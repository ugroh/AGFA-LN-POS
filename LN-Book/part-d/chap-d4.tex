% !TEX root = ../LN-Book.tex
%% -- 
%% --Stand 2025-06-09 final
%% --
\chapternopage{Asymptotics of Positive Semigroups on \texorpdfstring{C$^{*}$}{C*}- and \texorpdfstring{W$^{*}$}{W*}-Algebras}\label{chap:d4}
\chaptermark{Asymptotics of Positive Semigroups}
%% --
{\Large
\vspace*{-.75cm}
by \\[.25em]
Ulrich Groh
\vspace{.75cm}
\\
}
%% --
\section{Stability of Positive Semigroups}\label{sec:d4-1}
%% --
As explained in A-III, Section 1, it is possible to deduce uniform exponential stability of strongly continuous semigroups from the location of the spectrum of its generator if the spectral bound $ s(A) $ and the growth bound $ \omega_{0} $ coincide.
In this section we prove $s(A) = \omega_{0}$ for positive semigroups on \CA-algebras and preduals of \WA-algebras.
A more general discussion of the \enquote{$s(A) = \omega_{0}$} problem can be found in \citet{greinervoigtwolff:1981}.
For the results of this section the existence of a unit is essential.
%% --
\begin{theorem}\label{thm:d4-1.1}
Let $M$ be a \CA-algebra with unit and $\TT = (T(t))_{t \geq 0}$ a positive semigroup on $M$.
Then
%% --
\[
	-\infty < s(A) = \omega_{0} \in \sigma(A).
\]
%% --
\end{theorem}
%% --
\begin{proof}
For every  $t \geq 0 $ there exists $\phi_{t}$ in the state space $S(M)$ of $M$ such that
%% --
\[
	T(t)'\phi_{t} = r(T(t))\phi_{t} = \exp(\omega_{0} t)\phi_{t}
\]
%% --
(see, \eg \citet[2.1]{groh:1981}).

Let $n \in \N$ and
%% --
\[
	E_{n} \coloneqq \{\phi \in S(M) \colon T(2^{-n})\phi = \exp(\omega_{0} 2^{-n})\phi \}.
\]
%% --
Then $\emptyset \neq E_{n+1} \subseteq E_{n}$,  $(n \in \N)$.
Since $S(M)$ is $\sigma(M,M')$-compact, there exists $\phi \in \bigcap_{n \in \N} E_{n}$.
Then $ T(t)'\phi = \exp(\omega_{0} t)\phi$ follows for all $ 0 \leq t $ because the adjoint semigroup $(T(t)')_{t \geq 0}$ is a weak*-semigroup on $M'$.

Suppose $-\infty = \omega_{0}$.
Then for $t > 0$ either $r(T(t)) = 0$ (A-III, Prop.1.1) or $T(t)'\phi = 0$, in particular $\phi(T(t)\1) = 0$.
From this we obtain the contradiction $\phi(\1) = 0$.
Hence $-\infty < \omega_{0}$ and $\exp(\omega_{0} t) \in \rho(T(t)')$ for every $t \in \R_{+}$.
Thus $\omega_{0} \in \sigma(A)$ or $\omega_{0} = s(A)$.
\end{proof}
%% --
\begin{remark}\label{rem:d4-1.2}
%% --
\begin{enumerate}[\upshape (i), wide, labelindent=.5em]
\item
If we consider the nilpotent translation semigroup on the \CA-algebra $C_{0}( \left[0,1\right[ )$, then $\sigma(A) = \emptyset$ and $\omega_{0} = -\infty$.
This shows that the existence of a unit is essential.

\item
The equality $s(A) = \omega_{0}$ still holds for positive semigroups on commutative \CA-algebras without unit (see B-IV, Rem.1.2.b).
\end{enumerate}
\end{remark}
%% --
\begin{theorem}\label{thm:d4-1.3}
Let $M$ be a \WA-algebra with predual $M_{*}$ and let $(T(t))_{t \geq 0}$ be a positive semigroup on $M_{*}$.
Then $s(A) = \omega_{0}$.
\end{theorem}
%% --
\begin{proof}
For all $\lambda > s(A)$ and $\phi \in M_{*}$
%% --
\[
R(\lambda,A)\phi = \int_{0}^{\infty} \mathrm{e}^{-\lambda t}T(s)\phi ds
\]
%% --
which follows as in C-III, Section 1 or \citet[Theorem 3]{greinervoigtwolff:1981}.
Since $\|\phi\| = \phi(\1)$ for every $\phi \in M_{*}^{+}$ and since the norm is additive on the positive cone of $M_{*}$, the integral
%% --
\[
	\int_{0}^{\infty} \mathrm{e}^{\lambda t}\|T(s)\phi\|ds
\]
%% --
exists for all $\phi \in M_{*}$ and all $\lambda > s(A)$.
From this the assumption follows by A-IV,Thm.1.11.
\end{proof}
%% --
\begin{corollary}\label{cor:d4-1.4}
Let $M$ be a \CA-algebra and $(T(t))_{t \geq 0}$ a positive semigroup on $M'$.
Then $s(A) = \omega_{0}$ holds.
\end{corollary}
%% --
\noindent This follows from the fact that the bidual of a \CA-algebra is a \WA-algebra (see \citet[Theorem III.2.4.]{takesaki:1979}).
%% --
\begin{remark}\label{rem:d4-1.5}
A simple modification of A-III, Example 1.4 (take $c_{0}$ instead of $\ell^2$) shows that Theorem 1.3 is no longer true for non-positive semigroups (for details see \citet[Beispiel 2.5]{grohneubrander:1981}).

While the growth bound $\omega_{0}$ characterizes uniform exponential stability of the semigroup there are other (and weaker) stability concepts (cf. A-IV, Section 1).
\end{remark}
%% --
\begin{definition}\label{def:d4-1.6}
Let $E$ be a Banach space and $(T(t))_{t \geq 0}$ a semigroup on $E$.
We call the semigroup
%% --
\begin{enumerate}[\upshape (i)]
\item 
\emph{uniformly exponentially stable} if $\|T(t)\| \leq M\mathrm{e}^{-\omega t}$ for some $\omega$, $M > 0$ and all $t \geq 0$.

\item 
\emph{uniformly stable} if $\lim_{t \to \infty} T(t) = 0$ in the strong operator topology.

\item 
\emph{weakly stable} if $\lim_{t \to \infty} T(t) = 0$ in the weak operator topology.
\end{enumerate}
\end{definition}
%% --
Surprisingly all these properties coincide for positive semigroups on C*-algebras with unit.
%% --
\begin{theorem}\label{thm:d4-1.7}
Let $M$ be a C*-algebra with unit and $(T(t))_{t \geq 0}$ a positive semigroup on $M$.
Then the following assertion are equivalent.
%% --
\begin{enumerate}[\upshape (a)]
\item 
$s(A) < 0$.

\item 
The semigroup $(T(t))_{t \geq 0}$ is uniformly exponentially stable.

\item 
The semigroup $(T(t))_{t \geq 0}$ is uniformly stable.

\item 
The semigroup $(T(t))_{t \geq 0}$ is weakly stable.

\end{enumerate}
\end{theorem}
%% --
\begin{proof}
Since $s(A) = \omega_{0}$ by Theorem 1.3, it suffices to show that (d) implies (a).
For $t > 0$ there exists $\phi \in S(M)$ such that
%% --
\[
    T(t)'\phi = r(T(t))\phi .
\]
%% --
Then for $x \in M$
%% --
\[
    \phi(T(t)^{n}x) = (r(T(t)))^{n} \phi(x) \to 0
\]
%% --
as $n \to \infty$.
Therefore $r(T(t)) < 1$ or $\omega_{0} < 0$.
Since $s(A) \leq \omega_{0}$ the assertion follows.
\end{proof}
%% --
\begin{remark}\label{rem:d4-1.8}
Consider the translation semigroup $(T(t))_{t \geq 0}$ on $C_{0}(\R_{+})$. 
Then $\|T(t)\| = 1$, hence $s(A) = 1$, but $(T(t))_{t \geq 0}$ is strongly stable.
The same holds for the translation semigroup on $L^1(\R_{+})$.
Thus Theorem 1.7 is not true for semigroups on \CA-algebras without unit or on preduals of \WA-algebras.
For the discussion of the commutative situation we refer to B-IV, Section 1.
\end{remark}
%% --
\section{Stability of Implemented Semigroups}
%% --
Let $H$ be a Hilbert space, $\UG = (U(t))_{t \geq 0}$ a strongly continuous semigroup on $H$ with generator $B$ and $M \subseteq \BH$ a \WA-algebra, where $\BH$ is the \WA-algebra of all bounded linear operators on $H$.
Suppose $\UG(t)^{*}MU(t) \subseteq M$.
Then one can define a weak*-continuous semigroup $\TT$ on $M$ by 
%
\[
	T(t)x \coloneqq U(t)^{*}xU(t) \quad (t \in \R_{+}, x \in M) .
\]
%
We call $\TT$ an \emph{implemented semigroup}.
Every map $T(t) \in \TT$ of an implemented semigroup is weak*-continuous and $n$-positive for every $n \in \N$.
%% --
\begin{remarks}\label{rem:d4-2.1}

\begin{enumerate}[\upshape (i), wide, labelindent=.5em]
\item\label{item:d4-2.1-i}
Because of
%% --
\[
	\|T(t)\| = \|T(t)\1\| = \|U(t)^{*}U(t)\| = \|U(t)\|^2
\]
%% --
it follows that $\omega_{0}(\TT) = 2\,\omega_{0}(\UG)$.

\item\label{item:d4-2.1-ii}
If $\TT$ is an implemented semigroup, then the preadjoint semigroup is strongly continuous on $M_{*}$.
Therefore $s(A) = \omega_{0}$ for $\TT$ by Theorem~\ref{thm:d4-1.3}.

\item\label{item:d4-2.1-iii}
Since $\UG$ is a strongly continuous semigroup on $ H $, the same is true for the adjoint semigroup 
$\UG^{*} = \{ U(t)^{*} \colon U(t) \in \UG \}$ and its generator is given by $B^{*}$.
In analogy to \citet[3.2.55]{brattelirobinson:1979} the following assertions for $x \in M$ are equivalent.
%% --
\begin{enumerate}[\upshape (a)]
\item
$x \in D(A)$, $ A $ the generator of $ \TT $.

\item
For $\xi \in D(B)$ it follows $x\xi \in D(B^{*})$ and the linear mapping
%% --
\begin{equation}
	(\xi \mapsto x(B\xi)+B^{*}(x\xi)): D(B) \to H \tag{*}
\end{equation}
%% --
has a continuous extension to $H$.
\end{enumerate}
\end{enumerate}
\end{remarks}
%% --
%% -- d4-5
Then for $A$ is given as the continuous extension of (*), \ie 
 $Ax = xB + B^{*}x$ for $ x \in D(A) $

In the next theorem we give some equivalent conditions for the uniform exponential stability of an implemented semigroup.
As we shall see, the operator equality
%% --
\[
yB + B^{*}y = -x \quad (x, y \in M_{+})
\]
%% --
is necessary and sufficient, which is in complete analogy to the classical Liapunov stability result.
%% --
\begin{theorem}\label{thm:d4-2.2}
Let $M$ be a \WA-algebra on a Hilbert space $H$ and let $\TT = (T(t))_{t \geq 0}$ be a weak*-semigroup on $M$ with generator $A$ implemented by the semigroup $\UG$ on $H$ with generator $B$.
Then the following assertions are equivalent.
%% --
\begin{enumerate}[\upshape (a)]

\item\label{item:d4-2.2-a}
$\omega_{0}(\TT) = s(A) < 0$.

\item\label{item:d4-2.2-b}
The semigroup $(U(t))_{t \geq 0}$ is uniformly exponentially stable.

\item\label{item:d4-2.2-c}
There exists $0 \leq x \in D(A)$ such that $Ax = -\1$.

\item\label{item:d4-2.2-d}
There exists $0 \leq x \in D(A)$ such that $x(D(B)) \subseteq D(B^{*})$ \mbox{and $xB+B^{*}x = -\1$}.

\item\label{item:d4-2.2-e}
For every $0 \leq x \in D(A)$ there exists $0 \leq y \in D(A)$ such that $Ay = -x$.

\item\label{item:d4-2.2-f}
For every $0 \leq x \in D(A)$ there exists $0 \leq y \in D(A)$ such that $y(D(B)) \subseteq D(B^{*})$ and $yB+B^{*}y = -x$.

\item\label{item:d4-2.2-g}
$\int_{0}^{\infty} \|U(s)\xi\|^2ds$ exists for all $\xi \in H$.

\item\label{item:d4-2.2-h}
$\int_{0}^{\infty} |(T(s)x)\xi|\zeta)ds$ exists for all $\xi,\zeta \in H$ and all $x \in M$.
\end{enumerate}
\end{theorem}

\begin{proof}
The equivalence of \ref{item:d4-2.2-a} and \ref{item:d4-2.2-b} follows from Remark~\ref{rem:d4-2.1}\,\ref{item:d4-2.1-i}, whereas \ref{item:d4-2.2-c} and \ref{item:d4-2.2-d}), \resp  \ref{item:d4-2.2-e} and \ref{item:d4-2.2-f} are equivalent by the Remark~\ref{rem:d4-2.1}\,\ref{item:d4-2.1-iii}
%% --
\begin{enumerate}[wide, labelindent=.5em]

\item[(a) $\implies$ (c):] Since $s(A) < 0$ the resolvent $R(0,A)$ exists and is a positive map on $M$.
Therefore $R(0,A)1 \in D(A)_{+}$ or $Ax = -\1$ for some $x \in D(A)_{+}$.
%% --


\item[(c) $\implies$ (e):] Let $x \in D(A)_{+}$ such that $Ax = -\1$.
Then
%% --
\[
T(t)x - x = \int_{0}^{t} T(s)Ax \ds = -\int_{0}^{t} T(s) \1 \ds \quad (t \geq 0),
\]
%% --
hence
%% --
\[
0 \leq \int_{0}^{t} T(s) \1 \ds \leq x \quad (t \in \R_{+}).
\]
%% --
Since the family $(\int_{0}^{t} T(s)\1 \ds)_{t \geq 0}$ is increasing and bounded, 
%
\[
	\lim_{t \to \infty} \int_{0}^{t} T(s) \1 \ds
\]
%
exists in the weak operator topology on $\BH$.

Since on bounded sets of $M$, the weak operator topology is equivalent to the $\sigma(M,M_{*})$-topology, for every $\phi \in M_{*}$ the integral $ \int_{0}^{\infty}\phi(T(s) \1) \ds $ exists (\citet[1.15.2.]{sakai:1971}).
Take $x \in M_{+}$ and $\phi \in M_{*}^{+}$.
Then $x \leq \|x\| \1$ and therefore
%% --
\[
	\phi(T(s)x) \leq \|x\|\phi(T(s) \1) \quad (s \in \R_{+}).
\]
%% --
Hence $\int_{0}^{\infty} \phi(T(s)x)ds$ exists.
Since the positive cones of $M$ and $M_{*}$ are generating, $\int_{0}^{\infty}\phi(T(s)x)ds$ exists for every $x \in M$ and $\phi \in M_{*}$.
Therefore $R(0,A)$ exists and is positive which proves (e).

\item[(c) $\implies$ (g):] From the last paragraph we obtain that for all $\xi \in H$
%% --
\[
\int_{0}^{\infty}\|U(s)\|^2ds = \int_{0}^{\infty}(T(s)1\xi|\xi)ds
\]
%% --
exists.

\item[(g) $\implies$ (h):] It follows from the polarization identity that the integral
%% --
\[
\int_{0}^{\infty}(U(s)\xi|U(s)\zeta)ds
\]
%% --
exists for all $\xi,\zeta \in H$.
Using \citet[Theorem III.4.2 and Theorem II.2.6]{takesaki:1979}, we conclude as in the implication from (c) to (e) that for all $\xi,\zeta \in H$ the integral
%% --
\[
\int_{0}^{\infty}((T(s)x)\xi|\zeta)ds \quad (x \in M)
\]
%% --
is finite.

\item[(g) $\implies$ (a):] Since the vector states are dense in the predual of $M$ and since the preadjoint semigroup of\/ $\TT$ is strongly continuous, it is easy to see that the integral
%% --
\[
\int_{0}^{\infty} \phi(T(s)x)ds
\]
%% --
exists for all $x \in M$ and $\phi \in M_{*}$ (\citet[Theorem II.2.6]{takesaki:1979}).
Therefore, the resolvent $R(0,A)$ exists and is positive, hence $s(A) < 0$.
\end{enumerate}
\end{proof}
%% --
\section{Convergence of Positive Semigroups}\label{sec:d4-3}
%\index{Convergence of Positive Semigroups}
In this section the asymptotic behavior of positive semigroups $(T(t))_{t \geq 0}$ on \WA-algebras will be described in more detail.
Essentially we distinguish three cases.
%% --
\begin{enumerate}[\upshape (i), wide, labelindent=.5em]
\item
The Cesàro means $\frac{1}{s}\int_{0}^{s} T(t)dt$ converge strongly to a projection $P$ onto the fixed space of $(T(t))_{t \geq 0}$ (see Proposition~\ref{prop:d4-3.3} \& \ref{prop:d4-3.4}).

\item
The maps $T(t)$ converge strongly to $P$ (see Proposition~\ref{prop:d4-3.7}, \ref{thm:d4-3.8} \&. \ref{cor:d4-3.9}).

\item
The maps $T(t)$ behave asymptotically as a periodic group (Theorem~\ref{thm:d4-3.11}).
\end{enumerate}
%% --
Much of the following is based on the theory of weakly compact operator semigroups.
Therefore the following compactness criterium is quite useful.
%% --
\begin{proposition}\label{prop:d4-3.1}
Let $M$ be a \WA-algebra, $\TT$ a bounded semigroup of positive maps on $M_{*}$ and suppose that there exists a faithful family $\Phi$ of\/ $\TT$-subinvariant states in $M_{*}$.
Then $\TT$ is relatively compact in the weak operator topology of $ \L{M_{*}} $.
In particular, $\TT$ is strongly ergodic, \ie 
%
\[
	\lim_{s \to \infty} \frac{1}{s}\int_{0}^{s} T(t)x \dt
\]
%
exists for every $x$ in $M$ and yields a projection onto $\Fix{\TT}$.
\end{proposition}
%
\begin{proof}
Since the positive cone of $M_{*}$ is generating, it is enough to show that for every $0 \leq \phi \in M_{*}$ the orbit $\{T(t)\phi \colon t \in \R_{+}\}$ is relatively weak compact.
For this we use \citet[Theorem III.5.4.(iii)]{takesaki:1979}.

Let $(p_{n})_{n \in \N}$ be a decreasing sequence of projections in $M$ such that $\inf_{n} p_{n} = 0$.
Then $\lim_{n}\phi(p_{n}) = 0$ for every $\phi \in \Phi$.
Since
%% --
\[
	(T(t)p_{n})^2 \leq T(t)p_{n}, \quad t \in \R_{+},
\]
%% --
we obtain by a classical \emph{inequality of Kadison} that
%% --
\[
0 \leq \phi((T(t)p_{n})^2) \leq \phi(T(t)p_{n}) \leq \phi(p_{n}),
\]
%% --
hence $\lim_{n}\phi(T(t)p_{n}) = 0$ uniformly in $t \in \R_{+}$.
Since the family $\Phi$ is faithful on $M$, it follows from \citet[Proposition III.5.3]{takesaki:1979} that $(T(t)p_{n})$ converges to zero in the $s(M,M_{*})$-topology uniformly in $t \in \R_{+}$.
Since this topology is finer than the weak*-topology on $M$, we obtain the relative compactness of\/ $\TT$ which implies the strong ergodicity.
\end{proof}
%% --
Let $\TT$ be an identity preserving semigroup of Schwarz type on the predual of a \WA-algebra $M$.
We call
%% --
\[
p_{r} \coloneqq \sup\{s(|\phi|) \colon \phi \in \Fix{T}\}
\]
%% --
the recurrent projection associated with $\TT$.
For a motivation of this definition compare, \eg \citet[Section 6.3]{davies:1976}.

Since $T(t)|\phi| = |\phi|$ for all $\phi \in \Fix{T}$ (D-III, Cor. 1.5), we obtain $T(t)'p_{r} \geq p_{r}$ (see D-I,Sec.3.(c)).
Let $\TT^{(r)}$ be the reduced semigroup on $p_{r}M_{*}p_{r}$ with generator $A^{(r)}$.
Then $\TT^{(r)}$ is identity preserving and of Schwarz type.
Similarly, if $\RR$ is a pseudo-resolvent on $D = \{\lambda \in \C \colon \Re(\lambda) > 0\}$ with values in $M_{*}$ such that $\RR$ is identity preserving and of Schwarz type, then the recurrent projection associated with $\RR$ is defined using $\Fix{\RR}$.
%% --
\begin{remark}\label{rem:d4-3.2}
\begin{enumerate}[\upshape (i), wide, labelindent=.5em]

\item\label{item:d4-3.2-i}
Let $\phi \in M_{*}$ and $\alpha \in \R$ such that $(\mu - \im\alpha)R(\mu)\phi = \phi$ for some $\mu \in \R_{+}$.
Since $s(|\phi|)$ and $s(|\phi^{*}|)$ are majorized by $p_{r}$ (D-III,Prop.1.4), it follows that $\phi$ and $\phi^{*}$ are in $p_{r}M_{*}p_{r}$.

\item\label{item:d4-3.2-ii}
From (i) and the observation that the family $\{|\phi| \colon \phi \in \Fix{\TT}\}$ is
faithful on $p_{r}Mp_{r}$ and consists of $\TT^{(r)}$-invariant elements, it follows that
%% --
\begin{enumerate}[--]
\item
$P\sigma(A) \cap \im\R = P_{\sigma}(A^{(r)}) \cap \im\R$.

\item
$\Kern{(\im\alpha - A)} \subset p_{r}M_{*}p_{r}$ for all $\alpha \in \R$.

\item
The semigroup $\TT^{(r)}$ is relatively compact in the weak operator topology and therefore strongly ergodic.
\end{enumerate}


\item\label{item:d4-3.2-iii}
Similarly, let $\RR$ be an identity preserving pseudo-resolvent with values in $M_{*}$ on $D = \{\lambda \in \C \colon \Re(\lambda) > 0\}$ which is of Schwarz type.
It follows as in (b) that 
$\Fix{(\lambda - \im\alpha)R(\lambda)}$ 
is contained in $p_{r}M_{*}p_{r}$ for all $\lambda \in D$ and $\alpha \in \R$, where $p_{r}$ is the associated recurrent projection.
\end{enumerate}
\end{remark}
%% --
We now give a characterization of strong ergodicity of semigroups which are identity preserving and of Schwarz type.
For this we need that the Cesàro means
%% --
\[
C(s)x = \frac{1}{s}\int_{0}^{s} T(t)xdt \quad (x \in M, 0 \leq s \in \R)
\]
%% --
are Schwarz maps.
We omit the simple calculation (compare D-I,Thm.2.1).
%% --
\begin{proposition}\label{prop:d4-3.3}
Let $\TT$ be an identity preserving semigroup of Schwarz type on the predual of a \WA-algebra $M$.
Then the following assertions are equivalent.
%% --
\begin{enumerate}[\upshape (a)]
\item
$\TT$ is strongly ergodic on $M_{*}$.

\item
$\sigma(M,M_{*})\text{-}\lim_{s \to \infty} C(s)'p_{r} = 1$.

\item
$s^{*}(M,M_{*})\text{-}\lim_{s \to \infty} C(s)'p_{r} = 1$.
\end{enumerate}
\end{proposition}
%% --
\begin{proof}
Suppose that (a) holds.
Since $\Fix{T}$ separates $\Fix{T'}$, the fixed space of $\TT'$ is non trivial, hence $p_{r} \neq 0$ (see \citet[Chap.2, Thm.1.4]{krengel:1985}).
Let $0 \leq \psi \in M_{*}$, then $\psi_{0} \coloneqq \lim_{s \to \infty} C(s)\psi \in \Fix{T}$ and $s(\psi_{0}) \leq p_{r}$.
Therefore
%% --
\begin{align*}
\lim_{s \to \infty} \psi(C(s)'p_{r}) &= \lim_{s \to \infty} (C(s)\psi)(p_{r})  = \psi_{0}(p_{r})  \\
= \psi_{0}(1) &= \lim_{s \to \infty} (C(s)\psi)(1) = \psi(1)
\end{align*}
%% --
which proves (b).

Suppose that (b) is satisfied.
Since $C(s)'p_{r} \leq 1$ for all $s \in \R_{+}$, we obtain (c).
(Use that for $(x_{\alpha}) \in M_{+}$ we have $\lim_{\alpha}x_{\alpha} = 0$ in the weak*-topology if and only if $\lim_{\alpha}x_{\alpha} = 0$ in the $s^{*}(M,M_{*})$-topology.)

Suppose that (c) holds.
Since each $C(s)'$ is an identity preserving Schwarz map, we obtain for all $x \in M$
%% --
\begin{align*}
	\left( C(s)'((1-p_{r})x))(C(s)'((1-p_{r})x)^{*} \right) 
			&\leq C(s)'((1-p_{r})xx^{*}(1-p_{r}))  \\
 	&\leq \|x\|^2 C(s)'(1-p_{r}),
\end{align*}
%% --
hence
%% --
\[
s^{*}(M,M_{*})\text{-}\lim_{s \to \infty} C(s)'((1-p_{r})x) = 0.
\]
%% --
In particular, we obtain for all $x \in \Fix{\TT'}$ that $x = \sigma(M,M_{*})\text{-}\lim_{s \to \infty} C(s)'x = \sigma(M,M_{*})\text{-}\lim_{s \to \infty} C(s)'(p_{r}x)$.

Especially for $0 \neq x \in \Fix{\TT}$ we obtain $p_{r}xp_{r} \neq 0$.
Since the \WA-algebra $p_{r}Mp_{r}$ is the dual of $p_{r}M_{*}p_{r}$ and since $\TT^{(r)}$ is strongly ergodic, it follows that the fixed space of\/ $\TT$ separates the points of $\Fix{\TT'}$.
Thus $\TT$ is strongly ergodic (\citet[Chap. 2, Thm. 1.4]{krengel:1985}).
\end{proof}
%% --
It follows from the result above that the semigroup in \citet{evans:1977} cannot be strongly ergodic on $\BH_{*}$ since the associated recurrent projection is zero.
But for irreducible semigroups we have the following result.
%% --
\begin{proposition}\label{prop:d4-3.4}
Let $\TT$ be an identity preserving semigroup of Schwarz type on the predual of a \WA-algebra $M$.
Then the following assertions are equivalent.
%% --
\begin{enumerate}[\upshape (a)]
\item
$\TT$ is irreducible and $P\sigma(A) \cap \im\R \neq \emptyset$.

\item
$\TT$ is relatively compact in the weak operator topology and the fixed space of\/ $\TT$ is generated by a faithful state.

\item
$\TT$ is strongly ergodic and the fixed space of\/ $\TT$ is generated by a faithful state.

\item
The fixed space of\/ $\TT$ is generated by a faithful state.
\end{enumerate}
\end{proposition}
%% --
\begin{proof}
Suppose (a) is satisfied.
Since $\Fix{\TT} \neq \{0\}$, there exists a faithful normal state $\phi$ on $M$ such that $\Fix{\TT} = \C \,\phi$ (D-III, Thm.1.10.).
Therefore $\TT$ is relatively compact in the weak operator topology by Proposition 3.1., whence (b) holds and the  implications from (b) to (c) and (c) to (d) are obvious.

Suppose that (d) holds.
Let $\phi$ be a faithful normal state on $M$ such that $\Fix{T} = \phi\C$.
By Proposition 3.1 the semigroup $\TT$ is strongly ergodic.
Therefore the fixed space of\/ $\TT$ separates the points of $\Fix{T'}$.
Consequently $\Fix{T'} = \C1$.
Thus the ergodic projection associated with $\TT$ is given by $P = 1 \otimes \phi$, \ie $P\psi = \psi(1)\phi$ for all $\psi \in M_{*}$.
Let $F$ be a closed $\TT$-invariant face of $M_{*}^{+}$.
If $0 \neq \psi \in F$ then
%% --
\[
\lim_{s \to \infty} C(s)\psi = \psi(1)\phi \in F.
\]
%% --
Hence $\phi \in F$ and therefore $F = M_{*}^{+}$ by the faithfulness of $\phi$ which proves (a).
\end{proof}
%% --
The next theorem is an extension of D-III, Thm.1.10 and shows the usefulness of the theory of semitopological semigroups.
Assume $\TT \subseteq \L{M_{*}}$ to be relatively compact in the weak operator topology. 
Since $\TT$ is commutative its closure $\mathcal{S}  = (\TT)^{-} \subseteq L_{w}(M_{*})$ contains a unique minimal ideal $\mathcal{K}$, called the kernel of $\mathcal{S}$, which is a compact Abelian group (\citet{deleeuw:1961}, \citet{junghenn:1971} \& \citet[§ 2.4]{krengel:1985}).
The identity $Q$ of $\mathcal{K}$ is a projection onto the closed linear span of all eigenvectors of $ A $ pertaining to the eigenvalues in $ \im \R $. 

Moreover, the dual group of $\mathcal{K}$ can be identified with the subgroup of $\im\R$ generated by 
$P\sigma(A) \cap \im \R $.
We call $Q$ the semigroup projection associated with $\TT$.
On the other hand, $\TT$ is always strongly ergodic with projection $P$ onto $\Fix{\TT}$.
Obviously, the relation
%% --
\[
    0 \leq P \leq Q \leq \Id 
\]
%% --
holds, where the order relation is defined by the inclusion of the range spaces.

There are two extreme cases.
First, $Q = \Id$ and $ \rank (P)$.
This corresponds to the Halmos-von Neumann Theorem in commutative ergodic theory and is discussed, at least for irreducible semigroups, in \citet{olesen:1980}.

Second, $\Id > Q = P$, in particular $\rank(P) = 1$.
This latter case will be investigated in detail for $M = \BH $, the \WA-algebra of all bounded linear operators on a Hilbert space $H$.
But we first need some preparations.

\begin{theorem}\label{thm:d4-3.5}
Let $\TT$ be an identity preserving semigroup of Schwarz type on the predual of a \WA-algebra $M$ and suppose there exists a faithful family of\/ $\TT$-invariant states on $M$.
Let $N$ be the $\sigma(M,M_{*})$-closed linear span of all eigenvectors of $A'$ pertaining to the eigenvalues in $\im\R$.
If $Q$ is the semigroup projection associated with $\TT$, then the following holds.
%% --
\begin{enumerate}[\upshape (i)]
\item 
The adjoint of $Q$ is a faithful normal conditional expectation from $M$ onto the \WA-subalgebra $N$.

\item 
The restriction of $T'$ to $N$ can be embedded into a $\sigma(M,M_{*})$-continuous, one-parameter group of *-automorphisms.

\item 
If, in addition, $\TT$ is irreducible and if $\phi$ is the normal state generating the fixed space of\/ $\TT$, then $\phi|_{N}$ is a faithful normal trace.

\end{enumerate}
\end{theorem}
%% --
%% --
\begin{proof}
Consider $H \coloneqq P\sigma(A) \cap \im\R$ which is not empty by assumptions.
From Proposition 3.1 it follows that $\TT$ is relatively compact in the weak operator topology.
Let $K$ be the semigroup kernel of $\overline{\TT}{w} \subset L(M_{*})$ and $Q$ the unit of $K$.
Recall that $Q\psi_{n} = \psi_{n}$ for all $\psi_{n} \in M_{*}$ such that $A\psi_{n} = n\psi_{n}$ $(n \in H)$.
Let $\mathcal{E}$ be the family of all eigenvectors of $A'$ pertaining to the eigenvalues in $H$.

Then $\mathcal{E}$ is closed with respect to the multiplication in $M$ and the formation of adjoints.
Thus $N$ is a \WA-subalgebra of $M$, \citet[Corollary 1.7.9.]{sakai:1971}, and $\TT_{0}(t)' \coloneqq T(t)'_{|N}$ is multiplicative (for this see D-III, Lemma 1.1).

Since $Q \in \overline{\TT}{w}  \subseteq L_{w}(M_{*})$, there exists an ultrafilter $\mathfrak{U}$ on $\R_{+}$ such that 
%
\[
	\lim_{\mathfrak{U}}\langle T(t)\psi,x\rangle = \langle Q\psi,x\rangle
\]
%
for all $x \in M$ and $\psi \in M_{*}$.
If $n \in H$ and $\psi_{n} \in M_{*}$ such that $A\psi_{n} = n\psi_{n}$, then for all $x \in M$ we obtain
%% --
\[
\langle\psi_{n},x\rangle = \langle Q\psi_{n},x\rangle = \lim_{\mathfrak{U}} \langle T(t)\psi_{n},x\rangle = (\lim_{\mathfrak{U}} \mathrm{e}^{nt})\langle\psi_{n},x\rangle,
\]
%% --
hence $\lim_{\mathfrak{U}} \mathrm{e}^{nt} = 1$.
From this it follows that for all $\psi \in M_{*}$ we have
%% --
\[
\langle\psi,Q'(u_{n})\rangle = \lim_{\mathfrak{U}} \langle\psi,T(t)'u_{n}\rangle = (\lim_{\mathfrak{U}} \mathrm{e}^{nt})\langle\psi,u_{n}\rangle = \langle\psi,u_{n}\rangle.
\]
%% --
Hence $N \subseteq Q'(M)$.

For $\gamma$ in the dual group of $K$ and $x \in M$ we define $x_{\gamma}$ by
%% --
\[
\psi(x_{\gamma}) \coloneqq \int_{K} \langle S\psi,x\rangle\langle S,\gamma \langle\mathrm{e}^{*} \dm (S) \quad (\psi \in M_{*}^{+}).
\]
%% --
Then $x_{\gamma} \in M$ and $T(t)'x_{\gamma} = \langle QT(t),\gamma\rangle x_{\gamma}$.
Therefore $x_{\gamma} \in N$.
Thus the inclusion $Q'M \subseteq N$ is proved if we can show that $Q'M$ belongs to the $\sigma(M,M_{*})$-closed linear span of $\{x_{\gamma} \colon \gamma \in K, x \in M\}$.
For this it is enough to show that every linear form $\psi \in M_{*}$ such that $\psi(x_{\gamma}) = 0$ for all $\gamma \in K$ satisfies $\psi(Qx) = 0$ for all $x \in M$.
But if $\psi(x_{\gamma}) = 0$, then
%% --
\[
\int_{K} \langle S\psi,x\rangle\langle S,\gamma\rangle\mathrm{e}^{*} \dm (S) = 0, \gamma \in K.
\]
%% --
Since the map $(S \mapsto \psi(Sx))$ is continuous on $K$ and since the elements of $K$ form a complete orthonormal basis in $L^2(K,\dm)$, we obtain $\psi(Sx) = 0$ for all $S \in K$, in particular $\psi(Qx) = 0$ as desired.

Since the range of $Q'$ is a \WA-subalgebra of $M$ it follows from \citet[Theorem III.3.4]{takesaki:1979} that $Q'$ is a completely positive, normal conditional expectation.
This $Q'$ is faithful, \ie $\Kern{Q'} \cap M_{+} = \{0\}$ since $Q\phi = \phi$ for the faithful linear form $\phi$.

Let $\phi$ be the faithful normal state generating $\Fix{T}$ and let $\UG$ be a family of unitary eigenvectors of $A'$ pertaining to the eigenvalues in $H$ (see D-III, Remark 1.11).
If $u_{1}$, $u_{2} \in U$, then
%% --
\[
\phi(u_{1}u_{2}^{*}) = \phi(T_{0}(t)'(u_{1}u_{2}^{*})) = \mathrm{e}^{(n_{1}-n_{2})t}\phi(u_{1}u_{2}^{*}).
\]
%% --
Therefore
%% --
\[
\phi(u_{1}u_{2}^{*}) = 
    \begin{cases} 
        0 & \text{if } n_{1} \neq n_{2}, \\ 
        1 & \text{if } n_{1} = n_{2}. 
    \end{cases}
\]
%% --
Hence $\phi(u_{1}u_{2}^{*}) = \phi(u_{2}^{*}u_{1})$ from which it follows that $\tau \coloneqq \phi|_{N}$ is a faithful normal trace.
\end{proof}
%% --
\begin{remarks}\label{rem:d4-3.6}
\begin{enumerate}[\upshape (i), wide, labelindent=.5em]
\item
Since $QM_{*} = N_{*}$ and $Q'M = N$, where $N_{*}$ is as in D-III, Proposition 1.12, it follows from general duality theory that $(N_{*})' = N$.

\item
If $\psi \in N_{*}$, then $|\psi| \in N_{*}$.
To see this, note that $Q\psi = \psi$ and $Q$ is an identity preserving Schwarz map.
Then the assertion follows from D-III, Proposition 1.4.

\item
If $\psi \in N_{*}$, then $|T_{0}(t)\psi| = T_{0}(t)|\psi|$ for all $t \in \R$.
This follows immediately from the fact that $\TT_{0}(t)'$ is a *-automorphismus on $N$.

\item
Let us add a few words concerning the structure of $N$: If $\TT$ is irreducible and $K$ is the semigroup kernel of $\TT^{-} \subseteq L_{w}(M_{*})$, then $(S \mapsto S'): K \to L((N,\sigma(N,N_{*})))$ is a representation of the compact, Abelian group $K$ as group of *-automorphism such that the fixed space is one dimensional.
Therefore we are able to apply the results of \citet{olesen:1980}.
There are three possibilities for $N$.

\begin{enumerate}[(1)]
\item
$N = L^{\infty}(K,\dm)$ and $\TT|_{N}$ is the translation group on $N$.

\item
$N \cong R$ where $\RR$ is the (unique) hyperfinite factor of type II$_{1}$.
In that case (the image of) $K$ is approximately inner on $\RR$ [l.c., Theorem 5.8].


\item
There exists a closed subgroup $ G $  of $ K $  such that
%% --
\[
N = L^{\infty}(K/G, \dm) \otimes R
\]
%% --
where $ R $  is as in (ii) and $ \dm $  the normalized Haar measure on $ K/G $  [l.c., Theorem 5.15].
\end{enumerate}
\end{enumerate}
\end{remarks}
%% --
So far we have studied weak*-semigroups on general \WA-algebras.
We apply now these results to weak*-semigroup on $ \BH $.
%This is of interest in view of the results in \citet{davies:1976}.
To do this we call a triple $ (M,\phi,\TT) $  a \WA-dynamical system if $ M $  is a \WA-algebra, $ \TT$ a weak*-semigroup of identity preserving Schwarz maps on $ M $  and $ \phi $  a faithful family of $ \TT $-invariant normal states.
We call $ (M,\phi,\TT) $  irreducible, if the preadjoint semigroup is irreducible (alternatively, if the fixed space of $ \TT $  is one dimensional).
%% --
\begin{proposition}\label{prop:d4-3.7}
Let $ (\BH,\phi,\TT) $  be a \WA-dynamical system on the \WA-algebra $ \BH $  of all bounded linear operators on a Hilbert space $ H $.
Then the following assertions are equivalent.
\begin{enumerate}[\upshape (a)]
\item
$ P\sigma(A) \cap \im\R = \{0\} $,

\item
$ \lim_{s \to \infty} T(s)_{*} = P_{*} $  in the strong operator topology on $ \L{\BH_{*}} $.
\end{enumerate}
\end{proposition}
%% --
\begin{proof}
Obviously (b) implies (a).
Suppose that (a) is fulfilled.
Then the ergodic projection $ P_{*} $  of the preadjoint semigroup is equal to the associated semigroup projection.
Consequently there exists an ultrafilter $ \mathfrak{U} $  on $ \R_{+} $  such that $ \lim_{\mathfrak{U}} T(t) = P $  in the weak operator topology.
We claim that the convergence holds even in the strong operator topology.
Taking this for granted it follows, since for every $ t \in \R_{+} $  $ T(t) $  is a contraction, that
%% --
\[
\lim_{t \to \infty} \|T(t)_{*}\phi\| = 0
\]
%% --
for all $ \phi \in \Kern{P_{*}} $.
Since $ T(t)_{*}\psi = \psi $  for every 
$ \psi \in \Im (P_{*}) $  and
%% --
\[
\BH_{*} = \Im (P_{*}) \oplus \Kern{P_{*}}
\]
%% --
the assertion is proved.
%\end{proof}

It remains to show that $ \lim_{\mathfrak{U}} T(t)_{*} = P_{*} $  in the strong operator topology.
Choose $ 0 \leq \phi \in \BH_{*} $, $ \|\phi\| \leq 1 $  and let $ \phi_{t} \coloneqq T(t)_{*}\phi $  $ (t>0) $.
$ \phi_{0} \coloneqq P_{*}\phi $  and let $ \{p_{i}: i \in A\} $  be an increasing net of projections of finite rank in $ \BH $  with strong limit 1.
Since the set $ K \coloneqq \{\phi_{t}: t \geq 0\} $  is relatively compact in the $ \sigma(\BH_{*},\BH) $-topology, there exists for every $ \delta > 0 $  an index $ i_{0} \in A $  such that
%% --
\[
\|(1 - p_{i})\psi(1 - p_{i})\| \leq \delta
\]
%% --
for every $ \psi \in K $  and $ i \geq i_{0} $  (\citet[Theorem III.5.4.(vi)]{takesaki:1979}).
In particular
%% --
\[
|\psi(1 - p_{i})| \leq \delta, \quad \psi \in K,\, i(0) \leq i.
\]
%% --
Let $ p \coloneqq p_{i(0)} $.
Then for all $ x $  in the unit ball of $ M $  it follows that
%% --
\begin{align*}
|(\phi_{t} - \phi_{0})(x)| &\leq \\
|(\phi_{t} - \phi_{0})(pxp)| &+ |(\phi_{t} - \phi_{0})((1-p)xp)| \\
+ |(\phi_{t} - \phi_{0})(x(1-p))| &\leq 
\leq |(\phi_{t} - \phi_{0})(pxp)| + 4\sqrt{\delta} .
\end{align*}
%% --
Since the \WA-algebra $ p \BH p $  is finite dimensional, there exists $ U \in \mathfrak{U} $  such that
%% --
\[
	\|(\phi_{t} - \phi_{0})|_{p \BH p}\| \leq \delta .
\]
%% --
for all $ t \in U $.
Consequently
%% --
\[
\|(\phi_{t} - \phi_{0})\| \leq (\delta + 4\sqrt{\delta})
\]
%% --
for all $ t \in U $.
Therefore $ \lim_{\mathfrak{U}} T(t)_{*}\phi = P_{*}\phi $  in the strong operator topology.
Since the positive cone of $ \BH_{*} $  is generating, the assertion is proved.
\end{proof}
%% --
We show next, that for irreducible \WA-dynamical systems on $ \BH $  the above properties always hold.
%% --
\begin{theorem}\label{thm:d4-3.8}
Let $ (\BH,\phi,\TT) $  be an irreducible \WA-dynamical system.
Then
%% --
\[
P\sigma(A) \cap \im\R = \{0\} .
\]

%% --
\end{theorem}
%% --
\begin{proof}
Let $ N $  be the \WA-subalgebra of $ M = \BH $  generated by the eigenvectors of $ A $  pertaining to the eigenvalues on $ \im\R $  and let $ Q $  be the faithful normal conditional expectation from $ M $  onto $ N $  (Proposition~\ref{prop:d4-3.7}).
Since $ M $  is atomic, $ N $  is atomic (\citet{stormer:1972}).
$ N $  is finite since there exists a finite, faithful normal trace on $ N $.
In particular the center of $ N $  is isomorphic to $ \ell^{\infty} $.

Let $ \SG $  be the restriction of $ \TT $  to the center.
Then $ \SG $  is a weak*-semigroup such that every $ S(t) \in \SG $  is $ \sigma(\ell^{\infty},\ell^{1}) $-continuous and a *-automorphism.
From this it follows that $ S(t) $  is induced by some continuous flow $ \kappa_{t}: \N \to \N $.
Indeed, if $ \delta_{n}((\xi_{m})) = \xi_{n} $  $ (n \in \N, (\xi_{m}) \in \ell^{\infty}) $, then $ \delta_{n} \circ S(t) $  is a normal scalar valued *-homomorphism hence of the form $ \delta_{m} $  for some $ m = \kappa_{t}(n) $.
But the function $ t \mapsto \kappa_{t} $  is continuous from $ \R $  into $ \N $, whence constant.
Hence $ S(t) = \Id $.
But the semigroup $ S $  is weak*-irreducible on the center.
Consequently, the center is one dimensional.
Using [Takesaki, Theorem V.1.27] we obtain $ N = B(H_{n}) $  where $ H_{n} $  is a finite dimensional Hilbert space.
But if $ 0 \neq \im\alpha \in P\sigma(A) \cap \im\R $  then $ \im\alpha\Z \subset P\sigma(A) $  by D-III,Thm.1.10, whence $ N $  must be infinite dimensional.
Therefore $ P\sigma(A) \cap \im\R = \{0\} $  as desired.
\end{proof}
%% --
\begin{corollary}\label{cor:d4-3.9}
If $ (\BH,\phi,T) $  is an irreducible \WA-dynamical system, then
%% --
\[
	\lim_{s \to \infty} T(s) = 1 \otimes \phi
\]
%% --
in the strong operator topology on $ L(\BH_{*}) $, where $ \phi $  is the unique normal state generating the fixed space of $ T_{*} $.
\end{corollary}
%% --
We are now going to discuss the asymptotic behavior of positive semigroups whose generator has boundary point spectrum different from $ 0 $.
The standard example is the following.
%% --
If $ \Gamma $  is the unit circle, $ \dm $  the normalized Haar measure on $ \Gamma $  and $ 0 < \tau \in \R $, then we define the maps $ T_{\tau}(t) $, $ t \in \R_{+} $, on $ L^{1}(\Gamma,m) $  by
%% --
\[
(T_{\tau}(t)f)(\xi) = f(\xi\exp(\frac{2\pi i}{\tau}t)) \quad (f \in L^{1}(\Gamma,\dm), \xi \in \Gamma).
\]
%% --
Then $ \TT \coloneqq (T_{\tau}(t))_{t \geq 0} $  forms a strongly continuous one parameter semigroup which is identity preserving and of Schwarz type.
%% --
Since $ \TT $  is periodic of period $ \tau $, it follows that 0 is a pole of the resolvent of its generator $ B $  with residuum $ P = 1 \otimes 1 $  and $ \{\frac{2\pi \im}{\tau} \cdot k: k \in \Z\} = \sigma(B) $.
Thus $ \TT $  is irreducible and uniformly ergodic on $ L^{1}(\Gamma,\dm) $  (see A-II, Section 5).

Now let $ \TT $  be a semigroup on a predual $ M_{*} $ of a von Neumann-algebra $ M $.
It is called \emph{partially periodic}, if there exists a projection $ Q \in L(M_{*}) $  reducing $ T $  such that $ Q(M_{*}) \cong L^{1}(\Gamma,\dm) $  and $ T_{| \Im(Q)} $  is conjugate to a periodic semigroup on $ L^{1}(\Gamma,\dm) $.

In the main result we present a non commutative version of \citet{nagel:1984} showing that certain dynamical systems are partially periodic semigroups.
%% --
\begin{proposition}\label{prop:d4-3.10}
Let $ \TT $  be an irreducible, identity preserving semigroup of Schwarz type with generator $ A $  on the predual of a \WA-algebra $ M $.

If $ \TT $  is uniformly ergodic, then $ \sigma(A) \cap \im\R = P\sigma(A) \cap \im\R = \im\alpha\Z $  for some $ \alpha \in \R $.
If additionally $ \sigma(A) \cap \im\R \neq \{0\} $, there exists a strictly positive projection $ Q $  on $ M_{*} $  which is identity preserving and completely positive such that
%% --
\begin{enumerate}[\upshape (i)]
\item
$ Q $  reduces $ \TT $  and $ Q(M_{*}) \cong L^{1}(\Gamma) $, $ \Gamma $  being the one dimensional torus.

\item
The restriction $ T_{0} $  of $ \TT $  to $ \Im (Q) $  is irreducible and conjugate to a rotation semigroup of period $ \tau = \frac{2\pi}{\alpha} $  on $ \Gamma $.

\item
The spectral bound $ s(A_{| \Kern{Q} } $  is strictly smaller than $ 0 $.
\end{enumerate}
\end{proposition}
%% --
\begin{proof}
By D-III, Thm.1.11 and D-III, Thm.2.5 it follows that
%% --
\[
\sigma(A) \cap \im\R = P\sigma(A) \cap \im\R = \im\alpha\Z
\]
%% --
for some $ \alpha \in \R $.
Suppose $ \alpha \neq 0 $.
Since $ \sigma(A) + \im\alpha\Z = \sigma(A) $  and since every $ n \in \im\alpha\Z $  is isolated, it follows that there exists $ \delta > 0 $  such that
%% --
\[
\sigma(A) \setminus \im\alpha\Z \subseteq \{\lambda \in \C \colon  \Re(\lambda) \leq \delta\}.
\]
%% --
Let $ \{u_{\alpha}^{k}: k \in \Z\} $  be a family of unitary eigenvectors of $ A' $  pertaining to the eigenvalues in $ \im\R $.
Then $ Q'(M) $  is a commutative \WA-algebra.
For $ \tau \coloneqq \frac{2\pi}{\alpha} $, we obtain $ T(\tau)u_{\alpha}^{k} = u_{\alpha}^{k} $, hence $ T|_{\Im (Q)} $  is periodic.
From the Halmos-von Neumann theorem (see \citet[Thm. III.7.11]{schaefer:1974})
it follows that $ T|_{\Im (Q)} $  is conjugate to the rotation semigroup of period $ \tau $  on $ L^{1}(\Gamma,m) $.
\end{proof}
%% --
Using this proposition we obtain the following theorem.
%% --
\begin{theorem}\label{thm:d4-3.11}
Let $ T = (T(t))_{t \geq 0} $  be a uniformly ergodic, identity preserving semigroup of Schwarz type on the predual of a \WA-algebra $ M $  and suppose 
%
\[
	\sigma(A) \cap \im\R \neq \{0\} .
\]
%
Then there exists a partially periodic, identity preserving semigroup $ S = (S(t))_{t \geq 0} $  of Schwarz type on $ M_{*} $  such that
%% --
\[
\lim_{t \to \infty} (T(t) - S(t)) = 0
\]
%% --
in the strong operator topology.
\end{theorem}
%% --
\begin{proof}
Let $ \phi $  be the normal state on $ M $  generating the fixed space of $ \TT $.
Let $ \SG = (S(t))_{t \geq 0} $  where $ S(t) \coloneqq T(t) \circ Q $  and $ Q $  is as in 2.6.
Obviously, $ \SG $  is partially periodic and $ \phi \in \Fix{\SG} $.
Let $ H_{\phi} $  be the GNS-Hilbert space pertaining to $ \phi $.
Since $ \phi $  is fixed under $ \TT $, $ \SG $  and $ Q $,  these objects have a canonical extension to $ H_{\phi} $  (in the following denoted by the same symbols).
If $ H_{0} \coloneqq \Kern{Q} \subseteq H_{\phi} $,  then it is easy to see that $ H_{0} $  is invariant under the extension to $ H_{\phi} $  and for the multiplication maps we defined in D-III, Remark 1.3.

Consequently, using the results in \citet{grohkuemmerer:1982}, it follows that there exists $ c \in \R $  such that for all $ \gamma $  near $ 0 $  and all $ \beta \in \R $ :
%% --
\begin{equation}
\|R(\gamma + \im\beta A_{0})\| \leq c \tag{*} ,
\end{equation}
%% --
where $ A_{0} \coloneqq A_{|\Kern{Q}} $  (the norm taken in $ L(H_{\phi}) $ ).
Using the result in A-III, Cor.7.11 it follows that
%% --
\[
\lim_{t \to \infty} \|T(t)|_{H_{0}}\| = 0.
\]
%% --
Since the $ s(M,M_{*}) $-topology on the unit ball of $ M $  is nothing else than the restriction of the norm topology on $ H_{\phi} $, we obtain
%% --
\[
s(M,M_{*})\text{-}\lim_{t \to \infty} (T(t)' - S(t)')(x) = 0
\]
%% --
uniformly on $ M_{1} $.
From this the assertion follows.
\end{proof}
%% --
\section{Uniform Ergodic Theorems}
%% --
As we have seen, uniformly ergodic semigroups have strong spectral properties.
In this section we study sufficient conditions which imply uniform ergodicity thereby generalizing results of 
\citet{groh:1984b}.
We first need some preparations.
%% --
\begin{lemma}\label{lem:d4-4.1}
Let $ \RR $  be an identity preserving pseudo-resolvent of Schwarz type on $ D = \{\lambda \in \C \colon  \Re(\lambda) > 0\} $  with values in the predual of a \WA-algebra $ M $.
If the fixed space of $ \RR $  is infinite dimensional, then there exists a sequence of states in $ \Fix{\RR} $  such that the corresponding support projections are mutually orthogonal in $ M $.
\end{lemma}
%% --
\begin{proof}
Let $ \Phi = \{\phi \in \Fix{\RR}: \phi \text{ state on $ M $}\} $  and let $ p = \sup\{s(\phi): \phi \in \Phi\} $.
Since $ \lambda R(\lambda)\phi = \phi $  for all $ \phi \in \Phi $  and $ \lambda \in D $,  it follows $ \mu R(\mu)(\1 - s(\phi)) = (\1 - s(\phi)) $.
Hence $ \mu R(\mu)(\1 - p) = (\1 -p ) $  for all $ \mu \in \R_{+} $.

Let $ \RR_{1} $  be the induced pseudo-resolvent on $ pM_{*}p $  (D-I, Section 3.3).
Then the family $ \Phi $  is faithful on $ M_{p} $  and contained in the fixed space of $ \RR_{1} $.
The adjoint $ \mu R_{1}(\mu)' $  is an identity preserving Schwarz map.
Consequently it follows from D-III, Lemma 1.1.(ii) and the $ \sigma(M_{p},(M_{p})_{*}) $-continuity of $ \mu R_{1}(\mu)' $  that $ \Fix{R_{1}'} $  is a \WA-subalgebra of $ M_{p} $  and by D-III, Lemma 1.5, $ \dim \Fix{\RR} \leq \dim \Fix{R_{1}'} $.

If $ \Fix{\RR} $  is infinite dimensional, let $ (p_{n}) $  be a sequence of mutually orthogonal projections in 
$ \Fix{ R_{1}' } \subseteq M_{p} $  and choose a sequence $ (\phi_{n}) $  in $ \Phi $  such that $ \phi_{n}(p_{n}) \neq 0 $.
For $ n \in \N $  let $ \psi_{n} $  be the normal state
%% -- 
\[
\psi_{n}(x) = \phi_{n}(p_{n})^{-1}\phi_{n}(p_{n}xp_{n})
\]
%% --
on $ M $.
Because of $ s(\psi_{n}) \leq p_{n} \leq p $, the support projections of the $ \psi_{n} $'s are mutually orthogonal in $ M $.
For $ \mu \in \R_{+} $  and $ x \in M $  we obtain
%% --
\[
\begin{aligned}
\langle x,\mu R(\mu)\psi_{n}\rangle &= \phi_{n}(p_{n})^{-1}\langle\mu p_{n}(R(\mu)'x)p_{n},\phi_{n}\rangle = \\
&= \phi_{n}(p_{n})^{-1}\langle\mu p_{n}p(R(\mu)p'x)p_{n},\phi_{n}\rangle = \\
&= \phi_{n}(p_{n})^{-1}\langle\mu p_{n}(pR_{1}(\mu)'xp)p_{n},\phi_{n}\rangle = \\
&= \phi_{n}(p_{n})^{-1}\langle\mu(p_{n}R_{1}(\mu)'xp_{n}),\phi_{n}\rangle = \\
&= \phi_{n}(p_{n})^{-1}\phi_{n}(x) = \psi_{n}(x).
\end{aligned}
\]
%% --
Therefore $ \psi_{n} \in \Fix{\RR} $  for all $ n \in \N $.
\end{proof}
%% --
\begin{remark}\label{rem:d4-4.2}
\begin{enumerate}[\upshape (i), wide, labelindent=.5em]
\item
If $ \dim \Fix{\RR} \geq 2 $  then the Jordan decomposition of self adjoint linear functionals implies that at least two states in $ \Fix{\RR} $  have orthogonal support (compare D-III, Theorem 1.10.(i)).

\item
If $ \RR $  is a pseudo-resolvent with values in a \WA-algebra such that $ \Fix{\RR'} $  is contained in $ M_{*} $,  then by D-III, Lemma 1.2, there exists a sequence of normal states in $ \Fix{\RR'} $  with orthogonal supports in $ M $.
\end{enumerate}
\end{remark}
%% --
\begin{lemma}\label{lem:d4-4.3}
Let $ \RR $  be an identity preserving pseudo-resolvent of Schwarz type on $ D = \{\lambda \in \C \colon  \Re(\lambda) > 0\} $  with values in the predual of a \WA-algebra $ M $.
If the fixed space of the canonical extension $ \widehat{\RR} $  of\/ $ \RR $  to some ultrapower of\/ $ M_{*} $  is infinite dimensional, then there exists a sequence $ (z_{n}) $  in $ M_{1}^{+} $  and a sequence of states $ (\phi_{n}) $  in $ M_{*} $  such that
%% --
\begin{enumerate}[\upshape (i)]

\item\label{item:d4-4.3-i}
$ \lim_{n} z_{n} = 0 $  in the $ s^{*}(M,M_{*}) $-topology,

\item\label{item:d4-4.3-ii}
$ \lim_{n} \|(\Id - \lambda R(\lambda))\phi_{n}\| = 0 $  for all $ \lambda \in D $,

\item\label{item:d4-4.3-iii}
$ \phi_{n}(z_{n}) \geq \frac{1}{2} $  for all $ n \in \N $. 
\end{enumerate}
\end{lemma}
%% --
\begin{proof}
Let $ (M_{*})^{\wedge} $  be the ultrapower of $ M_{*} $  with respect to some free ultrafilter $ \mathfrak{U} $  on $ \N $.
Since $ (M_{*})^{\wedge} $  is the predual of a \WA-subalgebra of $ \widehat{M} $  
(see D-III, Remark 2.4.(ii)), there exists a sequence of states $ (\hat{\psi}_{n}) $  in $ \Fix{\RR}^{\wedge} $  such that the corresponding support projections are mutually orthogonal in $ \widehat{M} $  (Lemma 4.1).
For every $ n \in \N $  let $ (\psi_{n,k}) $  be a representing sequence of states, 
%% --
\[
\phi \coloneqq \sum_{n,k} 2^{-(n+k+1)} \psi_{n,k}
\]
%% --
and 
%% --
\[
p \coloneqq \sup\{s(\psi_{n,k}): n,k=1,\ldots\}
\]
%% --
in $ M $.
Then $ \phi $  is a normal state on $ M $  which is faithful on the \WA-algebra $ M_{p} $.
Since 
%% --
\[
1 = \langle\psi_{n,k},s(\psi_{n,k})\rangle = \psi_{n,k}(p) \quad (n,k \in \N) ,
\]
%% --
it follows $ \hat{\psi}_{n}(\hat{p}) = 1 $  where $ \hat{p} $  is the canonical image of $ p $  in $ \widehat{M} $.
But this implies $ s(\hat{\psi}_{n}) \leq \hat{p} $  in $ \widehat{M} $.

Since $ \widehat{M}_{1}^{+} $  is $ \sigma(\widehat{M},\widehat{M}') $-dense in $ (\widehat{M}'')_{1}^{+} $  (Kaplansky's Density Theorem \citet[1.9.1]{sakai:1971} and \citet[1.8.9 and 1.8.12]{sakai:1971}), there exists for all $ n \in \N $  a net $ (z_{n,\gamma}) $  in $ \widehat{M}_{1}^{+} $  such that
%% --
\[
\sigma(\widehat{M}'',\widehat{M}')\text{-}\lim_{\gamma} \hat{z}_{n,\gamma} = s(\hat{\psi}_{n}).
\]
%% --
From \citet[1.7.8]{sakai:1971} and the above considerations, we obtain that the net $ (p\hat{z}_{n,\gamma}\hat{p}) $  converges to $ s(\hat{\psi}_{n}) $  in the $ \sigma(\widehat{M}'',\widehat{M}') $-topology.
Therefore we may assume $ \hat{z}_{n,\gamma} \in (\widehat{M}_{p}')_{1}^{+} $.

In the following we denote by $ \hat{\phi} $  the canonical image of $ \phi $  in $ (M_{*})^{\wedge} $.

Since the projections $ s(\hat{\psi}_{n}) $  are mutually orthogonal, there exists a real sequence $ (r_{n}) $, $ 0 < r_{n} < 1 $, $ \lim_{n} r_{n} = 0 $  and $ \hat{\phi}(s(\hat{\psi}_{n})) \leq \frac{1}{2}r_{n} $.
For all $ n \in \N $  choose $ \hat{z}_{n} \in (\widehat{M}_{p}')_{1}^{+} $  such that
%% --
\begin{align*}
|\langle\hat{\phi},s(\hat{\psi}_{n}) - \hat{z}_{n}\rangle| &\leq \frac{1}{2}r_{n}, \\
|\langle\hat{\psi}_{n},s(\hat{\psi}_{n}) - \hat{z}_{n}\rangle| &\leq \frac{1}{2}r_{n}.
\end{align*}
%% --
Hence $ \hat{\phi}(\hat{z}_{n}) \leq r_{n} $  and $ \hat{\psi}_{n}(\hat{z}_{n}) \geq \frac{1}{2} $  for all $ n \in \N $.
For every $ n \in \N $  let $ (z_{n,k}) \in \hat{z}_{n} $  be a representing sequence in $ (M_{p})_{1}^{+} = p(M_{1}^{+})p $  (note that $ {M}_{\hat{p}} = \widehat{M_{p}} $ ) and fix $ \mu \in \R_{+} $.
Since $ \mu R(\mu)'\hat{\psi}_{n} = \hat{\psi}_{n} $, $ \hat{\phi}(\hat{z}_{n}) \leq r_{n} $  and $ \hat{\psi}_{n}(\hat{z}_{n}) \geq \frac{1}{2} $,  there exists for all $ n \in \N $  an element $ U_{n} \in \mathfrak{U} $  such that for all $ k \in U_{n} $ and we obtain
%% --
\begin{enumerate}[label=(\roman*$^{\prime}$)]
\item
$ \phi(z_{n,k}) \leq r_{n} $,

\item
$ \|(Id - \mu R(\mu))\psi_{n,k}\| \leq r_{n} $,

\item
$ \psi_{n,k}(z_{n,k}) \geq \frac{1}{2} $. 
\end{enumerate}
%% --
Inductively we find a sequence $ (z_{n}) $  in $ (M_{p})_{1}^{+} $  and a sequence of states $ (\phi_{n}) $  in $ M_{*} $  such that for all $ n \in \N $ 
%% --
\begin{enumerate}[label=(\roman*$^{\prime\prime}$)]
\item
$ \lim_{n} \phi_{n}(z_{n}) = 0 $, 

\item%\label{item:d4-4.3-ii}
$ \lim_{n} \|(Id - \mu R(\mu))\phi_{n}\| = 0 $, 

\item
$ \phi_{n}(z_{n}) \geq \frac{1}{2} $.
\end{enumerate}
%% --
But $ \phi $  is faithful on $ M_{p} $.
Therefore condition \ref{item:d4-4.3-ii} implies that $ \lim_{n} z_{n} = 0 $  in the $ s^{*}(M_{p},(M_{p})_{*}) $-topology (\citet[Proposition III.5.4]{takesaki:1979}).
Since 
%
\[
	s^{*}(M_{p},(M_{p})_{*}) = s^{*}(M,M_{*})|_{M_{p}} ,
\]
%
(i) follows immediately from \ref{item:d4-4.3-ii}.
Using the resolvent equation for $ \RR $  it is easy to see that \ref{item:d4-4.3-ii} implies
%% --
\[
\lim_{n} \|(Id - \lambda R(\lambda))\phi_{n}\| = 0
\]
%% --
for all $ \lambda \in D $  and the proof is complete.
\end{proof}
%% --
Without further comments, we will now use following facts.
%% --
\begin{enumerate}[(1)]
\item
A sequence $ (\phi_{n}) $  in $ M'_{+} $  converges in the $ \sigma(M',M) $-topology if and only if it converges in $ \sigma(M',M'') $-topology (\citet{akemann:1972}).

\item
We can decompose $ \phi \in M'_{+} $  into its normal and singular part $ \phi = \phi^{(n)} + \phi^{(s)} $, $ 0 \leq \phi^{(n)} \in M_{*} $, $ 0 \leq \phi^{(s)} \in M_*^{\perp} $  and $ \|\phi\| = \|\phi^{(n)}\| + \|\phi^{(s)}\| $  (\citet[Theorem III.2.14]{takesaki:1979}).

\item
If $ (\phi_{k}) $  is a sequence in $ M_{*} $  converging to zero in the $ \sigma(M_{*},M) $-topology and if $ (x_{n}) $  is a sequence in $ M $  converging to zero in the $ s^{*}(M,M_{*}) $-topology, then $ \lim_{n} \phi_{k}(x_{n}) = 0 $  uniformly in $ k \in \N $  (\citet[Lemma III.5.5]{takesaki:1979}).
\end{enumerate}
%% --
\begin{theorem}\label{thm:d4-4.4}
Let $ \RR $  be an identity preserving pseudo-resolvent on 
%
\[
	D = \{\lambda \in \C \colon  \Re(\lambda) > 0\} 
\]
%
with values in a \WA-algebra $ M $  which is of Schwarz type and let $ \RR' $  br its adjoint pseudo-resolvent.
Any one of the following conditions implies $ \dim \Fix{\widehat{\RR}} < \infty $  in some ultrapower of $ M $.
%% --
\begin{enumerate}[\upshape (i)]
\item
The fixed space of $ \RR' $  is finite dimensional.

\item
$ \lim_{\mu \to 0} \mu R(\mu) = P $  exists in the strong operator topology and $ \rank (P) < \infty $.

\item
The fixed space of\/ $ \RR' $  is contained in $ M_{*} $.

\item
Every map $ \mu R(\mu) $, $ \mu \in \R_{+} $, is irreducible on $ M $.
\end{enumerate}
\end{theorem}
%% --
\begin{proof}
Suppose that the dimension of the fixed space of\/ 
$ (\RR')^{\wedge} $  in some ultrapower $ (M')^{\wedge} $  of $ M' $  is infinite dimensional.
Since $ M' $ is the predual of the \WA-algebra $ M'' $  and $ \RR'' $  is identity preserving (since $ R''\1 = R\1 = \1 $ ) and of Schwarz type (because $ \mu R''(\mu) = (\mu R(\mu))'' $  is a Schwarz map for all $ \mu \in \R_{+} $ ), we can apply Lemma~\ref{lem:d4-4.3}.

%Suppose that the fixed space of the canonical extension of $ \RR' $  to some ultrapower of $ M' $  is infinite dimensional.
Thus we may choose a sequence of states $ (\phi_{k}) $  in $ M' $  and a sequence $ (z_{k}) $  in $ (M'')_{1}$, $ 0 \leq z_{k} $,  satisfying (i)--(ii) of Lemma 4.3.
Remark~(3) above implies that no subsequence of $ (\phi_{k}) $  can converge in the $ \sigma(M',M'') $-topology.
%% --
\begin{enumerate}[\upshape (i), wide, labelindent=.5em]
\item
If $ \phi $  is a $ \sigma(M',M) $-accumulation point of $ (\phi_{k}) $, then $ \phi \in \Fix{\RR'} $.
Since $ \Fix{\RR'} $  is finite dimensional, the set of accumulation points of the sequence $ (\phi_{k}) $  is metrizable in the $ \sigma(M',M) $-topology.
Hence there exists a sequence $ (k(n)) $  of natural numbers such that 
%% --
\[
    \sigma(M',M)\text{-}\lim_{n} \phi_{k(n)} = \phi .
\]
%% --
Consequently, by Remark (1) above, 
%% --
\[ 
    \sigma(M',M'')\text{-}\lim_{n} \phi_{k(n)}= \phi .
\]
%% --
But this leads to a contradiction proving (i).

\item
Since $ \dim \Fix{\RR} = \dim \Fix{\RR'} = \rank(P) < \infty $, (ii) follows from (i).

\item
Suppose that the fixed space of $ R' $  is infinite dimensional.
Since $ \Fix{\RR'} \subseteq M_{*} $,  there exists a sequence of states $ (\psi_{n}) $  in $ \Fix{\RR'} $  with mutually orthogonal support projections in $ M $  (Lemma~\ref{lem:d4-4.1}).
Since every $ \sigma(M',M) $-accumulation point of the $ \psi_{n} $'s belongs to $ \Fix{\RR'} $, hence is normal, the sequence $ (\psi_{n}) $  is relatively $ \sigma(M_{*},M) $-compact.

By Eberlein's theorem, we may assume that this sequence is weakly convergent (\citet{schaefer:1966}).
By the orthogonality of the $ s(\psi_{n}) $'s this sequence converges to zero in the $ s^{*}(M,M_{*}) $-topology, hence $ \lim_{n} \psi_{k}(s(\psi_{n})) = 0 $  uniformly in $ k \in \N $, a contradiction.
Consequently $ \dim \Fix{\RR} < \infty $  and (iii) is proved.

\item
We prove $ \dim \Fix{\RR'} = 1 $  and apply (i) once again and need the following observation: If $ \psi $  is a faithful state on $ M $,  then the normal part is faithful too.
Indeed, if $ 0 \neq x \in M $  such that $ \psi^{(n)}(x) = 0 $,  choose a projection $ 0 \neq p \in M $  such that $ \psi^{(n)}(p) = \psi^{(s)}(p) = 0 $ (use \citet[Theorem III.3.8]{takesaki:1979}). 
Hence $ \psi(p) = 0 $  which conflicts with the faithfulness of $ \psi $.

If $ 2 \leq \dim \Fix{\RR'} $  there are states $ \psi_{1} $  and $ \psi_{2} $  in $ \Fix{\RR'} $  such that the corresponding support projections are orthogonal in $ M'' $  (Remark~\ref{rem:d4-4.2}).
Since every $ \RR' $-invariant state $ \psi $  is faithful on $ M $, $ \psi_{i}^{(n)} \neq 0 $  (otherwise the norm closed face $ \{\psi(x) = 0: x \in M_{+}\} $  would
be non trivial and $ \mu R(\mu) $-invariant).
The support projections of the $ \psi_{i}^{(n)} $'s in $ M'' $  are orthogonal (since $ \psi_{1}^{(n)} \leq \psi_{i} $ ) and different from zero.
Let $ (z_{\gamma}) $  be a net in $ M_{1}^{+} $  such that
%% --
\[
	\sigma(M'',M')\text{-}\lim_{\gamma} z_{\gamma} = s(\psi_{1}^{(n)}).
\]
%% --
Then $ \lim_{\gamma} \psi_{1}^{(n)}(z_{\gamma}) = 1 $  but $ \lim_{\gamma} \psi_{2}^{(n)}(z_{\gamma}) = 0 $.
Let $ z $  be a $ \sigma(M,M_{*}) $-accumulation point of $ (z_{\gamma}) $  in $ M_{+} $.
Since every $ \psi_{i}^{(n)} $  is normal, $ \psi_{1}^{(n)}(z) = 1 $  but $ \psi_{2}^{(n)}(z) = 0 $.
The first condition implies $ z \neq 0 $  while the second shows that $ \psi_{2}^{(n)} $  cannot be faithful.
This is a contradiction and it implies $ \dim \Fix{\RR'} = 1 $, hence (iv) is proved.
\end{enumerate}
%% --
\end{proof}
%% --
The next corollary is an easy application of\/ Theorem~\ref{thm:d4-4.4} and of D-III, Proposition 2.3.
%% --
\begin{corollary}\label{cor:d4-4.5}
Let $ \TT $  be an identity preserving semigroup of Schwarz type on the predual of a \WA-algebra $ M $.
Then the following assertions are equivalent.
%% --
\begin{enumerate}[\upshape (a)]
\item
$ \TT $  is uniformly ergodic with finite dimensional fixed space.

\item
The adjoint weak*-semigroup is strongly ergodic with finite dimensional fixed space.

\item
Every $ \TT'' $-invariant state is normal.
\end{enumerate}
\end{corollary}
%% -- 
\begin{proof}
If (a) is fulfilled, then the semigroup $ \TT $  is strongly ergodic on $ M_{*} $.
Since
%% --
\[
\dim \Fix{\TT} = \dim \Fix{\TT'} < \infty,
\]
%% --
there exist normal states $ \phi_{1},\ldots,\phi_{n} $  in $ \Fix{\TT} $  and $ x_{1},\ldots,x_{k} $  in $ \Fix{\TT'} $  such that $ \phi_{n}(x_{m}) = \delta_{n,m} $  $ (1 \leq n, m \leq k) $.
Then 
%% --
\[
P = \sum_{i=1}^{k} \phi_{i} \otimes x_{i} 
\]
%% --
is the associated ergodic projection.
If $ (C(s))_{s>0} $  is the family of Cesàro means of $ \TT $, then
%% --
\[
\lim_{s \to \infty} C(s)''(\psi) = \sum_{i=1}^{k} \phi_{i}(\psi)x_{i} \in M_{*}
\]
%% --
for every $ \psi \in M' $.
Hence $ \Fix{\TT''} \subseteq M_{*} $  which implies (c).

If (c) is fulfilled, then $ \Fix{\TT'} = \Fix{\TT''} $.
Therefore the fixed space of $ \TT' $  separates the points of $ \Fix{\TT''} $, hence $ \TT' $  is strongly ergodic on $ M $  (\citet[Chap.2, Thm.1.4]{krengel:1985}).

If (b) holds, then
%% --
\[
P = \lim_{\mu \to 0} \mu R(\mu,A')
\]
%% --
exists in the strong operator topology with $ A' $  is the generator of $ \TT' $.
Therefore $ \dim \Fix{\widehat{\mu R(\mu)}} < \infty $  in some ultrapower of $ M $  (Theorem~\ref{thm:d4-4.4}).
It follows from D-III, Proposition 2.3 that 0 is a pole of the resolvent of $ R(\cdot,A) $.
Therefore $ \TT $  is uniformly ergodic.
\end{proof}
%% --
%\clearpage
\section*{Notes}\label{notes:d4-notes}
\addcontentsline{toc}{section}{Notes}
%% --
\begin{enumerate}[label=\emph{Section \arabic*:}, wide, itemsep=1ex]
\item
The stability concepts appearing in Theorem~\ref{thm:d4-1.7} coincide not only for positive semigroups on \CA-algebras but on any order unit Banach space.
We refer to \citet{battyrobinson:1984} for this more general setting and to B-IV, Section 1 for the analogous results on $ C_{0}(X) $.


\item
Theorem~\ref{thm:d4-2.2} generalizes the Liapunov stability theorem from the matrix algebra $ B(\C^{n}) $  to arbitrary \WA-algebras.
For the algebra $ \BH $  it is due to \citet{milstein:1975} and in the general form to \citet{grohneubrander:1981}.


\item
From the many papers dealing more or less explicitly with the asymptotic behavior of semigroups on operator algebras we quote \citet{frigerioverri:1982} and \citet{watanabe:1982}.
The background for our ergodic theorems (Proposition~\ref{prop:d4-3.3} \& \ref{prop:d4-3.4}) can be found best in \citet{krengel:1985}.
The \enquote{automatic} convergence theorem for an irreducible \WA-dynamical system on $ \BH $  stated in Corollary~\ref{cor:d4-3.9} is the continuous version of a result in \citet{groh:1984c}.
Finally, the characterization of convergence towards a periodic semigroup through spectral properties of the generator---Theorem~\ref{thm:d4-3.11}---is due to \citet{nagel:1984} in the commutative case, \ie in $ L^{1}(\mu) $ (see also C-IV, Thm.2.14).
%\medskip

\item
Again we refer to \citet{krengel:1985} for the (uniform) ergodic theory for a single operator or a one-parameter semigroup on a Banach space.
The characterization given in Corollary~\ref{cor:d4-4.5} for positive semigroups on \WA-algebras is based on a sophisticated use of ultrapower techniques and has its discrete forerunners in \citet{lotz:1981} and 
\citet{groh:1984b}.
%\medskip
\end{enumerate}

%% -- Literatur
%% --
\section*{References}
\addcontentsline{toc}{section}{References}
{\RaggedRight
\renewcommand{\bibsection}{}
\bibliographystyle{abbrvnat}
\bibliography{bib/ln-references}}

