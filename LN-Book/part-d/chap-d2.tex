% !TEX root = ../../LN-Book.tex
%% -- Stand 2025/01/17
%% -- ulgr
%% -- Part D2
%% --

%% -- Chapter D-II
%% --

\chapter{Characterization of Positive Semigroups on \WA-Algebras}\label{chap:D-II}


Since the positive cone of a \CA-algebra has non-empty interior many results of Chapter B-II can be applied verbatim to the characterization of the generator of positive semigroups on \CA-algebras.
On the other hand a concrete and detailed representation of such generators has been found only in the uniformly continuous case (see \citet{lindblad:1976}).
A third area of active research has been the following: Which maps on \CA-algebras (in particular, which derivations) commuting with certain automorphism groups are automatically generators of strongly continuous positive semigroups.
For more information we refer to the survey article of \citet{evans:1984}.

\section{Semigroups on Properly Infinite W*-Algebras}

The aim of this section is to show that strongly continuous semigroups of Schwarz maps on properly infinite W*-algebras are already uniformly continuous.
In particular, our theorem is applicable to such semigroups on $B(H)$.

It is worthwhile to remark, that the result of \citet{lotz:1985} on the uniform continuity of every strongly continuous semigroup on $L^\infty$ (see A-II, Sec.3) does not extend to arbitrary W*-algebras.
%% --
\begin{example}
Take $M = B(H)$, $H$ infinite dimensional, and choose a projection $p \in M$ such that $Mp$ is topologically isomorphic to $H$.
Therefore $M = H \oplus M_{0}$, where $M_{0} = \ker(x \mapsto xp)$.
Next take a strongly, but not uniformly continuous, semigroup $S$ on $H$ and consider the strongly continuous semigroup $S \oplus \Id$ on $M$.
\end{example}
%% --
For results from the classification theory of \WA-algebras needed in our approach we refer to \cite[2.2]{sakai:1971} and \cite[V.1]{takesaki:1979}.

\begin{theorem}\label{thm:schwarz_uniform}
Every strongly continuous one-parameter semigroup of Schwarz type on a properly infinite W*-algebra $M$ is uniformly continuous.
\end{theorem}

\begin{proof}
Let $T = (T(t)_{t \geq 0})$ be strongly continuous on $M$ and suppose $T$ not to be uniformly continuous.
Then there exists a sequence $(T_n) \subset T$ and $\varepsilon > 0$ such that $\|T_n - \Id\| \geq \varepsilon$ but $T_n \to \Id$ in the strong operator topology.
We claim that for every sequence $(P_k)$ of mutually orthogonal projections and all bounded sequences $(x_k)$ in $M$
%% --
\[
\lim_n \|(T_n - \Id)(P_k x_k P_k)\| = 0
\]
%% --
uniformly in $k \in \mathbb{N}$.
This follows from an application of the \emph{Lemma of Phillips} and the fact that the sequence $(P_{k} x_{k} P_{k})$ is summable in the $s^{*}(M,M_*)$-topology (compare Elliot (1972)).

Let $(P_k)$ be a sequence of mutually orthogonal projections in $M$ such that every $P_k$ is equivalent to $1$ via some $u_k \in M$ \cite[2.2]{sakai:1971}.
Without loss of generality we may assume $\|(T_n - \Id)(u_n)\| \leq n^{-1}$ since the semigroup $T$ is strongly continuous.
Thus we obtained the following:

\begin{enumerate}[(i)]
\item 
$\lim_n \|(T_n - \Id)(P_k x_k P_k)\| = 0$ uniformly in $k \in \mathbb{N}$ for every bounded sequence $(x_k)$ in $M$.
\item 
Every projection $P_k$ is equivalent to $1$ via some $u_k \in M$.
\item 
$\|(T_n - \Id)u_n\| \leq n^{-1}$ for all $n \in \mathbb{N}$.
\end{enumerate}

For the following construction see A-I,3.6 and D-II,Sec.2.
Let 

\begin{enumerate}[(i)]
\item
$\widehat{M}$ be an ultrapower of $M$,

\item
let $p := \widehat{(P_{k})} \in \widehat{M}$,

\item
$T := \widehat{(T_{n}) }\in L(\widehat{M})$

\item
 and $u := \widehat{(u_{k})}  \in \widehat{M}$.

\end{enumerate}
%% --  
Then $T$ is identity preserving and of Schwarz type on $\widehat{M}$.
Since $u^*u = p$ and $uu^* = 1$ it follows $pu^* = u^*$ and $(uu^*)x(uu^*) = x$ for all $x \in \widehat{M}$.
Finally, $T(pxp) = pxp$ for all $x \in \widehat{M}$, which follows from (i), and $T(u^*) = T(pu^*) = pu^* = u^*$ and $T(u) = u$, which follows from (iii).
Using the Schwarz inequality we obtain
%% --
\[
T(uu^*) = T(1) \leq 1 = uu^* = T(u)T(u)^*.
\]
%% --
Using D-III, Lemma 1.1. we conclude $T(ux) = uT(x)$ and $T(xu^*) = T(x)u^*$ for all $x \in \widehat{M}$.
Hence

\begin{align*}
T(x) &= T(uu^*xuu^*) = uT(u^*xu)u^* = uT(pu^*xup)u^* \\
&= upu^*xupu^* = uu^*xuu^* = x
\end{align*}

for all $x \in \widehat{M}$.
From this we obtain that for every bounded sequence $(x_k)$ in $M$

\[
\lim_m \|T_m x_m - x_m\| = 0
\]

for some subsequence of the $T_n$'s and of the $x_k$'s.
This conflicts with our assumption at the beginning, hence the theorem is proved.
\end{proof}

%% --
%\bibliographystyle{abbrvnat}
%\bibliography{ln-references}

\bibliographystyle{spmpsci}
\begin{thebibliography}{5}
\providecommand{\natexlab}[1]{#1}
\providecommand{\url}[1]{\texttt{#1}}
\expandafter\ifx\csname urlstyle\endcsname\relax
  \providecommand{\doi}[1]{doi: #1}\else
  \providecommand{\doi}{doi: \begingroup \urlstyle{rm}\Url}\fi

\bibitem[Evans(1984)]{evans:1984}
D.~Evans.
\newblock Quantum dynamical semigroups, symmetry groups, and locality.
\newblock \emph{Acta Applicandae Mathematicae}, 2:\penalty0 333--352, 1984.

\bibitem[Lindblad(1976)]{lindblad:1976}
G.~Lindblad.
\newblock On the generators of quantum dynamical semigroups.
\newblock \emph{Comm. Math. Phys.}, 48:\penalty0 119--130, 1976.

\bibitem[Lotz(1985)]{lotz:1985}
H.~P. Lotz.
\newblock Uniform convergence of operators on {$L^{\infty}$} and similar
  spaces.
\newblock \emph{Math. Z.}, 190:\penalty0 207--220, 1985.

\bibitem[Sakai(1971)]{sakai:1971}
S.~Sakai.
\newblock \emph{{C$^{*}$}-Algebras and {$W^*$}-Algebras}.
\newblock Springer, Berlin-Heidelberg-New York, 1971.

\bibitem[Takesaki(1979)]{takesaki:1979}
M.~Takesaki.
\newblock \emph{Theory of Operator Algebras {I}}.
\newblock Springer, New York-Heidelberg-Berlin, 1979.

\end{thebibliography}