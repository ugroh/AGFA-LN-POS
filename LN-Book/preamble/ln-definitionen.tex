%% -- Definitionen
%% -- Stand 2025/02/14
%% -- var-Symbole für griechisches Alphabet
%% --  Makro zum Tauschen von Symbolen 
%% --

\newcommand{\swapsymbols}[2]{%
  \let\temporaryhold#1%
  \let#1#2%
  \let#2\temporaryhold%
}

%% --  Griechische Buchstaben tauschen
\swapsymbols{\varepsilon}{\epsilon}
\swapsymbols{\varrho}{\rho}
\swapsymbols{\vartheta}{\theta}
\swapsymbols{\varphi}{\phi}

%% --  Vergleichsoperatoren tauschen
\swapsymbols{\leqslant}{\leq}
\swapsymbols{\geqslant}{\geq}


%% -- Zahlen
\newcommand{\N}{\mathbb{N}}		% Natuerliche Zahlen
\newcommand{\Z}{\mathbb{Z}}		% Ganze Zahlen
\newcommand{\Q}{\mathbb{Q}}		% Rationale Zahlen
\newcommand{\R}{\mathbb{R}}		% Reelle Zahlen
\newcommand{\C}{\mathbb{C}}		% Komplexe Zahlen
\newcommand{\K}{\mathbb{K}}		% Koerperzeichen
\newcommand{\T}{\mathbb{T}}		% Torus

%% -- Operatoren					
\DeclareMathOperator{\Id}{Id}
\DeclareMathOperator{\sign}{sign}				% signum-Operator
\newcommand{\im}{\ensuremath{\mathrm{i}}}		% imaginäre Konstante i


%% -- Differential Allgemeine Abkuerzungen
%% --
\newcommand*{\diff}[1]{\mathop{}\!\mathrm{d}{#1}}	% \diff{s} = ds, 
\renewcommand*{\d}[1]{\mathop{}\!\mathrm{d}{#1}}	% \d{\mu} = d.., 
\newcommand*{\ds}{\mathop{}\!\mathrm{d}{s}}       	% \ds = ds, 
\newcommand*{\dg}{\mathop{}\!\mathrm{d}{g}}       	% \dg = dg, 
\newcommand*{\dt}{\mathop{}\!\mathrm{d}{t}}       	% \dt = dt, 
\newcommand*{\dx}{\mathop{}\!\mathrm{d}{x}}       	% \dx = dx, 
\newcommand*{\dy}{\mathop{}\!\mathrm{d}{y}}       	% \dy = dy, 
\newcommand*{\dr}{\mathop{}\!\mathrm{d}{r}}       	% \dr = dr, 
\newcommand*{\dm}{\mathop{}\!\mathrm{d}{m}}       	% \dm = dm, 

%% --
\newcommand{\LE}{\mathcal{L}(E)} 		% 
\renewcommand{\L}[1]{\mathcal{L}(#1)} 	% \L{C(K)} zum Beispiel
\newcommand{\BH}{\mathcal{B}(H)} 		% 

\newcommand{\Fix}[1]{\mathop{}\mathrm{Fix}(#1)}		% \Fix{T} Fixraum
\newcommand{\rank}{\mathop{}\!\mathrm{rank}\,}		% 
\newcommand{\Kern}[1]{\mathop{}\mathrm{Ker}(#1)}


%% -- Enpunkt korrekt (momentan ignorieren)
\newcommand*{\eg}{e.g.,\xspace}
\newcommand*{\ie}{i.e.,\xspace}
\newcommand*{\resp}{resp.\xspace}
\newcommand*{\vs}{vs.\xspace}
\newcommand*{\cf}{cf.\xspace}

%% --
\newcommand{\CA}{$\mathrm{C}^{*}$}	% C*-Algebra
\newcommand{\WA}{$\mathrm{W}^{*}$}	% W*-Algbera

%%
\newcommand*{\TT}{\mathcal{T}}		% semigroup
\newcommand*{\RR}{\mathcal{R}}		% Pseudo Resolvente
\renewcommand*{\SS}{\mathcal{S}}		% Semigroup
\newcommand*{\F}{\mathcal{F}}		% F-product
       
%% -- Angepasstes E/F und 1/3
%% -- 1/3

\newcommand*{\sfrac}[2]{\leavevmode\kern.1em%
       \raise.5ex\hbox{\scriptsize #1}%
       \kern-.1em/\kern-.15em%
       \lower.25ex\hbox{\scriptsize #2}}

\newcommand{\nfrac}[2]{\leavevmode\kern.1em%
       \raise.5ex\hbox{\scriptsize #1}%
       \kern-.1em/\kern-.15em%
       \lower.25ex\hbox{\scriptsize #2}}
       
%% -- E/F im Textmodus, sonst \sfrac{E}{F}
%% -- \trfac{E}{F}
%% -- 
\renewcommand{\tfrac}[2]{\raisebox{0.5ex}{#1}\big/\raisebox{-0.5ex}{#2}}
%%

%% -- Eins einer Algebra oder \1_{A} charakteristische Funktion
%% -- 

\RequirePackage{bbm}
\newcommand{\1}{\mathbbm{1}}



