%% -- Chapter A-II
%% --

\chapter{Characterization of Semigroups on Banach Spaces\\
W. Arendt \& H. Lotz}\label{chap:A-II}

%In this chapter two different problems are treated:
%
%\begin{enumerate}[(i)]
%\item to characterize generators of strongly continuous semigroups;
%\item to characterize various properties of strongly continuous semigroups in terms of their generators.
%\end{enumerate}
%
%In Section 1 the first problem is solved by finding conditions on the Cauchy problem associated with $A$ and also by finding conditions on the resolvent of $A$.
%The second problem is treated for a hierarchy of smoothness properties of the semigroup.
%
%Contraction semigroups are considered in Section 2.
%Here, the first problem has a simple and extremely useful solution: A densely defined operator $A$ is generator of a contraction semigroup if and only if $A$ is dissipative and satisfies a range condition.
%
%Our approach is quite general.
%We do not only consider contractions with respect to the norm but also with respect to \enquote{half-norms}.
%This will allow us to obtain results on positive contraction semigroups simultaneously by choosing a suitable half-norm (cf.\ C-II,Sec.1).
%
%The last section contains a surprising result: on certain Banach spaces (e.g., $L^{\infty}$) only bounded operators are generators of strongly continuous semigroups.
%
%\section{The Abstract Cauchy Problem, Special Semigroups and Perturbation}
%
%Linear differential equations in Banach spaces are intimately connected with the theory of one-parameter semigroups.
%In fact, given a closed linear operator $A$ with dense domain $D(A)$ the following statement is true (with some reservation regarding a technical detail): The abstract Cauchy problem
%%% -- 
%\[
%\begin{aligned}
%\dot{u}(t) &= Au(t) \quad (t \geq 0) \\
%u(0) &= f
%\end{aligned}
%\]
%%% -- 
%has a unique solution for every $f \in D(A)$ if and only if $A$ is the generator of a strongly continuous semigroup.
%
%This is one characterization of generators which illustrates their important role for applications.
%The fundamental Hille-Yosida theorem gives a different characterization in terms of the resolvent and yields a powerful tool for actually proving that a given operator is the generator of a semigroup.
%
%Another problem we will treat here is how diverse properties of a semigroup can be described in terms of its generator.
%This is a reasonable question from the theoretical point of view (since the generator uniquely determines the semigroup).
%It is of interest from the practical point of view as well: the generator is the given object, defined by the differential equation.
%It is useful to dispose of conditions of the generator itself giving information on the solutions (which might not be known explicitly).
%We discuss smoothness properties such as analyticity, differentiability, norm continuity and compactness of the semigroup.
%
%A frequent method to obtain new generators out of a given one is by perturbation.
%We will have a brief look at this circle of problems at the end of this section.
%
%The results are explained and illustrated by examples.
%Proofs are only given when new aspects are presented which are not yet contained in the literature, otherwise we refer to the recent monographs Davies (1980), Goldstein (1985a), Pazy (1983).
%
%% chapter-A-II-1-section1.tex
%
%\subsection{The Abstract Cauchy Problem}\label{sec:acp}
%
%Let $A$ be a closed operator on a Banach space $E$ and consider the abstract Cauchy problem
%%% -- 
%\[
%\begin{aligned}
%\text{(ACP)} \quad \begin{cases}
%\dot{u}(t) &= Au(t) \quad (t \geq 0) \\
%u(0) &= f.
%\end{cases}
%\end{aligned}
%\]
%%% -- 
%By a solution of (ACP) for the initial value $f \in D(A)$ we understand a continuously differentiable function $u : [0,\infty) \to E$ satisfying $u(0) = f$ and $u(t) \in D(A)$ for all $t \geq 0$ such that $\dot{u}(t) = Au(t)$ for $t \geq 0$.
%
%By A-I,Thm.1.7 there exists a unique solution of (ACP) for all initial values in the domain $D(A)$ whenever $A$ is the generator of a strongly continuous semigroup.
%The converse does not hold (see Example 1.4.\ below).
%However, for the operator $A_{1}$ on the Banach space $E_{1} = D(A)$ (see A-I,3.5) with domain $D(A_{1}) = D(A^{2})$ given by $A_{1}f = Af$ $(f \in D(A_{1}))$ the following holds.
%
%\begin{theorem}\label{thm:1.1}
%The following assertions are equivalent.
%\begin{enumerate}[(a)]
%\item For every $f \in D(A)$ there exists a unique solution of (ACP).
%\item $A_{1}$ is the generator of a strongly continuous semigroup.
%\end{enumerate}
%\end{theorem}
%
%\begin{proof}
%(i) implies (ii).
%Assume that (i) holds; i.e., for every $f \in D(A)$ there exists a unique solution $u(\cdot,f) \in C^{1}([0,\infty),E)$ of (ACP).
%For $f \in E_{1}$ define $T_{1}(t)f := u(t,f)$ $(t\geq0)$.
%By the uniqueness of the solutions it follows that $T_{1}(t)$ is a linear operator on $E_{1}$ and $T_{1}(s+t) = T_{1}(s)T_{1}(t)$.
%Moreover, since $u(\cdot,f) \in C^{1}$, it follows that $t \mapsto T_{1}(t)f$ is continuous from $[0,\infty)$ into $E_{1}$.
%We show that $T_{1}(t)$ is a continuous operator for all $t>0$.
%
%Let $t>0$.
%Consider the mapping $\eta: E_{1} \to C([0,t],E_{1})$ given by $\eta(f) = T_{1}(\cdot)f = u(\cdot,f)$.
%We show that $\eta$ has a closed graph.
%In fact, let $f_{n} \to f$ in $E_{1}$ and $\eta(f_{n}) = u(\cdot,f_{n}) \to v$ in $C([0,t],E_{1})$.
%Then $u(s,f_{n}) = f_{n} + \int_0^s Au(r,f_{n})dr$.
%Letting $n\to\infty$ we obtain $v(s) = f + \int_0^s Av(r)dr$ for $0 \leq s \leq t$.
%Let $\tilde{v}(s) = T_{1}(s-t)v(t)$ for $s > t$, and $\tilde{v}(s) = v(s)$ for $0 \leq s \leq t$.
%% continuation of chapter-A-II-1-section1.tex
%
%Then $\tilde{v}$ is a solution of (ACP).
%It follows that $\tilde{v}(s) = T_{1}(s)f$ for all $s \geq 0$.
%Hence $v = \eta(f)$.
%We have shown that $\eta$ has a closed graph and so $\eta$ is continuous.
%This implies the continuity of $T_{1}(t)$.
%It remains to show that $A_{1}$ is the generator of $(T_{1}(t))_{t\geq0}$.
%
%We first show that for $f \in D(A^{2})$ one has
%%% -- 
%\[
%AT_{1}(t)f = T_{1}(t)Af. \tag{1.1}
%\]
%%% -- 
%
%In fact, let $v(t) = f + \int_{0}^{t} u(s,Af) ds$.
%Then $\dot{v}(t) = u(t,Af) = Af + \int_{0}^{t} Au(s,Af) ds = A(f + \int_{0}^{t} u(s,Af) ds) = Av(t)$.
%Since $v(0) = f$, it follows that $v(t) = u(t,f)$.
%Hence $Au(t,f) = Av(t) = \dot{v}(t) = u(t,Af)$.
%This is (1.1).
%
%Now denote by $B$ the generator of $(T_{1}(t))_{t\geq0}$.
%For $f \in D(A^{2})$ we have
%%% -- 
%\[
%\lim_{t \to 0} \frac{T_{1}(t)f - f}{t} = Af
%\]
%%% -- 
%and by (1.1),
%%% -- 
%\[
%\lim_{t \to 0} A \frac{T_{1}(t)f - f}{t} = \lim_{t \to 0} \frac{T_{1}(t)Af - Af}{t} = A^{2}f \text{ in the norm of } E.
%\]
%%% -- 
%
%Hence $\lim_{t \to 0} \frac{T_{1}(t)f - f}{t} = Af$ in the norm of $E_{1}$.
%
%This shows that $A_{1} \subset B$.
%In order to show the converse, let $f \in D(B)$.
%Then $\lim_{t \to 0} A \frac{T_{1}(t)f - f}{t}$ exists in the norm of $E$.
%Since $\lim_{t \to 0} \frac{T_{1}(t)f - f}{t} = Af$ in the norm of $E$, it follows that $Af \in D(A)$, since $A$ is closed.
%Thus $f \in D(A^{2}) = D(A_{1})$.
%We have shown that $B = A_{1}$.
%
%(ii) implies (i).
%Assume that $A_{1}$ is the generator of a strongly continuous semigroup $(T_{1}(t))_{t\geq0}$ on $E_{1}$.
%Let $f \in D(A)$ and set $u(t) = T_{1}(t)f$.
%Then $u \in C([0,\infty),E)$ and $Au(\cdot) \in C([0,\infty),E)$.
%Moreover, $\int_{0}^{t} u(s)ds = \int_{0}^{t} T_{1}(s)fds \in D(A_{1}) = D(A^{2})$ and $A\int_{0}^{t} u(s)ds = u(t) - u(0) = u(t) - f$ (by A-I, (1.3)).
%Consequently, $u(t) = f + A\int_{0}^{t} u(s)ds = f + \int_{0}^{t} Au(s)ds$.
%Hence $u \in C^{1}([0,\infty),E)$ and $\dot{u}(t) = Au(t)$.
%Thus $u$ is a solution of (ACP).
%
%% continuation of chapter-A-II-1-section1.tex
%
%In order to show uniqueness, assume that $u$ is a solution of (ACP) with initial value $0$.
%We have to show that $u \equiv 0$.
%Let $v(t) = \int_{0}^{t} u(s)ds$.
%Then $v(t) \in D(A)$ and $Av(t) = \int_{0}^{t} Au(s)ds = \int_{0}^{t} \dot{u}(s)ds = u(t) \in D(A)$.
%Consequently, $v(t) \in D(A^{2})$ for all $t\geq0$.
%Moreover, $\dot{v}(t) = u(t) = Av(t)$ and $\frac{d}{dt} Av(t) = Au(t) = A\dot{v}(t) = A^{2}v(t)$.
%Thus $v \in C^{1}([0,\infty),E_{1})$ and $\dot{v}(t) = A_{1}v(t)$.
%Since $v(0) = 0$, it follows that $v \equiv 0$.
%Thus $u \equiv v \equiv 0$.
%\end{proof}
%
%If (ACP) has a unique solution for every initial value in $D(A)$, then $A$ is the generator of a strongly continuous semigroup only if some additional assumptions on the solutions (continuous dependence from the initial value) or on $A$ $(\rho(A) \neq \emptyset)$ are made.
%
%\begin{corollary}\label{cor:1.2}
%Let $A$ be a closed operator.
%Consider the following existence and uniqueness condition.
%
%(EU) For every $f \in D(A)$ there exists a unique solution $u(\cdot,f)\in C^{1}([0,\infty),E)$ of the Cauchy problem associated with $A$ having the initial value $u(0,f) = f$.
%
%The following assertions are equivalent.
%\begin{enumerate}[(a)]
%\item $A$ is the generator of a strongly continuous semigroup.
%\item $A$ satisfies (EU) and $\rho(A) \neq \emptyset$.
%\item $A$ satisfies (EU) and for every $\mu \in \mathbb{R}$ there exists $\lambda > \mu$ such that $(\lambda-A)D(A) = E$.
%\item $A$ satisfies (EU), has dense domain and for every sequence $(f_{n})$ in $D(A)$ satisfying $\lim_{n \to \infty}f_{n} = 0$ one has $\lim_{n \to \infty}u(t,f_{n}) = 0$ uniformly in $t \in [0,1]$.
%\end{enumerate}
%\end{corollary}
%
%\begin{proof}
%It is clear that (i) implies the remaining assertions.
%So assume that $A$ satisfy (EU).
%Then by Theorem 1.1., $A_{1}$ is a generator.
%If there exists $\lambda \in \rho(A)$, then $(\lambda-A)$ is an isomorphism from $E_{1}$ onto $E$ and $A$ is similar to $A_{1}$ via this isomorphism [since $D(A_{1}) = \{(\lambda-A)^{-1}f : f \in D(A)\}$ and $Af = (\lambda-A)A_{1}(\lambda-A)^{-1}f$ for all $f \in D(A)$, see A-I,3.0].
%Thus $A$ is a generator on $E$ and we have shown that (ii) implies (i).
%
%If (iii) holds, then there exists $\lambda > s(A_{1})$ such that $(\lambda-A)D(A) = E$.
%We show that $(\lambda-A)$ is injective.
%Then $\lambda \in \rho(A)$ since $A$ is closed.
%Assume that $Af = \lambda f$ for some $f \in D(A)$.
%Then $f \in D(A^{2}) = D(A_{1})$, and so $f = 0$ since $\lambda \in \rho(A_{1})$.
%This proves that (iii) implies (ii).
%We have shown existence.
%
%
%It remains to show that (iv) implies (i).
%
%Assertion (iv) implies that for all $t \geq 0$ there exist bounded operators $T(t) \in \mathcal{L}(E)$ such that $u(t,f) = T(t)f$ if $f \in D(A)$.
%Moreover, $\sup_{0\leq t\leq1} \|T(t)\| < \infty$.
%It follows that $T(\cdot)f$ is strongly continuous for all $f \in E$ (since it is so for $f \in D(A)$ and $D(A)$ is dense).
%Let $t > 1$.
%There exist unique $n \in \mathbb{N}$ and $s \in [0,1)$ such that $t = n + s$.
%Let $T(t) := T(1)^{n}T(s)$.
%From the uniqueness of the solutions it follows that $T(t)f = u(t,f)$ for all $t \geq 0$ as well as $T(t+s)f = T(s)T(t)f$ for all $f \in D(A)$ and $s,t \geq 0$.
%Thus $(T(t))_{t\geq0}$ is a semigroup.
%Denote by $B$ its generator.
%It follows from the definition that $A \subset B$.
%Moreover, $D(A)$ is invariant under the semigroup.
%So by A-I,Prop.1.9.(ii) $D(A)$ is a core of $B$.
%Since $A$ is closed this implies that $A = B$.
%\end{proof}
%
%\begin{remark}\label{rem:1.3}
%It is surprising that from condition (ii) and (iii) in the corollary it follows automatically that $D(A)$ is dense.
%On the other hand this condition cannot be omitted in (iv).
%In fact, consider any generator $\tilde{A}$ and its restriction $A$ with domain $D(A) = \{0\}$.
%Then $A$ satisfies the remaining conditions in (iv) but is not a generator (if $\dim E > 0$).
%\end{remark}
%
%\begin{example}\label{ex:1.4}
%We give a densely defined closed operator $A$, such that there exists a unique solution of (ACP) for all initial values in $D(A)$, but $A$ is not a generator.
%Let $B$ be a densely defined unbounded closed operator on a Banach space $F$.
%Consider $E = F \oplus F$ and $A$ on $E$ given by
%%% -- 
%\[
%A = \begin{pmatrix} 0 & B \\ 0 & 0 \end{pmatrix}
%\]
%%% -- 
%with domain $F \times D(B)$.
%
%Then $D(A^{2}) = \{(f,g) \in F \times D(B) : Bg \in F\} = D(A)$ and so $A_{1} \in \mathcal{L}(E_{1})$.
%In particular, $A_{1}$ is a generator.
%But $A$ is not.
%For instance condition (ii) in Corollary 1.2.\ does not hold, since for each $\lambda \in \mathbb{C}$,
%%% -- 
%\[
%(\lambda-A)D(A) = \{(\lambda f-Bg,\lambda g) : f \in F, g \in D(B)\} \subset F \times D(B) \neq F \times F = E.
%\]
%%% -- 
%So $\rho(A) = \emptyset$.
%\end{example}
%
%As a further illustration, we note that the solution of the corresponding abstract Cauchy problem for the initial value $(f,g) \in F \times D(B)$ is given by $u(t) = (f + tBg,g)$.
%Since $B$ is unbounded, condition (iv) of Corollary 1.2.\ is clearly violated.
