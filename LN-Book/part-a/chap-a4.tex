%% Stand 2025-04-24 ulgr
%% --
\chapter{Asymptotics of Semigroups on Banach Spaces}\label{chap:a4}
\index{Semigroups!Asymptotics}
\index{Banach Spaces!Semigroups}
%% --
{\Large
\vspace*{-.75cm}
by \\[.25em]
Frank Neubrander
\vspace{.75cm}
\\
}
%% --
%% --
In this chapter, we study the asymptotic behavior of the solutions of the initial value problem
%% --
\begin{equation}\label{eq:c4-1}
\dot{u}(t) = Au(t) + F(t), \, u(0) = f \tag{*}
\end{equation}
%% --
for $ t \geq 0 $, where $ A $ is the generator of a strongly continuous semigroup $ (T(t))_{t \geq 0} $ on a Banach space $ E $, and $ F(\cdot) $ is a function from $ \mathbb{R}_{+} $ into $ E $.\\
%% --

%%\noindent


In Section 1, we investigate whether -- and at what rate -- a solution $ T(\cdot)f $ of the homogeneous problem converges to the zero solution as $ t \to \infty $. 
In Section 2, we consider the long-term behavior of the solutions of \eqref{eq:c4-1} for different classes of forcing terms $ F $.
%% --
\section{Stability: Homogeneous Case}\label{sec:c4-1}
\index{Stability!Homogeneous Case}
\index{Semigroups!Stability}
%% -- 
Let $ (T(t))_{t \geq 0} $ be a semigroup on $ E $ with generator $ A $.
An initial value $ f \in D(A) $ is said to be \emph{stable} if the solution $ t \mapsto T(t)f $ of
%% --
\begin{equation}\label{eq:c4-2}
\dot{u}(t) = Au(t), \quad u(0) = f \tag{ACP}
\end{equation}
%% --
converges to zero as $ t \to \infty $. 
The semigroup is called stable if every solution converges to zero; that is, if every initial value $ f \in D(A) $ is stable.
\\
%%\noindent


If the space $ E $ is finite-dimensional, then the stability of the semigroup implies exponential decay. 
More precisely, the following statements are equivalent:
%% --
\begin{enumerate}[(a)]
\item $ \|T(t)f\| \to 0 $ for every $ f \in \mathbb{C}^{n} $,
\item $ \|T(t)\| \leq Me^{-\omega t} $ for some $M \geq 1,\ \omega > 0 $
\end{enumerate}
%% --
Moreover, these statements hold if and only if
%% --
\[
s(A) = \sup\{\Re\,\lambda : \lambda \in \sigma(A)\} < 0,
\]
%% --
see A-III, Corollary~1.2.
%% --
\goodbreak

%\medskip
%%\noindent


As discussed in Chapter A-III, the situation becomes significantly more difficult in the infinite-dimensional case. For unbounded generators, we must distinguish between strong and generalized (mild) solutions of $\dot{u}(t) = Au(t)$, as well as between various notions of stability. 
Recall that for $f \in D(A)$, the function $T(\cdot)f$ is a strong solution of (ACP) (see A-II, Corollary~1.2). 
For an arbitrary $f \in E$, the function $T(\cdot)f$ is referred to as a generalized or mild solution of (ACP). 
Next, we introduce several constants that characterize the growth of solutions of (ACP).

%% --
\begin{definition} ($1^{st}$ part).\label{def:a4-1.1}
Let $A$ be the generator of a strongly continuous semigroup on a Banach space $E$. 
Then:
%% --
\begin{enumerate}[(i)]
\item $\omega(f) \coloneqq \inf\{\omega : \|T(t)f\| \leq Me^{\omega t} \text{ for some } M \text{ and every } t \geq 0\}$ is called the exponential growth bound of $T(\cdot)f$;
\item $\omega_{1}(A) \coloneqq \sup\{\omega(f) : f \in D(A)\}$ is called the exponential growth bound for the solutions of the Cauchy problem $\dot{u}(t) = Au(t)$;
\item $\omega_0(A) = \sup\{\omega(f) : f \in E\}$ is called the exponential growth bound for the mild solutions of the Cauchy problem $\dot{u}(t) = Au(t)$.
\end{enumerate}
\end{definition}
%% --
Note that, by the Principle of Uniform Boundedness, 
\[
\sup\{\omega(f) : f \in E\} = \inf\{\omega : \|T(t)\| \leq Me^{\omega t} \text{ for some } M \text{ and every } t \geq 0\}.
\]
Hence, $\omega_0(A)$ coincides with the growth bound of the semigroup $(T(t))_{t \geq 0}$, as defined in A-I, I.3. 
Using the constants defined above, we obtain the following stability concepts.

%%\bigskip
%%\noindent


{\bf Definition 1.1.} ($2^{nd}$ part). 
The semigroup is called
%% --
\begin{enumerate}
\item[(iv)] uniformly exponentially stable if $\omega_0(A) < 0$;
\item[(v)] exponentially stable if $\omega_{1}(A) < 0$;
\item[(vi)] uniformly stable if $\|T(t)f\| \to 0$ as $t \to \infty$ for every $f \in E$;
\item[(vii)] stable if $\|T(t)f\| \to 0$ as $t \to \infty$ for every $f \in D(A)$.
\end{enumerate}
%% --
%%\bigskip
%%\noindent


The interrelation between these stability concepts is given by
%% --
\[
\text{(iv)} \Rightarrow \text{(v)} \Rightarrow \text{(vii)} \text{ and } \text{(iv)} \Rightarrow \text{(vi)} \Rightarrow \text{(vii)}
\]
%% --
If $A$ is a bounded operator, that is, if $D(A) = E$, then (iv) $\Leftrightarrow$ (v) and (vi) $\Leftrightarrow$ (vii).
However, if $A$ is unbounded, the stability notions may differ, as illustrated in the following examples.
%%--
\begin{example}\label{ex:a4-1.2}
\begin{enumerate}[(a), wide]
\item 
Let $E = c_{0}$. 
Then $A: (x_{n})_{n \in \mathbb{N}} \mapsto (-1/n \cdot x_{n})_{n \in \mathbb{N}}$ generates the semigroup $T(t)(x_{n})_{n \in \mathbb{N}} = (e^{-t/n} x_{n})_{n \in \mathbb{N}}$.
It is easy to see that $\|T(t)\|=1$ and that $\|T(t)f\|\to 0$ for every $f \in c_{0}$.
Moreover, since $A$ is a bounded operator, $D(A) = E$.
This provides an example of a (uniformly) stable but not exponentially stable semigroup.
The translation semigroups generated by the first derivative on $C_0(\mathbb{R}_{+})$ or $L^{p}(\mathbb{R}_{+})$ for $1 < p < \infty$ offer further examples of (uniformly) stable but not exponentially stable semigroups.
Moreover, as shown in A-II, Example~1.14, the Laplacian $\Delta$ on $C_0(\mathbb{R}^{n})$ generates a bounded holomorphic semigroup given by
%% --
\[
T(t)f(x) = (4\pi t)^{-n/2} \int_{\mathbb{R}^{n}} e^{-(x-y)^{2}/4t} f(y) \,\dy
\]
%% --
which cannot be exponentially stable because $0 \in \sigma(\Delta)$ ($\text{im}(\Delta) \not= C_0(\mathbb{R}^{n})$), see Corollary~1.5 below.

By a straightforward $(2-\epsilon)$-argument using $(4\pi t)^{-n/2}\int_{\mathbb{R}^{n}} \exp(-y^{2}/4t) \,\dy = 1$ one can easily show that $\|T(t)f\| \to 0$ for all $f \in C_0(\mathbb{R}^{n})$ (see also B-III, Example~1.7).
%%--
Therefore, the Laplacian on $C_0(\mathbb{R}^{n})$ (and also on $L^{p}(\mathbb{R}^{n})$ for $1 < p < \infty$, see Example~1.15 below) generates a (uniformly) stable but not exponentially stable semigroup.

%%\medskip%%\noindent


%(b) 
\item 
Note that the condition $0 \leq \omega_0(A) = \inf\{\omega : \|T(t)\| \leq Me^{\omega t} \text{ for all } t \geq 0\}$ does not exclude the possibility that the semigroup is exponentially stable.
In fact, as shown in A-III, 1.3 the translation semigroup on $E \coloneqq C_0(\mathbb{R}_{+}) \cap L^{1}(\mathbb{R}_{+},e^{x}\dx)$ satisfies $\|T(t)\| = 1$, hence $\omega_0(A) = 0$. 
For every $\lambda \in \mathbb{C}$ with $\Re\,\lambda > -1$, the resolvent of the generator is given as $R(\lambda,A)f = \int_{0}^{\infty} e^{\lambda t} T(t)f \,\dt$ for every $f \in E$.
From the equation A-I, 3.2
%% --
\[
T(t)f = e^{\lambda t}\left(f - \int_{0}^{t} e^{-\lambda s} T(s) (\lambda - A)f \,\ds\right)
\]
%% --
and from the existence of $\lim_{t \to \infty} \int_{0}^{t} e^{-\lambda s} T(s) (\lambda - A)f \,\ds$, it follows that 
$\|T(t)f\| \leq Me^{\lambda t}$ for every $f \in D(A)$ and some constant $M$ depending on $f$. This yields $\omega_{1}(A) \le -1 < 0 = \omega_0(A)$. 
Thus, we have a semigroup that is exponentially stable, but not uniformly exponentially stable.

%%\medskip%%\noindent


%(c) 
\item
Rescaling this semigroup (see A-I, 3.1), we obtain a semigroup with $-1/2 = \omega_{1}(A)$ and $1/2 = \omega_0(A)$.
Therefore, there exist exponentially stable (and hence, stable) semigroups that are not bounded, and hence, not uniformly stable.
This example illustrates that there may be a significant difference between the long-term behavior of the semigroup $(T(t))_{t \geq 0}$ (\ie the set of all mild solutions) and the long-term behavior of the strong solutions $\{T(\cdot)f : f \in D(A)\}$ of (ACP). 
\end{enumerate}
\end{example}

%%%\noindent


In the following, we characterize the exponential growth bounds $\omega(f)$, $\omega_{1}(A)$, and $\omega_0(A)$ by certain abscissas of simple or absolute convergence of the Laplace transform of $T(\cdot)f$. 
These characterizations will serve as the main tool for showing that, for certain semigroups, the growth bounds $\omega_0(A)$ and/or $\omega_{1}(A)$ coincide with the spectral bound $s(A) = \sup\{\Re\,\lambda : \lambda \in \sigma(A)\}$.

%%%\bigskip%%\noindent


We begin by observing that $s(A)$ can be interpreted as the abscissa of holomorphy of the Laplace transform $\lambda \mapsto \int_{0}^{\infty} e^{-\lambda t} T(t) \,\dt$ of the semigroup $(T(t))_{t \geq 0}$.

%%%\noindent


Furthermore, we recall that the Laplace transform exists for every $\lambda \in \mathbb{C}$ with $\Re\,\lambda > \Re\,\mu$, provided it exists for some $\mu\in\C$.
This follows from 
%% --
\begin{align}\label{eq:a4-1.1}
\int_0^t e^{-\lambda s} f(s)\,\ds = &e^{-(\lambda-\mu)t} \int_{0}^{t} e^{\mu s} f(s) \,\ds \\ &+ (\lambda - \mu) \int_{0}^{t} e^{-(\lambda-\mu)s} \int_{0}^{s} e^{\mu r} f(r) \,\diff{r}\,\ds \notag
\end{align}
%% --
Note that even boundedness of $\int_{0}^{t} e^{-\mu s}f(s) \,\ds$ implies the existence of the Laplace transform for $\Re\,\lambda > \Re\,\mu$. 
Therefore, the subset of $\mathbb{C}$ for which the Laplace transform exists is always a half-plane 
$\{\lambda \in \mathbb{C} : \Re\,\lambda > \gamma\} \cup H$, where $H$ is a subset of the line $\{\lambda \in \mathbb{C} : \Re\,\lambda = \gamma\}$.

%%%\bigskip%%\noindent


In the following theorem, we show that the bound of the half-plane for which the Laplace transform of $T(\cdot)f$ ($f \in E$) exists absolutely, and the bound of the half-plane for which the Laplace transform of $T(\cdot)Af$ ($f \in D(A)$) exists strongly, coincide with the growth bound $\omega(f) = \inf\{\omega : \|T(t)f\| \leq Me^{\omega t} \text{ for all } t \geq 0\}$.

%%%\bigskip%%\noindent


\begin{theorem}\label{thm:a4-1.3}
Let $A$ be the generator of a strongly continuous semigroup on a Banach space $E$. 
Then, for every $f \in E$,
%% --
\begin{equation}\label{eq:a4-1.2}
\omega(f) = \limsup_{t \to \infty} \frac{1}{t}\log\|T(t)f\|,
\end{equation}
%% --
and
%% --
\begin{enumerate}[(i)]
\item $\omega(f) = \inf\{\Re\,\lambda : \int_{0}^{\infty} \|T(t)f\| \,\dt \text{ exists}\}.$
\end{enumerate}
If $\ker(A) = \{0\}$, then for every $f \in D(A)$ we have 
\begin{enumerate}
\item[(ii)] $\omega(f) = \inf\{\Re\lambda : \int_{0}^{\infty} T(t)Af \,\dt \text{ exists as an improper Riemann integral}\}.$
\end{enumerate}
%% --
\end{theorem}
%%--
\begin{proof}  The proof of \eqref{eq:a4-1.2} is omitted (see \citet[p.306]{hillephillips:1957}. 
To prove (i) and (ii), we need the following lemma.

%%\bigskip
%%\noindent


{\bf Lemma}. 
\label{lem:a4-1.3}
{\it Let $F \in C(\mathbb{R}_{+},\mathbb{R}_{+})$ be such that $\int_{0}^{\infty} F(t) \,\dt$ exists. 
If there is a positive number $m$ and an interval $[0,n]$ such that $F(t + s) \leq m \cdot F(s)$ for all $s \geq 0$ and $t \in [0,n]$, then $\lim_{s \to \infty} F(s) = 0$.}

%\medskip%%\noindent


\textit{Proof of the Lemma}. 
For all $\epsilon > 0$ there exists $a > 0$ such that $A(a) \coloneqq \int_{a}^{\infty} F(s) \,\ds < \frac{n}{m}\epsilon$.
For all $t > a+n$ there exists $r \in [t-n,t]$ such that $F(r) \leq \frac{1}{n}A(a)$.
Therefore, $F(t) = F(t-r+r) \leq m \cdot F(r) \leq m \cdot \frac{A(a)}{n} < \epsilon$. 
\qed

%%\bigskip
%%%\noindent


To prove (i), we define $b \coloneqq \inf\{\Re\,\lambda : \int_{0}^{\infty} \|e^{-\lambda t} T(t)f\| \,\dt \text{ exists}\}$. 
A straightforward application of the lemma shows that $\omega(f) \leq b$.
The definition of $\omega(f)$ gives the reverse inequality.

\goodbreak
%%\medskip%%\noindent


It remains to prove statement (ii) of Theorem \ref{thm:a4-1.3}.
Assume that $\ker(A) = \{0\}$ and let $f \in D(A)$, $\lambda \in \mathbb{C}$ with $\Re\,\lambda > \omega(f)$. 
From the equation
%% --
\[
\int_{0}^{t} e^{-\lambda s} T(s)Af \,\ds = e^{-\lambda t}T(t)f - f + \lambda \int_{0}^{t} e^{-\lambda s} T(s)f \,\ds
\]
%% --
it follows that $\int_{0}^{\infty} e^{-\lambda t} T(t)Af \,\dt$ exists. 
Therefore, 
%%--
\[b \coloneqq \inf\{\Re\,\lambda : \int_{0}^{\infty} e^{-\lambda t}T(t)Af \,\dt \text{ exists}\} \leq \omega(f).\]
%%--

%%\noindent


Next we show that $b < 0$ implies $b = \omega(f)$. 
Suppose $b < 0$. 
Then, by \eqref{eq:c4-1}, $\int_{0}^{\infty} T(s)Af \,\ds$ exists. 
Since $\int_{0}^{r} T(s)Af \,\ds = T(r)f - f$, we see that $g \coloneqq\lim_{r \to \infty} T(r)f$ exists. 
But for every $t \geq 0$, $T(t)g = g$ which implies that $g \in \ker(A) = \{0\}$ or $g = 0$. 
Hence, $\int_{0}^{\infty} T(s)Af \,\ds = -f$.
Now, choosing $r < 0$, $b < r < 0$, and integrating by parts, we obtain
%% --
\begin{align*}
-T(t)f &= \lim_{u \to \infty} \int_{t}^{u} e^{r s} e^{-r s} T(s)Af \,\ds\\
&= \lim_{u \to \infty} ( e^{ru}\int_0^ue^{-rs} T(s)Af \,\ds - e^{rt}\int_0^te^{-rs} T(s)Af \,\ds\\ 
& \qquad \qquad -r\int_t^u e^{rs}\int_0^s e^{-rv} T(v)Af\,\diff{v} \,\ds
) \\
&= - e^{rt}\int_0^te^{-rs} T(s)Af \,\ds -r\int_t^\infty e^{rs}\int_0^s e^{-rv} T(v)Af\,\diff{v} \,\ds.
\end{align*}
%% --
From $\left\|\int_{0}^{t} e^{-r s} T(s)Af \,\ds\right\| \leq M$ for some $M$ and every $t \geq 0$ we conclude that $\|T(t)f\| \leq \tilde{M}e^{rt}$ for all $t \geq 0$ and some constant $\tilde{M}$.
Hence, $\omega(f) \leq r$ for every $b < r < 0$; that is, $\omega(f) \leq b$.

%%\noindent


If $b \geq 0$ and $w > b$, then $\left\|\int_{0}^{t} e^{-w s} T(s)Af\,\ds\right\| \leq M$ for all $t \geq 0$. 
By 
\begin{align*}
T(t)f - f &= \int_{0}^{t} e^{w s} e^{-w s} T(s)Af \,\ds\\
&= e^{wt}\int_0^t e^{-ws} T(s)Af\,\ds - w\int_0^te^{ws}\int_0^s e^{-wr} T(r)Af\, \diff{r}\,\ds
\end{align*}
we obtain $\|T(t)f-f\| \leq Me^{w t} + M(e^{w t} - 1)
\le 2Me^{wt}$. 
Hence, $\omega(f)\le w$ for every $w > b$, \ie $\omega(f) \leq b$.
\end{proof}

%%%\bigskip%%\noindent


From \eqref{eq:a4-1.2} and the Uniform Boundedness Principle, it follows, that the growth bound $\omega_{1}(A) = \sup\{\omega(f) \colon f \in D(A)\}$ satisfies
%% --
\begin{align}\label{eq:a4-1.3}
\omega_1(A) &= \inf\{\omega \colon \text{for every } f \in D(A) \text{ there exists a constant } M \text{ such that }\\
& \hskip 2 true in \|T(t)f\| \leq Me^{\omega t} \text{ for every } t \geq 0\} \notag\\
&=
\limsup_{t \to \infty} \frac{1}{t} \log\|T(t)R(\lambda,A)\| \quad (\lambda \in \rho(A)). 
\notag
\end{align}
%% --

%%%\bigskip%%\noindent


The subsequent theorem will be of particular importance in the stability theory of positive semigroups.
We show, that the constant $\omega_{1}(A)$ coincides with the abscissa of simple convergence of the Laplace transform of the semigroup and, with the abscissa of absolute convergence of the Laplace transform of the strong solutions of (ACP).

\begin{theorem}\label{thm:a4-1.4}
\index{Theorem!Growth bound}
Let $A$ be the generator of a strongly continuous semigroup on a Banach space $E$. 
Then
%% --
\begin{align}\label{eq:a4-1.4}
\omega_{1}(A) &= \inf\{\Re\,\lambda \colon \int_{0}^{\infty} e^{-\lambda t} T(t)f \,\dt \text{ exists as an improper} \\
&\phantom{= \inf\{\Re\,\lambda \colon \int_{0}^{\infty} e^{-\lambda t} T(t)f} \text{Riemann integral for every } f \in E\} \notag \\
&= \inf\{\Re\,\lambda \colon \int_{0}^{\infty} \|e^{-\lambda t} T(t)f\| \,\dt \text{ exists for every } f \in D(A)\}.\notag
\end{align}
\end{theorem}
%--
\begin{remarks*}\label{rem:a4-1.4}
\index{Remark!Laplace transform convergence}

\begin{enumerate}[(a), wide, labelsep=1em, itemindent=\parindent]

\item 
One can show that the abscissas of uniform, strong, and weak convergence of the Laplace transform coincide (see C-III, Theorem~I.2, last part of the proof). 
Therefore, by Theorem \ref{thm:a4-1.4},
%--
\begin{align}\label{eq:a4-1.5}
\omega_1(A) &= \inf \left\{ \Re\,\lambda : \text{weak-} \lim_{t \to \infty} \int_0^t e^{-\lambda s} T(s) \,\ds \text{ exists} \right\}
\\
    &= \inf \left\{ \Re\,\lambda : \text{uniform-} \lim_{t \to \infty} \int_0^t e^{-\lambda s} T(s) \,\ds \text{ exists} \right\}.\notag
\end{align}

\item
In equations \eqref{eq:a4-1.4} and \eqref{eq:a4-1.5}, the term \enquote{$\Re\,\lambda$} may be replaced by "$\lambda \in \mathbb{R}$" (use \eqref{eq:c4-1}).
\end{enumerate}

\end{remarks*}
%--

%% --
\begin{proof}(Proof of Theorem \ref{thm:a4-1.4}) 
The equality 
 \[ 
 \omega_1(A) = \inf \left\{ \Re\,\lambda : \int_0^\infty \|e^{-\lambda t} T(t) f \| \,\dt \text{ exists for all } f \in D(A) \right\} 
 \] 
 follows from the definition of $\omega_1(A)$ and the lemma in the proof of Theorem \ref{thm:a4-1.3}.
We prove 
$
\omega_1(A) = \inf \left\{ \Re\,\lambda : \int_0^\infty e^{-\lambda s} T(s) f \,\ds \text{ exists for every } f \in E \right\}=: b. 
$
The identity
$    T(t) f = e^{\lambda t} \left\{ f - \int_0^t e^{-\lambda s} T(s) (\lambda - A) f \,\ds \right\}
$
yields
\[
\omega_1(A) \leq \inf \left\{ \Re\,\lambda : \int_0^\infty e^{-\lambda t} T(t) f \,\dt \text{ exists for every } f \in \text{im} (\lambda - A) \right\}.
\]
Therefore,
$\omega_1(A) \leq b.
$
%--
Take $\lambda \in \mathbb{C}$ with $\Re\,\lambda > \omega_1(A)$. 
Then $\int_0^\infty e^{-\lambda t} T(t) f \,\dt$ exists for every $f \in D(A)$. 
Define $g := \int_0^\infty e^{-\lambda t} T(t) f \,\dt$. 
Then $g \in D(A)$ and $\int_0^n e^{-\lambda t} T(t) f \,\dt = \sum_{k=0}^{n-1} e^{-\lambda k} T(k) g$. 
Since $\text{Re } \lambda > \omega_1(A)$, the sum converges for every $g \in D(A)$. 
Therefore, the integral converges as $n \to \infty$ for every $f \in E$.
For every $t \in \mathbb{R}^+$, define a bounded operator $T_t$ by $ f \mapsto \int_0^t e^{-\lambda s} T(s) f \,\ds$. 
As seen above, $T_n f$ converges as $n \to \infty$ for every $f \in E$. 
It follows from the Uniform Boundedness Principle that $(T_n)_{n \in \mathbb{N}}$ is uniformly bounded.
%%--
%%\noindent


For every $t \in \mathbb{R}^+$, there exist $n \in \mathbb{N}$ and $t' \in [0,1)$ such that $T_t = T_{t'} + e^{-\lambda t'} T(t') T_n$. 
Since the operator families on the right side are uniformly bounded, the same is true for $(T_t f)_{t \geq 0}$. 
Since $(T_t f)_{t \geq 0}$ converges for every $f \in D(A)$, it follows that $(T_t f)_{t \geq 0}$ converges for every $f \in E$. 
Thus, $b \leq \omega_1(A)$.
\end{proof}

%%--
%\medskip
%%\noindent


The inequality
\[
    \omega_0(A) \geq \inf \left\{ \text{Re } \lambda : \int_0^\infty \|e^{-\lambda t} T(t) f \| \,\dt 
    \text{ exists for every } f \in E \right\}
\]
combined with the lemma of Theorem \ref{thm:a4-1.3} implies that the growth bound $\omega_0(A)$ coincides with the abscissa of absolute convergence of the Laplace transform of $(T(t))_{t \geq 0}$; \ie
\begin{equation}\label{eq:a4-1.6}
   \omega_0(A) = \inf \left\{ \text{Re } \lambda : \int_0^\infty \|e^{-\lambda t} T(t) f \| \,\dt \text{ exists for every } f \in E \right\}.
\end{equation}
%--
%%\noindent


As seen in A-I, Prop.1.11, if $\int_0^\infty e^{-\lambda t} T(t) f \,\dt$ exists for every $f \in E$, then $\lambda \in \rho(A)$ and $R(\lambda, A) f = \int_0^\infty e^{-\lambda t} T(t) f \,\dt$. 
This and Theorem \ref{thm:a4-1.4} yield the following corollary.
%--
\begin{corollary} \label{cor:a4-1.5} Let $ A $ be the generator of a strongly continuous semigroup $ (T(t))_{t \geq 0} $ on a Banach space $ E $. 
Then  
\[
s(A) \leq \omega_1(A) \leq \omega_0(A).
\]
\end{corollary}

%%\noindent

 Example 1.2.(c) shows that uniform exponential stability is not equivalent to $ \sigma(A) \subset \{\lambda \in \mathbb{C} : \text{Re } \lambda \leq q < 0 \} $. 
The following example shows that even strong solutions can be unstable when $ s(A) < 0 $, constructing a semigroup where $ s(A) < \omega_1(A) < \omega_0(A) $.

\begin{example}\label{ex:a4-1.6} In A-III, Example~1.4, the semigroup $ (T(t))_{t \geq 0} $ on the Hilbert space $ E = \{(x^1, x^2, ...), x^n \in \mathbb{C}^n : \sum_{i=1}^{\infty} \|x^i\|^2 < \infty\} $, given by $ T(t) := (e^{2\pi i n t} \exp(t A_n))_{n \in \mathbb{N}} $ with  

\[
A_n =
\begin{bmatrix}
0 & 1 & 0 & \dots & 0 \\
& 0 & 1 & \dots & 0 \\
& & \ddots & \ddots & \vdots \\
& & & 0 & 1 \\
0 & & & & 0
\end{bmatrix}_{n \times n}
\]
has growth $ e^t $ (\ie $ \|T(t)\| = e^t $), so $ \omega_0(A) = 1 $ while $ A = (2\pi i n + A_n)_{n \in \mathbb{N}} $ has spectral bound $ 0 $. 
We show that $ \omega_1(A) = \omega_0(A) $ to construct a semigroup with $ s(A) < \omega_1(A) < \omega_0(A) $. 
Let $ e_n = n^{-1/2} (1, ..., 1) \in \mathbb{C}^n $. 

Then,
%% --  
\begin{align*}
\|\exp(&t A_n) e_n\|^2 = \\
&= \frac{1}{n} \left\| (1 + t + \dots + \frac{t^{n-1}}{(n-1)!}, 1 + t + \dots + \frac{t^{n-2}}{(n-2)!}, \dots, 1+t, 1) \right\|^2 \\
&=
\frac{1}{n} \sum_{r=0}^{n-1} \sum_{j,s=0}^{r} \frac{1}{j!s!} t^{j+s} \, = \,
\frac{1}{n} \sum_{r=0}^{n-1} \left(\sum_{j=0}^{r} \frac{1}{j!} t^j \right)^2 
\, = \, \frac{1}{n} \sum_{r=0}^{n-1} \sum_{j,s=0}^{r} \frac{1}{j!s!} t^{j+s} \\
&= 
\frac{1}{n} \sum_{r=0}^{n-1} \sum_{i=0}^{2r} t^i \sum_{j+s=i} \frac{1}{j!s!} \,
= \, 
\frac{1}{n} \sum_{r=0}^{n-1} \sum_{i=0}^{2r} \frac{(2t)^i}{i!} \,  = \,  \frac{1}{n^2} \sum_{i=0}^{n-1} \frac{(2t)^i}{i!}.
\end{align*}
%% --
 For $ 0 < q < 1 $, define $ x_q \in E $ as  
$
x_q := (q e_1, 2q^2 e_2, ..., n q^n e_n, ...).
$
Then $x_q \in D(A)$ and
%--
\begin{align*}
\|T(t)x_q\|^2 & = \sum_{n=1}^{\infty} q^{2n} \| \exp(t A) e_n \|^2
\geq \sum_{n=1}^{\infty} n^2 q^{2n} \left(\frac{1}{n^2} \sum_{i=0}^{n-1}  \frac{1}{i!} (2t)^i \right)\\
&= \sum_{i=0}^{\infty} \sum_{n=i+1}^{\infty} \left( q^{2n} \frac{1}{i!} (2t)^i \right)
= \sum_{i=0}^{\infty} q^{2i+2} (1 - q^2)^{-1} \frac{1}{i!} (2t)^i \\
&= \frac{q^2}{1 - q^2} \sum_{i=0}^{\infty} \frac{1}{i!} (2t q^2)^i
= \frac{q^2}{1 - q^2} e^{2t q^2}.
\end{align*}
%% -- \noindent kann man sich eraparen, wenn man nach \end{} keine Leerzeile macht, also
%% -- und kein eingerücktes it
It follows that $\omega(x_q) \geq q^2$. 
Thus,
\[
1 = \sup \{\omega(x_q) : 0 < q < 1\} \leq \omega_1(A) \leq \omega_0(A) = 1.
\]
%% --
Rescaling the semigroup (\ie looking at $e^{-3/2 \cdot t} T(t)$), we obtain a semigroup generator $A$ on the Hilbert space $E$ with $-3/2 = s(A)$ and $\omega_1(A) = \omega_0(A) = -1/2$. 
On the other hand, Example 1.2.(c) produces a semigroup in a Banach space $F$ with generator $B$ such that $-1 = s(B) = \omega_1(B)$ while $\omega(B) = 0$. 
Now the operator $C := A \oplus B$ on the Banach space $E \oplus F$ is a semigroup generator for which
\[
s(C) = \max \{s(A), s(B)\} = -1, \quad \omega_1(C) = \max\{\omega_1(A), \omega_1(B)\} = -1/2
\]
\[
\text{and} \quad \omega(C) = \max\{\omega_0(A), \omega(B)\} = 0.
\]
\end{example}
%% -- 
\begin{remark}\label{rem:a4-1.7}
For eventually norm continuous semigroups---particularly compact, differentiable, holomorphic, or nilpotent ones--- the spectral mapping theorem 
%% --
\begin{equation}\label{eq:a4-1.7}
\sigma(T(t)) \setminus \{0\} = e^{t \sigma(A)}
\end{equation}
%% --
holds. 
Consequently,
%% --
\begin{equation}\label{eq:a4-1.8}
s(A) = \omega_1(A) = \omega_0(A)
\end{equation}
%% --
is valid (Corollary \ref{cor:a4-1.5} and A-III, Corollary~6.7).
Hence, if $A$ is the generator of an eventually norm continuous semigroup, then the exponential growth bounds of the strong and the mild solutions of $\dot{u}(t) = A u(t)$ are determined by the spectral bound
$s(A) = \sup\{\Re\,\lambda : \lambda \in \sigma(A)\}.$
\end{remark}
%--
\begin{remark} In general, the growth bound $\omega_0(A)$ can be obtained using the Hille-Yosida theorem (see A-II, Theorem~1.7) as
\begin{align}\label{eq:a4-1.9}
\omega_0(A) = \inf \{ w : \|R(\lambda, A)^n\| \leq &M ( \Re\,\lambda - w)^{-n} \text{ for some } M \text{ and}\\
&\text{every } n \in \mathbb{N} \text{ and } \lambda \in \mathbb{C} \text{ with } \text{Re } \lambda > w\}.\notag
\end{align}
%\noindent
 In view of the difficulties involved in estimating all powers of the resolvent, this equation is of little practical use. 
If $A$ is the generator of a semigroup on a Hilbert space $H$, then it is shown in A-III, Corollary~7.11 that
%--
\begin{equation}\label{eq:a4-1.10}
 \omega_0(A) = \inf \{ w : \| R(\lambda, A) \| \leq M_w \quad \text{for } \Re\,\lambda > w \}.
\end{equation}
\end{remark}
%\noindent
 Unfortunately, the identity \eqref{eq:a4-1.10} does not hold on arbitrary Banach spaces. 
However, as we will see in Section 1 of C-IV, the identity 
%--
\begin{equation}\label{eq:a4-1.11}
 s(A)=\omega_1(A) = \inf \{ w : \| R(\lambda, A) \| \leq M_w \quad \text{for } \text{Re } \lambda > w \}
\end{equation}
%\noindent
 holds for every positive semigroup on a Banach lattice. 
Consequently, for positive semigroups with $s(A) = \omega_1(A) < \omega_0(A)$ (see Example \ref{ex:a4-1.2}(2)), the identity \eqref{eq:a4-1.10} does not apply. 
However, we can establish the following theorem.

\begin{theorem}\label{thm:a4-1.9} Let $A$ be the generator of a strongly continuous semigroup $(T(t))_{t \geq 0}$ on a Banach space $E$. 
Suppose there exist constants $a \geq 0$ and $q \geq s(A)$ such that there are $C \in \mathbb{R}_+$ and $n \in \mathbb{N}$ such that, 
    \[
    \| R(\lambda, A) \| \leq C | \lambda |^{n-2}
    \]
    for all $\lambda \in \mathbb{C}$ with $\Re\,\lambda > q$ and $| \Im \lambda | > a$. 
    Then
    \[
    \sup \{ \omega(f),  f \in D(A^n)  \leq q.\]
\end{theorem}

\begin{proof} The hypothesis $\| R(\lambda, A) \| \leq C | \lambda |^{n-2}$ is invariant under rescaling. 
That is, the resolvent $R(\lambda, -b+A)$ of the generator $-b+A$ of the rescaled semigroup $e^{-bt} T(t)$ satisfies  
$\| R(\lambda, -b+A) \| \leq \tilde{C} | \lambda |^{n-2}$ for every $\lambda \in \mathbb{C}$ with $\text{Re } \lambda > q-b$ and $| \text{Im } \lambda | > a+2b$ and a suitable constant $\tilde{C}$. 
Therefore, we may assume, without loss of generality, that $b := \max(\omega_0(A), q) < 0$. 
Let $\omega_0(A) < p < 0$. 
Then, for every $f \in D(A)$ and $p' := \max\{p, q\} < 0$, the inversion formula for the Laplace transform gives 
\begin{equation} \label{eq:a4-1.12}
T(t) f = \frac{1}{2\pi i} \int_{p' - i\infty}^{p' + i\infty} e^{\lambda t} R(\lambda, A) f \, d\lambda.
\end{equation}
%\noindent
 (For a proof of the vector-valued version of the inversion formula one may follow \citet[p.66]{widder:1946}; also see the notes to this section.)
By the resolvent equation we obtain 
   \[ R(\lambda, A)^n R(0, A) = \sum_{k=1}^{n} (-1)^{k+1} \lambda^{-k} R(0, A)^{n+1-k} + (-1)^n \lambda^{-n} R(\lambda, A).
   \]
%\noindent
  Using that  
$ \frac {1}{2\pi i} \int_{p' - i\infty}^{p' + i\infty} e^{\lambda t} \, \lambda^{-k} \, d\lambda = 0 $ for $ k \geq 1, p' < 0 $ and $ t > 0$, we obtain
 \begin{equation}\label{eq:a4-1.13}
   T(t) R(0, A)^n f = (-1)^n \frac{1}{2\pi i} \int_{p' - i\infty}^{p' + i\infty} e^{\lambda t} \lambda^{-n} R(\lambda, A) f \, d\lambda
    \end{equation}
    %\noindent
 for every $f \in E$ and $t > 0$.
    %\noindent
 If $q < p'$, then, by Cauchy's Integral Theorem and since $\| R(\lambda, A) \| \leq C | \lambda |^{n-2}$, we see that the path of integration can be shifted to $\text{Re} \lambda = q$; 
    %\noindent
 that is,
    \[
    T(t) R(0, A)^n f = (-1)^n \frac{1}{2\pi i} \int_{q - i\infty}^{q + i\infty} e^{\lambda t} \lambda^{-n} R(\lambda, A) f \, d\lambda.
    \]
Therefore,  
$
\| T(t) R(0,A)^n f \| \leq c' e^{q t} \| f \| \int_{-\infty}^{\infty} (q^2 + s^2)^{-1}\ds = M e^{q t} \| f \|$ 
or 
$
\| T(t) f \| \leq M e^{q t} \| A^n f \|$ for $f \in D(A^n)$.
\end{proof}
%% --
In view of the characterizations given in Section 1 of A-II, the semigroups occurring in the theorem are holomorphic if $ n = 1 $. 
In this case, one may apply \eqref{rem:a4-1.7} to obtain the stronger statement \eqref{eq:a4-1.8}.

Instead of making assumptions on the resolvent of $ A $, we now take a different view and characterize the property \enquote{$ \omega_0(A) < 0 $} in terms of the semigroup $ (T(t))_{t \geq 0} $ directly.
%% --
\begin{proposition} \label{prop:a4-1.10} Let $ A $ be the generator of a strongly continuous semigroup $ (T(t))_{t \geq 0} $ on a Banach space $E$. 
Then the following statements are equivalent:
\begin{itemize}
\item[(a)] $ \omega_0(A) < 0 $.
\item[(b)] $ \lim_{t \to \infty} \| T(t) \| = 0 $.
\item[(c)] $ \| T(t') \| < 1 $ for some $ t' > 0 $.
\end{itemize}
\end{proposition}

\begin{proof}
The implications (a) $ \Rightarrow $ (b) $ \Rightarrow $ (ac) are obvious; (c) $ \Rightarrow $ (a) follows from  $\omega_0(A) = \lim_{t \to \infty} \frac{1}{t} \log \| T(t) \|$ (see A-I,(1.1)) and  
\[
\displaystyle
\frac{\log \|T(t)\|}{t} \leq \frac{\log \|T(t')\|}{t'} + \frac{\log \|T(t)\|}{nt' + s} \quad \text{for } t = nt' + s, \quad s \in [0,t'].
\]
\end{proof}
%% --
%\noindent
 Other less obvious characterizations of the property ``$ \omega_0(A) < 0 $'' are given in the next theorem. 
The equivalence of (a) and (c) is known as \emph{Datko’s Theorem}.


\begin{theorem} \label{thm:a4-1.11} Let $ A $ be the generator of a strongly continuous semigroup $ (T(t))_{t \geq 0} $ on a Banach space $ E $. 
Then the following statements are equivalent:
\begin{itemize}
\item[(a)] $ \omega_0(A) < 0 $.
\item[(b)] $ s(A) < 0 $ and there is $ t_0 > 0 $ such that  
$
|\lambda| < 1$ for every $ \lambda \in A\sigma (T(t_0)).$
\item[(c)] For every (some) $ p \geq 1 $ exists $ \int_{0}^{\infty} \| T(t) f \|^p\dt $ for every $ f \in E $.
\end{itemize}
\end{theorem}

\begin{proof} The implication ``(a) $ \Rightarrow $ (b)'' follows from $ r(T(t)) = e^{\omega_0(A) t} < 1 $ and $ s(A) \leq \omega_0(A) < 0 $. For the point and residual spectrum, the spectral mapping theorem is valid (see A-III, Theorem~6.3). The approximate point spectrum is closed, hence, the additional information in (ii)
implies $ |\lambda| \leq r < 1 $ for some $ r $ and each $ \lambda \in A\sigma(T(t_0)) $. 
Consequently, 
\[
\exp(\omega_0(A) \cdot t_0) = r(T(t_0)) \leq \max\{\exp(t_0 \cdot s(A)), r\} < 1
\]
or $ \omega_0(A) < 0 $. 
This proves \textquotedblleft (b) $\Rightarrow$ (a)\textquotedblright. 
For a proof of the equivalence of (a) and (c) we refer to \citet{datko:1972} or \citet[Thm.4.4.1]{pazy:1983}. 
\end{proof}

%\medskip%\noindent

Rescaling a given semigroup $ (T(t))_{t \geq 0} $, one obtains the following corollary from \eqref{eq:a4-1.1} and statement (c) of the above theorem.

%\medskip
\begin{corollary} \label{cor:a4-1.12} 
Let $ (T(t))_{t \geq 0} $ be a strongly continuous semigroup on a Banach space $ E $. Then the set of complex numbers $ \lambda $ for which
\[
\int_{0}^{\infty} \|e^{-\lambda t} T (t) f \| \,\dt,
\]
exists for every $ f \in E $, is an open right half-plane.
\end{corollary}
%\medskip

%\noindent

In the next theorem, we give two necessary conditions for stability of $ (T(t))_{t \geq 0} $ in terms of the generator $ A $. We will see in Chapter C-IV that for positive semigroups a condition similar to statement (ii) below is even sufficient for stability of the semigroup. We emphasize that stable semigroups need not be uniformly bounded (see Example \ref{ex:a4-1.2}(3)) and that $ s(A) = \omega_0(A) = 0 $ does not imply boundedness or even stability of the semigroup (see also A-I, Example~1.4.(i)).

\begin{theorem} \label{thm:a4-1.13} Let $ A $ be the generator of a stable semigroup $ (T(t))_{t \geq 0} $ on a Banach space $ E $. Then the following assertions hold:

\begin{itemize}
    \item[(i)] $ s(A) \leq 0 $ and $ \operatorname{Re} \lambda < 0 $ for every $ \lambda \in P\sigma (A) \cup R\sigma(A) $.
    \item[(ii)] $ \lim_{\lambda \to 0^+} \lambda R(\lambda, A) f $ exists for every $ f \in D(A) $.
\end{itemize}
\end{theorem}
%\medskip

\begin{proof} If $ (T(t))_{t \geq 0} $ is stable, then $ \|T(t) f\| \leq M_f $ for every $ f \in D(A) $. Therefore, $ s(A) \leq \omega_1(A) \leq 0 $. 

%\noindent
 Assume there is $ \lambda \in P\sigma(A) $ with $ \Re\,\lambda = 0 $. Then, by A-III, Corollary~6.4, there exists $ g \neq 0 $ such that $ T(t) g = e^{\lambda t} g $ for all $ t \geq 0 $. Since $ |e^{\lambda t}| = 1 $, this contradicts the stability of the semigroup. 

%\noindent
 Assume there is $ \lambda \in R\sigma(A) = P\sigma(A') $ with $ \Re\,\lambda = 0 $. Then there exists $ 0 \neq \varphi \in E' $ with $ T(t)^* \varphi = \exp(\lambda t) \varphi $ for all $ t \geq 0 $. Choose $ f \in D(A) $ such that $ \langle f, \varphi \rangle \neq 0 $. Then $ |\langle T(t) f, \varphi \rangle| = | \langle f, \varphi \rangle| > 0 $ for every $ t \geq 0 $, which contradicts the stability of the semigroup.

%\medskip
%\noindent

(ii) From the stability of the semigroup and the identity   $\int_{0}^{t} T(s) A f \,\ds = T(t) f - f$, we see that  
$\int_{0}^{\infty} T(s) A f \,\ds$ exists for every $ f \in D(A) $.  
But $ \omega_1(A) \leq 0 $, and hence,  
$ R(\lambda, A) A f = \int_{0}^{\infty} e^{-\lambda s} T(s) A f \,\ds $ for every $ \lambda > 0 $ (see Theorem~1.4). By a classical theorem of Laplace transform theory, (for a proof of the vector-valued version one may follow \citet[p.196]{widder:1971}), we conclude that $\lim_{\lambda \to 0+} R(\lambda,A)Af$ exists
and is equal to $\int_{0}^{\infty} T(s)Af \,\ds$. 
The identity $R(\lambda,A)Af = \lambda R(\lambda,A)f - f$
yields the existence of $\lim_{\lambda \to 0+} \lambda R(\lambda,A)f$ for every $f \in D(A)$.
\end{proof}
%%--
%\noindent

Bounded holomorphic semigroups (see A-II, Definition~1.11) satisfy
$\|AT(t)\| \leq \frac{m}{t}$ \citet[p.33]{goldstein:1985a}, hence, $T(t)f \to 0$ as $t \to \infty$
whenever $f \in \image A$. 
If $\image A$ is dense (\ie $0 \not\in R\sigma(A)$), we obtain
uniform stability and the following corollary.

\begin{corollary}\label{cor:a4-1.14}
Let $A$ be the generator of a bounded holomorphic
semigroup $(T(t))_{t \geq 0}$ on a Banach space $E$. 
Then the following statements are equivalent.
\begin{enumerate}[(a)]
\item $0 \not\in P\sigma(A) \cup R\sigma(A)$.
\item $(T(t))_{t \geq 0}$ is uniformly stable.
\end{enumerate}
\end{corollary}

\begin{example}\label{ex:a4-15}
The Laplacian $\Delta$ generates a bounded holomorphic semigroups on $L^{p}(\mathbb{R}^{n})$ for $1 \leq p < \infty$ (see the example proceeding
Corollary~1.13 of Chap. A-II). 
All solutions of $\Delta f = 0$ are either constant
or unbounded, therefore $0 \not\in P\sigma(\Delta)$. 
If $1 < p < \infty$, then the adjoint
of the Laplacian on $L^{p}(\mathbb{R}^{n})$ is the Laplacian on $L^{q}(\mathbb{R}^{n})$ where
$\frac{1}{p} + \frac{1}{q} = 1$. 
Therefore $0 \not\in P\sigma(\Delta) \cup R\sigma(\Delta)$ and we obtain by Corollary \ref{cor:a4-1.14}
that $\Delta$ generates uniformly stable semigroups on the space $L^{p}(\mathbb{R}^{n})$
for $1 < p < \infty$ which are, by $\image \Delta \not= L^{p}(\mathbb{R}^{n})$ and Corollary \ref{cor:a4-1.5}, not exponentially stable. \qed
\end{example}
%%--
%\noindent

As seen in Theorem \ref{thm:a4-1.4}, exponential stability can be defined by saying that the abscissa of convergence of the Laplace transform of $(T(t))_{t \geq 0}$ is less than zero. 
This should be compared to the assertion of our final theorem.

\begin{theorem}\label{thm:a4-1.16}
Let $A$ be the generator of a strongly continuous semigroup $(T(t))_{t \geq 0}$ on a Banach space $E$. 
The following assertions are equivalent:
\begin{enumerate}[(a)]
\item $(T(t))_{t \geq 0}$ is stable.
\item $\ker(A) = \{0\}$ and $\int_{0}^{\infty} T(t)f \,\dt$ exists for all $f \in \image A$.
\end{enumerate}
Furthermore the following statements are equivalent:
\begin{enumerate}[(a$'$)]
\item $(T(t))_{t \geq 0}$ is stable and bounded.
\item $(T(t))_{t \geq 0}$ is uniformly stable.
\item $(T(t))_{t \geq 0}$ is bounded and there is a dense subspace $D$ such that $\int_{0}^{\infty} T(t)f \,\dt$ exists for every $f \in D$.
\end{enumerate}
\end{theorem}

\begin{proof}
If $(T(t))_{t \geq 0}$ is stable, then, by Theorem \ref{thm:a4-1.13}, $\ker(A) = \{0\}$ and
$\int_{0}^{t} T(s)Af \,\ds = T(t)f - f \to -f$ as $t \to \infty$. 
Therefore, (a) implies (b).
On the other hand, if $\int_{0}^{t} T(s)Af \,\ds$ converges as $t \to \infty$, then, by
the above equation, $g \coloneqq \lim_{t \to \infty} T(t)f$ exists. 
But $\ker(A) = \{0\}$ and
therefore $g = 0$. 
This proves "(b) $\Rightarrow$ (a)".
The implication "(a$'$) $\Rightarrow$ (b$'$)" is obvious. 
If $T(t)f \to 0$ for every $f \in E$, then $\|T(t)\| \leq M$ and $0 \not\in R\sigma(A)$ (Theorem \ref{thm:a4-1.13}). 
Therefore,
$D \coloneqq \image A$ is dense and $\int_{0}^{\infty} T(t)f \,\dt$ exists for every $f \in D$. 
This proves "(b$'$) $\Rightarrow$ (c$'$)". 
We have to show that (c$'$) implies (a$'$).
Define $G \coloneqq \{h \in E : h = \int_{0}^{\infty} T(t)g \,\dt \text{ for some } g \in D\}$. 
We will show
that $G$ is dense in $E$. 
First, we notice that $g - T(s)g \in D$ whenever $g \in D$
and $s \in \mathbb{R}_{+}$.
Define $h_{s} = \frac{1}{s} \int_{0}^{\infty} T(t)(g - T(s)g) \,\dt = \frac{1}{s} \int_{0}^{s} T(t)g \,\dt$. 
Then $h_{s} \in G$
and $h_{s} \to g$ as $s \to 0$. 
Therefore, $D \subset G$ or $G = E$. 
Now, let $h \in G$.
Then, $T(t)h = T(t) \int_{0}^{\infty} T(s)g \,\ds = \int_{t}^{\infty} T(s)g \,\ds \to 0$ as $t \to \infty$. 
But $\|T(t)\| \leq M$ and therefore $T(t)f \to 0$ for every $f \in E$.
\end{proof}

\begin{remark}\label{rem:a4-1.17}
(a) If $A$ is the generator of a stable semigroup
$(T(t))_{t \geq 0}$ on a Banach space $E$, then, by the previous theorem,
$\image A \subset \{f \in E : \int_{0}^{\infty} T(t)f \,\dt \text{ exists}\} =: H$.
If $g \in H$, then $\int_{0}^{\infty} T(t)g \,\dt \in D(A)$ and $A \int_{0}^{\infty} T(t)g \,\dt = -g$. 
Therefore $g \in \image A$ and we obtain that the dense subspace $\image A$ is given
by
\begin{equation}\label{eq:a4-1.14}
\image A = \{f \in E : \int_{0}^{\infty} T(t)f \,\dt \text{ exists}\}
\end{equation}
in case that $A$ is the generator of a stable semigroup $(T(t))_{t \geq 0}$.

%%\bigskip
%\noindent

(b) If $\omega(f) < 0$ for every $f \in D(A)$, then $(T(t))$ is stable (but
might not be exponentially stable if
$0 = \omega_{1}(A) = \supp\{\omega(f) : f \in D(A)\}$. 
In this case it can be seen by a
proof similar to the one of Theorem \ref{thm:a4-1.4}, that $\sigma(A)$ has to be contained
in the open left half-plane; i.e. $\Re\,\lambda < 0$ for $\lambda \in \sigma(A)$.

%%\bigskip
%\noindent

(c) If one defines a semigroup $(T(t))_{t \geq 0}$ to be weakly stable if
$\langle T(t)f,\phi \rangle \to 0$ as $t \to \infty$ for every $f \in D(A)$ and $\phi \in E'$ or as
weakly uniformly stable if $\langle T(t)f,\phi \rangle \to 0$ as $t \to \infty$ for every $f \in E$
and $\phi \in E'$, then Theorem \ref{thm:a4-1.13} and \ref{thm:a4-1.16} can be reformulated in a weak form (i.e.; we replace stable by weakly stable and \emph{lim} by
\emph{weak-lim}). 
The proofs require only some obvious modifications.
If A has a compact resolvent or if A is the generator of a bounded
holomorphic semigroup, then weak stability implies stability. 
In general, this is no longer true; e.g., the translation semigroup on
$L^{2}(\mathbb{R})$ is weakly uniformly stable but not stable (see also
B-IV, Example~1.2).
\end{remark}

\section{Stability: Inhomogeneous Case}
\index{Stability!Inhomogeneous Case}
%% --
Using the results of the first section, we now investigate the long
term behavior of the solutions of the inhomogeneous initial value
problem
\begin{equation}\label{eq:a4-2.1}
\dot{u}(t) = Au(t) + F(t) \quad , \quad u(0) = f,
\end{equation}
where A is the generator of a strongly continuous semigroup on a
Banach space E and $F(\cdot)$ is a locally integrable function from $\mathbb{R}_{+}$
into E referred to as the forcing term. 
A function $u(\cdot)$ is called a (strong)
solution of \eqref{eq:a4-2.1} if $u(\cdot) \colon \mathbb{R}_{+} \to D(A)$, $u(\cdot) \in C^{1}(\mathbb{R}_{+},E)$ and \eqref{eq:a4-2.1} is satisfied for $t \geq 0$.
The assumption that A is the generator of a semigroup $(T(t))_{t \geq 0}$
yields the uniqueness of the solution of \eqref{eq:a4-2.1}. 
If $u(\cdot)$ is a solution of \eqref{eq:a4-2.1}, then the function $v(s) := T(t-s)u(s)$, $0 \leq s \leq t$, is
differentiable and $v'(s) = T(t-s)F(s)$. 
But $F(\cdot)$ is locally integrable, and by $\int_{0}^{t} T(t-s)F(s) \,\ds = v(t) - v(0) = u(t) - T(t)f$ we see
that the solution $u(t)$ of \eqref{eq:a4-2.1} is given by
\begin{equation}\label{eq:a4-2.2}
u(t) = T(t)f + \int_{0}^{t} T(t-s)F(s) \,\ds
\end{equation}

%%\bigskip
%\noindent

{\bf Example.} \label{ex:a4-2.1}
Let $(T(t))_{t \geq 0}$ be a semigroup that is not eventually differentiable. 
Then there
exists $g \in E$ such that $t \mapsto T(t)g$ is not differentiable on $(0,\infty)$.
The initial value problem $\dot{u}(t) = Au(t) + T(t)g$, $u(0) = 0$ has no
(strong) solution $u(\cdot)$ because otherwise
\begin{align*}
u(t) = \int_{0}^{t} T(t-s)T(s)g \,\ds = tT(t)g
\end{align*}
has to be differentiable on $\mathbb{R}_{+}$.

%%\bigskip
%\noindent

Whenever the expression \eqref{eq:a4-2.2} is well-defined, we refer to it as a  \emph{generalized} (or
\emph{mild}) solution of \eqref{eq:a4-2.1}. 
If $F(\cdot)$ is continuous and $f \in D(A)$, then
the generalized solution of \eqref{eq:a4-2.1} is a strong solution if and only if
$v(t) := \int_{0}^{t} T(t-s)F(s) \,\ds$ is differentiable (see \citet[Chap.4,2.4]{pazy:1983}Pazy (1983). 
There are several sufficient conditions on the generator A,
the forcing term $F(\cdot)$ or the space E under which every mild solution
is a strong solution of \eqref{eq:a4-2.1} (see \citet{travis:1979}
\marginpar{GG: Travis (1979) ist nicht im LV!}
or  \citet[Sec.4.2]{pazy:1983}).

%%\bigskip
%\noindent

Our aim in this section is to study the asymptotic behavior of the solutions of \eqref{eq:a4-2.1} as $t \to \infty$. 
To that end, we consider absolutely integrable or periodic forcing terms $F(\cdot)$, and assume that the semigroup
is uniformly stable.

%\noindent

Similar results for integrable and convergent forcing terms $F(\cdot)$ can
be obtained if the semigroup is assumed to be uniformly  
stable (see \citet[p.119]{pazy:1983} or \citet{neubrander:1985b}). 
However, if the
semigroup is positive, these results hold even under the weaker assumption of stability (see Section C-IV).
From Theorem \ref{thm:a4-1.13}(i), we recall that for stable
semigroups, $\image A$ is dense in E.

\begin{theorem}\label{thm:a4-2.1}
Let $A$ be the generator of a uniformly stable semigroup
$(T(t))_{t \geq 0}$ on a Banach space $E$. 
If there is $g \in \image A$ such that
$\int_{0}^{\infty} \|F(s) - g\| \,\ds$ exists, then every generalized solution $u(\cdot)$ of
\eqref{eq:a4-2.1} converges as $t \to \infty$ and $\lim_{t \to \infty} u(t) = -h$ where $h \in D(A)$ with
$Ah = g$.
\end{theorem}

\begin{proof}
If $u(\cdot)$ is a generalized solution of \eqref{eq:a4-2.1}, then, by \eqref{eq:a4-2.2},
\[
u(t) = T(t)f + \int_{0}^{t} T(s)g \,\ds + \int_{0}^{t} T(t-s)(F(s)-g) \,\ds.
\]
By the uniform
stability and $\int_0^t T(s)Ah\,\ds = T(t)h - h$ (see A-I, Proposition 1.6), the first term converges to zero
and the second term converges to $-h$. 
We have to show that the
third term converges to zero. 
Take $\epsilon > 0$ and $G(s) \coloneqq F(s)-g$. 
Then
\begin{align*}
\|\int_{0}^{t} T(t-s)G(s) \,\ds\| &\leq \|\int_{0}^{r} T(t-r+r-s)G(s) \,\ds\| + \|\int_{r}^{t} T(t-s)G(s) \,\ds\|\\
&\leq \|T(t-r)\int_{0}^{r} T(r-s)G(s) \,\ds\| + M \int_{r}^{\infty} \|G(s)\| \,\ds.
\end{align*}
Since the semigroup is uniformly stable, we obtain
$T(t-r)\int_{0}^{r} T(r-s)G(s) \,\ds \to 0$ as $t \to \infty$ for every $r \geq 0$.
Therefore, $\|\int_{0}^{t} T(t-s)G(s) \,\ds\| \leq \epsilon$ for all sufficiently large $t$.
\end{proof}
%\noindent

In the following theorem, we show that if A generates a
uniformly stable semigroup, the forcing term $F(\cdot)$ is p-periodic,
and $\int_{0}^{p} T(p-s)F(s) \,\ds \in \image(\Id - T(p))$, then \eqref{eq:a4-2.1}
admits a unique p-periodic, asymptotically stable mild solution. (Notice that, by
Theorem \ref{thm:a4-1.13} and A-III, Lemma 5.3, $\image(\Id - T(p)) = E$.)

\begin{lemma}\label{lem:a4-2.2}
Let $A$ be the generator of a strongly continuous semigroup $(T(t))_{t \geq 0}$ on a Banach space $E$ and let $F(\cdot)$ be a $p$-periodic, locally integrable function, $p > 0$. 
Then the following statements are equivalent:
\begin{enumerate}[(a)]
\item $\dot{u}(t) = Au(t) + F(t)$ admits a (unique) generalized $p$-periodic solution.
\item There exists a (unique) $f \in E$ such that $(\Id - T(p))f = \int_{0}^{p} T(p-s)F(s) \,\ds$.
\end{enumerate}
\end{lemma}

\begin{proof}
$(a) \Rightarrow (b)$. 
Let $f \coloneqq u(0)$ be the initial value for which (2.1) has the $p$-periodic solution. 
Then we have
%% --
\begin{align*}
u(t) &= u(t+p) \\
&= T(t)T(p)f + \int_{0}^{p} T(t+p-s)F(s) \,\ds + \int_{p}^{t+p} T(t+p-s)F(s) \,\ds \\
&= T(t)\left[T(p)f + \int_{0}^{p} T(p-s)F(s) \,\ds\right] + \int_{0}^{t} T(t-s)F(s) \,\ds
\end{align*}
%% --
for every $t \geq 0$. 
Therefore, $f = u(0) = T(p)f + \int_{0}^{p} T(p-s)F(s) \,\ds$.
Clearly, if $u(\cdot)$ is a unique periodic solution with $u(0) = f$, then $f$ is the unique element for which $f = T(p)f + \int_{0}^{p} T(p-s)F(s) \,\ds$ holds.

%\medskip%\noindent

$(b) \Rightarrow (a)$. 
Define $u(\cdot)$ as in \eqref{eq:a4-2.2}. 
Then
%% --
\begin{align*}
u(t+p) = T(t)\left[T(p)f + \int_{0}^{p} T(p-s)F(s) \,\ds\right] + \int_{0}^{t} T(t-s)F(s) \,\ds = u(t).
\end{align*}
%% --
If $f$ is unique, then, by the considerations above, the solution is unique.
\end{proof}

\begin{remark}\label{rem:a4-2.3}
Let $A$ be the generator of a strongly continuous semigroup for which the spectral mapping theorem holds (see A-III, Sec.6), and let $F$ be a $p$-periodic forcing term.
If $\frac{2\pi in}{p} \in \rho(A)$ for every $n \in \mathbb{Z}$, then, by Lemma \ref{lem:a4-2.2}, $\dot{u}(t) = Au(t) + F(t)$ has a unique $p$-periodic solution with initial value $(\Id - T(p))^{-1} \left(\int_{0}^{p} T(p-s)F(s) \,\ds\right)$.
\end{remark}

%%\bigskip
%\noindent

As a consequence of Theorem \ref{thm:a4-1.13}  and A-III, Corollary~6.4, for a uniformly stable semigroup, there exists at most one $f \in E$ satisfying $(\Id-T(p))f = \int_{0}^{p} T(p-s)F(s) \,\ds$. 
This fact, together with Lemma \ref{lem:a4-2.2}, is used to prove the following theorem.


\begin{theorem}\label{thm:a4-2.4}
Let $A$ be the generator of a uniformly stable semigroup $(T(t))_{t \geq 0}$ and let $F(\cdot)$ be a $p$-periodic locally integrable function such that $(\Id - T(p))f = \int_{0}^{p} T(p-s)F(s) \,\ds$ for some $f \in E$. 
Then the unique $p$-periodic generalized solution
%% --
\[
u(t) = T(t)f + \int_{0}^{t} T(t-s)F(s) \,\ds
\]
%% --
is asymptotically stable; that is, for every generalized solution $v(\cdot)$ of \eqref{eq:a4-2.1}, 
\[
\lim_{t \to \infty} \|v(t) - u(t)\| = 0.
\]
\end{theorem}

\begin{example}\label{ex:a4-2.5}
Let $E$ be the Banach space $C_{0}(\mathbb{R}_{+})$. 
Then $A = \frac{d}{dx}$ with $D(A) = \{f \in E: f' \in C^{1} \text{ and } f' \in E\}$ is the generator of the uniformly stable translation semigroup $T(t)f(x) \coloneqq f(t+x)$. 
Applying \eqref{eq:a4-1.14}, we obtain that $\image A = \{f : \int_{0}^{\infty} f(x) \,\dx \text{ exists}\}$ is dense in $C_{0}(\mathbb{R}_{+})$. 
Let $r \in \image A$ and let $F(\cdot)$ be a $p$-periodic, real-valued function.

%\noindent

We apply Theorem \ref{thm:a4-2.4}  to the initial value problem
%% --
\begin{equation}
 \frac{d}{dt} u(t,x) = \frac{d}{dx}u(t,x) + r(x)F(x+t), \quad u(0,\cdot) \in D(A). 
 \tag{*}
\end{equation}
%% --
We may rewrite (*) as
%% --
\begin{equation}
\dot{v}(t) = Av(t) + G(t), \tag{**}
\end{equation}
%% --
where $v(t) = u(t,\cdot)$ and $G : \mathbb{R}_{+} \to E$ is defined by $G(t)(x) = r(x)F(x+t)$.
Then $G$ is $p$-periodic with values in $E$ and $h_{0} \coloneqq \int_{0}^{p} T(p-t)G(t) \,\dt$ is the function $x \mapsto \left[\int_{0}^{p} T(p-t)G(t) \,\dt\right](x) = F(x)\int_{x}^{x+p} r(s) \,\ds$. 
For the function $f = \sum_{k=0}^{\infty} T(kp)h_{0}$, given by $x \mapsto F(x)\int_{x}^{\infty} r(s) \,\ds$, it is clear that $(\Id - T(p))f = h_{0}$. 
Therefore, (**) has a unique $p$-periodic generalized solution (Theorem \ref{thm:a4-2.4}) although $\im \mathbb{R} \subset \sigma(A)$ (compare with Remark \ref{rem:a4-2.3}).

%\noindent

The unique $p$-periodic generalized solution $u(t,\cdot)$ is given by 
\[
u(t,x) = F(x+t)\int_{x+t}^{\infty} r(s)ds + F(x+t)\int_{x}^{x+t} r(s)ds = F(x+t)\int_{x}^{\infty} r(s)\,\ds.
\]
For every solution $v(t,\cdot)$ of (*) we have, by Theorem \ref{thm:a4-2.4} :
%% --
\[
\sup\{|v(t,x) - F(x+t)\int_{x}^{\infty} r(s) \,\ds| : x \in \mathbb{R}_{+}\} \to 0 \quad \text{as} \quad t \to 0.
\]
%% --
\end{example}


\section*{Notes}
\addcontentsline{toc}{section}{Notes}
%% --
\begin{enumerate}[%
label={\Large \emph{Section \arabic*:}}
, wide
, labelindent = 0.0em]

\item 
The exponential growth bounds $\omega(f)$ and $\omega_0(A)$ as well as the characterizations \eqref{eq:a4-1.2}, \eqref{eq:a4-1.6} and Theorem \ref{thm:a4-1.3} (i) can be found in \citet{hillephillips:1957}.
Growth bounds similar to $\omega_{1}(A)$ were considered first in \citet{djacenko:1976} and in \citet[Prop.2]{zabczyk:1979}. 
Example \ref{ex:a4-1.2}(2) is taken from \citet{wolff:1981}; other \emph{counterexamples} can be found in \citet{hillephillips:1957}, \citet{foias:1973}, \citet{triggiani:1975}, 
\marginpar{GG: Triggiani (1975a  oder b)} 
\citet{zabczyk:1975} and \citet{greineretal:1981}. 
Statements \eqref{eq:a4-1.2}, \eqref{eq:a4-1.6} and Theorem \ref{thm:a4-1.3} (i) are semigroup versions of results of classical Laplace transform theory, see \citet{hillephillips:1957} and \citet{widder:1946}. 
Theorem \ref{thm:a4-1.3} (ii) is a semigroup version of Theorem 1.2.7 and 1.2.8 in \citet{doetsch:1950}. 
The lemma in the proof of Theorem \ref{thm:a4-1.3} is taken from \citet{milstein:1975}. 
Theorem \ref{thm:a4-1.4} and Corollary \ref{cor:a4-1.5} can be found in \citet{neubrander:1985a}. 
Example \ref{ex:a4-1.6} follows Remark~2 in \citet{zabczyk:1975}. 
Statement \eqref{eq:a4-1.8} is sometimes called the \emph{spectrum determined growth assumption}, see, for example, \citet{triggiani:1975b}. 
Theorem \ref{thm:a4-1.9} is due to \citet{slemrod:1976}. 
The proof presented here is based on the following sharper version of the inversion formula for the Laplace transform, which improves upon the one given in \citet[p.349]{hillephillips:1957}. 
Using \citet[p.66]{widder:1946} or \citet[p.212]{doetsch:1950}Doetsch (1950) one can establish the following theorem (see \citet{neubrander:1984b}.

\begin{theorem}\label{thm:a4-2.6}
Let $A$ be the generator of a strongly continuous semigroup $(T(t))_{t \geq 0}$ on a Banach space $E$. 
For every $f \in D(A)$ and $p > \omega_{1}(A)$ we have
%% --
\[
T(t)f = \frac{1}{2\pi i} \int_{p-i\infty}^{p+i\infty} e^{\mu t}R(\mu,A)f \, d\mu.
\]
%% --
\end{theorem}
%\noindent

The equivalence of the statements \eqref{eq:a4-1.12}, \eqref{eq:a4-1.13} and \emph{$\omega_0(A) < 0$} were observed by many authors, see for example, \citet[p.178]{balakrishnan:1976}, 
\marginpar{GG Balakrishnan (1976) ist nicht im LV!} 
or \citet{benchimol:1978}. 
\marginpar{GG: Benchimol: 1978a oder 1978b}
Theorem \ref{thm:a4-1.11} is due to \citet{datko:1970} and \citet{delfour:1974} 
\marginpar{GG: "Delfour (1974)" ist nicht im LV!}; 
for a proof see \citet[p.116]{pazy:1983}. 
Theorems \ref{thm:a4-1.13} and \ref{thm:a4-1.16} can be found in \citet{neubrander:1985b} and Corollary \ref{cor:a4-1.14} is due to \citet{komatsu:1969}. 
An example of an unstable semigroup generator $A$ with Re $\mu < 0$ for all $\mu \in \sigma(A)$ is given in \citet{datko:1983}.

\item For a discussion of well-posedness of inhomogeneous Cauchy problems, we refer to \citet[p.83]{goldstein:1985a}, and \citet[p.105]{pazy:1983}. 
Further results on the asymptotic behavior of the solutions of the inhomogeneous problem can be found in \citet{raohengartner:1974}, \citet{zaidman:1979}, \citet{pazy:1983}, and \citet{neubrander:1985b}. 
Results similar to Lemma \ref{lem:a4-2.2} and Theorem \ref{thm:a4-2.4} are due to \citet{pruess:1984}. 
For a discussion of the asymptotic behavior of the solutions of $\dot{u}(t) = A(t)u(t) + F(t)$ see \citet{datko:1972} and \citet[p.172]{pazy:1983}.

\end{enumerate}

%% -- Literatur
%% --
\RaggedRight
\bibliographystyle{abbrvnat}
\bibliography{bib/ln-references} 
