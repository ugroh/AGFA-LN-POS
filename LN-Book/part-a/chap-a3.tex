% !TEX root = chap-a3-test.tex
%% -- Chapter A-III
%% -- ulgr 2025/03/16
%% -- gugr 2025/04/07  Kleinigkeiten korrigiert u.a. \eqref{..}

\setcounter{chapter}{2}
\setcounter{section}{1}
%\author{Günthßr Greiner and Rainer Nagel}
\chapter{Spectral Theory}\label{chap:a3}
%% --
\section{Introduction}
\index{Spectral Theory!Introduction}
%% --
In this chapter, we start a systematic analysis of the spectrum of a strongly continuous semigroup $\TT = (T(t))_{t\geq 0}$ on a complex Banach space $E$.
By the spectrum of the semigroup we understand the spectrum $\sigma(A)$ of the generator $A$ of $\TT$.
In particular, we are interested in the precise relations between $\sigma(A)$ and $\sigma(T(t))$.
The heuristic formula
%% --
\[
	T(t) = \mathrm{e}^{tA}
\]
%% --
serves as a leitmotiv and suggests relations of the form
%% --
\[
\sigma(T(t)) = \mathrm{e}^{t\sigma(A)} = \{ \mathrm{e}^{t\lambda} \colon \lambda \in \sigma(A) \}\,,
\]
%% --
called \emph{spectral mapping theorem}.
These---or similar---relations will be of great use in Chapter IV and enable us to determine the asymptotic behavior of the semigroup $\TT$ by the spectrum of its generator.

As motivation and also as a preliminary step, we concentrate here on the \emph{spectral radius}
%% -- 
\begin{equation}\label{eq:a3-1.1}
	r(T(t)) := \sup \{ |\lambda| : \lambda \in \sigma(T(t)) \}, \quad t \geq 0\,,
\end{equation}
%% -- 
and show how it is related to the \emph{spectral bound}
%% -- 
\begin{equation}\label{eq:a3-1.2}
	s(A) := \sup \{ \Re\,\lambda : \lambda \in \sigma(A) \}
\end{equation}
%% -- 
of the generator $A$ and to the \emph{growth bound}
%% -- 
\begin{equation}\label{eq:a3-1.3}
	\omega_{0} := \inf \{\omega \in \\R  : \|T(t)\| \leq M_{\omega}\cdot \mathrm{e}^{\omega t} \text{ for all } t \geq 0 \text{ and suitable } M_{\omega}\}
\end{equation}
%% -- 
of the semigroup $\TT = (T(t))_{t\geq 0}$.
(Recall that sometimes we write $\omega_{0}(\TT)$ or $\omega_{0}(A)$ instead of $\omega_{0}$).
%
The Examples~\ref{ex:a3-1.3} and \ref{ex:a3-1.4} below illustrate the main difficulties to be encountered.
%% -- Examples  1.3 and 1.4
\begin{proposition}\label{prop:a3-1.1}
Let $\omega_{0}$ be the growth bound of the strongly continuous semigroup $\TT = (T(t))_{t\geq 0}$.
Then
%% --
\begin{equation}\label{eq:a3-1.4}
	r(T(t)) = \mathrm{e}^{\omega_{0} t}
\end{equation}
%% --
for every $t \geq 0$.
\end{proposition}

%\newpage
%% -- a3-2
\begin{proof}
From A-I, (1.1) we know that
%% --
\[
    \omega_{0}(\TT) = \lim_{t \to \infty} \frac{1}{t} \log \|T(t)\|
\]
%% --
Since the spectral radius of $T(t)$ is given as
%% --
\[
    r(T(t)) = \lim_{n \to \infty} \|T(nt)\|^{1/n}\,,
\]
%% --
we obtain for $t > 0$
%% --
\[
    r(T(t)) = \lim_{n \to \infty} \exp\left(\frac{t}{nt} \log \|T(nt)\|\right) = \mathrm{e}^{\omega_{0} t}\,.
\]
%% --
\end{proof}
%% --
It was shown in A-I, Proposition~1.11 that the spectral bound $s(A)$ is always dominated by the growth bound $\omega_{0}$ and therefore $\mathrm{e}^{s(A)t} \leq r(T(t))$.
If the above mentioned spectral mapping theorem holds --- as is the case for bounded generators (\eg see Theorem~VII.3.11 of \citet{dunfordschwartz:1958}) --- we obtain the equality
%% --
\[
    \mathrm{e}^{s(A)t} = r(T(t)) = \mathrm{e}^{\omega_{0}(\TT) t}\,,
\]
%% --
hence $s(A) = \omega_{0}(\TT)$.
Therefore, the following corollary is a consequence of the definitions of $s(A)$ and $\omega_{0}(\TT)$.
%% --
\begin{corollary}\label{cor:a3-1.2}
Consider the semigroup $\TT = (T(t))_{t \geq 0}$ generated by some bounded linear operator $A \in \L{E}$.
If $\Re\,\lambda < 0$ for each $\lambda \in \sigma(A)$, then $\lim_{t \to \infty}\|T(t)\| = 0$.
\end{corollary}
%% --
Through this corollary we have re-established a famous result of Liapunov which assures that the solutions of the linear Cauchy problem
%% --
\[
    \dot{x}(t) = Ax(t), \quad x(0) = x_{0} \in \C^{n} \quad \text{and} \quad A = (a_{ij})_{n\times n}
\]
%% --
are \emph{stable}, \ie they converge to zero as $t \to \infty$ if the real parts of all eigenvalues of the matrix $A$ are smaller than zero.

For unbounded generators the situation is much more difficult and $s(A)$ may differ drastically from $\omega_{0}(\TT)$.
%% --
\begin{example}\label{ex:a3-1.3}
%\index{Examples!Banach Function Space}
(Banach function space, \citet{greinervoigtwolff:1981})
Consider the Banach space $E$ of all complex valued continuous functions on $\R_{+}$ which vanish at infinity and are integrable for $\mathrm{e}^{x}\dx$, \ie 
%% --
\[
    E \coloneqq C_{0}(\R_{+}) \cap L^{1}(\R_{+}, \mathrm{e}^{x}\dx)
\]
%% --
endowed with the norm
%% --
\[
    \|f\| \coloneqq \|f\|_{\infty} + \|f\|_{1} = \sup\{|f(x)| \colon x \in \R_{+}\} + \int_{0}^{\infty} |f(x)|\mathrm{e}^{x} \dx\,.
\]
%% --
\end{example}
%% --
The translation semigroup
%% --
\[
    T(t)f(x) \coloneqq f(x+t)
\]
%% --
is strongly continuous on $E$ and one shows as in A-I,2.4 that its generator is given by
%% --
\[
    Af = f', \quad D(A) = \{ f \in E \colon f \in C^{1}(\R_{+}), f' \in E \}\,.
\]
%% --
First we observe that $\|T(t)\| = 1$ for every $t \geq 0$, hence $\omega_{0}(\TT) = 0$.
Moreover it is clear that $\lambda$ is an eigenvalue of $A$ as soon as $\Re\,\lambda < -1$ (in fact: the function
%% --
\[
    x \mapsto e_{\lambda}(x) \coloneqq \mathrm{e}^{\lambda x}
\]
%% --
belongs to $D(A)$ and is an eigenvector of $A$), hence $s(A) \geq -1$.
For $f \in E$, $\Re\,\lambda > -1$,
%% --
\[
    \|\cdot\|_{1}\text{-}\lim_{t \to \infty} \int_{0}^{t} \mathrm{e}^{-\lambda s}T(s)f \, \ds
\]
%% --
exists since $\|T(s)f\|_{1} \leq \mathrm{e}^{-s}\|f\|_{1}$, $s \geq 0$, and
%% --
\[
    \|\cdot\|_{\infty}\text{-}\lim_{t \to \infty} \int_{0}^{t} \mathrm{e}^{-\lambda s}T(s)f \, \ds
\]
%% --
exists since $\int_{0}^{\infty} \mathrm{e}^{x}|f(x)| \, \dx < \infty$.
Therefore $\int_{0}^{\infty} \mathrm{e}^{-\lambda s}T(s)f \, \ds$ exists in $E$ for every $f \in E$, $\Re\,\lambda > -1$.
As we observed in A-I, Proposition~1.11, this implies $\lambda \in \rho(A)$.
Therefore $\TT = (T(t))_{t\geq 0}$ is a semigroup having $s(A) = -1$, but $\omega_{0}(\TT) = 0$.
%% --
\begin{example}\label{ex:a3-1.4}
%\index{Examples!Hilbert Space}
(Hilbert space, \citet{zabczyk:1975})
For every $n \in \N$ consider the $n$-dimensional Hilbert space $E_{n} \coloneqq \C^{n}$ and operators $A_{n} \in \mathcal{L}(E_{n})$ defined by the matrices
%% --
\[
    A_{n} =
    \begin{pmatrix}
    0 & 1 & \cdots & 0 \\
    \cdot & 0 & 1 & \cdot \\
    \cdot & \cdot & \cdot & 1 \\
    0 & \cdot & \cdot & 0
    \end{pmatrix}_{n \times n}\,.
\]
%% --
These matrices are nilpotent and therefore $\sigma(A_{n}) = \{0\}$.
The elements 
%% --
\[
	x_{n} \coloneqq n^{-1/2}(1, \ldots, 1) \in E_{n}
\]
%% --
 satisfy the following properties.
%% --
\begin{enumerate}[(i)]
\item
	$\|x_{n}\| = 1$ for every $n \in \N$\,,

\item
	$\lim_{n \to \infty} \|A_{n}x_{n} - x_{n}\| = 0$\,,

\item
	$\lim_{n \to \infty} \|\exp(tA_{n})x_{n} - \mathrm{e}^{t}x_{n}\| = 0$\,.

\end{enumerate}
%% --
Consider now the Hilbert space $E \coloneqq \bigoplus_{n \in \N} E_{n}$ and the operator\\ $A \coloneqq (A_{n} + 2\pi\im n)_{n \in \N}$ with maximal domain in $E$.
Analogously we define a semigroup $\TT = (T(t))_{t \geq 0}$ by
%% --
\[
    T(t) \coloneqq (\mathrm{e}^{2\pi\im nt}\exp(tA_{n}))_{n \in \N}\,.
\]
%% --
\end{example}
%% --
Since $\|\exp(tA_{n})\| \leq \mathrm{e}^{t}$ for every $n \in \N$, $t \geq 0$, and since $t \mapsto T(t)x$ is continuous on each component $E_{n}$, it follows that $\TT$ is strongly continuous.
Its generator is the operator $A$ as defined above.

For $\lambda \in \C$, $\Re\,\lambda > 0$, we have 
$\lim_{n \to \infty} \|R(\lambda-2\pi\im n,A_{n})\| = 0$, hence
%% --
\[
    (R(\lambda,A_{n}+2\pi\im n))_{n \in \N} = (R(\lambda-2\pi\im n,A_{n}))_{n \in \N}
\]
%% --
is a bounded operator on $E$ representing the resolvent $R(\lambda,A)$.
Therefore we obtain $s(A) \leq 0$.
On the other hand, each $2\pi\im n$ is an eigenvalue of $A$, hence $s(A) = 0$.

Take now $x_{n} \in E_{n}$ as above and consider the sequence $(x_{n})_{n \in \N}$.
From (iii) it follows that for $t > 0$ the number $\mathrm{e}^{t}$ is an approximate eigenvalue of $T(t)$ with approximate eigenvector $(x_{n})_{n \in \N}$ (see Definition~\ref{def:a3-2.1} below).
Therefore $\mathrm{e}^{t} \leq r(T(t)) \leq \|T(t)\|$ and hence $\omega_{0}(\TT) \geq 1$.
On the other hand, it is easy to see that $\|T(t)\| = \mathrm{e}^{t}$, hence $\omega_{0}(\TT) = 1$.


Finally, if we take $S(t) \coloneqq \mathrm{e}^{-t/2}T(t)$, we obtain a semigroup $\mathcal{S}$ 
having spectral bound $-\frac{1}{2}$ but satisfying $\lim_{t \to \infty} \|S(t)\| = \infty$ in contrast with Corollary~\ref{cor:a3-1.2}.

These examples show that neither the conclusion of Corollary~\ref{cor:a3-1.2}, \ie \enquote{$s(A) < 0$ implies stability}, nor the \enquote{spectral mapping theorem}
%% --
\[
    \sigma(T(t)) = \exp(t\cdot\sigma(A))
\]
%% --
is valid for arbitrary strongly continuous semigroups.
A careful analysis of the general situation will be given in Section 6 below, but we will first develop systematically the necessary spectral theoretic tools for unbounded operators.
%% --
\section{The Fine Structure of the Spectrum}\label{sec:a3-2}
\index{Spectrum!Fine Structure}

As usual, with a closed linear operator $A$ with dense domain $D(A)$ in a Banach space $E$, we associate its spectrum $\sigma(A)$, its resolvent set $\rho(A)$ and its resolvent
%% --
\[
    \lambda \mapsto R(\lambda,A) \coloneqq (\lambda - A)^{-1}
\]
%% --
which is a holomorphic map from $\rho(A)$ into $\mathcal{L}(E)$.
In contrast to the finite dimensional situation, where a linear operator fails to be surjective if and only if it fails to be injective, we now have to distinguish different cases of \emph{non-invertibility} of $\lambda - A$.
This distinction gives rise to a subdivision of $\sigma(A)$ into different subsets.
We point out that these subsets need not be disjoint. Our definitions are
justified by the fact that for each of the following subsets of $\sigma(A)$ there exist canonical constructions converting the corresponding spectral values into eigenvalues (see Proposition~\ref{prop:a3-2.2}.(ii) and Proposition~\ref{prop:a3-4.4} below).
%% --
\begin{definition}\label{def:a3-2.1}
For a closed, densely defined, linear operator $A$ with domain $D(A)$ in the Banach space $E$ denote by the
%% --
\begin{enumerate}[(i)]
\item 
\emph{point spectrum} $P\sigma(A)$ the set of all $\lambda \in \C$ such that 
$A - \lambda$ is not injective.

\item 
\emph{approximate point spectrum} $A\sigma(A)$ the set of all $\lambda \in \C$ such that $A - \lambda$ is not injective or $(A - \lambda)D(A)$ is not closed in $E$\,.

\item 
\emph{residual spectrum} $R\sigma(A)$ the set of all $\lambda \in \C$ such that $(A - \lambda)D(A)$ is not dense in $E$.
\end{enumerate}
\end{definition}
%% --
From these definitions it follows that $\lambda \in P\sigma(A)$ if and only if there exists a non-zero \emph{eigenvector} $f \in D(A)$ such that $Af = \lambda f$, \ie $\lambda$ is an \emph{eigenvalue}.
%% --
It follows from the \emph{Open Mapping Theorem} that $\lambda \in A\sigma(A)$ if and only if $\lambda$ is an \emph{approximate eigenvalue}, \ie there exists a sequence $(f_{n})_{n \in \N} \subset D(A)$, called an\emph{ approximate eigenvector}, such that $\|f_{n}\| = 1$ and $ \lim_{n \to \infty} \|Af_{n} - \lambda f_{n}\| = 0$.

%% -- ??? $\lambda \in P\sigma(A) \cup R\sigma(A)$ .

Clearly we have $P\sigma(A) \subset A\sigma(A)$ and $\sigma(A) = A\sigma(A) \cup R\sigma(A)$ where the union need not be disjoint.

The following proposition is a first indication that the subdivision we made yields nice properties.
%% --
\begin{proposition}\label{prop:a3-2.2}
%% --
For a closed, densely defined, linear operator $(A,D(A))$ in a Banach space $E$ the following holds.
%% --
\begin{enumerate}[(i)]

\item
The topological boundary $\partial\sigma(A)$ of $\sigma(A)$ is contained in $A\sigma(A)$.

\item
$R\sigma(A) = P\sigma(A')$ for the adjoint operator $A'$ on $E'$.

\end{enumerate}
\end{proposition}
%% --
\begin{proof}
\begin{enumerate}[(i), wide]

\item 
Take $\lambda_{0} \in \partial\sigma(A)$ and $\lambda_{n} \in \rho(A)$ such that $\lambda_{n} \to \lambda_{0}$.

Since $\|R(\lambda_{n},A)\| \geq r(R(\lambda_{n},A)) = (\text{dist}(x,\sigma(A)))^{-1}$ (see Proposition~\ref{prop:a3-2.5}.(ii)), by the uniform boundedness principle we find $f \in E$ such that
%% --
\[
\lim_{n \to \infty}\|R(\lambda_n ,A)f\| = \infty\,.
\]
%% --
Define $g_{n} \in D(A)$ by
%% --
\[
g_{n} \coloneqq \|R(\lambda_{n},A)f\|^{-1} R(\lambda_{n},A)f
\]
%% --
and use the identity
%% --
\[
(\lambda_{0} - A)g_{n} = (\lambda_{0} - \lambda_{n})g_{n} + (\lambda_{n} - A)g_{n}
\]
%% --
to show that $(g_{n})_{n \in \N}$ is an approximate eigenvector corresponding to $\lambda_{0}$.

\item 
This is a simple consequence of the Hahn-Banach theorem.
%% --
\end{enumerate}
\end{proof}
%% --
In order to illuminate the above definitons we now return to the Standard Examples introduced in Section 2 of A-I and discuss the fine structure of the spectrum of these strongly continuous semigroups, \ie of their generators and their semigroup operators.
%% --
\begin{example}{(The Spectrum of Multiplication Semigroups)}\label{ex:a3-2.3}
%{(The Spectrum of Multiplication Semigroups)}
%\subsection{The Spectrum of Multiplication Semigroups}\label{subsec:a3-2.3}
\index{Example!Multiplication Semigroups}
%% --

Take $E = C_{0}(X)$ for some locally compact space $X$ and take a continuous function $q \colon X \mapsto \C$ whose real part is bounded above.
As observed in A-I,2.3 the multiplication operator
%% --
\[
M_{q} \colon f \mapsto q \cdot f
\]
%% --
with maximal domain $D(M_{q})$ generates the multiplication semigroup
%% --
\[
T(t)f \coloneqq \mathrm{e}^{tq} \cdot f \, , \, f \in E\,.
\]
%% --
Since $M_{q}$ is bounded if and only if $q$ is bounded, we conclude that $M_{q}$ is invertible (with bounded inverse $M_{1/q}$) if and only if
%% --
\[
0 \notin \overline{\{q(x)  \colon x \in X\}}\,.
\]
%% --
Therefore we obtain
%% --
\[
\sigma(M_{q}) = \overline{q(X)} = \overline{\{q(x) \colon x \in X\}}\,,
\]
%% --
and
%% --
\[
\sigma(T(t)) = \overline{\{\exp(tq(x)) \colon x \in X\}}\,.
\]
%% --
In particular the following \emph{weak spectral mapping theorem} is valid
%% --
\[
\sigma(T(t)) = \overline{\exp(t\sigma(M_{q}))}\,.
\]
%% --
In addition, we observe that to each spectral value of $A$ (\resp of $T(t)$) there exists an approximate eigenvector and hence
%% --
\[
\sigma(A) = A\sigma(A) \text{ and } \sigma(T(t)) = A\sigma(T(t))\,.
\]
%% --
Since each Dirac functional is an eigenvector for the adjoint multiplication operator, we obtain
%% --
\[
q(X) \subset R\sigma(M_{q}) \text{ and } \mathrm{e}^{tq(X)} \subset R\sigma(T(t))\,.
\]
%% --
The eigenvalues of $M_{q}$ can be characterized as follows.
%% --
\begin{quote}
$\lambda \in P\sigma(M_{q})$ if and only if the set $\{x \in X \colon q(x) = \lambda\}$ has non empty interior (analogously for $P\sigma(T(t))$).
\end{quote}
%% --
For example, it follows that $P\sigma(M_{q}) = \emptyset$ for $E = C_{0}(\R_{+})$ and $q(x) = -x$, $x \in \R_{+}$.

On $E = L^{p}(X,\Sigma,\mu)$ analogous results are valid, but their exact formulation---using the notion \emph{essential range}, see \citet{goldstein:1985a}---is left to the reader.
%% --
\end{example}
%% --
\begin{example}{(The Spectrum of Translation Semigroups)}\label{ex:a3-2.4}
%\subsection{The Spectrum of Translation Semigroups}\label{subsec:a3-2.4}
\index{Examples!Translation Semigroups}
%% --

We consider the translation semigroup
%% --
\[
T(t)f(x) \coloneqq f(x+t)
\]
%% --
on $E = C_{0}(\R_{+})$ (or $L^{p}(\R_{+})$, see A-I,2.4).
Its generator $A$ is the first derivative and for every $\lambda \in \C$, $\Re\,\lambda < 0$, the function $\epsilon_{\lambda} \colon x \mapsto \mathrm{e}^{\lambda x}$ belongs to $D(A)$ and satisfies
%% --
\[
A\epsilon_{\lambda} = \lambda\epsilon_{\lambda}\,,
\]
%% --
hence $\lambda \in P\sigma(A)$.
Since $\TT = (T(t))_{t \geq 0}$ is a contraction semigroup it follows that $\sigma(A) = \{\lambda \in \C \colon 
\Re\lambda \leq 0\}$ and $\im\R \subset A\sigma(A)$ (use Proposition~\ref{prop:a3-2.2}.(i)) or show directly that $f_{n}(x) = \mathrm{e}^{ \im\alpha x}\mathrm{e}^{-x/n}$ defines an approximate eigenvector for $\im\alpha$, $\alpha \in \R$).
Using the same functions one obtains
%% --
\begin{align*}
	P\sigma(T(t)) &= \{\mathrm{e}^{\lambda t} \colon \Re\,\lambda < 0\} = \{z \in \C \colon |z| < 1\}\,,\\
	\sigma(T(t)) &= \{z \in \C \colon |z| \leq 1\} \text{ for every } t > 0\,.
\end{align*}
%% --
In the case of the translation group on $E = C_{0}(\R)$ one has $\sigma(A) \subset \im\R$.
As above one obtains approximate eigenvectors for every $\alpha \in \R$ from 
$f_{n}(x) = \mathrm{e}^{\im\alpha x}\mathrm{e}^{-|x|/n}$, hence
%% --
\[
\sigma(A) = A\sigma(A) = \im\R\,.
\]
%% --
The generator $A$ of the nilpotent translation semigroup A-I,2.6 has empty spectrum by A-I, Proposition~1.11.
The resolvent is given by
%% --
\[
R(\lambda,A)f(x) = \mathrm{e}^{\lambda x}\int_{x}^{\infty}\mathrm{e}^{-\lambda s}f(s) \, \ds \quad (f \in L^{p}([0,\tau]), \lambda \in \C)\,.
\]
%% --
Finally, the generator of the periodic translation group from A-I,2.5 on
\\
$E = \{f \in C[0,1] \colon f(0) = f(1)\}$ has point spectrum
%% --
\[
P\sigma(A) = 2\pi \im\Z
\]
%% --
with eigenfunctions $\epsilon_{n}(x) \coloneqq \exp(2\pi \im n x)$.
In Section 5 we show that $\sigma(A) = 2\pi \im\Z$.
%% --
\end{example}
%% --
We now return to the general theory and recall from Corollary~\ref{cor:a3-1.2} that it is very useful (\eg for stability theory) to be able to convert
spectral values of the generator $A$ into spectral values of the semigroup operator $T(t)$ and vice versa.
As shown in Examples~\ref{eq:a3-1.3} and \ref{eq:a3-1.4} this is not possible in general.
Therefore we tackle first a much easier \emph{spectral mapping theorem}: the relation between $\sigma(A)$ and $\sigma(R(\lambda_{0}))$, where $R(\lambda_{0}) \coloneqq R(\lambda_{0},A)$ for some $\lambda_{0} \in \rho(A)$.
%% --
\begin{proposition}\label{prop:a3-2.5}
Let $(A,D(A))$ be a densely defined closed linear operator with non-empty resolvent set $\rho(A)$.
For each $\lambda_{0} \in \rho(A)$ the following assertions hold.

\begin{enumerate}[(i)]
\item 
$\sigma(R(\lambda_{0})) \setminus \{0\} = (\lambda_{0} - \sigma(A))^{-1}$, in  particular, $r(R(\lambda_{0})) = (\mathrm{dist}(\lambda_{0},\sigma(A)))^{-1}$.

\item 
Analogous statements hold for the point-, approximate point-, residual spectra of $A$ and $R(\lambda_{0},A)$.

\item 
The point $\alpha$ is isolated in $\sigma(A)$ if and only if $(\lambda_{0}-\alpha)^{-1}$ is isolated in $\sigma(R(\lambda_{0}))$.
In that case the residues (\resp the pole orders) in $\alpha$ and in $(\lambda_{0}-\alpha)^{-1}$ coincide.
\end{enumerate}
\end{proposition}
%% --
\begin{proof}
\begin{enumerate}[(i), wide]
\item 
is well known. It can be found for example in \citet[VII.9.2]{dunfordschwartz:1958}.

\item 
We show that $\alpha \in A\sigma(A)$ if $(\lambda_{0}-\alpha)^{-1} \in A\sigma(R(\lambda_{0}))$ and leave the proof of the remaining statements to the reader.

Take $(f_{n})_{n \in \N} \subset E$ such that $\|f_{n}\| = 1$, $\|(\lambda_{0}-\alpha)^{-1}f_{n} - R(\lambda_{0},A)f_{n}\| \to 0$ and $\|R(\lambda_{0},A)f_{n}\| \geq \frac{1}{2}|\lambda_{0} - \alpha|^{-1}$.
Define
%% --
\[
g_{n} \coloneqq \|R(\lambda_{0},A)f_{n}\|^{-1}R(\lambda_{0},A)f_{n} \in D(A)
\]
%% --
and deduce from
%% --
\begin{align*}
(\alpha-A)g_{n} &= \|R(\lambda_{0},A)f_{n}\|^{-1} \cdot 
		[(\lambda_{0}-A) - (\lambda_{0}-\alpha)]R(\lambda_{0},A)f_{n} \\  
	&= \|R(\lambda_{0},A)f_{n}\|^{-1} \cdot 			(\lambda_{0}-\alpha)[(\lambda_{0}-\alpha)^{-1} - R(\lambda_{0},A)]f_{n}
\end{align*}
%% --
that $(g_{n})$ is an approximate eigenvector of $A$ to the eigenvalue $\alpha$.

\item 
First we recall the wellknown \emph{resolvent equation}. For any $z,\lambda_0 \in \rho(A)$ we have $R(\lambda_0,A) - R(z,A))= -(\lambda_0-z)R(\lambda_0,A)R(z,A))$\,. From this it follows that 
$(\lambda_0-z)^2\cdot R(z,A) = \left((\lambda_0-z)^{-1}-R(\lambda_0,A)\right)^{-1} - (\lambda_0-z)$\,.
If we now take a circle $\Gamma$ with center $\alpha$ and sufficiently small radius. 
Then the residue $P$ of $R(\cdot,A)$ at $\alpha$ is
%% --
\begin{align*}
P &=  \frac{1}{2\pi\im } \int_{\Gamma} R(z,A)   \diff{z} = \\
&=  \frac{1}{2\pi\im } \left[\int_{\Gamma} (\lambda_{0}-z)^{-2}R((\lambda_{0}-z)^{-1}\,R(\lambda_{0},A)) \, \diff{z}   - 
   \int_{\Gamma}(\lambda_{0}-z)^{-1} \diff{z} \right] . 
\end{align*}
%% --
If $\lambda_{0}$ lies in the exterior of $\Gamma$, the second integral is zero.
The substitution $\tilde{z} \coloneqq (\lambda_{0} - z)^{-1}$ yields a path $\tilde{\Gamma}$ around $(\lambda_{0}-\alpha)^{-1}$ and we obtain
%% --
\[
P = \frac{1}{2\pi\im } \int_{\tilde{\Gamma}} R(\tilde{z},R(\lambda_{0},A)) \, \mathrm{d}\tilde{z}
\]
%% --
which is the residue of $R(\cdot,R(\lambda_{0},A))$ at $(\lambda_{0}-\alpha)^{-1}$.
The final assertion on the pole order follows from the identities
%% --
\[
V_{-n} = ((\lambda_{0}-\alpha)^{-1}R(\lambda_{0},A))^{n-1}U_{-n}, \quad n \in \N\,,
\]
%% --
where $U_{n}$, \resp $V_{n}$ stand for the n-th coefficient in the Laurent series of $R(\cdot,A)$, \resp $R(\cdot,R(\lambda_{0},A))$ at $\alpha$, \resp $(\lambda_{0}-\alpha)^{-1}$.
This has already been proved for $n = 1$ and follows for $n > 1$ by induction using the relations
%% --
\[
U_{-n-1} = (A - \alpha)U_{-n} \quad \text{and} \quad V_{-n-1} = \left(R(\lambda_{0},A) - (\lambda_{0}-\alpha)^{-1}\right)V_{-n}\,.
\]
\end{enumerate}
%% --
\end{proof}
%% --
\section{Spectral Decomposition}\label{sec:a3-3}
\index{Spectral Decomposition}
%% --
In the next two sections we develop some important techniques for our further investigation of semigroups and their generators.
Even though these methods are well known (compare, e.g. Section VII.3 of \citet{dunfordschwartz:1958}) or rather technical, it is useful to present them in a coherent way.

Our interest in this section is the following: Let $E$ be a Banach space and $\TT = (T(t))_{t \geq 0}$ a strongly continuous semigroup with generator $A$.
Suppose that the spectrum $\sigma(A)$ splits into the disjoint union of two closed subsets $\sigma_{1}$ and $\sigma_{2}$.
Does there exist a corresponding decomposition of the space $E$ and the semigroup $\TT$\,?

In the following definition, we explain what we understand by \enquote{corresponding decomposition}.
%% --
\begin{definition}\label{def:a3-3.1}
Assume that $\sigma(A)$ is the disjoint union
%% --
\[
\sigma(A) = \sigma_{1} \cup \sigma_{2}
\]
%% --
of two non-empty closed subsets $\sigma_{1}$, $\sigma_{2}$.
A decomposition
%% --
\[
E = E_{1} \oplus E_{2}
\]
%% --
of $E$ into the direct sum of two non-trivial closed $\TT$-invariant subspaces is called a \emph{spectral decomposition} corresponding to $\sigma_{1} \cup \sigma_{2}$ if the spectrum $\sigma(A_{i})$ of the generator $A_{i}$ of $\TT_{i} \coloneqq (T(t)_{|E_{i}})_{t \geq 0}$ coincides with $\sigma_{i}$ for $i = 1$, $2$.
\end{definition}
%% --
For a better understanding of the above definition we recall that to every direct sum decomposition $E = E_{1} \oplus E_{2}$ there corresponds a continuous projection $P \in \LE$ such that $PE = E_{1}$ and $P^{-1}(0) = E_{2}$.
Moreover, the subspaces $E_{1}$, $E_{2}$ are $\TT$-invariant if and only if $P$ commutes with the semigroup $\TT$, \ie $T(t)P = PT(t)$ for every $t \geq 0$.
In this case it follows that the domain $D(A)$ of the generator $A$ splits analogously and $D(A) \cap E_{i}$ is the domain $D(A_{i})$ of the generator $A_{i}$ of the restricted semigroup $\TT_{i}$, $i = 1$, $2$.
We write
%% --
\[
A = A_{1} \oplus A_{2}\,.
\]
%% --
and say that \emph{$A$ commutes with $P$} and call $P$ a \emph{spectral projection}.
In terms of the generator $A$ this means that for $f \in D(A)$ we have $Pf \in D(A)$ and $APf = PAf$.

The existence of such projections reduces the semigroup $\TT$ into two (possibly simpler) semigroups $\TT_{1}$, $\TT_{2}$ such that
%% --
\[
\sigma(A) = \sigma(A_{1}) \cup \sigma(A_{2}) \quad \text{and} \quad \sigma(T(t)) = \sigma(T_{1}(t)) \cup \sigma(T_{2}(t))\,.
\]
%% --
For example, in some cases (see Theorem~\ref{thm:a3-3.3} below) it can be shown that one of the reduced semigroups has additional properties.

In order to achieve such decompositions we will assume that $\sigma(A)$ decomposes into sets $\sigma_{1}$ and $\sigma_{2}$ and will then try to find a corresponding spectral projection.
Unfortunately such spectral decompositions do not exist in general.
%% --
\begin{example}\label{ex:a3-3.2}
\index{Examples!Spectral Decomposition!Non-existence}
%% --
Take the rotation semigroup from A-I,2.4 on the Banach space $L^{p}(\Gamma)$, $1 \leq p < \infty$, $\tau = 2\pi$.
It was stated in Example~\ref{ex:a3-2.4} and will be proved in Section 5 that its generator $A$ has spectrum
%% --
\[
\sigma(A) = P\sigma(A) = \im\Z
\]
%% --
where $\epsilon_{k}(z) \coloneqq z^{k}$ spans the eigenspace corresponding to $\im k$, $k \in \Z$.

Now, $\sigma(A)$ is the disjoint union of 
$\sigma_{1} \coloneqq \{0,\im ,2\im, \ldots\}$ 
and $\sigma_{2} \coloneqq \{-\im, -2\im, \ldots\}$.
By a result of M. Riesz there is no projection $P \in \L{L^{1}(\Gamma)}$ satisfying $P\epsilon_{k} = \epsilon_{k}$ for $k \geq 0$, $P\epsilon_{k} = 0$ for $k < 0$ , hence there is no spectral decomposition of $L^{1}(\Gamma)$ corresponding to $\sigma_{1}$, $\sigma_{2}$ (\citet[p.165]{lindenstraustzafriri:1979}).

On the other hand, for $L^{p}(\Gamma)$, $1 < p < \infty$, such a spectral projection exists (l.c., 2.c.15).
As long as $p \neq 2$ we can always decompose $\sigma(A)$ into suitable subsets admitting no spectral decomposition (l.c., remark before 2.c.15).
Clearly, for $p = 2$ such spectral decompositions always exist.
\end{example}
%% --
In the above example both subsets $\sigma_{1}$, $\sigma_{2}$ of $\sigma(A)$ are unbounded.
But as soon as one of these sets is bounded a corresponding spectral decomposition can always be found.
%% --
\begin{theorem}\label{thm:a3-3.3}
Let $\TT$ be a strongly continuous semigroup on a Banach space $E$ and assume that the spectrum $\sigma(A)$ of the generator $A$ can be decomposed into the disjoint union of two non-empty closed subsets $\sigma_{1}$, $\sigma_{2}$.

If $\sigma_{1}$ is compact, then there exists a unique corresponding spectral decomposition $E = E_{1} \oplus E_{2}$ such that the restricted semigroup $\TT_{1}$ has a bounded generator.
\end{theorem}
%% --
\begin{proof}
We assume the reader to be familiar with the spectral decomposition theorem for bounded operators (see, \eg \citet[p.572]{dunfordschwartz:1958}) and apply the spectral mapping theorem for the resolvent (Proposition~\ref{prop:a3-2.5}.(i)) in order to decompose $R(\lambda,A)$ instead of $A$.

For $\lambda_{0} > \omega_{0}(\TT)$ it follows from Proposition~\ref{prop:a3-2.5} that $\sigma(R(\lambda_{0},A)) \setminus \{0\} = (\lambda_{0} - \sigma(A))^{-1}$.
From $\sigma(A) = \sigma_{1} \cup \sigma_{2}$ we obtain a decomposition of $\sigma(R(\lambda_{0},A)) \setminus \{0\}$ into
%% --
\[
\tau_{1} \coloneqq (\lambda_{0}-\sigma_{1})^{-1}, \quad \tau_{2} \coloneqq (\lambda_{0}-\sigma_{2})^{-1}\,.
\]
%% --
Since $\sigma_{1}$ is compact, the set $\tau_{1}$ is compact and does not contain $0$.
Only in the case that $\sigma_{2}$ is unbounded the point $0$ will be an accumulation point of $\tau_{2}$.
Therefore $\sigma(R(\lambda_{0},A)) \cup \{0\}$ is the disjoint union of the closed sets $\tau_{1}$ and $\tau_{2} \cup \{0\}$.

Take now $P$ to be the spectral projection of $R(\lambda_{0},A)$ corresponding to this decomposition.
Then $P$ commutes with $R(\lambda_{0},A)$ (by definition), with $R(\lambda,A)$ for every $\lambda > \omega_{0}(\TT)$ (use the series representation of the resolvent), with $T(t)$ for each $t \geq 0$ (use A-II, Proposition~1.10) and therefore with the generator $A$ (in the sense explained above).
In particular, we obtain
%% --
\[
R(\lambda_{0},A)P = R(\lambda_{0},A_{1}), \quad R(\lambda_{0},A)(Id-P) = R(\lambda_{0},A_{2})
\]
%% --
for the generator $A_{1}$ of $T_{1} = (T(t)P)_{t \geq 0}$ and $A_{2}$ of $T_{2} = (T(t)(Id-P))_{t \geq 0}$.
Applying the Spectral Mapping Theorem 2.5 we conclude
%% --
\[
\sigma(A_{1}) = \sigma_{1} \text{ and } \sigma(A_{2}) = \sigma_{2}\,,
\]
%% --
\ie $P$ is a spectral projection corresponding to $\sigma_{1}$, $\sigma_{2}$.
Finally, the above spectral decomposition of $R(\lambda_{0},A)$ is unique and satisfies $0 \notin \sigma(R(\lambda_{0},A_{1}))$.
Therefore $R(\lambda_{0},A_{1})^{-1} = (\lambda_{0}-A_{1})$ is bounded.
\end{proof}
%% --
If we do not require $\TT_{1}$ to be uniformly continuous, the above spectral decomposition need not be unique, as can be seen from the following example. 

Consider a decomposition $E = E_{1} \oplus E_{2}$ and add a direct summand $E_{3}$ with a strongly continuous semigroup $T_{3}$ whose generator $A_{3}$ has empty spectrum (\eg A-I,Example 2.6).
Then still $\sigma(A) = \sigma_{1} \cup \sigma_{2}$, but $E_{1} \oplus (E_{2} \oplus E_{3})$ and $(E_{1} \oplus E_{3}) \oplus E_{2}$ are two different spectral decompositions corresponding to $\sigma_{1}$, $\sigma_{2}$.
%% --

The importance of the above theorem stems from the fact that $\TT_{1}$ has a bounded generator and therefore is easy to deal with.
In particular the asymptotic behavior of $\TT_{1}$ can be deduced from the location of $\sigma_{1}$.
%% --
\begin{corollary}\label{cor:a3-3.4}
Assume that $\sigma(A)$ splits into non-empty closed sets $\sigma_{1}$, $\sigma_{2}$ where $\sigma_{1}$ is compact and consider the corresponding spectral decomposition $E = E_{1} \oplus E_{2}$ for which $\TT_{1}$ is uniformly continuous.

For all constants $\nu$, $\omega \in \R$ satisfying
%% --
\[
\nu < \inf \{\Re\,\lambda \colon \lambda \in \sigma_{1}\} \leq \sup \{\Re\,\lambda \colon \lambda \in \sigma_{1}\} < \omega
\]
%% --
there exist $m \leq 1$, $M \geq 1$ such that
%% --
\[
m \cdot \mathrm{e}^{\nu t}\|f\| \leq \|T_{1}(t)f\| \leq M \cdot \mathrm{e}^{\omega t}\|f\|
\]
%% --
for every $f \in E_{1}$, $t \geq 0$.
\end{corollary}
%% --
\begin{proof}
Since the generator $A_{1}$ of $\TT_{1}$ is bounded, we have $T_{1}(t) = \exp(tA_{1})$ and $\sigma(T_{1}(t)) = \exp(t\sigma(A_{1}))$.
Therefore by the remark following Proposition~\ref{prop:a3-1.1}, the spectral bound $s(A_{1})$ coincides with the growth bound $\omega_{0}(T_{1})$ and we have the upper estimate.
The lower estimate is obtained by applying the same reasoning to $-A_{1}$ which generates the semigroup $(T_{1}(t)^{-1})_{t \geq 0}$ on $E_{1}$.
\end{proof}
%% --
%% old It is clear from Examples 1.3, 1.4 
It is clear from Examples~\ref{ex:a3-1.3} and \ref{ex:a3-1.4} on page \pageref{ex:a3-1.4} that no norm estimates for $(T_{2}(t))_{t \geq 0}$ can be obtained from the location of $\sigma_{2}$.
Only by adding appropriate hypotheses we will achieve spectral decompositions admitting norm estimates on both components (see Theorem~\ref{thm:a3-6.6} below).

Another way of obtaining such norm estimates is by constructing spectral decompositions starting from a semigroup operator $T(t_{0})$ (instead of $A$, and $R(\lambda,A)$ \resp, as in Theorem~~\ref{thm:a3-3.3}).
%% --
\begin{corollary}\label{cor:a3-3.5}
If $\sigma(T(t_{0})) = \tau_{1} \cup \tau_{2}$ for two non-empty, closed, disjoint sets $\tau_{1}$, $\tau_{2}$ and if $P$ is the spectral projection corresponding to $T(t_{0})$ and $\tau_{1}$, $\tau_{2}$, then $\sigma(A)$ splits into closed subsets $\sigma_{1}$, $\sigma_{2}$ and $P$ is the corresponding spectral projection for $\TT$ and $\sigma_{1}$, $\sigma_{2}$.
\end{corollary}
%% --
\begin{proof}
The spectral projection $P$ of $T(t_{0})$ is obtained by integrating $R(\lambda,T(t_{0}))$ (see, \eg \citet[Section VII.3]{dunfordschwartz:1958}).
Since every $T(t)$, $t \geq 0$, commutes with $T(t_{0})$, it must commute with $R(\lambda,T(t_{0}))$, hence with $P$.
The statement on the decomposition $\sigma(A) = \sigma_{1} \cup \sigma_{2}$ follows from the Spectral Inclusion Theorem 6.2 below.
\end{proof}
%% --
This decomposition can be applied to the study of the asymptotic behavior of $\TT$. In the situation of Corollary~\ref{cor:a3-3.5} assume
%% --
\[
\sup \{|\lambda| \colon \lambda \in \tau_{2}\} < \alpha < \inf \{|\lambda| \colon \lambda \in \tau_{1}\}\,.
\]
%% --
for some $\alpha > 0$. If we set $\beta \coloneqq (\log\alpha)/t_{0}$ and use \citet[Chap.I, Theorem~6.5]{pazy:1983} we obtain $\omega_{0}(\TT_{2}) < \beta$ and $\omega_{0}(\TT_{1}^{-1}) < \beta$ by Proposition~\ref{prop:a3-1.1}.
Therefore we have constants $m$, $M$ with $m \le 1 \le M $ such that
%% --
\begin{align*}
	\|T(t)f\| &\leq M \cdot \mathrm{e}^{\beta t}\|f\| \quad \text{for } f \in E_{2}\,, \\
	\|T(t)f\| &\geq m \cdot \mathrm{e}^{-\beta t}\|f\| \quad \text{for } f \in E_{1}\,.
\end{align*}
%% --
As nice as they might look, results of this type are unsatisfactory. We need information on the semigroup in order to estimate its asymptotic behavior.
In Chapter IV we will try to obtain such results by exploiting information about the generator only.
%% --
\begin{example}{\textbf{(Isolated singularities and poles)}}\label{ex:a3-3.6}
\index{Examples!Isolated singularities and poles}
%% --

In case that $\lambda_{0}$ is an isolated point of $\sigma(A)$ the holomorphic function $\lambda \mapsto R(\lambda,A)$ can be expanded as a Laurent series
%% --
\[
R(\lambda,A) = \sum_{n=-\infty}^{+\infty} U_{n}(\lambda - \lambda_{0})^{n} \text{ for } 0 < |\lambda - \lambda_{0}| < \delta \text{ and some } \delta > 0\,.
\]
%% --
The coefficients $U_{n}$ are bounded linear operators given by
%% --
\begin{equation}\label{eq:a3-3.1}
U_{n} = \frac{1}{2\pi\im }\int_{\Gamma} (z - \lambda_{0})^{-(n+1)}R(z,A) \, \diff{z} \, , \, n \in \Z\,,
\end{equation}
%% --
where $\Gamma = \{z \in \C \colon |z - \lambda_{0}| = \delta/2\}$.
The coefficient $U_{-1}$ is the spectral projection corresponding to the spectral set $\{\lambda_{0}\}$ (see Definition~\ref{def:a3-3.1}) is called the \emph{residue} of $R(\cdot,A)$ at $\lambda_{0}$, and will be denoted by $P$.
From \eqref{eq:a3-3.1} one deduces
%% --
\begin{equation}\label{eq:a3-3.2}
U_{-(n+1)} = (A - \lambda_{0})^{n} \circ P \quad \text{and} \quad U_{-(n+1)} \circ U_{-(m+1)} = U_{-(n+m+1)} \text{ for } n, m \geq 0\,.
\end{equation}
%% --
If there exists $k > 0$ such that $U_{-k} \neq 0$ while $U_{-n} = 0$ for all $n > k$, the point $\lambda_{0}$ is called a \emph{pole of} $R(\cdot,A)$ \emph{of order} $k$.
In view of \eqref{eq:a3-3.2} this is true if $U_{-k} \neq 0$ and $U_{-(k+1)} = 0$.
In this case one can retrieve $U_{-k}$ as
%% --
\begin{equation}\label{eq:a3-3.3}
U_{-k} = \lim_{\lambda \to \lambda_{0}} (\lambda - \lambda_{0})^{k}R(\lambda,A)\,.
\end{equation}
%% --
The dimension of $PE$ (\ie the dimension of the spectral subspace corresponding to $\{\lambda_{0}\}$) is called the \emph{algebraic multiplicity} $m_{a}$ of $\lambda_{0}$, while the \emph{geometric multiplicity} is $m_{g} \coloneqq \text{dim ker}(\lambda_{0} - A)$.
In case $m_{a} = 1$, we call $\lambda_{0}$ an \emph{algebraically simple pole}.

If $k$ is the pole order ($k = \infty$ in case of an essential singularity), we have
%% --
\begin{equation}\label{eq:a3-3.4}
	\max\{m_{g},k\} \leq m_{a} \leq k \cdot m_{g}
\end{equation}
%% --
where $\infty \cdot 0 = \infty$.

These inequalities yield the following implications.
%% --
\begin{enumerate}[(i)]
\item
$m_{a} < \infty$ if and only if $\lambda_{0}$ is a pole with $m_{g} < \infty$,
\item
if $\lambda_{0}$ is a pole with order $k$, then $\lambda_{0} \in P\sigma(A)$ and $PE = \text{ker}(\lambda_{0} - A)^{k}$.
\end{enumerate}
%% --
If $A$ has compact resolvent, then every point of $\sigma(A)$ is a pole of finite algebraic multiplicity.
This is a consequence of Proposition~\ref{prop:a3-2.5}.(iii) and the well-known Riesz-Schauder Theory for compact operators (see \citet[VII.4.5]{dunfordschwartz:1958}).
\end{example}
%% --
\begin{example}{\textbf{(The essential spectrum)}}
\label{subsec:a3-3.7}	
%\medskip\noindent
%\textbf{Example}.\ \ (The essential spectrum) 
%\index{Examples!Essential Spectrum}
%% --

For an operator $T \in \LE$ the \emph{Fredholm domain} $\rho_{F}(T)$ is
%% --
\begin{equation}\label{eq:a3-3.5} 
\begin{aligned}
	\rho_{F}(T) & \coloneqq  \{\lambda \in \C \colon \lambda - T \text{ is a Fredholm operator}\}\\
	& =  \{\lambda \in \C \colon \text{ker}(\lambda - T) \text{ and }  E / \mathrm{im}(\lambda - T) \text{ are finite dimensional}\}.
\end{aligned}
\end{equation}
%% --
An equivalent characterization of $\rho_{F}(T)$ is obtained through the 
\hfill\break
Calkin algebra $\LE/\mathcal{K}(E)$, where $\mathcal{K}(E)$ stands for the closed ideal of all compact operators.
In fact, $\rho_{F}(T)$ coincides with the resolvent set of the canonical image of $T$ in the Calkin algebra.
The complement of $\rho_{F}(T)$ is called \emph{essential spectrum} of $T$ and denoted by $\sigma_{\text{ess}}(T)$.
The corresponding spectral radius, called \emph{essential spectral radius}, satisfies
%% --
\begin{equation}\label{eq:a3-3.6}
r_{\text{ess}}(T) \coloneqq \sup \{|\lambda| \colon \lambda \in \sigma_{\text{ess}}(T)\} = \lim_{n \to \infty} \|T^{n}\|_{\text{ess}}^{1/n}\,,
\end{equation}
%% --
where $\|T\|_{\text{ess}} = \text{dist}(T,\mathcal{K}(E)) \coloneqq \inf \{\|T - K\| \colon K \in \mathcal{K}(E)\}$ is the norm of $T$ in $\LE/\mathcal{K}(E)$\,.

For every compact operator $K$ we have $\|T - K\|_{\text{ess}} = \|T\|_{\text{ess}}$, hence
%% --
\begin{equation}\label{eq:a3-3.7}
r_{\text{ess}}(T - K) = r_{\text{ess}}(T)\,.
\end{equation}
%% --
A detailed analysis of $\rho_{F}(T)$ can be found in Section IV.5.6 of \citet{kato:1966}.
In particular we recall that the poles of $R(\cdot,T)$ with finite algebraic multiplicity belong to $\rho_{F}(T)$.
Conversely, an element of the unbounded component of $\rho_{F}(T)$ either belongs to $\rho(T)$ or is a pole of finite algebraic multiplicity.
Thus $r_{\text{ess}}(T)$ can be characterized as
%% --
\begin{equation}\label{eq:a3-3.8}
\begin{split}
r_{\text{ess}}(T) \text{ is the smallest } r \in \R_{+}
& \text{ such that every } \lambda \in \sigma(T), |\lambda| > r \\
& \text{ is a pole of finite algebraic multiplicity.}
\end{split}
\end{equation}
%% --
Now, if $\TT = (T(t))_{t \geq 0}$ is a strongly continuous semigroup, then VIII.1, Lemma~4 of \citet{dunfordschwartz:1958} applied to the function $t \mapsto \log \|T(t)\|_{\text{ess}}$ ensures that
%% --
\begin{equation}\label{eq:a3-3.9}
\omega_{\text{ess}}(\TT) \coloneqq \lim_{t \to \infty} \frac{1}{t} \log\|T(t)\|_{\text{ess}} = \inf \{\frac{1}{t} \log\|T(t)\|_{\text{ess}} \colon t > 0\}
\end{equation}
%% --
is well defined (possibly $-\infty$).
By the definition of $\omega_{\text{ess}}(\TT)$ and \eqref{eq:a3-3.6} we have
%% --
\begin{equation}\label{eq:a3-3.10}
r_{\text{ess}}(T(t)) = \exp(t\omega_{\text{ess}}(\TT)), \quad t \geq 0\,.
\end{equation}
%% --
Obviously, $\omega_{\text{ess}} \leq \omega_{0}$ and equality occurs if and only if $r_{\text{ess}}(T(t)) = r(T(t))$ for $t \geq 0$.

If $\omega_{\text{ess}} < \omega_{0}$, there exists an eigenvalue $\lambda$ of $T(t)$ satisfying $|\lambda| = r(T(t))$, hence by Theorem~\ref{thm:a3-6.3} below there exists $\lambda_{1} \in P\sigma(A)$ such that $\Re\,\lambda_{1} = \omega_{0}$.
Thus $\omega_{\text{ess}} < \omega_{0}$ implies $s(A) = \omega_{0}(\TT)$, \ie we have
%% --
\begin{equation}\label{eq:a3-3.11}
\omega_{0}(\TT) = \max\{\omega_{\text{ess}}(\TT),s(A)\}\,.
\end{equation}
%% --
As a final observation we point out that
%% --
\begin{equation}\label{eq:a3-3.12}
\omega_{\text{ess}}(\TT) = \omega_{\text{ess}}({\mathcal{S}})\,,
\end{equation}
%% --
whenever $\TT$ is generated by $A$ and $\mathcal{S}$ is generated by $A + K$ for some compact operator $K$ 
see Proposition~2.8  and Proposition~2.9 of B-IV).
%(see Proposition~\ref{prop:b4-2.8} and Proposition~\ref{prop:b4-2.9 of B-IV).
\end{example}
%% --

%% --
\section{The Spectrum of Induced Semigroups}\label{sec:a3-4}
\index{Spectrum!Induced Semigroups}
In the previous section we tried to decompose a semigroup into the direct sum of two, hopefully simpler objects.
Here we present other methods to reduce the complexity of a semigroup and its generator.
Forming subspace or quotient semigroups as in A-I,3.2, A-I,3.3 are such methods.
But also the constructions of new semigroups on canonically associated spaces such as the dual space, see A-I,3.4, or the $\F$-product, see A-I,3.6, might be helpful.
We review these constructions under the spectral theoretical point of view and collect a number of technical properties for later use.

We start by studying the spectrum of subspace and quotient semigroups.
To that purpose assume that the strongly continuous semigroup $\TT = (T(t))_{t \geq 0}$ leaves invariant some closed subspace $N$ of the Banach space $E$.
There are canonically induced semigroups $\TT_{|}$ on $N$, \resp $\TT_{/}$ on $E/N$ and their generators $A_{|}$, \resp $A_{/}$ are canonically obtained from the generator $A$ of $\TT$ (see A-I, Section 3).
The following example shows that the spectra of $A$, $A_{|}$ and $A_{/}$ may differ quite drastically.
%% --
\begin{example}\label{ex:a3-4.1}
\index{Examples!Spectrum of Induced Semigroups}
%% --
As in the example in A-I,3.3 we consider the translation semigroup on $E = L^{1}(\R)$ and the invariant subspace $N \coloneqq \{f \in E \colon f(x) = 0 \text{ for } x \geq 1\}$.
Then $\sigma(A) = \im\R$ but $\sigma(A_{|}) = \{\lambda \in \C \colon \Re\,\lambda \leq 0\}$.
Next we take the translation invariant subspace $M \coloneqq \{f \in N \colon f(x) = 0 \text{ for } 0 \leq x \leq 1\}$ and obtain $\sigma(A_{|/}) = \emptyset$ for the generator $A_{|/}$ of the quotient semigroup $\TT_{|/}$ (use the fact that $\TT_{|/}$ is nilpotent).
\end{example}
%% --

In the next proposition we collect the information on $\sigma(A)$ which in general can be obtained from the \enquote{subspace spectrum} $\sigma(A_{|})$ and the \enquote{quotient spectrum} $\sigma(A_{/})$.

%% --
\begin{proposition}\label{prop:a3-4.2}
\index{Spectrum!Induced Semigroups}
\index{Proposition!Spectrum Relations}

Using the standard notations the following inclusions hold.
\begin{enumerate}[(i)]
\item 
$\rho(A) \subset [\rho(A_{|}) \cap \rho(A_{/})] \cup [\sigma(A_{|}) \cap \sigma(A_{/})]$\,,

\item 
$[\rho(A_{|}) \cap \rho(A_{/})] \subset \rho(A) $\,,

\item 
$\rho_{+}(A) \subset [\rho(A_{|}) \cap \rho(A_{/})]$\,,
\end{enumerate}
 where $\rho_{+}(A)$ denotes the connected component of $\rho(A)$ which is unbounded to the right.

\end{proposition}
%% --

\begin{proof}
\begin{enumerate}[(i), wide]
\item 
Assume $\lambda \in \rho(A)$, \ie $(\lambda-A)$ is a bijection from $D(A)$ onto $E$.
Since $N$ is $T$-invariant, we have $D(A_{|}) = D(A) \cap N$ and $(\lambda-A)D(A_{|}) \subset N$.
If $(\lambda-A)D(A_{|}) = N$, then $R(\lambda,A)N = D(A_{|})$ and the induced operators $R(\lambda,A)_{|}$, \resp $R(\lambda,A)_{/}$ are the inverses of $(\lambda-A_{|})$, \resp $(\lambda-A_{/})$.
If $(\lambda-A)D(A_{|}) \neq N$, then $\lambda \in \sigma(A_{|})$.

In addition there exists $f \in D(A)\backslash N$ such that $g \coloneqq (\lambda-A)f \in N$.
Hence for $\hat{f} \coloneqq f+N$, $\hat{g} \coloneqq g+N \in E_{/}$ it follows that $(\lambda-A_{/})\hat{f} = \hat{g} = 0$, \ie $\lambda \in \sigma(A_{/})$

\item 
Take $\lambda \in \rho(A_{|}) \cap \rho(A_{/})$.
Then $(\lambda-A)$ is injective, since $(\lambda-A)f = 0$ implies $(\lambda-A_{/})\hat{f}= 0$, hence $\hat{f} = 0$, \ie $f \in N$ and therefore $f = 0$.
%% --
%\newpage
%% -- a3-16 / 76
%% --
In addition, $(\lambda-A)$ is surjective: For $g \in E$ there exists $\hat{f} \in E_{/}$ such that $(\lambda-A_{/})\hat{f} = \hat{g}$, \ie there exists $h \in N$ such that $(\lambda-A)f - g = h = (\lambda-A)k$ for some $k \in D(A_{|})$.
Therefore we obtain $(\lambda-A)(f-k) = g$\,.

\item 
The integral representation of the resolvent for $\lambda > \omega_{0}(\TT)$ (see A-I, Proposition~1.11) shows that $R(\lambda,A)N \subset N$.
By the power series expansion for holomorphic functions this extends to all $\lambda \in \rho_{+}(A)$.
Therefore the restriction $R(\lambda,A)_{|}$ coincides with the resolvent $R(\lambda,A_{|})$.
On the other hand $R(\lambda,A)_{/}$ is well defined on $E_{/}$ and satisfies
%% --
\[
R(\lambda,A)_{/}(f+N) = R(\lambda,A)f + N
\]
%% --
(use again the integral representation).
This proves that $R(\lambda,A)_{/} = R(\lambda,A_{/})$.
\end{enumerate}
\end{proof}
%% --
\begin{corollary}\label{cor:a3-4.3}
\index{Corollary!Spectrum of Induced Semigroups}
\index{Spectrum!Poles}

Under the above assumptions take a point $\mu$ in the closure of $\rho_{+}(A)$.

Then
\begin{enumerate}[(i)]
\item 
$\mu \in \sigma(A)$ if and only if $\mu \in \sigma(A_{|})$ or $\mu \in \sigma(A_{/})$\,.

\item 
$\mu$ is a pole of $R(\cdot,A)$ if and only if $\mu$ is a pole of $R(\cdot,A_{|})$ and of $R(\cdot,A_{/})$\,.

In that case,
%% --
\[
\max\{k_{|},k_{/}\} \leq k \leq k_{|} + k_{/}
\]
%% --
for the respective pole orders. Note that hereby pole orders $0$ are allowed.
\end{enumerate}
\end{corollary}
%% --

\begin{proof}
\begin{enumerate}[(i), wide]
\item 
follows from Proposition~\ref{prop:a3-4.2}, inclusions (ii) and (iii).

\item 
By the previous assertion we may assume that for some $\delta > 0$ the pointed disc
%% --
\[
\{\lambda \in \C \colon 0 < |\lambda-\mu| < \delta\}
\]
%% --
is contained in $\rho(A) \cap \rho(A_{|}) \cap \rho(A_{/})$.

Call $U_{n}$ the coefficients of the Laurent expansion of $R(\cdot,A)$.
Since $N$ is $R(\lambda,A)$-invariant for $\lambda \in \rho_{+}(A)$, the same holds for each $U_{n}$.
With the obvious notations we have \quad
$R(\lambda,A) = \sum_{n} U_{n}(\lambda-\mu)^{n}$, $\quad R(\lambda,A)_{|} = \sum U_{n|}(\lambda-\mu)^{n}$ \quad \text{and} \quad $R(\lambda,A)_{/} = \sum U_{n/}(\lambda-\mu)^{n}$~
which shows $\max\{k_{|},k_{/}\} \leq k$.

If $R(\cdot,A)_{|}$ has a pole in $\mu$ of order $\ell$, then $U_{-(\ell+1)|} = 0$, \ie $U_{-(\ell+1)}N = \{0\}$.
Similarly it follows that $U_{-(m+1)}E \subset N$ if $R(\cdot,A)_{/}$ has a pole in $\mu$ of order $m$.
%\newpage
%% -- a3-17 / 77
Therefore $U_{-(\ell+1)} \circ U_{-(m+1)} = 0$.

The relations \eqref{eq:a3-3.2} imply $U_{-(m+\ell+1)} = 0$, hence the pole order of $R(\cdot,A)$ is dominated by $\ell + m$.
\end{enumerate}
\end{proof}

%% --
\subsection{Spectrum of the adjoint semigroup}\label{subsec:a3-4.4}
%\begin{example}{\textbf{Spectrum of the adjoint semigroup}}
%\label{ex:a3-4.4}	
\index{Spectrum!Adjoint Semigroup}
%\index{Adjoint Semigroup!Spectrum}
%\index{Semigroups!Adjoint}
%% --
We recall from A-I,3.4 that to every strongly continuous semigroup $\TT = (T(t))_{t \geq 0}$ there corresponds a strongly continuous adjoint semigroup $\TT^* = (T(t)^*)_{t \geq 0}$ on the semigroup dual
%% --
\[
E^* = \{\phi \in E' \colon \lim_{t \to \infty} \|T(t)'\phi-\phi\| = 0\}\,.
\]
%% --
Its generator $A^*$ is the maximal restriction of the adjoint $A'$ to $E^*$.
For these operators the spectra coincide, or more precisely.
\begin{enumerate}[(i)]
\item 
$\sigma(T(t)) = \sigma(T(t)') = \sigma(T(t)^*)$,\\
$R\sigma(T(t)) = P\sigma(T(t)') = P\sigma(T(t)^*)$\,,

\item 
$\sigma(A) = \sigma(A') = \sigma(A^*)$, $R\sigma(A) = P\sigma(A') = P\sigma(A^*)$\,,
\item 
$s(A) = s(A^*)$, $\omega_{0}(A) = \omega_{0}(A^*)$\,.
\end{enumerate}

\begin{proof}
The left part of these equalities is either well known or has been stated in 
2.2(ii).
The first statement of (iii) follows from (ii), while the second is an immediate consequence of the estimate $\|T(t)^*\| \leq \|T(t)\| \leq M\cdot\|T(t)^*\|$ given in A-I,3.4.

As a sample for the remaining assertions we show that $0 \notin \sigma(A)$ if and only if $0 \notin \sigma(A^*)$:
If $A$ and therefore $A'$ is invertible, it follows from A-I,3.4 that $A^*$ is a bijection from $D(A^*)$ onto $\mathrm{e}^*$.

Conversely assume that $A^*$ is invertible.
Then $A'$ must be injective by the Proposition in A-I,3.4.
Moreover $A'(D(A'))$ contains $A^*(D(A^*)) = \mathrm{e}^*$ and is $\sigma(E',E)$-dense in $E'$.
By standard duality arguments it follows that $A$ is injective with dense image.
Next we show that $A(D(A))$ is closed: For $f \in D(A)$ choose $\phi \in D(A')$ such that $\|\phi\| \leq 1$ and $|\langle f,\phi \rangle| \geq \frac{1}{2}\|f\|$.
Then
%% --
\begin{align*}
\|(A^*)^{-1}\| \|Af\| &\geq \|(A^*)^{-1}\| |\langle Af,\phi \rangle| \geq |\langle Af,(A^*)^{-1}\phi \rangle| \\
&= |\langle f,\phi \rangle| \geq \frac{1}{2}\|f\|\,,
\end{align*}
%% --
hence
%% --
\[
\|Af\| \geq \frac{1}{2}\|(A^*)^{-1}\|^{-1}\|f\|\,,
\]
%% --
and $A(D(A))$ is closed since $A$ is closed.
\end{proof}
%\end{example}

%% --
\subsection{Spectrum of the }% $\mathcal{F}$-product semigroup}
\label{subsec:a3-4.5}
%
%\begin{example}{\textbf{Spectrum of the $\F$-product semigroup}}
%\label{ex:a3-4.5}
\index{Spectrum!$\F$-product Semigroup}
%% --
As stated in A-I,3.6 the $\F$-product semigroup $\TT_{\F} = (T_{\F}(t))_{t \geq 0}$ on $E_{\F}^{\TT}$ of a strongly continuous semigroup $\TT$ on $E$ serves to convert sequences in $E$ into points in $E_{\F}^{\TT}$.
In particular it can be used to convert approximate eigenvectors of the generator $A$ into eigenvectors of $A_{\F}$.
%% --
\begin{proposition}\label{prop:a3-4.4}
\index{$\F$-product!Spectrum}
%% --
Let $A$ be the generator of a strongly continuous semigroup. Then the generator $A_{\F}$ of the $\F$-product semigroup satisfies.
%% --
\begin{enumerate}[(i)]
\item 
	$A\sigma(A) = A\sigma(A_{\F}) = P\sigma(A_{\F})$\,,

\item 
	$\sigma(A) = \sigma(A_{\F})$\,.
\end{enumerate}
\end{proposition}
%% --
\begin{remark}
In case $A$ is bounded, then the canonical extension $A_{\F}$ is a generator and $E_{\F}^{\TT} = E_{\F}$ (cf. A-I,3.6).
Thus the proposition applies to bounded linear operators and their canonical extensions to the 
$\F$-product $E_{\F}$.
\end{remark}
%% --
\begin{proof}[Proof of the proposition]
\begin{enumerate}[(i), wide]
\item 
The inclusion $P\sigma(A_{\F}) \subset A\sigma(A_{\F})$ holds trivially.

We show that $A\sigma(A_{\F}) \subset A\sigma(A)$: Take $\lambda \in A\sigma(A_{\F})$ and an associated approximate eigenvector $({\hat{f}^{m}})_{n \in \N}$, \ie $\hat{f}^{m} \in D(A_{\F})$, $\|\hat{f}^{m}\| = 1$ and $(\lambda-A_{\F})\hat{f}^{m} \to 0$ as $m \to \infty$.

By the considerations in A-I,3.6 we can represent each $\hat{f}^{m}$ as a normalized sequence $(f_{n}^{m})_{n \in \N}$ in $D(A)$ such that
%% --
\[
\lim_{m \to \infty} \limsup_{n \to \infty} \|(\lambda-A)f_{n}^{m}\| = 0\,.
\]
%% --
Therefore we can find a sequence $g_{k} = f_{k}^{m(k)}$ satisfying
%% --
\[
\lim_{k \to \infty} \|(\lambda-A)g_{k}\| = 0\,,
\]
%% --
\ie $\lambda \in A\sigma(A)$.

Finally we show $A\sigma(A) \subset P\sigma(A_{\F})$: For $\lambda \in A\sigma(A)$ take a corresponding approximate eigenvector $(f_{n})$.
By A-I,(3.2) we have
%% --
\begin{align*}
\|T(t)f_{n} - f_{n}\| &\leq \|T(t)f_{n} - \mathrm{e}^{\lambda t}f_{n}\| + |\mathrm{e}^{\lambda t} - 1| \\
&= \left\|\int_{0}^{t} \mathrm{e}^{\lambda(t-s)}T(s)(\lambda-A)f_{n} \, \ds\right\| + |\mathrm{e}^{\lambda t} -1|
\end{align*}
%% --
which converges to zero uniformly in $n$ as $t \to 0$, \ie $(f_{n}) \in m^{\TT}(E)$.
From the characterization of $D(A_{\F})$ given in A-I,3.6 it follows that
%% --
\[
\hat{f} \coloneqq (f_{n}) + c_{F}(E) \in D(A_{\F})
\]
%% --
and $A_{\F}\hat{f} = \lambda\hat{f}$, \ie $\lambda \in P\sigma(A_{\F})$\,.

\item 
The inclusion $A\sigma(A) \subset \sigma(A_{\F})$ follows from (i). Now we show $R\sigma(A) \subset R\sigma(A_{\F})$:
For $\lambda \in R\sigma(A)$ choose $f \in E$ such that $\|(\lambda-A)g - f\| \geq 1$ for every $g \in D(A)$.
Then $\|(\lambda-A_{\F})g - \hat{f}\| \geq 1$ for every $\hat{g} \in D(A_{\F})$ and $\hat{f} = (f,f,\ldots) + c_{F}(E)$.
Therefore $\lambda \in R\sigma(A_{\F})$.

We now show $\rho(A) \subset \rho(A_{\F})$: Assume $\lambda \in \rho(A)$.
By (i) $(\lambda-A_{\F})$ has to be injective.
Choose $\hat{f} = (f_{1},f_{2},\ldots) + c_{\F}(E)$ such that $(f_{n}) \in m^{\TT}(E)$.
Then $(R(\lambda,A)f_{n}) \in m^{\TT}(E)$ and $(\lambda-A_{\F})((R(\lambda,A)f_{n})+c_{\F}(E)) = (f_{n}) + c_{\F}(E)$, \ie $(\lambda-A_{\F})$ is surjective and $\lambda \in \rho(A_{\F})$.
\end{enumerate}
\end{proof}
%% --
Applying this proposition to a single operator $T(t)$, we obtain
%% --
\[
A\sigma(T(t)) = P\sigma(T(t)_{\F}) .
\]
%% --
Note that in general $A\sigma(T(t)) \neq P\sigma(T_{\F}(t))$ (see the Examples~\ref{ex:a3-1.3} and \ref{ex:a3-1.4} in combination with Theorem~\ref{thm:a3-6.3}).
%\end{example}

%% --
\section{The Spectrum of Periodic Semigroups}\label{sec:a3-5}
\index{Spectrum!Periodic Semigroups}
%% --
In this section we determine the spectrum of a particularly simple class of strongly continuous semigroups and thereby achieve a rather complete description of the semigroup itself.
Besides being nice and simple these semigroups gain their importance as building blocks for the general theory.
%% --
\begin{definition}\label{def:a3-5.1}
A strongly continuous semigroup $\TT = (T(t))_{t \geq 0}$ on a Banach space $E$ is called \emph{periodic} if $T(t_{0}) = \Id$ for some $t_{0} > 0$.

The \emph{period} $\tau$ of $\TT$ is obtained as 
%
\[
	\tau \coloneqq \inf\{t_{0} > 0 \colon T(t_{0}) = \Id\}\,.
\]
%
\end{definition}
%% --
We immediately observe that periodic semigroups are groups with inverses $T(t)^{-1} = T(n\tau-t)$ for $0 \leq t \leq n\tau$, $\tau$ the period of $\TT$.
Moreover, they are bounded, hence the growth bound is zero and $\sigma(A) \subset \im\R$.
%% --
\begin{lemma}\label{lem:a3-5.2}
Let $T$ be a strongly continuous semigroup with period $\tau > 0$ and generator $A$.
Then
%% --
\[
\sigma(A) \subset 2\pi \im/\tau\cdot\Z
\]
%% --
and
%% --
\begin{equation}\label{eq:a3-5.1}
R(\mu,A) = (1-\mathrm{e}^{-\mu\tau})^{-1} \int_{0}^{\tau}\mathrm{e}^{-\mu s}T(s) \, \ds
\end{equation}
%% --
for $\mu \notin 2\pi\im /\tau\cdot\Z$.
\end{lemma}
%% --
\begin{proof}
From the basic identities A-I,(3.1) and A-I,(3.2) for $t = \tau$, it follows that $(\mu - A)$ has a left and right inverse if $\mu \neq 2\pi\im n/\tau$, $n \in \Z$, and that the inverse is given by the above expression.
\end{proof}
%% --
The representation of $R(\mu,A)$ given in A-I, Proposition~1.11 shows that the resolvent of the generator of a periodic semigroup is a meromorphic function having only poles of order one and the residues
%% --
\[
P_{n} \coloneqq \lim_{\mu \to \mu_{n}} (\mu-\mu_{n})R(\mu,A) \quad \text{in} \quad \mu_{n} \coloneqq 2\pi\im n/\tau, \quad n \in \Z\,,
\]
%% --
are
%% --
\begin{equation}\label{eq:a3-5.2}
P_{n} = \frac{1}{\tau}\int_{0}^{\tau}\exp(-\mu_{n}s)T(s) \, \ds\,.
\end{equation}
%% --
Moreover, it follows that the spectrum of $A$ consists of eigenvalues only and each $P_{n}$ is the spectral projection belonging to $\mu_{n}$ (see 3.6). 
Another way of looking at $P_{n}$ is given by saying that $P_{n}$ is the n-th Fourier coefficient of the $\tau$-periodic function $s \mapsto T(s)$.
From this it follows that no non-zero $\phi \in E'$ vanishes on all $P_{n}E$ simultaneously.
By the Hahn-Banach theorem we conclude that $\text{span } \cup_{n \in \Z} P_{n}E$ is dense in $E$.

Since $P_{n}E \subset D(A)$, we obtain from A-I,(3.1) that
%% --
\begin{equation}\label{eq:a3-5.3}
AP_{n}f = \mu_{n}P_{n}f
\end{equation}
%% --
for every $f \in E$, $n \in \Z$.
This and A-I,(3.2) imply
%% --
\begin{equation}\label{eq:a3-5.4}
T(t)P_{n}f = \exp(\mu_{n}t) \cdot P_{n}f
\end{equation}
%% --
for every $t \geq 0$.
Therefore $\mu_{n}$ is an eigenvalue of $A$ and $\exp(\mu_{n}t)$ is an eigenvalue of $T(t)$ if and only if $P_{n} \neq 0$.
In that case, $P_{n}E$ is the corresponding eigenspace and we have the following lemma.
%% --
\begin{lemma}\label{lem:a3-5.3}
For a $\tau$-periodic semigroup $\TT$ we take $\mu_{n} \coloneqq 2\pi\im n/\tau$, $n \in \Z$, and consider
%% --
\[
P_{n} \coloneqq \frac{1}{\tau}\int_{0}^{\tau} \exp(-\mu_{n}s)T(s) \, \ds\,.
\]
%% --
Then the following assertions are equivalent.
%% --
\begin{enumerate}[(a)]
\item 
$P_{n} \neq 0$\,,

\item 
$\mu_{n} \in P\sigma(A)$\,,

\item 
$\exp(\mu_{n}t) \in P\sigma(T(t))$ for every $t > 0$\,.

\end{enumerate}
\end{lemma}
%% --
The action of $A$, \resp $T(t)$ in the subspaces $P_{n}E$, $n \in \Z$, is determined by \eqref{eq:a3-5.3} and \eqref{eq:a3-5.4} \resp.
Moreover,
%% --
%\begin{align*}
%P_{m}P_{n}f &=  \frac{1}{\tau}\int_{0}^{\tau} \exp(-\mu_{m}s)T(s)P_{n}f \, \ds = \\
%&=  \frac{1}{\tau}\int_{0}^{\tau} \exp((\mu_{n}-\mu_{m})s)P_{n}f \, \ds = 0
%\end{align*}
%% --
\[\textstyle
P_{m}P_{n}f =  \frac{1}{\tau}\int_{0}^{\tau} \exp(-\mu_{m}s)T(s)P_{n}f \, \ds 
=  \frac{1}{\tau}\int_{0}^{\tau} \exp((\mu_{n}-\mu_{m})s)P_{n}f \, \ds = 0
\]
%% --
for $n \neq m$, \ie the subspaces $P_{n}E$ are \enquote{orthogonal}.
Since their union is total in $E$, one expects to be able to extend the representations \eqref{eq:a3-5.3} and \eqref{eq:a3-5.4} of $A$ and $T(t)$.
This is possible if
%% --
\[
\sum_{-\infty}^{+\infty} P_{n} = \Id\,,
\]
%% --
where the series should be summable for the strong operator topology.

Unfortunately this is false in general since the family of projections
%% --
\[
Q_{H} \coloneqq \sum_{n \in H} P_{n}\,,
\]
%% --
where $H$ runs through all finite subsets of $\Z$, may be unbounded (see the example below).
Nevertheless the following is true.
%% --
\begin{theorem}\label{thm:a3-5.4}
Let $\TT = (T(t))_{t \geq 0}$ be a $\tau$-periodic semigroup on a Banach space $E$ with generator $A$ and associated spectral projections
%% --
\[
P_{n} \coloneqq  \frac{1}{\tau}\int_{0}^{\tau} \exp(-\mu_{n}s)T(s) \, \ds, \quad \mu_{n} \coloneqq 2\pi\im n/\tau, \quad n \in \Z \,.
\]
%% --
For every $f \in D(A)$ one has $f = \sum_{-\infty}^{+\infty} P_{n}f$ and therefore
\begin{enumerate}[(i)]
\item 
$T(t)f = \sum_{-\infty}^{+\infty} \exp(\mu_{n}t)P_{n}f$ \quad if $f \in D(A)$\,,

\item 
$Af = \sum_{-\infty}^{+\infty} \mu_{n}P_{n}f$ \quad if $f \in D(A^{2})$\,.
\end{enumerate}
\end{theorem}
%% --
\begin{proof}
It suffices to prove the first statement. Then (i) and (ii) follow by \eqref{eq:a3-5.3} and \eqref{eq:a3-5.4}.

We assume $\tau = 2\pi$ and show first that $\sum_{-\infty}^{+\infty} P_{n}f$ is summable for $f \in D(A)$: For $g \coloneqq Af$ we obtain $P_{n}g = P_{n}Af = AP_{n}f = inP_{n}f$.
Take $H$ to be a finite subset of $\Z \setminus \{0\}$ and $\phi \in E'$. Then
%% --
%\begin{align*}
%\left|\sum_{n \in H} \langle P_{n}f,\phi \rangle\right|
%&= \left|\sum_{n \in H} (in)^{-1} \langle P_{n}g,\phi \rangle\right| \\
%&\leq \left(\sum_{n \in H} n^{-2}\right)^{1/2}\left(\sum_{n \in H} |\langle P_{n}g,\phi \rangle|^{2}\right)^{1/2}\,.
%\end{align*}
%% --
\[\textstyle
\left|\sum_{n \in H} \langle P_{n}f,\phi \rangle\right|
= \left|\sum_{n \in H} \frac{1}{in} \langle P_{n}g,\phi \rangle\right| 
\leq \left(\sum_{n \in H} \frac{1}{n^2}\right)^{1/2}\left(\sum_{n \in H} |\langle P_{n}g,\phi \rangle|^{2}\right)^{1/2}
\]
%% --
From Bessel's inequality we obtain for the second factor
%% --
%\begin{align*}
%\sum_{n \in H} |\langle P_{n}g,\phi \rangle|^{2} &\leq \frac{1}{2\pi} \cdot \int_{0}^{2\pi} |\langle T(s)g,\phi \rangle|^{2} \, \ds \\
%&\leq \|\phi\|^{2} \cdot \frac{1}{2\pi} \cdot \int_{0}^{2\pi} \|T(s)g\|^{2} \, \ds \,.
%\end{align*}
%% --
\[\textstyle
\sum_{n \in H} |\langle P_{n}g,\phi \rangle|^{2} \leq \frac{1}{2\pi} \cdot \int_{0}^{2\pi} |\langle T(s)g,\phi \rangle|^{2} \, \ds 
\leq \|\phi\|^{2} \cdot \frac{1}{2\pi} \cdot \int_{0}^{2\pi} \|T(s)g\|^{2} \, \ds \,.
\]
%% --
With the constant $c \coloneqq \left(\frac{1}{2\pi} \cdot \int_{0}^{2\pi} \|T(s)g\|^{2} \, \ds\right)^{1/2}$ we obtain
%% --
\[\textstyle
\left\|\sum_{n \in H} P_{n}f\right\| \leq c\left(\sum_{n \in H} n^{-2}\right)^{1/2}
\]
%% --
for every finite subset $H$ of $\Z$, \ie $\sum_{-\infty}^{+\infty} P_{n}f$ is summable.

Next we set $h \coloneqq \sum_{-\infty}^{+\infty} P_{n}f$ and observe that for every $\phi' \in E'$ the Fourier coefficients of the continuous, $\tau$-periodic functions
$s \mapsto \langle T(s)h,\phi \rangle$ and $s \mapsto \langle T(s)f,\phi \rangle$
coincide.
Therefore these functions are identical for $s \geq 0$ and in particular for $s = 0$, \ie $\langle h,\phi \rangle = \langle f,\phi \rangle$.
By the Hahn-Banach Theorem we obtain $f = h$.
\end{proof}

The above theorem contains rather precise information on periodic semigroups.
In particular, it characterizes periodic semigroups by the fact that $\sigma(A)$ is contained in $\im\alpha\Z$ for some $\alpha \in \R$ and the eigenfunctions of $A$ form a total subset of $E$.

%If we suppose in addition that a periodic semigroup has a bounded generator it follows that the spectrum of its generator is bounded.

For a periodic semigroup with bounded generator
only a finite number of spectral projections $P_{n}$ are distinct from $0$ and we have the following characterization.
%% --
\begin{corollary}\label{cor:a3-5.5}
Let $\TT = (T(t))_{t \geq 0}$ be a semigroup with bounded generator on some Banach space $E$.

This semigroup has period $\tau/k$ for some $k \in \N$ if and only if there exist finitely many pairwise orthogonal projections $P_{n}$, $-m \leq n \leq m$, $P_{-m} \neq 0$ or $P_{m} \neq 0$, such that
%% --
\begin{enumerate}[(i)]
\item 
$\sum_{-m}^{+m} P_{n} =  \Id$\,,

\item 
$T(t) = \sum_{-m}^{+m} \exp(2\pi\im nt/\tau)P_{n}$\,,

\item 
$A = \sum_{-m}^{+m} (2\pi\im n/\tau)P_{n}$\,.

\end{enumerate}
\end{corollary}
%% --
\begin{example}\label{ex:a3-5.6}
\index{Example!Rotation Group}
%% --
From A-I,2.5 we recall briefly the rotation group
%% --
\[
R_{\tau}(t)f(z) \coloneqq f(\exp(2\pi\im nt/\tau) \cdot z)
\]
%% --
on $E = C(\Gamma)$, \resp $E = L^{p}(\Gamma,m)$ for $1 \leq p < \infty$.
The spectrum of the generator\quad
$Af(z) = (2\pi\im /\tau)z \cdot f'(z)$\quad
%% --
is \quad $\sigma(A) = (2\pi\im /\tau)\cdot\Z$.
The eigenfunctions $\epsilon_{n}(z) \coloneqq z^{n}$ yield the projections
%% --
\[
P_{n} = (1/2\pi\im )\cdot\epsilon_{-(n+1)} \otimes \epsilon_{n}, \text{ \ie }
\]
%% --
\[
P_{n}f(z) = (1/2\pi\im )\cdot(\int_{\Gamma} f(w)w^{-(n+1)} \, dw)\cdot z^{n}\,.
\]
%% --
It is left as an exercise to compute the norms of $Q_{m} \coloneqq \sum_{-m}^{+m} P_{n}$ in $L^{p}(\Gamma,m)$ for various $p$ and then check the assertions of Theorem~\ref{thm:a3-5.4}.

Clearly, this proves some classical convergence theorems for Fourier series (compare \citet[Chap.8.1]{davies:1980}).
\end{example}


%% --
\section{Spectral Mapping Theorems}\label{sec:a3-6}
\index{Spectral Mapping Theorems}
%% --
We now return to the question posed in the introduction to this chapter: In which form and under which conditions is it true that the spectrum $\sigma(T(t))$ of the semigroup operators is obtained---via the exponential map---from the spectrum $\sigma(A)$ of the generator, or briefly
%% --
\[
\textit{Do we have } \sigma(T(t)) = \exp(t\sigma(A)) \textit{ or at least }
 \sigma(T(t)) = \overline{\exp(t\sigma(A))} ~~?
\]
%% --
This and similar statements will be called \emph{spectral mapping theorems} for the semigroup $\TT = (T(t))_{t \geq 0}$ and its generator $A$.
In addition, we saw in Proposition~\ref{prop:a3-1.1} that the validity of such a spectral mapping theorem implies
%% --
\[
s(A) = \omega_{0}(A)
\]
%% --
for the spectral- and growth bounds and therefore guarantees that the location of the spectrum of $A$ determines the asymptotic behavior of $\TT$.
As we have seen in Examples~\ref{ex:a3-1.3} and \ref{ex:a3-1.4} the last statement does not hold in general.
We therefore present a detailed analysis, where and why it fails and what additional assumptions are needed for its validity.
Before doing so, we have another look at the examples.
%% --
%\subsection{The counterexamples revisited}\label{subsec:a3-6.1}
%\index{Counterexamples!Spectral Mapping Theorem}
\begin{example}{(The counterexamples revisited)}
	\label{ex:a3-6.1}
%% --
\begin{enumerate}[(i), wide]
\item
Take the nilpotent translation semigroup from A-I,2.6. Then $\sigma(A) = \emptyset$ and $\sigma(T(t)) = 0$ for every $t > 0$.
By this trivial example and since $\mathrm{e}^{z} \neq 0$ for every $z \in \C$, it is natural to read the spectral mapping theorem modulo the addition of $\{0\}$, \ie 
%% --
\[
\sigma(T(t)) \setminus \{0\} = \exp(t\sigma(A))  \text{ for } t \geq 0\,.
\]
%% --
\item
The spectrum of the generator $A$ of the $\tau$-periodic rotation group $(R_{\tau}(t))_{t \geq 0}$ on $C(\Gamma)$ is $\sigma(A) = 2\pi\im /\tau\cdot\Z$ and $\exp(2\pi\im nt/\tau)$, $n \in \Z$, is an eigenvalue of $R_{\tau}(t)$ for every $t \geq 0$ (see Example~\ref{ex:a3-5.6}.
If $t/\tau$ is irrational, these eigenvalues form a dense subset of $\Gamma$.
Since the spectrum is closed, we obtain $\sigma(T(t)) = \Gamma$ for these $t$.
Therefore in this example the spectral mapping theorem is valid only in the following \enquote{weak} form
%% --
\[
\sigma(T(t)) = \overline{\exp(t\sigma(A))}, \quad t \geq 0\,.
\]
%% --

\item 
By Example~\ref{ex:a3-1.3} there exists a semigroup $\TT = (T(t))_{t \geq 0}$ with generator $A$ such that $s(A) = -1$ and $\omega_{0}(\TT) = 0$.
This implies that for preassigned real numbers $\alpha < \beta$ there exists a semigroup $\mathcal{S} = (S(t))_{t \geq 0}$ with generator $B$ such that $s(B) = \alpha$ and $\omega_{0}(\mathcal{S}) = \beta$. Indeed, take $S(t) = \mathrm{e}^{\beta t}T((\beta - \alpha)t)$ and observe that $B = (\beta-\alpha)A + \beta\Id$. 
In that case $\exp(t\sigma(B))$ is contained in the circle about $0$ with radius $\mathrm{e}^{\alpha t}$ 
while $\sigma(S(t))$ has spectral values satisfying $|\lambda| = r(S(t)) = \mathrm{e}^{\beta t} > \mathrm{e}^{\alpha t}$\,.


\item 
The Example~\ref{ex:a3-1.3} can be strengthend in order to yield a semigroup $\TT = (T(t))_{t \geq 0}$ with generator $A$ such that $\sigma(A) = \emptyset$, but $\|T(t)\| = r(T(t)) = 1$ for $t \geq 0$, \ie $s(A) = -\infty$, $\omega_{0} = 0$ and $s(A) < \omega_{0}$ 
%\textbf{???RAINER???} 
take the translation semigroup on the Banach space
%% --
\[
E \coloneqq C_{0}(\R_{+}) \cap L^{1}(\R_{+}, \mathrm{e}^{x^{2}}\dx)
\]
%% --
with $\|f\| \coloneqq \sup \{|f(x)| \colon x \in \R_{+}\} + \int_{0}^{\infty} |f(x)|\mathrm{e}^{x^{2}} \, \dx$ (see \citet{greinervoigtwolff:1981}).

\item 
Another modification of Example~\ref{ex:a3-1.3} yields a group $\TT = (T(t))_{t \in \R}$ satisfying $s(A) < \omega_{0}$.
Therefore the spectral mapping theorem does not hold in the setting of groups (see Wolff (1981)).
\end{enumerate}
\end{example}
%% --
The next few theorems form the core of this chapter. 
We show that only one part of the spectrum and one inclusion is responsible for the failure of the spectral mapping theorem.
The usefulness of this detailed analysis will become clear in the subsequent chapters on stability and asymptotics.
%% --
%\subsection{Spectral Inclusion Theorem}\label{subsec:a3-6.2}
\begin{proposition}{$\mathrm{(Spectral\ Inclusion)}$}
	\label{prop:a3-6.2}
	\index{Spectral Inclusion}
	
Let $A$ be the generator of a strongly continuous semigroup $\TT = (T(t))_{t \geq 0}$ on some Banach space $E$.
Then
%% --
\[
\exp(t\sigma(A)) \subset \sigma(T(t)) \text{ for } t \geq 0\,.
\]
%% --
More precisely we have the following inclusions.
%% --
\begin{align}
\exp(t \cdot P\sigma(A)) &\subset P\sigma(T(t)) \label{eq:a3-6.1}\,, \\
\exp(t \cdot A\sigma(A)) &\subset A\sigma(T(t)) \label{eq:a3-6.2}\,, \\
\exp(t \cdot R\sigma(A)) &\subset R\sigma(T(t)) \label{eq:a3-6.3}\,.
\end{align}
\end{proposition}
%% --
\begin{proof}
Since $\mathrm{e}^{\lambda t} - T(t) = (\lambda - A)\int_{0}^{t} \mathrm{e}^{\lambda(t-s)}T(s) \, \ds$ (see A-I,(3.1)), it follows that $(\mathrm{e}^{\lambda t} - T(t))$ is not bijective if $(\lambda - A)$ fails to be bijective proving the main inclusion.

The inclusion \eqref{eq:a3-6.1} becomes evident from the following proof of \eqref{eq:a3-6.2}. Take $\lambda \in A\sigma(A)$ and an associated approximate eigenvector $(f_{n}) \subset D(A)$.
Then
%% --
\[
g_{n} \coloneqq \mathrm{e}^{\lambda t}f_{n} - T(t)f_{n} = \int_{0}^{t} \mathrm{e}^{\lambda(t-s)}T(s)(\lambda-A)f_{n} \, \ds
\]
%% --
converges to zero as $n \to \infty$.
Consequently, $\mathrm{e}^{\lambda t} \in A\sigma(T(t))$ and, in fact, the same approximate eigenvector $(f_{n})$ does the job for all $t \geq 0$.

For the proof of \eqref{eq:a3-6.3} we take $\lambda \in R\sigma(A)$ and observe that $(\mathrm{e}^{\lambda t} - T(t))f = (\lambda - A)(\int_{0}^{t} \mathrm{e}^{\lambda(t-s)}T(s)f \, \ds) \in (\lambda - A)D(A)$ for every $f \in E$.
\end{proof}
%% --
As we know from the Examples~\ref{ex:a3-6.1}, the converse inclusions do not hold in general, \ie not every spectral value of a semigroup operator $T(t)$ comes - via the exponential map - from a spectral value of the generator.
But at least this is true for some important parts of the spectrum.
%% --
\begin{theorem}[Spectral Mapping Theorem for Point and Residual Spectrum]\label{thm:a3-6.3}
	
Let $A$ be the generator of a strongly continuous semigroup $\TT = (T(t))_{t \geq 0}$.
Then
%% --
\begin{equation}\label{eq:a3-6.4}
\exp(t \cdot P\sigma(A)) = P\sigma(T(t)) \setminus \{0\}\,,
\end{equation}
%% --
\begin{equation}\label{eq:a3-6.5}
\exp(t \cdot R\sigma(A)) = R\sigma(T(t)) \setminus \{0\} \text{ for } t \geq 0\,.
\end{equation}
\end{theorem}
%% --
\begin{proof}
For the proof of \eqref{eq:a3-6.4},  take $t_{0} > 0$ and $0 \neq \lambda \in P\sigma(T(t_{0}))$.

After rescaling the semigroup $\TT = (T(t))_{t \geq 0}$ to the semigroup
%% --
\[
(\exp(-t \cdot \log\lambda/t_{0})T(t))_{t \geq 0}\,,
\]
%% --
we may assume $\lambda = 1$. 
Then the closed, $\TT$-invariant subspace
%% --
\[
G \coloneqq \{g \in E \colon T(t_{0})g = g\}
\]
%% --
is non trivial.
The restricted semigroup $T_{|}$ is periodic with period $\tau \leq t_{0}$ and the spectrum of its generator $A_{|}$ contains at least one eigenvalue $\mu = 2\pi\im n/t_{0}$ for some $n \in \Z$ (see Lemma~\ref{lem:a3-5.3}).
Since every eigenvalue of $A_{|}$ is an eigenvalue of $A$, we obtain that $1 \in \exp(t_{0} \cdot P\sigma(A))$.
The converse inclusion has been proved in \eqref{eq:a3-6.1}.

In fact, even more can be said: Let $g \in G$ be an eigenvector of $T(t_{0})$ corresponding to the eigenvalue $\lambda = 1$.
For each $n \in \Z$ define
%% --
\[
g_{n} \coloneqq P_{n}g = 1/t_{0} \cdot \int_{0}^{t_{0}} \exp(-2\pi\im ns/t_{0})T(s)g \, \ds \in G
\]
%% --
as in Section 5.
If $g_{n} \ne 0$, then $G_n$ is an eigenvector of $A_{|}$, hence of $A$ with eigenvalue $2\pi\im n/t_{0}$ as soon as $g_{n}$ is distinct from zero.
Since $D(A_{|})$ is dense in $G$ it follows from Theorem~\ref{thm:a3-5.4} that this holds for at least one $n \in \Z$.
And from the proof of \eqref{eq:a3-6.1} we know that this $g_{n}$ is in fact an eigenvector for each $T(t)$, $t \geq 0$.

Since $R\sigma(A) = P\sigma(A^*)$ and $R\sigma(T(t)) = P\sigma(T(t)^*)$ (see Proposition~\ref{prop:a3-4.4}) the assertion \eqref{eq:a3-6.5}  follows from \eqref{eq:a3-6.4} .
\end{proof}
%% --
Note that the proof is essentially an application of the structure theorem for periodic semigroups as given in Theorem~\ref{thm:a3-5.4}.
The information gained there can be reformulated into statements on the eigenspaces of $A$ and $T(t)$.
%% --
\begin{corollary}\label{cor:a3-6.4}
For the eigenspaces of the generator $A$, \resp of the semigroup operators $T(t)$, $t > 0$, the following holds  for  $\mu \in \C$\,.
%% --
\begin{enumerate}[(i)]
\item 
$\ker(\mu - A) = \bigcap_{s \geq 0} \ker(\mathrm{e}^{\mu s} - T(s))$\,,

\item 
$\ker(\mathrm{e}^{\mu t} - T(t)) = \overline{\mathrm{span}_{n \in \Z} \{\ker(\mu + 2\pi\im  n/t - A)\}}$\,.

\end{enumerate}
\end{corollary}
%% --
We note that an analogous statements is valid for $\ker(\mu - A')$ and $\ker(\mathrm{e}^{\mu t} - T(t)')$ if we take in (ii) the $\sigma(E',E)$-closure.

Without proof (see \citet[Proposition~1.10]{greiner:1981}) we add another corollary showing that poles of the resolvent of $T(t)$ correspond necessarily to poles of the resolvent of the generator.
Again the converse is not true as shown by Example~\ref{ex:a3-5.6}.
%% --
\begin{corollary}\label{cor:a3-6.5}
Assume that $\mathrm{e}^{\mu t}$ is a pole of order $k$ of $R(\cdot,T(t))$ with residue $P$ and $Q$ as the $k$-th coefficient of the Laurent series.
Then
%% --
\begin{enumerate}[(i)]
\item 
$\mu + 2\pi\im  n/t$ is a pole of $R(\cdot,A)$ of order $\leq k$ for every $n \in \Z$\,,

\item 
the residues $P_{n}$ in $\mu + 2\pi\im  n/t$ yield $PE = \overline{\text{span}_{n \in \Z} \{P_{n}E\}}$\,,

\item 
the $k$-th coefficient of the Laurent series of $R(\cdot,A)$ at $\mu + 2\pi\im  n/t$ is
%% --
\[
Q_{n} = (t \cdot \mathrm{e}^{\mu t})^{1-k} \cdot Q \circ (1/t) \int_{0}^{t} \mathrm{e}^{-(\mu+2\pi\im  n/t)s}T(s) \ds\,.
\]
%% --
\end{enumerate}
\end{corollary}
%% --
From Proposition~\ref{prop:a3-6.2} and Theorem~\ref{thm:a3-6.3} it follows that the approximate point spectrum is the trouble maker in the sense that not every approximate eigenvalue of $T(t)$ corresponds to an approximate eigenvalue of the generator $A$.
Since nothing more can be said in general, we now look for additional hypotheses on the semigroup implying the spectral mapping theorem.

As a simple example we assume $T(t_{0})$ to be compact for some $t_{0} > 0$.
Then $\sigma(T(t)) \setminus \{0\} = P\sigma(T(t)) \setminus \{0\}$ for $t \geq t_{0}$ and the spectral mapping theorem is valid by \eqref{eq:a3-6.4}.
A different class of semigroups verifying the spectral mapping theorem is given by the uniformly continuous semigroups (compare Corollary~\ref{cor:a3-1.2}).

Both cases, and many more, are included in the following result.
%% --
\begin{theorem}[Spectral Mapping Theorem for Eventually Continuous Semigroups]\label{thm:a3-6.6}
Let $\TT = (T(t))_{t\geq 0}$ be an eventually norm continuous semigroup with generator $A$.
Then the spectral mapping theorem is valid, \ie 
%% --
\begin{equation}\label{eq:a3-6.6)}
	\sigma(T(t)) \setminus \{0\} = \mathrm{e}^{t \cdot \sigma(A)} \text{ for every } t \geq 0\,.
\end{equation}
%% --
\end{theorem}
%% --
\begin{proof}
By the previous considerations it suffices to show that $A\sigma(T(t)) \setminus \{0\} \subset \mathrm{e}^{t \cdot \sigma(A)}$ for $t > 0$.
This will be done by converting approximate eigenvectors into eigenvectors in the semigroup $\F$-product (see subsection~\ref{subsec:a3-4.5}).
The assertion then follows from \eqref{eq:a3-6.4} and Proposition~\ref{prop:a3-4.4}.(ii).

Assume $t \mapsto T(t)$ to be norm continuous for $t \geq t_{0}$.
Moreover it suffices to consider $1 \in A\sigma(T(t_{1}))$ for some $t_{1} > 0$, \ie we have a normalized sequence $(f_{n})_{n\in\N} \subset E$ such that
%% --
\[
\lim_{n\to\infty} \|T(t_{1})f_{n} - f_{n}\| = 0\,.
\]
%% --
Choose $k \in \N$ such that $kt_{1} > t_{0}$ and define $g_{n} \coloneqq T(kt_{1})f_{n}$.
Then
%% --
\[
\lim_{n\to\infty}\|g_{n}\| = \lim_{n\to\infty}\|T(t_{1})^{k}f_{n}\| = \lim_{n\to\infty}\|f_{n}\| = 1
\]
%% --
and
%% --
\[
\lim_{n\to\infty} \|T(t_{1})g_{n} - g_{n}\| = 0\,,
\]
%% --
\ie $(g_{n})_{n\in\N}$ yields an approximate eigenvector of $T(t_{1})$ with approximate eigenvalue $1$.
But the semigroup $\TT$ is uniformly continuous on sets of the form $T(t_{0})V$, $V$ bounded in $E$.
In particular, it is uniformly continuous on the sequence $(g_{n})_{n\in\N}$, which therefore defines an element $g$ in the semigroup $\F$-product $E_{\F}$.

Obviously, $g$ is an eigenvector of $T_{\F}(t_{1})$ with eigenvalue $1$ and by \eqref{eq:a3-6.4} we obtain an eigenvalue $2\pi\im  n/t_{1}$ of $A_{\F}$ for some $n \in \Z$.
The coincidence of $\sigma(A)$ and $\sigma(A_{\F})$ proves the assertion.
\end{proof}
%% --
We point out that the above spectral mapping theorem implies the coincidence of spectral bound and growth bound for eventually norm continuous semigroups, hence we have generalized the Liapunov Stability Theorem (see \ref{cor:a3-1.2}) to a much larger class of semigroups.
As mentioned before, this will be of great use in many applications.
Therefore we state explicitly the spectral mapping theorem for several important classes of semigroups all of which are eventually norm continuous (cf. the diagram preceding A-II, Example~1.27).
%% --
\begin{corollary}\label{cor:a3-6.7}
The spectral mapping theorem \ref{thm:a3-6.6}
%%% --
%\[
%\sigma(T(t)) \setminus \{0\} = \mathrm{e}^{t \cdot \sigma(A)}, \quad t \geq 0\,,
%\]
%%% --
holds for each of the following classes of strongly continuous semigroups.
%% --
\begin{enumerate}[(i)]
\item 
eventually compact semigroups,

\item 
eventually differentiable semigroups,

\item 
holomorphic semigroups,

\item 
uniformly continuous semigroups.
\end{enumerate}
%% --
\end{corollary}
%% --
Another application of the above spectral mapping theorem can be made to tensor product semigroups (see A-I,3.7).
Let $\TT_{1} = (T_{1}(t))_{t\geq 0}$, $\TT_{2} = (T_{2}(t))_{t\geq 0}$ be strongly continuous semigroups on Banach spaces $E_{1}$, $E_{2}$ with generator $A_{1}$, $A_{2}$.
The tensor product semigroup $\TT = \TT_{1} \otimes \TT_{2}$ on some (appropriate) tensor product $E \coloneqq E_{1} \otimes E_{2}$ has the generator $A = A_{1}\otimes\Id + \Id\otimes A_{2}$, but in general the spectrum of $A$ is not determined by the spectra of $A_{1}$, $A_{2}$.
But with an additional hypothesis the following can be proved.
%% --
\begin{corollary}\label{cor:a3-6.8}
If $\TT_{1}$ and $\TT_{2}$ are eventually norm continuous, then
%% --
\[
	\sigma(A) = \sigma(A_{1}) + \sigma(A_{2})\,,
\]
%% --
where $A$ is the generator of the tensor product semigroup
%% --
\[
	\TT_{1} \otimes \TT_{2} = (T_{1}(t) \otimes T_{2}(t))_{t\geq 0}\,.
\]
%% --
\end{corollary}
%% --
\begin{proof}
Clearly, the tensor product semigroup is eventually norm continuous and hence the Spectral Mapping Theorem~\ref{thm:a3-6.6} is valid for all three semigroups $\TT_{1}$, $\TT_{2}$ and $\TT$.
Moreover the spectrum of the tensor product of bounded operators is the product of the spectra \citet[XIII.9]{reedsimon:1978}.
Therefore
%% --
\[
	\sigma(T_{1}(t)\otimes T_{2}(t)) = \sigma(T_{1}(t))\cdot\sigma(T_{2}(t)), \quad t \geq 0\,.
\]
%% --
Consequently we have the following identity for every $t \geq 0$ 
%% --
\begin{align*}
\mathrm{e}^{t\cdot\sigma(A)} &= \sigma(T_{1}(t)\otimes T_{2}(t)) \setminus \{0\} \\
&= \sigma(T_{1}(t))\cdot\sigma(T_{2}(t)) \setminus \{0\} \\
&= \mathrm{e}^{t\cdot\sigma(A_{1})}\cdot \mathrm{e}^{t\cdot\sigma(A_{2})} \\
&= \mathrm{e}^{t(\sigma(A_{1})+\sigma(A_{2}))}\,.
\end{align*}
%% --
From this identity we want to deduce $\sigma(A) = \sigma(A_{1}) + \sigma(A_{2})$.

\enquote{$\subset$}\quad 
Take $\xi \in \sigma(A)$.
Then for every $t > 0$ there exist $\mu_{t} \in \sigma(A_{1})$, $\lambda_{t} \in \sigma(A_{2})$ and $n_{t} \in \Z$ such that $\xi = \mu_{t} + \lambda_{t} + 2\pi\im  n_{t}/t$.

Since the real parts of $\mu_{t}$, $\lambda_{t}$ are bounded above, they lie in some interval $[a,b]$.
But $\sigma(A_{i}) \cap ([a,b] + \im\R)$ is compact for $i = 1$, $2$  since $A_{i}$ is the generator of an eventually norm continuous semigroup (see A-II, Theorem~1.20).
By taking $t$ sufficiently small, we conclude that $n_{t'} = 0$ for some $t' > 0$, \ie $\xi = \mu_{t'} + \lambda_{t'}$.

\enquote{$\supset$}\quad 
Choose $\mu \in \sigma(A_{1})$, $\lambda \in \sigma(A_{2})$.
For every $t > 0$ there exist $\eta_{t} \in \sigma(A)$, $m_{t} \in \Z$ such that $\mu + \lambda = \eta_{t} + 2\pi \im m_{t}/t$.
Since $\Re\,\mu + \Re\,\lambda = \Re\,\eta_{t}$ and $\{\Im \eta_{t}: t > 0\}$ is bounded - $\TT = (T_{1}(t) \otimes T_{2}(t))_{t\geq 0}$ is eventually norm continuous - it follows that $m_{t'} = 0$ for some $t' > 0$.
\end{proof}
%% --
\section{Weak Spectral Mapping Theorems}\label{sec:a3-7}
\index{Spectrum!Weak Spectral Mapping Theorems}
%% --
In the previous section we showed under which hypotheses a spectral mapping theorem of the form
%% --
\begin{equation}\label{eq:a3-7.1}
\sigma(T(t)) \setminus \{0\} = \mathrm{e}^{t \cdot \sigma(A)}, \quad t \geq 0\,,
\end{equation}
%% --
is valid for the generator $A$ of a strongly continuous semigroup $(T(t))_{t\geq 0}$.

Among the various examples showing that \eqref{eq:a3-7.1} does not hold in general we recall the following.
Take the Banach space $E = c_{0}$, the multiplication operator $A(x_{n})_{n\in\N} = (\im n x_{n})_{n\in\N}$ with maximal domain and the corresponding semigroup $T(t)(x_{n})_{n \in \N} = 
(\mathrm{e}^{\im nt}x_{n})_{n \in \N}$.
Then $\sigma(A) = \{in \colon n \in \N\}$ and the spectral mapping theorem is valid only in the following weak form
%% --
\begin{equation}\label{eq:a3-7.2}
\sigma(T(t)) = \overline{\exp(t\cdot\sigma(A))}, \quad t \geq 0\,.
\end{equation}
%% --
In this section we prove similar weak spectral mapping theorems.
We start with a generalization of the above example.

Consider the Banach space $E = C_{0}(X,\C^{n})$ of all continuous $\C^{n}$-valued functions vanishing at infinity on some locally compact space $X$.
In analogy to A-I,2.3 we associate to every continuous function $q \colon X \to M(n)$, where $M(n)$ denotes the space of all complex $n\times n$-matrices, a "multiplication operator" $M_{q} \colon f \to q\cdot f$ such that $(q\cdot f)(x) = q(x)\cdot f(x)$, $x \in X$, on the maximal domain $D(M_{q}) = \{f \in E \colon q\cdot f \in E\}$.
If $\|\mathrm{e}^{tq(x)}\|$ is uniformly bounded for $0 \leq t \leq 1$ and $x \in X$, it follows that $M_{q}$ generates the multiplication semigroup
%% --
\[
(T(t)f)(x) = \mathrm{e}^{tq(x)}\cdot f(x), \quad f \in E, \quad x \in X, \quad t \geq 0\,.
\]
%% --
Since $M_{q}$ has a bounded inverse if and only if $q(x)^{-1}$ exists and is uniformly bounded for $x \in X$, it follows that the eigenvalues of each matrix $q(x)$ are always contained in $\sigma(M_{q})$.
In fact, much more can be said, in case the function is bounded.
%% --
\begin{lemma}\label{lem:a3-7.1}
The spectrum of the matrix valued multiplication operator $M_{q}$, where $q \colon X \to M(n)$, is bounded, is given by $\sigma(M_{q}) = \overline{\bigcup_{x\in X} \sigma(q(x))}$.
\end{lemma}
%% --
\begin{proof}
It remains to show that $0 \notin \overline{\bigcup_{x\in X} \sigma(q(x))}$ implies $0 \notin \sigma(M_{q})$.
Since $\det q(x)$ is the product of $n$ eigenvalues (according to their multiplicity) of $q(x)$, the hypothesis implies that $d \coloneqq \inf\{|\det q(x)| \colon x \in X\} > 0$.
By Formula 4.12 in Chapter I of \citet{kato:1966} we obtain
%% --
\[
\|q(x)^{-1}\| \leq \gamma \cdot \|q(x)\|^{n-1} \cdot |\det q(x)|^{-1} \leq \gamma/d \cdot \|M_{q}\|^{n-1}
\]
%% --
for every $x \in X$ and a constant $\gamma$ depending only on the norm chosen on $\C^{n}$.
Therefore, $x \mapsto q(x)^{-1}$ defines a bounded continuous function on $X$, which obviously yields the inverse of $M_{q}$, \ie $0 \notin \sigma(M_{q})$.
\end{proof}

%% --
\begin{theorem}\label{thm:a3-7.2}
Let $A = M_{q}$ be a matrix multiplication operator on $C_{0}(X,\C^{n})$ generating a strongly continuous semigroup $(T(t))_{t\geq 0}$, $T(t) = M_{\mathrm{e}^{tq}} \textit{ for } t \geq 0$\,.
Then the Weak Spectral Mapping Theorem~\ref{thm:a3-7.2} holds true, \ie 
%% --
\[
\sigma(T(t)) = \overline{\exp(t\cdot\sigma(A))}\,.
\]
%% --

\end{theorem}
%% --
\begin{proof}
By the Spectral Inclusion Proposition~\ref{prop:a3-6.2} we always have $\exp(t\sigma(A)) \subset \sigma(T(t))$.
Since $T(t)$ is a matrix multiplication operator with a bounded function, we obtain from Lemma~\ref{lem:a3-7.1}
%% --
\[
\sigma(T(t)) = \overline{\bigcup_{x\in X} \sigma(\exp(tq(x)))} = \overline{\bigcup_{x\in X} \exp(t\sigma(q(x)))} \subset \overline{\exp(t\sigma(A))}
\]
%% --
which proves the assertion.
\end{proof}
%% --
\begin{corollary}\label{cor:a3-7.3}
The growth bound $\omega_{0}(A)$ and the spectral bound $s(A)$ coincide for matrix multiplication semigroups.
\end{corollary}
%% --
The above results remain valid for other Banach spaces of $\C^{n}$-valued functions such as $L^{p}(X,\C^{n})$, $1 \leq p < \infty$.

The example given at the beginning of this section can be generalized in a different way.
In fact, $A(x_{n}) \coloneqq (\im nx_{n})$ on $E = c_{0}$ generates a bounded group, and we will show that this property too ensures that the Weak Spectral Mapping Theorem~\ref{thm:a3-7.2} holds.
Without any boundedness assumption on $(T(t))_{t\in\R}$ this result cannot be true (see \citet[Sec.23.16]{hillephillips:1957} or \citet{wolff:1981}).
%% --
\begin{theorem}\label{thm:a3-7.4}
Let $\TT = (T(t))_{t\in\R}$ be a strongly continuous group on some Banach space $E$ such that $\|T(t)\| \leq p(t)$ for some polynomial $p$ and all $t \in \R$.
Then the Weak Spectral Mapping Theorem~\ref{thm:a3-7.2} holds, \ie 
%% --
\[
\sigma(T(t)) = \overline{\exp(t\cdot\sigma(A))} \text{ for all } t \in \R\,.
\]
%% --
\end{theorem}
%% --
From the proof we isolate a series of lemmas for which we always assume the hypothesis made in Theorem~\ref{thm:a3-7.4}.
Moreover we recall from Fourier analysis that the Fourier transformation $\phi \mapsto \hat{\phi}$,
%% --
\[
\hat{\phi}(\alpha) \coloneqq \int_{-\infty}^{\infty} \phi(x)\mathrm{e}^{-\im\alpha x} \dx
\]
%% --
and its inverse $\Psi \mapsto \check{\Psi}$,
%% --
\[
\check{\Psi}(x) \coloneqq \frac{1}{2\pi}\int_{-\infty}^{\infty}\Psi(\alpha)\mathrm{e}^{\im\alpha x} 				\diff{\alpha}
\]
%% --
are topological isomorphisms of the Schwartz space $\mathcal{S} ( = \mathcal{S}(\R))$.
Since the subspace $\mathcal{D}$ of all functions having compact support is dense in $\mathcal{S}$, it follows that $\{\phi \in \mathcal{S} \colon \check{\phi} \in \mathcal{D}\}$ is also dense in $\mathcal{S}$.
%% --
\begin{lemma}\label{lem:a3-7.5}
For every function $\phi \in \mathcal{S}$ we obtain an operator $T(\phi) \in \LE$ by
%% --
\[
T(\phi)f \coloneqq \int_{-\infty}^{\infty} \phi(s)T(s)f \ds, \quad f \in E\,.
\]
%% --
This operator can be represented as
\begin{equation}\label{eq:a3-7.3}
T(\phi)f = \lim_{\epsilon\to 0} \frac{1}{2\pi}\int_{-\infty}^{\infty} \check{\phi}(\alpha)[R(\epsilon-\im\alpha,A)f - R(-\epsilon-\im\alpha,A)f] d\alpha, \quad f \in E\,.
\end{equation}
\end{lemma}
%% --
\begin{proof}
That $T(\phi)$ is well-defined follows from the polynomial boundedness of $(T(t))_{t\in\R}$.
In fact, $\phi \to T(\phi)$ is continuous from $\mathcal{S}$ into $(\LE,\|\cdot\|)$.
We obtain
\begin{align*}
T(\phi)f &= \lim_{\epsilon\to 0} \int_{-\infty}^{\infty} \phi(s)\mathrm{e}^{-\epsilon|s|}T(s)f \ds \\
&= \lim_{\epsilon\to 0} \int_{-\infty}^{\infty} \frac{1}{2\pi}\int_{-\infty}^{\infty} \check{\phi}(\alpha)\mathrm{e}^{\im\alpha s}\mathrm{e}^{-\epsilon|s|}T(s)f d\alpha \ds \\
&= \lim_{\epsilon\to 0} \frac{1}{2\pi}\int_{-\infty}^{\infty} \check{\phi}(\alpha)\int_{-\infty}^{\infty} \mathrm{e}^{\im\alpha s}\mathrm{e}^{-\epsilon|s|}T(s)f \ds d\alpha \\
&= \lim_{\epsilon\to 0} \frac{1}{2\pi}\int_{-\infty}^{\infty} \check{\phi}(\alpha)[R(\epsilon-\im\alpha,A)f - R(-\epsilon-\im\alpha,A)f] d\alpha \,.
\end{align*}
Here we used that polynomially bounded groups have growth bound $0$, hence $\omega_{0}(A) = \omega_{0}(-A) = 0$.
Therefore the integral representation of $R(\epsilon-\im\alpha,A)$ (cf. A-I, Proposition~1.11) exists for $\epsilon \neq 0$.
\end{proof}
%% --
\begin{lemma}\label{lem:a3-7.6}
If $E \neq \{0\}$, then $\sigma(A) \neq \emptyset$.
\end{lemma}
%% --
\begin{proof}
If $\sigma(A) = \emptyset$, then \eqref{eq:a3-7.3} implies $T(\phi) = 0$ whenever $\check{\phi}$ has compact support.
Since these functions form a dense subspace of $\mathcal{S}$, we conclude that $T(\phi) = 0$ for all $\phi \in \mathcal{S}$.
Choosing an approximate identity $(\psi_{n})_{n\in\N} \subset \mathcal{D}$, we obtain
%% --
\[
f = T(0)f = \lim_{n\to\infty} T(\psi_{n})f = 0
\]
%% --
for every $f \in E$.
\end{proof}
%% --
\begin{proof}[Proof of Theorem~\ref{thm:a3-7.4} (1st part)]
By the Spectral Inclusion (see Proposition~\ref{prop:a3-6.2}), we have to show that every spectral value of $T(t)$ can be approximated by exponentials of spectral values of $A$.
In view of the rescaling procedure it suffices to prove this when $-1 \in \rho(T(\pi))$, provided that the following condition is satisfied.
%% --
\begin{equation}\label{eq:a3-7.4}
\text{There exists } \epsilon > 0 \text{ such that } \bigcup_{k\in\Z} i[2k+1-2\epsilon,2k+1+2\epsilon] \subset \rho(A)\,.
\end{equation}
%% --
Assume now that \eqref{eq:a3-7.4} holds.
Then each of the sets 
\[
\sigma_{k}\coloneqq \{\im\alpha \in \sigma(A) \colon \alpha \in [2k-1,2k+1]\}
\] 
is a spectral set of $A$ with corresponding spectral projection $P_{k}$.
If we choose $\phi_{0} \in \mathcal{D}$ such that 
\[
\supp(\phi_{0}) \subset [-1+\epsilon,1-\epsilon] \text{ and } \phi_{0}(x) = 1 \text{ for } x \in [-1+2\epsilon,1-2\epsilon]\,,
\]
it follows from \eqref{eq:a3-7.3} and the integral representation of $P_{k}$ (cf.\ \eqref{eq:a3-3.1}) that $P_{0} = T(\check{\phi_{0}})$.
More generally, since $\left(\mathrm{e}^{i2k\cdot}\check{\phi_{0}}\right){\hat{ }}\,(\alpha) = \phi_{0}(\alpha-2k)$, the assertions \eqref{eq:a3-7.3} and \eqref{eq:a3-7.4} imply
%% --
\begin{equation}\label{eq:a3-7.5}
P_{k} = \int_{-\infty}^{\infty} \mathrm{e}^{i2ks}\check{\phi_{0}}(s)T(s)\ds \text{ for } k \in \Z\,.
\end{equation}
%% --
\end{proof}
%% --
At this point we isolate another lemma.
%% --
\begin{lemma}\label{lem:a3-7.7}
$\mathrm{span }\bigcup_{k\in\Z} P_{k}E$ is dense in $E$.
\end{lemma}
%% --
\begin{proof}
The closure of $\text{span }\bigcup_{k\in\Z} P_{k}E$ is a $\TT$-invariant subspace $G$ of $E$.
Consider the quotient group $(T(t)_/)_{t\in\R}$ induced on $E/G$.
The spectrum of its generator $A_{/}$ is contained in $\sigma(A)$ by Proposition~\ref{prop:a3-4.2}.(ii).
Moreover, the spectral projection corresponding to $\sigma(A_{/}) \cap \sigma_{k}$ is the quotient operator $P_{k/}$.
Obviously $P_{k/} = 0$, hence $\sigma(A_{/}) \cap \sigma_{k} = \emptyset$ for every $k \in \Z$ and $\sigma(A_{/}) = \emptyset$.
By Lemma~\ref{lem:a3-7.6} this implies $E/G = \{0\}$, \ie $G = E$.
\end{proof}
%% --
\begin{proof}[Proof of Theorem~\ref{thm:a3-7.4} (2nd part)]
We return to the situation of the first part of the proof.
Using \eqref{eq:a3-7.5} the spectral projection $P_{k}$ can be transformed into
\begin{align*}
P_{k} &= \int_{-\infty}^{\infty} \mathrm{e}^{\im 2ks}\check{\phi_{0}}(s)T(s)\ds \\
&= \sum_{m\in\Z} \int_{(m-1/2)\pi}^{(m+1/2)\pi} \mathrm{e}^{\im 2ks}\check{\phi_{0}}(s)T(s)\ds \\
&= \int_{-\pi/2}^{\pi/2} \mathrm{e}^{\im 2ks}\sum_{m\in\Z}\check{\phi_{0}}(s+m\pi)T(s+m\pi)\ds\,,
\end{align*}
%% --
\ie $P_{k}f$ is the $k$-th Fourier coefficient of the $\pi$-periodic, continuous function $\xi_{f} \colon s \mapsto \sum_{m\in\Z} \check{\phi_{0}}(s+m\pi)T(s+m\pi)f$, $f \in E$.
Since the projections $P_{k}$ are mutually orthogonal, \ie $P_{k}P_{m} = 0$ for $k \neq m$, it follows that $g = \sum_{n\in\Z} P_{n}g$ for every $g \in \text{span }\bigcup_{k\in\Z} P_{k}E$.
In particular, the Fourier coefficients of the function $\xi_{g}$ are absolutely summable, hence the Fourier series of $\xi_{g}$ converges to $\xi$.

For $s = 0$ we obtain
%% --
\[
g = \sum_{n\in\Z} P_{n}g\cdot \mathrm{e}^{-\im n0} = \sum_{m\in\Z} \check{\phi_{0}}(0 + m\pi)T(0 + m\pi)g \quad 
\left(g \in \text{span }\bigcup_{k\in\Z} P_{k}E\right)\,.
\]
%% --
Since $\text{span }\bigcup_{k\in\Z} P_{k}E$ is dense (Lemma~\ref{lem:a3-7.7}), we conclude that
%% --
\begin{equation}\label{eq:a3-7.6}
\sum_{m\in\Z} \phi_{0}(m\pi)T(m\pi) = \Id\,.
\end{equation}
%% --
As the final step we construct the inverse operator of $\Id + T(\pi)$ showing that $-1 \in \rho(T(\pi))$.
We define $\psi_{0}(\alpha) \coloneqq \phi_{0}(\alpha)\cdot(1 + \mathrm{e}^{\im\pi\alpha})^{-1}$, $\alpha \in \R$.
Then we have $\psi_{0} \in \mathcal{S}$ and $\psi_{0}\cdot(1 + \mathrm{e}^{\im\pi\cdot}) = \phi_{0}$,
hence $\check{\psi_{0}}(x) + \check{\psi_{0}}(x + \pi) = \check{\phi_{0}}(x)$ for all $x \in \R$.
Then \eqref{eq:a3-7.6} implies
%% --
\begin{align*}
\Id &= \sum_{m\in\Z} \check{\phi_{0}}(m\pi)T(m\pi) \\
&= \sum_{m\in\Z} (\check{\psi_{0}}(m\pi) + \check{\psi_{0}}((m+1)\pi))T(m\pi) \\
&= [\sum_{m\in\Z} \check{\psi_{0}}(m\pi)T(m\pi)](\Id + T(\pi))\,.
\end{align*}
\end{proof}
%% --
In the rest of this section we discuss the behavior of the single spectral values $\lambda$ of $T(t)$, $t > 0$.
The goal is a characterization of $\sigma(T(t))$ involving only properties of the generator.
By the rescaling procedure A-I,3.1 we may assume $\lambda = 1$ and $t = 2\pi$.

From the Spectral Inclusion (see )Proposition~\ref{prop:a3-6.2}) we know that $1 \in \rho(T(2\pi))$ implies $\im\Z \subset \rho(A)$.
As seen in many examples the converse does not hold and we are now looking for additional conditions.
Henceforth we assume $\im\Z \subset \rho(A)$ and define for $k \in \Z$
%% --
\begin{equation}\label{eq:a3-7.7}
Q_{k}\coloneqq \frac{1}{2\pi}\int_{0}^{2\pi} \mathrm{e}^{-\im ks}T(s)\ds = \frac{1}{2\pi}(1 - T(2\pi))R(\im k,A)
\end{equation}
%% --
(cf. Formula A-I, (3.1)).

Our approach is based on Fejér's Theorem (for Banach space valued functions).
Let us recall this result.
Suppose $\xi \colon [0,2\pi] \to E$ is a continuous function and let $\xi_{k}\coloneqq \frac{1}{2\pi}\int_{0}^{2\pi} \mathrm{e}^{-\im ks}\xi(s)\ds$ be its $k$-th Fourier coefficient.
Then the Fourier series is Césaro summable to $\xi$ in every point $t \in (0,2\pi)$.
Moreover one has
%% --
\begin{equation}\label{eq:a3-7.8}
\frac{1}{2}(\xi(0) + \xi(2\pi)) = C_{1}\text{-}\sum_{k\in\Z} \xi_{n} \coloneqq \lim_{N\to\infty} \frac{1}{N}\sum_{n=0}^{N-1}(\sum_{k=-n}^{n} \xi_{k})\,.
\end{equation}
%% --
This result enables us to prove the following proposition.
%% --
\begin{proposition}\label{prop:a3-7.8}
\index{Spectral Theory!Cesaro Summability}
Let $(T(t))_{t\geq 0}$ be a semigroup on a Banach space $E$ and denote its generator by $A$.
Then the following conditions are equivalent
%% --
\begin{enumerate}[(a)]
\item $1 \in \rho(T(2\pi))$\,,

\item 
$\im\Z \subset \rho(A)$ and the series $\sum_{k\in\Z} R(\im k,A)f$ is Césaro-summable for every $f \in E$\,,

\item 
$\im\Z \subset \rho(A)$ and the series $\sum_{k\in\Z} R(\im k,A)Q_{k}f$ is Césaro-summable for every $f \in E$\,.

\end{enumerate}
\end{proposition}
%% --
\begin{proof}
(a) $\Rightarrow$ (b) The Spectral Inclusion (see )Proposition~\ref{prop:a3-6.2}) implies $\im\Z \subset \rho(A)$.
By \eqref{eq:a3-7.7} we have $R(\im k,A) = 2\pi\cdot(1-T(2\pi))^{-1}Q_{k}$.
Since $\sum_{k\in\Z} Q_{k}f$ is Césaro-summable (towards $\frac{1}{2}(f + T(2\pi)f)$) (see \eqref{eq:a3-7.8}, it follows that $\sum_{k\in\Z} R(\im k,A)f$ is Césaro-summable as well.

(b) $\Leftrightarrow$ (c) If we use A-I,(3.1) and integrate by parts, we obtain
\begin{align*}
R(\im k,A)Q_{k}f &= \frac{1}{2\pi}\int_{0}^{2\pi} \mathrm{e}^{-\im ks} T(s)R(\im k,A)f \ds \\
&= \frac{1}{2\pi}\int_{0}^{2\pi} \left[R(\im k,A)f - \int_{0}^{s} \mathrm{e}^{-\im kt} T(t)f \dt\right] \ds \\
&= R(\im k,A)f - \frac{1}{2\pi}\int_{0}^{2\pi} \mathrm{e}^{-\im kt} (2\pi-t) T(t)f \dt\,.
\end{align*}
Fejér's theorem ensures that $\sum_{k\in\Z} (1/2\pi)\int_{0}^{2\pi} \mathrm{e}^{-\im kt} (2\pi-t) T(t)f \dt$ is Césaro summable.
Hence $\sum_{k\in\Z} R(\im k,A)Q_{k}f$ is Césaro-summable if and only if $\sum_{k\in\Z} R(\im k,A)f$ is.

(b) $\Rightarrow$ (a) We have $Q_{k} = \frac{1}{2\pi}(1 - T(2\pi))R(\im k,A)$.
Furthermore we know by \eqref{eq:a3-7.7} and \eqref{eq:a3-7.8} that $\sum_{k\in\Z} Q_{k}f$ is Césaro-summable towards $\frac{1}{2}(f + T(2\pi)f)$.
If we define $S \colon E \to E$ by $Sf \coloneqq \frac{1}{2}f + \frac{1}{2\pi}\cdot C_{1}\text{-}\sum_k R(\im k,A)f$, then we have
\begin{align*}
(1 - T(2\pi))Sf &= \frac{1}{2}(f - T(2\pi)f) + \frac{1}{2\pi}\cdot C_{1}\text{-}\sum_k (1 - T(2\pi))R(\im k,A)f \\
&= \frac{1}{2}(f - T(2\pi)f) + \frac{1}{2}(f + T(2\pi)f) = f\,.
\end{align*}
Since $S$ commutes with $T(2\pi)$, it follows that $S$ is the inverse of $(1 - T(2\pi))$ thus $1 \in \rho(T(2\pi))$.
\end{proof}
%% --
Based on the equivalence of (a) and (b), one can state a characterization of the spectrum of $T(t)$ in terms of the generator and its resolvent only.
However, in general it is difficult to verify the summability condition stated in (b).

In Hilbert spaces the boundedness of the resolvents will suffice (see Theorem~\ref{thm:a3-7.10} below).
%% --
\begin{lemma}\label{lem:a3-7.9}
Let $(T(t))_{t\geq 0}$ be a semigroup on some Hilbert space $H$ and assume $\im\Z \subset \rho(A)$ for the generator $A$.
Then we have
%% --
\begin{enumerate}[(i)]
\item 
$(Q_{k}f)_{k\in\Z} \subset \ell^{2}(H)$ for every $f \in H$, and

\item 
if\quad $\sup_{k\in\Z}\|R(\im k,A)\| < \infty$~, then\quad $\sum_{k\in\Z} R(\im k,A)f_{k}$\quad is summable 

whenever\quad $(f_{k})_{k\in\Z} \in \ell^{2}(H)$\,.
\end{enumerate}
\end{lemma}
%% --
\begin{proof}
(i) We consider the Hilbert space $L^{2}([0,2\pi],H)$ and obtain
\begin{align*}
0 &\leq \left\|T(\cdot)f - \sum_{k=-n}^{n} Q_{k}f\cdot \mathrm{e}^{\im k\cdot}\right\|^{2} \\
&= \int_{0}^{2\pi} \|T(s)f\|^{2} \ds - \int_{0}^{2\pi} \sum_{k=-n}^{n} (T(s)f|\mathrm{e}^{\im ks}Q_{k}f) \ds - \\
&\quad \int_{0}^{2\pi} \sum_{k=-n}^{n} (\mathrm{e}^{\im ks}Q_{k}f|T(s)f) \ds + \int_{0}^{2\pi} \left(\sum_{k=-n}^{n} \mathrm{e}^{\im ks}Q_{k}f|\sum_{\ell=-n}^{n} \mathrm{e}^{i\ell s}Q_{\ell}f\right) \ds \\
&= \int_{0}^{2\pi} \|T(s)f\|^{2} \ds - 2\pi\sum_{k=-n}^{n} \|Q_{k}f\|^{2} \text{ (use \eqref{eq:a3-7.5} }\,.
\end{align*}
It follows that $\sum_{k\in\Z} \|Q_{k}f\|^{2} \leq \frac{1}{2\pi}\int_{0}^{2\pi} \|T(s)f\|^{2} \ds < \infty$.

(ii) Fix $\lambda > 0$ sufficiently large and set 
\[
g_{k}\coloneqq (1 + \lambda R(\im k,A))f_{k}, k \in \Z\,.
\]
Using the resolvent equation and then (A-I,(3.1)), we obtain
%% --
\[
R(\im k,A)f_{k} = R(\lambda + \im k,A)g_{k} = [1 - \mathrm{e}^{-2\pi\lambda}T(2\pi)]^{-1}\int_{0}^{2\pi} \mathrm{e}^{-\lambda s}\mathrm{e}^{-\im ks}T(s)g_{k} \ds\,.
\]
%% --
This yields for every finite subset $N$ of $\Z$ that
%\begin{align*} %% besser mit align*
%% --
\[
\left\|\sum_{k\in N} R(\im k,A)f_{k}\right\| \leq \|(1 - \mathrm{e}^{-2\pi\lambda}T(2\pi))^{-1}\|\cdot\int_{0}^{2\pi} \|T(s)\| \left\|\sum_{k\in N} \mathrm{e}^{-\im ks}g_{k}\right\| \ds
\]
%% --
\[
\leq \|(1 - \mathrm{e}^{-2\pi\lambda}T(2\pi))^{-1}\|\cdot\left(\int_{0}^{2\pi}\|T(s)\|^{2} \ds\right)^{1/2} \cdot \left(\int_{0}^{2\pi}\left\|\sum_{k\in N} \mathrm{e}^{-\im ks}g_{k}\right\|^{2} \dx\right)^{1/2}
\]
%% --
\[
= c(\sum_{k\in N} \|g_{k}\|^{2})^{1/2} \leq c(1 + \lambda M)\left(\sum_{k\in N} \|f_{k}\|^{2}\right)^{1/2}\,.
\]
%% --
%\end{align*}
Here $c\coloneqq \|(1 - \mathrm{e}^{-2\pi\lambda}T(2\pi))^{-1}\|\cdot\left(\int_{0}^{2\pi}\|T(s)\|^{2} \ds\right)^{1/2}$ and $M\coloneqq \sup_{k\in\Z}\|R(\im k,A)\|$.
\end{proof}
%% --
\begin{theorem}\label{thm:a3-7.10}
\index{Spectral Mapping Theorem!Hilbert Space}
Let $A$ be the generator of a semigroup $(T(t))_{t\geq 0}$ on some Hilbert space $H$.
Then the following form of the spectral mapping theorem is valid.
\begin{align*}
\sigma(T(t))\setminus\{0\} = \{\mathrm{e}^{\lambda t} \colon
& \text{ either } \mu_{k}\coloneqq \lambda + 2\pi\im  k/t \in \sigma(A) \text{ for some } k \in \Z\\
& \text{ or } (\|R(\mu_{k},A)\|)_{k\in\Z} \text{ is unbounded}\}\,.
\end{align*}
\end{theorem}
%% --
\begin{proof}
If $\mathrm{e}^{\lambda t} \not\in \sigma(T(t))$, it follows from the spectral inclusion theorem that $\mu_{k} \not\in \sigma(A)$ for every $k \in \Z$ and from Formula (3.1) in A-I, that $\|R(\mu_{k},A)\|$ is bounded.
For the converse inclusion it suffices to assume $t = 2\pi$ and $\lambda = 0$ (use the rescaling procedure A-I,3.1).
Assuming that $\im\Z \subset \rho(A)$ and $\|R(\im k,A)\|$ is bounded, then $\sum_{k\in\Z} R(\im k,A)Q_{k}f$ is summable by Lemma~\ref{lem:a3-7.9}.
Since every summable series is Cesàro-summable, condition (c) of Proposition~\ref{prop:a3-7.8} is satisfied and we conclude $1 \in \rho(T(2\pi))$.
\end{proof}
%% --
As an immediate consequence we obtain an interesting characterization of the growth bound $\omega_{0}$ of semigroups on Hilbert spaces.
%% --
\begin{corollary}\label{cor:a3-7.11}
\index{Growth Bound!Hilbert Space}
The growth bound of a semigroup $(T(t))_{t\geq 0}$ on a Hilbert space $H$ satisfies
%% --
\begin{equation}\label{eq:a3-7.9}
\omega_{0} = \inf \{\lambda \in \R \colon \lambda + \im\R \subset \rho(A) \text{ and } \|R(\lambda+\im\mu,A)\| \text{ is bounded for } \mu \in \R\}\,.
\end{equation}
\end{corollary}
%% --
The Example~\ref{ex:a3-1.3} above in combination with C-III,Corollary~1.3 shows that \eqref{eq:a3-7.9} is not valid in arbitrary Banach spaces.

\section*{NOTES}
\addcontentsline{toc}{section}{Notes}
%% --
\begin{enumerate}[label=\emph{Section \arabic*:}, wide]
\item 
It was already known to \citet{hillephillips:1957} that for strongly continuous semigroups $(T(t))_{t\geq 0}$ with generator $A$ the spectral mapping theorem \enquote{$\sigma(T(t)) = \exp(t\sigma(A))$} may be violated in various ways [l.c.,Sec.23.16].
The simple Examples~\ref{ex:a3-1.3} and ~\ref{ex:a3-1.4} are due to Wolff (see \citet{greinervoigtwolff:1981}) and \citet{zabczyk:1975}.
A more sophisticated example of a positive group with 
\enquote{$s(A) < \omega_{0}(A)$} is given in \citet{wolff:1981}.

\item 
In Definition~\ref{def:a3-2.1} we define the residual spectrum of $A$ in such a way that it coincides with the point spectrum of the adjoint $A'$ (see Proposition~\ref{prop:a3-2.2}.(ii)).
It therefore differs slightly from the one used, \eg by \citet{schaefer:1974}.
The spectral mapping theorem for the resolvent (Theorem~\ref{prop:a3-2.5}) is well known and can, \eg be deduced from Lemma~9.2 and Theorem~3.11 of Chap.VII in \citet{dunfordschwartz:1958}.

\item 
The general theory of spectral decompositions can be found in \citet{kato:1966}, Chap.III,§ 6.4].
Applications to isolated singularities like 3.6 are discussed extensively in [l.c., Chap. III, §6.5] and \citet[Chap.VIII, Sec.8]{yosida:1965}.
There are many ways to introduce an \enquote{essential spectrum} (see the footnote on page 243 of \citet{kato:1966}).
However each one yields the same \enquote{essential spectral radius}.

\item 
These results are taken from \citet{derndinger:1980} and \citet{greiner:1981}.

\item 
Periodic semigroups are studied explicitly in \citet{bart:1977}, but most of the results of this section seem to be well known.

\item
The Spectral Inclusion (see Proposition~\ref{prop:a3-6.2}) and the Spectral Mapping Theorem~\ref{thm:a3-6.6} for eventually norm continuous semigroups date back to \citet{hillephillips:1957}.
\citet{davies:1980} gives an elegant proof using Banach algebra techniques.
Tensor products of operators and their spectral theory have been studied by Ichinose and others (see Chap. XIII.9 of \citet{reedsimon:1978}).
The spectral mapping theorem in Corollary~\ref{cor:a3-6.8} generalizes Theorem~XIII.35 of \citet{reedsimon:1978} (see also \citet{herbst:1982}).

\item 
Matrix valued multiplication semigroups appear as solution, via Fourier transformation, of systems of partial differential equations.
Kreiss initiated a systematic investigation (see, \eg \citet{kreiss:1958}, \citet{kreiss:1959}, \citet{millerstrang:1966}) and the Weak Spectral Mapping Theorem~\ref{thm:a3-7.2} must have been known to him. The direct proof of the Weak Spectral Mapping Theorem~\ref{thm:a3-7.4} for polynomially bounded groups seems to be new. The result can also be deduced from the theory of spectral subspaces of group representations (see, \eg \citet{combesdelaroche:1978}), since the Arveson spectrum of a strongly continuous one-parameter group can be identified with the spectrum of the generator (see \citet{evans:1976}). The final part of this section is taken from \citet{greiner:1985} and yields a new approach to Gearhart's characterization of the spectrum of semigroups on Hilbert spaces \citet{gearhart:1978}. Different proofs can be found in \citet{herbst:1983}, \citet{howland:1984} and \citet{pruess:1984}. 
\end{enumerate}



