% !TEX root = ../LN-Book.tex
%% --
%% -- Stand 2025-06-06  
%% -- Final
%% --
\setcounter{chapter}{1}
\chapter{Extended Notes: Chapter B-II}
\label{chap:eb2}%
%\index{$C_{0}(X)$!Positive Semigroups}
%% --
%{\Large
%\vspace*{-.75cm}
%by \\[.25em]
%Wolfgang Arendt 
%\vspace{.75cm}
%\\
%}
%% --
\section[Positive semigroups generated by elliptic operators]{Positive semigroups generated by elliptic operators on spaces of continuous functions}
Important examples of semigroups on $C_0(\Omega)$ or $C(\overline{\Omega})$, where $\Omega \subset \R^n$ is open and bounded, are generated by elliptic differential operators. In the following we put together a series of results starting with the Laplacian subject to Dirichlet and to Robin boundary conditions and ending with the Dirichlet-to-Neumann operator on $C(\partial\Omega)$. Each time we obtain a positive irreducible semigroup. We consider $\mathbb{K} = \R$ throughout this section.
%% --
\subsection{The Laplacian}
%% --
Let $\Omega \subset \R^d$ be open and bounded. We say that $\Omega$ is \emph{Dirichlet-regular} if for every $g \in C(\partial\Omega)$ there exists a (unique) function $u \in C(\overline{\Omega}) \cap C^2(\Omega)$ such that
%% --
\begin{align*}
 	\Delta u &= 0 \quad \text{and} \\
	u|_{\partial\Omega} &= g.
\end{align*} 
%% --
This means that the Dirichlet problem is well-posed. 
This property is very well understood and precise characterizations in terms of barriers or of capacity are known. 
If $\Omega$ has Lipschitz boundary, then $\Omega$ is Dirichlet regular. 
In dimension $d = 2$ it suffices that $\Omega$ is simply connected.

We refer to \citet[Section 6.9]{Au23}  or \citet[Section 2.8]{GT83}  for further information on the Dirichlet Problem. 

The Dirichlet Laplacian $\Delta_0$ on $C_0(\Omega)$ is defined by
%% --
\begin{align*}
 	\Delta_0 u &\coloneq \Delta u \\
 	D(\Delta_0) &\coloneq \{ u \in C_0(\Omega) \colon \Delta u \in C_0(\Omega) \}.
\end{align*}
%% --
Here $\Delta u$ is to be understood in the sense of distributions.

In thes section we consider always real spaces.
Then a semigroup is called \emph{holomorphic} if its extension to the coerresponding complexification (here $ C_{0}( \Omega, \C ) $) is holomorphic.
%% --
\begin{theorem} % 4.1
The following are equivalent.
%% --
\begin{enumerate}[\upshape (a)]

\item 
$\Omega$ is Dirichlet regular;

\item  
$\Delta_0$ generates a positive semigroup $\mathcal{T}$ on $C_0(\Omega)$.

\end{enumerate}
%% -- 
In that case the semigroup $\mathcal{T}$ is holomorphic of angle $\pi/2$. 
Moreover $T(t)$ is compact for all $t > 0$.
If $\Omega$ is connected, then the semigroup is irreducible. 
Moreover,
%% --
\begin{equation}
	\|T(t)\| \leq Me^{-\varepsilon t} \quad (t \geq 0)
\end{equation}
%%
for some $\varepsilon > 0$, $M \geq 1$.
\end{theorem}
%% --
This result is due to Arendt-Bénilan \cite{ArBe99} besides irreducibility on which we comment later.
In Example C-II.1.5 (e), the generation result was obtained if $\Omega$ has $C^2$-boundary.

The implication (a) $\Rightarrow$ (b) of Theorem 4.1 is proved below in order to show how the Dirichlet problem comes into play and leads to a result with minimal regularity assumptions on the boundary of $\Omega$.

We use the following abstract generation result which is of independent interest. 
By C-II, Theorem 1.2 a densely defined operator $A$ generates a contractive positive semigroup if and only if $A$ is dispersive and $(\lambda - A)$ is surjective for some $\lambda > 0$.
We now describe the case $\lambda = 0$.
%% --
\begin{theorem}
Let $A$ be a densely defined operator on a real or complex Banach lattice $E$. 
The following are equivalent.
%% --
\begin{enumerate}[\upshape (a)]
\item  
$A$ generates a positive, contractive semigroup and $s(A) \leq 0$.
\item  
$A$ is dispersive and surjective.
\end{enumerate}
%% --
In particular, (b) implies that $A$ is closed.
\end{theorem}
%% --
Dispersive operators are defined before C-II, Theorem 1.2. A densely defined operator $A$ on $C_0(\Omega)$ is dispersive iff for $u \in D(A)$, $x_0 \in \overline{\Omega}$:
\[u(x_0) = \sup_{x \in \overline{\Omega}} u(x) > 0 \text{ implies } (Au)(x_0) \leq 0.\]

\begin{proof}(Theorem 4.2.) 
(b) $\Rightarrow$ (a)

Consider the equivalent norm
\[
\|u\|_1 \coloneq \|u^+\| + \|u^-\|
\]
on $E$. 
Since $A$ is dispersive it is dissipative with respect to this new norm as is easy to see. Now Theorem 4.5 of Arendt, Chalendar and Moletsare \cite{ACM24} implies that $A$ generates a contraction semigroup $\mathcal{T}$ and $A$ is invertible. Since $A$ is dispersive, it follows from C-II, Theorem 1.2 that $\mathcal{T}$ is positive and contractive (with respect to the original norm). Since $R(\lambda, A) \geq 0$ for $\lambda > 0$, it follows that $-A^{-1} \geq 0$. Now C-I, Theorem 1.1 (vi) implies that $s(A) \leq 0$.

(a) $\Rightarrow$ (b) is obvious from C-II, Theorem 1.2. 

\end{proof}
%% --
\begin{proof}(Theorem 4.1.)  
(a) $\Rightarrow$ (b)

The operator $\Delta_0$ is dispersive by the maximum principle. 
If $\Omega$ is Dirichlet regular, then $\Delta_0$ is surjective. In fact, let $f \in C_0(\Omega)$. Extend $f$ by $0$ to $\R^n$ and let $w = \Gamma * f$, where $\Gamma$ is the fundamental solution of Laplace's equation (see Gilbarg and Trudinger \cite[2.12]{GT83}). 
Then $w \in C(\R^n)$ and $\Delta w = f$ in the sense of distributions. 
Let $g = w|_{\partial\Omega}$ and let $v \in C^2(\overline{\Omega}) \cap C(\overline{\Omega})$ be the solution of the Dirichlet problem, \ie 
%
\[
	v|_{\partial\Omega} = g 
	\quad \text{and} \quad \Delta v = 0 \text{in $\Omega$.}
\]
%
Then $u \coloneq w - v \in D(\Delta_0)$ and $\Delta u = f$.

We have shown that $\Delta_0$ satisfies condition (b) of Theorem 4.2. 
Thus $\Delta_0$ generates a positive, contractive $C_0$-semigroup $(T(t))_{t \geq 0}$ on $C_0(\Omega)$ and $s(\Delta_0) \leq 0$. 
Since by C-IV Theorem 1.1 (iv) $s(\Delta_0) = \omega_{0}(\Delta_0)$, it is exponentially stable.

We refer to Arendt and Bénilan \cite{ArBe99} for the proof of (b) $\Rightarrow$ (a).
%% --
\end{proof}
%% --
We want to add two further comments on the Dirichlet Laplacian $\Delta_0$ on $C_0(\Omega)$. The first concerns its domain
%% --
\[
D(\Delta_0) = \{u \in C_0(\Omega) \colon \Delta u \in C_0(\Omega)\}.
\]
%% --
This distributional domain is not contained in $C^2(\Omega)$ for any open set $\Omega \subset \R^n$, $n \geq 2$, see Arendt-Urban \cite[Theorem 6.60]{Au23}. 

Our second comment concerns the proof of holomorphy. It can be given via Gaussian estimates (see the Extended Notes for C-II). In our context, a short proof based on Kato's inequality of C-II, Section 2 is more appealing (see Arendt-Batty \cite{ArBa92}).

Finally, we comment on irreducibility. 
On $C_0(\Omega)$ it is a strong property. 
By C-III, Theorem 3.2 (ii) it means that for $0 \leq f \in C_0(\Omega)$, $f \neq 0$,
%% --
\[
(T(t)f)(x) > 0 \text{ for all } x \in \Omega, t > 0.
\]
%% --
On $L^2(\Omega)$ irreducibility is much weaker (meaning that $(T(t)f)(x) > 0$ $x$-a.e.), but easy to prove (see the Extended Notes to C-I). In the paper Arendt, ter Elst, Glück \cite{AEG20} an argument based on Banach lattice technique shows how irreducibility on $L^2(\Omega)$ can be carried over to $C_0(\Omega)$ or even to $C(\overline{\Omega})$ in the case of Robin boundary conditions which we consider now.

By %
\[
	H^1(\Omega) \coloneq \{u \in L^2(\Omega) \colon \partial_j u \in L^2(\Omega) \text{ for } j = 1, \ldots, n\}
\]
%
we denote the first Sobolev space.
We assume that $\Omega$ has Lipschitz boundary. 
Then there exists a unique bounded operator %
\[
	\text{tr} \colon H^1(\Omega) \to L^2(\partial\Omega)
\]
%
such that $\text{tr }(u) = u|_{\partial\Omega}$ for all $u \in C^1(\overline{\Omega})$. 
It is called the \emph{trace operator}.

Here the space $L^2(\partial\Omega)$ is defined with respect to the surface measure (i.e. the $(d-1)$-dimensional Hausdorff measure) on $\partial\Omega$.

The normal derivative $\partial_\nu u$ of $u$ is defined as follows. Let $u \in H^1(\Omega)$ such that $\Delta u \in L^2(\Omega)$. Let $h \in L^2(\partial\Omega)$. We say that $h$ is the \emph{(outer) normal derivative} of $u$ and write $\partial_\nu u = h$ if
\[\int_\Omega \Delta u v + \int_\Omega \nabla u \nabla v = \int_{\partial\Omega} h v\]
for all $v \in C^1(\overline{\Omega})$.

If $u \in H^1(\Omega)$ such that $\Delta u \in L^2(\Omega)$ we say $\partial_\nu u \in L^2(\partial\Omega)$ if there exists $h \in L^2(\partial\Omega)$ such that $\partial_\nu u = h$.
%% --
\begin{remark*}
Since $\Omega$ has Lipschitz boundary the outer normal $\nu(z)$ exists for almost all $z \in \partial\Omega$ and $\nu \in L^\infty(\partial\Omega)$. 
But we do not use this outer normal and rather define $\partial_\nu u$ weakly by the validity of Green's formula.
\end{remark*}
%% --
Let $\beta \in L^\infty(\partial\Omega)$. 
We define the Laplacian $\Delta^\beta$ with Robin boundary conditions as follows:
%% --
\begin{align}
	D(\Delta^\beta) &\coloneq \{u \in H^1(\Omega) : \Delta u \in L^2(\Omega), \partial_\nu u + \beta 				\text{tr}(u) = 0\} \\
	\Delta^\beta u &\coloneq \Delta u.
\end{align}
%% --
We call $\Delta^\beta$ briefly the Robin-Laplacian. Note that for $\beta = 0$, we obtain Neumann boundary conditions, and $\Delta^0 =: \Delta^N$ is the Neumann Laplacian.

The following result is valid.

\begin{theorem}[4.3]
Assume that $\Omega \subset \R^d$ is bounded, open, connected with Lipschitz boundary, and let $\beta \in L^\infty(\partial\Omega)$. Then $\Delta^\beta$ generates a positive, irreducible, holomorphic semigroup $\mathcal{T} = (T(t))_{t \geq 0}$ on $C(\overline{\Omega})$. Moreover, $T(t)$ is compact for all $t > 0$.
\end{theorem}

The generation property on $C(\overline{\Omega})$ is due to Nittka \cite{Ni11}. A major point is to show that the resolvent of the corresponding operator on $L^2(\Omega)$ leaves $C(\overline{\Omega})$ invariant. Given $f \in C(\overline{\Omega})$, $u \in H^1(\Omega)$ such that $u - \Delta u = f$, $\partial_\nu u + \beta u|_{\partial\Omega} = 0$.
%% --
% !TEX root = Extended_Notes_B-II.tex


\begin{theorem}[4.4]
Assume (4.2) and (4.4). Then $\Delta - V$ generates a positive, irreducible semigroup on $C(\partial\Omega)$. If $V \geq 0$, then the semigroup is contractive.
\end{theorem}

If $\Omega$ is of class $C^\infty$ similar results have been obtained by Escher \cite{Es94} and Engel \cite{En03}. Under the very general conditions here, Theorem 4.8 is due to Arendt and ter Elst \cite{AtE20}. There it is shown that $N_V$ is resolvent-

The generation property on $C(\overline{\Omega})$ is due to Nittka \cite{Ni11}. A major point is to show that the resolvent of the corresponding operator on $L^2(\Omega)$ leaves $C(\overline{\Omega})$ invariant. Given $f \in C(\overline{\Omega})$, $u \in H^1(\Omega)$ such that $u - \Delta u = f$, $\partial_\nu u + \beta u|_{\partial\Omega} = 0$.

One has to show that $u \in C(\overline{\Omega})$. Nittka extends $u$ to an open set $\widetilde{\Omega}$ containing $\overline{\Omega}$ by reflecting $u$ along the graph. Then $u$ becomes the solution of an elliptic problem on $\widetilde{\Omega}$. Continuity on $\widetilde{\Omega}$, and hence on $\overline{\Omega}$, follows from the De Giorgi-Nash Theorem.

Irreducibility is due to Arendt, ter Elst and Glück \cite{AEG20}, Theorem 4.5.
These results have first been proved for $ \beta \geq 0 $.
Daves \cite{Dav09} has shown how one can treat general $ \beta \in L^{\infty}( \delta \Omega ) $.

Since the semigroup is holomorphic, by C-II, Theorem 3.2 (ii), it implies that
%% --
\begin{equation} \tag{4.1}
\inf_{x \in \overline{\Omega}} (T(t)f)(x) > 0
\end{equation}
%% --
for all $t > 0$ and $0 \leq f \in C(\overline{\Omega})$, $f \neq 0$.

Denote by $s(\Delta^\beta)$ the spectral bound of $\Delta^\beta$. 
By C-III, Theorem 3.8 (iv), $s(\Delta^\beta)$ is the unique eigenvalue with a positive eigenfunction $u_0 \geq 0$, $u_0 \neq 0$. 
It follows from (4.1) that $u_0$ is strictly positive; \ie
%% --
\[
	\inf_{x \in \overline{\Omega}} u_0(x) > 0,
\]
%% --
a remarkable property, which has important applications to semi-linear problems, see Arendt-Daners \cite{AD25}.

The spectral bound $s(\Delta^\beta)$ determines the asymptotic behavior of the semigroup $\mathcal{T}$. 
In fact, the following corollary follows from B-III Proposition 3.5.
%% --
\begin{corollary} 
%% --
There exist a strictly positive Borel measure $\mu$ on $\overline{\Omega}$, $M \geq 0$ and $\varepsilon > 0$ such that $\langle \mu, u_0 \rangle = 1$ and
\[\|T(t) - e^{s(\Delta^\beta)t}P\| \leq Me^{-\varepsilon t}\]
for all $t \geq 0$, where $P \in \mathcal{L}(C(\overline{\Omega}))$ is given by
\[Pf = \langle \mu, f \rangle u_0.\]
\end{corollary}
%% --
The theorem says that the rescaled semigroup $(e^{-s(\Delta^\beta)t}T(t))_{t \geq 0}$ converges in the operator norm to the rank-1-projection $P$ exponentially fast.
%% --
\subsection{Elliptic operators in divergence form}
%% --
The preceding results extend to elliptic operators in divergence form with bounded measurable coefficients. Let $\Omega \subset \R^n$ be open and bounded.

Let $a_{k,\ell}, b_k, c_k, c_0 \in L^\infty(\Omega)$, $k, \ell = 1, \ldots, n$ such that for some $\alpha > 0$
%% --
\[
	\sum_{k,\ell=1}^n a_{k,\ell}(x)\xi_k\xi_\ell \geq \alpha|\xi|^2
\]
%% --
for all $x \in \Omega$, $\xi \in \R^n$, where $|\xi|^2 = \xi_1^2 + \cdots + \xi_n^2$.
Let 
%
\[
	H^1_{\text{loc}}(\Omega) := \{u \in L^2_{\text{loc}}(\Omega) \colon \partial_k u \in L^2_{\text{loc}}(\Omega), k = 1, \ldots, n\}.
\]
%
Define $\mathcal{A} \colon H^1_{\text{loc}}(\Omega) \to C_c^\infty(\Omega)'$ by
%% --
\[
	\langle \mathcal{A}u, v \rangle = \int_{\Omega}
		\left( \sum_{k,\ell=1}^d \partial_k(a_{k\ell}\partial_\ell u) + 
		\sum_{k=1}^d \partial_k(b_k u) + 
		\sum_{k=1}^d c_k \partial_k u + c_0 u \right) \dx .
\]
%% --
We define $A_0$ as the part of $\mathcal{A}$ in $C_0(\Omega)$; \ie
%% --
\begin{align*}
	D(A_0) &\coloneq \{u \in C_0(\Omega) \cap H^1_{\text{loc}}(\Omega) \colon \mathcal{A}u \in C_0(\Omega)\} \\
	A_0 u &\coloneq  \mathcal{A}u.
\end{align*}
%% -- 
Then Theorem 4.1 holds with $\Delta_0$ replaced by $A_0$. It is remarkable that Dirichlet regularity of $\Omega$ is the right regularity condition at the boundary, a discovery due to Stampacchia. We refer to Arendt and Bénilan \cite{ArBe99}, Section 4 for a proof of the following result.

\begin{theorem}
Assume that $\Omega \subset \R^n$ is a bounded, open, connected, Dirichlet regular set. Then $A_0$ generates a positive, irreducible, holomorphic semigroup $\mathcal{T} = (T(t))_{t \geq 0}$ on $C_0(\Omega)$. Moreover, $T(t)$ is compact for all $t > 0$.
\end{theorem}

\begin{remark*}
The proof of holomorphy depends on Gaussian estimates, which in \cite{ArBe99} were merely known if the $b_k \in W^{1,\infty}(\Omega)$. Later, it was shown by Davies \cite{Da00} that they always hold.
\end{remark*}

Also the results for Robin boundary conditions Theorem 4.3 and 4.4 can be extended to elliptic operators in divergence form on $C(\overline{\Omega})$; see Theorem 4.5 in Arendt, ter Elst, Glück \cite{AEG20}. It uses results of Nittka \cite{Ni11}.
%% --
\subsection{Elliptic operators in non-divergence forms}
%% --
The techniques for elliptic operators in non-divergence form are quite different than those used in the divergence-case form. But the results are similar.

Let $\Omega \subset \R^n$ be open and connected. We assume that $\Omega$ satisfies the uniform exterior cone condition. This means the following. There exists a finite, right circular cone $V$ such that for each $x \in \partial\Omega$ there exists a cone $V_x$ which is congruent to $V$ such that $V_x \cap \overline{\Omega} = \{x\}$.

Let $a_{k\ell} = a_{\ell k} \in C(\overline{\Omega})$, $b_k \in L^\infty(\Omega)$, $c \in L^\infty(\Omega)$, $c \leq 0$ such that
\[\sum_{k,\ell=1}^n a_{k\ell}(x)\xi_k\xi_\ell \geq \mu|\xi|^2\]
for all $x \in \overline{\Omega}$, $\xi \in \R^n$ and some $\mu > 0$.

For $u \in W^{2,n}_{\text{loc}}(\Omega)$ we define
\[\mathcal{A}u = \sum_{k,\ell=1}^n \partial_k a_{k\ell} \partial_\ell u + \sum_{k=1}^n b_k \partial_k u + cu.\]

Thus $\mathcal{A} \colon W^{2,n}_{\text{loc}}(\Omega) \to L^n_{\text{loc}}(\Omega)$ is linear. Here
\[W^{2,n}_{\text{loc}}(\Omega) := \{u \in L^n_{\text{loc}}(\Omega) \colon \partial_k u \in L^n_{\text{loc}}(\Omega), \partial_k \partial_\ell u \in L^n_{\text{loc}}(\Omega) \text{ for all } k, \ell = 1, \ldots, n\}.\]

We consider the operator $A$ on $C_0(\Omega)$ defined by
\[D(A) := \{u \in C_0(\Omega) \cap W^{2,n}_{\text{loc}}(\Omega) \colon \mathcal{A}u \in C_0(\Omega)\}\]
\[Au := \mathcal{A}u.\]

Then the following holds.

\begin{theorem}
The operator $A$ generates a positive, irreducible, contractive holomorphic semigroup $(T(t))_{t \geq 0}$ on $C_0(\Omega)$. Moreover
\[\|T(t)\| \leq Me^{-\varepsilon t} \quad (t \geq 0)\]
for some $\varepsilon > 0$, $M \geq 1$. The resolvent of $A$ is compact.
\end{theorem}
%% --
This result is proved by Arendt and Schätzle \cite{AS14}, Proposition 4.7. 

Positivity and irreducibility are a consequence of the Alexandrov maximum principle. For $C^2$-boundary also results on $L^p(\Omega)$ spaces are obtained by Denk, Hieber and Prüss \cite{DHP03} whose main interest lies in proving maximal regularity and establishing a bounded $H^\infty$-calculus.

However, in the situation of Theorem 4.6, without assuming merely the uniform exterior cone condition on $\Omega$, it seems not to be known whether the semigroup extends to a strongly continuous semigroup on $L^p(\Omega)$ for some $p \in [1,\infty)$.

Theorem 4.6 is extended by Arendt and Schätzle \cite{AS25} to unbounded open sets which satisfy the locally uniform exterior cone condition. However, in the case of unbounded $\Omega$ the semigroup converges merely strongly to $0$ (and not exponentially fast).

The monograph of Lunardi \cite{Lu95} is devoted to the study of holomorphic semigroups generated by elliptic operators in non-divergence form.
%% --
\subsection{The Dirichlet-to-Neumann operator on $C(\partial\Omega)$}
%% --
Let $\Omega$ be a bounded, open, connected subset of $\R^n$ with Lipschitz boundary and let $V \in L^\infty(\Omega)$. We consider the Dirichlet-to-Neumann operator with respect to $\Delta - V$ on the space $C(\partial\Omega)$. For that we first establish well-posedness of the Dirichlet Problem.

We assume throughout this subsection that
\begin{equation} \tag{4.2}
u \in C_0(\Omega), \Delta u - Vu = 0 \text{ implies } u = 0.
\end{equation}

This is exactly the condition that the solutions of the Dirichlet problem with respect to $\Delta - V$ formulated in Proposition 4.7 are unique. An equivalent condition is
\begin{equation} \tag{4.3}
u \in H^1_0(\Omega), \Delta u - Vu = 0 \text{ implies } u = 0;
\end{equation}
(which means that $0 \notin \sigma(\Delta_0 - V)$ where $\Delta_0$ is the Dirichlet Laplacian on $L^2(\Omega)$).

\begin{proposition}
Assume (4.2). Let $g \in C(\partial\Omega)$. Then there exists a unique $u_g \in C(\overline{\Omega})$ such that
\[(\Delta - V)u_g = 0 \quad \text{and} \quad u_g|_{\partial\Omega} = g.\]

Thus, $u_g$ is a harmonic function with respect to $\Delta - V$ which has to be understood in the sense of distributions; i.e.,
\[\int_\Omega u_g \Delta\varphi - \int_\Omega Vu_g \varphi = 0\]
for all $\varphi \in C_c^\infty(\Omega)$.
\end{proposition}

For a simple proof of Proposition 4.7 we refer to \cite{AtE19}.

Next we define the Dirichlet-to-Neumann operator $N_V$ with respect to $\Delta - V$ on $C(\partial\Omega)$ as follows.
\begin{align}
D(N_V) &:= \{g \in C(\partial\Omega) : u_g \in H^1(\Omega), \text{ and } \partial_\nu u_g \in C(\partial\Omega)\} \\
N_V g &:= -\partial_\nu u_g.
\end{align}

Recall that $\partial_\nu u_g \in C(\partial\Omega)$ means that there exists $h \in C(\partial\Omega)$ such that
\[\int_\Omega \Delta u_g \varphi + \int_\Omega \nabla u_g \nabla\varphi = \int_{\partial\Omega} h\varphi\]
for all $\varphi \in C^1(\overline{\Omega})$. Then we put $\partial_\nu u_g := h$.

We will need the hypothesis that $-\Delta_0 + V$ is form-positive i.e.
\begin{equation} \tag{4.4}
\int_\Omega (|\nabla u|^2 + V|u|^2) \geq 0
\end{equation}
for all $u \in H^1_0(\Omega)$.

\begin{theorem}
Assume (4.2) and (4.4). Then $\Delta - N_{V}$ generates a positive, irreducible semigroup on $C(\partial\Omega)$. If $V \geq 0$, then the semigroup is contractive.
\end{theorem}

If $\Omega$ is of class $C^\infty$ similar results have been obtained by Escher \cite{Es94} and Engel \cite{En03}. 
Under the very general conditions here, Theorem 4.8 is due to Arendt and ter Elst \cite{AtE20}. There it is shown that $N_V$ is resolvent-positive and that the domain is dense (which is the main difficulty). 
Then by B-II, Theorem 1.8 $N_V$ generates a positive semigroup. 
Irreducibility is surprising. 
In fact, even though $\Omega$ is supposed to be connected, $\partial\Omega$ might not be connected (consider a ring for example). 
The fact that the semigroup is irreducible shows that the operator $N_V$ is non-local in quite a dramatic way.

A first result on irreducibility (on $L^2(\partial\Omega)$) was obtained by Arendt and Mazzeo \cite{AM12}.

It is not known so far whether the semigroup generated by $N_V$ is holomorphic if $\partial\Omega$ has Lipschitz boundary. If the boundary is of class $C^{n+\alpha}$ with $\alpha > 0$, then it is holomorphic of angle $\pi/2$. This is due to ter Elst and Ouhabaz \cite{tEO19}.

The operator $N_V$ is also called \emph{voltage-to-current map} and has physical meaning. 
One version of the famous Calderón-Problem is the question whether for $V_1, V_2 \in L^\infty(\Omega)$, such that $0 \notin \sigma(\Delta_{V_1}) \cup \sigma(\Delta_{V_2})$,
\[N_{V_1} = N_{V_2} \text{ implies } V_1 = V_2.\]

This is true under the only assumption that $\Omega$ has Lipschitz boundary; see Theorem 1.1 by Krupchyk and Uhlmann \cite{KU14}.

Finally we mention that $N_V$ may generate a positive semigroup even if (4.4) is violated. This and other surprising phenomena were discovered by Daners \cite{Da14}, and led to the new theory of eventually positive semigroups; see e.g. \cite{DEK16}.




%%

%% -- Literatur
%% --
{\RaggedRight
\bibliographystyle{abbrvnat}
\bibliography{bib/eb2-references}
}

