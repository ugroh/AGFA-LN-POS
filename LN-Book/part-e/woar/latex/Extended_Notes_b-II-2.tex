% !TEX root = Extended_Notes_B-II.tex


\begin{theorem}[4.4]
Assume (4.2) and (4.4). Then $\Delta - V$ generates a positive, irreducible semigroup on $C(\partial\Omega)$. If $V \geq 0$, then the semigroup is contractive.
\end{theorem}

If $\Omega$ is of class $C^\infty$ similar results have been obtained by Escher \cite{Es94} and Engel \cite{En03}. Under the very general conditions here, Theorem 4.8 is due to Arendt and ter Elst \cite{AtE20}. There it is shown that $N_V$ is resolvent-

The generation property on $C(\overline{\Omega})$ is due to Nittka \cite{Ni11}. A major point is to show that the resolvent of the corresponding operator on $L^2(\Omega)$ leaves $C(\overline{\Omega})$ invariant. Given $f \in C(\overline{\Omega})$, $u \in H^1(\Omega)$ such that $u - \Delta u = f$, $\partial_\nu u + \beta u|_{\partial\Omega} = 0$.

One has to show that $u \in C(\overline{\Omega})$. Nittka extends $u$ to an open set $\widetilde{\Omega}$ containing $\overline{\Omega}$ by reflecting $u$ along the graph. Then $u$ becomes the solution of an elliptic problem on $\widetilde{\Omega}$. Continuity on $\widetilde{\Omega}$, and hence on $\overline{\Omega}$, follows from the De Giorgi-Nash Theorem.

Irreducibility is due to Arendt, ter Elst and Glück \cite{AEG20}, Theorem 4.5.
These results have first been proved for $ \beta \geq 0 $.
Daves \cite{Dav09} has shown how one can treat general $ \beta \in L^{\infty}( \delta \Omega ) $.

Since the semigroup is holomorphic, by C-II, Theorem 3.2 (ii), it implies that
%% --
\begin{equation} \tag{4.1}
\inf_{x \in \overline{\Omega}} (T(t)f)(x) > 0
\end{equation}
%% --
for all $t > 0$ and $0 \leq f \in C(\overline{\Omega})$, $f \neq 0$.

Denote by $s(\Delta^\beta)$ the spectral bound of $\Delta^\beta$. 
By C-III, Theorem 3.8 (iv), $s(\Delta^\beta)$ is the unique eigenvalue with a positive eigenfunction $u_0 \geq 0$, $u_0 \neq 0$. 
It follows from (4.1) that $u_0$ is strictly positive; \ie
%% --
\[
	\inf_{x \in \overline{\Omega}} u_0(x) > 0,
\]
%% --
a remarkable property, which has important applications to semi-linear problems, see Arendt-Daners \cite{AD25}.

The spectral bound $s(\Delta^\beta)$ determines the asymptotic behavior of the semigroup $\mathcal{T}$. 
In fact, the following corollary follows from B-III Proposition 3.5.
%% --
\begin{corollary} 
%% --
There exist a strictly positive Borel measure $\mu$ on $\overline{\Omega}$, $M \geq 0$ and $\varepsilon > 0$ such that $\langle \mu, u_0 \rangle = 1$ and
\[\|T(t) - e^{s(\Delta^\beta)t}P\| \leq Me^{-\varepsilon t}\]
for all $t \geq 0$, where $P \in \mathcal{L}(C(\overline{\Omega}))$ is given by
\[Pf = \langle \mu, f \rangle u_0.\]
\end{corollary}
%% --
The theorem says that the rescaled semigroup $(e^{-s(\Delta^\beta)t}T(t))_{t \geq 0}$ converges in the operator norm to the rank-1-projection $P$ exponentially fast.
%% --
\subsection{Elliptic operators in divergence form}
%% --
The preceding results extend to elliptic operators in divergence form with bounded measurable coefficients. Let $\Omega \subset \R^n$ be open and bounded.

Let $a_{k,\ell}, b_k, c_k, c_0 \in L^\infty(\Omega)$, $k, \ell = 1, \ldots, n$ such that for some $\alpha > 0$
%% --
\[
	\sum_{k,\ell=1}^n a_{k,\ell}(x)\xi_k\xi_\ell \geq \alpha|\xi|^2
\]
%% --
for all $x \in \Omega$, $\xi \in \R^n$, where $|\xi|^2 = \xi_1^2 + \cdots + \xi_n^2$.
Let 
%
\[
	H^1_{\text{loc}}(\Omega) := \{u \in L^2_{\text{loc}}(\Omega) \colon \partial_k u \in L^2_{\text{loc}}(\Omega), k = 1, \ldots, n\}.
\]
%
Define $\mathcal{A} \colon H^1_{\text{loc}}(\Omega) \to C_c^\infty(\Omega)'$ by
%% --
\[
	\langle \mathcal{A}u, v \rangle = \int_{\Omega}
		\left( \sum_{k,\ell=1}^d \partial_k(a_{k\ell}\partial_\ell u) + 
		\sum_{k=1}^d \partial_k(b_k u) + 
		\sum_{k=1}^d c_k \partial_k u + c_0 u \right) \dx .
\]
%% --
We define $A_0$ as the part of $\mathcal{A}$ in $C_0(\Omega)$; \ie
%% --
\begin{align*}
	D(A_0) &\coloneq \{u \in C_0(\Omega) \cap H^1_{\text{loc}}(\Omega) \colon \mathcal{A}u \in C_0(\Omega)\} \\
	A_0 u &\coloneq  \mathcal{A}u.
\end{align*}
%% -- 
Then Theorem 4.1 holds with $\Delta_0$ replaced by $A_0$. It is remarkable that Dirichlet regularity of $\Omega$ is the right regularity condition at the boundary, a discovery due to Stampacchia. We refer to Arendt and Bénilan \cite{ArBe99}, Section 4 for a proof of the following result.

\begin{theorem}
Assume that $\Omega \subset \R^n$ is a bounded, open, connected, Dirichlet regular set. Then $A_0$ generates a positive, irreducible, holomorphic semigroup $\mathcal{T} = (T(t))_{t \geq 0}$ on $C_0(\Omega)$. Moreover, $T(t)$ is compact for all $t > 0$.
\end{theorem}

\begin{remark*}
The proof of holomorphy depends on Gaussian estimates, which in \cite{ArBe99} were merely known if the $b_k \in W^{1,\infty}(\Omega)$. Later, it was shown by Davies \cite{Da00} that they always hold.
\end{remark*}

Also the results for Robin boundary conditions Theorem 4.3 and 4.4 can be extended to elliptic operators in divergence form on $C(\overline{\Omega})$; see Theorem 4.5 in Arendt, ter Elst, Glück \cite{AEG20}. It uses results of Nittka \cite{Ni11}.
%% --
\subsection{Elliptic operators in non-divergence forms}
%% --
The techniques for elliptic operators in non-divergence form are quite different than those used in the divergence-case form. But the results are similar.

Let $\Omega \subset \R^n$ be open and connected. We assume that $\Omega$ satisfies the uniform exterior cone condition. This means the following. There exists a finite, right circular cone $V$ such that for each $x \in \partial\Omega$ there exists a cone $V_x$ which is congruent to $V$ such that $V_x \cap \overline{\Omega} = \{x\}$.

Let $a_{k\ell} = a_{\ell k} \in C(\overline{\Omega})$, $b_k \in L^\infty(\Omega)$, $c \in L^\infty(\Omega)$, $c \leq 0$ such that
\[\sum_{k,\ell=1}^n a_{k\ell}(x)\xi_k\xi_\ell \geq \mu|\xi|^2\]
for all $x \in \overline{\Omega}$, $\xi \in \R^n$ and some $\mu > 0$.

For $u \in W^{2,n}_{\text{loc}}(\Omega)$ we define
\[\mathcal{A}u = \sum_{k,\ell=1}^n \partial_k a_{k\ell} \partial_\ell u + \sum_{k=1}^n b_k \partial_k u + cu.\]

Thus $\mathcal{A} \colon W^{2,n}_{\text{loc}}(\Omega) \to L^n_{\text{loc}}(\Omega)$ is linear. Here
\[W^{2,n}_{\text{loc}}(\Omega) := \{u \in L^n_{\text{loc}}(\Omega) \colon \partial_k u \in L^n_{\text{loc}}(\Omega), \partial_k \partial_\ell u \in L^n_{\text{loc}}(\Omega) \text{ for all } k, \ell = 1, \ldots, n\}.\]

We consider the operator $A$ on $C_0(\Omega)$ defined by
\[D(A) := \{u \in C_0(\Omega) \cap W^{2,n}_{\text{loc}}(\Omega) \colon \mathcal{A}u \in C_0(\Omega)\}\]
\[Au := \mathcal{A}u.\]

Then the following holds.

\begin{theorem}
The operator $A$ generates a positive, irreducible, contractive holomorphic semigroup $(T(t))_{t \geq 0}$ on $C_0(\Omega)$. Moreover
\[\|T(t)\| \leq Me^{-\varepsilon t} \quad (t \geq 0)\]
for some $\varepsilon > 0$, $M \geq 1$. The resolvent of $A$ is compact.
\end{theorem}
%% --
This result is proved by Arendt and Schätzle \cite{AS14}, Proposition 4.7. 

Positivity and irreducibility are a consequence of the Alexandrov maximum principle. For $C^2$-boundary also results on $L^p(\Omega)$ spaces are obtained by Denk, Hieber and Prüss \cite{DHP03} whose main interest lies in proving maximal regularity and establishing a bounded $H^\infty$-calculus.

However, in the situation of Theorem 4.6, without assuming merely the uniform exterior cone condition on $\Omega$, it seems not to be known whether the semigroup extends to a strongly continuous semigroup on $L^p(\Omega)$ for some $p \in [1,\infty)$.

Theorem 4.6 is extended by Arendt and Schätzle \cite{AS25} to unbounded open sets which satisfy the locally uniform exterior cone condition. However, in the case of unbounded $\Omega$ the semigroup converges merely strongly to $0$ (and not exponentially fast).

The monograph of Lunardi \cite{Lu95} is devoted to the study of holomorphic semigroups generated by elliptic operators in non-divergence form.