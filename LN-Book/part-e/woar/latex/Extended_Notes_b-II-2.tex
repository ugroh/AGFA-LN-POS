% !TEX root = Extended_Notes_B-II.tex
%\documentclass[11pt]{article}
%\usepackage{amsmath,amsthm,amssymb}
%\usepackage[utf8]{inputenc}
%\usepackage[T1]{fontenc}
%\usepackage{geometry}
%\geometry{a4paper,margin=2.5cm}
%
%\newtheorem{theorem}{Theorem}[section]
%\newtheorem{corollary}[theorem]{Corollary}
%\newtheorem{remark}[theorem]{Remark}
%
%% Custom commands
%\newcommand{\R}{\mathbb{R}}
%\newcommand{\CC}{\mathbb{C}}
%\newcommand{\N}{\mathbb{N}}
%\newcommand{\D}{\mathcal{D}}
%\newcommand{\dom}{\text{dom}}
%\renewcommand{\Re}{\text{Re}}
%\newcommand{\s}{\mathfrak{s}}
%\newcommand{\wo}{\omega_0}
%
%\title{Extended Notes for B-II\\
%Section 4: Positive semigroups generated by elliptic operators\\
%--- Continuation ---}
%\author{}
%\date{}
%
%\begin{document}
%\maketitle

\bigskip
\textbf{This document continues the Extended Notes for B-II, Section 4.}
\bigskip

The generation property on $C(\overline{\Omega})$ is due to Nittka \cite{Ni11}. A major point is to show that the resolvent of the corresponding operator on $L^2(\Omega)$ leaves $C(\overline{\Omega})$ invariant. Given $f \in C(\overline{\Omega})$, $u \in H^1(\Omega)$ such that $u - \Delta u = f$, $\partial_\nu u + \beta u|_{\partial\Omega} = 0$.

one has to show that $u \in C(\overline{\Omega})$. Nittka extends $u$ to an open set $\widetilde{\Omega}$ containing $\overline{\Omega}$ by reflecting $u$ along the graph. Then $u$ becomes the solution of an elliptic problem on $\widetilde{\Omega}$. Continuity on $\widetilde{\Omega}$, and hence on $\overline{\Omega}$, follows from the De Giorgi-Nash Theorem.

Irreducibility is due to Arendt, ter Elst and Glück \cite{AEG20}, Theorem 4.5.

Since the semigroup is holomorphic, by C-II, Theorem 3.2 (ii), it implies that
\begin{equation} \tag{4.1}
\inf_{x \in \overline{\Omega}} (T(t)f)(x) > 0
\end{equation}
for all $t > 0$ and $0 \leq f \in C(\overline{\Omega})$, $f \neq 0$.

Denote by $\s(\Delta^\beta)$ the spectral bound of $\Delta^\beta$. By C-III, Theorem 3.8 (iv), $\s(\Delta^\beta)$ is the unique eigenvalue with a positive eigenfunction $u_0 \geq 0$, $u_0 \neq 0$. It follows from (4.1) that $u_0$ is strictly positive; i.e.,
\[\inf_{x \in \overline{\Omega}} u_0(x) > 0,\]
a remarkable property, which has important applications to semi-linear problems, see Arendt-Daners \cite{ArDa25}.

The spectral bound $\s(\Delta^\beta)$ determines the asymptotic behavior of the semigroup $\mathcal{T}$. In fact, the following follows from B-III Proposition 3.5.

\begin{corollary}[4.4]
There exist a strictly positive Borel measure $\mu$ on $\overline{\Omega}$, $M \geq 0$ and $\varepsilon > 0$ such that $\langle \mu, u_0 \rangle = 1$ and
\[\|T(t) - e^{\s(\Delta^\beta)t}P\| \leq Me^{-\varepsilon t}\]
for all $t \geq 0$, where $P \in \mathcal{L}(C(\overline{\Omega}))$ is given by
\[Pf = \langle \mu, f \rangle u_0.\]
\end{corollary}

The theorem says that the rescaled semigroup $(e^{-\s(\Delta^\beta)t}T(t))_{t \geq 0}$ converges in the operator norm to the rank-1-projection $P$ exponentially fast.

\section{Elliptic operators in divergence form}

The preceding results extend to elliptic operators in divergence form with bounded measurable coefficients. Let $\Omega \subset \R^n$ be open and bounded.

Let $a_{k,\ell}, b_k, c_k, c_0 \in L^\infty(\Omega)$, $k, \ell = 1, \ldots, n$ such that for some $\alpha > 0$
\[\sum_{k,\ell=1}^n a_{k,\ell}(x)\xi_k\xi_\ell \geq \alpha|\xi|^2\]
for all $x \in \Omega$, $\xi \in \R^n$, where $|\xi|^2 = \xi_1^2 + \cdots + \xi_n^2$.

Let $H^1_{\text{loc}}(\Omega) := \{u \in L^2_{\text{loc}}(\Omega) : \partial_k u \in L^2_{\text{loc}}(\Omega), k = 1, \ldots, n\}$.

Define $\mathcal{A} : H^1_{\text{loc}}(\Omega) \to C_c^\infty(\Omega)'$ by
\[\langle \mathcal{A}u, v \rangle = \sum_{k,\ell=1}^d \partial_k(a_{k\ell}\partial_\ell u) + \sum_{k=1}^d \partial_k(b_k u) + \sum_{k=1}^d c_k \partial_k u + c_0 u.\]

We define $A_0$ as the part of $\mathcal{A}$ in $C_0(\Omega)$; i.e.
\[\D(A_0) := \{u \in C_0(\Omega) \cap H^1_0(\Omega) : \mathcal{A}u \in C_0(\Omega)\}\]
\[A_0 u := \mathcal{A}u.\]

Then Theorem 4.1 holds with $\Delta_0$ replaced by $A_0$. It is remarkable that Dirichlet regularity of $\Omega$ is the right regularity condition at the boundary, a discovery due to Stampacchia. We refer to Arendt and Bénilan \cite{ArBe99}, Section 4 for a proof of the following result.

\begin{theorem}[4.5]
Assume that $\Omega \subset \R^n$ is a bounded, open, connected, Dirichlet regular set. Then $A_0$ generates a positive, irreducible, holomorphic semigroup $\mathcal{T} = (T(t))_{t \geq 0}$ on $C_0(\Omega)$. Moreover, $T(t)$ is compact for all $t > 0$.
\end{theorem}

\begin{remark}
The proof of holomorphy depends on Gaussian estimates, which in \cite{ArBe99} were merely known if the $b_k \in W^{1,\infty}(\Omega)$. Later, it was shown by Davies \cite{Da00} that they always hold.
\end{remark}

Also the results for Robin boundary conditions Theorem 4.3 and 4.4 can be extended to elliptic operators in divergence form on $C(\overline{\Omega})$; see Theorem 4.5 in Arendt, ter Elst, Glück \cite{AEG20}. It uses results of Nittka \cite{Ni11}.

\section{Elliptic operators in non-divergence forms}

The techniques for elliptic operators in non-divergence form are quite different than those used in the divergence-case form. But the results are similar.

Let $\Omega \subset \R^n$ be open and connected. We assume that $\Omega$ satisfies the uniform exterior cone condition. This means the following. There exists a finite, right circular cone $V$ such that for each $x \in \partial\Omega$ there exists a cone $V_x$ which is congruent to $V$ such that $V_x \cap \overline{\Omega} = \{x\}$.

Let $a_{k\ell} = a_{\ell k} \in C(\overline{\Omega})$, $b_k \in L^\infty(\Omega)$, $c \in L^\infty(\Omega)$, $c \leq 0$ such that
\[\sum_{k,\ell=1}^n a_{k\ell}(x)\xi_k\xi_\ell \geq \mu|\xi|^2\]
for all $x \in \overline{\Omega}$, $\xi \in \R^n$ and some $\mu > 0$.

For $u \in W^{2,n}_{\text{loc}}(\Omega)$ we define
\[\mathcal{A}u = \sum_{k,\ell=1}^n \partial_k a_{k\ell} \partial_\ell u + \sum_{k=1}^n b_k \partial_k u + cu.\]

Thus $\mathcal{A} : W^{2,n}_{\text{loc}}(\Omega) \to L^n_{\text{loc}}(\Omega)$ is linear. Here
\[W^{2,n}_{\text{loc}}(\Omega) := \{u \in L^n_{\text{loc}}(\Omega) : \partial_k u \in L^n_{\text{loc}}(\Omega), \partial_k \partial_\ell u \in L^n_{\text{loc}}(\Omega) \text{ for all } k, \ell = 1, \ldots, n\}.\]

We consider the operator $A$ on $C_0(\Omega)$ defined by
\[\D(A) := \{u \in C_0(\Omega) \cap W^{2,n}_{\text{loc}}(\Omega) : \mathcal{A}u \in C_0(\Omega)\}\]
\[Au := \mathcal{A}u.\]

Then the following holds.

\begin{theorem}[4.6]
The operator $A$ generates a positive, irreducible, contractive holomorphic semigroup $(T(t))_{t \geq 0}$ on $C_0(\Omega)$. Moreover
\[\|T(t)\| \leq Me^{-\varepsilon t} \quad (t \geq 0)\]
for some $\varepsilon > 0$, $M \geq 1$. The resolvent of $A$ is compact.
\end{theorem}

This result is proved by Arendt and Schätzle \cite{AS14}, Proposition 4.7. The monograph of Lunardi \cite{Lu95} is devoted to the study of holomorphic semigroups generated by elliptic operators in non-divergence form and Theorem 4.6 is a generalization of results found therein.

%\begin{thebibliography}{99}
%\bibitem{AEG20} W. Arendt, A.F.M. ter Elst, J. Glück, \emph{Positive semigroups and elliptic operators}, 2020.
%
%\bibitem{ArBa92} W. Arendt, C.J.K. Batty, \emph{Holomorphic semigroups and the geometry of Banach spaces}, 1992.
%
%\bibitem{ArBe99} W. Arendt, Ph. Bénilan, \emph{Wiener criteria and semigroups}, 1999.
%
%\bibitem{ArDa25} W. Arendt, D. Daners, \emph{Spectral theory and applications}, 2025.
%
%\bibitem{AS14} W. Arendt, R. Schätzle, \emph{Elliptic operators in non-divergence form}, 2014.
%
%\bibitem{AtE20} W. Arendt, A.F.M. ter Elst, \emph{Robin boundary conditions and semigroups}, 2020.
%
%\bibitem{AU24} W. Arendt, K. Urban, \emph{Vector-valued Laplace Transforms and Cauchy Problems}, 2024.
%
%\bibitem{Da00} E.B. Davies, \emph{Heat kernel estimates and parabolic equations}, 2000.
%
%\bibitem{En03} K.-J. Engel, \emph{Generator property and regularity}, 2003.
%
%\bibitem{Es94} J. Escher, \emph{Semigroups and boundary value problems}, 1994.
%
%\bibitem{GT83} D. Gilbarg, N.S. Trudinger, \emph{Elliptic Partial Differential Equations of Second Order}, 1983.
%
%\bibitem{Lu95} A. Lunardi, \emph{Analytic Semigroups and Optimal Regularity in Parabolic Problems}, 1995.
%
%\bibitem{Ni11} R. Nittka, \emph{Regularity of solutions and semigroups}, 2011.
%\end{thebibliography}
%
%\end{document}