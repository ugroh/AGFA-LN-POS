% !TEX root = Extended_Notes_B-II.tex
%\documentclass[11pt]{article}
%\usepackage{amsmath,amsthm,amssymb}
%\usepackage[utf8]{inputenc}
%\usepackage[T1]{fontenc}
%\usepackage{geometry}
%\geometry{a4paper,margin=2.5cm}
%
%\newtheorem{theorem}{Theorem}[section]
%\newtheorem{proposition}[theorem]{Proposition}
%\newtheorem{corollary}[theorem]{Corollary}
%\newtheorem{remark}[theorem]{Remark}
%
%% Custom commands
%\newcommand{\R}{\mathbb{R}}
%\newcommand{\CC}{\mathbb{C}}
%\newcommand{\N}{\mathbb{N}}
%\newcommand{\D}{\mathcal{D}}
%\newcommand{\dom}{\text{dom}}
%\renewcommand{\Re}{\text{Re}}
%\newcommand{\s}{\mathfrak{s}}
%\newcommand{\wo}{\omega_0}
%
%\title{Extended Notes for B-II\\
%Section 4: Positive semigroups generated by elliptic operators\\
%--- Final Part ---}
%\author{}
%\date{}
%
%\begin{document}
%\maketitle

\bigskip
\textbf{This document continues the Extended Notes for B-II, Section 4.}
\bigskip

of Lunardi's results which were obtained under the more restrictive assumption of a $C^2$-boundary. Positivity and irreducibility are a consequence of the Alexandrov maximum principle. For $C^2$-boundary also results on $L^p(\Omega)$ spaces are obtained by Denk, Hieber and Prüss \cite{DHP03} whose main interest lies in proving maximal regularity and establishing a bounded $H^\infty$-calculus.

However, in the situation of Theorem 4.6, without assuming merely the uniform exterior cone condition on $\Omega$, it seems not to be known whether the semigroup extends to a strongly continuous semigroup on $L^p(\Omega)$ for some $p \in [1,\infty)$.

Theorem 4.6 is extended by Arendt and Schätzle \cite{AS25} to unbounded open sets which satisfy the locally uniform exterior cone condition. However, in the case of unbounded $\Omega$ the semigroup converges merely strongly to $0$ (and not exponentially fast).

\section{The Dirichlet-to-Neumann operator on $C(\partial\Omega)$}

Let $\Omega$ be a bounded, open, connected subset of $\R^n$ with Lipschitz boundary and let $V \in L^\infty(\Omega)$. We consider the Dirichlet-to-Neumann operator with respect to $\Delta - V$ on the space $C(\partial\Omega)$. For that we first establish well-posedness of the Dirichlet Problem.

We assume throughout this subsection that
\begin{equation} \tag{4.2}
u \in C_0(\Omega), \Delta u - Vu = 0 \text{ implies } u = 0.
\end{equation}

This is exactly the condition that the solutions of the Dirichlet problem with respect to $\Delta - V$ formulated in Proposition 4.7 are unique. An equivalent condition is
\begin{equation} \tag{4.3}
u \in H^1_0(\Omega), \Delta u - Vu = 0 \text{ implies } u = 0;
\end{equation}
(which means that $0 \notin \sigma(\Delta_0 - V)$ where $\Delta_0$ is the Dirichlet Laplacian on $L^2(\Omega)$).

\begin{proposition}[4.7]
Assume (4.2). Let $g \in C(\partial\Omega)$. Then there exists a unique $u_g \in C(\overline{\Omega})$ such that
\[(\Delta - V)u_g = 0 \quad \text{and} \quad u_g|_{\partial\Omega} = g.\]

Thus, $u_g$ is a harmonic function with respect to $\Delta - V$ which has to be understood in the sense of distributions; i.e.,
\[\int_\Omega u_g \Delta\varphi - \int_\Omega Vu_g \varphi = 0\]
for all $\varphi \in C_c^\infty(\Omega)$.
\end{proposition}

For a simple proof of Proposition 4.7 we refer to \cite{AtE19}.

Next we define the Dirichlet-to-Neumann operator $N_V$ with respect to $\Delta - V$ on $C(\partial\Omega)$ as follows.
\begin{align}
\D(N_V) &:= \{g \in C(\partial\Omega) : u_g \in H^1(\Omega), \text{ and } \partial_\nu u_g \in C(\partial\Omega)\} \\
N_V g &:= -\partial_\nu u_g.
\end{align}

Recall that $\partial_\nu u_g \in C(\partial\Omega)$ means that there exists $h \in C(\partial\Omega)$ such that
\[\int_\Omega \Delta u_g \varphi + \int_\Omega \nabla u_g \nabla\varphi = \int_{\partial\Omega} h\varphi\]
for all $\varphi \in C^1(\overline{\Omega})$. Then we put $\partial_\nu u_g := h$.

We will need the hypothesis that $-\Delta_0 + V$ is form-positive i.e.
\begin{equation} \tag{4.4}
\int_\Omega (|\nabla u|^2 + V|u|^2) \geq 0
\end{equation}
for all $u \in H^1_0(\Omega)$.

\begin{theorem}[4.8]
Assume (4.2) and (4.4). Then $\Delta - V$ generates a positive, irreducible semigroup on $C(\partial\Omega)$. If $V \geq 0$, then the semigroup is contractive.
\end{theorem}

If $\Omega$ is of class $C^\infty$ similar results have been obtained by Escher \cite{Es94} and Engel \cite{En03}. Under the very general conditions here, Theorem 4.8 is due to Arendt and ter Elst \cite{AtE20}. There it is shown that $N_V$ is resolvent-positive and that the domain is dense (which is the main difficulty). Then by B-I, Theorem 1.8 $N_V$ generates a positive semigroup. Irreducibility is surprising. In fact, even though $\Omega$ is supposed to be connected, $\partial\Omega$ might not be connected (consider a ring for example). The fact that the semigroup is irreducible shows that the operator $N_V$ is non-local in quite a dramatic way.

A first result on irreducibility (on $L^2(\partial\Omega)$) was obtained by Arendt and Mazzeo \cite{AM12}.

It is not known so far whether the semigroup generated by $N_V$ is holomorphic if $\partial\Omega$ has Lipschitz boundary. If the boundary is of class $C^{n+\alpha}$ with $\alpha > 0$, then it is holomorphic of angle $\pi/2$. This is due to ter Elst and Ouhabaz \cite{tEO19}.

The operator $N_V$ is also called voltage-to-current map and has physical meaning. One version of the famous Calderón-Problem is the question whether for $V_1, V_2 \in L^\infty(\Omega)$, such that $0 \notin \sigma(\Delta_{V_1}) \cup \sigma(\Delta_{V_2})$,
\[N_{V_1} = N_{V_2} \text{ implies } V_1 = V_2.\]

This is true under the only assumption that $\Omega$ has Lipschitz boundary; see Theorem 1.1 by Krupchyk and Uhlmann \cite{KU14}.

Finally we mention that $N_V$ may generate a positive semigroup even if (4.4) is violated. This and other surprising phenomena were discovered by Daners \cite{Da14}, and led

%\begin{thebibliography}{99}
%\bibitem{AM12} W. Arendt, R. Mazzeo, \emph{Spectral properties of the Dirichlet-to-Neumann operator on Lipschitz domains}, Ulmer Seminare 17 (2012), 15--30.
%
%\bibitem{AS14} W. Arendt, R. Schätzle, \emph{Semigroups generated by elliptic operators in non-divergence form on $C_0(\Omega)$}, 2014.
%
%\bibitem{AS25} W. Arendt, R. Schätzle, \emph{Extension to unbounded domains}, 2025.
%
%\bibitem{AtE19} W. Arendt, A.F.M. ter Elst, \emph{The Dirichlet-to-Neumann operator via hidden compactness}, J. Funct. Anal. 266 (2014), 1757--1786.
%
%\bibitem{AtE20} W. Arendt, A.F.M. ter Elst, \emph{The Dirichlet-to-Neumann operator on rough domains}, J. Differential Equations 251 (2011), 2100--2124.
%
%\bibitem{Da14} D. Daners, \emph{Non-positivity of the semigroup generated by the Dirichlet-to-Neumann operator}, Positivity 18 (2014), 235--256.
%
%\bibitem{DHP03} R. Denk, M. Hieber, J. Prüss, \emph{Optimal $L^p$-$L^q$-estimates for parabolic boundary value problems with inhomogeneous data}, Math. Z. 257 (2007), 193--224.
%
%\bibitem{En03} K.-J. Engel, \emph{Generator property and regularity}, 2003.
%
%\bibitem{Es94} J. Escher, \emph{Semigroups and boundary value problems}, 1994.
%
%\bibitem{KU14} K. Krupchyk, G. Uhlmann, \emph{Uniqueness in an inverse boundary problem for a magnetic Schrödinger operator with a bounded magnetic potential}, Comm. Math. Phys. 327 (2014), 993--1009.
%
%\bibitem{Lu95} A. Lunardi, \emph{Analytic Semigroups and Optimal Regularity in Parabolic Problems}, Progress in Nonlinear Differential Equations and their Applications, 16. Birkhäuser Verlag, Basel, 1995.
%
%\bibitem{tEO19} A.F.M. ter Elst, E.M. Ouhabaz, \emph{Analysis of the heat kernel of the Dirichlet-to-Neumann operator}, J. Funct. Anal. 267 (2014), 4066--4109.
%\end{thebibliography}
%
%\end{document}