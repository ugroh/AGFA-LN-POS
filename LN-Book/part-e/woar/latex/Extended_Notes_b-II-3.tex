% !TEX root = Extended_Notes_B-II.tex
\subsection{The Dirichlet-to-Neumann operator on $C(\Omega)$}

Let $\Omega$ be a bounded, open, connected subset of $\R^n$ with Lipschitz boundary and let $V \in L^\infty(\Omega)$. We consider the Dirichlet-to-Neumann operator with respect to $\Delta - V$ on the space $C(\Omega)$. For that we first establish well-posedness of the Dirichlet Problem.

We assume throughout this subsection that for $u \in C_{0}(\Omega)$,
%% --
\begin{equation}\label{eq:eb2-4.1} %% \tag{4.1}
 \Delta u - Vu = 0 \text{ implies } u = 0.
\end{equation}
%% --
This is exactly the condition that the solution of the Dirichlet problem with respect to $\Delta - V$ formulated in Proposition 4.7 is unique. 
An equivalent condition is
%% --
\begin{equation*}%\label{eq:eb2-4.2}  % \tag{4.3}
	u \in H^1_0(\Omega), \text{ $\Delta u - Vu = 0$ implies $u = 0$;}
\end{equation*}
%% --
(which means that $0 \notin \sigma(\Delta_0 - V)$ where $\Delta_0$ is the Dirichlet Laplacian on $L^2(\Omega)$).
%% --
\begin{proposition}
Assume \eqref{eq:eb2-4.1} and let $g \in C(\Omega)$. 
Then there exists a unique $u_{g} \in C(\overline{\Omega})$ such that
%% --
\[(
	\Delta - V)u_{g} = 0 \quad \text{and} \quad u_{g}|_{\Omega} = g.
\]
%% --
\end{proposition}
%% --
Thus, $u_{g}$ is a harmonic function with respect to $\Delta - V$ which has to be understood in the sense of distributions; \ie
%% --
\[
	\int_{\Omega} u_{g} \Delta \phi - \int_\Omega V u_{g} \phi = 0 
\]
%%% --
for all $\phi \in C_c^\infty(\Omega)$.
For a simple proof of Proposition 4.7 we refer to \citet{AtE19}.

Next we define the \emph{Dirichlet-to-Neumann operator} $N_{V}$ with respect to $\Delta - V$ on $C(\Omega)$ as follows.
%% --
\begin{align*}
	D(N_{V}) &:= \{g \in C(\Omega)\colon u_{g} \in H^1(\Omega)  \text{ and } \partial_\nu u_{g} \in C(\Omega)\} \\
N_{V} g &:= -\partial_\nu u_{g}.
\end{align*}
%% --
Recall that $\partial_\nu u_{g} \in C(\Omega)$ means that there exists $h \in C(\Omega)$ such that
\[\int_\Omega \Delta u_{g} \phi + \int_\Omega \nabla u_{g} \nabla\phi = \int_{\Omega} h\phi\]
for all $\phi \in C^1(\overline{\Omega})$. Then we put $\partial_\nu u_{g} := h$.

We will need the hypothesis that $-\Delta_0 + V$ is \emph{form-positive} \ie
%% --
\begin{equation}\label{eq:eb2-4.2}
	\int_\Omega (|\nabla u|^2 + V|u|^2) \geq 0
\end{equation}
%% --
for all $u \in H^1_0(\Omega)$.
%% --
\begin{theorem}
Assume \eqref{eq:eb2-4.1} on page \pageref{eq:eb2-4.1} and \eqref{eq:eb2-4.2}. 
Then $- N_{V}$ generates a positive, irreducible semigroup on $C(\Omega)$. If $V \geq 0$, then the semigroup is contractive.
\end{theorem}
%% --
If $\Omega$ is of class $C^\infty$ similar results have been obtained by \citet{Es94} and \citet{En03}. 
Under the very general conditions here, Theorem 4.8 is due to \citet{AtE20}. 
There it is shown that $N_{V}$ is resolvent-positive and that the domain is dense (which is the main difficulty). 
Then by B-II, Theorem 1.8 $N_{V}$ generates a positive semigroup. 
Irreducibility is surprising. 
In fact, even though $\Omega$ is supposed to be connected, $\Omega$ might not be connected (consider a ring for example). 
The fact that the semigroup is irreducible shows that the operator $N_{V}$ is non-local in quite a dramatic way.

A first result on irreducibility (on $L^2(\Omega)$) was obtained by \citet{AM12}.

It is not known so far whether the semigroup generated by $N_{V}$ is holomorphic if $\Omega$ has Lipschitz boundary. 
If the boundary is of class $C^{1+\alpha}$ with $\alpha > 0$, then it is holomorphic of angle $\pi/2$. 
This is due to \citet{tEO19}.

The operator $N_{V}$ is also called \emph{voltage-to-current map} and has physical meaning. 
One version of the famous Calderón-Problem is the question whether for $V_1, V_2 \in L^\infty(\Omega)$, such that $0 \notin \sigma(\Delta_{V_1}) \cup \sigma(\Delta_{V_2})$,
\[N_{V_1} = N_{V_2} \text{ implies } V_1 = V_2.\]

This is true under the only assumption that $\Omega$ has Lipschitz boundary; see \citet[Theorem 1.1]{KU14}.

Finally we mention that $N_{V}$ may generate a positive semigroup even if (4.2) is violated. 
This and other surprising phenomena were discovered by  \citet{Da14}, and led to the new theory of eventually positive semigroups; see \eg \citet{DEK16}.

