% !TEX root = Extended_Notes_B-II.tex



\subsection{The Dirichlet-to-Neumann operator on $C(\partial\Omega)$}

Let $\Omega$ be a bounded, open, connected subset of $\R^n$ with Lipschitz boundary and let $V \in L^\infty(\Omega)$. We consider the Dirichlet-to-Neumann operator with respect to $\Delta - V$ on the space $C(\partial\Omega)$. For that we first establish well-posedness of the Dirichlet Problem.

We assume throughout this subsection that
\begin{equation} \tag{4.2}
u \in C_0(\Omega), \Delta u - Vu = 0 \text{ implies } u = 0.
\end{equation}

This is exactly the condition that the solutions of the Dirichlet problem with respect to $\Delta - V$ formulated in Proposition 4.7 are unique. An equivalent condition is
\begin{equation} \tag{4.3}
u \in H^1_0(\Omega), \Delta u - Vu = 0 \text{ implies } u = 0;
\end{equation}
(which means that $0 \notin \sigma(\Delta_0 - V)$ where $\Delta_0$ is the Dirichlet Laplacian on $L^2(\Omega)$).

\begin{proposition}
Assume (4.2). Let $g \in C(\partial\Omega)$. Then there exists a unique $u_g \in C(\overline{\Omega})$ such that
\[(\Delta - V)u_g = 0 \quad \text{and} \quad u_g|_{\partial\Omega} = g.\]

Thus, $u_g$ is a harmonic function with respect to $\Delta - V$ which has to be understood in the sense of distributions; i.e.,
\[\int_\Omega u_g \Delta\varphi - \int_\Omega Vu_g \varphi = 0\]
for all $\varphi \in C_c^\infty(\Omega)$.
\end{proposition}

For a simple proof of Proposition 4.7 we refer to \cite{AtE19}.

Next we define the Dirichlet-to-Neumann operator $N_V$ with respect to $\Delta - V$ on $C(\partial\Omega)$ as follows.
\begin{align}
D(N_V) &:= \{g \in C(\partial\Omega) : u_g \in H^1(\Omega), \text{ and } \partial_\nu u_g \in C(\partial\Omega)\} \\
N_V g &:= -\partial_\nu u_g.
\end{align}

Recall that $\partial_\nu u_g \in C(\partial\Omega)$ means that there exists $h \in C(\partial\Omega)$ such that
\[\int_\Omega \Delta u_g \varphi + \int_\Omega \nabla u_g \nabla\varphi = \int_{\partial\Omega} h\varphi\]
for all $\varphi \in C^1(\overline{\Omega})$. Then we put $\partial_\nu u_g := h$.

We will need the hypothesis that $-\Delta_0 + V$ is form-positive i.e.
\begin{equation} \tag{4.4}
\int_\Omega (|\nabla u|^2 + V|u|^2) \geq 0
\end{equation}
for all $u \in H^1_0(\Omega)$.

\begin{theorem}
Assume (4.2) and (4.4). Then $\Delta - N_{V}$ generates a positive, irreducible semigroup on $C(\partial\Omega)$. If $V \geq 0$, then the semigroup is contractive.
\end{theorem}

If $\Omega$ is of class $C^\infty$ similar results have been obtained by Escher \cite{Es94} and Engel \cite{En03}. 
Under the very general conditions here, Theorem 4.8 is due to Arendt and ter Elst \cite{AtE20}. There it is shown that $N_V$ is resolvent-positive and that the domain is dense (which is the main difficulty). 
Then by B-II, Theorem 1.8 $N_V$ generates a positive semigroup. 
Irreducibility is surprising. 
In fact, even though $\Omega$ is supposed to be connected, $\partial\Omega$ might not be connected (consider a ring for example). 
The fact that the semigroup is irreducible shows that the operator $N_V$ is non-local in quite a dramatic way.

A first result on irreducibility (on $L^2(\partial\Omega)$) was obtained by Arendt and Mazzeo \cite{AM12}.

It is not known so far whether the semigroup generated by $N_V$ is holomorphic if $\partial\Omega$ has Lipschitz boundary. If the boundary is of class $C^{n+\alpha}$ with $\alpha > 0$, then it is holomorphic of angle $\pi/2$. This is due to ter Elst and Ouhabaz \cite{tEO19}.

The operator $N_V$ is also called \emph{voltage-to-current map} and has physical meaning. 
One version of the famous Calderón-Problem is the question whether for $V_1, V_2 \in L^\infty(\Omega)$, such that $0 \notin \sigma(\Delta_{V_1}) \cup \sigma(\Delta_{V_2})$,
\[N_{V_1} = N_{V_2} \text{ implies } V_1 = V_2.\]

This is true under the only assumption that $\Omega$ has Lipschitz boundary; see Theorem 1.1 by Krupchyk and Uhlmann \cite{KU14}.

Finally we mention that $N_V$ may generate a positive semigroup even if (4.4) is violated. This and other surprising phenomena were discovered by Daners \cite{Da14}, and led to the new theory of eventually positive semigroups; see e.g. \cite{DEK16}.

