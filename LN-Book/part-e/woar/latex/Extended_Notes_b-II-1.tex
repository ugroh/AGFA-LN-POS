% !TEX root = Extended_Notes_B-II.tex
%% --
\section{Positive semigroups generated by elliptic operators on spaces of continuous functions}
Important examples of semigroups on $C_0(\Omega)$ or $C(\overline{\Omega})$, where $\Omega \subset \R^n$ is open and bounded, are generated by elliptic differential operators. In the following we put together a series of results starting with the Laplacian subject to Dirichlet and to Robin boundary conditions and ending with the Dirichlet-to-Neumann operator on $C(\partial\Omega)$. Each time we obtain a positive irreducible semigroup. We consider $\mathbb{K} = \R$ throughout this section.

\subsection{The Laplacian}

Let $\Omega \subset \R^d$ be open and bounded. We say that $\Omega$ is \emph{Dirichlet-regular} if for every $g \in C(\partial\Omega)$ there exists a (unique) function $u \in C(\overline{\Omega}) \cap C^2(\Omega)$ such that
\begin{align}
\Delta u &= 0 \quad \text{and} \\
u|_{\partial\Omega} &= g.
\end{align} 

This means that the Dirichlet problem is well-posed. This property is very well understood and precise characterizations in terms of barriers or of capacity are known. If $\Omega$ has Lipschitz boundary, then $\Omega$ is Dirichlet regular. In dimension $d = 2$ it suffices that $\Omega$ is simply connected.

We refer to Arendt-Urban \cite{Au23}, Section 6.9 or Gilbarg-Trudinger \cite{GT83}, Section 2.8 for further information on the Dirichlet Problem.

The Dirichlet Laplacian $\Delta_0$ on $C_0(\Omega)$ is defined by
\begin{align}
\Delta_0 u &:= \Delta u \\
D(\Delta_0) &:= \{u \in C_0(\Omega) : \Delta u \in C_0(\Omega)\}.
\end{align}

Here $\Delta u$ is to be understood in the sense of distributions.

\begin{theorem} % 4.1
The following are equivalent.

\begin{enumerate}[\upshape (a)]
\item 
$\Omega$ is Dirichlet regular;

\item  
$\Delta_0$ generates a positive semigroup $\mathcal{T}$ on $C_0(\Omega)$.

\end{enumerate}
%% -- 
In that case the semigroup $\mathcal{T}$ is holomorphic of angle $\pi/2$. 
Moreover $T(t)$ is compact for all $t > 0$.
If $\Omega$ is connected, then the semigroup is irreducible. 
Moreover,
%% --
\begin{equation}
\|T(t)\| \leq Me^{-\varepsilon t} \quad (t \geq 0)
\end{equation}
%%
for some $\varepsilon > 0$, $M \geq 1$.
\end{theorem}
%% --
This result is due to Arendt-Bénilan \cite{ArBe99} besides irreducibility on which we comment later.
In Example C-II.1.5 (e), the generation result was obtained if $\Omega$ has $C^2$-boundary.

The implication (a) $\Rightarrow$ (b) of Theorem 4.1 is proved below in order to show how the Dirichlet problem comes into play and leads to a result with minimal regularity assumptions on the boundary of $\Omega$.

We use the following abstract generation result which is of independent interest. 
By C-II, Theorem 1.2 a densely defined operator $A$ generates a contractive positive semigroup if and only if $A$ is dispersive and $(\lambda - A)$ is surjective for some $\lambda > 0$.
We now describe the case $\lambda = 0$.
%% --
\begin{theorem}
Let $A$ be a densely defined operator on a real or complex Banach lattice $E$. 
The following are equivalent.
%% --
\begin{enumerate}[\upshape (a)]
\item  
$A$ generates a positive, contractive semigroup and $\s(A) \leq 0$.
\item  
$A$ is dispersive and surjective.
\end{enumerate}
%% --
In particular, (b) implies that $A$ is closed.
\end{theorem}
%% --
Dispersive operators are defined before C-II, Theorem 1.2. A densely defined operator $A$ on $C_0(\Omega)$ is dispersive iff for $u \in D(A)$, $x_0 \in \overline{\Omega}$:
\[u(x_0) = \sup_{x \in \overline{\Omega}} u(x) > 0 \text{ implies } (Au)(x_0) \leq 0.\]

\begin{proof}(Theorem 4.2.) 
(b) $\Rightarrow$ (a)

Consider the equivalent norm
\[
\|u\|_1 := \|u^+\| + \|u^-\|
\]
on $E$. 
Since $A$ is dispersive it is dissipative with respect to this new norm as is easy to see. Now Theorem 4.5 of Arendt, Chalendar and Moletsare \cite{ACM24} implies that $A$ generates a contraction semigroup $\mathcal{T}$ and $A$ is invertible. Since $A$ is dispersive, it follows from C-II, Theorem 1.2 that $\mathcal{T}$ is positive and contractive (with respect to the original norm). Since $R(\lambda, A) \geq 0$ for $\lambda > 0$, it follows that $-A^{-1} \geq 0$. Now C-I, Theorem 1.1 (vi) implies that $s(A) \leq 0$.

(a) $\Rightarrow$ (b) is obvious from C-II, Theorem 1.2. 

\end{proof}
%% --
\begin{proof}(Theorem 4.1.)  
(a) $\Rightarrow$ (b)

The operator $\Delta_0$ is dispersive by the maximum principle. 
If $\Omega$ is Dirichlet regular, then $\Delta_0$ is surjective. In fact, let $f \in C_0(\Omega)$. Extend $f$ by $0$ to $\R^n$ and let $w = \Gamma * f$, where $\Gamma$ is the fundamental solution of Laplace's equation, see Gilbarg and Trudinger \cite{GT83}, (2.12). Then $w \in C(\R^n)$ and $\Delta w = f$ in the sense of distributions. Let $g = w|_{\partial\Omega}$ and let $v \in C^2(\overline{\Omega}) \cap C(\overline{\Omega})$ be the solution of the Dirichlet problem; i.e. $v|_{\partial\Omega} = g$ and $\Delta v = 0$ in $\Omega$. Then $u := w - v \in D(\Delta_0)$ and $\Delta u = f$.

We have shown that $\Delta_0$ satisfies condition (b) of Theorem 4.2. Thus $\Delta_0$ generates a positive, contractive $C_0$-semigroup $(T(t))_{t \geq 0}$ on $C_0(\Omega)$ and $s(\Delta_0) \leq 0$. Since by C-IV Theorem 1.1 (iv) $\s(\Delta_0) = \wo(\Delta_0)$, it is exponentially stable.

We refer to Arendt and Bénilan \cite{ArBe99} for the proof of (b) $\Rightarrow$ (a).

\end{proof}
%% --
We want to add two further comments on the Dirichlet Laplacian $\Delta_0$ on $C_0(\Omega)$. The first concerns its domain
%% --
\[
D(\Delta_0) = \{u \in C_0(\Omega) \colon \Delta u \in C_0(\Omega)\}.
\]
%% --
This distributional domain is not contained in $C^2(\Omega)$ for any open set $\Omega \subset \R^n$, $n \geq 2$, see Arendt-Urban \cite[Theorem 6.60]{Au23}. 

Our second comment concerns the proof of holomorphy. It can be given via Gaussian estimates (see the Extended Notes for C-II). In our context, a short proof based on Kato's inequality of C-II, Section 2 is more appealing (see Arendt-Batty \cite{ArBa92}).

Finally, we comment on irreducibility. 
On $C_0(\Omega)$ it is a strong property. 
By C-III, Theorem 3.2 (ii) it means that for $0 \leq f \in C_0(\Omega)$, $f \neq 0$,
%% --
\[
(T(t)f)(x) > 0 \text{ for all } x \in \Omega, t > 0.
\]
%% --
On $L^2(\Omega)$ irreducibility is much weaker (meaning that $(T(t)f)(x) > 0$ $x$-a.e.), but easy to prove (see the Extended Notes to C-I). In the paper Arendt, ter Elst, Glück \cite{AEG20} an argument based on Banach lattice technique shows how irreducibility on $L^2(\Omega)$ can be carried over to $C_0(\Omega)$ or even to $C(\overline{\Omega})$ in the case of Robin boundary conditions which we consider now.

By $H^1(\Omega) := \{u \in L^2(\Omega) : \partial_j u \in L^2(\Omega) \text{ for } j = 1, \ldots, n\}$ we denote the first Sobolev space.

We assume that $\Omega$ has Lipschitz boundary. Then there exists a unique bounded operator $\text{tr} : H^1(\Omega) \to L^2(\partial\Omega)$ such that $\text{tr }(u) = u|_{\partial\Omega}$ for all $u \in C^1(\overline{\Omega})$. It is called the \emph{trace operator}.

Here the space $L^2(\partial\Omega)$ is defined with respect to the surface measure (i.e. the $(d-1)$-dimensional Hausdorff measure) on $\partial\Omega$.

The normal derivative $\partial_\nu u$ of $u$ is defined as follows. Let $u \in H^1(\Omega)$ such that $\Delta u \in L^2(\Omega)$. Let $h \in L^2(\partial\Omega)$. We say that $h$ is the \emph{(outer) normal derivative} of $u$ and write $\partial_\nu u = h$ if
\[\int_\Omega \Delta u v + \int_\Omega \nabla u \nabla v = \int_{\partial\Omega} h v\]
for all $v \in C^1(\overline{\Omega})$.

If $u \in H^1(\Omega)$ such that $\Delta u \in L^2(\Omega)$ we say $\partial_\nu u \in L^2(\partial\Omega)$ if there exists $h \in L^2(\partial\Omega)$ such that $\partial_\nu u = h$.

\begin{remark*}
Since $\Omega$ has Lipschitz boundary the outer normal $\nu(z)$ exists for almost all $z \in \partial\Omega$ and $\nu \in L^\infty(\partial\Omega)$. But we do not use this outer normal and rather define $\partial_\nu u$ weakly by the validity of Green's formula.
\end{remark*}

Let $\beta \in L^\infty(\partial\Omega)$. We define the Laplacian $\Delta^\beta$ with Robin boundary conditions as follows:
\begin{align}
D(\Delta^\beta) &:= \{u \in H^1(\Omega) : \Delta u \in L^2(\Omega), \partial_\nu u + \beta \text{tr}(u) = 0\} \\
\Delta^\beta u &:= \Delta u.
\end{align}

We call $\Delta^\beta$ briefly the Robin-Laplacian. Note that for $\beta = 0$, we obtain Neumann boundary conditions, and $\Delta^0 =: \Delta^N$ is the Neumann Laplacian.

The following result is valid.

\begin{theorem}[4.3]
Assume that $\Omega \subset \R^d$ is bounded, open, connected with Lipschitz boundary, and let $\beta \in L^\infty(\partial\Omega)$. Then $\Delta^\beta$ generates a positive, irreducible, holomorphic semigroup $\mathcal{T} = (T(t))_{t \geq 0}$ on $C(\overline{\Omega})$. Moreover, $T(t)$ is compact for all $t > 0$.
\end{theorem}

The generation property on $C(\overline{\Omega})$ is due to Nittka \cite{Ni11}. A major point is to show that the resolvent of the corresponding operator on $L^2(\Omega)$ leaves $C(\overline{\Omega})$ invariant. Given $f \in C(\overline{\Omega})$, $u \in H^1(\Omega)$ such that $u - \Delta u = f$, $\partial_\nu u + \beta u|_{\partial\Omega} = 0$.

\begin{theorem}[4.8]
Assume (4.2) and (4.4). Then $\Delta - V$ generates a positive, irreducible semigroup on $C(\partial\Omega)$. If $V \geq 0$, then the semigroup is contractive.
\end{theorem}

If $\Omega$ is of class $C^\infty$ similar results have been obtained by Escher \cite{Es94} and Engel \cite{En03}. Under the very general conditions here, Theorem 4.8 is due to Arendt and ter Elst \cite{AtE20}. There it is shown that $N_V$ is resolvent-