% !TEX root = ../LN-Book.tex
%% -- 
%% --Stand 2025-06-06 final
%% --
%% --
\chapter[Spectral Theory on Banach Lattices]{Spectral Theory of Positive Semigroups on Banach Lattices}\label{chap:c3}
\chaptermark{Spectral Theory of Positive Semigroups}
%%%\index{Spectral Theory}
%% --
{\Large
\vspace*{-.75cm}
by \\[.25em]
Günther Greiner
\vspace{.75cm}
\\
}
%% --
In Chapter B-III we have shown that positive semigroups on spaces $C_{0}(X)$ possess several interesting spectral properties.
Now we are going to extend many of these results to the more general setting of Banach lattices.
We will improve some of them considerably and give the complete proof of B-III, Theorem~4.1.

Throughout this chapter we will assume that $E \neq \{0\}$ is a complex Banach lattice.

%% --
\section{The Spectral Bound}\label{sec:c3-1}
%%\index{Spectral Bound}
%% --
The fact that the spectral bound of a positive semigroup is always contained in the spectrum (provided that the spectrum is non-empty) is also true in the general setting of Banach lattices.
The proof given in B-III, Theorem~1.1 for spaces $C_{0}(X)$ works in the general case too.
Another proof is given below (\cf Corollary~\ref{cor:c3-1.4}).
Furthermore, Corollary~1.3 and Proposition~1.5 of B-III are true in the setting of Banach lattices and their proofs can be carried over to the general case.
For the sake of completeness we summarize these results in the following theorem.
%% --
\begin{theorem}\label{thm:c3-1.1}
%%\index{Positive Semigroups!Generator}
Let $A$ be the generator of a positive semigroup $(T(t))_{t \geq 0}$ on a Banach lattice $E$.
%% --	
	\begin{enumerate}[\upshape (i)]
		\item 
		$s(A) \in \sigma(A)$ unless $\sigma(A) = \emptyset$.
		
		\item 
		For $\lambda_{0} \in \rho(A)$ we have
		$R(\lambda_{0},A)$ is positive if and only if $\lambda_{0} > s(A)$.
		In this case $r(R(\lambda_{0},A)) = (\lambda_{0} - s(A))^{-1}$.
		
		\item 
		If $T(1)$ has a positive fixed vector $h_{0}$, then $\ker(A)$ contains a positive element $h$ such that $h_{0} \in E_{|h|}$.

		\item 
		If \ $T(1)'\phi_{0} = \phi_{0}$ \ for some \ $\phi_{0} \in E'_{+}$, then there exists $\phi \in D(A^*)_{+}$ with $\{f \in E \colon \langle |f|,\phi\rangle = 0\} \subseteq \{f \in E \colon \langle |f|,\phi_{0}\rangle = 0\}$ such that $A^*\phi = 0$.
	\end{enumerate}
\end{theorem}
%% --
The fact that $s(A)$ is always an eigenvalue of the adjoint (\cf B-III Theorem~1.6) is characteristic for spaces $C(K)$, $K$ compact, as can be seen by considering the Laplacian on $L^p(\R^n)$, where $1 < p < \infty$, or on $C_{0}(\R^n)$ (see B-III, Example~1.7).
Another result which cannot be extended to arbitrary Banach lattices is that spectral bound and growth bound coincide (\cf B-IV, Theorem~1.4); an example is given in A-III, Example~1.3.
Despite of this, the resolvent $R(\lambda,A)$ of a positive semigroup is given as the Laplace transform of the semigroup in the half-plane $\{z \in \C  \colon \Re  z > s(A)\}$ (even in case that $\omega_0(A) > s(A)$).
Note however, that the integral exists only as an improper Riemann integral.
By Datko's Theorem (A-IV, Theorem~1.11) the function $t \mapsto \mathrm{e}^{-\lambda t}T(t)f$ cannot be Bochner integrable for all $f \in E$ in the case $\Re  \lambda \leq \omega_0(A)$.
%% --
\begin{theorem}\label{thm:c3-1.2}
	%%\index{Positive Semigroups!Resolvent}
	Suppose $A$ is the generator of a positive semigroup $(T(t))_{t\geq 0}$\,.
	
	For $\Re \lambda > s(A)$ we have
	%% --
	\begin{equation}\label{eq:c3-1.1}
		R(\lambda,A)f = \lim_{t\to\infty} \int_{0}^t \mathrm{e}^{-\lambda s} T(s)f \ds  \ \text{ for all } \ f \in E.
	\end{equation}
	%% --
	Moreover, the operators $\int_{0}^t \mathrm{e}^{-\lambda s} T(s) \ds $ tend to $R(\lambda,A)$ with respect to the operator norm as $t \to \infty$.
\end{theorem}
%% --
\begin{proof}
	We fix $\lambda_{0} > \omega_0(A)$.
	Then, by A-I,  Proposition~1.11,
	%% --
	\begin{equation}\label{eq:c3-1.2}
		R(\lambda_{0},A)^{n+1}f = \frac{1}{n!} \int_{0}^{\infty} s^n \exp(-\lambda_{0}s) T(s)f \ds  \quad (n \in \N_{0}, f \in E)\,.
	\end{equation}
	%% --
	Given $\mu$ such that $s(A) < \mu < \lambda_{0}$, $f \in E_{+}$, $\phi \in E'_{+}$, then
	%% --
	\begin{equation}\label{eq:c3-1.3}
	\begin{aligned}
		\langle R(\mu,A)f,\phi\rangle &= \sum_{n=0}^{\infty} (\lambda_{0}-\mu)^n\langle R(\lambda_{0},A)^{n+1}f,\phi\rangle = \\
		&= \sum_{n=0}^{\infty} \int_{0}^{\infty} \frac{1}{n!}(\lambda_{0}-\mu)^n\exp(-\lambda_{0}s)\langle T(s)f,\phi\rangle\ds = \\
		&= \int_{0}^{\infty} \sum_{n=0}^{\infty} \frac{1}{n!}(\lambda_{0}-\mu)^n\exp(-\lambda_{0}s)\langle T(s)f,\phi\rangle\ds = \\
		&= \int_{0}^{\infty} \exp((\lambda_{0}-\mu)s)\exp(-\lambda_{0}s)\langle T(s)f,\phi\rangle\ds = \\
		&= \int_{0}^{\infty} \exp(-\mu s)\langle T(s)f,\phi\rangle\ds = \\ & = \lim_{t\to\infty}\left\langle \int_{0}^t \exp(-\mu s)T(s)f \ds,\phi\right\rangle\,.
	\end{aligned}
	\end{equation}
%% --
	Note that one can interchange summation and integration because all the integrands are positive functions.
	
	It follows from \eqref{eq:c3-1.3} at the net $(\int_{0}^t \mathrm{e}^{-\mu s}T(s) f \ds )_{t\geq 0}$ converges weakly to $R(\mu,A)f$ for $t \to \infty$.
	Because it is monotone increasing $(f \geq 0)$, we have strong convergence (see the corollary to II.Theorem~5.9 in \citet{schaefer:1974}).

If $\lambda = \mu + \im\nu$ with $\mu$, $\nu$ real and $\mu > s(A)$, we have for arbitrary $f \in E$, $\phi \in E'$
%% --
\begin{align*}
	\left|\left\langle \int_{r}^t \mathrm{e}^{-\lambda s}T(s)f \ds ,\phi\right\rangle\right| &\leq \int_{r}^t \mathrm{e}^{-\mu s}\langle T(s)|f|,|\phi|\rangle\ds \\
	\text{ hence } \left\|\int_{r}^t \mathrm{e}^{-\lambda s}T(s)f \ds \right\| &\leq \left\|\int_{r}^t \mathrm{e}^{-\mu s}T(s)|f| \ds \right\| \
\end{align*}
%% --
which shows that $\lim_{t\to\infty} \int_{0}^t \mathrm{e}^{-\lambda s}T(s)f \ds $ exists.
Thus, by A-I,\,Proposition~1.11 we have  $R(\lambda,A)f = \int_{0}^{\infty} \mathrm{e}^{-\lambda s} T(s)f \ds $\,. 

It remains to prove that the net $\left(\int_{0}^t \exp(-\mu s)T(s) \ds \right)_{t\geq 0}$ converges with respect to the operator norm.
We fix $\mu$ such that $s(A) < \mu < \Re  \lambda$.
As we have seen above, the map $s \mapsto \mathrm{e}^{-\mu s}\langle T(s)f,\phi\rangle$ is Lebesgue integrable for every $(f,\phi) \in E \times E'$, thus defining a bilinear map $b \colon E \times E' \to L^1(\R _{+})$.
Using the closed graph theorem, it is easy to see that $b$ is separately continuous, hence jointly continuous by \citet[III.Theorem~5.1]{schaefer:1966}.
Thus there is a constant $M$ such that
%% --
\begin{equation}\label{eq:c3-1.4}
	\int_{0}^{\infty} \mathrm{e}^{-\mu s}|\langle T(s)f,\phi\rangle| \ds  = \|b(f,\phi)\| \leq M\|f\|\|\phi\| \quad (f \in E, \phi \in E')\,.
\end{equation}
%% --
Given $0 \leq t < r$ and setting $\epsilon \coloneqq \Re  \lambda - \mu$ we have
%% --
\begin{align*}
	\left|\int_{t}^r \mathrm{e}^{-\lambda s}\langle T(s)f,\phi\rangle \ds \right| &\leq \int_{t}^r \exp(-(\Re \lambda-\mu)s)\mathrm{e}^{-\lambda s}|\langle T(s)f,\phi\rangle| \ds  \\
	&\leq \mathrm{e}^{-\epsilon t}\int_{t}^r \mathrm{e}^{-\lambda s}|\langle T(s)f,\phi\rangle| \ds  \leq \mathrm{e}^{-\epsilon t} M\|f\|\|\phi\|.
\end{align*}
%% --
It follows that $\left\|\int_{t}^r \mathrm{e}^{-\lambda s}T(s) \ds \right\| \leq M\mathrm{e}^{-\epsilon t}$, hence $\left(\int_{0}^t \mathrm{e}^{-\lambda s}T(s) \ds \right)_{t\geq 0}$ is a Cauchy net with respect to the operator norm.
\end{proof}
%% --
Theorem~\ref{thm:c3-1.2} has many consequences.
In particular, we can conclude that $s(A) \in \sigma(A)$ whenever $s(A) > -\infty$ (without using the analogous result for bounded operators, \cf Corollary~\ref{cor:c3-1.4} below).

In each of the following corollaries we assume that $A$ is the generator of a positive semigroup on a Banach lattice $E$.

%% --
\begin{corollary}\label{cor:c3-1.3}
If $\Re  \lambda > s(A)$, then we have
%% --
\begin{equation}\label{eq:c3-1.5}
	|R(\lambda,A)f| \leq R(\Re \lambda,A) |f| \quad (f \in E).
\end{equation}
\end{corollary}
%% --
The proof is an immediate consequence of Theorem~\ref{thm:c3-1.2}.

\begin{corollary}\label{cor:c3-1.4}
We have $s(A) \in \sigma(A)$ unless $s(A) = -\infty$.
\end{corollary}
%% --
\begin{proof}
Assume that $s(A) > -\infty$ and $s(A) \notin \sigma(A)$, then it follows from \eqref{eq:c3-1.5} that $\{R(\lambda,A) \colon \Re \lambda > s(A)\}$ is uniformly bounded in $\mathcal{L}(E)$, by $M$ say. 
Then
\[
\{z \in \C  \colon \Re  z = s(A)\} \subseteq \rho(A) \text{ and } 
\|R(z,A)\| \leq M \text{ for } z  \text{ with } \Re z = s(A)\,.
\]
It follows that $\{z \in \C  \colon |\Re  z - s(A)| < M^{-1}\} \subseteq \rho(A)$, which is absurd by the definition of $s(A)$.
\end{proof}
%% --

\begin{corollary}\label{cor:c3-1.5}
Suppose that $s(A)$ is a pole of order $m$ of the resolvent $R(\lambda,A)$.
Then $m$ is a majorant for the order of any other pole on the line $s(A) + \im\R $.
\end{corollary}
%% --
\begin{proof}
Without loss of generality we may assume that $s(A) = 0$.
By \eqref{eq:c3-1.5} we have $\|R(\epsilon+\im\beta ,A)\| \leq \|R(\epsilon,A)\|$ for every $\beta \in \R $, $\epsilon > 0$.
Therefore 
\[
\lim_{\epsilon\to 0}\|\epsilon^k R(\epsilon+\im\beta ,A)\| \leq \lim_{\epsilon\to 0}\|\epsilon^k R(\epsilon,A)\| = 0 \text{ for } k > m\,.
\]
\end{proof}
%% --
The spectrum of a positive semigroup may be empty (see B-III, Example~1.2(a)) and the spectrum of a general group may be empty as well (see \citet[Section~23.16]{hillephillips:1957}).
However, for positive groups this cannot occur.
More precisely, we have the following result
%% --
\begin{corollary}\label{cor:c3-1.6}
If $A$ is the generator of a positive group, then $\sigma(A)\cap\R  \neq \emptyset$\,.
\end{corollary}
%% --
\begin{proof}
Both $A$ and $-A$ are generators of positive semigroups, hence if $\sigma(A) = \emptyset$, then $s(A) = s(-A) = -\infty$ and \eqref{eq:c3-1.5} implies that 
\[
\{R(\lambda,A) \colon \Re  \lambda \geq 0\} \cup \{R(\lambda,-A) \colon \Re  \lambda \geq 0\}
\]
is bounded in $\mathcal{L}(E)$, \ie  $\{R(\lambda,A) \colon \lambda \in \C \}$ is bounded.
By Liouville's Theorem the function $\lambda \mapsto R(\lambda,A)$ is constant, hence identically zero because $\displaystyle \lim_{\lambda\to\infty}R(\lambda,A) = 0$.
Thus we arrive at a contradiction.
\end{proof}
%% --
We conclude this section by indicating possible extensions and further consequences of the results stated above.

%% --
\begin{remarks}\label{rem:c3-1.7}
\begin{enumerate}[\upshape (i), wide, labelindent=.5em]
	\item 
	Many of the results of this section remain true for positive semigroups on ordered Banach spaces more general than Banach lattices.
	The interested reader is referred to \citet{greinervoigtwolff:1981}.
	
	\item 
	From Theorem~\ref{thm:c3-1.2} one can easily deduce that for positive semigroups on $L^1$-spaces spectral bound and growth bound coincide.
	To prove the analogous result for $L^2$-spaces one makes use of Corollary~\ref{cor:c3-1.3}.
	For details we refer to C-IV, Theorem~1.1.
\end{enumerate}
\end{remarks}
%% --

\section{The Boundary Spectrum}\label{sec:c3-2}
%%\index{Boundary Spectrum}
%%\index{Spectrum!Boundary}
%% --
In Chapter B-III we have seen that under suitable assumptions the boundary spectrum $\sigma_{b}(A)$, consisting of all spectral values with maximal real part is a cyclic set (\cf B-III, Definition~2.5).
In the main theorem of this section we prove a result which is more general and which is true for arbitrary Banach lattices.

We first want to extend some of the notions used in B-III to the more general setting considered here.
If $E$ is a Banach lattice and $f,g \in E$ such that $g \in E_{|f|}$, then $(\sign  f)g$ is well-defined (\cf Section~8 of C-I).
Thus the following definition makes sense
%% --
\begin{definition}\label{def:c3-2.1}
	%%\index{Banach Lattices!Recursive Definition}
	If $E$ is a Banach lattice then for $f \in E$, $n \in \Z$ we define $f^{[n]}$ recursively as follows
	%% --
	\begin{equation}\label{eq:c3-2.1}
	\begin{aligned}
		f^{[0]} &\coloneqq |f|\,, \\
		f^{[n]} &\coloneqq (\sign  f)f^{[n-1]} \text{ if } n > 0\,,\\
		f^{[n]} &\coloneqq (\sign  \overline{f})f^{[n+1]} \text{ if } n < 0\,.
	\end{aligned}
	\end{equation}
\end{definition}
%% --
Obviously, for $E = C_{0}(X)$ this amounts to the same as B-III, Definition~2.2.
Moreover, in case $E$ is an $L^p$-space, then $f^{[n]}$ is the function given by
%% --
\begin{equation}\label{eq:c3-2.2}
	f^{[n]}(x) = \begin{cases}
		(f(x)/|f(x)|)^{n-1}f(x) & \text{if } f(x) \neq 0\,, \\
		0 & \text{otherwise}.
	\end{cases}
\end{equation}
%% --
The following properties are immediate consequences Definition~\ref{def:c3-2.1}
%% --
\begin{equation}\label{eq:c3-2.3}
	f^{[0]} = |f|, f^{[1]} = f, f^{[-1]} = \overline{f}, |f^{[n]}| = |f| \quad (n \in \Z)
\end{equation}
%% --
\begin{equation}\label{eq:c3-2.4}
	f^{[n]} = (\sign  f)f^{[n-1]} = (\sign  \overline{f})f^{[n+1]} \text{ for all } n \in \Z;
\end{equation}
%% --
\begin{equation}\label{eq:c3-2.5}
	(\alpha f)^{[n]} = \alpha(\alpha/|\alpha|)^{n-1}f^{[n]} \text{ for } n \in \Z, \alpha \in \C , \alpha \neq 0.
\end{equation}
%% --
Next we show that B-III, Theorem~2.4 is true for arbitrary Banach lattices.
For definition and simple properties of the signum operator $S_{h}$ see C-I, Section~8.

%% --
\begin{theorem}\label{thm:c3-2.2}
	%%\index{Positive Semigroups!Generator Properties}
	Let $(T(t))_{t\geq 0}$ be a positive semigroup on a Banach lattice $E$ with generator $A$ and suppose that for $h \in E$, $\alpha,\beta \in \R $ we have
	%% --
	\begin{equation}\label{eq:c3-2.6}
		Ah = (\alpha+\im\beta )h, \quad A|h| = \alpha|h|.
	\end{equation}
Then the following holds true
%% --
\begin{equation}\label{eq:c3-2.7}
	Ah^{[n]} = (\alpha+\im n\beta)h^{[n]} \text{ for all } n \in \Z\,.
\end{equation}
%% --
In case $|h|$ is a quasi-interior point of $E_{+}$, then 
%% --
\[
S_{h}D(A) = D(A) \text{ and } A + \im\beta \Id = S_{h}^{-1}AS_{h}\,.
\]
%% --
\end{theorem}
%% --
\begin{proof}
	Without loss of generality we may assume that $\alpha = 0$.
	Then the assumption \eqref{eq:c3-2.6} implies that $T(t)|h| = |h|$ and $T(t)h = \mathrm{e}^{\im\beta t}h$ for $t \geq 0$ (see A-III, Corollary~6.4).
	In particular, the principal ideal $E_{|h|}$ is invariant under every operator $T(t)$.
	By the Kakutani-Krein Theorem (C-I, Section~4) we can identify $E_{|h|}$ with a space $C(K)$, $K$ compact.
	Then the restrictions $\overline{T}(t) \coloneq T(t)|_{E_{|h|}}$ are positive operators on $C(K)$ satisfying $\overline{T}(t)|\overline{h}| = |\overline{h}|$ and $\overline{T}(t)\overline{h} = \mathrm{e}^{\im\beta  t}\overline{h}$.
	
	From B-III, Theorem~2.4(i) we conclude $\overline{T}(t)\overline{h}^{[n]} = \mathrm{e}^{\im\beta  t}\overline{h}^{[n]}$ for all $t \geq 0$, $n \in \Z$.
	Translating this back to $T(t)$ and $E$ this means precisely $T(t)h^{[n]} = \mathrm{e}^{\im n\beta}h^{[n]}$, hence $h^{[n]} \in D(A)$ and $Ah^{[n]} = \im  n\beta h^{[n]}$.
	
	Moreover, by B-III, Theorem~2.4(i) we have $\mathrm{e}^{\im\beta  t}\overline{T}(t) = S_{h}^{-1}\overline{T}(t)S_{h}$.
	If $|h|$ is a quasi-interior point, this relation extends by continuity from the dense subspace $E_{|h|}$ to the whole space $E$, \ie  we have $\mathrm{e}^{\im\beta  t}T(t) = S_{h}^{-1}T(t)S_{h}$ for all $t \geq 0$.
\end{proof}
%% --
In the proof above we could not apply assertion (ii) of B-III, Theorem~2.4 because the semigroup $(\overline{T}(t))$ on $E_{|h|} \cong C(K)$ need not be strongly continuous with respect to the sup-norm.

As a first application of Theorem~\ref{thm:c3-2.2} we prove a cyclicity result for the point spectrum of contraction semigroups on a class of Banach lattices which includes the $L^p$-spaces.

%% --
\begin{corollary}\label{cor:c3-2.3}
	Suppose $E$ is a Banach lattice such that the norm is strictly monotone on $E_{+}$ (\ie  $0 \leq f < g \Rightarrow \|f\| < \|g\|$).
	If $(T(t))$ is a positive contraction semigroup on $E$ with $s(A) = 0$, then $P\sigma_{b}(A) = P\sigma(A) \cap \im\R $ is imaginary additively cyclic.
\end{corollary}
%% --
\begin{proof}
	Suppose that $Ah = \im\beta  h$ $(\beta \in \R , h \in E)$.
	Then we have $T(t)h = \mathrm{e}^{\im\beta  t}h$ $(t \geq 0)$ and $|h| = |T(t)h| \leq T(t)|h|$ since $T(t)$ is positive.
	Moreover, $\|h\| \leq \|T(t)|h|\| \leq \|h\|$ since $\|T(t)\| \leq 1$.
	The assumption on the norm of $E$ implies that $T(t)|h| = |h|$ for all $t \geq 0$, equivalently $A|h| = 0$.
	Now we can apply Theorem~\ref{thm:c3-2.2} in order to obtain the desired result.
\end{proof}
%% --
A more general result on cyclicity of the eigenvalues in the boundary spectrum will be proved in Corollary~\ref{cor:c3-4.3}.
In the remaining part of this section we focus our interest on the entire boundary spectrum.
We will prove that it is cyclic provided that the resolvent $R(\lambda,A)$ does not grow too fast as $\lambda \to s(A)$.
We start with some preparations.
An important tool in the proof are pseudo-resolvents.
%% --
\begin{definition}\label{def:c3-2.4}
	%%\index{Pseudo-resolvent}
	Let $D$ be an open (non-empty) subset of $\C $ and let $G$ be a Banach space.
	A mapping $R \colon D \to \mathcal{L}(G)$ which satisfies
	%% --
	\begin{equation}\label{eq:c3-2.8}
		R(\lambda) - R(\mu) = -(\lambda - \mu)R(\lambda)R(\mu) \quad (\lambda,\mu \in D)
	\end{equation}
	%% --
	is called a \emph{pseudo-resolvent} on $G$.
\end{definition}
%% --
An equivalent (often quite useful) version of \eqref{eq:c3-2.8} is the following
%% --
\begin{equation}\label{eq:c3-2.9}
	(1 - (\lambda - \mu)R(\lambda))(1 - (\mu - \lambda)R(\mu)) = 1 \quad (\lambda,\mu \in D)
\end{equation}
%% --
Obviously, the resolvent of a closed linear operator $A$ on $G$ is a pseudo-resolvent on $D = \rho(A)$.
In general a pseudo-resolvent need not be the resolvent of an operator.
Further information can be found in \citet{hillephillips:1957}, \citet{pazy:1983} or \citet{yosida:1965}.
For our purposes the following examples are of particular interest

%% --
\begin{example}\label{ex:c3-2.5}
	\begin{enumerate}[\upshape (i), wide, labelindent=.5em]
		\item 
		Suppose $A$ is a densely defined linear operator on $G$ with $\rho(A) \neq \emptyset$ and let $G_{\F}$ be an F-product of $G$ (\cf A-I, Section~3.7).
		Then the canonical extensions $R(\lambda,A)_{\F}$ of $R(\lambda,A)$ form a pseudo-resolvent $R_{\F}$ on $G_{\F}$ with $\rho(A)$ as domain of definition.
		If $A$ is unbounded, then $0 \in A\sigma(R(\lambda,A))$ hence $0 \in P\sigma(R_{\F}(\lambda,A))$ (\cf A-III, Section~4.2).
		It follows that $R_{\F}$ is not the resolvent of an operator on $G$.
		
		\item 
		If $\{R(\lambda)\}_{\lambda\in D}$ is a pseudo-resolvent on $G$, then $\{R(\lambda)'\}_{\lambda\in D}$ is a pseudo-resolvent on $G'$.
		Moreover, if $H$ is a closed linear subspace of $G$ which is $\{R(\lambda)\}_{\lambda\in D}$-invariant $(R(\lambda)H \subset H$ for all $\lambda \in D)$, then the operators on $H$ and $G/H$ induced by $R(\lambda)$ in the canonical way form a pseudo-resolvent on $H$ and $G/H$, respectively.
	\end{enumerate}
\end{example}
%% --
In the following we list several simple properties.
We assume that $R \colon D \to \mathcal{L}(G)$ is a pseudo-resolvent on a Banach space $G$
%% --
\begin{equation}\label{eq:c3-2.10}
	\begin{aligned}
	&\text{Given  $\lambda_{0} \in D, \mu \in \C $  there exists at most one operator  $S \in \mathcal{L}(G)$ }\\
	&\text{ such that \quad $R(\lambda_{0}) - S = -(\lambda_{0}-\mu)R(\lambda_{0})S = -(\lambda_{0}-\mu)SR(\lambda_{0})$}\,.\\
	&\text{ In case such an operator exists w,e have} \\
	&\text{$R(\lambda) - S = -(\lambda-\mu)R(\lambda)S = -(\lambda-\mu)SR(\lambda)$ for all $\lambda \in D$}\,.
	\end{aligned}
\end{equation}
%% -- 
\begin{equation}\label{eq:c3-2.11}
	\begin{aligned}
	&\text{Given  $\lambda_{0} \in D$,  then for  $\mu \in D$  with  $|\mu-\lambda_{0}| < \|R(\lambda_{0})\|^{-1}$}\\ 
	&\text{ we have $ R(\mu) = \sum_{n=0}^{\infty} (\lambda_{0}-\mu)^n R(\lambda_{0})^{n+1}$ }\,.
	\end{aligned}
\end{equation}
%% --
%% --
\begin{equation}\label{eq:c3-2.12}
	\begin{aligned}
	&\text{The map $\lambda \mapsto R(\lambda)$ is a locally holomorphic function defined on  $D \subseteq \C $}\\ 
	&\text{ with values in  $\mathcal{L}(G)$}\,.
	\end{aligned}
\end{equation}
%% --
We only sketch the proof of these assertions. \eqref{eq:c3-2.12} follows from \eqref{eq:c3-2.11} and the latter is a consequence of \eqref{eq:c3-2.10}.
The identity stated in \eqref{eq:c3-2.10} can be rewritten as follows
%% --
\[
(\Id-(\lambda_{0}-\mu)R(\lambda_{0}))(\Id-(\mu-\lambda_{0})S) = \Id = (\Id-(\mu-\lambda_{0})S)(\Id-(\lambda_{0}-\mu)R(\lambda_{0}))\,.
\]
%% --
Thus $S = (\mu-\lambda_{0})^{-1}(1 - (1-(\lambda_{0}-\mu)R(\lambda_{0}))^{-1})$ has to be unique.

It follows from \eqref{eq:c3-2.11} and \eqref{eq:c3-2.12} that every pseudo-resolvent has a unique maximal extension.
Further properties of pseudo-resolvents are given in the following two propositions.

%% --
\begin{proposition}\label{prop:c3-2.6}
	Suppose $G$ is a Banach space, $D \subseteq \C $ and $R \colon D \to \mathcal{L}(G)$ is a pseudo-resolvent.
	
	\begin{enumerate}[\upshape (i)]
		\item Given $\alpha \in \C $, $x \in G$ one has $(\lambda-\alpha)R(\lambda)x = x$ either for all $\lambda \in D$ or for none.
		
		\item Suppose $\mu \in \overline{D}\backslash D$. Then $R$ can be extended to an open set containing $\mu$ if and only if there exists a sequence $(\lambda_{n}) \subset D$ converging to $\mu$ such that $\|R(\lambda_{n})\|$ is bounded.
	\end{enumerate}
\end{proposition}
%% --
\begin{proof}
	\begin{enumerate}[\upshape (i), wide, labelindent=.5em]
		\item 
        Suppose that $(\lambda-\alpha)R(\lambda)x = x$ for some fixed $\lambda \in D$, $x \in G$.
		Then using \eqref{eq:c3-2.8} we have for $\mu \in D$ 
        \[
        (\mu-\alpha)(\mu-\lambda)R(\mu)R(\lambda)x = (\lambda-\alpha)(R(\lambda)x - R(\mu)x) = x - (\lambda-\alpha)R(\mu)x\,.
        \]
		It follows that $(\mu-\alpha)R(\mu)x = x$ for all $\mu \in D$.
		
		\item 
        If there exists an extension, then $\|R(\lambda_{n})\|$ is bounded for every sequence $(\lambda_{n})$ converging to $\mu$ by \eqref{eq:c3-2.12}.
		On the other hand assuming that $\|R(\lambda_{n})\|$ is bounded by $M$ for a fixed sequence $(\lambda_{n}) \subset D$ with $\lambda_{n} \to \mu$ $(M \geq 0)$, we have
		\[
		\|R(\lambda_{n}) - R(\lambda_{m})\| = |\lambda_{n}-\lambda_{m}| \|R(\lambda_{n})R(\lambda_{m})\| \leq M^2|\lambda_{n}-\lambda_{m}|
        \]
        which shows that $(R(\lambda_{n}))$ is a Cauchy sequence in $\mathcal{L}(G)$, hence $S \coloneq \lim_{n\to\infty}R(\lambda_{n})$ exists.
		The assertion now follows from \eqref{eq:c3-2.10} and \eqref{eq:c3-2.11}.
	\end{enumerate}
\end{proof}
%% --
In the next proposition, we consider a positive pseudo-resolvent $R$ on a Banach lattice $E$, \ie  we assume that the domain $D$ of $R$ contains the positive real axis and that $R(\mu)$ is a positive operator for every $\mu > 0$.
Applying Pringsheim's Theorem (see Theorem~2.1 in the appendix of \citet{schaefer:1966}) to the expansion given in \eqref{eq:c3-2.11} one can conclude that $R$ has an extension to the half-plane $\{z \in \C  \colon \Re  z > 0\}$.

This shows that without loss of generality one can assume that the domain of a positive pseudo-resolvent contains the half-plane $\{z \in \C  \colon \Re  z > 0\}$.
%% --
\begin{proposition}\label{prop:c3-2.7}
	%%\index{Pseudo-resolvent!Positive}
	%%\index{Banach lattice!Pseudo-resolvent}
	%%\index{Examples!Positive pseudo-resolvent}
	Suppose $R \colon \Delta \to \mathcal{L}(E)$ is a positive pseudo-resolvent on a Banach lattice $E$ and $\Delta \coloneqq \{z \in \C  \colon \Re  z > 0\}$.
	
	If for some $\beta \in \R $, $h \in E$ we have
	$\lambda R(\lambda+\im\beta)h = h$ and $\lambda R(\lambda)|h| = |h|$ $(\lambda \in \Delta)$, then
	\[
    \lambda R(\lambda+\im n\beta)h^{[n]} = h^{[n]} \text{ for all } n \in \Z, \lambda \in \Delta\,.
    \]
\end{proposition}
%% --
\begin{proof}
At first we prove the following domination property which is the extension of \eqref{eq:c3-1.5} to positive pseudo-resolvents
%% --
\begin{equation}\label{eq:c3-2.13}
	|R(\lambda)f| \leq R(\Re  \lambda) |f| \ \text{ for every } \ \lambda \in \Delta , f \in E\,.
\end{equation}
%% --
To do this we fix $\lambda \in \Delta$.
Then there exists $r_{0} > 0$ such that $|r-\lambda| < r$ whenever $r > r_{0}$.
Therefore $R(\lambda) = \sum_{n=0}^{\infty}(r-\lambda)^{n}R(r)^{n+1}$ for $r > r_{0}$, which implies for $f \in E$
%% --
\begin{align*}
	|R(\lambda)f| &\leq \sum_{n=0}^{\infty}|r-\lambda|^{n}R(r)^{n+1}|f| = \\
	& =\sum_{n=0}^{\infty}(r - (r-|r-\lambda|))^{n}R(r)^{n+1}|f| = R(r - |\lambda-r|) |f|\,.
\end{align*}
%% --
Since $\lim_{r \to \infty} (r - |\lambda-r|) = \Re  \lambda$, we obtain \eqref{eq:c3-2.13}.

As a consequence of \eqref{eq:c3-2.13} and the assumption $rR(r) |h| = |h|$ we have that the principal ideal $E_{|h|}$ is $\{R(\lambda)\}_{\lambda \in \Delta}$-invariant.
Identifying, according to the Kakutani-Krein Theorem, $E_{|h|}$ with a space $C(K)$, $K$ compact, and by restricting the operators $R(\lambda)$ to $E_{|h|} \cong C(K)$, we obtain a positive pseudo-resolvent $\tilde{R} \colon \Delta \to \mathcal{L}(C(K))$.
Then we have for every $\alpha > 0$ and $f \in E$
%% --
\[
\alpha\tilde{R}(\alpha+\im\beta)h = h, \alpha\tilde{R}(\alpha) |h| = |h| = \1_{K}, \alpha|\tilde{R}(\alpha+\im\beta)f| \leq \alpha\tilde{R}(\alpha) |f|\,.
\]
%% --
Applying B-III, Lemma 2.3 we obtain $\tilde{R}(\alpha) = S_{h}^{-1}\tilde{R}(\alpha + \im\beta)S_{h}$ for every $\alpha > 0$ and using the uniqueness theorem for holomorphic functions we obtain
%% --
\begin{equation}\label{eq:c3-2.14}
\tilde{R}(z) = S_{h}^{-n}\tilde{R}(z + \im n\beta)S_{h}^{n} \ \text{ for all } \ z \in \Delta \ \text{ and } \ n \in \Z\,.
\end{equation}
In particular, $S_{h}^{n}|h| = S_{h}^{-n}z\tilde{R}(z) |h| = z\tilde{R}(z+\im n\beta)S_{h}^{n}|h|$.

In terms of the initial space this means precisely $h^{[n]} = zR(z+\im n\beta)h^{[n]}$, and the proposition is proved.
\end{proof}
%% --
We will prove cyclicity of the boundary spectrum under a \emph{growth condition} which is stated in the following definition.

\begin{definition}\label{def:c3-2.8}
	%\index{Generator!Positive semigroup}
	%\index{Semigroup!Spectral bound}
	%\index{Examples!Slowly growing resolvent}
	Let $A$ be the generator of a positive semigroup $(T(t))_{t \geq 0}$ with spectral bound $s(A) > -\infty$.
	The resolvent is said to \emph{grow slowly} if one of the following (equivalent) conditions is satisfied.
	\begin{equation}\label{eq:c3-2.15}
	\begin{aligned}
	\text{(a)}\quad & \text{$\{(\lambda-s(A))R(\lambda,A) \colon \lambda > s(A)\}$ is bounded in $\mathcal{L}(E)$}\,.\\
	\text{(b)}\quad & \text{$\left\{\frac{1}{t} \int_{0}^{t} \exp(-\tau s(A))T(\tau) \diff{\tau} \colon t > 0\right\}$ is bounded in $\mathcal{L}(E)$}\,.
	\end{aligned}
	\end{equation}
\end{definition}
%% --
Before proving the equivalence of the two assertions we make some remarks.
\begin{remarks*}
\begin{enumerate}[\upshape (i), wide, labelindent=0.5em]
    \item 
    Since one always has \quad $\lambda R(\lambda,A) \to \Id$\quad for \quad $\lambda \to \infty$, 
    the set \quad $\{(\lambda-s(A))R(\lambda,A) \colon \lambda > s(A)+\epsilon\}$ is bounded for every $\epsilon > 0$. 
    Thus in \eqref{eq:c3-2.15}(a) the essential fact is boundedness near $s(A)$. 
    On the other hand, the set of operators $\{\frac{1}{t} \int_{0}^{t} \exp(-\tau s(A))T(\tau) \diff{\tau} \colon 0 \leq t \leq t_{0} \}$ is bounded for every $t_{0} \geq 0$, hence in \eqref{eq:c3-2.15}(b) the boundedness for $t \to \infty$ is important.

    \item 
    By Corollary~\ref{cor:c3-1.4} we have $\|R(\lambda,A)\| \geq r(R(\lambda,A)) = (\lambda-s(A))^{-1}$. 
    Hence $\|R(\lambda,A)\|$ grows at least as fast as $(\lambda-s(A))^{-1}$. 
    Thus if \eqref{eq:c3-2.15}(a) is satisfied, the resolvent grows as slowly as it possibly can for $\lambda \to s(A)$.

    \item 
    We do not assume in Definition~\ref{def:c3-2.8} that spectral bound and growth bound coincide.
    A slight modification of A-III, Example 1.3 shows that there are semigroups with slowly growing resolvent and $s(A) < \omega_0(A)$.
\end{enumerate}
\end{remarks*}
%% --
\begin{proof} 
To prove equivalence of \eqref{eq:c3-2.15}(a) and \eqref{eq:c3-2.15}(b), we assume $s(A) = 0$ and write $C(t) \coloneqq \frac{1}{t} \int_{0}^{t} T(\tau) \diff{\tau}$.

\eqref{eq:c3-2.15}(a) $\Rightarrow$ \eqref{eq:c3-2.15}(b)\quad 
Consider $\lambda > 0$, $t > 0$ such that $\lambda t = 1$. 
Then we have
%% --
\[
\lambda \cdot \mathrm{e}^{-\lambda s} \geq 
\begin{cases}
	(\mathrm{e}t)^{-1} & \text{if } s \leq t\,, \\
	0 & \text{if } s > t\,.
\end{cases}
\]
%% --
Now \eqref{eq:c3-1.1} implies 
\[\textstyle
\lambda R(\lambda,A) = \int_{0}^{\infty} \lambda \exp(-\lambda s)T(s) \ds \geq \mathrm{e}^{-1}C(t) = \mathrm{e}^{-1}\cdot C(\frac{1}{\lambda}) \geq 0\,.
\]
Thus $C(t)$ is bounded for $t \to \infty$ whenever $\lambda R(\lambda,A)$ is bounded for $\lambda \downarrow 0$.

\eqref{eq:c3-2.15}(b) $\Rightarrow$ \eqref{eq:c3-2.15}(a)\quad
 Let $M \coloneqq \sup\{\|C(t)\| \colon t > 0\}$.
Given $f \in E$, $\lambda > 0$ and $r > 0$ then integration by parts yields
%% --
\[
	\lambda \int_{0}^{r} \mathrm{e}^{-\lambda s}T(s)f \ds = \lambda \mathrm{e}^{-\lambda r} \int_{0}^{r} T(\sigma)f \diff{\sigma} + \lambda^{2}\int_{0}^{r} s\mathrm{e}^{-\lambda s}\left(\frac{1}{s}\int_{0}^{s} T(\sigma)f \diff{\sigma}\right)\ds\,.
\]
It follows that
\begin{align*}
	\left\|\lambda \int_{0}^{r} \mathrm{e}^{-\lambda s}T(s)f \ds\right\| &\leq \left(r\lambda \mathrm{e}^{-r} + \lambda^{2}\int_{0}^{r} s\mathrm{e}^{-\lambda s} \ds\right)M\|f\| \\
	&= (1 - \mathrm{e}^{-\lambda r})M\|f\|\,.
\end{align*}
%% --

Letting $r \to \infty$, we obtain by \eqref{eq:c3-1.1} $\|\lambda R(\lambda,A)f\| \leq M\|f\|$ $(f \in E, \lambda > 0)$ hence $\|\lambda R(\lambda,A)\| \leq M$
\end{proof}
%% -- 
Two sufficient conditions for a resolvent to grow slowly are stated in the following proposition.
Its simple proof is omitted.
%% --
\begin{proposition}\label{prop:c3-2.9}
	%\index{Resolvent!Slow growth}
	%\index{Semigroup!Resolvent growth}
	%\index{Examples!Slowly growing resolvent}
	Suppose $(T(t))_{t\geq 0}$ is a positive semigroup with generator $A$.
	Each of the following conditions guarantees that the resolvent grows slowly.
	\begin{enumerate}[\upshape (i)]
		\item 
		$(T(t))_{t\geq 0}$ is bounded and $s(A) = 0$;
	
		\item 
		$s(A)$ is a first order pole of the resolvent.
	\end{enumerate}
\end{proposition}
%% --
In case $s(A)$ is a pole of order greater than $1$, the resolvent does not grow slowly.
We will treat this case in Corollary~\ref{cor:c3-2.12}
%% --
\begin{theorem}\label{thm:c3-2.10}
	%\index{Spectrum!Boundary}
	%\index{Semigroup!Cyclic boundary spectrum}
	The boundary spectrum of a positive semigroup with slowly growing resolvent is cyclic.
\end{theorem}
%% --
\begin{proof}
Without loss of generality we can assume that $s(A) = 0$.

Given $\im\beta \in \sigma(A)$ $(\beta \in \R )$, then $\im\beta \in A\sigma(A)$ (A-III, Proposition~2.2) and $(\lambda-\im\beta)^{-1} \in A\sigma(R(\lambda,A))$ (A-III, Proposition~2.5).
We consider an $\mathcal{F}$-product $E_{\mathcal{F}}$ of $E$ and for convenience write $E_{1}$ instead of $E_{\mathcal{F}}$.
The canonical extensions of $R(\lambda,A)$ to $E_{1}$ form a positive pseudo-resolvent $\{(R_{1}(\lambda)\}_{\Re \lambda>0}$ on $E_{1}$.
By Proposition~\ref{prop:c3-2.6}(i) and A-III,4.5 there exists $h_{1} \in E_{1}$, $h_{1} \neq 0$ such that
%% --
\begin{equation}\textstyle\label{eq:c3-2.16}
	\lambda R_{1}(\lambda+\im\beta)h_{1} = h_{1} \ \text{ for }  \Re \lambda > 0\,.
\end{equation}
%% --
By \eqref{eq:c3-2.13} we have
%% --
\begin{equation}\label{eq:c3-2.17}\textstyle
	|h_{1}| = |\tau R_{1}(\tau+\im\beta)h_{1}| \leq \tau R_{1}(\tau)|h_{1}| \ (r > 0)\,.
\end{equation}
%% --
Next, we choose any $\phi \in E_{1}'$ such that $\langle  h_{1},\phi \rangle \neq 0$.
Since $\|R_{1}(\lambda)'\| = \|R_{1}(\lambda)\| = \|R(\lambda,A)\|$, the assumption of slow growth implies that $\{\lambda R_{1}(\lambda)'\phi \colon \lambda > 0\}$ is bounded in $E_{1}'$, hence $\sigma(E_{1}',E_{1})$-relatively compact by \emph{Alaoglu's Theorem}.
Thus, there exist $\phi_{1} \in \cap_{\epsilon>0}\{\tau R_{1}(\tau)'\phi \colon 0<\tau<\epsilon\}^{-\sigma(E'_1,E_1)}$.

Using the resolvent equation \eqref{eq:c3-2.8} we obtain for $r > 0$, $\epsilon > 0$
%% --
\[
(1 - \tau R_{1}(\tau)')\epsilon R_{1}(\epsilon)'\phi = \epsilon(\tau-\epsilon)^{-1}(\tau R_{1}(\tau)'\phi - \epsilon R_{1}(\epsilon)'\phi)\,.
\]
%% --
Since the right hand side tends to $0$ as $\epsilon \to 0$, we have $(1 - \tau R_{1}(\tau)')\phi_{1} = 0$ or
%% --
\begin{equation}\label{eq:c3-2.18}
\lambda R_{1}(\lambda)'\phi_{1} = \phi_{1} \ \text{ for } \Re\lambda > 0\,.
\end{equation} 
%% --
Moreover, from 
%% --
\[
0 < |\langle h_{1},\phi \rangle| \leq \langle |h_{1}|,|\phi| \rangle \leq \langle R_{1}(\tau)|h_{1}|,|\phi| \rangle = \langle |h_{1}|,\tau R_{1}(\tau)'\phi \rangle 
\]
%% --
it follows that
%% --
\begin{equation}\label{eq:c3-2.19}
		\langle |h_{1}|,\phi_{1} \rangle > 0\,.
\end{equation}
for arbitrary $f_{1} \in E_{1}$, $\Re  \lambda > 0$ we have $\langle |R_{1}(\lambda)f_{1}|,\phi_{1} \rangle \leq \langle R_{1}(\Re \lambda)|f_{1}|,\phi_{1} \rangle = \langle |f_{1}|,R_{1}(\Re \lambda)'\phi_{1} \rangle = (\Re \lambda)^{-1}\langle |f_{1}|,\phi_{1} \rangle$.

Therefore the ideal $I \coloneqq \{f_{1} \in E_{1} \colon \langle |f_{1}|,\phi_{1} \rangle = 0\}$ is invariant under $\{(R_{1}(\lambda)\}_{\Re \lambda>0}$.
Furthermore we have (see \eqref{eq:c3-2.17}, \eqref{eq:c3-2.18}) 
\begin{align*}
	\langle |rR_{1}(r)|h_{1}| - |h_{1}||,\phi_{1} \rangle &= \langle rR_{1}(r)|h_{1}| - |h_{1}|,\phi_{1} \rangle = \\
	& = \langle |h_{1}|,rR_{1}(r)'\phi_{1} - \phi_{1} \rangle = 0\ 
\end{align*}	
which implies
\begin{equation}\label{eq:c3-2.20}
rR_{1}(r)|h_{1}| - |h_{1}| \in I \ \text{ for } \ r > 0)\,.
\end{equation}
Denote by $E_{2}$ the quotient space $E_{1}/I$ and by $\{(R_{2}(\lambda)\}_{\Re \lambda>0}$ the pseudo-resolvent on $E_{2}$ induced by $\{(R_{1}(\lambda)\}_{\Re \lambda>0}$ in the canonical way. 
Then $h_{2} \coloneqq h_{1} + I \neq 0$ (by \eqref{eq:c3-2.19}.
Moreover, $\lambda R_{2}(\lambda+\im\beta)h_{2} = h_{2}$ (by \eqref{eq:c3-2.16} and $\lambda R_{2}(\lambda)|h_{2}| = |h_{2}|$ (by \eqref{eq:c3-2.20} and Proposition~\ref{prop:c3-2.6}(i)).

Now we apply Proposition~\ref{prop:c3-2.7}(b) and obtain
\begin{equation}\label{eq:c3-2.21}
	\lambda R_{2}(\lambda+\im n\beta)h_{2}^{[n]} = h_{2}^{[n]}\ \text{ for } \ \Re\lambda > 0 , n \in \Z\,.
\end{equation}

In particular, we have $\|R_{2}(r+\im n\beta)\| \geq \frac{1}{r}$, thus 
\[
\|R(r+\im n\beta,A)\| = \|R_{1}(r+\im n\beta)\| \geq \|R_{2}(r+\im n\beta)\| \geq \frac{1}{r} \ \text{ for } r > 0\,.
\]
This finally implies that $ \im n\beta \in \sigma(A)$ for $n \in \Z$.
\end{proof}
%% --
To prove cyclicity of the boundary spectrum in case $s(A)$ is a pole (of arbitrary order), one applies B-III, Lemma~2.8 to reduce the problem to the case of first order poles. Actually, B-III, Lemma~2.8 is true for arbitrary Banach lattices and the proof given in chapter B-III works in the general case as well. For completeness we recall this result. 

%% --
\begin{proposition}\label{prop:c3-2.11}
	%\index{Generator!Pole properties}
	%\index{Semigroup!Spectral properties}
	Let $A$ be the generator of a positive semigroup $\TT$ on a Banach lattice $E$ and suppose that the spectral bound $s(A)$ is a pole of the resolvent of order $k$.
	Then there is a sequence
	\begin{equation}\label{eq:c3-2.22}
		I_{-1} \coloneqq \{0\} \subset I_{0} \subsetneq I_{1} \subsetneq ... \subsetneq I_{k} \coloneqq E
	\end{equation}
	of $\TT$-invariant closed ideals with the following properties.
	If $A_{n}$ is the generator of the semigroup induced by $\TT$ on the quotient $I_{n}/I_{n-1}$, then we have
	\begin{enumerate}[\upshape (i)]
		\item 	
		$s(A_{0}) < s(A)$\,,

		\item 
		If $n \geq 1$, then $s(A_{n}) = s(A)$ is a first order pole of the resolvent $R(\cdot,A_{n})$.
		The corresponding residue is a strictly positive operator.

	\end{enumerate}
\end{proposition}
%% --
Combining Theorem~\ref{thm:c3-2.10} and Proposition~\ref{prop:c3-2.11} one obtains the following generalization of B-III,Theorem~2.9.
In contrast with the former result we do not assume that every point of $\sigma_{b}(A)$ is a pole.
%% --
\begin{corollary}\label{cor:c3-2.12}
	%\index{Spectrum!Boundary cyclicity}
	%\index{Generator!Pole properties}
	If $A$ is the generator of a positive semigroup such that $s(A)$ is a pole of the resolvent, then $\sigma_{b}(A)$ is cyclic.
\end{corollary}
%% --
\begin{proof}
Considering the sequence of ideals as described in Proposition~\ref{prop:c3-2.11} and the corresponding generators $A_{n}$ $(0 \leq n \leq k)$, then we have by A-III, Proposition~4.2 that $\sigma_{b}(A) = \cup_{n=1}^{k} \sigma_{b}(A_{n})$.

By Theorem~\ref{thm:c3-2.10} each set $\sigma_{b}(A_{n})$ is cyclic, hence so is $\sigma_{b}(A)$.
\end{proof}
%% --
The proof of the preceding corollary indicates a possible generalization of Theorem~\ref{thm:c3-2.10}.
Instead of assuming that the resolvent grows slowly one merely needs that there exist a finite chain of closed $\TT$-invariant ideals as described in \eqref{eq:c3-2.22} such that the semigroups induced on the corresponding quotient spaces have slowly growing resolvents.

In case that $\sigma_{b}(A)$ is cyclic one has the alternative (\cf B-III,(2.19))
%% --
\[
\text{Either $\sigma_{b}(A) = \{s(A)\}$ or else $\sigma_{b}(A)$ is an unbounded set.}
\]
%% -- 
Obviously one can use this fact to prove the existence of a dominant spectral value (\cf the explanation preceding B-III, Corollary~2.11).
We give a typical example.
%% --
\begin{corollary}\label{cor:c3-2.13}
	%\index{Spectral value!Dominant}
	%\index{Semigroup!Norm-continuous}
	Let $A$ be the generator of a positive, eventually norm-continuous semigroup.
	Suppose either that the resolvent grows slowly or that $s(A)$ is a pole of the resolvent.
	Then $s(A)$ is a dominant spectral value.
\end{corollary}
%% --
\begin{proof}
The boundary spectrum $\sigma_{b}(A)$ is cyclic (Theorem~\ref{thm:c3-2.10} and Corollary~\ref{cor:c3-2.12} resp.) and compact (A-II, Theorem~1.20).
Hence $\sigma_{b}(A) = \{s(A)\}$.
\end{proof}
%% --
A situation in which Corollary~\ref{cor:c3-2.13} can be applied is described in the following example.
For more details and proofs we refer to \citet{amann:1983}.
%% --
\begin{example}\label{ex:c3-2.14}
	%\index{Domain!Bounded}
	%\index{Example!Bounded domain}
	Let $\Omega$ be a bounded domain in $\R ^{n}$ of class $C^{2}$.
	
	Assume that $\partial\Omega = \Gamma_{0}\cup\Gamma_{1}$ where $\Gamma_{0}$ and $\Gamma_{1}$ are disjoint open and closed subsets of $\partial\Omega$.
    On $E = L^{p}(\Omega)$ $(1 \leq p < \infty)$ we consider a differential operator $L_{p,o}$ which is defined as follows
%% --
\begin{equation}\label{eq:c3-2.23}
	\begin{aligned}
\text{$L_{p,0}f$} &\text{$\coloneqq \sum_{i,j=1}^{n} a_{ij} f_{ij}'' + \sum_{i=1}^{n} b_{i} f_{i}' + cf$ , with domain} \\
\text{ $D(L_{p,0})$} & \text{$\coloneqq \{f \in C^{2}(\bar{\Omega}) \colon$} 
\text{$ f(x) = 0$ for $x \in \Gamma_{0}$}\\ 
& \text{\phantom{aaaaaaa} and $\partial f/\partial\nu (x) + \gamma(x)f(x) = 0$ for $x \in \Gamma_{1}\}$}
	\end{aligned}
\end{equation}
%% --
Here $f_{i}'$ stands for $\partial f/\partial x_{i}$, thus $f_{ij}'' = \partial^{2}f/\partial x_{i}\partial x_{j}$.
We assume that $L_{p,0}$ is elliptic and that all coefficients are real-valued satisfying $a_{ij} \in C^{2}(\bar{\Omega})$, $b_{i} \in C^{1}(\bar{\Omega})$, $\gamma \in C^{1}(\bar{\Omega})$, $c \in C^{1}(\bar{\Omega})$.

Then $L_{p,0}$ is closable and its closure $L_{p}$ is the generator of a holomorphic semigroup of positive operators.
Moreover, the resolvent is compact.
Thus Corollary~\ref{cor:c3-2.13} is applicable and one obtains that $s(A)$ is strictly dominant provided that $\sigma(A) \neq \emptyset$.
Using the results of Section 3 one can show that $\sigma(A) \neq \emptyset$ and that $s(A)$ is an algebraically simple eigenvalue (see Theorem~\ref{thm:c3-3.7} and Proposition~\ref{prop:c3-3.5}).
\end{example}
%% --
We conclude with some remarks.
%% --
\begin{remarks}\label{rem:c3-2.15}
	%\index{Resolvent!Growth properties}
	%\index{Semigroup!Boundary spectrum}
\begin{enumerate}[\upshape (i), wide, labelindent=.5em]	
	\item 
	In the proof of Theorem~\ref{thm:c3-2.10} we did not use the assumption that $R$ is the resolvent of a semigroup.
	In fact one can state this theorem for closed operators having positive resolvent.
	In this case one has to assume that $\{(\lambda-s(A))R(\lambda,A) \colon s(A) < \lambda < s(A)+1\}$ is bounded in $\mathcal{L}(E)$.
	
	One can go even further and consider positive pseudo-resolvents $\{R(\lambda)\}_{\lambda\in D}$.
	This is also possible with respect to Corollary~\ref{cor:c3-2.12}.
	
	\item 
	If $s(A)$ is a pole, then the criteria stated in B-III, Remark~2.15(a) for $s(A)$ to be a first order pole are valid in the setting of arbitrary Banach lattices as well.
	In particular, one has a first order pole provided that $\ker(s(A) - A)$ contains a quasi-interior point or in case that $\ker(s(A) - A')$ contains a strictly positive linear form.
	
	\item 
	It is not difficult to give examples of semigroups whose resolvent do not grow slowly or cannot be reduced by a finite chain of invariant ideals as described after Corollary~\ref{cor:c3-2.12}.
	E.g., one can take a bounded positive operator $A$ which is not nilpotent and satisfies $\sigma(A) = \{0\}$.
	However, the following question is still unanswered.
	\begin{equation*}
		\textit{Does every positive semigroup have cyclic boundary spectrum?}
	\end{equation*}
\end{enumerate}
\end{remarks}

\section{Irreducible Semigroups}\label{sec:c3-3}
%\index{Irreducible Semigroup}
%\index{Theory!Irreducible Semigroup}

The concept of irreducibility is very natural in the general setting of Banach lattices.
However, some of the (equivalent) assertions stated in B-III, Definition~3.1 do not make sense here, others need a slightly different formulation.
%% --
\begin{definition}\label{def:c3-3.1}
	%\index{Semigroup!Irreducible}
	%\index{Banach lattice!Irreducible semigroup}
	
	A positive semigroup $(T(t))_{t\geq 0}$ on a Banach lattice $E$ with generator $A$ is called \emph{irreducible} if one of the following (mutually equivalent) conditions is satisfied
	\begin{enumerate}[\upshape (a)]
	\item
	There is no $(T(t))$-invariant closed ideal except $\{0\}$ and $E$.
	
	\item 
	Given $f \in E$, $\phi \in E'$ such that $f > 0$, $\phi > 0$ then $\langle T(t_{0})f,\phi \rangle > 0$ for some $t_{0} \geq 0$.
	
	\item 
	For arbitrary $f,g \in E_{+}$, $f > 0$, $g > 0$ there exists $t_{0}$ such that $\inf\{T(t_{0})f,g\} > 0$.
	
	\item 
	For some (every) $\lambda > s(A)$ there is no closed ideal other than $\{0\}$ or $E$ which is invariant under $R(\lambda,A)$.
	
	\item 
	For some (every) $\lambda > s(A)$ we have $R(\lambda,A)f$ is a quasi-interior point of $E_{+}$ whenever $f > 0$.
	\end{enumerate}
\end{definition}
%% --
Equivalence of the five conditions above is obtained by a slight modification of the arguments given in B-III, Definition~3.1.
Since there are no difficulties we omit a detailed proof.
Obviously, a semigroup is irreducible if one single operator $T(t_{0})$ is irreducible.
In general the converse does not hold (see p.65 in \citet{greiner:1982}).
The situation is different when holomorphic semigroups are considered.
Then an even stronger assertion holds. In fact irreducibility of a holomorphic semigroup implies that every single operator maps the positive elements onto quasi-interior points.
This is the second statement of the following theorem (see also B-III, Remark~3.2).

%% --
\begin{theorem}\label{thm:c3-3.2}
	%\index{Semigroup!Strictly positive}
	%\index{Semigroup!Holomorphic properties}
	\begin{enumerate}[\upshape (i)]
	\item
 	If $(T(t))$ is an irreducible semigroup then every operator $T(t)$ is strictly positive, \ie 
	given $f > 0$, $t \geq 0$, then $T(t)f > 0$.
	
	\item 
	Suppose $(T(t))_{t\geq 0}$ is a holomorphic positive semigroup.
	If $(T(t))$ is irreducible, then $T(t)f$ is a quasi-interior point of $E_{+}$ whenever $f > 0$ and $t > 0$.
	Equivalently, given $f \in E$, $\phi \in E'$ such that $f > 0$, $\phi > 0$, then $\langle T(t)f,\phi \rangle > 0$ for all $t > 0$\,.
	\end{enumerate}
\end{theorem}
%% --
\begin{proof}
\begin{enumerate}[\upshape (i), wide, labelindent=.5em]
	\item
	Given $t > 0$ and $f > 0$ such that $T(t)f = 0$ and $\lambda > s(A)$, then we have $T(t)(R(\lambda,A)f) = R(\lambda,A)T(t)f = 0$.
	Since $R(\lambda,A)f$ is a quasi-interior point, it follows that $T(t) = 0$.
	Thus for fixed $t \in \R _{+}$ we have either $T(t)$ is strictly positive or else $T(t) = 0$.
	Then strong continuity and $T(0) = \Id \neq 0$ implies that there exists $\tau > 0$ such that $T(t)$ is strictly positive for $0 \leq t \leq \tau$.
	For arbitrary $t \in \R _{+}$ we find $n \in \N$ such that $\frac{t}{n} \leq \tau$.
	Then $T(t) = T(\frac{t}{n})^{n}$ is also strictly positive.
	
	\item
	We prove that for an arbitrary holomorphic positive semigroup $(T(t))_{t\geq 0}$ the following holds
	
	Given $f > 0$, $\phi > 0$ then either $\langle T(t)f,\phi \rangle = 0$ for all $t \geq 0$ or $\langle T(t)f,\phi \rangle > 0$ for all $t > 0$.
	
	Then it follows from Definition~\ref{def:c3-3.1}(b) that for irreducible semigroups always the second case occurs.
	
	Assume that $\langle T(t_{0})f,\phi \rangle = 0$ for some $t_{0} > 0$.
	
	We consider a null sequence $(t_{n})$, $0 < t_{n} < t_{0}$, such that $\|T(t_{n})f - f\| \leq 2^{-n}$ and define $f_{n} \coloneqq T(t_{n})f$, $g_{n} \coloneqq f - \sum_{k=n}^{\infty}(f-f_{k})^{+}$.
	
	Then for $g_{n} \leq f$, $f = \lim_{n\to\infty}g_{n}$ and $m \geq n$, 
	we have 
	
	$g_{n} \leq f - (f-f_{m})^{+} = \inf\{f,f_{m}\} \leq f_{m}$.
	
	For $n \in \N$ fixed and $m \geq n$ we obtain 
	
	$0 \leq \langle T(t_{0}-t_{m})g_{n}^{+},\phi \rangle \leq \langle T(t_{0}-t_{m})f_{m},\phi \rangle = \langle T(t_{0})f,\phi \rangle = 0$.
	
	Thus the function $t \mapsto \langle T(t)g_{n}^{+},\phi \rangle$ is identically zero by the uniqueness theorem for analytic functions.
	Since $f = \lim_{n\to\infty}g_{n}^{+}$, we have $\langle T(t)h,\phi \rangle = 0$ for all $t \in \R _{+}$.
\end{enumerate}
\end{proof}
%% --
The next result can be used to check irreducibility for a semigroup whose generator is a bounded perturbation of a known semigroup. It is a generalization (and an extension to Banach lattices) of B-III, Proposition~3.3. 

%% --
\begin{proposition}\label{prop:c3-3.3}
	%\index{Semigroup!Generator perturbation}
	%\index{Operator!Bounded perturbation}
	Suppose that $A$ is the generator of a positive semigroup $\TT$, further assume that $K$ is a bounded positive operator and $M$ is a bounded real multiplier (\cf C-I, Section~8).
	Let $\mathcal{S}$ be the semigroup generated by $B \coloneqq A + K + M$.
	
	For a closed ideal $I \subset E$ the following assertions are equivalent.
	\begin{enumerate}[\upshape (a)]
		\item 
		$I$ is $\mathcal{S}$-invariant.
	
		\item 
		$I$ is invariant both under $\TT$ and $K$.
	\end{enumerate}
\end{proposition}
%% --
\begin{proof}
We recall that a closed subspace $I \subset E$ is invariant for a semigroup generated by $C$ if and only if $C(D(C)\cap I) \subset I$ and the restriction $C_{|I}$ of $C$ with domain $D_{|I} \coloneqq D(C)\cap I$ generates a semigroup on $I$ (see A-I,3.3).
Now let $I$ be a closed ideal of $E$.

$(b)\Rightarrow (a)$.
If $I$ is $\TT$-invariant then $A_{|I}$ generates a semigroup on $I$.
The restrictions $K_{|I}$ nd $M_{|I}$ of $K$ and $M$ respectively are bounded linear operators on $I$.
Note that each closed ideal is invariant for $M$, \cf C-I, Section~8.
Thus $B_{|I} = A_{|I} + M_{|I} + K_{|I}$ with domain $D(A_{|I}) = D(A)\cap I = D(B)\cap I$ is the generator of a semigroup on $I$.
It follows that $I$ is invariant for the semigroup generated by $B$.

$(a) \Rightarrow (b)$.
Without loss of generality we assume $M \geq 0$.
Then $0 \leq T(t) \leq S(t)$ for all $t \geq 0$.
It follows that $I$ is $T$-invariant.
Thus for $x \in D(A)\cap I = D(B)\cap I$, we have $Kx = Bx - Ax - Mx$.
This shows that $K(D(B)\cap I) \subset I$.
Since $B|_{I}$ is a generator $D(B)\cap I$ is dense in $I$.
Then, by continuity, we have $KI \subset I$, \ie  $I$ is $K$-invariant.
\end{proof}

Next we consider some concrete examples.

\begin{examples}\label{ex:c3-3.4}
	%\index{Examples!Semigroups}
	%\index{Semigroups!Examples}
	%\index{Concrete Examples}
	\begin{enumerate}[\upshape (i), wide, labelindent=.5em]
	\item 
	Suppose that on $E = L^{p}(\mu)$ $(1 \leq p < \infty)$ the semigroup $(T(t))$ is given by
	%% --
	\begin{equation}\label{eq:c3-3.1}
		(T(t)f)(x) = \int_{X} k(t,x,y)f(y) \diff{\mu}(y) \quad (x \in X,\, t > 0)
	\end{equation}
	%% --
	for some measurable function $k \colon \R _{+} \times X \times X \to \R _{+}$.
	Then $(T(t))$ is irreducible if and only if for any two measurable sets $M$ and $N$ with $0 < \mu(M) < \infty$, $0 < \mu(N) < \infty$, $\mu(M\cap N) = 0$ there exist $t_{0} > 0$ such that $\int_{M}\int_{N} k(t_{0},x,y)\diff{\mu}(x)\diff{\mu}(y) > 0$\,.
	
	\item 
	Consider the first derivative on $\R $, $\R _{+}$ or $\R _{2\pi} = \Gamma$ as operator on the corresponding $L^{p}$-space (with respect to the Lebesgue measure.)
	Then the statements made in B-III, Example~3.4(c) are true.
	The same is true for B-III, Example~3.4(e) and (f) (second order differential operator) when the corresponding $L^{p}$-spaces are considered.
	
	\item 
	Let $E = L^{1}[-1,0]$ and for $g \in L^{\infty}$ consider the operator $A_{g}$ given by
	%% --
	\begin{equation}\label{eq:c3-3.2}
		A_{g}f \coloneqq f', \quad D(A_{g}) \coloneqq \{f \in W^{1}[-1,0] \colon f(0) = \int_{-1}^{0} f(x)g(x)\dx\}
	\end{equation}
	%% --
	If $g \geq 0$ then $A_{g}$ generates a positive semigroup.
	In case there exist $\epsilon > 0$ such that $g$ vanishes a.e. on $[-1,-1+\epsilon]$, then $I \coloneqq \{f \in L^{1} \colon f|_{[-1+\epsilon,0]} = 0\}$ is a non-trivial closed ideal which is invariant under the semigroup.
	It is not difficult to see that the condition on $g$ stated above is also necessary for $(T(t))$ to be reducible (\ie  not irreducible.)

	\item 
	Let $E = L^{1}([0,1]\times[-1,1])$ and consider the semigroup $(T(t))_{t\geq 0}$ defined as follows
	%% --
	\begin{equation}\label{eq:c3-3.3}
		(T(t)f)(x,v) \coloneqq \begin{cases}
			f(x-vt,v) & \text{for } 0 \leq x-vt \leq 1 \,,\\
			0 & \text{otherwise}.
		\end{cases}
	\end{equation}
	%% --
		$(T(t))_{t\geq 0}$ is a positive semigroup on $E$ and
	%% --
	\begin{align*}\label{eq:c3-D0}
		D_{0} \coloneqq \{f \in C^{1}([0,1]\times[-1,1]) \colon  
		& f(0,v) = f_{x}(0,v) = 0 \text{ if } v \geq 0\,,\\ 
		& f(1,v) = f_{x}(1,v) = 0 \text{ if } v \leq 0\}
	\end{align*}
	%% --
	is a core for its generator $A$ (\cf A-I, Corollary~1.34).
	We have
	%% --
	\begin{equation}\label{eq:c3-3.4}
		(Af)(x,v) = -v\frac{\partial f}{\partial x}(x,v) \quad (f \in D_{0}).
	\end{equation}
	%% --
	The Laplace transform of $(T(t))$ is the resolvent of $A$.
	An explicit calculation yields
	%% --
	\begin{equation}\label{eq:c3-3.5}
		(R(\lambda,A)f)(x,v) = \int_{0}^{1} r_{\lambda}(x,x',v)f(x',v)\dx' \quad (\lambda > 0)
	\end{equation}
	%% --
	where $r_{\lambda} \colon [0,1] \times [0,1] \times [-1,1] \to \R $ is given by
	\begin{equation*}\label{eq:c3-r_lambda}
		r_{\lambda}(x,x',v) = \begin{cases}
			|v|^{-1}\exp(-\lambda(x-x')v^{-1}) & 
			\begin{cases}\text{if either } v>0 \text{ and } x'\leq x \\
		    \text{or } v<0 \text{ and } x'\geq x, 
		    \end{cases}\\
			0 & \text{otherwise}.
		\end{cases}
	\end{equation*}
	Let $\sigma \colon [0,1]\times[-1,1] \to \R _{+}$ and $\kappa \colon [0,1]\times[-1,1]\times[-1,1] \to \R _{+}$ be bounded measurable functions and consider the operators $M$ and $K$ given by
	%% --
	\begin{equation}\label{eq:c3-3.6}
		Mf \coloneqq \sigma f, \quad Kf \coloneqq \int_{-1}^{1} \kappa(\cdot,\cdot,v')f(\cdot,v') \diff{v'}.
	\end{equation}
	%% --
	Then $B \coloneqq A - M + K$ with domain $D(B) \coloneqq D(A)$ is the generator of a positive semigroup.
	
	Using Proposition~\ref{prop:c3-3.3} we can prove the following irreducibility criterion for the semigroup $(S(t))_{t\geq 0}$ generated by $B$.
	%% --
	\begin{equation}\label{eq:c3-3.7}
		\text{If } \kappa \text{ is strictly positive, then } (S(t))_{t\geq 0} \text{ is irreducible.}
	\end{equation}
	%% --
	
	Actually, in view of Proposition~\ref{prop:c3-3.3}, we have to show that a closed ideal which is invariant under $R(\lambda,A)$ and $K$ has to be $\{0\}$ or $E$.
	
	We recall that closed ideals of $E$ are uniquely determined (up to sets of measure zero) by measurable subsets $Y$ of $[0,1]\times[-1,1]$, \ie  every closed ideal has the form
	%% --
	\[
		I_{Y} = \{f \in E \colon f \text{ vanishes (a.e.) on } [0,1]\times[-1,1] \setminus Y\}.
	\]
	%% --
	Since we assumed that $\kappa$ is strictly positive, $I_{Y}$ is $K$-invariant if and only if $Y = X\times[-1,1]$ for some measurable set $X \subset [0,1]$.
	If we assume that $X$ has positive measure and define
    \[\textstyle
	\alpha \coloneqq \sup\left\{x \in [0,1] \colon \int_{0}^{x} \1_{X}(s)\ds  = 0\right\}, \beta \coloneqq \inf\left\{x \in [0,1] \colon \int_{x}^{1} \1_{X}(s)\ds  = 0 \right\},
	\] 
	then we have $\alpha < \beta$ and the support of the function $h \coloneqq R(\lambda,A)\1_{Y}$, where 
	$(Y \coloneqq X\times[-1,1])$ is given by $\supp h = [\alpha,1]\times[0,1] \cup [0,\beta]\times[-1,0]$.
	Since we assumed that $I_{Y}$ is $R(\lambda,A)$-invariant, we have $h \in I_{Y}$, \ie  $\supp h \subset Y = X\times[-1,1]$.
	Obviously, this is true only if $Y = [0,1]\times[-1,1]$ or $I_{Y} = E$.
	
	A weaker condition than \eqref{eq:c3-3.7} entailing irreducibility is the following.
	%% --
	\begin{equation}\label{eq:c3-3.8}
		\begin{aligned}
		&\text{There exists } \delta > 0 \text{ such that } \kappa \text{ is strictly positive}\\
		&\text{on the sets } [0,\delta]\times[-1,1] \text{ and } [1-\delta,1]\times[-1,1]\,.
		\end{aligned}
	\end{equation}
	%% --
	For details we refer to \citet{greiner:1984d}.
	\end{enumerate}	
\end{examples}
	
In the following proposition we list some properties which are consequences of irreducibility. This extends Proposition~3.5 of B-II to the setting of Banach lattices. The first assertion of the latter proposition is no longer true in the general setting (see Example~\ref{ex:c3-3.6} and Theorem~\ref{thm:c3-3.7}). 

\begin{proposition}\label{prop:c3-3.5}
	%\index{Irreducibility!Properties}
	%\index{Semigroups!Irreducible Properties}
	%\index{Properties!Irreducible Semigroups}
	
	Suppose $A$ to be the generator of an irreducible, positive semigroup on a Banach lattice $E$.
	Then the following assertions are true.
	\begin{enumerate}[\upshape (i)]
	\item 
	Every positive eigenvector of $A$ is a quasi-interior point.
	
	\item 
	Every positive eigenvector of $A'$ is strictly positive.
	
	\item 
	$\ker(s(A) - A')$ contains a positive element, then $\dim(\ker(s(A) - A)) \leq 1$.
	
	\item If $s(A)$ is a pole of the resolvent, then it has algebraic (and geometric) multiplicity $1$.
	The corresponding residue has the form $P = \phi\otimes u$, where $\phi \in E'$ is a positive eigenvector of $A'$, $u \in E$ is a positive eigenvector of $A$ and $\langle u,\phi \rangle = 1$.
	\end{enumerate}
\end{proposition}
%% --	
\begin{proof}
    To prove (i), (ii) and (iv) one can proceed as in the case $C_{0}(X)$ (see B-III, Proposition~3.5).
    We only prove (iii) and assume $s(A) = 0$.
    By assumption and by assertion (i) there exists $\phi \gg 0$ such that $T(t)'\phi = \phi$ $(t\geq0)$.
    Given $f \in \ker A$, then $T(t)f = f$ hence $|f| = |T(t)f| \leq T(t)|f|$.
    Since $\phi$ is strictly positive and $\langle|f|,\phi\rangle \leq \langle T(t)|f|,\phi\rangle = \langle|f|,\phi\rangle$, it follows that $|f| = T(t)|f|$.
    We have shown that $\ker A$ is a sublattice.
    Then for $f \in \ker A$, $f$ real, \ie  $f = \overline{f}$, we have that $f^{+}$ and $f^{-}$ are elements of $\ker A$.
    Hence the principal ideals generated by $f^{+}$ and $f^{-}$ are $T$-invariant.
    Since these ideals are orthogonal, the irreducibility of $T$ implies that either $f^{+}$ or $f^{-}$ is zero.
    We have shown that $\ker A \cap E_{\R }$ is totally ordered, hence at most one-dimensional (see Proposition~3.4 of \citet{schaefer:1974}).
\end{proof}
%% --
In arbitrary Banach lattices it is no longer true that an irreducible semigroup has necessarily nonvoid spectrum.
We indicate how an irreducible semigroup having empty spectrum can be constructed.

\begin{example}\label{ex:c3-3.6}
%\index{Example!Empty Spectrum}
%\index{Spectrum!Empty Example}
%\index{Irreducible Semigroup!Empty Spectrum}

Consider the Banach lattice $E = L^{p}(\Gamma\times\Gamma)$.
For (fixed) positive numbers $\alpha,\beta$ such that $\frac{\alpha}{\beta}$ is irrational we define a positive semigroup $(T_{0}(t))_{t\geq 0}$ as follows
%% --
\begin{equation}\label{eq:c3-3.9}
	(T_{0}(t)f)(z,w) \coloneqq f(z\cdot \mathrm{e}^{\im\alpha t},w\cdot \mathrm{e}^{\im\beta t}) \quad (z,w \in \Gamma = \{\xi\in\C :|\xi|=1\})\,.
\end{equation}
%% --
Next we define for a positive function $m \colon \Gamma\times\Gamma \to \R $ which is continuous on $\Gamma\times\Gamma\setminus(1,1)$ functions $m_{t}$, $t\geq 0$, as follows
%% --
\begin{equation}\label{eq:c3-3.10}
	m_{t}(z,w) \coloneqq \exp(-\int_{0}^{t} m(z\cdot \mathrm{e}^{\im\alpha s},w\cdot \mathrm{e}^{\im\beta s})\ds )\,.
\end{equation}
%% --
Then $(T(t))_{t\geq 0}$ defined by
%% --
\begin{equation}\label{eq:c3-3.11}
	T(t)f \coloneqq m_{t}\cdot(T_{0}(t)f)
\end{equation}
%% --
is a strongly continuous semigroup of positive contractions on $E$.
Since $\frac{\alpha}{\beta}$ is irrational, the semigroup $(T_{0}(t))$ is irreducible.
Moreover, each $m_{t}$ is strictly positive (\ie  $m_{t} > 0$ a.e.) thus $(T(t))$ is irreducible as well.
If one chooses $m$ such that $m(z,w)$ tends to $+\infty$ sufficiently fast as $(z,w) \to (1,1)$, one obtains $\|T(t)\| = \|m_{t}\|_{\infty} \leq \exp(-t^{2})$ for all $t \geq 0$.
Obviously such an estimate of $\|T(t)\|$ implies $\omega_0(A) = -\infty$, hence $\sigma(A) = \emptyset$.
\end{example}
%% --
\begin{theorem}\label{thm:c3-3.7}
%\index{Theorem!Nonvoid Spectrum}
%\index{Spectrum!Nonvoid Conditions}
%\index{Irreducible Semigroup!Spectrum}

Suppose that $(T(t))_{t\geq 0}$ is an irreducible, positive semigroup on the Banach lattice $E$.
Each of the following conditions on $E$ and $(T(t))$, respectively, implies that $\sigma(A) \neq \emptyset$
%% --
\begin{enumerate}[\upshape (i)]
	\item
	$E = C_{0}(X)$ where $X$ is locally compact.

	\item 
	$E = \ell^{p}$ $(1 \leq p < \infty)$ (more generally, $E$ contains atoms).

	\item either $T(t_{0})$ is compact for some $t_{0}$ or $R(\lambda_{0},A)$ is compact for some $\lambda_{0} \in \rho(A)$.

	\item 
	$E$ has order continuous norm and either $T(t_{0})$ or $R(\lambda_{0},A)$ is a kernel operator for some $t_{0} \geq 0$ $(\lambda_{0} \in \rho(A))$. 	
	\footnote[1]{For a precise definition of a kernel operator we refer to Section~IV.9 of \citet{schaefer:1974} or Chapter~13 of \citet{zaanen:1983}.}

	\item 
	$E$ is reflexive and there exist $t_{0} > 0$, $h \in E_{+}$ such that $T(t_{0})E \subset E_{h}$\,.
\end{enumerate}
%% --
\end{theorem}
%% --
\begin{proof}
	(i) is proved in B-III, Proposition~3.5(a).
	
	Assertion (ii)-(f) will be proved utilizing the analogous results for a single operator.
	In view of A-III, Proposition~2.5 we have to show that $r(R(\lambda,A)) > 0$ for some $\lambda \in \rho(A)$.
	Moreover, from A-I,(3.1) we deduce
	%% --
	\begin{align*}
		T(t)R(\lambda,A) = \mathrm{e}^{\lambda t}R(\lambda,A) - \mathrm{e}^{\lambda t}\int_{0}^{t} \mathrm{e}^{-\lambda s}T(s)\ds \leq \mathrm{e}^{\lambda t}R(\lambda,A) \ (t\geq 0, \lambda>s(A))\,.
	\end{align*}
	%% --
	Since the spectral radius is an isotone function on the cone of positive operators, it is enough to show that
	%% --
	\begin{equation}\label{eq:c3-3.12}
		r(T(t)R(\lambda,A)) > 0 \text{ for some } t \geq 0,\, \lambda > s(A).
	\end{equation}
	%% --
	
	Using Theorem~\ref{thm:c3-3.2}(i) it is easy to see that $T(t)R(\lambda,A) = R(\lambda,A)T(t)$ is irreducible.
	
	The assertions (ii),(iv) and (v) now follow from the corresponding results for a single operator as presented in Sect.V.6 of \citet{schaefer:1974} (see Proposition~6.1, Theorem~6.5 Corollary and Theorem~6.5 l.c.).
	(iii) follows from the recent result of \citet{depagter:1986} which ensures that every positive operator on a Banach lattice which is compact and irreducible has positive spectral radius.
\end{proof}
%% --
The theorem can be used to prove that elliptic operators as described in Example~2.14 have non-empty spectrum.
It is shown in \citet{amann:1983} that these operators have compact resolvent and generate irreducible semigroups.
Thus the assumption of (iii) is satisfied.

Concerning the eigenvalues of an irreducible semigroup which are contained in $\sigma_{b}(A)$ all statements established for spaces $C_{0}(X)$ in B-III, Theorem~3.6 are true in the setting of Banach lattices.
The proof can be translated without difficulties and will be omitted (see also \citet[Theorem~2.6]{greiner:1982}).

\begin{theorem}\label{thm:c3-3.8}
	%\index{Theorem!Irreducible Semigroup Eigenvalues}
	%\index{Eigenvalues!Irreducible Semigroup}
	%\index{Irreducible Semigroup!Eigenvalues}
	
	Suppose $\TT$ is an irreducible semigroup on the Banach lattice $E$ and let $A$ be its generator.
	Assume that $s(A) = 0$ and that there exists a positive linear form $\psi \in D(A')$ with $A'\psi \leq 0$\,.

    If $P\sigma(A)\cap \im\R $ is non-empty, then the following assertions are true.
	\begin{enumerate}[\upshape (i]
		\item 
			$P\sigma(A)\cap \im\R $ is a (additive) subgroup of $\im\R $.
	
		\item 
		The eigenspaces corresponding to $\lambda \in P\sigma(A)\cap \im\R $ are one-dimensional.
	
		\item 
		If $Ah = \im\alpha h$ $(h \neq 0,\, \alpha \in \R )$ then $|h|$ is a quasi-interior point and the following holds
		%% --
		\begin{equation}\label{eq:c3-3.13}
			S_{h}(D(A)) = D(A) \text{ and } S_{h}^{-1}\circ A\circ S_{h} = (A + \im\alpha\Id).
		\end{equation}
		%% --
		
		\item 
		$0$ is the only eigenvalue of $A$ admitting a positive eigenvector.
	\end{enumerate}
\end{theorem}
%% -- 
One can apply the theorem in order to prove that the rotation semigroup on $\Gamma$ (\cf A-I,2.5) is the only positive periodic semigroup which is irreducible.

\begin{corollary}\label{cor:c3-3.9}
	%\index{Corollary!Positive Periodic Semigroup}
	%\index{Semigroup!Positive Periodic}
	%\index{Periodic Semigroup!Positive}
	
	Let $(T(t))_{t\geq 0}$ be a positive irreducible semigroup on a Banach lattice $E$ which is periodic of period $\tau$.
	Assume that $\dim E > 1$.
	Then there exist continuous lattice homomorphisms
	$i \colon C(\Gamma) \to E$ and $j \colon E \to L^{1}(\Gamma)$,
	both injective with dense range,
	such that the diagram commutes for all $t\geq 0$.
	Hereby, $j\circ i$ is the canonical inclusion of $C(\Gamma)$ in $L^{1}(\Gamma)$\,.
	%% --
	\[ 
	\begin{tikzcd}
		C(\Gamma) \arrow[r, "i"] \arrow[d, "R_\tau(t)"'] & E  \arrow[r, "j"] \arrow[d, "T(t)"'] & L^1(\Gamma) \arrow[d, "R_\tau(t)"] \\
		C_(\Gamma) \arrow[r, "i"'] & E \arrow[r, "j"'] & L^{1}(\Gamma)
	\end{tikzcd}
	\]
	%% --
\end{corollary}
%% --
\begin{proof}
    By Theorem~\ref{thm:c3-3.8} and A-III, Thm~5.4 we have $R\sigma(A) = P\sigma(A) = \sigma(A) = \im\alpha\Z$ with $\alpha \coloneqq \frac{2\pi}{n\tau}$ for suitable $n \in \N$.
    We fix $h \in \ker(\im\alpha - A)$, $h \neq 0$.
    Then $|h| \in \ker A$ and there exists $\phi \in \ker A'$ such that $\langle|h|,\phi\rangle = 1$.
    According to the Kakutani-Krein Theorem we identify $E_{|h|}$ with $C(K)$.
    Then $h$ is a unimodular function onto $\Gamma$ (use the argument given in the proof of B-III, Theorem~3.6(iii)).
    
    We define $i \colon C(\Gamma) \to E$ by $i(f) \coloneqq f\circ h \in C(K) \cong E_{h} \subset E$, then $i$ is injective.
    For the monomials $e_{n}(z) \coloneqq z^{n}$ $(n \in \Z)$ we have $i(e_{n}) = h^{[n]}$ thus $i$ has dense image in $E$ (by A-III, Theorem~5.4).
    Moreover, 
    \[
    \langle h^{[n]},\phi\rangle = \langle i(e_{n}),\phi\rangle = \int_{0}^{2\pi} e_{n}(\mathrm{e}^{it})\dt  =2\pi\cdot\delta_{n0}\  \text{ for all } n \in \Z\,,
    \] 
    hence $\int_{0}^{2\pi} f(\mathrm{e}^{it})\dt   = \langle i(f),\phi\rangle$ for all $f \in C(\Gamma)$.
    It follows that $(E,\phi) \cong L^{1}(\Gamma)$, and we define $j$ to be the canonical mapping from $E$ into $(E,\phi) \cong L^{1}(\Gamma)$ (see C-I, Section~4).
    Then $j$ has dense image and is injective since $\phi$ is strictly positive (\cf Proposition~\ref{prop:c3-3.5}(ii)).
    One easily verifies that the diagram commutes.
\end{proof}
%% --
Now we are going to prove the main result of this section.
As in the proof of Theorem~\ref{thm:c3-2.10} we will utilize pseudo-resolvents on a suitable $\mathcal{F}$-product of the Banach lattice.
To simplify the proof we isolate two lemmas.

%% --
\begin{lemma}\label{lem:c3-3.10}
	%\index{Lemma!F-product}
	%\index{F-product!Kernel Dimension}
	
	Let $\mathcal{F}$ be a filter on $\N$ which is finer than the Frechet filter and let $E_{\mathcal{F}}$ be the $\mathcal{F}$-product of the Banach lattice $E$.
	Given $R \in \mathcal{L}(E)$ and denoting its canonical extension to $E_{\mathcal{F}}$ by $R_{\mathcal{F}}$ the following is true.
\begin{quote}
	If $\alpha \in A\sigma(R)\setminus P\sigma(R)$, then $\ker(\alpha - R_{\mathcal{F}})$ is infinite dimensional.
\end{quote}
\end{lemma}
%% --	
\begin{proof}
    Let $(f_{n})_{n\geq 1}$ be a normalized approximate eigenvector of $R$ corresponding to $\alpha$.
    Since every accumulation point of $(f_{n})$ is an eigenvector of $R$, the assumption $\alpha \notin P\sigma(A)$ implies that $(f_{n})$ does not have any accumulation points.
    Then there exist an $\epsilon > 0$ and a subsequence $(g_{n})$ of $(f_{n})$ such that
    %% --
    \begin{equation}\label{eq:c3-3.14}
        \|g_{n} - g_{m}\| \geq \epsilon \text{ whenever } n \neq m.
    \end{equation}
    %% --
    
    Obviously, $(g_{n})$ is a normalized approximate eigenvector of $R$ and so is every subsequence of $(g_{n})$.
    In particular, for $k \in \N$ the sequence $(g_{n+k})_{n\geq 1}$ is a normalized approximate eigenvector of $R$.
    Then the elements $\hat{g}^{k} \in E_{\mathcal{F}}$ given by $\hat{g}^{k} \coloneqq ((g_{n+k})_{n\geq 1} + c_{\mathcal{F}}(E))$ are normalized eigenvectors of $R_{\mathcal{F}}$ corresponding to $\alpha$.
    As a consequence of \eqref{eq:c3-3.14} we obtain
    $\|\hat{g}^{k} - \hat{g}^{m}\| = \mathcal{F}\text{-}\lim\sup\|g_{n+k} - g_{n+m}\| \geq \epsilon$ provided that $k \neq m$.
    This shows that the unit ball in $\ker(\alpha - R_{\mathcal{F}})$ is not relatively compact, hence $\ker(\alpha - R_{\mathcal{F}})$ has to be infinite dimensional.
\end{proof}

%% --
\begin{lemma}\label{lem:c3-3.11}
	%\index{Lemma!Banach Lattice Subspaces}
	%\index{Banach Lattice!Subspace Dimension}
	
	Let $E$ be a Banach lattice and let $M$, $L$ be two linear subspaces of $E$.
	Assume that $f \in M$ implies $|f| \in L$, then $\dim L \geq \dim M$.
\end{lemma}
%% --	
\begin{proof}
    Let $\{g_{1},g_{2},\ldots,g_{m}\}$ $(m\geq 1)$ be any (finite) subset of $M$ which is linearly independent.
    For $u \coloneqq \sum_{n=1}^{m}|g_{n}|$ all vectors $g_{n}$ are contained in the principal ideal $E_{u}$ which (by the Kakutani-Krein Theorem) is isomorphic to a space $C(K)$.
    Considering $g_{1}$, $g_{2}$, $\ldots$, $g_{m}$ as continuous functions on $K$, there exist points $x_{1}$, $x_{2}$, $\ldots$, $x_{m} \in K$ and functions $h_{1}$, $h_{2}$, $\ldots$, $h_{m} \in \mathrm{span}\{g_{1},g_{2},\ldots,g_{m}\}$ such that $h_{i}(x_{j}) = \delta_{ij}$.
    Then $|h_{i}|(x_{j}) = \delta_{ij}$ hence $\{|h_{j}| \colon 1\leq j\leq m\}$
    is a subset of $m$ linearly independent vectors which (by assumption) is contained in $L$.
\end{proof}

The surprising fact in the following theorem is the conclusion that every point in the boundary spectrum is a simple algebraic pole if only $s(A)$ is supposed to be a pole.

%% --
\begin{theorem}\label{thm:c3-3.12}
%\index{Theorem!Boundary Spectrum}
%\index{Spectrum!Boundary Poles}
%\index{Irreducible Semigroup!Spectral Properties}

Let $\TT$ be an irreducible semigroup on a Banach lattice and let $A$ be its generator.
If $s(A)$ is a pole of the resolvent, then $\sigma_{b}(A) = s(A) + \im\alpha\Z$ for some $\alpha \geq 0$.
Moreover, $\sigma_{b}(A)$ contains only algebraically simple poles.
\end{theorem}
%% --
\begin{proof}
	We will assume that $s(A) = 0$.
	Assuming first that every element of $\sigma_{b}(A)$ is an eigenvalue of $A$, one can conclude the following.
	By Theorem~\ref{thm:c3-3.8}(i) we know that $\sigma_{b}(A)$ is an additive subgroup of $\im\R $.
	Since it is a closed subset and $0$ is an isolated point, it follows that $\sigma_{b}(A) = \im\alpha\Z$ for some $\alpha \geq 0$.
	Moreover, as a consequence of \eqref{eq:c3-3.13}, for every $k \in \Z$ we obtain
	%% --
	\begin{equation}\label{eq:c3-3.15}
		R(\lambda+\im k\alpha,A) = S_{h}^{-k}\circ R(\lambda,A)\circ S_{h}^{k} \quad (\lambda \in \rho(A),\, k \in \Z).
	\end{equation}
	%% --
	
	By Proposition~\ref{prop:c3-3.5}(iv) $0$ is an algebraically simple pole.
	Then \eqref{eq:c3-3.15} implies that every point $\im k\alpha$ has the same property.
	
	We now show that every element $\im\beta$ is an eigenvalue of $A$.
	By Proposition~\ref{prop:c3-3.5}(iv) the residue of $R(\cdot,A)$ in $\lambda = 0$ has the form $P = \phi\otimes u$ with $\phi(u) = 1$.
	Given an ultrafilter $\mathcal{U}$ on $\N$ which is finer than the Frechet filter, then $\lim_{\mathcal{U}}\phi(f_{n})$ exists for every bounded sequence $(f_{n}) \subset E$.
	Using this fact it is easy to see that the canonical extension $P_{\mathcal{U}}$ of $P$ to the $\mathcal{U}$-product $E_{\mathcal{U}}$ of $E$ has the following form.
	%% --
	\begin{equation}\label{eq:c3-3.16}
		\begin{aligned}
		&P_{\mathcal{U}} = \hat{\phi}\otimes\hat{u} \quad \ \text{where } \hat{u} \coloneqq (u,u,u,\ldots)+c_{\mathcal{U}}(E) \in E_{\mathcal{U}} \ \text{ and } \hat{\phi} \in (E_{\mathcal{U}})'
		\\
		&\text{is given by } \hat{\phi}((f_{n})+c_{\mathcal{U}}(E)) \coloneqq \lim_{\mathcal{U}}\phi(f_{n}) \ ((f_{n})+c_{\mathcal{U}}(E) \in E_{\mathcal{U}})\,.
		\end{aligned}
	\end{equation}
	%% --
	
	Given $\im\beta \in \sigma_{b}(A)$ then $\im\beta \in A\sigma(A)$ hence $1 \in A\sigma(\lambda R(\lambda+\im\beta,A))$.
	If we assume that  $\im\beta \notin P\sigma(A)$, then $1 \notin P\sigma(\lambda R(\lambda+\im\beta,A))$.
	Then Lemma~\ref{lem:c3-3.10} implies that $M \coloneqq \ker(1 - \lambda R(\lambda+\im\beta,A)_{\mathcal{U}})$ is infinite dimensional (and independent of $\lambda$ by Proposition~\ref{prop:c3-2.6}(i).)
	For $\hat{f} \in M$ we have $|\hat{f}| = \gamma R(\gamma+\im\beta,A)_{\mathcal{U}}|\hat{f}| \leq \gamma R(\gamma,A)_{\mathcal{U}}|\hat{f}|$ for every $\gamma > 0$.
	It follows that $\hat{\phi}(|\hat{f}|) = P_{\mathcal{U}}|\hat{f}| = \lim_{\gamma\to 0}\gamma R(\gamma,A)_{\mathcal{U}}|\hat{f}| \geq |\hat{f}|$.
	Thus considering the closed ideal $I \coloneqq \{\hat{f} \in E_{\mathcal{U}} \colon \hat{\phi}(|\hat{f}|) = 0\}$ we have
	%% --
	\begin{equation}\label{eq:c3-3.17}
		\tilde{\phi}(|\hat{f}|) - |\hat{f}| \in I \ \text{ for every } \hat{f} \in M\,.
	\end{equation}
	%% --
	This implies that $M \cap I = \{0\}$. 
	Hence the canonical image $M_/$ of $M$ in the quotient space $E_{\mathcal{U}/{I}}$ is infinite dimensional as well. 
	By \eqref{eq:c3-3.16} and \eqref{eq:c3-3.17} the absolute value of an element $\tilde{f} \in M_/$ is a scalar multiple of $\tilde{u} \coloneqq \hat{u} + I$. 
	This is a contradiction by Lemma~\ref{lem:c3-3.11}.
\end{proof}
%% --
In view of A-III, Proposition~4.2 the result above has consequences for semigroups which can be reduced (by considering restrictions to invariant ideals or quotients) to semigroups which satisfy the hypothesis of Theorem~\ref{thm:c3-3.12}. 
Semigroups having this property are precisely those for which $s(A)$ is a pole of the resolvent of finite algebraic multiplicity. 
The latter claim is a consequence of Proposition~\ref{prop:c3-2.11} and the following lemma.

\begin{lemma}\label{lem:c3-3.13}
	Suppose that $\TT = (T(t))_{t \ge 0}$ is a positive semigroup such that $s(A)$ is a first order pole of the resolvent. 
	Moreover assume that the corresponding residue is a strictly positive operator of finite rank.
%% --	
	Then there are closed $\TT$-invariant ideals $J_{1},J_{2}, .. J_{m}$ which are mutually orthogonal. 
	For the restrictions  $\TT_{k}$ of $\TT$ to $J_{k}$ the following is true.
    %% --
	\begin{enumerate}[\upshape (i)]
		\item $\TT_{k}$ is irreducible with spectral bound $s(A_k) = s(A)$\,.
		\item $s(A_{/J}) < s(A)$  where $J \coloneqq  J_{1}\oplus J_{2}\oplus .. \oplus J_{m}\,.$
	\end{enumerate}
\end{lemma}
%% --
\begin{proof}
	We assume $s(A) = 0$. 
	Since $P$ is a strictly positive projection $PE = \ker A$ is a sublattice of $E$. 
	Actually, if $x \in PE$ \ie  $Px = x$, then $P|x| \geq |Px| = |x|$. 
	Hence $P(|P|x|-|x||) = P^{2}|x| - P|x| = 0$ which implies that $P|x| - |x| = 0$ or $|x| \in PE$.
	Thus we know that $\ker A$ is a finite dimensional sublattice of $E$ hence it is isomorphic to a space $\C ^{m}$ for some $m \in \N$ (see Section~II.4 of \citet{schaefer:1974}). 
	
	Then there exist vectors $e_{j} \in E_{+}$ $(1\leq j\leq m)$ such that the following holds.
	%% --
	\begin{equation}\label{eq:c3-3.18}
		\ker A = \mathrm{span} \{e_{1},e_{2}, .. ,e_{m}\} \text{ and } \inf\{e_{i},e_{j}\} = 0 \ \text{ for }  i \neq j\,.
	\end{equation}
	%% --
	We have $T(t)e_{k} = e_{k}$ hence the closed ideal generated by $e_{k}$ is $\TT$-invariant. 
	We denote this ideal $J_{k}$ and define $J\coloneqq  J_{1}\oplus J_{2}\oplus .. \oplus J_{m}$.
	The ideal $J$ is closed 
    (see \citet[III Theorem~1.2]{schaefer:1974}, $\TT$-invariant and we have $PE = \ker A \subset J$. 
	Then $P_{/J} = 0$ hence the spectral bound $s(A_{/J})$
	is strictly less than zero (by Theorem~\ref{thm:c3-1.1}(i)). 
	Moreover, the residue corresponding to the resolvent of $\TT_{k}$, we denote it $P_{k}$, is the restriction of $P$ to $J_{k}$.
	Then $P_{k}$ is strictly positive and $P(J_{k}) = \mathrm{span}\{e_{k}\}$. 
	To show that $T_{k}$ is irreducible, we consider an invariant ideal $I$. 
	Then we have $R(\lambda,A_{k})I \subset I$ for $\lambda > 0$ hence $P_{k} = \lim_{\lambda \to 0}\lambda R(\lambda,A_{k})$ leaves $I$ invariant. 
	If $I \neq \{0\}$, then $P_{k}I \neq \{0\}$ since $P_{k}$ is strictly positive. 
	Then $e_{k} \in P_{k}J \subset I$ which implies that $J_{k} \subset I$.
	%$\square$
\end{proof}
%% --
Combining the lemma with Proposition~\ref{prop:c3-2.11} one obtains the following.

If $s(A)$ is a pole of finite algebraic multiplicity, then there exists a finite chain of $\TT$-invariant ideals $I_{-1} \coloneqq  \{0\} \subset I_{0}\subset ... \subset I_{N} \coloneqq  E$ $(N\in\N)$ such that the following is true.
%% --
\begin{equation}\label{eq:c3-3.19}
	\begin{aligned}
	&\textit{For the semigroup $\TT_{n}$ on $I_{n}/I_{n-1}\, \ (0 \leq n \leq N)$ which is induced by $\TT$}
		\\
	& \textit{we have either $s(A_{n}) = s(A)$ and $\TT_{n}$ is irreducible or $s(A_{n}) < s(A)$}\,.
	\end{aligned}
\end{equation}
%% --
The following theorem is an immediate consequence of \eqref{eq:c3-3.19}, Theorem~\ref{thm:c3-3.12} and A-III, Proposition~4.2.

\begin{theorem}\label{thm:c3-3.14}
	Let $\TT$ be a positive semigroup on a Banach lattice with generator $A$. 
	If $s(A)$ is a pole of finite algebraic multiplicity, then $\sigma_{b}(A)$ is a finite union of discrete subgroups (\ie  $\sigma_{b}(A) = s(A) + \cup_{k=1}^{N}\im\alpha_{k}\Z$ with $\alpha_{k} \in \R )$.
	
	Moreover, $\sigma_{b}$ contains only poles of finite algebraic multiplicity.
\end{theorem}

Here the assumption that the multiplicity of $s(A)$ is finite is essential as can be seen from the following example.

\begin{example}\label{ex:c3-3.15}
	Consider $X \coloneqq  [0,1] \times V$, $V \coloneqq  \{v \in \R  \colon v_{1} < |v| < v_{2}\}$ $(0 < v_{1} < v_{2} < \infty)$. 
	The flow in the phase space $X$ which describes the free motion in the interval $[0,1]$ with velocities in $V$ assuming that the particles are reflected at the endpoints generates a positive semigroup on $L^{p}(X,\mu)$ $(\mu$ the Lebesgue measure). 
	For the spectrum of the generator $A$ one obtains 
    \[
    \sigma(A) = \{\im \gamma \colon n\gamma_{1} \leq |\gamma| \leq n\gamma_{2} \text{ for some } n \in \N_{0}\}
    \]
    with $\gamma_{1} \coloneqq  \pi/v_{1}$, $\gamma_{2} \coloneqq  \pi/v_{2}$. 
	Moreover, $0$ is a first order pole of the resolvent, obviously the only pole in $\sigma_{b}(A) = \sigma(A)$.
	These statements can be verified by calculating the resolvent explicitely. 
	This can be done using the integral representation. 
    The semigroup is given as follows.
%% --
\begin{equation}\label{eq:c3-3.20}
	\textit{\small $
	(T(t)f)(x,v) = \begin{cases}
		f(x-vt+k,v) & \text{if } k-1\leq vt-x\leq k \text{ and } k \textit{ even}, \\
		f(1-(x-vt+k),-v) & \text{if } k-1\leq vt-x\leq k \text{ and } k \text{ odd}.
	\end{cases}
$}
\end{equation} 

Obviously one can apply Theorem~\ref{thm:c3-3.12} and Theorem~\ref{thm:c3-3.14}, respectively, in order to prove existence of strictly dominant eigenvalues. 
We consider two typical cases in the following corollaries. 
The meaning of $r_{ess}(T(t))$ and $\omega_{ess}(\TT)$ and its properties are explained in A-III, Example~3.7.
\end{example}

\begin{corollary}\label{cor:c3-3.16}
	Suppose that $\TT$ is a positive semigroup such that $\omega_{ess}(\TT) < \omega_0(\TT)$. 
	Then $s(A) = \omega_0(\TT)$ is a strictly dominant eigenvalue.
	If, in addition, there exists an eigenfunction which is a quasi-interior point of $E_{+}$ (\eg if $\TT$ is irreducible), then $s(A)$ is a first order pole of $R(.,A)$.
\end{corollary}
%% --
\begin{proof}
	There exist $\epsilon > 0$ such that 
    %for every $t > 0 the set 
    $\{\lambda \in \sigma(T(t)) \colon |\lambda| \geq \exp((s(A)-\epsilon)t)\}$ contains only (finitely many) poles of $R(\cdot,T(t))$ each being of finite algebraic multiplicity. 
	In view of A-III, Corollary~6.5 the set $\{\lambda \in \sigma(A) \colon \Re \lambda > s(A)-\epsilon\}$ is finite and contains only poles of $R(\cdot,A)$. 
	Thus we can apply Theorem~\ref{thm:c3-3.14}. 
	It follows that $s(A)$ is strictly dominant.
	For the final assertion we refer to Remark~\ref{rem:c3-2.15}(b).
\end{proof}

\begin{corollary}\label{cor:c3-3.17}
	Suppose that $\TT$ is an irreducible, eventually norm continuous semigroup having compact resolvent.
	Then $s(A) = \omega_0(\TT)$ is an algebraically simple pole and a strictly dominant eigenvalue.
\end{corollary}
%% --
\begin{proof}
	By Theorem~\ref{thm:c3-3.7}(iii) we know that $s(A) > -\infty$. 
	Theorem~\ref{thm:c3-3.12} is applicable since we assumed that $T$ is irreducible and has compact resolvent.
	Thus $s(A)$ is an algebraically simple pole and $\sigma_{b}(A) = s(A) + \im\alpha\Z$ for some $\alpha \geq 0$. 
	In addition $\{\lambda \in \sigma(A) \colon \Re \lambda \geq -1\}$ is compact since $T$ is eventually norm-continuous (see A-II, Theorem~1.20). 
	It follows that $s(A)$ is strictly dominant.
	By A-III, Theorem~6.6 we have $s(A) = \omega_0(\TT)$.
	%$\square$
\end{proof}

In the following proposition we give a condition which ensures that Theorem~\ref{thm:c3-3.14} can be applied to certain perturbations. 
Moreover, we state a criterion ensuring the existence of a dominant eigenvalue.
%% --
\begin{proposition}\label{prop:c3-3.18}
	Suppose that $A$ is the generator of a positive semigroup and that $K \in \mathcal{L}(E)$ is a positive linear operator.
	If $K$ is A-compact (\ie  if $KR(\lambda_{o},A)$ is compact for some $\lambda_{o} \in \rho(A))$ and if $s(A+K) > s(A)$, then $B \coloneqq  A + K$ satisfies the assumptions of Theorem~\ref{thm:c3-3.14}.
	
	If, in addition, $K$ is irreducible, then $s(B)$ is a dominant eigenvalue and the semigroup generated by $B$ is irreducible.
\end{proposition}
%% --
\begin{proof}
The resolvent equation $R(\lambda,A) = R(\lambda_{0},A)(1 - (\lambda-\lambda_{0})R(\lambda,A))$ implies that $KR(\lambda,A)$ is a compact operator for every $\lambda \in \rho(A)$. 
For $\lambda > s(A)$ we have $\lambda - B = (1 - KR(\lambda,A))(\lambda-A)$ and $(1 - KR(\lambda,A))^{-1}$ exists for $\lambda > s(B)$.
Therefore \citet[Theorem XIII.13]{reedsimon:1979} 
implies that $R(\lambda,B) = R(\lambda,A)(1 - KR(\lambda,A))^{-1}$ has only poles of finite algebraic multiplicity in $\{\lambda \in \C  \colon \Re \lambda > s(A)\}$. 
This proves the first claim.

In order to prove the second, we denote the semigroup corresponding to $A$ and $B$ by $(T(t))$ and $(S(t))$,  respectively.
It follows from Proposition~\ref{prop:c3-3.3} that $(S(t))$ is irreducible and we have $S(t) = T(t) + \int_{0}^{t} T(t-s)KS(s)\ds$ (see A-II, (1.9)). 
Iterating this identity we obtain for every $m \in \N$, $t \geq 0$
%% --
\begin{align}\label{eq:c3-3.21}
    S(t) &= \sum_{n=0}^{m-1} T_{n}(t) + R_{m}(t) \ \text{ where}\\\notag
    T_{0}(t)&\coloneqq T(t),\  T_{n}(t) \coloneqq  \int_{0}^{t} T(t-s)KT_{n-1}(s)\ds \text{ for } n \in \N\,,\\\notag
    R_{m}(t) &\coloneqq \textit{\small $\int_{0}^{t}\int_{0}^{t_{1}} .. \int_{0}^{t_{m-1}} T(t-t_{1})KT(t_{1}-t_{2})K .. T(t_{m-1}-t_m)KS(t_{m})\dt _{m}..\dt _{1}$}\,.
\end{align}
%% --
We fix $0 < f \in E$, $0 < \phi \in E'$, $t > 0$. 
By Theorem~\ref{thm:c3-3.2}(i), $S(t)f > 0$.
Since $K$ is irreducible, there exists $m \in \N$ such that $\langle K^{m}f,\phi\rangle > 0$. 
Thus the integrand appearing in the the representation \eqref{eq:c3-3.21} of $\langle R_{m}(t)f,\phi\rangle$ is non-zero at $t_{1}=t_{2}= .. =t_{m-1}=t$, $t_{m}= t$.

Since the integrand is positive and continuous, we conclude
\begin{equation}\label{eq:c3-3.22}%\textstyle
    \langle S(t)f,\phi\rangle \geq \langle R_{m}(t)f,\phi\rangle > 0 \ \text{ for } \ 0 < f \,,  \ 0 < \phi \ \text{ and } \ t > 0\,.
\end{equation}
%% --
It follows that $(\mathrm{e}^{-ts(B)}S(t))_{t \geq 0}$ cannot contain the rotation semigroup on $\Gamma$. 
On the other hand, assuming that $s(B)$ is not dominant, then $\dim(\ker((\mathrm{e}^{\tau\cdot s(B)} - S(\tau))) > 1$ for some $\tau > 0$. 
Hence, the restriction $(\mathrm{e}^{-ts(B)}S(t)_{|F})_{t \geq 0}$, $F \coloneqq  \ker(\mathrm{e}^{\tau\cdot s(B)}- S(\tau))$, contains the rotation semigroup by Corollary~\ref{cor:c3-3.9}.
\end{proof}

We conclude this section considering once again Example~\ref{ex:c3-3.4}(d).
The generator considered there is $B = (A - M) + K$, where $K$ is positive linear. 
From \eqref{eq:c3-3.5} and \eqref{eq:c3-3.6} one deduces that 
\[
(KR(\lambda,A)f)(x,v) = \int_{0}^{1}\int_{-1}^{1} k(x,v,x',v')f(x',v')\dx'dv'
\]
where the kernel $k$ is given by $k(x,v,x',v') \coloneqq  \kappa(x,v,v')r_{\lambda}(x,x',v')$ (\cf \eqref{eq:c3-3.5}, \eqref{eq:c3-3.7}). 
Using this representation of $KR(\lambda,A)$ it follows that $K$ is A-compact. 
Moreover for $\lambda$ sufficiently large one has $R(\lambda,A-M) = R(\lambda,A)(1 - MR(\lambda,A))^{-1}$ which shows that $K$ is also $(A-M)$-compact. 
In order to apply Theorem~\ref{thm:c3-3.14} one needs $s(B) > s(A-M)$ (see Proposition~\ref{prop:c3-3.18}) which is difficult to verify. 
In case the function $\sigma$ is continuous one can state a sufficient condition as follows.
There exist $r \in \R $ and $g \in L^{1}([0,1]\times[-1,1])$, $g > 0$ such that $r < \inf\{\sigma(x,0) \colon x \in [0,1]\}$ and $Bg \geq -rg$.
The additional assumption made in the second part of Proposition~\ref{prop:c3-3.18} is not satisfied in this example. 
Nevertheless one can show that $s(B)$ is strictly dominant in this situation (provided that $s(B) > s(A)$). 
For details we refer to \citet{greiner:1984d}  or \citet{voigt:1985} where the linear transport equation in higher dimensional spaces is discussed.


\section{Semigroups of Lattice Homomorphisms}\label{sec:c3-4}
%\index{Lattice Homomorphism}
%\index{Theory!Lattice Homomorphism}

In Section 2 we proved that the boundary spectrum of certain positive semigroups is a cyclic set. 
For semigroups of lattice homomorphisms much more can be said. The whole spectrum is an imaginary additively cyclic subset of $\C $ (\cf Theorem~\ref{thm:c3-4.2}). 
This result can be used to derive cyclicity results for the eigenvalues in the boundary spectrum of positive semigroups (\cf Corollary~\ref{cor:c3-4.3}). 
In the last part of this section we discuss a spectral decomposition of positive groups (\cf Theorem~\ref{thm:c3-4.8}).

\begin{lemma}\label{lem:c3-4.1}
	Suppose that $(T(t))_{t \ge 0}$ is a semigroup of lattice homomorphisms on a Banach lattice $E$ with generator $A$.
	In case $\im\alpha \in R\sigma(A)$, $\alpha \in \R $, then \underline{one} of the following assertions are true
	\begin{enumerate}[\upshape (a)]
		\item $\im\alpha\Z \subset R\sigma(A)$;
		\item $\{\lambda \in \C  \colon \Re \lambda < 0 \} \subset R\sigma(A)$.
	\end{enumerate}
\end{lemma}
%% --
\begin{proof}
	There exists $\phi \in E'$, $\phi \neq 0$ such that $T(t)'\phi = \mathrm{e}^{\im\alpha t}\phi$ $(t \geq 0)$. 
	Then we have $|\phi| = |T(t)'\phi| \leq T(t)'|\phi|$ $(t \geq 0)$.
	If we fix $r > \omega_0(\TT)$ and define $\psi \coloneqq  rR(r,A)'|\phi|$, we have
	%% -- 
	\begin{equation}\label{eq:c3-4.1}
		T(t)'\psi \leq \mathrm{e}^{rt}\psi \ \text{ and } \ \ T(t)'\psi \geq \psi \ (t \geq 0) \ \text{ and } \ |\phi| \leq \psi\,.
	\end{equation}
	%% --
	In fact, A-I,(3.1) implies $(\mathrm{e}^{rt} - T(r))R(r,A) \geq 0$ hence 
    \[
    T(t)'\psi = rR(r,A)'T(t)'|\phi| \leq r\cdot \mathrm{e}^{rt}R(r,A)'|\phi| = \mathrm{e}^{rt}\psi\,.
    \]
	Moreover, the inequality $T(t)'|\phi| \geq |\phi|$ $(t \geq 0)$ implies 
    \[
    T(t)'\psi = rR(r,A)'T(t)'|\phi| \geq rR(r,A)'|\phi| = \psi
    \]
    and 
    \[
    \psi = rR(r,A)'|\phi| = r \int_{0}^{\infty} \mathrm{e}^{-rt}T(t)'|\phi|\dt  \geq r\int_{0}^{\infty} \mathrm{e}^{-rt}|\phi|\dt  = |\phi|\,.
    \]
    %% --
	Now, considering the AL-space $(E,\psi)$ (see C-I, Section~4) the first inequality of \eqref{eq:c3-4.1} implies that $(T(t))_{t \geq 0}$ induces a strongly continuous semigroup $(T_{1}(t))_{t \geq 0}$ on $(E,\psi)$.
	That is we have
	%% --
	\begin{equation}\label{eq:c3-4.2}
    \textstyle
		T_{1}(t)\circ q_{\psi} = q_{\psi}\circ T(t) \ (t \geq 0)\,.
	\end{equation}
	%% --
	\[
	\begin{tikzcd}
		E \arrow[r, "T(t)"] \arrow[d, "q_\psi"'] & E \arrow[d, "q_\psi"] \\
		(E,\psi) \arrow[r, "T_1(t)"'] & (E,\psi))
	\end{tikzcd}
	\]
	%% --
	Denoting by $A_{1}$ the generator of $(T_{1}(t))$ we have $R\sigma(A_{1}) \subset R\sigma(A)$.
	Indeed, assuming $A_{1}^*x = \lambda x$ then $T_{1}(t)'x = \mathrm{e}^{\lambda t}x$, hence  $T(t)'q_{\psi}'(x) = \mathrm{e}^{\lambda t}q_{\psi}'(x)$ (by \eqref{eq:c3-4.2}) or equivalently, $A^*(q_{\psi}'(x)) = q_{\psi}'(x)$. 
	Thus it remains to show that either $\im\alpha\Z$ or $\{\lambda \in \C  \colon \Re \lambda < 0 \}$ is contained in $R\sigma(A_{1})$. 
	
	Obviously, $(T_{1}(t))$ is a semigroup of lattice homomorphisms as well. 
	The second inequality of \eqref{eq:c3-4.1} implies
	%% --
	\begin{equation}\label{eq:c3-4.3}
	\|T_{1}(t)f\|_{\psi} = \langle|T_{1}(t)f|,\psi\rangle = \langle|f|,T_{1}(t)'\psi\rangle \geq \langle|f|,\psi\rangle = \|f\|_{\psi}\,.
	\end{equation}
	%% --
	Then for $\lambda \in \C $ with $\Re \lambda < 0$ we have 
	\[ 
	\|(\mathrm{e}^{\lambda t}-T_{1}(t))f\|_{\psi} \geq \|T_{1}(t)f\|_{\psi}-\|\mathrm{e}^{\lambda t}f\|_{\psi} \geq (1-|\mathrm{e}^{\lambda t}|)\|f\|_{\psi} \ (f \in (E,\psi))
	\]
	and we obtain for the corresponding generator
	%% --
	\begin{equation}\label{eq:c3-4.4}
		\begin{aligned}
		\|(\lambda-A_{1})f\|_{\psi} &= \lim_{t \to 0}\left\|\frac{1}{t}(\mathrm{e}^{-\lambda t}T_{1}(t)f-f)\right\|_{\psi} \geq \lim_{t \to 0}\frac{1}{t}(\mathrm{e}^{-t\Re \lambda}-1)\|f\|_{\psi}\\
		&= -\Re \lambda\cdot\|f\|_{\psi} \ \text{ for } \ \Re \lambda < 0 \ \text{ and } \ f \in (E,\psi)\,.
		\end{aligned}
	\end{equation}
	%% --
	It follows from \eqref{eq:c3-4.3} and \eqref{eq:c3-4.4} that $A\sigma(T_{1}(t))\cap\{z \in \C  \colon |z| < 1\} = \emptyset$ and that $A\sigma(A_{1})\cap\{\lambda \in \C  \colon \Re \lambda < 0\} = \emptyset$. 
	Since the toplogical boundary of the spectrum is always contained in the approximate point spectrum (see A-III, Proposition~2.2) and $R\sigma(T(t))\backslash\{0\} = \exp(tR\sigma(A))$ (see A-III, Theorem~6.3), precisely one of the following two cases occurs.
	\begin{enumerate}[\upshape (A)]
		\item
		$\{\lambda \in \C  \colon \Re \lambda < 0\} \subset \rho(A_{1})$ and $\{z \in \C  \colon |z| < 1\} \subset \rho(T_{1}(t))$;
		
		\item
		$\{\lambda \in \C  \colon \Re \lambda < 0\} \subset R\sigma(A_{1})$ and $\{z \in \C  \colon |z| < 1\} \subset R\sigma(T_{1}(t))$.
	\end{enumerate}
	
	We mentioned above that $R\sigma(A_{1}) \subset R\sigma(A)$. 
	Thus we only have to analyze case (A). 
	In this case each operator $T_{1}(t)$ is an invertible lattice homomorphism hence a lattice isomorphism. 
	It follows that $T_{1}(t)'$ is a lattice isomorphism as well. 
	The third inequality in \eqref{eq:c3-4.1} implies that $\phi$ can be considered as an element of $(E,\psi)'$ and $T(t)'\phi = \mathrm{e}^{\im\alpha t}\phi$ $(t \geq 0)$ implies $T_{1}(t)'\phi = \mathrm{e}^{\im\alpha t}\phi$. 
	Furthermore, we have
	$T_{1}(t)'|\phi| = |T_{1}(t)'\phi| = |\mathrm{e}^{\im\alpha t}\phi|$ or equivalently $A_{1}^*|\phi| = 0$.
	Now we can apply 
	Theorem~\ref{thm:c3-2.2} 
	and obtain $\im\alpha\Z \subset P\sigma(A_{1}^*) = R\sigma(A_{1})$.
	\end{proof}

\begin{theorem}\label{thm:c3-4.2}
	Let $A$ be the generator of a semigroup $(T(t))_{t \geq 0}$ of lattice homomorphisms on a Banach lattice $E$. 
	Then $\sigma(A)$, $A\sigma(A)$ and $P\sigma(A)$ are imaginary additively cyclic subsets of $\C $.
\end{theorem}

\begin{proof}
	We first consider the point spectrum. 
	If $\lambda \in P\sigma(A)$, $\lambda = \alpha + \im\beta$ $(\alpha,\beta \in \R )$, then there exists $f \in E$, $f \neq 0$ such that $Af = \lambda f$. 
	It follows that $T(t)f = \mathrm{e}^{\lambda t}f$ $(t \geq 0)$ hence $T(t)|f| = |T(t)f| = \mathrm{e}^{\alpha t}|f|$ $(t \geq 0)$, or equivalently, $A|f| = \alpha|f|$. 
	Now Theorem~\ref{thm:c3-2.2} is applicable and we obtain $A(f^{[n]}) = (\alpha + \im n\beta)f^{[n]}$ for all $n \in \Z$.
	
	To prove the assertion for $A\sigma(A)$ we consider an $\mathcal{F}$-product semigroup in order to reduce the problem to the point spectrum. 
	We use the notation of A-I, Section~3.7. 
	Obviously the space $m(E)$ is a Banach lattice and every operator $\hat{T}(t)$ is a lattice homomorphism. 
	We have $|T(t)|f| - |f|| = ||T(t)f| - |f|| \leq |T(t)f - f|$ $(f \in E)$, hence $(|f_{n}|) \in m^{\TT}(E)$ whenever $(f_{n}) \in m^{\TT}(E)$. 
	This proves that $m^{\TT}(E)$ is a sublattice, hence a Banach lattice as well. 
	Obviously, $c_{\mathcal{F}}(E)\cap m^{\TT}(E)$ is an order ideal. 
	Thus $E_{\mathcal{F}}^{\TT}$ is a Banach lattice and $(T_{\mathcal{F}}(t))$ is a semigroup of lattice homomorphisms. 
	It follows that $P\sigma(A_{\mathcal{F}})$ is cyclic, hence $A\sigma(A)$ is cyclic by A-III, Section~4.2.
	
	Cyclicity of the entire spectrum follows from the cyclicity of $A\sigma(A)$ and Lemma~\ref{lem:c3-4.1}.
\end{proof}

One can use Theorem~\ref{thm:c3-4.2} in order to prove cyclicity for the eigenvalues in the boundary spectrum of positive semigroups. 
We list some typical cases in the following corollary.

\begin{corollary}\label{cor:c3-4.3}
	Let $\TT = (T(t))_{t \geq 0}$ be a positive semigroup on a Banach lattice $E$ which is bounded. 
	Each of the following conditions implies that $P\sigma(A)\cap i\R $ is imaginary additively cyclic.
	\begin{enumerate}[\upshape (i)]
		\item E is weakly sequentially complete (\eg $E = L^{p}(\mu)$, $1 \leq p < \infty$);
		\item Every operator $T(t)$ is mean ergodic (\ie the Césaro means $\frac{1}{n}\sum_{k=0}^{n-1}T(t)^{k}$ converge strongly as $n \to  \infty$);
		\item There is a strictly positive linear form which is $\TT$-invariant.
	\end{enumerate}
\end{corollary}
%% --
\begin{proof}
	We will show that each of the conditions (i), (ii), (iii) implies that $\ker(1 - T(s))$ is a Banach lattice (not necessarily a sublattice of $E$) for every $s \geq 0$. 
	Then one argues as follows. Given $\im\alpha \in P\sigma(A)$, $\alpha \in \R $, then $T(t)g = \mathrm{e}^{\im\alpha t}g$ for suitable $g \neq 0$. 
	For $\tau \coloneqq  2\pi|\alpha|^{-1}$ we have $g \in F \coloneqq  \ker (1 - T(\tau))$. 
	Then the restriction $(T(t)_{|F})_{t \geq 0}$ is a $\tau$-periodic positive semigroup on $F$. 
	Since $\left(T(t)_{|F}\right)^{-1} = T(n\tau-t)_{|F} \geq 0$, it follows that $(T(t)_{|F})$ is a semigroup of lattice isomorphisms. 
	Since $g \in F$, we have $\im\alpha \in P\sigma(A_{|F})$ hence $\im\alpha\Z \in P\sigma(A_{|F}) \subset P\sigma(A)$ by Theorem~\ref{thm:c3-4.2}.
	
	Now we show that $\ker (1 - T(s))$ is a vector lattice for the induced order and a Banach lattice for an equivalent norm.
	
	In case (iii), $\ker (1 - T(s))$ is even a sublattice of $E$. 
	Indeed, assume $T(t)'\phi = \phi$ and $\phi \gg 0\,, (t \geq 0)$, then $T(s)f = f$ implies $T(s)|f| \geq |f|$. 
	Thus from $\langle T(s)|f| - |f|,\phi\rangle = \langle |f|,T(s)'\phi - \phi \rangle = 0$ it follows that $T(s)|f| = |f|$.
	
	Now we assume that $E$ is weakly sequentially complete, which is equivalent to (\cf Section~5 of Chapter C-I)
	%% --
	\begin{equation}\label{eq:c3-4.5}
	\textit{Every increasing norm-bounded net of $E_{+}$ converges}.
	\end{equation}
	%% --
	We fix $s > 0$ and define $T \coloneqq  T(s)$ and $F \coloneqq  \ker (1 - T)$, . 
	Obviously $f \in F$ implies $\bar{f} \in F$ hence $F = F\cap E_{\R } + iF\cap E_{\R }$. 
	Thus we have to show that $F_{\R } = F\cap\mathbb{E}_{\R }$ is a sublattice. 
	Given $f \in F_{\R }$, then $Tf = f$ hence $|f| \leq T|f|$. 
	Iterating this inequality we obtain $|f| \leq T|f| \leq T^{2}|f| \leq T^{3}|f| \leq \dots\,.$ 
	By \eqref{eq:c3-4.5} $|f|_{o} \coloneqq  \lim_{n\to \infty} T^{n}|f|$ exists and we have $T|f|_{o} = \lim_{n\to \infty} T^{n+1}|f| = |f|_{o}$, \ie $|f|_{o} \in F_{\R }$.
	And, for $g \in F_{\R }$ satisfying $\pm f \leq g$, we have $|f|_{o} \leq g$ thus $|f|_{o} = \sup_{F}\{f,-f\}$. 
	Moreover, $\|f\|_{o} \coloneqq  \||f|_{o}\|$ $(f \in F)$ is an equivalent norm on $F$ such that $(F,\|\cdot\|_{o})$ is a Banach lattice.
	
	(ii) If $T(s)$ is mean-ergodic, then we have $\ker (1 - T(s)) = PE$ where $P$ is the mean-ergodic projection, \ie $Pf = \lim_{n\to \infty}\frac{1}{n}\sum_{k=0}^{n-1}T(s)^{k}f$.
	Obviously $P$ is positive, hence II.11.5 of \citet{schaefer:1974} implies that $PE$ is a Banach lattice (for the induced order and an equivalent norm).
	%$\square$
\end{proof}
%% --
The assumptions made in Corollary~\ref{cor:c3-4.3} can be weakened slightly (\cf \citet{greiner:1982}). 
However, one still cannot prove cyclicity of $P\sigma_{b}(A)$ for arbitrary positive semigroups.

\begin{example}\label{ex:c3-4.4}
	At first we recall Example~2.13 of Chapter B-III. 
	There we constructed a bounded semigroup on $C(\Gamma)\times C_{0}(\R )$ such that $P\sigma_{b}(A) = \{\im k \colon k \in \Z, k \neq 0\}$.

    Let us perform the same construction on the Hilbert space
    $H \coloneqq L^2(\Gamma)\times L^2(\R )$.
    For a fixed positive, non-zero function $k \in C_c(\R )$,
    we define $T(t)$ on $H$ as follows.
    %% --
\begin{equation}\label{eq:c3-4.6}
	\begin{aligned} 
		T(t)([f,g]) &\coloneqq [f_t,g_t] \quad \ \text{with}\\
		f_t(z) &\coloneqq f(z \cdot \mathrm{e}^{it}) \quad (z \in \Gamma) \quad \text{and}\\
		g_t(x) &\coloneqq g(x+t) + \frac{1}{2\pi}\int_{0}^{2\pi} f(z \cdot \mathrm{e}^{is}) \ds  \cdot \int_{x}^{x+t} k(u) \du \,.
	\end{aligned}
\end{equation}
%% --
Then $(T(t))_{t \geq 0}$ is a positve semigroup on $H$ and for the spectrum of
the generator we obtain $\sigma(A) = i\R $, $P\sigma(A) = i\R \backslash\{0\}$.
In view of Corollary~\ref{cor:c3-4.3}(i) the semigroup cannot be bounded.
(The explicit representation \eqref{eq:c3-4.6} only allows the estimate $\|T(t)\| \leq \sqrt{2} + t \cdot \|k\|_2$ $(t\geq 0)$.)
\end{example}

In the next proposition we show that for semigroups of lattice homomorphisms on $L^1$-spaces, there is a spectral mapping theorem for the
real part of the spectrum.

\begin{proposition}\label{prop:c3-4.5}
%\index{Spectral mapping theorem}
%\index{Theorem!Spectral mapping}
%\index{Examples!Spectral mapping theorem}
Let $(T(t))_{t \geq 0}$ be a strongly continuous semigroup
of lattice homomorphisms on an $L^1$-space and denote by $A$ its generator.
Then we have
%% --
\begin{equation}\label{eq:c3-4.7}
	\exp(t\sigma(A) \cap \R ) = \sigma(T(t)) \cap (0,\infty) \quad \text{for every} \quad t \geq 0\,.
\end{equation}
%% --
\end{proposition}
%% --
\begin{proof}
In view of A-III,6.2 it is enough to prove that the left hand
side contains the set on the right.

Fix $t > 0$ and assume $r \in \sigma(T(t))$, $r > 0$ and let $\alpha \coloneqq \frac{1}{t} \log r$.
At first we assume $r \in R(\sigma(T(t)))$.
Then by A-III, Theorem~6.3 there exists
$\beta \in \R $ such that $\alpha+\im\beta \in R\sigma(A)$.
By Lemma~\ref{lem:c3-4.1} either $\alpha+\im\beta\Z \subset R\sigma(A)$
or $\{\lambda \in \C  \colon \Re  \lambda < \alpha\} \subset R\sigma(A)$.
In both cases we have $\alpha \in \sigma(A)$.

Now we assume $r \in A\sigma(T(t))$. 
Then there exists a normalized sequence
$(f_n) \subset E$ such that $\lim_{n \to \infty}\|T(t)f_n - rf_n\| = 0$.
Since we have
%% --
\[
|(T(t)|f| - r|f|)| = |(|T(t)f| - r|f|)| \leq |T(t)f - rf|, \  (f \in E) 
\]
%% --
we may assume that $(f_n)$ is a sequence in $E_+$.

Defining $g_n \coloneqq \int_{0}^{t} \mathrm{e}^{-\alpha s}T(s)f_n \ds $ we have $g_n \in D(A)$ and
\[
(A-\alpha)g_n = (A-\alpha)\int_{0}^{t} \mathrm{e}^{-\alpha s}T(s)f_n \ds = \mathrm{e}^{-\alpha t}T(t)f_n - f_n = \frac{1}{r}(T(t)f_n - rf_n)\,.
\]
Therefore $\lim_{n \to \infty}\|(A - \alpha) g_n\| = 0$\,. It remains to prove 
$\liminf_{n \to \infty}\|g_n\| > 0$\,.
The latter assertion is a consequence of the following facts.
\begin{enumerate}[\upshape (i),wide, labelindent=.5em]
	\item
	Since $f_n$ is positive and the norm is additive on $E_+$,
	we have
	\[
    \|g_n\| = \int_{0}^{t} \mathrm{e}^{-\alpha s}\|T(s)f_n\| \ds\,.
    \]
	
	\item 
    The inequality \quad
    $\displaystyle
    \|T(t)f\| \leq \|T(t-s)\|\|T(s)f\|
    $\quad 
    implies
	\[
    \|T(s)f\| \geq M^{-1}\|T(t)f\|
    \]
    \[
    \text{for } \ 0 \leq s \leq t\,, f \in E \ \text{ and } \ M \coloneqq \sup_{0 \leq s \leq t}\|T(s) \|\,.
    \]
    
	 \item 
	 \ \text{Since} \  
     $\displaystyle\lim_{n \to \infty}\|T(t)f_n - rf_n\| = 0$ 
     and $\|f_n\| = 1$, we have
    $\displaystyle
     \lim_{n \to \infty}\|T(t)f_n\| = r > 0\,.
    $
\end{enumerate}
\end{proof}
%% --
For semigroups satisfying the assumption of Proposition~\ref{prop:c3-4.5}, the spectrum $\sigma(A)$ is additively cyclic (by Theorem~\ref{thm:c3-4.2}) and $\sigma(T(t))$ is multiplicatively cyclic
(by \citet[V.Theorem~4.4]{schaefer:1974}). 
Then the relation \eqref{eq:c3-4.7} implies that
decompositions of the spectrum by vertical lines allow a spectral
decomposition of the semigroup (\cf A-III, Definition~3.1) (One simply performs a spectral decomposition of a single operator $T(t)$). 

In the
following we will show that for positive groups (on arbitrary Banach
lattices) spectral decompositions of this type always exist. Moreover,
it will turn out that the decomposition is compatible with the lattice
structure. The proof of this result uses Kato's equality (see Section~5 of
C-II). As a consequence of C-II, Corollary~5.8 we have the following.

Let $E$ be a Banach lattice with order continuous norm and $(T(t))_{t \in \R }$
be a group of positive operators on $E$ with generator $A$.
Then the domain $D(A)$ is a sublattice of $E$ and
%% --
\begin{equation}\label{eq:c3-4.8}
	A|f| = \Re [(\text{sign\,} f)Af] \ \text{ for every } \ f \in D(A)\,,
\end{equation}
%% --
For real $\mu$ one has $\mu|f| = \Re [(\text{sign } f)\mu f]$, hence
%% --
\[
(\mu - A)|f| = \Re[(\text{sign } f)(\mu - A)f] \ \ \text{ for } \ \mu \in \R , f \in D(A)\,.
\]
%% --
The relations $f^+ = \frac{1}{2}(|f| + f), \ f^- = \frac{1}{2}(|f| - f)$ yield
%% --
\begin{align*}
(\mu - A)f^+ &= \frac{1}{2}[(\text{sign } f)(\mu - A)f + (\mu - A)f] \quad\ \text{and }\\
(\mu - A)f^- &= \frac{1}{2}[(\text{sign } f)(\mu - A)f - (\mu - A)f]\,,
\end{align*}
%% --
in case $f$ is contained in the underlying real Banach lattice $E_{\R }$.
For $\mu \in \rho(A) \cap \R $, we can apply $R(\mu,A)$ on both sides and the substitution $f = R(\mu,A)g$ finally leads to
%% --
\begin{equation}\label{eq:c3-4.9}
	\begin{aligned}
	(R(\mu,A)g)^+ &= \frac{1}{2}R(\mu,A) [(\text{sign } R(\mu,A)g)g + g] \\
	(R(\mu,A)g)^- &= \frac{1}{2}R(\mu,A) [(\text{sign } R(\mu,A)g)g - g]
	\end{aligned}
\end{equation}
%% --
for all $g \in E_{\R }$. If we set 
\[ 
g_1 \coloneqq \frac{1}{2}\cdot(g + (\text{sign } R(\mu,A)g)g) \ \text{ and } \ 
g_2 \coloneqq \frac{1}{2}\cdot(g - (\text{sign } R(\mu,A)g)g)\,,
\]
then obviously $g = g_1 + g_2$. Moreover, $g$ is positive if and only if
both, $g_1$ and $g_2$ are positive. We summarize these considerations in the following lemma.
%%  bis hierher
\begin{lemma}\label{lem:c3-4.6}
%\index{Positive group}
%\index{Group!Positive}
%\index{Banach lattice}
Let $A$ be the generator of a positive group on a Banach
lattice $E$ which has order continuous norm. Given $\mu \in \rho(A) \cap \R $, then
every $g \in E_{\R }$ is representable as sum of two elements $g_1$ and $g_2$
such that
\begin{enumerate}[\upshape (i)]
	\item 
	$g \geq 0$ if and only if both $g_1$ and $g_2$ are positive,
	
	\item
	$R(\mu,A)g_1 = (R(\mu,A)g)^+$\,,

	\item
	$R(\mu,A)g_2 = -(R(\mu,A)g)^-$\,.
\end{enumerate}	
\end{lemma}
%% --
We need another lemma. It can be formulated for arbitrary positive
semigroups on Banach lattices.

\begin{lemma}\label{lem:c3-4.7}
%\index{Positive semigroup}
%\index{Semigroup!Positive}
Let $(T(t))_{t \geq 0}$ be a positive semigroup on a Banach
lattice $E$ with generator $A$. Given $\mu \in \rho(A) \cap \R $ and $h \in E_+$, then the following assertions are equivalent.
\begin{enumerate}[\upshape (a)]
	\item 
	$R(\mu,A)h \geq 0$\,,

	\item 
	$\left\{\int_{0}^{t} \mathrm{e}^{-\mu s}T(s)h \ds \colon t \in \R _+\right\}$ is bounded in $E$.
\end{enumerate}
\end{lemma}
%% --
\begin{proof}
$(a)\Rightarrow(b)$  We have
$\int_{0}^{t} \mathrm{e}^{-\mu s}T(s)h \ds = (Id - \mathrm{e}^{-\mu t}T(t))R(\mu,A)h$ (see A-I,(3.2)).
Since $R(\mu,A)h \geq 0$ and $T(t)$ is a positive operator we obtain
\[\textstyle
\int_{0}^{t} \mathrm{e}^{-\mu s}T(s)h \ds = R(\mu,A)h - \mathrm{e}^{-\mu t}T(t)R(\mu,A)h \leq R(\mu,A)h
\] 
which implies assertion (b).

$(b)\Rightarrow(a)$  The assumption implies that 
\[\textstyle
\int_{0}^{\infty} \mathrm{e}^{-\nu s}T(s)h \ds \coloneqq 
\lim_{t \to \infty} \int_{0}^{t} \mathrm{e}^{-\nu s}T(s)h \ds \ \text{ exists for } \ \nu > \mu\,. 
\]
Using that $A$ is a closed operator it follows that 
\[\textstyle
(\nu - A)\left(\int_{0}^{\infty} \mathrm{e}^{-\nu s}T(s)h \ds\right) = h\,.
\]
For $\nu$
sufficiently close to $\mu$ such that $\nu \in \rho(A) \cap \R $ we have 
\[\textstyle
R(\nu,A)h = \int_{0}^{\infty} \mathrm{e}^{-\nu s}T(s)h \ds \geq 0\,. 
\] 
By continuity we conclude $R(\mu,A)h \geq 0$.
\end{proof}
%% --
By now we are prepared to prove the spectral decomposition for positive groups. Before we formulate the theorem we recall the following
consequence of Theorem~\ref{thm:c3-4.2}. For any $\mu \in \rho(A) \cap \R $ the line $\mu+i\R $ is a
subset of the resolvent set and divides $\sigma(A)$ into disjoint sets.
Both sets will be unbounded in general.

\begin{theorem} \label{thm:c3-4.8}
%\index{Spectral decomposition}
%\index{Decomposition!Spectral}
%\index{Positive group}
Let $(T(t))_{t \in \R }$ be strongly continuous group of positive
operators on a Banach lattice $E$ with order continuous norm.
If $A$ is the generator and $\mu \in \rho(A) \cap \R $, then
$I_{\mu} \coloneqq  \{f \in E \colon R(\mu,A)|f| \geq 0\}$ and $J_{\mu} \coloneqq  \{f \in E \colon R(\mu,A)|f| \leq 0\}$
are $(T(t))_{t \in \R }$-invariant projection bands, 
the direct sum of them is $E$, and the spectra of the restrictions satisfy
%% --
\begin{align*}
	\sigma(A_{|I_{\mu}}) &= \sigma(A) \cap \{\lambda \in \C  \colon \Re  \lambda < \mu\} ,\\
	\sigma(A_{|J_{\mu}}) &= \sigma(A) \cap \{\lambda \in \C  \colon \Re  \lambda > \mu\} .
\end{align*}
\end{theorem}
%% --
\begin{proof}
At first we consider $I_{\mu}$. Obviously it is a closed subset.
From Lemma~\ref{lem:c3-4.7} we deduce that it is a lattice ideal. Moreover, $I_{\mu}$
is $R(\mu,A)$-invariant and $(T(t))_{t \in \R }$-invariant as well (use Lemma~\ref{lem:c3-4.7}
again).

Since $-A$ is the generator of the positive group $(T(-t))_{t \in \R }$ and
%
\[
	 J_{\mu} = \{f \in E \colon R(-\mu,-A)|f| \geq 0\} ,
\]
%
$J_{\mu}$ has the same properties.

If $f \in I_{\mu} \cap J_{\mu}$, then $R(\mu,A)|f| = 0$, hence $f = 0$ which shows that
$I_{\mu} \cap J_{\mu} = \{0\}$. On the other hand, decomposing $0 \leq h = h_1 + h_2 \in E_+$
according to Lemma~\ref{lem:c3-4.6}, then assertion (ii) of this lemma implies
that $h_1 \in I_{\mu}$, while assertion (iii) ensures that $h_2 \in J_{\mu}$. Since the
positive cone $E_+$ is generating, we have $E = I_{\mu} \oplus J_{\mu}$ and the first part
of the theorem is proved.

Since $I_{\mu}$ is $R(\mu,A)$-invariant, we have $\mu \in \rho(A_{|I_{\mu}})$ and
$R(\mu,A_{|I_{\mu}}) = R(\mu,A)_{|I_{\mu}} \geq 0$. Theorem~\ref{thm:c3-1.1}(ii) then implies
$\sigma(A|_{I_{\mu}}) \subset \{\lambda \in \C  \colon \Re  \lambda < \mu\}$. The same argument applied to $-A$ and
$-\mu$ yields $\sigma(A_{|J_{\mu}}) \subset \{\lambda \in \C  \colon \Re  \lambda > \mu\}$. Now the assertion follows
from A-III, Proposition~4.2.
%$\square$
\end{proof}

The spectral projections corresponding to the spectral decomposition
described above have the expected representation as an integral
'around' the spectral sets (see Corollary 3 in \citet{greiner:1984c}).

\begin{corollary}\label{cor:c3-4.9}
%\index{Spectral projections}
%\index{Projections!Spectral}
Assume that the assumptions of the theorem are satisfied, $\mu \in \rho(A) \cap \R $, $\beta > s(A)$, $\alpha < -s(-A)$. If we denote the projections corresponding to the decomposition $E = I_{\mu} \oplus J_{\mu}$ by $P_{\mu}$ and $Q_{\mu}$
(i.e., $P_{\mu}E = \ker Q_{\mu} = I_{\mu}$, $Q_{\mu}E = \ker P_{\mu} = J_{\mu}$), then for $f \in D(A^2)$ we
have
%% --
\begin{equation}\label{eq:c3-4.10}
\begin{aligned}
	P_{\mu}f &= \frac{1}{2\pi} \cdot \int_{-\infty}^{\infty} R(\mu+\im\tau,A)f \diff{\tau} - \frac{1}{2\pi} \cdot \int_{-\infty}^{\infty} R(\alpha+\im\tau,A)f \diff{\tau} ,\\
	Q_{\mu}f &= \frac{1}{2\pi} \cdot \int_{-\infty}^{\infty} R(\beta+\im\tau,A)f \diff{\tau} - \frac{1}{2\pi} \cdot \int_{-\infty}^{\infty} R(\mu+\im\tau,A)f \diff{\tau} .
\end{aligned}
\end{equation}
%% --
(The integrals appearing in \eqref{eq:c3-4.10} are improper Riemann integrals.)
\end{corollary}
We mention another consequence of Theorem~\ref{thm:c3-4.8}. Just like Proposition~\ref{prop:c3-4.5} it is a
spectral mapping theorem for the real part of the spectrum.

\begin{corollary}\label{cor:c3-4.10}
%\index{Spectral mapping theorem}
%\index{Positive group}
If $(T(t))_{t \in \R }$ is a positive group on a space $L^2$ or
$C_0(X)$ with generator $A$, then
%% --
\begin{equation}\label{eq:c3-4.11}
	\sigma(T(t)) \cap \R _+ = \exp(t\sigma(A) \cap \R ) \  \text{for every} \  t \geq 0 .
\end{equation}
%% --
\end{corollary}
%% --
\begin{proof} 
We borrow from the next chapter that for positive semigroups on
spaces $L^1$, $L^2$ or $C_0(X)$ spectral bound and growth bound coincide
(see C-IV, Theorem~1.1).

We only have to show that $\exp(t\rho(A) \cap \R ) \subset \rho(T(t)) \cap \R _+$.

If we consider a positive semigroup on an $L^2$-space, Theorem~\ref{thm:c3-4.8} can be
applied directly. Given $\mu \in \rho(A) \cap \R $, then $E = I_{\mu} \oplus J_{\mu}$ according to
Theorem~\ref{thm:c3-4.8}. 
The result mentioned above implies 
$r(T(t)_{|I_{\mu}}) < \mathrm{e}^{\mu t}$ and
$r(T(-t)_{|J_{\mu}}) < \mathrm{e}^{\mu t}$. 
Hence 
%
\[
	\sigma(T(t)_{|I_{\mu}}) \subset \{\lambda \in \C  \colon |\lambda| < \mathrm{e}^{\mu t}\}
\]
%% --
and
\[
	\sigma(T(t)_{|J_{\mu}}) = (\sigma(T(-t)_{|J_{\mu}}))^{-1} 
		\subset \{\lambda \in \C  \colon |\lambda| > \mathrm{e}^{\mu t}\}.
\]
%
Thus $\sigma(T(t)) = \sigma(T(t)_{|I_{\mu}}) \cup \sigma(T(t)_{|J_{\mu}})$ does not contain $\mathrm{e}^{\mu t}$.

In case $(T(t))$ is a positive group on $C_0(X)$, then the adjoint
group $(T(t)')$ is a group of lattice homomorphisms on $E'$. It
follows that $E^*$ is a sublattice of $C_0(X)'$ which is isomorphic to $M_b(X)$, hence an
$L^1$-space. 
The argument given for the $L^2$-space yields
$\sigma(T(t)^*) \cap \R _+ = \exp(t\sigma(A^*) \cap \R )$ for every $t \geq 0$. 
Thus the assertion follows from A-III, Section~4.1.
\end{proof}

We conclude by describing a general situation of lattice semigroups. 
In Section 4 of B-III we constructed semigroups of lattice
homomorphisms on $C_0(X)$ starting with a continuous (semi-) flow on the
locally compact space $X$ and a multiplication operator. 
One can perform similar constructions on spaces $L^p(\mu)$ for $1 \leq p < \infty$ under
certain conditions on the flow. 
We consider an example which shows where the problems are.

Define the semiflow $\phi$ on $\R _+$ as follows. 
$\phi(t,x) \coloneqq  x-t$ for $x \geq t$ and $\phi(t,x) \coloneqq  0$ for $x < t$. 
For $f \in L^p(\mu)$ one has difficulties to define $f \circ \phi_t$ properly since the preimage of the zero-set $\{0\}$ does
not have measure zero. This problem does not arise in case every
transformation $\phi_t$ is measure preserving, i.e. $\mu(\phi_t^{-1}(C)) = \mu(C)$
for every Borel set $C$. A more general criterion is stated in the
following proposition.

\begin{proposition}\label{prop:c3-4.11}
%\index{Flow!Measure preserving}
%\index{Measure preserving flow}
Let $X$ be a locally compact space and let $\mu$ be a
regular, positive Borel measure on $X$. Assume that the continuous
semiflow $\phi \colon \R _+ \times X \to X$ satisfies the condition
%% --
\begin{equation}\label{eq:c3-4.12}
	\phi_t^{-1}(K)  \text{ is compact for every compact set }  K \subset X , t \geq 0\,.
\end{equation}
%% --
\begin{enumerate}[\upshape (i)]
\item		
For every $p$, $1 \leq p < \infty$ the following assertions are equivalent.

\begin{enumerate} [(a)]
\item
The operators $T(t)$ defined by $T(t)f \coloneqq  f \circ \phi_t$ for $f \in L^p(\mu)$, $t \geq 0$, are well-defined as bounded linear operator on $L^p(\mu)$ and $(T(t))_{t \geq 0}$ is a strongly continuous semigroup.
			
\item 
There exist constants $t_0 > 0$, $M > 0$ such that $\mu(\phi_t^{-1}(C)) \leq M \cdot \mu(C)$ for every open (compact) set $C \subset X$ and every $t \leq t_0$.
\end{enumerate}
	
\item 
In case the conditions (a) and (b) are fulfilled, then $(T(t))_{t \geq 0}$ is a semigroup of lattice homomorphisms on $L^p(\mu)$ and $C_c(X) \cap D(A)$ is a core of the generator.
\end{enumerate}
\end{proposition}
%% --
\begin{proof}
\begin{enumerate}[\upshape (i), wide, labelindent=.5em]
\item
Since $\mu$ is assumed to be regular, the inequality stated in (b) holds true for all Borel sets provided it is true for all open sets (or all compact sets, respectively).
\end{enumerate}

%% --
\begin{enumerate}[wide, labelindent=.5em]

\item[$(a)\Rightarrow(b)$:] 
Assume that $(T(t))$ is a strongly continuous semigroup on
$L^p(\mu)$ with $1 \leq p < \infty$. For $t_0 > 0$ we define $M \coloneqq  (\sup\{\|T(t)\| \colon 0 \leq t \leq t_0\})^{1/p}$.
Given a Borel set $C \subset X$ we write $C(t) \coloneqq  \phi_t^{-1}(C)$.
Then we have $T(t)\1_C = \1_{C(t)}$, hence
$\mu(\phi_t^{-1}(C)) = \|\1_{C(t)}\|_p^p = \|T(t)\1_C\|_p^p \leq M \cdot \|\1_C\|_p^p = M \cdot \mu(C)$.

\item[$(b)\Rightarrow(a)$:] 
Since the inequality in (b) holds for all Borel sets,
$\phi_t^{-1}(C)$ is a $\mu$-null set whenever $C$ is a $\mu$-null set. 
Thus given
Borel functions $f$, $g$ such that $f = g$ $\mu$-a.e., then $f \circ \phi_t = g \circ \phi_t$
$\mu$-a.e.. 
Moreover, for $0 \leq f \in L^p(\mu)$, there exists an increasing
sequence $(h_n)$ of simple functions converging pointwise to $f$. Then
$(h_n \circ \phi_t)$ is a monotone sequence converging pointwise to $f \circ \phi_t$. Using
the fact that $\1_C \circ \phi_t = \1_{C(t)}$, $C(t)$ as above, and the assumption
$\mu(C(t)) \leq M \cdot \mu(C)$, it is easy to see that $\|h_n \circ \phi_t\|_p^p \leq M \cdot \|h_n\|_p^p \leq M \cdot \|f\|_p^p$.
Thus by the Monotone Convergence Theorem we have $f \circ \phi_t \in L^p(\mu)$ and
$\|f \circ \phi_t\|_p \leq M^{1/p}\|f\|_p$. It follows that $T(t)$ is a bounded linear
operator on $L^p(\mu)$ and $\|T(t)\| \leq M^{1/p}$ for $0 \leq t \leq t_0$. Since $\phi$ is a
semiflow, we have $T(0) = \text{Id}$ and $T(t+s) = T(s)T(t)$ $(0 \leq s,t < \infty)$. 
\end{enumerate}
%% --
It remains to prove strong continuity. Since $\phi$ is continuous and \eqref{eq:c3-4.12}
holds, we know that $T(t)(C_c(X)) \subset C_c(X)$ and that $T(t)f$ tends to $f$
uniformly as $t \to 0$ provided that $f \in C_c(X)$. It follows that
$\lim_{t \to 0}\|T(t)f - f\|_p = 0$ for $f \in C_c(X)$. Since $C_c(X)$ is dense in
$L^p(\mu)$ and $\|T(t)\| \leq M^{1/p}$ for $0 \leq t \leq t_0$, the semigroup is strongly
continuous.

\begin{enumerate}[\upshape (i), wide, labelindent=.5em, start=2]
\item
Obviously every operator $T(t)$ defined in assertion (a) of (i)
is a lattice homomorphism. Above we pointed out that $C_c(X)$ is
invariant under $(T(t))$, then the intersection $D(A) \cap C_c(X)$ is invariant as well. It
is dense because the elements of the form $\int_{0}^{r} T(s)f \ds$, $f \in C_c(X)$,
$r > 0$ belong to $C_c(X)$ and to $D(A)$. Hence $D(A) \cap C_c(X)$ is a core
(by A-I, Proposition~1.9).
\end{enumerate}

\end{proof}
%% --
Proposition~\ref{prop:c3-4.11} can be used to prove that flows corresponding to certain
ordinary differential equations on $\R ^n$ generate strongly continuous
semigroups on $L^p(\R ^n)$ (where $\R ^n$ is equipped with the Lebesgue
measure). One has to impose conditions on the corresponding vector
field. Note that for continuous flows condition \eqref{eq:c3-4.12} is automatically satisfied because for a compact $K \subset X$ the set $\phi_t^{-1}(K) = \phi_{-t}(K)$
is compact as the continuous image of a compact set.

\begin{example}\label{ex:c3-4.12}
%\index{Differential equations}
%\index{Flow!Generated by ODE}
Let $F \colon \R ^n \to \R ^n$ be a $C^1$-vector field and assume that
the derivative $DF$ is uniformly bounded on $\R ^n$. Then the ordinary
differential equation $y' = F(y)$ possesses a global flow
$\phi \colon \R  \times \R ^n \to \R ^n$ which is $C^1$. Moreover, we have
%% --
\begin{equation}\label{eq:c3-4.13}
\begin{minipage}{.8\textwidth} 
$\|D\phi_t(x)\| \leq \mathrm{e}^{M|t|}$  for all  $x \in \R ^n$, $t \in \R$  where   
%
\[
	M \coloneqq  \sup \{\|DF(x)\| \colon x \in \R ^n\}.
\]
% 
\end{minipage}
\end{equation}
%% --

All these properties were proven in Example~3.15 of B-II.
We will show that $\phi$ satisfies condition (b) of Proposition~\ref{prop:c3-4.11}(i),
hence it induces a strongly continuous (semi-)group of lattice homomorphisms
on $L(\R ^n)$ $(1 \leq p < \infty)$ via $T(t)f = f \circ \phi_t$.
This is done using the change of variables formula as follows.

Let $U$ be an open subset of $\R ^n$, then $\phi_t^{-1}(U) = \phi_{-t}(U) =\colon U(-t)$. 
If $\lambda$ denotes the Lebesgue measure, then
%% --
\begin{equation}\label{eq:c3-4.14}
	\begin{aligned}
		\lambda(\phi_t^{-1}(U)) &= \int_{U(-t)} 1 \dx  = \int_{U} 1 \circ \phi_{-t}(x) \cdot |\det D\phi_{-t}(x)| \dx  = \\
		&\int_{U} |\det D\phi_{-t}(x)| \dx  \leq \int_{U} \mathrm{e}^{nM|t|} \dx  = \mathrm{e}^{nM|t|} \cdot \lambda(U).
	\end{aligned}
\end{equation}
%% --
Here we used \eqref{eq:c3-4.13} and the fact that the determinant of an n$\times$n-matrix
$B$ satisfies $|\det B| \leq \|B\|^n$\,.
\end{example}

In general, existence of a global flow does not ensure that one can
associate a semigroup of bounded linear operators on $L^p(\R ^n)$, even if
the vector field is $C^{\infty}$. For example the differential equation
$y' = \sin(y^2)$ does not induce a semigroup on $L^p(\R )$.
There is another important class of differential equations, \emph{Hamiltonian differential equations}, which do induce semigroups of lattice homomorphisms on $L^p$-spaces. In fact, Liouville's Theorem states that the flow corresponding to a Hamiltonian vector field preserves the volume (see \citet[Section~3.3]{abraham:1978}). Thus assertion (b) of Proposition~\ref{prop:c3-4.11}(i) is trivially satisfied.
Further examples of flows which are measure preserving and therefore induce semigroups of lattice homomorphisms on $L^p$-spaces are billiard flows on compact convex subsets of $\R ^n$ and geodesic flows on Riemannian manifolds (see \citet{cornfeldetal:1982}).



\clearpage
\section*{Notes}
\addcontentsline{toc}{section}{Notes}

Spectral theory for a single positive operator as developed in Chapter V of \citet{schaefer:1974} is an essential tool for this chapter. 
Various results on the spectral theory of positive one-parameter semigroups can be found in Chapter~7 of \citet{davies:1980} and in the second part of \citet{battyrobinson:1984}.

\begin{enumerate}[label=\emph{Section \arabic*:}, wide, itemsep=1ex]
\item
That the spectral bound is always an element of the spectrum was stated by \citet{karlin:1959}, but a valid proof was given by \citet{derndinger:1980}. 
This assertion as well as assertion (b) of Theorem~\ref{thm:c3-1.1} allow generalizations in various directions.  
They are valid for ordered Banach spaces (see \citet{greinervoigtwolff:1981} and \citet{klein:1984}) and one only needs that $A$ has positive resolvent (see \citet{kato:1982} or \citet{nussbaum:1984}). 
Theorem~\ref{thm:c3-1.2} as well as its corollaries are also valid in ordered Banach spaces. 
For the analogue in the theory of the Laplace transform we refer to Section~10.5 in \citet{widder:1971} and \citet{voigt:1982}.

\item 
Theorem~\ref{thm:c3-2.2} is the basis for the subsequent cyclicity results. 
Pseudoresolvents are discussed \eg in \citet{hillephillips:1957} or \citet{yosida:1965}. 
For nonpositive semigroups the two assertions stated in Definition~\ref{def:c3-2.8} are no longer equivalent. 
A special case of Theorem~\ref{thm:c3-2.10} was proven by\citet{derndinger:1980} while the general result is due to \citet{greiner:1981}. Instead of pseudo-resolvents on the whole $\mathcal{F}$-product Derndinger works with the semigroup on the semigroup $\mathcal{F}$-product. 
Therefore he can only consider eigenvalues. 
Elliptic differential operators as generators of positive semigroups are discussed by many authors, \eg  \citet{amann:1983}, \citet{fattorini:1983}, \citet{friedman:1969} or \citet{pazy:1983}.

\item
There exist various notions which are (more or less closely) related to irreducibility, \eg \enquote{positivity improving} in \citet{reedsimon:1979}, \enquote{$u_0$-positivity} in 
\citet{krasnoselskii:1964}
%[Krasnosel'skii (1964)]
or \enquote{quasi-strictly positive} in \citet{karlin:1959}). \citet{sawashima:1964} uses \enquote{non-support} instead of irreducible. 
She also discusses several modifications (\enquote{semi-non-support}, \enquote{strictly non-support}, \enquote{strongly positive}) and the interrelation between these notions. 
The notion of irreducibility can be extended to the (non-lattice) ordered setting (see \citet{battyrobinson:1984}). 
Assertion (b) of Theorem~\ref{thm:c3-3.2} is due to \citet{majewskirobinson:1983} while special cases can be found in Section~XIII.12 of \citet{reedsimon:1979} and in \citet{kishimotorobinson:1981}. 
Proposition~\ref{prop:c3-3.3} is due to \citet{voigt:1984a}. 
Retarded equations as dicussed in Example 3.4(c) will be discussed in more detail in Section 3 of C-IV. Example 3.4(d) is a one-dimensional version of the linear transport equation. 
The higher dimensional equation is more delicate but can be treated similarly (see \eg \citet{greiner:1984b}, \citet{kaperetal:1983a}, or \citet{voigt:1984b}). 
A special case of Proposition~\ref{prop:c3-3.5} can be found in \citet{davies:1980}. 
Theorem~3.7 and Example~3.6 are taken from \citet{schaefer:1985}. 
The most interesting criterion of Theorem~\ref{thm:c3-3.7} seems to be condition (iii), since it gives a sufficient condition for the existence of eigenvalues for a sufficiently large class of semigroups. 
For semigroups induced by measure-preserving flows Theorem~\ref{thm:c3-3.8} and Corollary~\ref{cor:c3-3.9} are proven in \citet{cornfeldetal:1982}. 
Corollary~\ref{cor:c3-3.9} is a special case of the Halmos-von Neumann Theorem which classifies irreducible semigroups having discrete spectrum (see \citet{cornfeldetal:1982}, \citet{greiner:1982} and \citet{schaefer:1974} for the general result). 

Lemma~\ref{lem:c3-3.10} is taken from 
\citet{groh:1984b}
and Theorems~\ref{thm:c3-3.12} and \ref{thm:c3-3.14} can be found (with slightly different proofs) in \citet{greiner:1981}. 

\item 
lt was \citet{derndinger:1980} who proved Theorem~\ref{thm:c3-4.2}. 
In Corollary~\ref{cor:c3-4.3} one can replace boundedness of the semigroup by the assumption that the resolvent grows slowly (see \citet{greiner:1982}). 
Example 4.4 is due to Davies and Proposition~\ref{prop:c3-4.5} to H.~Kellermann (both unpublished). 
The spectral decomposition for positive groups as described in Theorem~\ref{thm:c3-4.8} is valid in arbitrary Banach lattices (see \citet{arendt:1982} and \citet{greiner:1984c}). 
This also holds for Corollaries 4.9 and 4.10. 
Proposition~\ref{prop:c3-4.11} and Example 4.12 indicate the relationship of positive groups to dynamical systems. 
For example, the \enquote{Annular Hull Theorem} (see \citet{chiconeswanson:1981}) is closely related to the results of this section.
\end{enumerate}

%% -- References
{\RaggedRight
\bibliographystyle{abbrvnat}
\bibliography{bib/ln-references}}