% !TEX root = ../LN-Book.tex
%% -- 
%% --Stand 2025-05-31
%% --
\setcounter{chapter}{1}
\chapter{Characterization of Positive Semigroups on Banach Lattices}\label{chap:c2}
\chaptermark{Characterization of Positive Semigroups}
\index{Positive Semigroups!Characterization}
%% --
{\Large
\vspace*{-.75cm}
by \\[.25em]
Wolfgang Arendt
\vspace{.75cm}
\\
}
%% --
In this chapter our first goal is to find conditions on a generator $A$ of a semigroup $(T(t))_{t \geq 0}$ which are equivalent to the positivity of the semigroup.
After the preparations in A-II, Section 2 this is easy if in addition we ask that the semigroup be contractive: $T(t)$ is a positive contraction for all $t \geq 0$ if and only if $A$ is dispersive (Section 1).
For arbitrary (not necessarily contractive) semigroups a condition on the generator had been found in the case when $E = C(K)$ ($K$ compact), namely the positive minimum principle (P) (see B-II, Theorem 1.6).
One may easily reformulate this condition in arbitrary Banach lattices and show its necessity.
However, only in special cases (for example if $A$ is bounded (see Section 1)) the positive minimum principle is sufficient for the positivity of the semigroup.
In fact, on $L^{2}(\R)$ there exists a non-positive semigroup whose generator satisfies (P) (Section 3).

Looking for another condition we consider the Laplacian $\Delta$ as a prototype.
Defined on a suitable domain, $\Delta$ generates a positive semigroup on $L^p(\R^n)$.
Kato proved the following distributional inequality for the Laplacian:
%% -- 
\[
\Re(\sign \bar{f}) \Delta f \leq \Delta|f|
\]
%% -- 
for all $f \in L^1_{\text{loc}}(\R^{n})$ such that $\Delta f \in L^1_{\text{loc}}(\R^{n})$.
In Section 3 we will show that an abstract version of Kato's inequality for a generator $A$ together with an additional condition is equivalent to the positivity of the semigroup generated by $A$.

Domination of one semigroup by another can be characterized by an analoguous condition for the generators (Section 4).
The results will be applied to Schrödinger operators on $L^p(\R^n)$.

Finally, in Section 5 we show that $(T(t))_{t \geq 0}$ is a lattice semigroup (\ie $|T(t) f| = T(t) |f|$ for all $t \geq 0$, $f \in E$) if and only if $A$ satisfies Kato's equality.
This parallels the case when $E = C_{0}(X)$, but if $E$ has order continuous norm the strong form of Kato's equality can be considered (in particular, $f \in D(A)$ implies $|f| \in D(A)$ if $A$ is the generator of such a semigroup).
%% --
\section{Positive Contraction Semigroups and Bounded Generators}\label{sec:c2-1}
\index{Positive Semigroups!Characterization!Positive Contraction Semigroups and Bounded Generators}
%% --
In this section we first characterize generators of positive contraction semigroups on a real Banach lattice $E$.
For that we use the results developed in A-II, Section 2 for the canonical half-norm $N^{+} \colon E \to \R$ given by
\begin{equation}\label{eq:c2-1.1}
N^{+}(f) = \|f^{+}\| \quad (f \in E).
\end{equation}
%% --
\begin{remark*}\label{rem:c2-1.1}
%\index{Half-norm}
%\index{Canonical half-norm}
It is easy to see that $N^{+}(f) = \inf \{\|f+g\| \colon g \in E_{+}\} = \text{dist}(-f,E_{+})$ (cf. A-II, Remark 2.8). 
\end{remark*}
%% --
It is obvious that $N^{+}$ is a strict half-norm (see A-II, (2.12) ). 

The subdifferential of $N^{+}$ is given by
\begin{equation}\label{eq:c2-1.2}
	\partial N^{+}(f) = \{\phi \in E'_{+} \colon \|\phi\| \leq 1, \langle f,\phi \rangle = \|f^{+}\|\}
\end{equation}
%% --
(this follows from the definition, see A-II, (2.5)). 
%% --
\begin{examples}\label{ex:c2-1.1}
%\index{Examples!Subdifferentials}
%\index{Subdifferentials!Examples}
\begin{enumerate}[\upshape (i), wide, labelsep=.5em]  % , itemindent=\parindent]
%%
\item \label{ex:c2-1.1-1}
Let $E = C_{0}(X)$ (X locally compact). 
Let $f \in E$.
There exists $x \in X$ such that $f(x) = \|f^{+}\|_{\infty}$. 
Then $\delta_{x} \in \partial N^{+}(f)$.
%% \medskip
%%
\item \label{ex:c2-1.1-2}
Let $E = L^{p}(X,\Sigma,\mu)$, where $(X,\Sigma,\mu)$ is a $\sigma$-finite measure space and $1 < p < \infty$. Let $f \in E$ satisfy $f^{+} \neq 0$. Let
%% -- 
\begin{align*}
\phi(x) = 
    \begin{cases}
        c \cdot f(x)^{p-1} & \text{if } f(x) > 0 \\
        0 & \text{if } f(x) \leq 0
    \end{cases}
\end{align*}
%% -- 
where $c > 0$ is such that $\int f(x) \phi(x) \dx = \|f^{+}\|$.
Then $\diff{N^{+}}(f) = \{\phi\}$.
%% \medskip
%%
\item \label{ex:c2-1.1-3}
Let $E = L^{1}(X,\Sigma,\mu)$, where $(X,\Sigma,\mu)$ is a $\sigma$-finite measure space, and $f \in E$. Let $\phi \in L^{\infty}(X,\Sigma,\mu)_{+}$. Then $\phi \in \partial N^{+}(f)$ if and only if
%% -- 
\begin{align*}
\phi(x) &= 1 & \text{if } f(x) > 0, \\
0 \leq \phi(x) &\leq 1 & \text{if } f(x) = 0 \text{ and} \\
\phi(x) &= 0 & \text{if } f(x) < 0.
\end{align*}
%% -- 
\end{enumerate}
\end{examples}
%% --
An operator $A$ on $E$ is called [strictly] dispersive if $A$ is [strictly] $N^{+}$-dissipative; that is, for every $f \in D(A)$ one has $\langle Af,\phi \rangle \leq 0$ for some [resp., all] $\phi \in \diff{N^{+}}(f)$ (see A-II, Section 2). 
Generators of positive contraction semigroups are characterized by the following theorem which is due to 
\citet{phillips:1962}
%% --
\begin{theorem}\label{thm:c2-1.2}
%\index{Dispersive Operators}
%\index{Operators!Dispersive}
%\index{Positive Contraction Semigroups!Generators}
Let $A$ be a densely defined operator on a real Banach lattice $E$. 
The following assertions are equivalent.
\begin{enumerate}[\upshape (a)]
%%
\item \label{thm:c2-1.2-1}
$A$ is the generator of a positive contraction semigroup.
%%--
\item \label{thm:c2-1.2-2}
$A$ is dispersive and $(\lambda-A)$ is surjective for some $\lambda > 0$.
\end{enumerate}
\end{theorem}
%% --
Frequently an operator is known explicitly only on a core. In that case one can use the following result.
%% --
\begin{corollary}\label{cor:c2-1.3}
%\index{Dispersive Operators!Closable}
%\index{Closable Operators}
Let $A$ be a densely defined dispersive operator on a real Banach lattice $E$. 
If $(\lambda - A)D(A)$ is dense in $E$ for some $\lambda > 0$, then $A$ is closable and the closure $\overline{A}$ of $A$ is the generator of a positive contraction semigroup.
\end{corollary}
%% --
Theorem \ref{thm:c2-1.2}  and Corollary \ref{cor:c2-1.3}  immediately follow from A-II, Theorem 2.11 and A-II, Corollary 2.12  if one observes the following.
%% --
\begin{lemma}\label{lem:c2-1.4}
%\index{Positive Contraction}
%\index{$N^{+}$-contractive}
A bounded linear operator  $T$ on a Banach lattice $E$ is a positive contraction if and only if $\|(Tf)^{+}\| \leq \|f^{+}\|$ for all $f \in E$ (\ie if $T$ is $N^{+}$-contractive).
\end{lemma}
%% --
\begin{proof}[Proof of the lemma]
If $T$ is a positive contraction, then $0 \leq (Tf)^{+} \leq Tf^{+}$ and so 
%
\[
	N^{+}(Tf) = \|(Tf)^{+}\| \leq \|Tf^{+}\| \leq \|f^{+}\| = N^{+}(f)
\]
%
for all $f \in E$. 

Conversely, assume that $T$ is an $N^{+}$-contraction and let $f \geq 0$. 
Then $\|(Tf)^{-}\| = N^{+}(T(-f)) \leq N^{+}(-f) = \|f^{-}\| = 0$. 
Hence $(Tf)^{-} = 0$; \ie $Tf \geq 0$. 
We have proved that $T$ is positive. 
In particular, $|Tf| \leq T|f|$ for all $f \in E$. 
Hence $\|Tf\| = \|\,|Tf|\,\| \leq \|T|f|\,\| \leq N^{+}(T|f|) \leq N^{+}(|f|) = \|f\|$ for all $f \in E$. 
So $T$ is a contraction.
\end{proof}
%% --
\begin{examples}\label{ex:c2-1.5}
%\index{Examples!Second Derivative Operator}
%\index{Dirichlet Boundary Condition}
\begin{enumerate}[\upshape (i), wide, labelsep=.5em] %, itemindent=\parindent]  %%enum:c2-1.5-beg
%%
\item \label{ex:c2-1.5-1}
Consider the second derivative with Dirichlet boundary condition on $E = C_{0}(0,1)$; \ie we let $Af = f''$ with domain $D(A) = \{f \in C^{2}[0,1] \colon f(0) = f(1) = f''(0) = f''(1) = 0\}$.
$A$ is dispersive. 
In fact, let $f \in D(A)$. 
Then there exists $x \in (0,1)$ such that $f(x) = \sup_{y \in [0,1]} f(y) = \|f^{+}\|_{\infty}$. 
Thus $\delta_{x} \in \diff{N^{+}}(f)$. 
But $\langle Af,\delta_{x} \rangle = f''(x) \leq 0$ since $f$ has a maximum in $x$.
Let $g \in E$. 
Define 
%%--
\[
f_{0}(x) = 1/2 [\mathrm{e}^{x} \int_{0}^{1} \mathrm{e}^{-y}g(y) \dy - \mathrm{e}^{-x} \int_{0}^{1} \mathrm{e}^{y}g(y) \dy].
\]
%%
Then $f_{0} \in C^{2}[0,1]$ and $f_{0} - f_{0}'' = g$. 
There exist $a, b \in \R$ such that $f(x) = f_{0}(x) + a\mathrm{e}^{x} + b\mathrm{e}^{-x}$ defines a function $f \in C^{2}[0,1]$ satisfying $f(0) = f(1) = 0$. Since $f - f'' = f_{0} - f_{0}'' = g$ this implies that $f \in D(A)$ and $f - Af = g$. We have shown that $(Id - A)$ is surjective. It follows from Theorem 
\ref{thm:c2-1.2} that $A$ is the generator of a positive contraction semigroup.
%%-- 
\item \label{ex:c2-1.5-2}
Let $E = L^{p}[0,1]$ $(1 \leq p < \infty)$ and $A$ be given by $Af = f''$ on $D(A) = \{f \in E \colon f \in C^{1}[0,1], f' \in AC[0,1], f'' \in L^{p}[0,1], f(0) = f(1) = 0\}$. 
Then $A$ is the generator of a positive contraction semigroup.
%%--
\begin{proof}
$A$ is dispersive. 
In fact, let $f \in D(A)$. 
Since the set $M = \{x \in (0,1) \colon f(x) > 0\}$ is open, there exists a countable set of disjoint intervals $(a_{n},b_{n})$ such that $M = \cup_{n \in \N} (a_{n},b_{n})$.

First case: $p > 1$.
Let $\phi \in \partial N^{+}(f)$. 
Then there exists $c \geq 0$ such that $\phi(x) = c f(x)^{p-1}$ for all $x \in M$ and and $\phi(x) = 0$ if $f(x) \leq 0$ (see Example \ref{ex:c2-1.1}, \ref{ex:c2-1.1-2} ). 
Thus integration by parts yields
%% -- 
\begin{align*}
\langle Af,\phi \rangle &= \sum_{n} \int_{a_{n}}^{b_{n}} f''(x)\phi(x) \, \dx \\
&= -c \sum_{n} \int_{a_{n}}^{b_{n}} f'(x)f'(x) (p-1)f(x)^{p-2} \, dx \\
&\leq 0.
\end{align*}
%% -- 
Second case: $p = 1$.
Let $\phi(x) = 1$ for $x \in M$ and $\phi(x) = 0$ for $x \not\in M$. 
Then $\phi \in \diff{N^{+}}(f)$ and
%% -- 
\begin{align*}
\langle Af,\phi \rangle &= \sum_{n} \int_{a_{n}}^{b_{n}} f''(x) \, \dx = \sum_{n} (f'(b_{n}) - f'(a_{n})) \leq 0
\end{align*}
%% -- 
since $f'(b_{n}) \leq 0$ and $f'(a_{n}) \geq 0$ for all $n \in \N$.

We have shown that $A$ is dispersive. 
As in \ref{ex:c2-1.5-1}   one shows that $(Id - A)$ is surjective. 
Now the claim follows from Theorem \ref{thm:c2-1.2}  .
\end{proof}
%% \medskip
%%
\item \label{ex:c2-1.5-3} 
Consider $E = C_{0}(\R^{n})$. 
Let $D(A) = \mathcal{S}(\R^{n})$ (the Schwartz space of all infinitely differentiable rapidly decreasing functions) and $Af = \Delta f$ $(f \in D(A))$. 
Then $A$ is closable and the closure of $A$ generates a positive contraction semigroup on $E$.
%%--
%%--
%%KGK p-C2-05; LNMp-251
%%
\begin{proof}
$A$ is dispersive. 
In fact, let $f \in D(A)$. 
If $f^{+} = 0$, then $\phi \coloneqq 0 \in \diff{N^{+}}(f)$. 
So assume that $f^{+} \neq 0$. 
Then there exists $x \in \R^{n}$ such that $f(x) = \|f\|_{\infty} = \sup\{f(y) \colon y \in \R^{n}\}$. 
Thus $\delta_{x} \in \diff{N^{+}}(f)$.
Since $f$ has a maximum in $x$ it follows that $\langle Af,\delta_{x} \rangle = (\Delta f)(x) = \text{tr}(\partial^{2}f/\partial x_{i}\partial x_{j})(x) \leq 0$. 
Moreover,
%% --
\begin{equation}\label{eq:c2-1.3}
(Id - \Delta) \text{ is an isomorphism from } \mathcal{S}(\R^{n}) \text{ onto } \mathcal{S}(\R^{n}).
\end{equation}
%% --
In fact, the Fourier transform $f \mapsto \hat{f}$ is a bijection from $\mathcal{S}(\R^{n})$ onto $\mathcal{S}(\R^{n})$.
But $[(Id - \Delta)f]^{\hat{}} = M\hat{f}$ where $(Mg)(y) = (1 + \sum_{i=1}^{n} y_{i}^{2})g(y)$ $(g \in \mathcal{S}(\R^{n}))$. It follows from \eqref{eq:c2-1.3}    that $(Id - A)D(A)$ is dense in $E$. So the claim follows from Corollary~\ref{cor:c2-1.3}.
\end{proof}
%% --
\begin{remark*}\label{rem:c2-1.2}
%\index{Closure of Laplacian}
%\index{Laplacian!Closure}
In addition one can show that the closure $\overline{A}$ of $A$ is given by $\overline{A}f = \Delta f$ with domain $D(\overline{A}) = \{f \in E \colon \Delta f \in E\}$ where for
$f \in C_{0}(\R^{n})$ the expression $\Delta f$ is understood in the sense of distributions. 
Moreover, the space $C_{c}^{\infty}(\R^{n})$ (of all infinitely differentiable functions with compact support) is a core of $\overline{A}$ (cf. \ref{ex:c2-1.5-4}).
\end{remark*}
%%
\item \label{ex:c2-1.5-4} 
Let $E = L^{p}(\R^{n})$ $(1 \leq p < \infty)$ and $A$ be given by $Af = \Delta f$ with domain $D(A) = \{f \in L^{p}(\R^{n}) \colon \Delta f \in L^{p}(\R^{n})\}$ where for $f \in L^{p}(\R^{n})$ the expression $\Delta f$ is understood in the sense of distributions. Then $A$ is the generator of a positive contraction semigroup. Moreover, the space $C_{c}^{\infty}(\R^{n})$ is a core of $A$.
%%
%% \smallskip
\begin{proof}
It is easy to see that $A$ is closed. 
Let $A_{0}$ denote the restriction of $A$ to $\mathcal{S} \coloneq \mathcal{S}(\R^{n})$. 
Then $A_{0}f = \Delta f$ in the classical sense for all $f \in \mathcal{S}$.
One can show in an analogous way as in \ref{ex:c2-1.5-2}   that $A_{0}$ is dispersive. 
Moreover, it follows from \eqref{eq:c2-1.3} that $(Id - A_{0})D(A_{0})$ is dense. 
Hence by Corollary \ref{cor:c2-1.3}   the closure $\overline{A_{0}}$ of $A_{0}$ is the generator of a positive contraction semigroup.

By construction one has $\overline{A_{0}} \subset A$. 
We prove that $\overline{A_{0}} = A$. 
For that it is enough to show that
\begin{equation}\label{eq:c2-1.4}
(Id - A) \text{ is injective.}
\end{equation}
In fact, since the restriction $(Id - \overline{A_{0}})$ of $(Id - A)$ is bijective from $D(\overline{A_{0}})$ onto $E$ it follows from  \eqref{eq:c2-1.4}  that $D(\overline{A_{0}}) = D(A)$. 
So let us show  \eqref{eq:c2-1.4}  . 
Assume that there is $f \in E$ such that $f - \Delta f = 0$.
Let $\phi \in C_{c}^{\infty}(\R^{n})$. 
Then
%% --
\begin{equation} \label{eq:c2-1.5}
	\langle \phi - \Delta\phi , f \rangle = 0.
\end{equation}
%%
Since  \,$C_{c}^{\infty}(\R^{n})$\, is dense in \,$\mathcal{S}$\, for the topology of $\mathcal{S}$, it follows from  \eqref{eq:c2-1.3}  that \\
$(Id~-~\Delta)C_{c}^{\infty}(\R^{n})$ is dense in $\mathcal{S}$. Hence  \eqref{eq:c2-1.5}   implies
%%--
%%KGK p-C2-06; LNMp-252
%%
that $\langle \phi, f \rangle = 0$ for all $\phi \in S$. 
Consequently, $f = 0$.
\end{proof}
%% \smallskip
\begin{remark*}\label{rem:c2-1.3}
%\index{Fourier transform}
%\index{Semigroups!examples}
%\index{Examples!Fourier transform}
Using the Fourier transform one can show that the semigroups in example \ref{ex:c2-1.5-3}   and \ref{ex:c2-1.5-4}   are given by
%% --
\begin{equation}\label{eq:c2-1.6}
(T(t)f)(x) = (4\pi t)^{-n/2} \int_{\R^n} \exp(-(x-y)^{2}/4t) f(y) \, \dy
\end{equation}
%% --
$(f \in E)$, where $z^{2} \coloneqq \sum_{i=1}^n z_{i}^{2}$ $(z \in \R^n)$.
\end{remark*}
%% \medskip
%%
\item \label{ex:c2-1.5-5}
The following example is the analog of  \ref{ex:c2-1.5-1}   for higher dimension.
Let $\Omega \subset \R^n$ be a bounded open and connected set and $E = C_{0}(\Omega)$.
We assume that the Dirichlet problem
%% --
\begin{equation} \label{eq:c2-1.7}
\begin{array}{lll}
u(x) - \Delta u(x) &= 0 \quad &(x \in \Omega)\\
u(x) &= b(x) \quad &(x \in \partial\Omega)
\end{array}
\end{equation}
%% --
has a solution $u \in C^{2}(\Omega) \cap C(\overline{\Omega})$ for every $b \in C(\partial\Omega)$.
For example, this is the case if the boundary $\partial\Omega$ is $C^{2}$ (see \citet[Section~2.8 and Theorem. 6.13]{gilbargtrudinger:1977}).
Let $A$ be given by $Af = \Delta f$ on
$D(A) = \{f \in C^{2}(\Omega) \cap C_{0}(\Omega) \colon \Delta f \in C_{0}(\Omega)\}$.
Then $A$ is closable and the closure of $A$ is the generator of a positive contraction semigroup.
%% \smallskip
\begin{proof}
$D(A)$ is clearly dense in $E$.
Moreover, one can show as in \ref{ex:c2-1.5-3} that $A$ is dispersive.
It remains to prove that $(Id - A)D(A)$ is dense in $E$.
The space $C_{c}^{\infty}(\Omega)$ of all infinitely differentiabel functions on $\Omega$ with compact support contained in $\Omega$ is dense in $E \cap C_{c}^{\infty}(\Omega)$.
We show that there exists $f \in D(A)$ satisfying $(Id - A)f = g$.
Let $\bar{g} \colon \R^{n} \to \R$ be given by $\bar{g}(x) = g(x)$ if $x \in \Omega$ and $0$ if $x \not\in \Omega$.
Then $\bar{g} \in \mathcal{S}(\R^n)$.
By  \eqref{eq:c2-1.3}  there exists $\bar{f} \in \mathcal{S}(\R^n)$ such that $\bar{f} - \Delta\bar{f} = \bar{g}$.
Consider the function $b \in C(\partial\Omega)$ given by $b(x) = \bar{f}(x)$ for all $x \in \partial\Omega$.
Then by our hypothesis there exists $u \in C(\overline{\Omega}) \cap C^{2}(\Omega)$ satisfying  \eqref{eq:c2-1.7}.
Let $f(x) = \bar{f}(x) - u(x)$ $(x \in \Omega)$.
Then $f \in C^{2}(\Omega) \cap C(\Omega)$ and $(f - \Delta f)(x) = g(x)$ $(x \in \Omega)$.
Thus $\Delta f = f - g$ vanishes on $\partial\Omega$.
Hence $f \in D(A)$ and $f - Af = g$.
\end{proof}
%% \medskip
%%
\item \label{ex:c2-1.5_6}   
Let $\Omega \subset \R^{n}$ be as in \ref{ex:c2-1.5-5}   and $E = L^{p}(\Omega)$.
Define $Af = \Delta f$ on $D(A) = \{f \in C^{2}(\Omega) \cap C_{0}(\Omega) \colon \Delta f \in C_{0}(\Omega)\}$.
Then $A$ is closable and the closure of $A$ is the generator of a positive contraction semigroup on $E$.
%%--
%%KGK p-C2-07; LNMp-253
%%
\begin{proof}
$D(A)$ is dense and one can show in an analoguous manner as in \ref{ex:c2-1.5-4}   that $A$ is dispersive.
We know from \ref{ex:c2-1.5-4}   that $C_{c}^{\infty}(\Omega) \subset (Id - A)D(A)$.
Thus $(Id - A)D(A)$ is dense in $E$ and the claim follows from Corollary \ref{cor:c2-1.3}.
\end{proof}
\end{enumerate}
\end{examples}
%% \bigskip
We now turn to the problem to characterize generators of arbitrary (not necessarily contractive) positive semigroups.
Of course, as in B-II, Section 1 
%%KGK: \ref{sec:b2-1}
one sees that a semigroup $(T(t))_{t \geq 0}$ is positive if and only if $R(\lambda,A) \geq 0$ for all $\lambda > \omega(A)$ where $A$ denotes the generator of $(T(t))_{t \geq 0}$.
We are looking for an intrinsic condition on~$A$.

The positive minimum principle which is characteristic for generators of strongly continuous semigroups on $C(K)$ (see 
%%KGK: Thm\ref{thm:b2-1.6}
B-II, Theorem 1.6) can be reformulated on an arbitrary Banach lattice $E$.
%% --
\begin{definition}\label{def:c2-1.6}
%\index{Positive minimum principle}
%\index{Generators!positive minimum principle}
%\index{Examples!Positive minimum principle}
An operator $A$ on $E$ satisfies the \emph{positive minimum principle} if for all $f \in D(A)_{+}$, $\phi \in E'_{+}$,
%% --
\begin{equation}\label{eq:c2-P} \tag{P}
\langle f,\phi \rangle = 0 \quad \text{implies} \quad \langle Af,\phi \rangle \geq 0  .
\end{equation}
%% --
\end{definition}
%% --
\begin{remark*}\label{rem:c2-1.6}
%\index{Positive minimum principle!equivalence}
%\index{C(K)!positive minimum principle}
It is easy to see that this definition coincides with that given in B-II, Section 1 
%%KGK: \ref{sec:b2-1} 
in the case when $E = C(K)$ (K compact).
[In fact, suppose that for all $f \in D(A)_{+}$ and $x \in K$, $f(x) = 0$ implies $(Af)(x) \geq 0$.
Let $g \in D(A)_{+}$, $\mu \in M(K)_{+}$ such that $\langle g,\mu \rangle = 0$.
Then $g(x) = 0$ for all $x \in \text{supp } \mu$.
Thus by hypothesis, $(Ag)(x) \geq 0$ for all $x \in \text{supp } \mu$.
Consequently $\langle Ag,\mu \rangle \geq 0$.
This proves one direction.
The other is obvious by considering point measures.]
\end{remark*}
%% --
\begin{proposition}\label{prop:c2-1.7}
%\index{Generators!positive minimum principle}
%\index{Positive minimum principle!necessity}
The generator of a strongly continuous positive semigroup satisfies the positive minimum principle  \eqref{eq:c2-P}.  
\end{proposition}
%%
\begin{proof}
Let $(T(t))_{t \geq 0}$ be a strongly continuous positive semigroup with generator $A$ and $0 \leq f \in D(A)$, $\phi \in E'_{+}$ such that $\langle f,\phi \rangle = 0$.
Then $\langle Af,\phi \rangle = \lim_{t \to 0} 1/t \langle T(t)f - f, \phi \rangle = \lim_{t \to 0} 1/t \langle T(t)f, \phi \rangle \geq 0$.
\end{proof}
%%
We will see that the positive minimum principle is not sufficient for the positivity of the semigroup, in general (Remark \ref{rem:c2-3.16}).
However, the following special case is of interest.
%%--
%%KGK p-C2-08; LNMp-254
%%
\begin{theorem}\label{thm:c2-1.8}
%\index{Generators!characterization}
%\index{Semigroups!positive}
%\index{Positive minimum principle!sufficiency}
Let $A$ be the generator of a strongly continuous semigroup $(T(t))_{t \geq 0}$ on a Banach lattice $E$.
Assume that
\begin{enumerate}[\upshape (i)]
%%
\item \label{thm:c2-1.8-1}
there exists $w \in \R$ such that $\|T(t)\| \leq \mathrm{e}^{wt}$ for all $t \geq 0$;
%%--
\item \label{thm:c2-1.8-2}
there exists a core $D_{0}$ of $A$ such that $f \in D_{0}$ implies $|f| \in D_{0}$.
\end{enumerate}
If the restriction of $A$ to $D_{0}$ satisfies the positive minimum principle, then the semigroup is positive.
\end{theorem}
%% --
\begin{remark*}\label{rem:c2-1.8-Claude}
%\index{Generators!conditions}
%\index{Positive semigroups}
Elementary examples show that neither a) nor b) hold for generators of positive semigroups, in general.
\end{remark*}
%% --
The proof of Theorem \ref{thm:c2-1.8}   is based on the following proposition.
%% --
\begin{proposition}\label{prop:c2-1.9}
%\index{Dispersive operators}
%\index{Positive minimum principle!dispersiveness}
Let $A$ be a densely defined dissipative operator which possesses a core $D_{0}$ such that $f \in D_{0}$ implies $|f| \in D_{0}$.
If the restriction of $A$ to $D_{0}$ satisfies the positive minimum principle (P), then $A$ is dispersive.
\end{proposition}
%% --
\begin{proof}[Proof of Proposition \ref{prop:c2-1.9}  ]
By A-II, Proposition 2.9, it is enough to show that $A_{0} \coloneqq A|_{D_{0}}$ is dispersive.
Let $f \in D_{0}$ and $\phi \in \diff{N^{+}}(f)$.
Then $\phi \in E'_{+}$, $\|\phi\| \leq 1$ and $\langle f,\phi \rangle = \|f^{+}\|$.
Hence, $\langle f^{-},\phi \rangle = \langle f^{-},\phi \rangle + \langle f,\phi \rangle - \|f^{+}\| = \langle f^{+},\phi \rangle - \|f^{+}\| \leq 0$.
Thus $\langle f^{-},\phi \rangle = 0$.
Consequently, $\langle f^{+},\phi \rangle = \langle f,\phi \rangle = \|f^{+}\|$; and so $\phi \in \diff{N}(f^{+})$.
Since $A$ is dissipative it follows that $\langle Af^{+},\phi \rangle \leq 0$.
Moreover, since $A$ satisfies \eqref{eq:c2-P} we have $\langle Af^{-},\phi \rangle \geq 0$.
So we finally obtain, $\langle Af,\phi \rangle = \langle Af^{+},\phi \rangle - \langle Af^{-},\phi \rangle \leq 0$.
\end{proof}
%% --
\begin{proof}[Proof of Theorem \ref{thm:c2-1.8}  ]
%\index{Generators!positive semigroups}
The operator $A - w$ satisfies (P) 
%%KGK:  \eqref{eq:c2-P}
as well.
So it follows from Proposition \ref{prop:c2-1.9}   that $A - w$ is dispersive.
Consequently, the semigroup $(\mathrm{e}^{-wt}T(t))_{t \geq 0}$, which is generated by $A - w$, is positive.
Thus $(T(t))_{t \geq 0}$ is positive as well.
\end{proof}
%% --
Next we give a reformulation of the positive minimum principle.
For $0 < u \in E_{+}$ we denote by $E_{u}$ the principal ideal generated by $u$.
If $g \in E_{+}$, then $g \in \overline{E_{u}}$ if and only if $\lim_{n \to \infty} \|u - n u \wedge g\| = 0$.
%% --
\begin{lemma}\label{lem:c2-1.10}
%\index{Positive minimum principle!reformulation}
%\index{Principal ideal}
An operator $A$ on $E$ satisfies (P) if and only if
%% --
\begin{equation}\label{eq:c2-1.8}
(Au)^{-} \in E_{u} \text{ for all } u \in D(A)_{+} \coloneqq D(A) \cap E_{+}.
\end{equation}
%% --
\end{lemma}
%%--
%%KGK p-C2-09; LNMp-255
%%
\begin{proof}
Let $u \in D(A)_{+}$, $g = Au$.
Assume that $(Au)^{-} \in \overline{E}_{u}$.
Then, if $0 \leq \phi \in E_{+}'$ such that $\langle u,\phi \rangle = 0$ one has $\langle f,\phi \rangle = 0$ for all $f \in \overline{E}_{u}$, hence $\langle (Au)^{-},\phi \rangle = 0$ and consequently $\langle Au,\phi \rangle \geq 0$.
This proves one direction.
To prove the other assume that $g^{-} \not\in \overline{E}_{u}$.
Then there exists $\phi \in (E_{u})^{\circ}$ such that $\langle g^{-},\phi \rangle \not= 0$.
%%Since $(E_{u})^{\circ}$ has a generating cone (by \citet[II,4.7]{schaefer:1974}), we can assume that $\phi > 0$.
Define $\psi_{0}(f) = \sup \phi([0,f] \cap E_{(g^{-})})$ for $f \in E_{+}$.
Then $\psi_{0}$ is positive homogeneous on $E_{+}$.
Thus the linear extension of $\psi_{0}$ defines a positive linear form $\psi$ on $E$.
We have $\langle g^{-},\psi \rangle = \langle g^{-},\phi \rangle > 0$ and $\langle g^{+},\psi \rangle = 0$.
Thus $\langle Au,\psi \rangle = - \langle g^{-},\psi \rangle < 0$.
But $\langle u,\psi \rangle \leq \langle u,\phi \rangle = 0$.
Thus  \eqref{eq:c2-P}   does not hold.
\end{proof}
%% --
Bounded generators of positive semigroups can now be characterized as follows.
%% --
\begin{theorem}\label{thm:c2-1.11}
%\index{Bounded generators}
%\index{Positive semigroups!Bounded generators}
Let $A$ be a bounded operator on a Banach lattice $E$.
The following assertions are equivalent:
\begin{enumerate}[\upshape (a)]
%%
\item \label{thm:c2-1.11-1}
$\mathrm{e}^{tA} \geq 0$ $(t \geq 0)$.
%%--
\item \label{thm:c2-1.11-2}
$f \in E_{+}$, $\phi \in E'_{+}$, $\langle f,\phi \rangle = 0$ implies $\langle Af,\phi \rangle \geq 0$.
%%--
\item \label{thm:c2-1.11-3}
$(Af)^{-} \in \overline{E_{f}}$ for all $f \in D(A)_{+}$.
%%--
\item \label{thm:c2-1.11-4}
$A + \|A\| \cdot \Id \geq 0$.
\end{enumerate}
\end{theorem}
%% --
\begin{proof}
It follows by Proposition \ref{prop:c2-1.7}   that  \ref{thm:c2-1.11-1}   implies \ref{thm:c2-1.11-2}  .
Since $\|\mathrm{e}^{tA}\| \leq \mathrm{e}^{t\|A\|}$ $(t\geq0)$, \ref{thm:c2-1.11-2}   implies \ref{thm:c2-1.11-1}   by Theorem \ref{thm:c2-1.8}   .
The equivalence of \ref{thm:c2-1.11-2}   and \ref{thm:c2-1.11-3}   is established by Lemma \ref{lem:c2-1.10}.
If \ref{thm:c2-1.11-4} holds, then $\mathrm{e}^{t(A+\|A\|)} \geq 0$ $(t\geq0)$.
Thus $\mathrm{e}^{tA} = \mathrm{e}^{-t\|A\|} \mathrm{e}^{t(A+\|A\|)} \geq 0$, $(t\geq0)$.
We have shown that \ref{thm:c2-1.11-1}, \ref{thm:c2-1.11-2}  and \ref{thm:c2-1.11-3} are equivalent and \ref{thm:c2-1.11-4}  implies \ref{thm:c2-1.11-1} 
It remains to show that \ref{thm:c2-1.11-1}  implies \ref{thm:c2-1.11-4}  .
Since assertions \ref{thm:c2-1.11-1}   and \ref{thm:c2-1.11-4}   are satisfied for $A$ if and only if they are satisfied for $A'$, we can assume that $E$ is order complete (considering $A'$ instead of $A$ if necessary).
Assume that \ref{thm:c2-1.11-1}   holds.
Then by what we have proved above \ref{thm:c2-1.11-3}   holds as well.
In particular
%% --
\begin{equation}\label{eq:c2-1.9}
(Au)^{-} \in \{u\}^{dd} \text{ for all } u \in E_{+}.
\end{equation}
%% --
Let $\lambda \geq 0$ and $f \in E_{+}$ such that $g = (A + \lambda)f \not\geq 0$.
We have to show that $\lambda \leq \|A\|$.
Denote by $P$ the band projection onto the band generated by $g^{-}$.
Then $PAf + \lambda Pf = Pg = g^{-} < 0$.
Since by  \eqref{eq:c2-1.9}, $[A(\Id - P)f]^{-} \in (\Id - P)E$, it follows 
$0 > \lambda Pf + PAf = \lambda Pf + PAPf + PA(\Id-P)f \geq \lambda Pf + PAPf + P(A(\Id-P)f)^{+} \geq \lambda Pf + PAPf$.
%%--
%%KGK p-C2-10; LNMp-256
%%
Hence $0 \leq \lambda Pf < -PAPf$.
This implies that $Pf \neq 0$ and $\lambda\|Pf\| \leq \|PAPf\| \leq \|A\|\cdot\|Pf\|$.
Consequently $\lambda \leq \|A\|$.
\end{proof}
%%
\begin{remark}\label{rem:c2-1.12}
%\index{Banach lattice!$\sigma$-order complete}
%\index{Semigroups!positivity}
%\index{Examples!$\sigma$-order complete Banach lattice}
It follows from the proof of Theorem \ref{thm:c2-1.11}   that on a $\sigma$-order complete Banach lattice condition  \eqref{eq:c2-1.9} is equivalent to the positivity of the semigroup $(\mathrm{e}^{tA})_{t \geq 0}$.
\end{remark}
%%
\begin{examples}\label{ex:c2-1.13}
%\index{Semigroups!in $\ell^p$ spaces}
%\index{Generators!matrix representation}
%\index{Examples!Matrix generators}
Let $E = \ell^p$ $(1 \leq p \leq \infty)$ or $E = c_{0}$.
%% --
\begin{enumerate}[\upshape (i), wide, labelsep=.5em] %%, 	 itemindent=\parindent]
%%
\item \label{ex:c2-1.13-1}
An operator $A \in \LE)$ can be canonically represented by a matrix $(a_{ij})$. 
It follows from Theorem \ref{thm:c2-1.11} that $\mathrm{e}^{tA} \geq 0$ for all $t \geq 0$ if and only if $a_{ij} \geq 0$ whenever $i \neq j$.
%% \medskip
%%--
\item \label{ex:c2-1.13-2}
Let $A$ be the generator of a strongly continuous contraction semigroup $(T(t))_{t \geq 0}$ on $E$.
Suppose that the space $c_{00}$ of all sequences which vanish off a finite set is a core of $A$.
Let $(a_{nm})_{m \in \N} = (Ae_{n})$ where $e_{n} = (\delta_{nm})_{m \in \N}$ denotes the $n^{th}$ unit vector.
Then it follows from Theorem \ref{thm:c2-1.8}   that the semigroup is positive if and only if $a_{nm} \geq 0$ whenever $n \neq m$.
\end{enumerate}
\end{examples}
%% --
\section{Kato's Inequality} \label{sec:c2-2}
%\index{Kato's inequality}
%\index{Inequality!Kato's}
A strongly continuous semigroup on $C(K)$ $(K$ compact$)$ or a norm continuous semigroup on an arbitrary Banach lattice is positive if and only if its generator $A$ satisfies the positive minimum principle \eqref{eq:c2-P} from Definition \ref{def:c2-1.6}.
However, we will see that in general \eqref{eq:c2-P} is not sufficient for the positivity of the semigroup.
One reason seems to be that \eqref{eq:c2-P} involves merely positive elements in $D(A)$ but $D(A)_{+}$ can be small if the semigroup is not positive (cf. Remark \ref{rem:c2-3.16}).
Our aim in this section is to find a different condition on the generator which is necessary for the positivity of the semigroup.

We recall from Chapter C-I, Section 8 definition and properties of the signum operator.
%% --
\begin{proposition}\label{prop:c2-2.1}
%\index{Signum operator}
%\index{Operators!signum}
%\index{$\sigma$-order complete Banach lattice}
Let $E$ be a $\sigma$-order complete (real or complex) Banach lattice.
For every $f \in E$ there exists a unique linear operator (sign $f$) on $E$ which satisfies
%%--
%%KGK p-C2-11; LNMp-257
%%
%% --
\begin{equation}\label{eq:c2-2.1}
	|(\sign f)g| \leq |g| \quad (g \in E)
\end{equation}
%% --
\begin{equation}\label{eq:c2-2.2}
(\sign f)g = 0 \quad \text{if} \quad \inf\{|f|,|g|\} = 0
\end{equation}
%% --
%% --
\begin{equation}\label{eq:c2-2.3}
(\sign \bar{f})f = |f| \quad \text{(where } \bar{f} \coloneqq \Re f - i\Im f \text{)}.
\end{equation}
%% --
\end{proposition}
%%--
The operator $(\widehat{\sign} f)$ (which is non-linear in general) is defined by
%% --
\begin{equation}\label{eq:c2-2.4}
(\widehat{\sign}f)g = (\sign f)g + (\Id - P_{|f|})|g|
\end{equation}
%% --
where for $h \in E_{+}$ we denote by $P_h$ the band projection onto the band $\{h\}^{dd}$ generated by $h$.

If $E$ is a real $\sigma$-order complete Banach lattice, then
%% --
\begin{equation}\label{eq:c2-2.5}
	\sign f = P_{(f^{+})} - P_{(f^{-})}  .
\end{equation}
%% --
\begin{example}\label{ex:c2-2.2}
%\index{Signum operator!in $L^p$ spaces}
%\index{$L^p$ spaces!signum operator}
%\index{Examples!Signum operator}
Let $f \in E \coloneqq L^p(X,\Sigma,\mu)$ (real or complex) where $(X,\Sigma,\mu)$ is a $\sigma$-finite measure space and $1 \leq p \leq \infty$.
Define
%% --
\begin{align*}
	m(x) = 
	\begin{cases}
    	f(x)/|f(x)| & \text{if } f(x) \neq 0 \\
    	0 & \text{if } f(x) = 0  .
	\end{cases}
\end{align*}
%% --
Then $\sign f$ is the multiplication operator defined by $m$; \ie
%% --
\[
	(\sign f)g = m \cdot g \quad (g \in E), 
\]
%% --
Moreover, 
\[
	(\widehat{\sign} f)g = m \cdot g + 1_{[f(x)=0]}|g| \quad (g \in E).
\]
\end{example}
%% --
The operator $\widehat{\sign} f$ is related to the Gateaux-derivative (B-II, Definition 3.2) of the modulus.
We explain this by an example.
%% --
\begin{example}\label{ex:c2-2.3}
%\index{Gateaux-derivative!of modulus}
%\index{Signum operator!Gateaux-derivative}
%\index{Examples!Gateaux-derivative}
Let $E$ be the real or complex space $L^p(X,\Sigma,\mu)$ where $(X,\Sigma,\mu)$ is a $\sigma$-finite measure space and $1 \leq p < \infty$.
Let $f,g \in E$ and $x \in X$.
Then by B-II, Lemma 2.4 
%%KGK: \ref{ken:b2-2.4}
%% --
\begin{align*}
\lim_{t \downarrow 0} 1/t\, (|f(x)+tg(x)|-|f(x)|) = 
\begin{cases}
    \Re (\sign \overline{f(x)})g(x) & \text{if } f(x) \neq 0 \\
    |g(x)| & \text{if } f(x) = 0.
\end{cases}
\end{align*}
%% --
If $\Theta \colon E \to E_{+}$ denotes the modulus function given by $\Theta(h) = |h|$, then it follows from the dominated convergence theorem that $\Theta$ is right-sided Gateaux-differentiable and
%% --
\begin{equation}\label{eq:c2-2.6}
D_{g}\Theta(f) = \Re(\widehat{\sign} \bar{f})g.
\end{equation}
%% --
\end{example}
Later we will see that \ref{thm:c2-2.6}   holds in every Banach lattice with order continuous norm.
%%--
%%KGK p-C2-12; LNMp-258
%%
Now let $A$ be the generator of a strongly continuous positive semigroup $(T(t))_{t \geq 0}$.
The positivity of the semigroup is equivalent to
%% --
\begin{equation}\label{eq:c2-2.7}
|T(t)f| \leq T(t)|f| \quad (t \geq 0 \,, f \in E)  .
\end{equation}
%% --
In order to deduce from \ref{eq:c2-2.7} a property for the generator $A$ it is natural trying to differentiate \ref{eq:c2-2.7} at $t = 0$.
Let us assume for a moment that $E = L^p(X,\Sigma,\mu)$ (as in Example \ref{ex:c2-2.3}).
Let $f \in D(A)$ and $0 \leq \phi \in D(A')$.
Then by \eqref{eq:c2-2.7}  ,
%% --
\begin{equation}\label{eq:c2-2.8}
	\langle |T(t)f|,\phi \rangle \leq \langle T(t)|f|,\phi \rangle \quad (t \geq 0)
\end{equation}
%% --
where the equality holds for $t = 0$.
Hence the inequality remains valid if we differentiate at $0$ on both sides of \eqref{eq:c2-2.8}  .
Since $\phi \in D(A')$ we obtain $\diff{}/\dt|_{t=0} \langle T(t)|f|,\phi \rangle = \langle |f|,A'\phi \rangle$ on the right side.
By \eqref{eq:c2-2.6} and the chain rule B-II, Proposition 2.3 
%%KGK: \ref{prop:b2-2.3} 
one obtains $\diff{}/\dt|_{t=0}|T(t)f| = \Re((\widehat{\sign} \bar{f})Af)$ on the left side.
Since $\Re((\sign \bar{f})Af) \leq \Re((\widehat{\sign} \bar{f})Af)$, this finally gives
%% --
\begin{equation*}\label{eq:c2-K} \tag{K}
\Re\langle(\sign \bar{f})Af,\phi \rangle \leq \langle |f|,A'\phi \rangle \quad (f \in D(A) \,, 0 \leq \phi \in D(A'))  .
\end{equation*}
%% --
We refer to this as \emph{Kato's inequality}, since it represents an abstract version of the classical inequality proved by Kato for the Laplacian (see Example \ref{ex:c2-2.5}).

We will see in the next section that, together with an additional condition, this inequality is characteristic for the positivity of the semigroup.

By a different proof, we now show that Kato's inequality holds for generators of positive semigroups in general.
%% --
\begin{theorem}\label{thm:c2-2.4}
%\index{Kato's inequality!necessity}
%\index{Generators!Kato's inequality}
%\index{Positive semigroups!Kato's inequality}
The generator $A$ of a strongly continuous positive semigroup $(T(t))_{t \geq 0}$ on a $\sigma$-order complete (real or complex) Banach lattice $E$ satisfies Kato's inequality; \ie
%% --
\begin{equation*}\label{eq:c2-K-2} %%\tag{K}
\Re\langle(\sign \bar{f})Af,\phi \rangle \leq \langle |f|,A'\phi \rangle \quad (f \in D(A), 0 \leq \phi \in D(A'))  .
\end{equation*}
%% --
\end{theorem}
%%
\begin{proof}
Let $f \in D(A)$, $0 \leq \phi \in D(A')$.
Then
%% --
\begin{align*}
\Re\langle(\sign \bar{f})Af, \phi \rangle 
	&= \lim_{t \to 0} 1/t \Re\langle(\sign \bar{f})(T(t)f - f), \phi \rangle \\
	&= \lim_{t \to 0} 1/t \Re\langle(\sign \bar{f})T(t)f - |f|, \phi \rangle \\
	&\leq \lim_{t \to 0} 1/t \langle |T(t)f| - |f|, \phi \rangle \\
	&\leq \lim_{t \to 0} 1/t \langle T(t)|f| - |f|, \phi \rangle\\
	&= \lim_{t \to 0} \langle |f|, 1/t(T(t)'\phi - \phi) \rangle\\
	&= \langle |f|, A'\phi \rangle.
\end{align*}
%%--
%%KGK p-C2-13; LNMp-259
%%
\end{proof}
%%
Let $D(A')_{+} = E'_{+} \cap D(A')$. 
Consider the condition
%% --
\begin{equation}\label{eq:c2-2.9}
\overline{D(A')_{+}}^{\sigma(E',E)} = E'_{+}
\end{equation}
%% --
(which is satisfied if the semigroup is positive). 
If \eqref{eq:c2-K}   and \eqref{eq:c2-2.9}   hold, then Kato's inequality holds in the strong form as well, whenever it makes sense; \ie
%% --
\begin{equation}\label{eq:c2-2.10}
\Re((\sign \bar{f})Af) \leq A|f| \quad \text{(whenever } f, |f| \in D(A)\text{)}.
\end{equation}
%% --
\begin{example}\label{ex:c2-2.5}
%\index{Kato's inequality!classical form}
%\index{Laplacian!Kato's inequality}
%\index{Examples!Kato's inequality}
Kato's inequality in its classical form says the following (see Kato (1973) or [Reed-Simon (1975); X.27]).
Let $f \in L^1_{\text{loc}}(\R^n)$ be such that the distributional Laplacian satisfies $\Delta f \in L^1_{\text{loc}}(\R^n)$.
Then the inequality
%% --
\begin{equation*} \label{eq:c2-K-classic} \tag*{}
\Re((\sign \bar{f})\Delta f) \leq \Delta|f| 
\end{equation*}
%% --
holds in the sense of distributions; \ie
$\langle \phi, \Re((\sign \bar{f})\Delta f) \rangle \leq \langle \phi,\Delta|f| \rangle$ (= $\langle \Delta \phi,|f| \rangle$) holds for all $0 \leq \phi \in C_{c}^{\infty}(\R^n)$.
Note that the closure of $\Delta$ defined on the domain $C_{c}^{\infty}(\R^n)$ generates a strongly continuous positive semigroup on $L^p(\R^n)$ $(1~\leq~p <~\infty)$ (see Example \ref{ex:c2-1.5-4} and Example \ref{ex:c2-4.7}).
\end{example}
%% --
We want to discuss the relation between the classical (distributional) inequality and our version given in Theorem \ref{thm:c2-2.4} .
Let 
%%--
\[
\mathcal{A} = \sum_{|\alpha| \leq m} a_{\alpha} D^{\alpha}
\]
%%-- 
be a differential operator, where $a_{\alpha} \in C_{c}^{\infty}(\R^n)$.
Here we let 
%%--
\[
%%$
D^{\alpha} = (\partial/\partial x_1)^{\alpha_1} \ldots (\partial/\partial x_{n})^{\alpha_{n}}
%%$ 
\]
%%--
for all multi-indices $\alpha = (\alpha_1,\ldots,\alpha_{n}) \in \N_{0}^n$, 
$(\N_{0} \coloneqq \N\cup\{0\})$ of order $|\alpha| \coloneqq \alpha_1 + \ldots + \alpha_{n}$.
We say that $\mathcal{A}$ satisfies Kato's inequality in the sense of distributions if
%% --
\begin{equation}\label{eq:c2-Kd} \tag{$K_{d}$}
\Re\langle(\sign \bar{f})\mathcal{A}f, \phi \rangle \leq \langle|f|, \mathcal{A}^{*}\phi \rangle
\end{equation}
%% --
for all $f \in C_{c}^{\infty}(\R^n)$, $0 \leq \phi \in C_{c}^{\infty}(\R^n)$, where $\mathcal{A}^{*}$ denotes the formal adjoint of $\mathcal{A}$. 

Let now $A$ be the generator of a positive semigroup $(T(t))_{t \geq 0}$ on $E \coloneqq L^p(\R^n)$ $(1 \leq p < \infty)$ or $C_{0}(\R^n)$.
Assume that there exists a core
%%--
%%KGK p-C2-14; LNMp-260
%%
$D_{0}$ of $A$ such that $C_{c}^{\infty} \subset D_{0}$ and $Af = \mathcal{A}f$ in the sense of distributions for all $f \in D_{0}$.
Then $\mathcal{A}$ satisfies Kato's inequality in the sense of distributions.

In fact, let $0 \leq \phi \in C_{c}^{\infty}(\R^n)$.
Then $\langle Af,\phi \rangle = \langle \mathcal{A}f,\phi \rangle = \langle f,\mathcal{A}^{*}\phi \rangle$ for all $f \in D_{0}$.
Since $D_{0}$ is a core of $A$, this implies that $\phi \in D(A')$ and $A'\phi = \mathcal{A}^{*}\phi$.
Thus (\ref{eq:c2-K})   gives $\Re\langle(\sign \bar{f})\mathcal{A}f,\phi \rangle = \Re\langle(\sign \bar{f})Af),\phi \rangle \leq \langle |f|,A'\phi \rangle = \langle |f|,\mathcal{A}^{*}\phi \rangle = \langle \mathcal{A}|f|,\phi \rangle$ for all $f \in C_{c}^{\infty}(\R^n)$, $0 \leq \phi \in C_{c}^{\infty}(\R^n)$.
This is Kato's inequality in the distributional sense.

\begin{remark*}\label{rem:c2-3.0}
%\index{Kato's inequality!ellipticity}
%\index{Elliptic operators}
%\index{Miyajima-Okasawa theorem}
It has been proved by \citet{miyajimaokazawa:1984}] that 

 (\ref{eq:c2-Kd}) implies that $m \leq 2$ and that the principal part $\mathcal{A}_{0} = \sum_{|\alpha|=2} a_{\alpha} D^{\alpha}$ of $\mathcal{A}$ is elliptic; \ie if we write the operator $\mathcal{A}_{0}$ in the form $\mathcal{A}_{0} = \sum_{i,j=1}^{2} b_{ij} \partial^{2}/\partial x_{i} \partial x_{j}$, then the matrix $(b_{ij}(x))$ is positive semidefinite for all $x \in \R^n$.
\end{remark*}

Finally we formulate Theorem \ref{thm:c2-2.4}   for the space $E \coloneq C_{0}(X)$ ($X$ locally compact) (which is not $\sigma$-order complete unless $X$ is $\sigma$-Stonian).
Recall, for $f \in C_{0}(X)$ sign $f$ is defined as a Borel function and for any bounded Borel function $g$ on $X$ and any $\phi \in M(X) = C_{0}(X)'$, we let $\langle g,\phi \rangle = \int g(x) d\phi(x)$ (see B-II, Section 2).
%%\ref{sec:b2-2}
\begin{theorem}\label{thm:c2-2.6}
%\index{Kato's inequality!in $C_{0}(X)$}
%\index{Generators!in $C_{0}(X)$}
%\index{Positive semigroups!in $C_{0}(X)$}
Let $X$ be a locally compact space and $A$ be the generator of a strongly continuous positive semigroup on 
$C_{0}(X)$.
Then
%% --
\begin{equation*}\label{eq:c2-K-3} \tag{K}
\Re\langle(\sign \bar{f})Af,\phi \rangle \leq \langle |f|,A'\phi \rangle \quad (f \in D(A), \phi \in D(A')_{+}).
\end{equation*}
%% --
\end{theorem}
%% --
The proof of Theorem \ref{thm:c2-2.4} can be taken over literally.
Also the analogue of the proof given for $L^p$-spaces (preceding Theorem \ref{thm:c2-2.4})  is valid if one uses B-II, Lemma 2.6.
%%KGK: \ref{lem:b2-2.6}
\section{A Characterization of Generators of Positive Semigroups}\label{sec:c2-3}
\index{Generators of Positive Semigroups}
%\index{Positive semigroups!characterization}
%\index{Characterization!of generators}
In this section we confine ourselves to real Banach lattices.
This does not mean a restriction since every positive semigroup on a complex Banach lattice leaves the real part of the space invariant.
%%--
%%KGK p-C2-15; LNMp-261
%%
\begin{remark}\label{rem:c2-3.1}
%\index{Complex Banach lattice}
%\index{Generators!on complex Banach lattice}
%\index{Semigroups!on complex Banach lattice}
Let $(S(t))_{t \geq 0}$ be a semigroup on a complex Banach lattice $E$ with generator $A$.
Then $S(t)E_{\R} \subset E_{\R}$ for all $t \geq 0$ if and only if
%% --
\begin{equation}\label{eq:c2-3.1}
f \in D(A) \implies \bar{f} \in D(A) \text{ and } A\bar{f} = (Af)^{-}  .
\end{equation}
%% --
In that case the generator $A_{\R}$ of the restriction semigroup on $E_{\R}$ is given by\\  $A_{\R}f = Af$ on $D(A_{\R}) = D(A) \cap E_{\R}$.
\end{remark}
%% --
We will see below that for generators of a strongly continuous semigroup Kato's inequality alone is not sufficient to ensure the positivity of the semigroup.
So we introduce another condition.
%% --
\begin{definition}\label{def:c2-3.2}
%\index{Strictly positive set}
%\index{Positivity!strict}
%\index{Dual space!strictly positive elements}
A subset $M'$ of $E'$ is called \emph{strictly positive} if for every $f \in E_{+}$, $\langle f,\phi \rangle = 0$ for all $\phi \in M'$ implies $f = 0$.
Accordingly, an element $\phi$ of $E'_{+}$ is called \emph{strictly positive} if the set $\{\phi\}$ is strictly positive.
\end{definition}
%% --
\begin{example}\label{ex:c2-3.3}
%\index{Strictly positive elements!in $L^p$ spaces}
%\index{$L^p$ spaces!strictly positive elements}
%\index{Examples!Strictly positive elements}
Let $E = L^p(X,\mu)$ $(1 \leq p < \infty)$, where $(X,\mu)$ is a $\sigma$-finite measure space.
Then $\phi \in E'
%%Claude: _{+} 
= L^q(X,\mu)$ (where $1/p + 1/q = 1$) is strictly positive if and only if $\phi(x) > 0$ $\mu$-a.e.
Note that strictly positive elements of $E'$ always exist in this case.
\end{example}
%% --
\begin{definition}\label{def:c2-3.4}
%\index{Subeigenvector!positive}
%\index{Positive subeigenvector}
%\index{Operators!positive subeigenvector}
Let $B$ be an operator on a Banach lattice $F$ and let $u \in F$.
Then $u$ is called a \emph{positive subeigenvector} of $B$ if
\begin{enumerate}[\upshape (i)]
%%
\item \label{def:c2-3.4-1}
$0 < u \in D(B)$ and
%%--
\item \label{def:c2-3.4-2}
$Bu \leq \lambda u$ for some $\lambda \in \R$.
\end{enumerate}
\end{definition}
%% --
\begin{proposition}\label{prop:c2-3.5}
%\index{Subeigenvector!of adjoint generator}
%\index{Positive semigroups!subeigenvectors}
%\index{Generators!subeigenvectors of adjoint}
Let $(T(t))_{t \geq 0}$ be a positive semigroup on a real Banach lattice with generator $A$.
Then there exists a strictly positive set $M'$ of subeigenvectors of $A'$ (the adjoint of the generator $A$).
Moreover, if there exist strictly positive linear forms on $E$, then there exists a strictly positive subeigenvector of $A'$.
\end{proposition}
%% --
\begin{proof}
Let $\lambda > 0$ be such that $R(\lambda,A) = (\lambda - A)^{-1}$ exists and such that $R(\lambda,A) \geq 0$.
Let $N' \subset E'_{+}$ be strictly positive.
Then $M' \coloneq \{R(\lambda,A)'\psi \colon \psi \in N'\} \subset D(A')_{+}$ .
We show that $M'$ is strictly positive.
Indeed, let $f \in E_{+}$ such that $\langle f,\phi \rangle = 0$ for all $\phi \in M'$.
Then $\langle R(\lambda,A)f,\psi \rangle = 0$ for all $\psi \in N'$.
Hence $R(\lambda,A)f = 0$ since $N'$ is strictly positive.
Consequently, $f = 0$.
The set $M'$ consists of
%%--
%%KGK p-C2-16; LNMp-262
%%--
subeigenvectors of $A'$. In fact, let $\psi \in N'$, $\phi = R(\lambda,A)'\psi$. Then
$A'\phi = \lambda\phi - \psi \leq \lambda\phi$.
\end{proof}

The fact that $\phi \in D(A')_{+}$ is a subeigenvector can be expressed by the
semigroup (if it is positive).
%% --
\begin{proposition}\label{prop:c2-3.6}
Assume that $A$ is the generator of a positive semigroup $(T(t))_{t \geq 0}$ on a real Banach lattice $E$. 
Let $\phi \in D(A')_{+}$ and $\lambda \in \R$. 
Then
%%
%%A'\phi \leq \lambda \phi if and only if \,T(t)'\phi \leq \mathrm{e}^{\lambda t}\phi \quad (t \geq 0).
\begin{center}
$A'\phi \leq \lambda \phi$ if and only if  $T(t)'\phi \leq \mathrm{e}^{\lambda t}\phi$ $(t \geq 0)$.
\end{center}
%\marginpar{?text upright?}
\end{proposition}
%% --
\begin{proof}
If $T(t)'\phi \leq \mathrm{e}^{\lambda t} \phi$ for all $t \geq 0$, then
%% --
\begin{align*}
A'\phi = \sigma(E',E)\text{-}\lim_{t \to 0} 1/t \cdot (T(t)'\phi - \phi) \leq \lim_{t \to 0} 1/t \cdot (\mathrm{e}^{\lambda t}\phi - \phi) = \lambda\phi.
\end{align*}
%% --
For the converse let $f \in E_{+}$. Then
%% --
\begin{align*}
\langle f,T(t)'\phi \rangle &= \langle f,\phi \rangle + \int_{0}^{t} \langle f,T(s)'A'\phi \rangle \, \ds\\
&\leq \langle f,\phi \rangle + \lambda \int_{0}^{t} \langle f,T(s)'\phi \rangle \, \ds.
\end{align*}
%% --
It follows from Gronwall's lemma that $\langle f,T(t)'\phi \rangle \leq \mathrm{e}^{\lambda t} \langle f,\phi \rangle$.
\end{proof}
%% --
\begin{remark}\label{rem:c2-3.7}
\begin{enumerate}[\upshape (i), wide, labelsep=.5em]%%, itemindent=\parindent]
%%
\item \label{rem:c2-3.7-1}
Using Proposition \ref{prop:c2-3.6} it is immediately clear that
$(T(t))_{t \geq 0}$ is irreducible if and only if every positive subeigenvector
of $A'$ is strictly positive (cf. C-III, Definition 3.1). 

%%--
\item \label{rem:c2-3.7-2}
In the proof of the "only if" - part of Proposition \ref{prop:c2-3.6} we needed the
positivity of the semigroup in order to be able to apply Gronwall's
lemma.
However, if instead of assuming that the semigroup is positive
we suppose that $A$ satisfies Kato's inequality and $A'\phi \leq \lambda\phi$ for some strictly positive $\phi \in D(A')$ then we will show that $T(t)'\phi \leq
\mathrm{e}^{\lambda t} \phi$ and that the semigroup is positive (see Corollary~\ref{cor:c2-3.9}).
\end{enumerate}
\end{remark}
%% --
The following is our characterization.
%% --
\begin{theorem}\label{thm:c2-3.8}
Let $(T(t))_{t \geq 0}$ be a semigroup on a $\sigma$-order complete real Banach lattice $E$. 
The semigroup is positive if and only if its generator $A$ satisfies the following condition.
There exists a core $D_{0}$ of $A$ and a strictly positive set $M'$ of subeigenvectors of $A'$ such that
%% --
\begin{equation}\label{eq:c2-K4} \tag{K}
\langle(\sign f) Af,\phi \rangle \leq \langle|f|,A'\phi\rangle \quad \text{for all } f \in D_{0} \, , \, \phi \in M'.
\end{equation}
%% --
\end{theorem}
%%--
%%KGK p-C2-17; LNMp-263
%%
\begin{corollary}\label{cor:c2-3.9}
Assume in addition that $E'$ contains a strictly positive functional. 
Then the semigroup is positive if and only if there
exists a core $D_{0}$ of $A$ and a strictly positive subeigenvector $\phi$
of $A'$ such that
%% --
\begin{equation}\label{eq:c2-K5} \tag{K}
\langle(\sign f) Af,\phi \rangle \leq \langle|f|,A'\phi\rangle \quad \text{for all } f \in D_{0}.
\end{equation}
%% --
\end{corollary}
%% --
From the proof of Theorem \ref{thm:c2-3.8}   we isolate the following
%% --
\begin{proposition}\label{prop:c2-3.10}
Let $B$ be a densely defined operator on $E$ and $D_{0}$ be a core of $B$.
Suppose that $\phi \in D(B')_{+}$ is such that $B'\phi \leq 0$.
Denote by $p$ the sublinear functional given by $p(f) = \langle f^{+},\phi \rangle$. 
If
%% --
\begin{equation*}\label{eq:c2-K6} \tag{K}
	\langle(\sign f) Bf,\phi \rangle \leq \langle|f|,B'\phi\rangle \quad (f \in D_{0}),
\end{equation*}
%% --
then $B$ is $p$-dissipative.
\end{proposition}
%% --
\begin{proof}
Let $f \in D_{0}$. 
Set $P_{+} \coloneqq P_{f^{+}}$, $P_{-} \coloneqq P_{f^{-}}$ and let
$P \coloneqq \Id - P_{+} - P_{-}$, $Q = P_{+} + 1/2 \, P$ and $\psi = Q'\phi$. We show that
%% --
\begin{equation}\label{eq:c2-3.2}
\psi \in \ddp(f).
\end{equation}
%% --
Let $g \in E$. 
Since $0 \leq Q \leq \Id$ we have 
$\langle g,\psi \rangle = \langle Qg,\phi \rangle \leq \langle Qg^{+},\phi \rangle \leq
\langle g^{+},\phi \rangle = p(g)$. 
Moreover, $\langle f, \psi \rangle = \langle Qf,\phi \rangle = \langle P_{+}f + 1/2 \, Pf,\phi \rangle = \langle f^{+},\phi \rangle = p(f^{+})$. 
So \eqref{eq:c2-3.2} follows by the definition of $\ddp(f)$. 
We show that
%% --
\begin{equation}\label{eq:c2-3.3}
\langle Bf,\psi \rangle \leq 0.
\end{equation}
%% --
One has trivially
%% --
\begin{equation}\label{eq:c2-3.4}
\langle(P_{+} + P_{-} + P) Bf,\phi \rangle = \langle f,B'\phi \rangle.
\end{equation}
%% --
Addition of \eqref{eq:c2-3.4} and (\ref{eq:c2-K6})   gives
$\langle(2P_{+} + P)Bf,\phi \rangle \leq \langle 2f^{+},B'\phi \rangle \leq 0$.
Hence $\langle Bf,\psi \rangle = \langle QBf,\phi \rangle \leq 0$.
Thus we have proved that $B|D_{0}$ is $p$-dissipative. 
Hence $B$ is $p$-dissipative as well (by A-II, Corollary 2.5).
%%KGK: \ref{cor:a2-2.5}
\end{proof}
%% --
\begin{proof}[Proof of Theorem \ref{thm:c2-3.8}]
Proposition \ref{prop:c2-3.5} and Theorem \ref{thm:c2-2.4}   yield one implication. 
In order to show the other assume that the condition in
Theorem \ref{thm:c2-3.8}   is satisfied. 
We have to show that $T(t) \geq 0$ for all $t \geq 0$.
Let $\phi \in M'$. 
Consider the half-norm $p(f) = \langle f^{+},\phi \rangle$ and the operator $B = A - \lambda$, where $\lambda \in \R$ is such that $A'\phi \leq \lambda\phi$. 
Then $B$ satisfies $B'\phi \leq 0$ and \ref{eq:c2-K6}   as well. 
So it follows from Proposition \ref{prop:c2-3.10}   that $B$ is $p$-dissipative.
Since $B$ generates the semigroup $(\mathrm{e}^{-\lambda t}T(t))_{t \geq 0}$ we obtain fromA-II, Theorem 2.6 that $p(\mathrm{e}^{-\lambda t}T(t)f) \leq p(f)$ $(f \in E\, , t\geq 0)$.
Hence,
%% --
\begin{equation} \label{eq:c2-3.5}
\langle (T(t)f)^{+}, \phi \rangle \leq \mathrm{e}^{\lambda t} \langle f^{+}, \phi \rangle \quad (f \in E, t \geq 0).
\end{equation}
%% --
Now let $t > 0$ and $f \leq 0$; then $f^{+} = 0$. 
It follows from \eqref{eq:c2-3.5} that $\langle (T(t)f)^{+},\phi \rangle \leq 0$.
Since $\phi \in M'$ is arbitrary and $M'$ is strictly positive, it follows that $(T(t)f)^{+} = 0$; \ie $T(t)f \leq 0$. 
This implies that $T(t) \geq 0$.
\end{proof}
%% --
\begin{remark}\label{rem:c2-3.11} ~ 
%% --
\begin{enumerate}[\upshape (i), wide, labelsep=.5em]%%, itemindent=\parindent]
%%
\item \label{rem:c2-3.11-1}
The proof of Theorem \ref{thm:c2-3.8}   shows the following. 
If $A$ is the generator of a positive semigroup and $E'$ contains strictly
positive linear forms, then there exist a continuous half-norm $p$ on
$E$ and $w \in \R$ such that $A - w$ is $p$-dissipative. 
We stress that $p$ cannot be replaced by the norm (or by $N^{+}$), since in general none of the semigroups $(\mathrm{e}^{-wt}T(t))_{t \geq 0}$ $(w \in \R)$ is contractive for the norm (cf. \citet{derndinger:1984} and \citet{battydavies:1983}).
%% \medskip
%%--
\item  \label{rem:c2-3.11-2}
Using Proposition \ref{prop:c2-3.10}   one can show with the help of the proof of 
%%KGK: \ref{prop:a2-2.9}
A-II,  Proposition 2.9 that a densely defined operator is closable whenever
there exists a strictly positive set $M'$ of subeigenvectors of $A'$
such that \eqref{eq:c2-K6}   holds for all $f \in D(A)$ and $\phi \in M'$.
\end{enumerate}
\end{remark}
%% --
\begin{remark}\label{rem:c2-3.12}
In Theorem \ref{thm:c2-3.8} and Corollary \ref{cor:c2-3.9}  one can replace inequality \eqref{eq:c2-K}   by the inequality
%% --
\begin{equation}\label{eq:c2-3.6}
\langle P_{(f^{+})}Af,\phi \rangle \leq \langle f^{+},A'\phi \rangle,
\end{equation}
%% --
(with the notation of Proposition \ref{prop:c2-3.10}).

Indeed, \eqref{eq:c2-3.6} for $-f$ yields $\langle -P_{(f^{-})}Af,\phi \rangle \leq \langle f^{-},A'\phi \rangle$. 
Adding up both inequalities one obtains $\langle(\sign f)Af,\phi \rangle \leq \langle |f|,A'\phi \rangle$.
On the other hand, if $A$ generates a positive semigroup, one sees by
the obvious alterations in the proof of Theorem \ref{thm:c2-2.4}   that \eqref{eq:c2-3.6} holds for all $f \in D(A)$ and $\phi \in D(A')_{+}$. 

Next we formulate the result for the space $C_{0}(X)$, where $X$ is a locally compact space (concerning the notation cf. Theorem \ref{thm:c2-2.6} and Section~2 of B-II).
%%KGK: \ref{sec:b2-2}   
\end{remark}
%% --
\begin{theorem}\label{thm:c2-3.13}
Let $A$ be the generator of a semigroup on $C_{0}(X)$.
The semigroup is positive if and only if there exists a core $D_{0}$ of
$A$ and a strictly positive set $M'$ of subeigenvectors of $A'$ such
that
%% --
\begin{equation}\label{eq:c2-K7} \tag{K}
\langle(\sign f)Af,\phi \rangle \leq \langle |f|,A'\phi \rangle \quad \text{for all } f \in D_{0}, \phi \in M'.
\end{equation}
%% --
\end{theorem}
%%--
%%KGK p-C2-19; LNMp-265
%%
This theorem can be proved in the same way as Theorem \ref{thm:c2-3.8}  .
%% --
\begin{remark*}
If $X$ is separable, then there exist strictly positive
measures on $C_{0}(X)$. 
In that case the analogue of Corollary \ref{cor:c2-3.9}   holds
as well.
\end{remark*}
%% --
Now we want to discuss the results obtained so far.

As a first example we consider the first derivative with boundary
conditions on %
\[
	E = L^{p}[0,1] \quad (1 \leq p < \infty) .
\]
%
By $\text{AC}[0,1]$ we denote the space of all absolutely continuous functions on $[0,1]$. 
Let $A_{\max}$ be given by
%% --
\begin{align*}
D(A_{\max}) &= \{f \in \text{AC}[0,1] \colon f' \in L^{p}[0,1]\}\\
A_{\max}f &= f' \quad (f \in D(A_{\max})).
\end{align*}
%% --
The following lemma is easy to prove.
%% --
\begin{lemma}\label{lem:c2-3.14}
Let $f \in \text{AC}[0,1]$. 
Then $|f| \in \text{AC}[0,1]$ and $|f|' = (\sign  f) \cdot f'$ (a.e.).
\end{lemma}
%% \medskip
As a consequence of the lemma, $D(A_{\max})$ is a sublattice of $E$ and
%% --
\begin{equation}\label{eq:c2-3.7}
(\sign  f)A_{\max}f = A_{\max}|f| \quad (f \in D(A_{\max})).
\end{equation}
%% --
For $\lambda > 0$ one has
%% --
\begin{equation}\label{eq:c2-3.8}
\text{ker}(\lambda - A_{\max}) = \R \cdot e_{\lambda} \quad \text{where }  e_{\lambda}(x) = \mathrm{e}^{\lambda x}.
\end{equation}
%% --

Hence $A_{\max}$ is not a generator. 
We impose the following boundary conditions.
Let $d \in \R$. 
Consider the restriction $A_{d}$ of $A_{\max}$ to the domain
%% --
\begin{equation*}
D(A_{d}) = \{f \in D(A_{\max}) \colon f(1) = \diff{}f(0)\}.
\end{equation*}
%% --
Then $A_{d}$ is the generator of the semigroup $(T_{d}(t))_{t \geq 0}$ given by
%% --
\begin{equation}\label{eq:c2-3.9}
T_{d}(t)f(x) = d^{n} \cdot f(x+t-n) \quad \text{if } x+t \in [n, n+1) \quad (n \in \N ).
\end{equation}
%% --

This is not difficult to prove. 
Actually \eqref{eq:c2-3.9} defines a group if $d \neq 0$ and if we allow $t \in \R$, $n \in \Z $. For $d = 0$ one obtains the nilpotent shift semigroup on $E$.
It follows from \eqref{eq:c2-3.9} that the semigroup $(T_{d}(t))_{t \geq 0}$ is positive if and only if $d \geq 0$.

Let us fix $d < 0$. Let $A = A_{d}$ and $T(t) = T_{d}(t)$ for $t \geq 0$. Then
%%--
%%KGK p-C2-20; LNMp-266
%%
$(T(t))_{t \geq 0}$ is a semigroup which is \emph{not positive}. Nevertheless its generator $A$ satisfies Kato's inequality. Even the equality is valid;
\ie
%% --
\begin{equation}\label{eq:c2-3.10}
\langle(\sign  f) Af,\phi\rangle = \langle|f|,A'\phi \rangle \quad \text{for all } f \in D(A), 0 \leq \phi \in D(A').
\end{equation}
%% --

\begin{proof}
It is not difficult to see that
%% --
%%Claude. \begin{align*}
\begin{equation}\label{eq:c2-3.11}
%%\text{(3.11)} 
\begin{array}{rl}
D(A') &= \{\phi \in \text{AC}[0,1] \colon \phi' \in L^{q}[0,1], \phi(0) = d\phi(1)\}\\ 
A'\phi &= -\phi' \quad \text{for all } \phi \in D(A').
\end{array}
%%Claude. \end{align*}
\end{equation}
%% --
where $1/p + 1/q = 1$. Let $\phi \in D(A')_{+}$. 
Since $d < 0$, it follows that $\phi(0) = \phi(1) = 0$. 
Hence for $f \in D(A)$,
%% --
\begin{align*}
\langle(\sign  f)Af,\phi\rangle &= \langle(\sign  f)f',\phi\rangle = \langle|f|',\phi\rangle\\
&= \int_{0}^{1} |f|'(x) \, \phi(x) \,\ dx\\
&= |f(1)|\phi(1) - |f(0)|\phi(0) - \int_{0}^{1} |f(x)| \, \phi'(x) \, \dx\\
&= |f(1)|\phi(1) - |f(0)|\phi(0) + \langle|f|,A'\phi\rangle\\
&= \langle|f|,A'\phi\rangle
\end{align*}
%% --
\end{proof}

\begin{remark}\label{rem:c2-3.15}
The equality \eqref{eq:c2-3.10} does not hold for all $\phi \in D(A')$. 
In fact, this would imply that $|f| \in D(A)$ and $(\sign  f)Af = A|f|$ for
all $f \in D(A)$. 
Thus by Corollary  5.8 
%%KGK: forward \ref{cor:c2-3.10}
below the semigroup would be positive. 
The reason why in this example the equality holds will be explained from a more general point of view in Section 5 (see Remark 5.12).
%%KGK: forward ref{rem:c2-5.12}
\end{remark}
%% --
Relation \eqref{eq:c2-3.10} shows that $A$ also satisfies Kato's inequality
formally in the strong sense. 
In order to formulate this more precisely, observe that it follows from \eqref{eq:c2-3.8} that $D(A_{\max}) = D(A) + \R \cdot e_{\lambda}$ (for any fixed $0 < \lambda \in \rho(A)$). Thus the extension $A_{\max}$ of $A$ satisfies the following.
%% --
\begin{align}\label{eq:c2-3.12}
A_{\max} 	& \quad \text{is closed}. \\
D(A_{\max}) & \quad \text{is a sublattice of } E. \\
D(A) & \quad \text{has codimension one in } D(A_{\max}) \\
(\sign  f)Af &= A_{\max}|f| \quad \text{for all } f \in D(A).
\end{align}
%% --
It is also remarkable that there exists a dense sublattice 
%
\[
	D_{0} \coloneq \{f \in D(A) \colon f(0) = f(1) = 0\} 
\]
%
of $E$ which is included in $D(A)$.
But $D_{0}$ is not a core of $A$ (this would imply the positivity of the semigroup by Theorem \ref{thm:c2-1.8}   if $|d| \leq 1$).
%%--
%%KGK p-C2-21; LNMp-267
%%
Since $(T(t))_{t \geq 0}$ is not positive but Kato's inequality holds, it
follows from Theorem \ref{thm:c2-3.8}   that there does not exist a strictly positive subeigenvector of $A'$. 
In fact, even the following is true.
%% --
\begin{equation} \label{eq:c2-3.16} 
0 \leq \phi \in D(A'), \, A'\phi \leq \mu\phi \text{ for some } \mu \in \R \text{ implies } \phi = 0.
\end{equation}
%% --
\begin{proof}
Suppose that $0 \leq \phi \in D(A')$ such that $-\phi' = A'\phi \leq \mu\phi$. 
We can assume that $\mu \geq 0$. 
Let $\psi(x) = \phi(1-x)$. 
Then $\psi'(x) = -\phi'(1-x) \leq \mu\phi(1-x) = \mu\psi(x)$. 
Since $\psi(0) = 0$, we obtain
%% --
\begin{equation*} \label{eq:c2-3-KGK} 
\psi(x) = \int_{0}^{x} \psi'(y) \, \dy \leq \mu \int_{0}^{x} \psi(y) \, \dy \quad (x \in [0,1]).
\end{equation*}
%% --
It follows from Gronwall's lemma that $\psi \leq 0$. Hence $\phi = \psi = 0$.
\end{proof}
%% --
\begin{remark}\label{rem:c2-3.16}
Let $B$ be the generator of a strongly continuous semigroup on a real Banach lattice with order continuous norm. Assume that the following two conditions hold.
%% --
\begin{align}
\langle(\sign  f)Bf,\phi\rangle &\leq \langle|f|,B'\phi\rangle \quad (f \in D(B), \, \phi \in D(B')_{+}). \label{eq:c2-3.17-K} \tag{K}\\
(D(B')_{+})^{-\sigma(E',E)} &= E'_{+}. \label{eq:c2-3.17}
\end{align}
%% --
Because of (3.17) condition (K) implies that $P_{f}Bf \leq (\sign  f)Bf \leq Bf$
whenever $f \in D(B)_{+}$.

This is Kato's inequality in the strong form for positive $f \in D(B)$ and is equivalent to $(Bf)^{-} \in \{f\}^{dd} = \overline{E_f}$ $(f \in D(A)_{+})$ 
(recall that $E$ has order continuous norm). By Lemma 
\ref{lem:c2-1.10}  this again is equivalent to
%% --
\begin{equation}\label{eq:c2-P2} \tag{P}
0 \leq f \in D(B), \, \phi \in E'_{+}, \, \langle f,\phi \rangle = 0 \text{ implies } \langle Bf,\phi \rangle \geq 0.
\end{equation}
%% --
It is easy to see that the operator $A$ in the example satisfies conditions \eqref{eq:c2-3.17-K}  and \eqref{eq:c2-3.17}. Thus the positive minimum principle (P) is not sufficient for the positivity of the semigroup.
\end{remark}
%%--
In view of the preceding example and remarks one might presume that the existence of a strictly positive set of subeigenvectors of the adjoint of the generator actually implies the positivity of the semigroup. 
This is not the case.

To give an example consider $E = L^{2}(\R)$ and the operator $B$ given by 
$Bf = f^{(3)}$ with domain
\[
D(B) = \{f \in L^{2}(\R) \colon f \in C^{2}(\R); f'' \in \text{AC}(\R) ; f,f',f'',f^{(3)} \in L^{2}(\R)\}
\]
%%--
%%KGK p-C2-22; LNMp-268
%%
Then $B$ is the generator of a unitary group $(U(t))_{t \in \R}$. 
In particular, $B$ is skew-adjoint, i.e. $B' = -B$.
Moreover, we claim that
%% --
\begin{equation}\label{eq:c2-3.18}
B' \text{ has a strictly positive subeigenvector } \phi.
\end{equation}
%% --
\begin{proof}
Let $\lambda > 0$ and $\phi \in C^{3}(\R)$ such that $\phi(x) = \mathrm{e}^{-|x|}$ for
$|x| \geq 1$, $\phi(x) > 0$ for all $x \in \R$, $\phi(0) = 1$ and $\phi'(0) = \phi''(0) = 0$.
Then $\phi \in D(B')$. 
Moreover, $-\phi^{(3)}(x)~\leq~\phi(x)$ for $|x| \geq 1$. 
Hence there exists $\mu \geq 1$ such that
%
\[
	B'\phi = -\phi^{(3)} \leq \mu\phi .
\]
%
\phantom{x}
\end{proof}

But the semigroup $(U(t))_{t \geq 0}$ is not positive. 
In fact, we show that there exists $f~\in~D(B)$ such that
%% --
\begin{equation}\label{eq:c2-3.19}
\langle(\sign  f)Bf,\phi\rangle > \langle|f|,B'\phi\rangle.
\end{equation}
%% --
\begin{proof}
Let $f \in D(B)$ be such that $f(x) = \mathrm{e}^{-x} \sin x$ in a neighborhood of $0$, while $f(x) > 0$ for $x > 0$ and $f(x) < 0$ for $x < 0$. 
Then
%% --
\begin{align*}
\langle(\sign  f)Bf,\phi\rangle = - \int_{-\infty}^{0} f^{(3)}(x)\phi(x) \, \dx + \int_{0}^{\infty} f^{(3)}(x)\phi(x) \, \dx.
\end{align*}
%% --
Hence, 
%% --
\begin{align*}
\langle|f|,B'\phi\rangle &= \int_{-\infty}^{0} (-f(x))(-\phi^{(3)}(x)) \, \dx + \int_{0}^{\infty} f(x)(-\phi^{(3)}(x)) \, \dx\\
&= - \int_{-\infty}^{0} f^{(3)}(x)\phi(x) \, \dx + \int_{0}^{\infty} f^{(3)}(x)\phi(x) \, \dx\\
&+ [f''\phi]_{-\infty}^{0} - [f''\phi]_{0}^{\infty} \quad \text{(since $\phi''(0)=\phi'(0)=0$)}\\
&= \langle(\sign  f) Bf,\phi\rangle + 2f''(0)\phi(0)\\
&< \langle(\sign  f) Bf,\phi\rangle \quad \text{(since $f''(0)\phi(0) = f''(0) = -2$)}.
\end{align*}
%% --
\phantom{x}
\end{proof}
%% --
We now show that $B$ satisfies Kato's inequality for positive elements, though; \ie
%% --
\begin{equation}\label{eq:c2-3.20}
P_{f} Bf \leq Bf \quad \text{for all } f \in D(B)_{+}.
\end{equation}
%% --
In fact, more is true. $B$ is \emph{local}, i.e.
%% --
\begin{equation}\label{eq:c2-3.21}
f \perp g \quad \text{implies} \quad Bf \perp g \quad \text{for all } f \in D(B), g \in L^{2}(\R).
\end{equation}
%% --
\begin{proof}
Let $A$ be the generator of the translation group which, in particular, is a lattice semigroup (see Section 5). We obtain from Proposition \ref{prop:c2-5.4} below that $A$ is local. 
Hence $B = A^{3}$ is local as well.
\end{proof}
%%--
%%KGK p-C2-23; LNMp-269
%%
This example shows that even if there exists a strictly positive subeigenvector of the adjoint of the generator, Kato's inequality for positive elements alone does not suffice for the positivity of the semigroup. 
Note also that (because of the order continuous norm) Kato's inequality holds for positive elements if and only if the positive minimum principle is satisfied (see Remark \ref{rem:c2-3.16}).
%%--
\section{Domination of Semigroups} \label{sec:c2-4}
%%--
Frequently it is useful to be able to compare two semigroups on a
Banach lattice with respect to the ordering (for example, in order to
decide whether a semigroup is stable (see Chapter A-IV and Example \ref{ex:c2-4.14}   ).

In this section we assume that $E$ is a $\sigma$-order complete complex Banach lattice. 
Let $(T(t))_{t \geq 0}$ be a positive semigroup with generator $A$ and $(S(t))_{t \geq 0}$ a semigroup with generator $B$. 
We say, $(T(t))_{t \geq 0}$ \emph{dominates} $(S(t))_{t \geq 0}$ if
%% --
\begin{equation}\label{eq:c2-4.1}
|S(t)f| \leq T(t)|f| \quad \text{for all } f \in E, t > 0.
\end{equation}
%% --

We first observe that domination of the semigroup is equivalent to domination of the resolvents.

\begin{proposition}\label{prop:c2-4.1}
The semigroup $(T(t))_{t \geq 0}$ dominates $(S(t))_{t \geq 0}$ if and only if
%% --
\begin{equation}\label{eq:c2-4.2}
|R(\lambda,B)f| \leq R(\lambda,A)|f| \quad (f \in E) \quad \text{for large real }  \lambda.
\end{equation}
%% --
\end{proposition}
%% --
\begin{proof}
\eqref{eq:c2-4.2} follows from \eqref{eq:c2-4.1} since the resolvent is given by the Laplace transform of the semigroup. 
Conversely, if \eqref{eq:c2-4.2} holds, then
%% --
\begin{align*}
|S(t)f| &= \lim_{n \to \infty} |((n/t)R(n/t,B))^{n}f|\\
&\leq \lim_{n \to \infty} ((n/t)R(n/t,A))^{n}|f|\\
&= T(t)|f| \quad (t \geq 0, f \in E).
\end{align*}
%% --
\end{proof}
%%--
%%KGK p-C2-24; LNMp-270
%%
One can describe domination by an inequality for the generators in a manner analoguous to the characterization of positive semigroups in Section 1; however, no positive subeigenvectors are needed here.

\begin{theorem}\label{thm:c2-4.2}
Let $(T(t))_{t \geq 0}$ be a positive semigroup with generator $A$ and $(S(t))_{t \geq 0}$ a semigroup with generator $B$. The following assertions are equivalent.
\begin{enumerate}[\upshape (a)]
%%
\item \label{thm:c2-4.2-1}
$|S(t)f| \leq T(t)|f|$ for all $f \in E$, $t \geq 0$.
%%--
\item \label{thm:c2-4.2-2}
$\Re\langle(\sign  f)Bf,\phi\rangle \leq \langle|f|,A'\phi\rangle$ for all $f \in D(B)$, $\phi \in D(A')_{+}$.
\end{enumerate}
\end{theorem}

\begin{proof}
\ref{thm:c2-4.2-1}   implies \ref{thm:c2-4.2-2}  . Let $f \in D(B)$, $\phi \in D(A')_{+}$. Then
%% --
\begin{align*}
\Re\langle(\sign  \bar{f})Bf, \phi\rangle &= \Re  \langle(\sign  \bar{f})\lim_{t \to 0}1/t(S(t)f - f),\phi\rangle\\
&= \langle\lim_{t \to 0}1/t(\Re\langle(\sign  \bar{f}) S(t)f\rangle - |f|),\phi\rangle\\
&\leq \lim_{t \to 0}\langle 1/t(|S(t)f| - |f|),\phi\rangle\\
&\leq \lim_{t \to 0}\langle 1/t(T(t)|f| - |f|),\phi\rangle = \langle|f|,A'\phi\rangle.
\end{align*}
%% --
\ref{thm:c2-4.2-2}   implies \ref{thm:c2-4.2-1}  . 
Let $\lambda > \max\{\omega (A),\omega (B)\}$ and $g \in E$. 
We show that
%% --
\begin{equation}\label{eq:c2-4.3}
|R(\lambda,B)g| \leq R(\lambda,A)|g|.
\end{equation}
%% --
Let $\psi \in E'_{+}$. Then $\phi \coloneq R(\lambda,A)'\psi \in D(A')_{+}$.
Setting $f \coloneq R(\lambda,B)g \in D(B)$ we obtain by \ref{thm:c2-4.2-2}  
%% --
\begin{align*}
\langle|R(\lambda,B)g|,\psi\rangle &= \langle|f|,(\lambda-A')\phi\rangle \leq \langle\lambda|f|,\phi\rangle - \Re\langle(\sign  \bar{f})Bf,\phi\rangle\\
&= \Re\langle(\sign  \bar{f})(\lambda f - Bf),\phi\rangle \\
&= \Re\langle(\sign  \bar{f})g,\phi\rangle \\
&\leq \langle|g|,\phi\rangle = \langle R(\lambda,A)|g|,\psi\rangle
\end{align*}
%% --
Since $\psi \in E'_{+}$ is arbitrary \eqref{eq:c2-4.3} follows.
\end{proof}

In order to deduce that \ref{thm:c2-4.2-2}   implies \ref{thm:c2-4.2-1}   in Theorem \ref{thm:c2-4.2}  , it is not
necessary to assume that $B$ is a generator. 
Merely a range condition is sufficient. 
The precise formulation is the following.

\begin{theorem}\label{thm:c2-4.3}
Let $(T(t))_{t \geq 0}$ be a positive semigroup with generator $A$. 
Let $B$ be a densely defined operator such that
%% --
\begin{equation}\label{eq:c2-4.4}
\Re\langle(\sign  \bar{f})Bf,\phi\rangle \leq \langle|f|,A'\phi\rangle
\end{equation}
%% --
for all $f \in D(B)$, $\phi \in D(A')_{+}$.
Then $B$ is closable. 
Moreover, if $(\lambda - B)D(B)$ is dense in $E$ for some $\lambda > \max\{0,s(A)\}$, then $\overline{B}$ (the closure of $B$) generates a
semigroup which is dominated by $(T(t))_{t \geq 0}$.
\end{theorem}

\begin{proof}
1. We show that $B$ is closable.
Let $u_{n} \in D(B)$ satisfy $u_{n} \to 0$ and $Bu_{n} \to v$ $(n \to \infty)$.
We have to
%%--
%%KGK p-C2-25; LNMp-271
%%
show that $v = 0$. 
Considering $A - \mu$ and $B - \mu$ for some $\mu > s(A)$ instead of $A$ and $B$ we may assume that $s(A) < 0$. 
Then there exists a strictly positive set $M' \subset E'$ such that
%% --
\begin{equation}\label{eq:c2-4.5}
\phi \in D(A') \text{ and } A'\phi \leq 0 \text{ for all } \phi \in M'
\end{equation}
%% --
(see the proof of Proposition \ref{prop:c2-3.5}).

Let $\phi \in M'$ and $p$ be the seminorm given by $p(f) = \langle|f|,\phi\rangle$. 
We show that $B$ is $p$-dissipative (see end of A-II, Section 2).
%%KGK \ref{sec:a2-2}
Let $f \in D(B)$, $\psi = (\sign  \bar{f})'\phi$. 
Then it is easy to see that $\psi \in \ddp(f)$. 
Moreover, by \eqref{eq:c2-4.4} and \eqref{eq:c2-4.5} one obtains that 
$\Re\langle Bf,\psi\rangle = \Re\langle(\sign  \bar{f})Bf,\phi\rangle \leq \langle|f|,A'\phi\rangle \leq 0$. 
Thus $B$ is $p$-dissipative. 
By the proof of A-II, Proposition 2.9 
%%KGG: \ref{prop:a2-2.9}
one sees that $p(v) = 0$; \ie $\langle|v|,\phi\rangle = 0$.
Since $\phi \in M'$ was arbitrary we conclude that $v = 0$.

2. Let $\lambda > \lambda_{0} \coloneq \max\{s(A),0\}$. 
We show that for $f \in D(B)$,
%% --
\begin{equation}\label{eq:c2-4.6}
g = (\lambda - B)f \text{ implies } |f| \leq R(\lambda,A)|g|.
\end{equation}
%% --
Let $\psi \in E'_{+}$. 
We have to show that 
$\langle|f|,\psi\rangle \leq \langle R(\lambda,A)|g|,\psi\rangle$.
Let $\phi = R(\lambda,A)'\psi \in D(A')_{+}$. 
Then by \eqref{eq:c2-4.4}
%% --
\begin{align*}
\langle|f|,\psi\rangle &= \langle|f|, (\lambda - A')\phi\rangle \\
&= \Re\langle(\sign  \bar{f})(\lambda f),\phi\rangle - \langle|f|,A'\phi\rangle\\
&\leq \Re\langle(\sign  \bar{f})(\lambda - B)f,\phi\rangle = \Re\langle(\sign  \bar{f})g,\phi\rangle\\
&\leq \langle|g|,\phi\rangle = \langle R(\lambda,A)|g|,\psi\rangle .
\end{align*}
%% --
It follows from \eqref{eq:c2-4.6} that for $\lambda > \lambda_{0}$ and $f \in D(\overline{B})$
%% --
\begin{equation}\label{eq:c2-4.7}
g = (\lambda - \overline{B})f \text{ implies } |f| \leq R(\lambda,A)|g|.
\end{equation}
%% --
In particular, $(\lambda - \overline{B})$ is injective for $\lambda > \lambda_{0}$. 
Moreover,
%% --
\begin{equation}\label{eq:c2-4.8}
\begin{split}
&|R(\lambda,\overline{B})g| \leq R(\lambda,A)|g| \quad \text{for all } g \in E\\
&\text{whenever} \lambda_{0} < \lambda \in \rho(\overline{B}).
\end{split}
\end{equation}
%% whenever $\lambda_{0} < \lambda \in \rho(\overline{B})$.
%% --
Assume now that $\mu > \lambda_{0}$ such that $(\mu - B)D(B)$ is dense in $E$. 
Then 
$(\mu - \overline{B})D(\overline{B}) = E$. 
(Indeed, let $h \in E$. 
There exists $f_{n} \in D(B)$ such that $g_{n} \coloneq (\mu - B)f_{n} \to h$ $(n \to \infty)$. 
By \eqref{eq:c2-4.6} it follows that $|f_{n} - f_{m}| \leq R(\lambda,A)|g_{n} - g_{m}|$. 
Thus $(f_{n})$ is a Cauchy sequence. 
Let $f = \lim_{n \to \infty} f_{n}$. 
Then $f \in D(\overline{B})$ and $(\mu - \overline{B})f = h$.) 
Thus $\mu \in \rho(\overline{B})$.

It follows from the hypothesis that there exists $\lambda_{1} \in \rho(\overline{B})$ such that $\lambda_{0} < \lambda_{1}$. 
Since $R(\lambda,A) \leq R(\lambda_{1},A)$ (by B-II, Lemma 1.9)
%%KGK \ref{lem:b2-1.9}
, it follows from \eqref{eq:c2-4.8} that $\|R(\lambda,\overline{B})\| \leq \|R(\lambda,A)\| \leq \|R(\lambda_{1},A)\| \coloneq c$\, ; 
hence \\
$\text{dist}(\lambda,\sigma(\overline{B})) = r(R(\lambda,\overline{B}))^{-1} \geq \|R(\lambda,\overline{B})\|^{-1} \geq 1/c$ 
for all
$\lambda \in \rho(\overline{B}) \cap [\lambda_{1},\infty)$. 
This implies that $\left[\lambda_{1},\infty\right) \subset \rho(\overline{B})$. Moreover, it follows from \eqref{eq:c2-4.8} that
%% --
\begin{equation}\label{eq:c2-4.9}
|R(\lambda,\overline{B})^{n}f| \leq R(\lambda,A)^{n}|f| \quad (f \in E, n \in \N , \lambda_{1} < \lambda).
\end{equation}
%% --
%%--
%%KGK p-C2-26; LNMp-272
%%
Let $w > \omega(A), \,\lambda_{1}$. 
Then it follows from \eqref{eq:c2-4.9} that\\
$\|(\lambda - w)^{n}R(\lambda,\overline{B})^{n}\| \leq \|(\lambda - w)^{n}R(\lambda,A)^{n}\|$ for all $\lambda > w$, $n \in \N $. 
So by the Hille-Yosida theorem, $\overline{B}$ is the generator of a semigroup $(S(t))_{t \geq 0}$. 
Finally, the domination of $(S(t))_{t \geq 0}$ by $(T(t))_{t \geq 0}$ follows from \eqref{eq:c2-4.8} and  Proposition \ref{prop:c2-4.1} .
\end{proof}

\begin{example}\label{ex:c2-4.4} 
\begin{enumerate}[\upshape (i), wide, labelsep=.5em] %%, itemindent=\parindent]
%%
\item \label{ex:c2-4.4-1}
Let $E$ be a $\sigma$-order complete complex Banach lattice and $(T(t))_{t \geq 0}$ be a positive semigroup with generator $A$. 
Let $M \in \ZE$ (the center of $E$ (see C-I, Section 9). 
For example, if $E =$ $L^{p}(X,\mu)$ (where $(X,\mu)$ is a $\sigma$-finite measure space and $1 \leq p \leq \infty$) then $M$ is the multiplication operator defined by a function in $L^{\infty}(X,\mu)$.

Let $B = A + M$. Then $B$ generates a semigroup $(S(t))_{t \geq 0}$.
Assume that $\Re M \leq 0$. Let $f \in D(B)$ and $\phi \in D(A')_{+}$. Then
%% --
\begin{align*}
\Re\langle(\sign  \bar{f})Bf,\phi\rangle &= \Re\langle(\sign  \bar{f})Af,\phi\rangle + \Re\langle(\sign  \bar{f})Mf,\phi\rangle\\
&= \Re\langle(\sign  \bar{f})Af,\phi\rangle + \Re\langle M|f|,\phi\rangle\\
&\leq \langle|f|,A'\phi\rangle.
\end{align*}
%% --
Thus, by Theorem \ref{thm:c2-4.2}  , $S(t))_{t \geq 0}$ is dominated by $(T(t))_{t \geq 0}$.
%% \medskip
%%-
\item \label{ex:c2-4.4-2}
Let $E$ be an order complete complex Banach lattice and $B$ be a regular bounded operator on $E$. 
Then $B$ can be written as $B = B_{0} + M$ where $M \in \ZE$ and $B_{0} \in \LE^{r}$ such that $\inf \{|B_{0}|, \Id\} = 0$.
Let $A = |B_{0}| + \Re  M$. 
Then the semigroup $(\mathrm{e}^{tB})_{t \geq 0}$ is dominated by $(\mathrm{e}^{tA})_{t \geq 0}$.

In fact, let $f \in E$. Then \\
$\Re[(\sign  \bar{f})Bf] = \Re[(\sign \bar{f})B_{0}f] + \Re M \cdot |f| \leq |B_{0}||f| + \Re M \cdot |f| = A|f|$. \\
This implies condition \ref{thm:c2-4.2-2} in Theorem \ref{thm:c2-4.2}.
\end{enumerate}
\end{example}

Domination and positivity are characterized simultaneously as
follows.

\begin{proposition}\label{prop:c2-4.5}
Let $E$ be a $\sigma$-order complete real Banach lattice.
Let $(T(t))_{t \geq 0}$ be a positive semigroup with generator $A$ and let
$(S(t))_{t \geq 0}$ be a semigroup with generator $B$. The following are equivalent.
\begin{enumerate}[\upshape (a)]
%%
\item \label{prop:c2-4.5-1}
$0 \leq S(t) \leq T(t)$ for all $t \geq 0$.
%%-
\item \label{prop:c2-4.5-2}
$\langle P_{(f^{+})}Bf,\phi\rangle \leq \langle f^{+},A'\phi\rangle$ for all $f \in D(B)$, $\phi \in D(A')_{+}$.
%%--
\item \label{prop:c2-4.5-3}
$\langle P_{(f^{+})}Bf,\phi\rangle \leq \langle f^{+},A'\phi\rangle$ for all $f \in D_{0}$, $\phi \in D(A')_{+}$,\\
where $D_{0}$ is a core of $B$.
\end{enumerate}
\end{proposition}
%%
\begin{remark}\label{rem:c2-4.6}
Condition \ref{prop:c2-4.5-2}   implies \eqref{eq:c2-4.4} (cf. Remark \ref{rem:c2-3.12}).
\end{remark}
%%--
%%KGK p-C2-27; LNMp-273
%%
\begin{proof}[Proof of Proposition \ref{prop:c2-4.5}]
One proves as in Theorem \ref{thm:c2-4.2}   that \ref{prop:c2-4.5-1} implies \ref{prop:c2-4.5-2}  .
It is trivial that \ref{prop:c2-4.5-2}   implies \ref{prop:c2-4.5-3}. 
Assume that \ref{prop:c2-4.5-3}   holds. 
Let $\lambda > \lambda_{0} = \max \{s(A),s(B),0\}$. 
In a similar way as \eqref{eq:c2-4.6} one shows that for all $f \in D_{0}$
%% --
\begin{equation}\label{eq:c2-4.10}
\lambda f - Bf = g \text{ implies } f^{+} \leq R(\lambda,A)g^{+}.
\end{equation}
%% --
Since $D_{0}$ is a core of $B$ it follows that \eqref{eq:c2-4.10} also holds for all $f \in D(B)$. 
This implies that $(R(\lambda,B)g)^{+} \leq R(\lambda,A)g^{+}$ for all $g \in E$, $\lambda > \lambda_{0}$. 
Consequently, $0 \leq R(\lambda,B) \leq R(\lambda,A)$ for all $\lambda > \lambda_{0}$. 
Hence \ref{prop:c2-4.5-1}   holds.
\end{proof}

In the following example we apply Theorem \ref{thm:c2-4.3} to Schrödinger operators. 
Here the range condition is proved by an elegant argument due to \citet{kato:1986} with the help of Kato's classical inequality.

\begin{example}\label{ex:c2-4.7}
(Schrödinger operators on $L^{p}$).
Let $E = L^{p}(\R^{n})$, $1 \leq p < \infty$, and $V \in L^{p}_{\text{loc}}(\R^{n})$ such that $\Re V \geq 0$.
Define $B$ on $E$ by $Bf = \Delta f - Vf$ with domain $D(B) = C^{\infty}_{c}(\R^{n})$. 
Then $B$ is closable and $\overline{B}$ is the generator of a semigroup $(S(t))_{t \geq 0}$ which is dominated by the diffusion semigroup (Example \ref{ex:c2-1.5}\ref{ex:c2-1.5-4} and A-I, 2.8). 
If $V \geq 0$, then $(S(t))_{t \geq 0}$ is positive.
\end{example}

\begin{proof}
Denote by $A$ the generator of the diffusion semigroup. 
Then $C^{\infty}_{c} \coloneq C^{\infty}_{c}(\R^{n})$ is a core of $A$ and $Af = \Delta f$ for $f \in C^{\infty}_{c}$ (see Example \ref{ex:c2-1.5}\ref{ex:c2-1.5-4}). 
Let $0 \leq \phi \in D(A')$. Then
%% --
\begin{align*}
\Re\langle(\sign \bar{f})Bf,\phi\rangle &= \Re\langle(\sign \bar{f})Af,\phi\rangle - \langle(\Re V)|f|,\phi\rangle \leq \Re\langle(\sign \bar{f})Af,\phi\rangle\\
&\leq \langle|f|,A'\phi\rangle \quad \text{for all } f \in C^{\infty}_{c} 
\end{align*}
%% --
by Theorem \ref{thm:c2-2.4}  .
Thus \eqref{eq:c2-4.4} holds.
We show that $(\lambda - B)$ has dense range for $\lambda > 0$. 
If not, then there exists $0 \neq \phi \in E' = L^{q}(\R^{n})$ such that $\langle(\lambda - \Delta + V)f,\phi\rangle = 0$ for all $f \in C^{\infty}_{c}$; \ie $(\lambda - \Delta + V)\phi = 0$ in the sense of distributions. 
By Kato's classical inequality (see Example \ref{ex:c2-2.5}) this implies that
\[
%%$
(\lambda - \Delta + \Re V)|\phi| \leq \lambda|\phi| - \Re[(\sign \overline{\phi})(\lambda\phi - \Delta\phi + V\phi)] = 0
%%$
\] 
(here we use that $\Delta\phi = \lambda\phi + V\phi \in L^{1}_{\text{loc}}$). 
Hence $(\lambda - \Delta)|\phi| \leq -(\Re V)|\phi| \leq 0$. 
Since $(\lambda - \Delta)^{-1}$ is a positive linear mapping from $\mathcal{S}(\R^{n})'$ onto $\mathcal{S}(\R^{n})'$, this implies that $\phi = 0$. 
It follows from Theorem \ref{thm:c2-2.4}   that $\overline{B}$ is the
generator of a semigroup $(S(t))_{t \geq 0}$ which is dominated by the semigroup generated by $A$.

If $V = \Re V \geq 0$, we may consider the real space $L^{p}(\R^{n})$. 
Then for every $f \in C^{\infty}_{c}$, $0 \leq \phi \in D(A')$ we have
%%--
%%KGK p-C2-28; LNMp-274
%%
%% --
\begin{align*}
\langle P_{(f^{+})}Bf,\phi\rangle &= \langle P_{(f^{+})}Af,\phi\rangle - \langle Vf^{+},\phi\rangle\\
&\leq \langle P_{(f^{+})}Af,\phi\rangle\\
&\leq \langle f^{+},A'\phi\rangle
\end{align*}
%% --
by \eqref{eq:c2-3.6}. 
It follows from  Proposition \ref{prop:c2-4.5} that $(S(t))_{t \geq 0}$ is positive.
\end{proof}

Finally, if it is known that the semigroup $(S(t))_{t \geq 0}$ is positive,
domination can be characterized as follows.

\begin{proposition}\label{prop:c2-4.8}
Let $E$ be a real Banach lattice, $(T(t))_{t \geq 0}$ a
positive semigroup with generator $A$ and $(S(t))_{t \geq 0}$ a positive
semigroup with generator $B$. Consider the following conditions.
\begin{enumerate}[\upshape (a)]
%%
\item \label{prop:c2-4.8-1}
$S(t) \leq T(t) \quad (t \geq 0)$.
%%..
\item \label{prop:c2-4.8-2}
$\langle Bf,\phi\rangle \leq \langle f,A'\phi\rangle \quad \text{for all } f \in D(B)_{+}, \phi \in D(A')_{+}$.
%%--
\item \label{prop:c2-4.8-3}
$Bf \leq Af \quad \text{for } 0 \leq f \in D(A) \cap D(B)$.
\end{enumerate}
Then \ref{prop:c2-4.8-1} and \ref{prop:c2-4.8-2}  are equivalent and imply \ref{prop:c2-4.8-3}  .
Moreover, if $D(A) \subset D(B)$ or $D(B) \subset D(A)$, then \ref{prop:c2-4.8-3}   implies \ref{prop:c2-4.8-1}  .
%\marginpar{daher: (a)}
\end{proposition}

\begin{proof}
Assume that \ref{prop:c2-4.8-1}   holds. 
Then for $f \in D(B)_{+}$, $\phi \in D(A')_{+}$,
%% --
\begin{align*}
\langle Bf,\phi\rangle &= \lim_{t \to 0} 1/t \langle S(t)f - f, \phi\rangle \leq \lim_{t \to 0} 1/t \langle T(t)f - f, \phi\rangle\\
&= \langle f, A'\phi\rangle.
\end{align*}
%% --
So \ref{prop:c2-4.8-2}   holds. 
\ref{prop:c2-4.8-3}   is proved similarly.

Now assume \ref{prop:c2-4.8-2}  . 
Let $\lambda > \max \{s(A), s(B)\}$. Let $g \in E_{+}$, $\psi \in E'_{+}$.
Then 
%% --
\begin{align*}
\langle R(\lambda,B)g &- R(\lambda,A)g, \psi\rangle\\
&= \langle R(\lambda,A)g, \lambda R(\lambda,B)'\psi - \psi\rangle - \langle \lambda R(\lambda,A)g - g, R(\lambda,B)'\psi\rangle\\
&= \langle f, B'\phi\rangle - \langle Af, \phi\rangle \leq 0,
\end{align*}
%% --
where $f = R(\lambda,A)g \in D(A)_{+}$ and $\phi = R(\lambda,B)'\psi \in D(B')_{+}$. 
Hence $R(\lambda,B) \leq R(\lambda,A)$ and \ref{prop:c2-4.8-1}   follows.

Finally, we prove that \ref{prop:c2-4.8-3}   implies  \ref{prop:c2-4.8-1}   if $D(B) \subset D(A)$, say.
Let $\lambda > \max\{s(A),s(B)\}$. 
Then $(A - B)R(\lambda,B)$ is a positive operator.\\
Hence $R(\lambda,A) - R(\lambda,B) = R(\lambda,A)(A - B)R(\lambda,B) \geq 0$. 
This implies \ref{prop:c2-4.8-1}  .
\end{proof}

The preceding results can be applied to the perturbation by multiplication operators. 
Let $(X,\mu)$ be a $\sigma$-finite measure space and $E = L^{p}(X,\mu)$ $(1 \leq p < \infty)$. 
Consider a positive semigroup $(T(t))_{t \geq 0}$ with generator $A$. 
Let $m \colon X \to \R$ be a measurable function such that $m(x) \leq 0$ for all $x \in X$. 
Let $D(m) = \{f \in E \colon f \cdot m \in E\}$. 
Define the operator $B$ with domain $D(B) = D(A) \cap D(m)$ by $Bf = Af + m \cdot f$ $(f \in D(B))$.
%%--
%%KGK p-C2-29; LNMp-275
%%
\begin{theorem}\label{thm:c2-4.9}
If there exists a quasi-interior subeigenvector $u$ of $A$
such that $u \in D(m)$, then $B$ is closable and the closure $\overline{B}$ of $B$ is the generator of a positive semigroup $(S(t))_{t \geq 0}$ which is dominated by $(T(t))_{t \geq 0}$.
\end{theorem}
%% --
For the proof of the theorem we need the following lemma.
%% --
\begin{lemma}\label{lem:c2-4.10}
Let $A$ and $B$ be generators of positive semigroups
$(T(t))_{t \geq 0}$ and $(S(t))_{t \geq 0}$, respectively.
If $(T(t))_{t \geq 0}$ dominates
$(S(t))_{t \geq 0}$, then $s(B) \leq s(A)$.
\end{lemma}
%% --
\begin{proof}[Proof of Lemma \ref{lem:c2-4.10}]
Let $\lambda > s(A)$.
Then for all $\mu > \max \{\lambda,s(B)\}$ one has $0 \leq R(\mu,A) \leq R(\lambda,A)$ 
%%KGK: \ref{lem:b2-1.9} 
(by B-II, Lemma 1.9), and so $\|R(\mu,B)\| \leq  \|R(\mu,A)\| \leq \|R(\lambda,A)\|$.
Thus $\text{dist}(\mu,\sigma(B)) \geq \|R(\mu,B)\|^{-1} \geq \|R(\lambda,A)\|^{-1}$.
This implies that $[\lambda,\infty) \subset \rho(B)$.
Hence $s(B) \leq \lambda$.
\end{proof}
%% --
\begin{proof}[Proof of Theorem \ref{thm:c2-4.9}]  
There exists $\mu > 0$ such that $Au \leq \mu u$.
Let $\lambda > \max \{s(A),\mu\}$.
Then $\lambda R(\lambda,A)u = \lambda R(\lambda,A)u+u \leq \mu R(\lambda,A)u + u$.
Hence $R(\lambda,A)u \leq c \cdot u$ where $c > 0$.
It follows that $R(\lambda,A)E_{u} \subset E_{u} \cap D(A) \subset D(B)$.
Hence $D(B)$ is dense.
Let $f \in D(B)$, $\phi \in D(A')_{+}$ and set $P_{+} \coloneqq P_{f^{+}}$, $P_{-} \coloneqq P_{f^{-}}$.
Then
%% --
\begin{equation}\label{eq:c2-4.11}
\langle P_{+} Bf,\phi \rangle \leq \langle f^{+},A'\phi \rangle.
\end{equation}
%% --
In fact, 
%% --
\begin{align*}
\langle P_{+} Bf, \phi \rangle &= \langle P_{+} Af, \phi \rangle + \langle P_{+} m \cdot f, \phi \rangle\\
&= \langle P_{+} Af, \phi \rangle + \langle m \cdot f^{+}, \phi \rangle\\
&\leq \langle P_{+} Af, \phi \rangle\\
&\leq \langle f^{+}, A'\phi \rangle \quad \text{(by \eqref{eq:c2-3.6})}.
\end{align*}
%% --
But \eqref{eq:c2-4.11} implies \eqref{eq:c2-4.4}.
So it follows from Theorem \ref{thm:c2-4.3}   that $B$ is closable.
Moreover, if we can show that $(\lambda - \overline{B})D(\overline{B})$ is dense in $E$, it follows that $\overline{B}$ is the generator of a semigroup $(S(t))_{t \geq 0}$.
In that case \eqref{eq:c2-4.11} implies that $(S(t))_{t \geq 0}$ is dominated by $(T(t))_{t \geq 0}$ 
(by Proposition \ref{prop:c2-4.5}).

Now we show that $(\lambda - \overline{B})D(\overline{B})$ is dense in $E$.
Let $m_{n} = \sup \{m, -n1_{X}\}$ $(n \in \N )$ and $B_{n} = A + m_{n}$.
Then $B_{n}$ is the generator of a positive semigroup and it follows from Proposition \ref{prop:c2-4.8}   that $0 \leq R(\lambda,B_{n+1}) \leq R(\lambda,B_{n}) \leq R(\lambda,A)$ for all $n \in \N $, $\lambda > s(A)$.
(Note that $s(B_{n}) \leq s(A)$ by Lemma \ref{lem:c2-4.10}   ).
Let $0 \leq f \in E_{u}$ and $g_{n} = R(\lambda,B_{n})f$.
Then $g = \inf_{n \in \N } g_{n} = \lim_{n \to \infty} g_{n}$ exists.
Moreover $g_{n} \in D(B)$ and $\lim_{n \to \infty} (\lambda - B)g_{n} = f + \lim_{n \to \infty} (B_{n} - B)g_{n} = f$, since
$|(B_{n} - B)g_{n}| \leq (m_{n} - m)|g_{n}| = (m_{n} - m)|R(\lambda,B_{n})f| \leq (m_{n} - m)R(\lambda,A)|f| \leq$
$c' (m_{n} - m)u$ for some positive constant $c'$.
%%--
%%KGK p-C2-30; LNMp-276
%%
But $\lim_{n \to \infty} (m_{n} - m)u = 0$ since $u \in D(m)$.
Thus $g \in D(\overline{B})$ and $(\lambda - \overline{B})g = f$.
We have shown that $E_{u} \subset (\lambda - \overline{B})D(\overline{B})$.
Hence $(\lambda - \overline{B})D(\overline{B})$ is dense in $E$.
\end{proof}
%% --
\begin{example}\label{ex:c2-4.11}
If in the situation explained before Theorem \ref{thm:c2-4.9}  
$D(A) \subset L^{\infty}(X,\mu)$ and $m \in L^{p}(X,\mu)$, then the hypotheses of Theorem \ref{thm:c2-4.9}   are satisfied.
\end{example}
%% --
Now we want to indicate how the results of this section look like for $C_{0}(X)$.
In fact, most of them carry over with a different interpretation of \emph{``sign''} but the same proofs.
We want to state the analogs of Theorem \ref{thm:c2-4.2}   and Theorem \ref{thm:c2-4.3}   explicitly but omit the proof.
Here we use the notation of B-II, Section 2.
%%KGK: ref{sec:b2-2}
%% --
\begin{theorem}\label{thm:c2-4.12}
Let $E = C_{0}(X)$ where $X$ is locally compact.
Let $(T(t))_{t \geq 0}$ be a strongly continuous positive semigroup with
generator $A$ and $(S(t))_{t \geq 0}$ a semigroup with generator $B$.
The following assertions are equivalent.
\begin{enumerate}[\upshape (a)]
%%
\item \label{thm:c2-4.12-1}
$|S(t)f| \leq T(t)|f|$ for all $f \in E$, $t > 0$.
%%--
\item \label{thm:c2-4.12-2}
$\Re\langle(\sign  \bar{f})Bf,\phi\rangle \leq \langle|f|,A'\phi\rangle$ for all $f \in D(B)$, $\phi \in D(A')_{+}$.
\end{enumerate}
\end{theorem}
%% --
Recall that by definition
$\Re  \langle(\sign  \bar{f})Bf,\phi\rangle = \int [\text{(sign } \overline{f(x)})\cdot(Bf)(x)] \, \diff{\phi(x)}$ where $\sign  f(x) =$
$f(x)/|f(x)|$ if $f(x) \neq 0$ and $\sign  0 = 0$.
%% --
\begin{theorem}\label{thm:c2-4.13}
Let $E = C_{0}(X)$ (X locally compact) and let $(T(t))_{t \geq 0}$
be a positive semigroup on $E$ with generator $A$.
Let $B$ be a densely defined operator such that
%% --
\begin{equation}\label{eq:c2-4.12}
\Re  \langle(\sign  \bar{f})Bf,\phi\rangle \leq \langle|f|,A'\phi\rangle \quad 
\text{for all} \quad f \in D(B),\, \phi \in D(A')_{+} .
\end{equation}
%% --
%%for all $f \in D(B)$, $\phi \in D(A')_{+}$.
Then $B$ is closable.
Moreover, if $(\lambda - B)D(B)$ is dense in $E$ for some $\lambda > \max\{0,s(A)\}$, then $\overline{B}$ (the closure of $B$) generates a
semigroup which is dominated by $(T(t))_{t \geq 0}$.
\end{theorem}
%% --
\begin{example}\label{ex:c2-4.14}
Let $E \coloneq C([-1,0],\C)$, $\alpha \in \R$, $\beta \in \C$, $\mu \in M[-1,0]_{+}$
and $\nu \in M[-1,0]$ such that $\mu(\{0\}) = \nu(\{0\}) = 0$.

Then the operator $A$ given by 
%%--
\[
Af = f' \text{ on } D(A) = \{f \in C^{1}([-1,0],\C) \colon f'(0) = \alpha f(0) + \langle f,\mu\rangle\}
\]
%%--
generates a strongly continuous positive semigroup $(T(t))_{t \geq 0}$ 
(see B-II, Example 1.22).
%%KGK: \ref{ex:b2-1.22}
%%--
%%KGK p-C2-31; LNMp-277
%%

Consider the operator $B$ given by
%%--
\[Bf = f' \text{ with domain } D(B) = \{f \in C^{1}([-1,0],\C) \colon f'(0) = \beta f(0) + \langle f,\nu \rangle\}.
\]
%%--
We claim that
%% --
\begin{equation}\label{eq:c2-4.13}
\begin{minipage}{.75\textwidth}
$ B $ is the generator of a strongly continuous semigroup  $(S(t))_{t \geq 0}$.
Moreover,  $(S(t))_{t \geq 0}$ is dominated by  $(T(t))_{t \geq 0}$ 
if and only if $\Re\beta \leq \alpha$ and  $|\nu| \leq \mu$.
\end{minipage}
\end{equation}
\end{example}
%% --
\begin{proof}[Proof of equation \eqref{eq:c2-4.13}]
We first assume that $\alpha \coloneqq \Re  \beta$ and $\mu = |\nu|$.
We show that \eqref{eq:c2-4.12} is satisfied.
Consider the operator $A_{\max}$ on $C[-1,0]$ given by $A_{\max}f = f'$ with domain $D(A_{\max}) = C^{1}[-1,0]$.
We know by B-II, Example 2.12 
%%KGK: \ref{ex:b2-2.12}
that \\ $\Re\langle(\sign  \bar{f})Af,\phi\rangle \leq \Re\langle(\widehat{\sign}  \bar{f})(Af),\phi\rangle = \langle|f|, (A_{\max})'\phi\rangle$ for all $f \in D(A_{\max})$, $0 \leq \phi \in D((A_{\max})')$.
In particular
%% --
\begin{equation}\label{eq:c2-4.14}
\Re\langle(\sign  \bar{f})Bf,\phi\rangle \leq \langle|f|,A'\phi\rangle
\end{equation}
%% --
holds for all $f \in D(B)$, $0 \leq \phi \in D((A_{\max})')$.
It is not difficult to see that $D(A') = D((A_{\max})') + \C\delta_{0}$, and since $D((A_{\max})') = BV[-1,0]$
(see B-II, Example 2.12)
%%KGK: \ref{ex:b2-2.12}
this is an order direct sum.
Thus, in view of \eqref{eq:c2-4.14}, it remains to show that
%% --
\begin{equation}\label{eq:c2-4.15}
\Re\langle(\sign  \bar{f})Bf,\delta_{0}\rangle \leq \langle|f|,A'\delta_{0}\rangle
\end{equation}
%% --
for all $f \in D(B)$.
By the definition of $A$, $\delta_{0} \in D(A')$ and $A'\delta_{0} =  \alpha\delta_{0} + \mu$.
Hence for $f \in D(B)$,
%% --
\begin{align*}
\Re\langle(\sign \bar{f})Bf,\delta_{0}\rangle &= \Re[(\sign \bar{f})f'](0) \\
&= \Re[(\sign  \overline{f(0)})\cdot(\beta f(0) +  \langle f,\nu\rangle)] \\
&\leq \Re\beta \, |f(0)| + |\langle f,\nu\rangle| \\
&\leq \alpha|f(0)| + \langle|f|,\mu\rangle = \langle|f|,A'\delta_{0}\rangle.
\end{align*}
%% --
Thus \eqref{eq:c2-4.15} and so also \eqref{eq:c2-4.12} are proved.

As in the proof in Example B-II, 1.22 one shows that $\lambda - B$ is surjective for large real $\lambda$.
Hence by Theorem \ref{thm:c2-4.13}, $B$ is the generator of a
strongly continuous semigroup $(S(t))_{t \geq 0}$ which is dominated by
$(T(t))_{t \geq 0}$.
This proves the first assertion of \eqref{eq:c2-4.13} and the sufficiency of the second.

Now we assume that the semigroup $(S(t))_{t \geq 0}$ is dominated by $(T(t))_{t \geq 0}$.
We have to show that $\Re\beta \leq \alpha$ and $|\nu| \leq \mu$.
Since $\delta_{0} \in D(A') \cap D(B')$ we have for all $f \in C[-1,0]_{+}$ satisfying $f(0) = 0$,
\begin{align*}
|\langle f,\nu\rangle| = |\langle f,B'\delta_{0}\rangle| &= \lim_{t \to 0+} 1/t \, |\langle S(t)f - f , \delta_{0}\rangle|\\
%%KGK: Eignelicher Newpage
&= \lim_{t \to 0+} 1/t \, |(S(t)f)(0)|  \\
&\leq \lim_{t \to 0+} 1/t \, ((T(t)|f|)(0) \\
&= \lim_{t \to 0+} \langle|f|,1/t(T(t)'\delta_{0} - \delta_{0})\rangle \\
&= \langle|f|,A'\delta_{0}\rangle = \langle|f|,\mu\rangle.
\end{align*}
%%\end{align*}
%%--
%%KGK p-C2-32; LNMp-278
%%
%% --
%%\begin{align*}
%% --
Since $\mu(\{0\}) = \nu\{0\}) = 0$, this implies that $|\nu| \leq \mu$.

Moreover, for arbitrary $f \in C[-1,0]_{+}$ we have
%% --
\begin{align*}
\langle f,\Re\beta \, \delta_{0} + \Re\nu\rangle &= \lim_{t \to 0+} 1/t \, \Re\langle(S(t)f - f),\delta_{0}\rangle\\
&\leq \lim_{t \to 0+} 1/t \, \Re\langle(T(t)f - f),\delta_{0}\rangle \\
&= \langle f,A'\delta_{0}\rangle = \langle f,\alpha\delta_{0} + \mu\rangle.
\end{align*}
%% --
Consequently, $(\Re\beta)\delta_{0} + \Re\nu \leq \alpha\delta_{0} + \mu$.
This implies $\Re\beta \leq \alpha$ since $\mu(\{0\}) = \nu\{0\}) = 0$.
\end{proof}
%%
\begin{remark*}
	It is of interest to find a condition on $B$ which implies that the semigroup $(S(t))_{t \geq 0}$ is stable (see A-IV, Section 1).
	Using the positivity of $(T(t))_{t \geq 0}$ it is shown in B-IV, Example 3.9, that
	$(T(t))_{t \geq 0}$ is stable if and only if $\|\mu\| + \alpha < 0$.
	Since a semigroup which is dominated by a stable semigroup is itself stable we obtain from \eqref{eq:c2-4.13} that $(S(t))_{t \geq 0}$ is stable if $\|\nu\| + \Re\beta < 0$.
\end{remark*}
%%
We conclude this section discussing the following question.
Let $(S(t))_{t \geq 0}$ be a semigroup which is dominated by some positive semigroup.
Does there exist a smallest semigroup $(T(t))_{t \geq 0}$ which dominates $(S(t))_{t \geq 0}$?
More precisely, we look for a positive semigroup $(T(t))_{t \geq 0}$ dominating $(S(t))_{t \geq 0}$ such that $(T(t))_{t \geq 0}$ is dominated
by any other positive semigroup which dominates $(S(t))_{t \geq 0}$.
If such a minimal dominating semigroup exists, it is unique and we call it the
\emph{modulus semigroup} of $(S(t))_{t \geq 0}$.

\begin{example}\label{ex:c2-4.15}
(the modulus semigroup associated with $\Delta - V$).
Let $E$ be the complex space $L^{p}(\R^{n})$ $(1 \leq p < \infty)$ and $V \in L^{p}_{\text{loc}}(\R^{n})$ satisfying $\Re V \geq 0$.
Denote by $B$ the closure of $\Delta - V$ on $C^{\infty}_{c}$
(cf. Example \ref{ex:c2-4.7}) .
The modulus semigroup of the semigroup $(S(t))_{t \geq 0}$ generated by $B$ exists and its generator $A$ is given by $Af = \Delta f - (\Re V)f$ for all $f \in C^{\infty}_{c}$ (and $C^{\infty}_{c}$ is a core of $A$, see Example \ref{ex:c2-4.7}).
\end{example}

\begin{proof}
The operator $A$ defined above generates a positive semigroup (see Example \ref{ex:c2-4.7}).
For $f \in C^{\infty}_{c}$, $\phi \in D(A')_{+}$ one has
%% --
\begin{align*}
\Re\langle(\sign  \bar{f})Bf,\phi\rangle &= \Re\langle(\sign  \bar{f})(\Delta f - Vf),\phi\rangle \\
&=\Re\langle(\sign  \bar{f})\Delta f,\phi\rangle - \langle(\Re V)|f|,\phi\rangle &= \Re\langle(\sign  \bar{f})Af,\phi\rangle \\
&\leq \langle|f|,A'\phi\rangle 
\end{align*}
%% --
by Theorem \ref{thm:c2-2.4}  .
Since $C^{\infty}_{c}$ is a core of $B$, it follows from Theorem \ref{thm:c2-4.3}   that the semigroup generated by $A$ dominates $(S(t))_{t \geq 0}$.
Let $C$ be the generator of a semigroup $(U(t))_{t \geq 0}$ dominating $(S(t))_{t \geq 0}$.
Then
%% --
\begin{align*}
\Re\langle(\sign  \bar{f})Af,\phi\rangle &= \Re\langle(\sign  \bar{f})\Delta f,\phi\rangle - \langle(\Re V)|f|,\phi\rangle \\
&= \Re\langle(\sign  \bar{f})Bf,\phi\rangle\\
&\leq \langle|f|,C'\phi\rangle 
\end{align*}
%% --
for all  $f \in C^{\infty}_{c}$, $\phi \in D(C')_{+}$  by Theorem \ref{thm:c2-4.2}  . 
It follows from Theorem \ref{thm:c2-4.3}   that $(U(t))_{t \geq 0}$ dominates the semigroup generated by $A$.
\end{proof}
%%--
%%KGK p-C2-33; LNMp-279
%%
\begin{example}\label{ex:c2-4.16}
Let $A_{0}$ be the generator of a positive semigroup on an order complete Banach lattice $E$ and $M \in \ZE$.
The semigroup generated by $A_{0} + M$ possesses a modulus semigroup. Its generator is $A_{0} + \Re M$. (This can be proved as the assertion in Example \ref{ex:c2-4.15}  .)
\end{example}

If a semigroup has a bounded regular generator, then it possesses a modulus semigroup. Its generator is bounded too (see C-I, Section 6 for the notion of regular operators).

\begin{theorem}\label{thm:c2-4.17}
Let $B$ be a regular, bounded operator on an order complete complex Banach lattice $E$. The semigroup $(\mathrm{e}^{tB})_{t \geq 0}$ possesses a modulus semigroup. Its generator is $A = |B_{0}| + \Re M$, where $B = B_{0} + M$ is the unique decomposition of $B$ in $\mathcal{L}^{r}(E)$ satisfying $M \in \ZE)$, $B_{0} \in \mathcal{Z}(E)^{d}$.
\end{theorem}

For the proof we need the following result which is of independent interest.

\begin{lemma}\label{lem:c2-4.18}
Let $A$ be the generator of a positive semigroup on a Banach lattice $E$. 
If $Af \geq 0$ for all $f \in D(A)_{+}$, then $A$ is bounded.
\end{lemma}
%%
\begin{proof}[Proof of Lemma \ref{lem:c2-4.18}]
There exists $M \geq 1$ such that $\|R(\lambda,A)\| \leq M/\lambda$ for all $\lambda \geq$ $\omega(A)+1$.
Fix $\mu \geq \omega(A)+1$.
Then $AR(\mu,A)Af = \mu R(\mu,A)Af - Af = \mu^{2}R(\mu,A)f - \mu f - Af$; hence $0 \leq Af \leq \mu^{2}R(\mu,A)f$ whenever $f \in D(A)_{+}$.
Thus $\|Af\| \leq c\|f\|$ for all $f \in D(A)_{+}$ (where $c \coloneq \mu^{2}\|R(\mu,A)\|$).
Consequently,
%% --
\[
\|(\lambda R(\lambda,A) - \Id)f\| = \|AR(\lambda,A)f\| \leq c\|R(\lambda,A)f\| \leq (Mc/\lambda)\|f\| 
\]
for all $f \in E_{+}$ and all $ \lambda \geq \omega(A)+1$. 
Hence
\[
\|(\lambda R(\lambda,A) - \Id)g\| \leq Mc/\lambda(\|g^{+}\| + \|g^{-}\|) \leq (2Mc/\lambda)\|g\| \quad \text{for all} \quad g \in E.
\]
%% --
Thus $\lambda R(\lambda,A)$ is invertible if $\lambda$ is large enough and
$D(A) = \text{im}(\lambda R(\lambda,A)) = E$.
\end{proof}
%%
\begin{proof}[Proof of Theorem \ref{thm:c2-4.17}]
Let $A = |B_{0}| + \Re M$. 
It has been shown in Example \ref{ex:c2-4.4}\ref{ex:c2-4.4-2} that $(\mathrm{e}^{tA})_{t \geq 0}$ dominates $(\mathrm{e}^{tB})_{t \geq 0}$.
Let $(U(t))_{t \geq 0}$ be a positive semigroup dominating $(\mathrm{e}^{tB})_{t \geq 0}$ and $C$ its generator.
We first show that $C$ is bounded.
Let $f \in D(C)_{+}$. Then $\Re(Bf) = \lim_{t \to 0} 1/t(\Re(\mathrm{e}^{tB}f) - f) \leq \lim_{t \to 0} 1/t(U(t)f - f) = Cf$.
Hence $(C + |B|)f \geq (C - \Re B)f \geq 0$ for all $f \in D(C)_{+}$.
By Lemma \ref{lem:c2-4.18} this implies that $C + |B|$ is bounded.
Hence $C$ is bounded as well.
%%--
%%KGK p-C2-34; LNMp-280
%%
Since $C + \|C\| \cdot \Id \geq 0$ by Theorem \ref{thm:c2-1.11}, $C$ is regular.
Let $C = C_{0} + N$ where $C_{0} \in \mathcal{Z}(E)^{d}$ and $N \in \ZE$.
Since $C \geq \Re B$ by what we just proved, it follows that $N \geq \Re M$.
Let $f \in E_{+}$, $\phi \in E'_{+}$ satisfy $\langle f,\phi \rangle = 0$.
Then for all $\alpha \in \R$,
%% --
\begin{align*}
\langle \Re(\mathrm{e}^{i\alpha}B)f,\phi \rangle &= \lim_{t \to 0}1/t \langle \Re(\mathrm{e}^{i\alpha}\mathrm{e}^{tB})f,\phi \rangle \\
&\leq \lim_{t \to 0}1/t \langle \mathrm{e}^{tC}f,\phi \rangle \\
&= \langle Cf,\phi \rangle.
\end{align*}
%% --
Thus $C - \Re(\mathrm{e}^{i\alpha}B)$ satisfies the positive minimum principle (P) 
%%KGK \ref{ex:c2-P}
for all $\alpha \in \R$.
It follows from Theorem \ref{thm:c2-1.11} that $C - \Re(\mathrm{e}^{i\alpha}B) + (\|C\|+\|B\|)\Id \geq 0$ for all $\alpha \in \R$.
Applying the band projection onto $\ZE^{d}$ on both sides of this inequality one obtains that
\[
|B_{0}| = \sup_{\alpha \in \R}\Re(\mathrm{e}^{i\alpha}B) \leq C_{0} 
\]
(since $|T| = \sup_{\alpha \in \R}\Re(\mathrm{e}^{i\alpha}T)$ for all $T \in \mathcal{L}^{r}(E)$, see C-I, Section 7).
We have proved that $\Re M \leq N$ and $|B_{0}| \leq C_{0}$.
This implies that
\[
\Re((\sign  \bar{f})Bf) = \Re((\sign  \bar{f})B_{0}f) + (\Re M)|f| \leq C_{0}|f| + N|f| = C|f|
\]
for all $f \in E$.
It follows from Theorem \ref{thm:c2-4.2}   that $(\mathrm{e}^{tB})_{t \geq 0}$ is dominated by $(\mathrm{e}^{tC})_{t \geq 0}$.
\end{proof}

\begin{remark*}
The proof of Theorem \ref{thm:c2-4.17}   shows that any semigroup dominating a semigroup whose generator is bounded and regular has a bounded generator as well.
\end{remark*}

\begin{example}\label{ex:c2-4.19}
Let $E = \ell^{p}$ $(1 \leq p < \infty)$ or $c_{0}$ and $B \in \mathcal{L}^{r}(E)$ be
given by the matrix $(b_{ij})$.
The generator $A$ of the modulus semigroup of $(\mathrm{e}^{tB})_{t \geq 0}$ is given by the matrix $(a_{ij})$ where $a_{ij} = |b_{ij}|$ when $i \neq j$ and $a_{ii} = \Re  b_{ii}$.
\end{example}

A related question is under which condition a semigroup $(S(t))_{t \geq 0}$ is
dominated by some positive semigroup.
Of course, a necessary condition is that every $S(t)$ is a regular operator.
On an AL-space this condition is automatically satisfied.
But \citet{kipnis:1974} gives an example of a strongly continuous semigroup on $\ell^{1}$ which is not dominated.
On the other hand, it has been independently shown by \citet{kipnis:1974} and
\citet{kubokawa:1975} that every contraction semigroup on an $L^{1}$-space
possesses a modulus semigroup (which is contractive as well).
%%--
%%KGK p-C2-35; LNMp-281
%%
\section{Semigroups of Disjointness Preserving Operators} \label{sec:c2-5}
\index{Semigroups of Disjointness Preserving Operators}
In this section we consider a special case of domination. 
Recall from C-I, Section 6 
%%KGK: \ref{sec;c1.6} 
that a linear operator $S$ on $E$ is called \emph{lattice homomorphism} if
%% --
\begin{equation}\label{eq:c2-5.1}
|Sf| = S|f| \quad \text{for all } f \in E.
\end{equation}
%% --
An operator $S \in \LE$ is called \emph{disjointness preserving} if
%% --
\begin{equation}\label{eq:c2-5.2}
f \perp g \quad \text{implies} \quad Sf \perp Sg \quad \text{for all } f,g \in E.
\end{equation}
%% --
Note that an operator $S$ is a lattice homomorphism if and only if $S$
is positive and disjointness preserving.

In the following we will consider \emph{disjointness preserving semigroups}
(by this we mean semigroups of disjointness preserving operators) and
\emph{lattice semigroups} (\ie semigroups of lattice homomorphisms). For
example, the semigroup $(T_{d}(t))_{t \geq 0}$ defined in Section \ref{sec:c2-3} after Theorem \ref{thm:c2-3.13}  is disjointness preserving for all $d \in \R$ and a lattice semigroup if $d \geq 0$.

\begin{proposition}\label{prop:c2-5.1}
A bounded operator $S$ on a complex Banach lattice $E$ is disjointness preserving if and only if there exists a linear operator $|S|$ on $E$ such that
%% --
\begin{equation}\label{eq:c2-5.3}
|Sf| = |S||f| \quad (f \in E).
\end{equation}
%% --
In that case the operator $|S|$ is uniquely determined by \eqref{eq:c2-5.3}.
$|S|$ is a lattice homomorphism and the \emph{modulus} of $S$ (\ie one has
$|S| \leq T$ for all $T \in \LE$ such that $|Sf| \leq T|f|$ $(f \in E)$).
\end{proposition}

For the proof of the proposition we refer to \citet{arendt:1983} and \citet{depagter:1984}.

\begin{proposition}\label{prop:c2-5.2}
Let $(S(t))_{t \geq 0}$ be a disjointness preserving semigroup. 
Let $T(t) = |S(t)|$ $(t \geq 0)$. 
Then $(T(t))_{t \geq 0}$ is a strongly continuous semigroup.
\end{proposition}

\begin{proof}
Let $0 \leq s,t$ and $f \in E_{+}$. 
Then by \eqref{eq:c2-5.1}, $T(s)T(t)f = T(s)|S(t)f| = |S(s)S(t)f| = |S(s+t)f| = T(s+t)f$. 
Since $\text{span } E_{+} = E$, it follows that $(T(t))_{t \geq 0}$ is a semigroup. 
Moreover, for $f \in E_{+}$, $\lim_{t \to 0} T(t)f = \lim_{t \to 0} |S(t)f| = |f| = f$. 
This implies that $(T(t))_{t \geq 0}$ is strongly continuous.
\end{proof}
%%--
%%KGK p-C2-40; LNMp-282
%%
\begin{example}\label{ex:c2-5.3}
Let $d \in \C$ and $S(t) = T_{d}(t)$ be given by \ref{eq:c2-3.8}  . 
Then $|T_{d}(t)| = T_{|d|}(t)$ $(t \geq 0)$.
\end{example}

\begin{proposition}\label{prop:c2-5.4}
Let $B$ be the generator of a disjointness preserving semigroup $(S(t))_{t \geq 0}$ on a Banach lattice $E$. 
Then $B$ is \emph{local}; i.e.
%% --
\begin{equation}\label{eq:c2-5.4}
f \perp g \quad \text{implies} \quad Bf \perp g \quad \text{for all } f \in D(B), g \in E.
\end{equation}
%% --
\end{proposition}

\begin{proof}
Let $f \in D(B)$ and $g \in E$ such that $\inf\{|f|,|g|\} = 0$. 
Then
%% --
\begin{align*}
|1/t(S(t)f - f)| \wedge |g| &\leq |1/tS(t)f| \wedge |g| + 1/t|f| \wedge |g|\\
&= 1/t |S(t)f| \wedge |g|\\
&\leq 1/t |S(t)f| \wedge |S(t)g -g| + (1/t|S(t)f|) \wedge |S(t)g|\\
&= 1/t |S(t)f| \wedge |S(t)g - g|\\
&\leq |S(t)g - g|.
\end{align*}
%% --
Letting $t \to \infty$ one obtains $|Bf| \wedge |g| = 0$.
\end{proof}

We now describe the relation between the generator of a disjointness preserving semigroup and the generator of the modulus semigroup.

\begin{theorem}\label{thm:c2-5.5}
Assume that E is a complex Banach lattice with order continuous norm. 
Let $(S(t))_{t \geq 0}$ be a semigroup with generator $B$.
The following assertions are equivalent.
\begin{enumerate}[\upshape (a)]
%%
\item \label{thm:c2-5.5-1}
$(S(t))_{t \geq 0}$ is disjointness preserving.
%%--
\item \label{thm:c2-5.5-2}
There exists a semigroup $(T(t))_{t \geq 0}$ with generator $A$ such
that
%% --
\begin{equation}\label{eq:c2-5.5}
f \in D(B) \text{ implies } |f| \in D(A) \text{ and } \Re((\widehat{\sign}  \bar{f})Bf) = A|f|.
\end{equation}
%% --
\end{enumerate}
Moreover, if these equivalent conditions are satisfied, then
$T(t) = |S(t)|$ for all $t \geq 0$.
\end{theorem}

\begin{remark*} ~ 
\begin{enumerate}[i), wide, labelsep=.5em] %%, itemindent=\parindent] \label{enum:c2-kgk.1}
%%
\item  \label{enum:c2-kgk.1-1} %% Nicht im LNM von Chlaude angelegt
By B-II, Lemma 2.9 
%%KGK: \ref{lem:b2-2.9} 
the relation \eqref{eq:c2-5.5} is equivalent to
\[
\langle\Re((\sign  \bar{f})Bf),\phi \rangle = \langle|f|,A'\phi\rangle  (f \in D(B), \phi \in D(A')).
\]
%%
\item \label{enum:c2-kgk.1-2}
It is remarkable that, in contrast with the situation considered in
Theorem \ref{thm:c2-3.8}, here condition \ref{thm:c2-5.5-2}  implies the positivity of $(T(t))_{t \geq 0}$
without further assumptions.
\end{enumerate}
\end{remark*}

The basic idea of the proof of Theorem \ref{thm:c2-5.5}   is to differentiate the equation $|S(t)f| = T(t)|f|$ (where $T(t) = |S(t)|$, cf. \eqref{eq:c2-5.3}). For
%%--
%%KGK p-C2-37; LNMp-283
%%
that we need that the modulus function is differentiable.
If $E = L^{p}(X,\Sigma,\mu)$ $(1 \leq p < \infty)$ this had been proved in Section 2 Example 
\ref{ex:c2-2.3}.
We extend this result to Banach lattices with order continuous norm.

\begin{proposition}\label{prop:c2-5.6}
Let $E$ be a real or complex Banach lattice with order continuous norm. 
Then the modulus function $\Theta \colon E \to E$ (given by $\Theta(h) = |h|$) is right-sided Gateaux differerentiable and
%% --
\begin{equation}\label{eq:c2-5.6}
D_{g}\Theta(f) = \Re((\widehat{\sign}  \bar{f})g) \quad (f, g \in E).
\end{equation}
%% --
\end{proposition}

\begin{proof}
Let $f, g \in E$. Define $k \colon \R \to E$ by $k(t) = |f+tg| - |f|$.
Then $k(0) = 0$ and $k$ is convex (\ie $k(\lambda s + (1-\lambda)t) \leq \lambda k(s) + (1-\lambda)k(t)$ for all $s, t \in \R, \lambda \in [0,1]$).
We show that
%% --
\begin{equation}\label{eq:c2-5.7}
k(s)/s \leq k(t)/t
\end{equation}
%% --
whenever $s < t$, $s,t \neq 0$.

First case: $s < t < 0$.
Choose $\lambda = t/s \in (0,1)$. 
Then $t = (1-\lambda)0 + \lambda s$. 
Consequently, $k(t) \leq (1-\lambda)k(0) + \lambda k(s) = t/s \, k(s)$.

Second case: $s < 0 < t$.
Let $0 < \lambda \coloneqq t/(t-s) < 1$. 
Then $0 = \lambda s + (1-\lambda)t$. 
Hence $0 = k(0) \leq \lambda k(s) + (1-\lambda)k(t) = t/(t-s) \, k(s) - s/(t-s) \, k(t)$, which implies \eqref{eq:c2-5.7}.

Third case: $0 < s < t$.
Let $\lambda = s/t \in (0,1)$. Then $s = (1-\lambda)0 + \lambda t$. Consequently, $k(s) \leq (1-\lambda)k(0) + \lambda k(t) = s/t \, k(t)$, which implies \eqref{eq:c2-5.7}.

It follows from \eqref{eq:c2-5.7} that the net $(k(t)/t)_{t>0}$ is decreasing and bounded below (by $-k(-1)$, for instance). Since $E$ has order continuous norm, it follows that $D_{g}\Theta(f) = \lim_{t \to 0+} k(t)/t$ exists.

It remains to show that $D_{g}\Theta(f) = \Re(\widehat{\sign}  \bar{f})g$.
First of all denote by $P$ the band projection onto $\{f\}^{dd}$. Then it
follows from the definition of $D_{g}\Theta(f)$ that $D_{g}\Theta(f) = PD_{g}\Theta(f) + (\Id-P)D_{g}\Theta(f) = D_{Pg}\Theta(f) + |(\Id-P)g|$. 
Thus it remains to show that
%% --
\begin{equation}\label{eq:c2-5.8}
D_{h}\Theta(f) = \Re((\sign  \bar{f})h) \quad \text{whenever} \quad h \in \{f\}^{dd}.
\end{equation}
%% --

According to the Kakutani-Krein theorem there exists a compact space
$K$ such that $E_{|f|}$ can be identified with $C(K)$. Then by B-II,
Lemma 2.4
%%KGK: \ref{lem:b2-2.4} 
%% --
\begin{equation}\label{eq:c2-5.9}
\lim_{t \to 0+} 1/t(|f+th| - |f|)(x) = \Re(\sign (\overline{f(x)})h(x)) \quad (x \in K).
\end{equation}
%% --
Let $\phi \in E'_{+}$. Then $\phi$ restricted to $E_{|f|}$ can be identified with a regular Borel measure $\mu$ on $C(K)$.
%%--
%%KGK p-C2-38; LNMp-284
%%
So it follows from \eqref{eq:c2-5.9} and the dominated convergence theorem that
%% --
\begin{align*}
\langle D_{h}\Theta(f),\phi \rangle &= \lim_{t \to 0+} 1/t \langle(|f + th| - |f|),\phi \rangle\\
&= \int_{K} \Re(\sign (\bar{f}(x)))h(x)) \, \diff{\mu(x)} \\
&= \langle\Re((\sign  \bar{f})h),\phi\rangle
\end{align*}
%% --
(the last identity holds since by the definition of $\sign  \bar{f} \in \LE$,
we have $(\sign  \bar{f})h \in E_{|f|} = C(K)$ whenever $h \in C(K)$ and
$
%%\[
((\sign  \bar{f})h)(x) = (\sign  \overline{f(x)})h(x)
%%\]
$  
(see C-I, Section 8)\,).\\
Consequently, $D_{h}\Theta(f) = \Re(\sign  \bar{f})h$ whenever $h \in E_{|f|}$. 
Since $D_{h}\Theta(f)$ is continuous in $h$ (in fact, $|D_{h}\Theta(f) - D_{k}\Theta(f)| \leq |h - k|$ for all $h,k \in E$) and $E_{|f|}$ is dense in $\{f\}^{dd}$, it follows that \eqref{eq:c2-5.8} holds for all $h \in \{f\}^{dd}$.
\end{proof}

\begin{remark}\label{rem:c2-5.7}%% ~ 
\begin{enumerate}[a), wide, labelsep=.5em] %%, itemindent=\parindent]
%%
\item \label{rem:c2-5.7-1}
By the same argument as given in the proof one sees
that $\Theta$ is left-sided Gateaux differentiable and
%% --
\begin{equation*}
D^{-}_{g}\Theta(f) = \Re((\sign  \overline{f})g) - P^{d}_{f}|g|
\end{equation*}
%% --
for all $f, g \in E$, where $D^{-}_{g}\Theta(f) = \lim_{t \to 0} 1/t(\Theta(f + tg) - \Theta(f))$ and
$P^{d}_{f}$ denotes the band projection onto $\{f\}^{d}$. 
In particular,
%% --
\begin{equation}\label{eq:c2-5.10}
D^{+}_{g}\Theta(f) = D^{-}_{g}\Theta(f) \quad \text{whenever} \quad g \in \{f\}^{dd}.
\end{equation}
%% --
\item \label{rem:c2-5.7-2}
The proof of  Proposition \ref{prop:c2-5.6}   shows that every convex function $\Theta \colon E \to \R_{+}$
(where $E$ is a Banach lattice with order continuous norm) is right- (and left-) sided Gateaux differentiable [(cf. Arendt (1982)]).
\end{enumerate}
\end{remark}

\begin{proof}[Proof of Theorem \ref{thm:c2-5.5}  ]
Assume that \ref{thm:c2-5.5-1}   holds. 
Let $f \in D(B)$. 
Then $S(t)f$ is differentiable at $t$. 
By the chain rule B-II, Proposition 2.3,
%%KGK: \ref{prop:b2-2.3}
$T(t)|f| = |S(t)f|$ is also differentiable and $\diff{}/\dt|_{t=0} T(t)|f| = \diff{}/\dt|_{t=0} |S(t)f| = \Re(\widehat{\sign}  \bar{f})Bf$ (by  Proposition \ref{prop:c2-5.6} .
Hence $|f| \in D(A)$ and $A|f| = \Re(\widehat{\sign}  \bar{f})Bf$.

Conversely, assume that \ref{thm:c2-5.5-2}   holds. 
Let $s > 0, f \in E$. 
We show that $|S(s)f| = T(s)|f|$. 
This implies that $S(s)$ is disjointness preserving and $|S(s)| = T(s)$ (by Proposition \ref{prop:c2-5.1}  . 
Since $D(B)$ is dense we can assume that $f \in D(B)$. 
Let $\xi(t) = T(s-t)|S(t)f|$ $(t \in [0,s])$.
Since by assumption $|S(t)f| \in D(A)$ one obtains
%% --
\begin{align*}
\diff{}/\dt \, \xi(t) &= -AT(s-t)|S(t)f| + T(s-t)\diff{}/\dr|_{r=t} |S(r)f|\\
&= -AT(s-t)|S(t)f| + T(s-t)(\Re(\widehat{\sign}  \overline{S(t)f})BS(t)f)\\
&\quad \text{(by  Proposition \ref{prop:c2-5.6}   and the chain rule B-II, Proposition 2.3)
%%KGK: ref{prop:b2-2.3}
}\\
&= 0 \quad \text{by the assumption \ref{thm:c2-5.5-2}  }.
\end{align*}
%% --
Hence $\xi(0) = \xi(s)$; \ie $|S(s)f| = T(s)|f|$.
\end{proof}
%%--
The case when $S(t) = T(t)$ $(t \geq 0$ is of special interest: it yields a characterization of generators of lattice semigroups.

Recall that if a semigroup $(T(t))_{t \geq 0}$ is positive, \ie if
%% --
\begin{equation}\label{eq:c2-5.13}
|T(t)f| \leq T(t) |f| \quad (f \in E) ,
\end{equation}
%% --
then its generator $A$ satisfies Kato's inequality. 
We now obtain from Theorem \ref{thm:c2-5.5} the semigroup consists of lattice homomorphisms (\ie the equality holds in \eqref{eq:c2-5.13}) if and only if $A$ satisfies Kato's equality. 
The precise statement is the following.

\begin{corollary}\label{cor:c2-5.8}
%\index{Kato's equality}
%\index{Semigroups!Lattice}
%\index{Generators!Lattice semigroups}
Let $A$ be the generator of a strongly continuous semigroup $(T(t))_{t \geq 0}$ on a Banach lattice $E$ with order continuous norm. 
The following assertions are equivalent.
%% --
\begin{enumerate}[\upshape (a)]
%%
\item \label{cor:c2-5.8-1}
$(T(t))_{t \geq 0}$ is a lattice semigroup.
%%--
\item \label{cor:c2-5.8-2}
$f \in D(A)$ implies $|f| \in D(A)$ and $\Re((\widehat{\sign}  \bar{f})Af) = A|f|$ 
%%--
\item \label{cor:c2-5.8-3}
$f \in D(A)$ implies $|f|,\bar{f} \in D(A)$ and $\Re((\sign  \bar{f})Af) = A|f|$ 
(Kato's equality).
\end{enumerate}
%% --
\end{corollary}

\begin{proof}
%%KGK: Check Claudes output
The equivalence of \ref{cor:c2-5.8-1}   and \ref{cor:c2-5.8-2}   follows directly from Theorem \ref{thm:c2-5.5}  . 
If \ref{cor:c2-5.8-1}   holds, then $A$ is local by  Proposition \ref{prop:c2-5.4}  .
Thus $(\sign  \bar{f})Af = (\widehat{\sign}  \bar{f})Af$ for all $f \in D(A)$ and so \ref{cor:c2-5.8-3}   holds since \ref{cor:c2-5.8-2}   is valid.

Assume now that \ref{cor:c2-5.8-3}   holds. 
Then Kato's equality implies that $Af \in \{f\}^{dd}$ whenever $f \in D(A)_{+}$. 
Since $D(A)$ is a sublattice of $E$ by hypothesis, this implies that $A$ is local, Thus \ref{cor:c2-5.8-2}   follows from \ref{cor:c2-5.8-3}   .
\end{proof}

In the case when $E$ is real this result can be reformulated.

\begin{corollary}\label{cor:c2-5.9}
%\index{Semigroups!Lattice!Real case}
%\index{Generators!Local}
Let $A$ be the generator of a strongly continuous semigroup $(T(t))_{t \geq 0}$ on a real Banach lattice $E$ with order continuous norm. 
The following assertions are equivalent.
%% --
\begin{enumerate}[\upshape (a)]
%%
\item \label{cor:c2-5.9-1}
$(T(t))_{t \geq 0}$ is a lattice semigroup.
%%--
\item \label{cor:c2-5.9-2}
$D(A)$ is a sublattice and $A$ is local.
\end{enumerate}
%% --
\end{corollary}

\begin{proof}
Assume that \ref{cor:c2-5.9-2}   holds. 
Let $f \in D(A)$ and $P_{+} \coloneqq P_{f^{+}}$ and $P_{-} \coloneqq P_{f^{-}}$.
Then $(P_{+})Af^{-} = (P_{-})Af^{+} = 0$ since $A$ is local. 
Hence $(\sign  f)Af = (P_{+} - P_{-})Af = (P_{+} - P_{-})(Af^{+} + Af^{-}) = 
(P_{+})Af^{+} + (P_{-})Af^{-} = Af^{+} + Af^{-} = A|f|$
\end{proof}
%%--
%%KGK p-C2-40; LNMp-286
%%
%% 286 \quad \text{C-II} \quad \text{CHARACTERIZATION}

Thus Kato's equality holds and it follows from Corollary \ref{cor:c2-5.8}   that $(T(t))_{t \geq 0}$ is a lattice semigroup. The other implication follows directly from Corollary  \ref{cor:c2-5.8}  .

\begin{example}\label{ex:c2-5.10}
%\index{Lattice Homomorphisms}
%\index{Semigroups!Lattice Homomorphisms}
%\index{Examples!Disjointness Preserving Semigroups}
Let $E = L^{p}(X,\mu)$ (where $(X,\mu)$ is a $\sigma$-finite measure space and $1 \leq p < \infty$) and let $A_{0}$ be the generator of a semigroup of lattice homomorphisms. 
Let $h \in L^{\infty}$ and $B = A_{0} + h$ (\ie $B$ is given by $Bf = A_{0}f + h \cdot f$ for $f \in D(B) = D(A_{0})$). 
Let $A = A_{0} + \Re \, h$.
Since $A_{0}$ generates a semigroup of lattice homomorphisms, we have $|f| \in D(A_{0})$ whenever $f \in D(A_{0})$ and $\Re((\sign  f)A_{0}f) = A_{0}|f|$.
Hence $\Re((\widehat{\sign}  \bar{f})Bf) = \Re((\widehat{\sign}  \bar{f})A_{0}f) + (\Re \, h) \cdot |f|) = A_{0}|f| + (\Re \, h) \cdot |f| = A|f|$ for all $f \in D(B)$. 
Thus it follows from Theorem \ref{thm:c2-5.5}   that $B$ generates a disjointness preserving semigroup whose modulus semigroup is generated by $A$.
\end{example}

Next we describe when a disjointness preserving semigroup is positive.

\begin{proposition}\label{prop:c2-5.11}
%\index{Disjointness Preserving Semigroups!Positivity}
%\index{Positive Semigroups}
Let $E$ be a complex Banach lattice with order continuous norm and $B$ be the generator of a disjointness preserving semigroup $(S(t))_{t \geq 0}$. 
The semigroup is positive if and only if $B$ is real and $\text{span } D(B)_{+} = D(B)$.
\end{proposition}

\begin{proof}
The conditions are clearly necessary. 
In order to prove sufficiency, we can assume that $E$ is real. 
Denote by $A$ the generator of $(T(t))_{t \geq 0}$, where $T(t) = |S(t)|$. 
Let $f \in D(B)_{+}$. 
Since $B$ is local we have $Bf = P_{f}Bf = (\sign  f)Bf = A|f| = Af$. 
By assumption, $\text{span } D(B)_{+} = D(B)$. 
Thus it follows that $B \subset A$. 
This implies that $B = A$ since $\rho(B) \cap \rho(A) \neq \emptyset$.
\end{proof}

\begin{remark}\label{rem:c2-5.12}
%\index{Kato's Inequality!Reverse}
%\index{Disjointness Preserving Semigroups!Kato's Inequality}
If $B$ is the generator of a disjointness preserving semigroup $(S(t))_{t \geq 0}$ on a real Banach lattice $E$ with order continuous norm then Kato's inequality holds in the reverse sense; \ie
%% -- 
\[
\langle(\sign  f)Bf,\phi \rangle \geq \langle |f|, B'\phi \rangle \text{ for all } f \in D(B), \phi \in D(B')_{+}.
\]
%% -- 
(cf. \eqref{eq:c2-3.9} for a concrete example). 
In fact, let $T(t) = |S(t)|$ and denote by $A$ the generator of $(T(t))_{t \geq 0}$. 
Let $f \in D(B)$, $\phi \in D(B')_{+}$.
Then 
\begin{align*}
\langle(\sign  f)Bf,\phi \rangle &= \langle A|f|,\phi \rangle \\
 &= \lim_{t \to 0} (1/t) \langle T(t)|f| - |f|, \phi \rangle \\
 &\geq \lim_{t \to 0} 1/t \langle S(t)|f| - |f|, \phi \rangle \\
 &= \langle |f|, B'\phi \rangle.
\end{align*}
\end{remark}
%%--
%%KGK p-C2-41; LNMp-287
%%
%% \section*{C-II CHARACTERIZATION 287}

Finally, we come back to Corollary \ref{cor:c2-5.9}  .
If in condition \ref{cor:c2-5.9-2}   we demand that $D(A)$ is not only a sublattice but an ideal of $E$ we obtain a characterization of multiplication semigroups.

Here we call a semigroup $(T(t))_{t \geq 0}$ \emph{multiplication semigroup} if $T(t)$ is a multiplication operator (\ie an element of the center) for every $t > 0$.

\begin{theorem}\label{thm:c2-5.13}
%\index{Multiplication semigroup}
%\index{Semigroups!Multiplication}
%\index{Characterization!Multiplication semigroups}
Let $A$ be the generator of a strongly continuous semigroup $(T(t))_{t \geq 0}$ on a $\sigma$-order complete real or complex Banach lattice $E$.
The following assertions are equivalent.
\begin{enumerate}[\upshape (a)]
%%
\item \label{thm:c2-5.13-1}
$(T(t))_{t \geq 0}$ is a multiplication semigroup.
%%--
\item \label{thm:c2-5.13-2}
There exists $\lambda \in \rho(A)$ such that $R(\lambda,A)$ is a multiplication operator.
%%-- 
\item \label{thm:c2-5.13-3}
$R(\lambda,A)$ is a multiplication operator for all $\lambda \in \rho(A)$
%-- 
\item \label{thm:c2-5.13-4}
$A$ is local and $D(A)$ is an ideal in $E$
%%-- 
\item \label{thm:c2-5.13-5}
If $f \in D(A)$ then $Pf \in D(A)$ for every band projection $P$ on $E$ and $APf = PAf$.
\end{enumerate}
\end{theorem}

\begin{proof}
Assume that \ref{thm:c2-5.13-1}   holds and let $\lambda \in \rho(A)$.
Since $R(\lambda,A)$ is the Laplace transform of the semigroup, it follows that $R(\lambda,A)$ is local since $T(t)$ is local for all $t \geq 0$.
This implies $R(\lambda,A) \in \mathcal{Z}(E)$ (see C-I, Section 9).
%%KGK: \ref{sec:c1.9}).

We show that \ref{thm:c2-5.13-2}   implies \ref{thm:c2-5.13-5}  .
Assume that $\lambda \in \rho(A)$ such that $R(\lambda,A)$ is a multiplication operator.
Let $P$ be a band projection.
Then $PR(\lambda,A) = R(\lambda,A)P$.
Let $f \in D(A)$, $g \coloneqq (\lambda-A)f$.
Then $Pf = PR(\lambda,A)g = R(\lambda,A)Pg$.
Hence $Pf \in D(A)$ and $(\lambda-A)Pf = Pg$.
Thus $APf= \lambda Pf - Pg = P(\lambda f - g) = PAf$.

We show that \ref{thm:c2-5.13-5}   implies \ref{thm:c2-5.13-3}  .
Let $\lambda \in \rho(A)$ and $P$ be a band projection.
We have to show that $PR(\lambda,A) = R(\lambda,A)P$.
Let $g \in E$, $f \coloneqq R(\lambda,A)g$.
Then $Pf \in D(A)$ and $APf = PAf$.
Hence $PR(\lambda,A)g = Pf = R(\lambda,A)P(\lambda-A)f = R(\lambda,A)Pg$.
It follows from C-I, Section 9 
%% \ref{sec:c1-9}
that $R(\lambda,A) \in \ZE$.

 \ref{thm:c2-5.13-3}   implies  \ref{thm:c2-5.13-1}   since $T(t) = \lim_{n \to \infty} [n/t\,R(n/t,A)]^{n}$ strongly for all $t > 0$.

It remains to show the equivalence of  \ref{thm:c2-5.13-4}   and  \ref{thm:c2-5.13-5}  .
Assume that  \ref{thm:c2-5.13-4}   holds, let $f \in D(A)$ and $P$ be a band projection.
Then $Pf \in D(A)$ and $(Id-P)f \in D(A)$ by the assumption.
Since $A$ is local we have 
\[
APf = PAPf + (Id-P)APf = PAPf = PAPf + PA(Id-P)f = PAf.
\]
Conversely, assume  \ref{thm:c2-5.13-5}  .
Let $f \in D(A)$ and $|g| \leq |f|$.
Then there exists a band projection $P$ such that $Pf = g$.
Hence $g \in D(A)$.
We have shown
%%--
%%KGK p-C2-42; LNMp-288
%%
that $D(A)$ is an ideal.
Assume that $\inf\{|h|,|f|\} = 0$.
Denote by $P$ the band projection onto $\{|h|\}^{dd}$.
Then $PAf = APf = A0 = 0$.
Thus $Af \in \{|h|\}^{d}$.
We have proved that $A$ is local.
\end{proof}


\begin{corollary}\label{cor:c2-5.14}
%\index{Multiplication semigroup}
%\index{Semigroups!Multiplication}
%\index{Real generator}
A multiplication semigroup $(T(t))_{t \geq 0}$ on a complex Banach lattice $E$ with order continuous norm is positive if and only if its generator $A$ is real; \ie $f \in D(A)$ implies $\bar{f} \in D(A)$ and $A\bar{f} = \overline{(Af)}$ .
\end{corollary}

\begin{proof}
The condition is equivalent to $T(t)E_{\R} \subset E_{\R}$ $(t \geq 0)$ (cf. Remark \ref{rem:c2-3.1}), so it is clearly necessary.
Conversely, if $A$ is real, then denote by $(T_{\R}(t))_{t \geq 0}$ the restriction semigroup on $E_{\R}$ and by $A_{\R}$ its generator.
Then $A_{\R}$ is local (since $A$ is local) and $D(A_{\R})$ is a sublattice of $E_{\R}$.
Thus $(T_{\R}(t))_{t \geq 0}$ is a lattice semigroup (and so positive) by Corollary \ref{cor:c2-5.9}  .
\end{proof}

The class of bounded operators which generate a lattice semigroup is very restricted.

\begin{proposition}\label{prop:c2-5.15}
%\index{Lattice semigroup}
%\index{Semigroups!Lattice}
%\index{Center!Operator in}
%\index{Disjointness preserving operator}
Let $E$ be a real or complex Banach lattice and $A \in \LE$. 
The following assertions are equivalent.
\begin{enumerate}[\upshape (a)]
%%
\item \label{prop:c2-5.15-1}
$A \in \mathcal{Z}(E)$
%%--
\item \label{prop:c2-5.15-2}
$\mathrm{e}^{tA}$ is disjointness preserving for all $t \geq 0$
%%-- 
\item \label{prop:c2-5.15-3}
$\mathrm{e}^{tA} \in \mathcal{Z}(E)$ for all $t \in \R$ .
\end{enumerate}
Moreover, if $A \in \ZE$ is real, then $\mathrm{e}^{tA} \geq 0$ for all $t \in \R$ .
\end{proposition}

\begin{proof}
Since $\ZE$ is a closed subalgebra of $\LE$ (see C-I, Section 9)
%%KGK: \ref{sec:c2-9}
, it is clear that \ref{prop:c2-5.15-1}   implies \ref{prop:c2-5.15-3} .
Assertion \ref{prop:c2-5.15-2}  follows trivially from \ref{prop:c2-5.15-3}  .
If \ref{prop:c2-5.15-2}   holds, then $A$ is local by  Proposition \ref{prop:c2-5.4}  .
Hence $A \in \mathcal{Z}(E)$

The last assertion follows from the fact that $\mathcal{Z}(E)$ is isomorphic to a space $C(K)$ as a Banach lattice and a Banach algebra.
\end{proof}

\begin{proposition}\label{prop:c2-5.16}
%\index{Group!Strongly continuous}
%\index{Group!Real operators}
%\index{Center!Operator in}
Let $E$ be a complex Banach lattice.
Every strongly continuous group $(T(t))_{t \in \R}$ of real operators contained in $\mathcal{Z}(E)$ has a bounded generator.
\end{proposition}
%%--
%%KGK p-C2-43; LNMp-289
%%
\begin{proof}
Let $(T(t))_{t \geq 0}$ be a strongly continuous multiplication semigroup.
There exist $\omega \in \R$, $M \geq 1$ such that $\|T(t)\| \leq M\mathrm{e}^{\omega|t|}$ $(t \geq 0)$.
Then 
$\|f\|_1 \coloneqq \sup_{t \geq 0}\|\mathrm{e}^{-\omega t}T(t)f\|$ 
defines an equivalent lattice norm on $E$ for which 
$\|T(t)\|_1 \leq \mathrm{e}^{\omega t}$ $(t \geq 0)$.
Since $\mathcal{Z}(E)$ is isometrically isomorphic to a space $C(K)$ (as a Banach lattice), for an operator $S \in \mathcal{Z}(E)$ one has $\|S\| = \inf\{c > 0 \colon |S| \leq c \cdot Id\}$.
Hence the operator norm of $S$ is independent of which lattice norm equivalent to the given one is considered on $E$.
Consequently, $\|T(t)\| = \|T(t)\|_1 \leq \mathrm{e}^{\omega t}$ $(t \geq 0)$.

If $(T(t))_{t \in \R}$ is a strongly continuous group contained in $\mathcal{Z}(E)$ then it follows that $\|T(t)\| \leq \mathrm{e}^{\omega|t|}$ $(t \in \R)$ for some $\omega \geq 0$. 
If in addition the operators $T(t)$ are real one obtains from the above expression for the operator norm that 
%%Claude: $\|T(t) - Id\| = \|Id - T(t)\|$ $(t \geq 0)$.
\[
\mathrm{e}^{- \omega t} \cdot \Id \leq T(t) \leq \mathrm{e}^{\omega t} \cdot \Id \quad (t \geq 0) .
\] 
Consequently, $\lim_{t \to 0}\|T(t) - Id\| = 0$.
\end{proof}

The assumption that the group consists of real operators is essential in Proposition \ref{prop:c2-5.16}  .
In fact, many differential operators on $L^{2}(\R^n)$ generate a strongly continuous group which (via Fourier transformation) is similar to a multiplication group.
A concrete example is the Laplacian (A-I, Example 2.8).
%%KGK: \ref{ex:a1-2.8}

On the other hand, if $E = C(K)$ (K compact), then every strongly continous multiplication semigroup $(T(t))_{t \geq 0}$ has a bounded generator.

[In fact, let $m_t \coloneqq T(t)1$ $(t \neq 0)$.
Then $\lim_{t \downarrow 0}\|T(t) - \Id\| =    \lim_{t \downarrow 0}\|m_t - 1\|_{\infty} = 0$.]

\begin{lemma}\label{lem:c2-5.17}
%\index{Center!Operator in}
%\index{Order continuous norm}
%\index{Banach lattice}
Let $E$ be a real Banach lattice with order continuous norm.
Let $A \in \LE$.
Assume that there exists a dense sublattice $D$ of $E$ such that for all $f \in D$, $g \in E$, $f \perp g$ implies $Af \perp g$.
Then $A \in \mathcal{Z}(E)$.
\end{lemma}

\begin{proof}
Let $0 \leq f \in D$, $\varphi \in E^*$ such that $\langle f,\phi \rangle = 0$.
Since $Af \in (f)^{dd}$ by assumption, it follows that $\langle Af,\phi \rangle = 0$.
Thus $A|_D$ and $-A|_D$ satisfy (P).
It follows from Theorem \ref{thm:c2-1.8}    that $(\mathrm{e}^{tA})_{t \in \R}$ is a positive group.
Thus $A \in \ZE$ by  Proposition \ref{prop:c2-5.15}  .
\end{proof}

Let $A$ be the generator of a positive semigroup and $B \in \LE$. 
The semigroup generated by $A + B$ is positive whenever $(\mathrm{e}^{tB})_{t \geq 0}$ is positive (this follows from \ref{eq:c2-1.8}).
However this condition is not
%%--
%%KGK p-C2-44; LNMp-290
%%
necessary. [For example, let $A \in \LE$ such that $(\mathrm{e}^{tA})_{t \geq 0}$ is positive and let $B = -A$.
Then $A + B$ generates a positive semigroup, but $(\mathrm{e}^{tB})_{t \geq 0}$ is positive only if $A \in \ZE$]
The situation is different when $A$ generates a lattice semigroup.

\begin{theorem}\label{thm:c2-5.18}
%%\index{Lattice semigroup}
%%%\index{Semigroups!Lattice}
%%\index{Positive semigroup}
%%\index{Semigroups!Positive}
Let $E$ be a real Banach lattice with order continuous norm and $A$ be the generator of a lattice semigroup.
Let $B \in \LE$.
The semigroup generated by $A + B$ is positive if and only if $(\mathrm{e}^{tB})_{t \geq 0}$ is positive.
The semigroup generated by $A + B$ is a lattice semigroup if and only if $B \in \ZE$.
\end{theorem}

\begin{proof}
Assume that $A + B$ generates a positive semigroup.
Let $f \in D(A)_{+}$, $\phi \in E_{+}^*$ such that $\langle f,\phi \rangle = 0$.
Since $A$ is local, it follows that $\langle Af,\phi \rangle = 0$.\\
But $\langle (A+B)f,\phi \rangle \geq 0$ by  Proposition  \ref{prop:c2-1.7}  .
Hence $\langle Bf,\phi \rangle \geq 0$.
We have shown that $B|_{D(A)}$ satisfies the positive minimum principle (Definition \ref{def:c2-1.6}).
Since $D(A)$ is a sublattice of $E$ (by Corollary \ref{cor:c2-5.9}  , it follows from Theorem \ref{thm:c2-1.8}   that $(\mathrm{e}^{tB})_{t \geq 0}$ is positive.

By Corollary \ref{cor:c2-5.9} the operator $A + B$ generates a lattice semigroup if and only if $A + B$ is local.
Since $A$ is local, this is equivalent to $B|_{D(A)}$ being local.
By Lemma~\ref{lem:c2-5.17} this is true if and only if $B \in \ZE$.
\end{proof}

%% -- Bitte die Notes so setzen
%% --
\section*{Notes}
\addcontentsline{toc}{section}{Notes}

\begin{enumerate}[label=\emph{Section \arabic*:}, wide]
%%
\item   The notion of dispersiveness is due to \citet{phillips:1962} who uses a semi-scalar product instead of the subdifferential of the canonical half-norm.
Our approach follows \citet{arendtchernoffkato:1982}.
Bounded generators of positive semigroups on a special class of ordered Banach spaces (which includes Banach lattices and C*-algebras) were characterized by the positive minimum principle by \citet{evanshancheolsen:1979}.
The equivalence of \ref{thm:c2-1.11-1} and \ref{thm:c2-1.11-4}  in Theorem \ref{thm:c2-1.11}  is due to \citet{nageluhlig:1981}.
Theorem \ref{thm:c2-1.8}   has been obtained independently by Arendt (1984a) and van Casteren (1984).

%% \smallskip
\item  
The classical distributional Kato's inequality for the Laplacian is due to \citet{kato:1973}.
It is a most elegant tool to prove essential selfadjointness of Schrödinger operators with domain $C^{\infty}_0(\R^n)$ (cf. Example \ref{ex:c2-4.7}).

The relation between Kato's inequality and positivity of $(\mathrm{e}^{t\Delta})_{t \geq 0}$ was pointed out by \citet{simon:1977}.
A criterion for a form negative operator on a space $L^{2}$ to generate a positive semigroup is given by \citet{beurlingdeny:1958}, see also \citet[Vol. IV, Section XIII.12]{reedsimon:1978}.
It was a conjecture of Nagel that some abstract version of
%%--
%%KGK p-C2-45; LNMp-291
%%
Kato's inequality characterizes the positivity of the semigroup (cf., \citet{nageluhlig:1981}). The necessity of Kato's inequality in the form given in Theorem \ref{thm:c2-2.4}   was first proved in \citet[Remark 3.10]{arendt:1982} with a different proof. 
The proof we give here appeared in \citet{arendt:1984a}. 
\citet{miyajimaokazawa:1984} use this inequality to show that a differential operator on which generates a positive semigroup is necessarily of order 2 and has an elliptic principal part. 
This result is generalized to the spaces $L^p(\Omega)$, $\Omega \subset \R^n$ suitable, by \citet{miyajima:1986}.

%%
\item  In this section we closely follow \citet{arendt:1984a}. Theorem \ref{thm:c2-3.8}  , in a similar form but with different proof, has been obtained independently by \citet{schep:1985}.

%%
\item  The characterization of domination by Kato's inequality on a Hilbert space is due to \citet{simon:1977}. Further contributions are due to \citet{hessetal:1977} and \citet{kishimotorobinson:1980}. Theorem \ref{thm:c2-4.3} is due to [Arendt (1984b)]. The result on Schrödinger operators on $L^p(\R^n)$ stated in Example \ref{ex:c2-4.7} is due to \citet{kato:1986}. The case $p = 2$ was proved in \citet{kato:1973}, where the classical Kato's inequality was established. Extensive information on Schrödinger semigroups on $L^p(\R^n)$ is given in \citet{simon:1982}. Other recent results on the $L^p$-theory of Schrödinger operators are obtained by \citet{davies:1986}, \citet{okazawa:1984} and \citet{voigt:1984a}.

The existence of the modulus semigroup of semigroups with bounded, regular generator (Theorem \ref{thm:c2-4.17}) is due to \citet{derndinger:1984} (in the real case).

Proposition \ref{prop:c2-5.15}   had been proved in \citet{schaeferetal:1978} by a completely different method.
 
%%
\item  
The characterization of generators of lattice semigroups on a Banach lattice with order continuous norm (Corollary  \ref{cor:c2-5.8}) is due to \citet{nageluhlig:1981}. An extension of this result to arbitrary Banach lattices is given by \citet{arendt:1982} from which the proof of  Proposition  \ref{prop:c2-5.6}   is taken as well.

Local closed operators having an ideal as domain (\ie operators satisfying condition \ref{thm:c2-5.13-4} of Theorem \ref{thm:c2-5.13}  are investigated in detail by \citet{nakano:1950} who calls them dilatators. \citet{peetre:1959} characterizes differential operators by locality (see also \citet{luxemburg:1979}). In the context of C*-algebras local operators are investigated by \citet{batty:1985} and \citet{battyrobinson:1985}.

\end{enumerate}
%%--

%% -- References
\RaggedRight
\bibliographystyle{abbrvnat}
\bibliography{bib/ln-references}
