% !TEX root = ../LN-Book.tex
%% --
%% -- Stand 2025-06-03  
%% -- Final
%% --
\setcounter{chapter}{2}
\chapternopage{Spectral Theory of Positive Semigroups on \texorpdfstring{$C_{0}(X)$}{C\textunderscore 0(X)}}\label{chap:b3}%
%\index{Spectral Theory on $ C_{0}(X)$}
%% --
{\Large
\vspace*{-.75cm}
by \\[.25em]
Günther Greiner
\vspace{.75cm}
\\
}
%% --
It is known that for a single operator $T \in \L{C_{0}(X)}$ the positivity of $T$ has influence on the spectrum of $T$, mainly on the peripheral spectrum, \ie the part of the spectrum containing all spectral values of maximal absolute value.
This part of the spectrum is of interest because it determines the asymptotic behavior of the iterates $T^{n}$ for large $n \in \N$.
The spectral properties indicated above were first proved by \citet{perron:1907} and \citet{frobenius:1909} for positive square matrices, \ie for positive operators on the Banach lattice $\C^{n}$.
Later these results were extended to the infinite dimensional setting; important contributions are due to Jentzsch, Karlin, Krein, Krasnoselski'i, Lotz, Rota, Rutman, Schaefer and others (see Chapt.V of \citet{schaefer:1974}).

In this chapter we investigate the spectrum $\sigma(A)$ of the generator $A$ of a positive semigroup $\TT = (T(t))_{t \geq 0}$ on the Banach space $C_{0}(X)$.
Throughout this chapter we assume that $C_{0}(X)$ is the space of all \emph{complex-valued} functions on the locally compact space $X$.
In case we restrict to compact spaces we write $K$ instead of $X$.
%% --
\section{The Spectral Bound}\label{sec:b3-1}%
%\index{Spectral Theory on $C_{0}(X)$!Spectral Bound}
%% --
One of the basic results on the spectrum of a positive operator is the fact that its spectral radius is an element of the spectrum (\citet[V.Proposition~4.1]{schaefer:1974}).
We begin the investigation of the spectrum of positive semigroups with the analogous result.
To that purpose we recall that the spectral bound $s(A)$ of a generator $A$ is defined as the least upper bound of the real parts of all spectral values (cf.\ A-III,(1.2)).
%\pagebreak[1]
%
\begin{theorem}\label{thm:b3-1.1}
%%
%\index{Theorem!Spectral Bound}
%%
%\index{Spectral Bound!Properties}
If $A$ is the generator of a positive semigroup $\TT$ on $E = C_{0}(X)$, then $s(A) \in \sigma(A)$ unless 
$s(A) = -\infty$.
In case $X$ is compact, we always have $s(A) > -\infty$.
\end{theorem}
\vfill
%% --
\begin{proof}
	We suppose $\sigma(A) \neq \emptyset$ (\ie $s(A) > -\infty$) and assume $s(A) \notin \sigma(A)$.
	Then there exist $\epsilon > 0$ and $\alpha_{0}$, $\beta_{0} \in \R$ such that
	%% --
	\begin{equation}\label{eq:b3-1.1}
		[s(A)-\epsilon,\infty) \subset \rho(A), \quad \mu_{0} \coloneqq \alpha_{0} + \im\beta_{0} \in \sigma(A) \quad \text{and} \quad \alpha_{0} > s(A)-\epsilon.
	\end{equation}
	%% --
	Now we choose $\lambda_{0} \in \R$ large enough such that
	%% --
	\begin{equation}\label{eq:b3-1.2}
		|\lambda_{0} - \mu_{0}| < \lambda_{0} - (s(A) - \epsilon)
	\end{equation}
	%% --
	and, in addition, such that $\lambda_{0} > \omega_{0}(A)$.
	Then the resolvent $R(\lambda_{0},A)$ is a positive bounded operator, hence its spectral radius $r(R(\lambda_{0},A))$ is a spectral value.
	From A-III, Proposition~2.5 it follows that
	%% --
	\begin{equation}\label{eq:b3-1.3}
		\lambda_{0} - r(R(\lambda_{0},A))^{-1} \in \sigma(A) \quad \text{and} \quad r(R(\lambda_{0},A)) \geq |\lambda_{0} - \mu_{0}|^{-1}
	\end{equation}
	%% --
	This and \eqref{eq:b3-1.2} implies that $\lambda_{0} - r(R(\lambda_{0},A))^{-1}$ is a real spectral value which is greater than $s(A) - \epsilon$.
	We have derived a contradiction to \eqref{eq:b3-1.1} and thus have proved the first statement of the theorem.
	
	To establish the second statement we recall that $\lim_{\lambda \to \infty}\lambda R(\lambda,A)f = f$ for every $f \in E$.
	In particular, for $f = \1_{X}$ we find a (large) $\lambda_{0} \in \R$ such that
	%% --
	\begin{equation}\label{eq:b3-1.4}
		\lambda_{0}R(\lambda_{0},A)\1_{X} \geq 1/2\cdot\1_{X} \quad \text{hence} \quad R(\lambda_{0},A)\1_{X} \geq (2\lambda_{0})^{-1}\cdot\1_{X}
	\end{equation}
	%% --
	We may assume $\lambda_{0} > \omega_{0}(A)$ then $R(\lambda_{0},A) \geq 0$, and iterating \eqref{eq:b3-1.4} we obtain
	%% --
	\begin{equation}\label{eq:b3-1.5}
		R(\lambda_{0},A)^{n}\1_{X} \geq (2\lambda_{0})^{-n}\cdot\1_{X} > 0 \quad \text{for every } n \in \N.
	\end{equation}
	%% --
	It follows that $\|R(\lambda_{0},A)^{n}\| \geq (2\lambda_{0})^{-n}$ and therefore
	%% --
	\begin{equation}\label{eq:b3-1.6}
		r(R(\lambda_{0},A)) = \lim_{n \to \infty}\|R(\lambda_{0},A)^{n}\|^{1/n} \geq (2\lambda_{0})^{-1} > 0.
	\end{equation}
	%% --
	Thus $\sigma(R(\lambda_{0},A))$ contains non-zero spectral values which in view of A-III, Proposition~2.5 is equivalent to $\sigma(A) \neq \emptyset$.
\end{proof}
%% --
The following examples show that the spectrum may be empty in case $X$ is not compact or if the semigroup is not positive.
%% --
\begin{examples}\label{ex:b3-1.2}
%
%\index{Spectral Theory on $ C_{0}(X)$!Empty Spectrum}

\begin{enumerate}[\upshape (i), wide, labelindent=.5em]	
\item  
Consider $X = [0,1)$ and $(T(t))$ on $C_{0}(X)$ given by
%
\begin{equation}\label{eq:b3-1.7}
	(T(t)f)(x) \coloneqq \begin{cases} 
		f(x+t) & \text{if } x+t < 1\,, \\
		0 & \text{if } x+t \geq 1.
	\end{cases}
\end{equation}
%%--
$(T(t))_{t\geq 0}$ is nilpotent (we have $T(t) = 0$ for $t \geq 1$).
It follows that $\sigma(T(t)) = \{0\}$ for all $t > 0$ and by A-III, Theorem~6.2 we have $\sigma(A) = \emptyset$.

\item 
The operator $A$ on $E \coloneqq C_{0}[0,\infty)$ given by
%%--
\begin{equation}\label{eq:b3-1.8}
	(Af)(x) = f'(x) - xf(x), D(A) = \{f \in E \colon f \in C^1, Af \in E\}
\end{equation}
%%--
has empty spectrum.
It is the generator of a positive non-nilpotent semigroup which is given by
%%--
\begin{equation}\label{eq:b3-1.9}
	(T(t)f)(x) = \exp(-(t^2/2) - xt) \cdot f(x+t).
\end{equation}
	
\item 
Taking into account that $C_{0}(\left[0,1\right[))$ as well as 
$C_{0}(\left[0,\infty\right[)$ both are topologically (but not isometrically) isomorphic to $C(\left[0,1\right])$ (see \citet[Section~21.5]{semadeni:1971}), one obtains from (i) and (ii) (non-positive) semigroups on $C(\left[0,1\right])$ whose generators have empty spectrum.
\end{enumerate}
\end{examples}
%% --
The proof of Theorem~\ref{thm:b3-1.1} given above is based on the fact that the spectral radius of a bounded positive operator is an element of the spectrum.
A direct proof not using this fact is given in C-III, Corollary~1.4.
%% --
\begin{corollary}\label{cor:b3-1.3}
	Suppose $\lambda_0 \in \rho(A)$.
	Then $R(\lambda_0,A)$ is a positive operator if and only if $\lambda_0 > s(A)$.
	For $\lambda > s(A)$ we have $r(R(\lambda,A)) = (\lambda - s(A))^{-1}$.
\end{corollary}
%% --
\begin{proof}
	The second statement is an immediate consequence of Theorem~\ref{thm:b3-1.1} and A-III, Proposition~2.5.
	
	Given $\lambda_0 > s(A)$ we choose $\lambda_1 > \max\{\lambda_0,\omega_{0}(A)\}$.
	Since 
	%%--
	\[
	\text{$|\lambda_1 - \lambda_0| < |\lambda_1 - s(A)| = r(R(\lambda_1,A))^{-1}$}
	\]
	%%--
	we have
	\begin{equation}\label{eq:b3-1.10}
		R(\lambda_0,A) = \sum_{n=0}^{\infty} (\lambda_1 - \lambda_0)^n \cdot R(\lambda_1,A)^{n+1}.
	\end{equation}
	%%--
	Since $R(\lambda_1,A)$ is positive, it follows that $R(\lambda_0,A)$ is positive as well.
	On the other hand, assuming that $R(\lambda_0,A)$ is a positive operator, then $\lambda_0$ has to be a real number (note that for $g \geq 0$ we have $f \coloneqq R(\lambda_0,A)g \geq 0$ hence $\lambda_0f - Af = g = \overline{g} = \overline{(\lambda_0 - A)f} = \overline{\lambda_0}f - Af$ ).
	As we have shown above, $R(\lambda,A)$ is positive for $\lambda > \max\{\lambda_0,s(A)\}$ hence an application of the resolvent equation yields
	%%--
	\begin{equation}\label{eq:b3-1.11}
		R(\lambda_0,A) = R(\lambda,A) + (\lambda - \lambda_0)R(\lambda,A)R(\lambda_0,A) \geq R(\lambda,A) \geq 0
	\end{equation}
	%%--
	It follows that for all $\lambda > \max\{\lambda_{0},s(A)\}$ we have
	%% --
	\begin{equation}\label{eq:b3-1.12}
		(\lambda - s(A))^{-1} = r(R(\lambda,A)) \leq \|R(\lambda,A)\| \leq \|R(\lambda_{0},A)\|
	\end{equation}
	%% --
	which can be true only if $\lambda_{0}$ is greater than $s(A)$.
\end{proof}
%% --
\begin{corollary}\label{cor:b3-1.4}
%%
%\index{Corollary!Compact Resolvent}
%%
%\index{Resolvent!Compact Case}
Suppose $X$ is compact and $A$ has compact resolvent.	
	Then there exists a real eigenvalue $\lambda_{0}$ admitting a positive eigenfunction such that 
	$\Re\lambda \leq \lambda_{0}$ for every $\lambda \in \sigma(A)$.
\end{corollary}
%% --
\begin{proof}
	From Theorem~\ref{thm:b3-1.1} we conclude that $\lambda_{0} \coloneqq s(A)$ is a real number, contained in the spectrum and obviously $\Re\lambda \leq \lambda_{0}$ for every $\lambda \in \sigma(A)$.
	Since $A$ has compact resolvent it follows that $\lambda_{0}$ is a pole of the resolvent.
	Let $k$ be its order, then the highest coefficient in the Laurent series is given by
	%% --
	\begin{equation}\label{eq:b3-1.13}
		Q \coloneqq \lim_{\lambda \to s(A)} (\lambda - s(A))^{k}R(\lambda,A).
	\end{equation}
	%% --
It follows from Corollary~\ref{cor:b3-1.3} that $Q$ is a positive operator.
Since $Q \neq 0$, there exists $g \geq 0$ such that $h \coloneqq Qg > 0$.
Moreover, by A-III,3.6., we have 
%
\[
	 (\lambda_{0} - A)h = (\lambda_{0} - A)Qg = 0 .
\]
%
\end{proof}
% --
The example of the rotation semigroup (A-III,Example~5.6) shows that the assumptions in Corollary~\ref{cor:b3-1.4} do not imply that $s(A)$ is dominant.
Additional hypotheses ensuring this stronger property will be given below (see Corollary~\ref{cor:b3-2.11} and \ref{cor:b3-2.12}).

The following result is elementary. 
However, positivity is the crucial point in its proof. 	
Note that it is not just a consequence of the spectral mapping theorem for the point spectrum.
%% --
\begin{proposition}\label{prop:b3-1.5}
%%
%\index{Proposition!Eigenvalue Properties}
%%
%\index{Positive Semigroups!Eigenvalues}
Suppose $A$ is the generator of the positive semigroup $(T(t))_{t \geq 0}$.
Take $\tau > 0$, $r > 0$ and let $\alpha \coloneqq \tau^{-1}\log(r)$.
%% --	
\begin{enumerate}[\upshape (i)]
\item 
If $r$ is an eigenvalue of $T(\tau)$ with positive eigenfunction $h_{0}$, then there is a positive $h \in D(A)$ such that $Ah = \alpha h$ and $\{x \in X \colon h_{0}(x) > 0\} \subset \{x \in X \colon h(x) > 0\}$.
		
\item 
If $r$ is an eigenvalue of $T(\tau)'$ with positive eigenvector $\phi_{0}$, then there is a positive $\phi \in D(A^*)$ such that $A^*\phi = \alpha\phi$ and $\supp\phi_{0} \subset \supp\phi$.
	\end{enumerate}
\end{proposition}
%% --
\begin{proof}
	Without loss of generality we may assume $r = 1$, \ie $\alpha = 0$ and $T(\tau)h_{0} = h_{0}$.
%% --
	\begin{enumerate}[\upshape (i), wide, labelindent=.5em]
    \item 
	Defining
	%% --
	\begin{equation}\label{eq:b3-1.14}
		h \coloneqq \int_{0}^{\tau} T(s)h_{0} \, \ds \,,
	\end{equation}
	%% --
	then for $0 \leq t \leq \tau$ we have
	%% --
	\begin{align*}
		T(t)h &= \int_{0}^{\tau} T(s+t)h_{0} \, \ds  = \int_{t}^{\tau} T(s)h_{0} \, \ds  + \int_{\tau}^{\tau+t} T(s-\tau)T(\tau)h_{0} \, \ds  \\
		&= \int_{t}^{\tau} T(s)h_{0} \, \ds  + \int_{0}^{t} T(s)T(\tau)h_{0} \, \ds  = h
	\end{align*}
	%% --
	It follows that $Ah = \lim_{t \to 0} t^{-1}(T(t)h - h) = 0$.
	So far, positivity was not used. 
    The point is that in general, $h$ may be zero.
	But if $(T(t))$ is positive and $h_{0} \geq 0$, then $s \mapsto (T(s)h_{0})(x)$ is a continuous positive function, hence $0 < h_{0}(x_{0}) = (T(0)h_{0})(x_{0})$ implies $h(x_{0}) = \int_{0}^{\tau} (T(s)h_{0})(x_{0}) \, \ds  > 0$.
	
	\item 
	Defining $\phi \coloneqq \int_{0}^{\tau} T(s)'\phi_{0} \, \ds $, one can proceed as in (i) to obtain the desired result.
	\end{enumerate}
	%% --
\end{proof}
%% --
We use Proposition~\ref{prop:b3-1.5} to prove an analogue of the famous Krein-Rutman result.
For the sake of completeness we include the proof of this classical result, stating that the spectral radius of a positive operator $T$ on $C(K)$ (or more generally on an order unit space) is an eigenvalue of the adjoint $T'$ (see the Corollary of Theorem~2.6 in the appendix of \citet{schaefer:1966}).
%% --
\begin{theorem}\label{thm:b3-1.6}
%%
%\index{Theorem!Krein-Rutman}
%%
%\index{Positive Semigroups!Krein-Rutman}	
	Suppose $K$ is compact and $(T(t))_{t \geq 0}$ is a positive semigroup on $C(K)$ with generator $A$.
	Then there exists a positive probability measure $\phi \in D(A')$ such that $A'\phi = \omega_{0}(A)\phi$.
\end{theorem}
%% --
\begin{proof}
	Consider $T \coloneqq T(1)$, $r \coloneqq r(T) = \mathrm{e}^{\omega_{0}(A)}$.
	In view of Proposition~\ref{prop:b3-1.5} it is enough to show that $r$ is an eigenvalue of $T'$ with a positive eigenvector.
	Given $\lambda \in \C$, $|\lambda| > r$ and $f \in C(K)$ we have
	%% --
	\[
	|R(\lambda,T)f| = \left|\sum_{n=0}^{\infty} \lambda^{-n-1} T^{n}f\right| \leq \sum_{n=0}^{\infty} |\lambda|^{-n-1} T^{n}|f| = R(|\lambda|,T) |f|.
	\]
	%% --
	It follows that $\|R(\lambda,T)\| \leq \|R(|\lambda|,T)\|$ and therefore
	%% --
	\begin{equation}\label{eq:b3-1.15}
		\lim_{\lambda \downarrow r}\|R(\lambda,T)\| = \infty.
	\end{equation}
	%% --
	By the uniform boundedness principle there exist a sequence $(\lambda_{n})$, $\lambda_{n} \downarrow r$ and a positive $\psi \in M(K)$ such that $\|R(\lambda_{n},T)'\psi\| \to \infty$.
	Defining 
    \[
    \psi_{n} \coloneqq \|R(\lambda_{n},T)'\psi\|^{-1}R(\lambda_{n},T)'\psi
    \] we have
%% --
	\begin{equation}\label{eq:b3-1.16}
		\begin{split}
		(r - T')\psi_{n} & = \|R(\lambda_{n},T)'\psi\|^{-1}\cdot((r-\lambda_{n}) + (\lambda_{n} - T'))R(\lambda_{n},T)'\psi\\ 
		& = (r-\lambda_{n})\psi_{n} + \|R(\lambda_{n},T)'\psi\|^{-1}\psi \to 0.
		\end{split}
	\end{equation}
	%% --
	Since $(r - T')$ is $\sigma(M(K),C(K))$-continuous, equation \eqref{eq:b3-1.16} implies that every $\sigma(M(K),C(K))$ cluster point of $(\psi_{n})$ is a positive eigenvector, provided that it is non-zero.
	Because $K$ is compact we have 
%
\[
	\{\phi \in M(K) \colon \phi \geq 0, \|\phi\| = 1\} 
	= \{\phi \in M(K) \colon \phi \geq 0, \langle \phi,1 \rangle = 1\} , 
\]
%% --
which shows that the set of probability measures is $\sigma(M(K),C(K))$-compact.
	Therefore the sequence $(\psi_{n})$ has non-zero cluster points.
\end{proof}
%% --
This theorem implies that for positive semigroups on $C(K)$ the growth and spectral bounds coincide (cf.\ A-III,4.4).
Actually, this is true for locally compact spaces as well and can be proved directly (see B-IV, Theorem~1.4).
Using this result one can prove Theorem~\ref{thm:b3-1.6} by applying the classical Krein-Rutman theorem to any resolvent operator $R(\lambda,A)$ for $\lambda \in \R$ sufficiently large.

The theorem ensures that $A'$ always has eigenvalues, but the generator itself may have no eigenvalue at all.
Multiplication operators have no eigenvalues unless the multiplier is constant on an open subset.
Theorem~\ref{thm:b3-1.6} fails to be true for locally compact spaces as the following example shows.
%% --
\begin{example}\label{ex:b3-1.7}
%%
%\index{Examples!Laplacian Semigroup}
%%
%\index{Semigroups!Laplacian Example}
	Consider $E = C_{0}(\R^{n})$ and the semigroup $(T(t))_{t \geq 0}$ generated by the Laplacian (cf.\ A-I,2.8).
From the explicit representation of $T(t)$,
	%% --
	\begin{equation}\label{eq:b3-1.17}
		(T(t)f)(x) = (4\pi t)^{-n/2}\int_{\R^{n}} \exp(-(x-y)^{2}/4t)\cdot f(y) \, \dy \,,
	\end{equation}
	%% --
	it follows that $\lim_{t \to \infty}T(t)f = 0$ for every $f \in C_{0}(\R^{n})$. Note that $\|T(t)\| = 1$ for all $t \geq 0$ and, for $f$ having compact support,      
    $\|T(t)f\| \leq (4\pi t)^{-n/2}\int_{\R^{n}} |f(y)| \dy \to 0$. 
    	
	If $\phi$ is an eigenvector of $A'$ corresponding to $s(A) = \omega_{0}(A) = 0$, we have $T(t)'\phi = \phi$ for all $t \geq 0$, hence $\langle \phi,f \rangle = \lim_{t \to \infty}\langle T(t)f,\phi \rangle = 0$ for every $f$, \ie $\phi = 0$.
\end{example}
%%% --
\section{The Boundary Spectrum}\label{sec:b3-2}%
%\index{Spectral Theory on $ C_{0}(X)$The Boundary Spectrum}
%% --
In this section we restrict our attention to the \emph{boundary spectrum} $\sigma_{b}(A)$ of a generator $A$, which, by definition, is the intersection of $\sigma(A)$ with the line $\{\lambda \in \C \colon \Re\lambda = s(A)\}$.
Thus $\sigma_{b}(A)$ contains all spectral values of $A$ which have maximal real part.
Note that in general the boundary spectrum is a proper subset of the topological boundary of $\sigma(A)$.
Our aim is to prove results ensuring that $\sigma_{b}(A)$ is a cyclic set (see Definition~2.5).

While most of the results of the preceding section were obtained by transforming the problem to a resolvent operator $R(\lambda,A)$ ($\lambda \in \R$ large enough), this procedure fails here.
The reason is that there is no one-to-one correspondence between the boundary spectrum of $A$ and the peripheral spectrum of $R(\lambda,A)$.
Actually, from Theorem~\ref{thm:b3-1.1} and A-III, Proposition~2.5 it follows that the peripheral spectrum of $R(\lambda,A)$ (\ie the set of spectral values having maximal absolute value) is trivial, since it only contains the spectral radius $r(R(\lambda,A)) = (\lambda - s(A))^{-1}$.

We begin our discussion with two lemmas.
%% --
\begin{lemma}\label{lem:b3-2.1}
%%
%\index{Lemma!Linear Operators}
%%
%\index{Linear Operators!Positivity}
	Suppose $K$, $L$ are compact and $T \colon C(K) \to C(L)$ is a linear operator satisfying $T \1_{K} = \1_{L}$.
	Then we have $T \geq 0$ if and only if $\|T\| \leq 1$.
\end{lemma}
%% --
\begin{proof}
	If $T$ is positive, then
	%% --
	\begin{equation}\label{eq:b3-2.1}
	|Tf| \leq T|f| \leq T(\|f\|\cdot \1_{K}) = \|f\|\cdot T(\1_{K}), \quad f \in C(K)\,,
	\end{equation}
	%% --
	hence $\|T\| = \|T\1_{K}\|$ whenever $T$ is positive.
	This shows that $T \geq 0$ implies $\|T\| \leq 1$ whenever $T\1_{K} = \1_{L}$.
	
	To prove the reverse direction, we first observe that for complex numbers and hence for complex-valued functions the following equivalence holds.
	%% --
	\begin{equation}\label{eq:b3-2.2}
	-1 \leq f \leq 1 \text{ if and only if } \|f - \im\cdot r\cdot 1\| \leq \rho_{r} \coloneqq (1+r^{2})^{1/2} \text{ for every } r \in \R.
	\end{equation}
	%% --
	Now suppose $f \in C(K)$, $0 \leq f \leq 2\cdot \1_{K}$.
	Then we have $-\1_{K} \leq f - \1_{K} \leq \1_{K}$ hence by \eqref{eq:b3-2.2} $\|f - \1_{K} - \im\cdot r\cdot \1_{K}\| \leq \rho_{r}$ for every $r \in \R$.
	From $T\1_{K} = \1_{L}$ and $\|T\| \leq 1$ it follows that $\|Tf - \1_{L} - \im\cdot r\cdot \1_{L}\| \leq \rho_{r}$ for every $r \in \R$.
	Using \eqref{eq:b3-2.2} once again, we obtain $-\1_{L} \leq Tf - \1_{L} \leq \1_{L}$ or $0 \leq Tf \leq 2\cdot \1_{L}$.
\end{proof}
%% --
Before we can formulate the second lemma we have to fix some notation.
	\begin{definition}\label{def:b3-2.2}
%	%
%\index{Definition!Sign Operator}
%	%
%\index{Operator!Sign}
%% -> $ \sign h $
\begin{enumerate}[\upshape (i)]
		
\item
Given $h \in C_{0}(X)$ such that $h(x) \neq 0$ for all $x \in X$ then the operator $S_{h}$ is defined to be the multiplication operator with $ \sign h $, \ie
		%% --
		\begin{equation}\label{eq:b3-2.3}
			S_{h}f = h|h|^{-1}f \quad (f \in C_{0}(X)).
		\end{equation}
		%% --
		
\item
For $f \in C_{0}(X)$, $n \in \Z$ we define $f^{[n]} \in C_{0}(X)$ by
%% --
\begin{equation}\label{eq:b3-2.4}
	f^{[n]}(x) = 
	\begin{cases}
			(f(x)/|f(x)|)^{n-1}\cdot f(x) & \text{if } f(x) \neq 0\,,\\			
			0 & \text{if } f(x) = 0.
			\end{cases}
\end{equation}
\end{enumerate}
%% --
\end{definition}
%% --	
The following assertions are immediate consequences of the definition. 
They will be used frequently in the following.
%% --
\begin{equation}\label{eq:b3-2.5}
	\begin{split}
		&S_{h} \text{  is a linear isometry satisfying  } |S_{h}f| = |f|, \\ 
		&\text{its inverse being  } S_{\overline{h}} \text{  where  } \overline{h} \text{ is the complex conjugate of $ h $,}
	\end{split}
\end{equation}
\begin{equation}\label{eq:b3-2.6}
	    f^{[1]} = f, f^{[0]} = |f|, f^{[-1]} = \overline{f}, |f^{[n]}| = |f| \text{  for every $n \in \Z$, }
\end{equation}
\begin{equation}\label{eq:b3-2.7}
	\text{If  } h(x) \neq 0 \text{  for all  } x \in X, \text{  then  } h^{[n]} = S_{h}^{n}|h| = S_{h}^{n-1}h
	\text{  for every $ n \in \Z $.} 
\end{equation}
%% --
%% --
%\begin{align*}\label{eq:b3-2.6}
%	\begin{split}
%		&S_{h} \text{  is a linear isometry satisfying  } |S_{h}f| = |f|, \\ 
%		&\text{its inverse being  } S_{\overline{h}} \text{  where  } \overline{h} \text{ is the complex conjugate of $ h $,}
%	\end{split}\\
%	&f^{[1]} = f, f^{[0]} = |f|, f^{[-1]} = \overline{f}, |f^{[n]}| = |f| \text{  for every $n \in \Z$, }\\ 
%	&\text{If  } h(x) \neq 0 \text{  for all  } x \in X, \text{  then  } h^{[n]} = S_{h}^{n}|h| = S_{h}^{n-1}h
%	\text{  for every $ n \in \Z $.} 
%\end{align*}
%%%%%%%%%%%%%%%%%%%%%%%%%%%%%%%%%%%%%%%%%%%%%%%%%%%%%%%%%
%\begin{equation}\label{eq:b3-2.5}
%	\begin{aligned}
%	&S_{h} \text{  is a linear isometry satisfying  } |S_{h}f| = |f|, \text{ \phantom{aaaaaaaaaaaaaa}}\ \\ 
%	&\text{its inverse being  } S_{\overline{h}} \text{  where  } \overline{h} \text{ is the complex conjugate of $ h $,}
%	\end{aligned}
%\end{equation}
%%% --
%\begin{align}\label{eq:b3-2.6}
%	&f^{[1]} = f, f^{[0]} = |f|, f^{[-1]} = \overline{f}, |f^{[n]}| = |f| \text{  for every $n \in \Z$, }\\ 
%	&\text{If  } h(x) \neq 0 \text{  for all  } x \in X, \text{  then  } h^{[n]} = S_{h}^{n}|h| = S_{h}^{n-1}h
%	\text{  for every $ n \in \Z $.} 
%\end{align}
	%% --
\begin{lemma}\label{lem:b3-2.3}
%%
%\index{Lemma!Similar Operators}
%%
%\index{Operators!Similarity}
Let $T$ and $R$ be bounded linear operators on $C_{0}(X)$ and assume that $h \in C_{0}(X)$ has no zeros. 
Suppose we have
%% --
\begin{equation}\label{eq:b3-2.8}
	Rh = h, T|h| = |h| \text{ and } |Rf| \leq T|f| \text{ for every $f \in C_{0}(X)$.} 
\end{equation}
%% --
Then $R$ and $T$ are similar, more precisely, $T = S_{h}^{-1}RS_{h}$.
In particular, the spectra (and point spectra, resp.) of $T$ and $R$ coincide.
\end{lemma}
%% --
\begin{proof}
	We first note that the assertion $|Rf| \leq T|f|$ ($f \in E$) implies that $T$ is a positive operator.
	Therefore the assumption $T|h| = |h|$ implies that the principal ideal $E_{h} = \{f \in C_{0}(X) \colon |f| \leq n|h| \text{ for some } n \in \N\}$ is an invariant subspace for $T$ and for $R$ as well.
	$E_{h}$ is isomorphic to $C^{b}(X) \cong C(\beta X)$ ($\beta X$ denotes the Stone-Čech compactification of $X$), an isomorphism is given by $f \mapsto f|h|$.
	Considering the restrictions $T_{|E_{h}}$ and $R_{|E_{h}}$ as operators on $C(\beta X)$ and denoting them $\tilde{T}$ and $\tilde{R}$ respectively, we have
	%% --
\begin{equation}\label{eq:b3-2.9}
		\tilde{R}\tilde{h} = \tilde{h}, \quad \tilde{T}\1 = \1, \quad \tilde{T} \geq 0, \quad |\tilde{R}f| \leq \tilde{T}|f| \text{ for all } f.
	\end{equation}
	%% --
	Here $\tilde{h}$ denotes the continuous extension of $h/|h|$ to $\beta X$.
	
	Defining $T_{1} \coloneqq M_{\tilde{h}}^{-1}\tilde{R}M_{\tilde{h}}$ we have by \eqref{eq:b3-2.9}
	%% --
	\begin{equation}\label{eq:b3-2.10}
	T_{1}\1 = M_{\tilde{h}}^{-1}\tilde{R}\tilde{h} = \1 \text{ and }
	\end{equation}
	%% --
	%% --
	\begin{equation}\label{eq:b3-2.11}
	|T_{1}f| = |M_{\tilde{h}}^{-1}\tilde{R}M_{\tilde{h}}f| = |\tilde{R}M_{\tilde{h}}f| \leq \tilde{T}|M_{\tilde{h}}f| = \tilde{T}|f| \text{ for all } f.
	\end{equation}
	%% --
	Hence we have $\|T_{1}\| \leq \|\tilde{T}\| = 1$ (by \eqref{eq:b3-2.11}, \eqref{eq:b3-2.9}, \eqref{eq:b3-2.1}.
	Then it follows from Lemma~\ref{lem:b3-2.1} that $T_{1}$ is a positive operator.
	Thus \eqref{eq:b3-2.11} implies that $0 \leq T_{1} \leq \tilde{T}$ and therefore $\|\tilde{T} - T_{1}\| = \|(\tilde{T} - T_{1})\1\| = 0$ (by \eqref{eq:b3-2.10}, \eqref{eq:b3-2.9}, \eqref{eq:b3-2.1}.
\end{proof}
%% --
We are now able to prove a result which in some sense is the key to cyclicity results for the spectrum.
	These general results will be proved by reducing the problem in such a way that the following theorem can be applied.
%% --
\begin{theorem}\label{thm:b3-2.4}
%%
%\index{Theorem!Spectral Cyclicity}
%%
%\index{Cyclicity!Spectral Theorem}
\begin{enumerate}[\upshape (i)]	
\item
Let $T$ be a positive linear operator on $C_{0}(X)$, let $h \in C_{0}(X)$ and $\lambda \in \C$, $|\lambda| = 1$.
	If $Th = \lambda h$ and $T|h| = |h|$, then we have $Th^{[n]} = \lambda^{n}h^{[n]}$ for every $n \in \Z$ (cf.\ \eqref{eq:b3-2.4}).
	If $h$ does not have zeros in $X$, then 
	%% --
	\[
	\lambda T = S_{h}^{-1}TS_{h}.
	\]
	%% --
\item
Suppose $A$ is the generator of a positive semigroup, $h \in C_{0}(X)$, $\alpha,\beta \in \R$ such that $Ah = (\alpha+\im\beta )h$ and $A|h| = \alpha|h|$. 
	Then we have $Ah^{[n]} = (\alpha+\im n\beta)h^{[n]}$ for every $n \in \Z$.
	If $h$ does not have zeros, then 
	%% --
	\[
	S_{h}D(A) = D(A) \text{ and \ } \im\beta + A = S_{h}^{-1}AS_{h}.
	\]
	%% --
\end{enumerate}
\end{theorem}
%% --
\begin{proof}
	(i) The closed principal ideal $\overline{E_{h}}$, which is canonically isomorphic to $C_{0}(X_{1})$ with $X_{1} = \{x \in X \colon h(x) \neq 0\}$, is $T$-invariant.
	We give an object a tilde when we consider it as an element of $\overline{E_{h}} \cong C_{0}(X_{1})$.
	Defining $\tilde{R} \coloneqq \lambda\tilde{T}$, then $\tilde{T}$, $\tilde{R}$, $\tilde{h}$ satisfy \eqref{eq:b3-2.8}, hence we have
	%% --
	\begin{equation}\label{eq:b3-2.12}
		\tilde{T} = S_{\tilde{h}}^{-1} \circ \tilde{R} \circ S_{\tilde{h}} = \bar{\lambda} \cdot S_{\tilde{h}}^{-1} \circ \tilde{T} \circ S_{\tilde{h}}
	\end{equation}
	%% --
	which by iteration yields
	%% --
	\begin{equation}\label{eq:b3-2.13}
		\tilde{T} = \bar{\lambda}^{n} \cdot S_{\tilde{h}}^{n} \circ \tilde{T} \circ S_{\tilde{h}}^{n} \text{ for all } n \in \Z.
	\end{equation}
	%% --
	It follows that
	%% --
		\begin{equation*}
		\tilde{T}\tilde{h}^{[n]} = \tilde{T} \circ S_{\tilde{h}}^{n}|\tilde{h}| = \lambda^{n} \cdot S_{\tilde{h}}^{n} \circ \tilde{T}|\tilde{h}| 
		= \lambda^{n} \cdot S_{\tilde{h}}^{n}\tilde{h} = \lambda^{n} \cdot \tilde{h}^{[n]}
	\end{equation*}
	%% --
	(see \eqref{eq:b3-2.7} and \eqref{eq:b3-2.12}, which is precisely $Th^{[n]} = \lambda^{n}h^{[n]}$ for all $n \in \Z$.
	
	If $h$ does not have zeros, then $\overline{E_{h}} = E$, hence $T = \tilde{T}$, $h = \tilde{h}$ and the remaining assertion follows from \eqref{eq:b3-2.12}.
%% --

	(ii) This can be deduced easily from (i) as follows. 
	If $Ah = (\alpha+\im\beta )h$, $A|h| = \alpha|h|$, then we have by A-III,Corollary~6.4
	%% --
	\[
	\mathrm{e}^{-\alpha t}T(t)h = \mathrm{e}^{\im\beta  t}h \text{ and } \mathrm{e}^{-\alpha t}T(t)|h| = |h| \text{ for every } t \geq 0.
	\]
	%% --
    %Hence by (a) $\mathrm{e}^{-\alpha t}T(t)h^{[n]} = \mathrm{e}^{\im n\beta t}h^{[n]}$ ($t \geq 0$, $n \in \Z$) which is equivalent to $Ah^{[n]} = (\alpha+\im n\beta)h^{[n]}$.
	By (a) for $t \geq 0$ and $n \in \Z$ we have
    $\mathrm{e}^{-\alpha t}T(t)h^{[n]} = \mathrm{e}^{\im n\beta t}h^{[n]}$, which is equivalent to $Ah^{[n]} = (\alpha+\im n\beta)h^{[n]}$.
	If $h$ does not have zeros, then $\mathrm{e}^{-\alpha t}T(t) = \mathrm{e}^{-\alpha t}\mathrm{e}^{\im\beta  t}S_{h}^{-1}T(t)S_{h}$ for every $t \geq 0$ which is equivalent to the final statement of (ii).
	\end{proof}
%% --	
Before we state a first cyclicity result we give the definition and illustrate it by some examples.
%% --
\begin{definition}\label{def:b3-2.5}
%	%
%\index{Definition!Cyclic Sets}
%	%
%\index{Sets!Imaginary Additively Cyclic}
A subset $M \subset \C$ is called \emph{imaginary additively cyclic} (or simply \emph{cyclic}), if it satisfies the following condition. 
\begin{quote}
$\alpha + \im\beta \in M$, $\alpha,\beta \in \R$ implies that $\alpha +  \im k\beta \in M$ for every $k \in \Z$\,
\end{quote}
\end{definition}
%% --
Every subset of $\R$ is cyclic. 
On the other hand, if $M$ is cyclic and $M \not\subset \R$, then $M$ has to be unbounded.
	
For a subset $M$ of $\im \R$ we give the following equivalent conditions.
	\begin{enumerate}[\upshape (a)]
		\item 
		$M$ is imaginary additively cyclic,
		
		\item 
		$M$ is the union of (additive) subgroups of $\im \R$\,,
		
		\item 
		$M = \cup_{\alpha \in S} \im\alpha\Z$ for some set $S \subset \R$.
	\end{enumerate}
	
	Here are some concrete cyclic subsets of $\im \R$.
	%% --
	\begin{align*}
		M_{1} &= \{0\}, \quad M_{2} = \im\R, \quad M_{3} = \im\alpha\Z \quad (\alpha > 0), \\
		M_{4} &= \im\alpha\Z + \im\beta\Z = \{\im n\alpha +  \im\beta \colon n,m \in \Z\} \quad (\alpha, \beta \in \R), \\
		M_{5} &= \{0\} \cup \{\im \lambda \colon \lambda \in \R, |\lambda| \geq 1\}, \\
		M_{6} &= \cup_{n=0}^{\infty}\{\im \lambda \colon \lambda \in \R, n\alpha \leq |\lambda| \leq n\beta\} \quad (0 < \alpha \leq \beta, \alpha,\beta \in \R)
	\end{align*}
	%% --
	In the following we consider the boundary spectrum of several semigroups. 
    The notation $M_{k}$ refers to the sets just defined.
%% --
\begin{examples}\label{ex:b3-2.6}%
%\index{Spectral Theory on $ C_{0}(X)$!Examples!Boundary Spectrum}

\begin{enumerate}[\upshape (i), wide, labelindent=.5em]
\item
For the Laplacian $\Delta$ on $\R^{n}$ or the second derivative on $[0,1]$ with Neumann boundary conditions the boundary spectrum is $M_{1}$.
	
\item
The first derivative on $\R$ or $\R_{+}$ is an example where the boundary spectrum of the generator is $M_{2}$.
	
\item
The rotation semigroup on $C(\Gamma)$ (see A-III,Example~5.6) with period $2\pi/\alpha$ has boundary spectrum $M_{3}$.

\item
For the semigroup on $C(\Gamma \times \Gamma)$ given by
	%% --
	\[
	(T(t)f)(z,w) = f(z\cdot \mathrm{e}^{\im\alpha t},w\cdot \mathrm{e}^{\im\beta  t}) \quad (f \in C(\Gamma \times \Gamma), (z,w) \in \Gamma \times \Gamma)
	\]
	%% --
	we have $P\sigma(A) = M_{4}$.
	If $\alpha/\beta$ is irrational, then this is a dense subset of $\im \R$ and $\sigma_{b}(A) = \sigma(A) = \im\R$.
	
\item
Consider $D \coloneqq \{z \in \C \colon |z| \leq 1\} = \{r\cdot \mathrm{e}^{\im\omega} \colon r \in [0,1], \omega \in \R\}$, and a strictly positive function $\kappa \in C[0,1]$.
	The flow on $D$ governed by the differential equation $\dot{r} = 0$, $\dot{\omega} = \kappa(r)$ induces a strongly continuous semigroup on $C(D)$ (which is given by
	%% --
	\[
	(T(t)f)(z) = f(z\cdot \mathrm{e}^{\im\kappa(|z|)t}).
	\]
	%% --
	The boundary spectrum is $M_{6}$ with $\alpha \coloneqq \inf_r \kappa(r)$, $\beta \coloneqq \sup_r \kappa(r)$.
	In particular, for $\kappa(r) = 1 + r$ we obtain as boundary spectrum the set $M_{5}$.
	
\item
Suppose $M$ is a closed cyclic subset of $\im \R$, $M = \cup_{\alpha \in S} \im\alpha\Z$ for a suitable $S \subset \R$ (\eg $S = M$).
	
	The space $E_{1} \coloneqq \{(f_{\alpha})_{\alpha \in S} \colon f_{\alpha} \in C(\Gamma), \sup_\alpha \|f_{\alpha}\| < \infty\}$ is a Banach space under the norm $\|(f_{\alpha})\| \coloneqq \sup_\alpha \|f_{\alpha}\|$.
	The closure of the linear subspace $E_{0} \coloneqq \{(f_{\alpha}) \in E_{1} \colon f_{\alpha} \neq 0 \text{ only for finitely many } \alpha \in S\}$ is isomorphic to $C_{0}(X)$ where $X$ is the topological sum of $|S|$ copies of $\Gamma$.
	
	Let $(T_{\alpha}(t))_{t \geq 0}$ denote the rotation semigroup on $C(\Gamma)$ with period $2\pi/\alpha$, then we define a semigroup $(T(t))_{t \geq 0}$ on $E \coloneqq C_{0}(X)$ as
	%% --
	\[
	(T(t)(f_{\alpha})) \coloneqq (T_{\alpha}(t)f_{\alpha}) \quad ((f_{\alpha})_{\alpha \in S} \in E).
	\]
	%% --
	This is a positive semigroup on $E = C_{0}(X)$ whose boundary spectrum is precisely the given closed cyclic set $M$.
	We leave the verification as an excercise.
	
\end{enumerate}
\end{examples}
%% --
Our first result concerns cyclicity of the eigenvalues contained in the boundary spectrum, \ie of the set
%% --
	\[
	P\sigma_{b}(A) \coloneqq P\sigma(A) \cap \sigma_{b}(A) = \{\lambda \in P\sigma(A) \colon \Re\lambda  = s(A)\}
	\]
	%% --
	It is an almost straightforward consequence of Theorem~\ref{thm:b3-2.4}.
	%% --
	\begin{proposition}\label{prop:b3-2.7}
%	%
%\index{Proposition!Cyclicity of Eigenvalues}
	%
%\index{Eigenvalues!Cyclicity}
%	%
%\index{Boundary Spectrum!Eigenvalues}
		
		Assume that for some $t_{0} > 0$ there is a strictly positive measure $\phi$ such that $T(t_{0})'\phi = \exp(s(A)t_{0})\cdot\phi$.
		
		Then $P\sigma_{b}(A)$ is imaginary additively cyclic.
	\end{proposition}
	%% --	
\begin{proof}
If $P\sigma_{b}(A)$ is empty, there is nothing to prove.
Otherwise we have $s(A) > -\infty$.
In view of the rescaling procedure we may assume $s(A) = 0$.
By Proposition~\ref{prop:b3-1.5}(ii) there exists $\phi \gg 0$ such that $T(t)'\phi = \phi$ for all $t \geq 0$.
Given $\im \alpha \in P\sigma_{b}(A)$ then there is $h \in C_{0}(X)$, $h \neq 0$, such that $Ah = \im\alpha h$ or $T(t)h = \mathrm{e}^{\im\alpha t}h$ for all $t$ (A-III, Corollary~6.4).

	Then we have
	%% --
	\begin{equation}\label{eq:b3-2.14}
		|h| = |\mathrm{e}^{\im\alpha t}h| = |T(t)h| \leq T(t)|h| \text{ or } T(t)|h| - |h| \geq 0
	\end{equation}
	%% --
	%% --
	\begin{equation}\label{eq:b3-2.15}
		\langle T(t)|h| - |h|,\phi \rangle = \langle |h|,T(t)'\phi \rangle - \langle |h|,\phi \rangle = 0.
	\end{equation}
	%% --
	Since $\phi$ is strictly positive, \eqref{eq:b3-2.14}  and \eqref{eq:b3-2.15} imply that $T(t)|h| = |h|$ for $t \geq 0$ or equivalently $A|h| = 0$.
	
	Now Theorem~\ref{thm:b3-2.4} implies that $Ah^{[n]} = \im n\alpha h^{[n]}$ $(n \in \Z)$.
\end{proof}
%% --
Concerning the hypothesis $T(t_{0})'\phi = \exp(s(A)t_{0})\cdot\phi \gg 0$ we recall that in case $X$ is compact there are always positive linear forms such that $T(t)'\phi = \mathrm{e}^{s(A)t}\phi$ (see Theorem~\ref{thm:b3-1.6}).
If the semigroup is irreducible, then one also has $\phi \gg 0$ (see Section~3 below).

In a second result we consider semigroups having compact resolvent.
An important step of the proof is isolated as a lemma.
Before stating it, we recall that given a closed ideal $I \subset C_{0}(X)$, then $I$ as well as $C_{0}(X)/I$ are spaces of continuous functions on a locally compact space vanishing at infinity.
More precisely, if $I = \{f \in C_{0}(X) \colon f|_{M} = 0\}$ for a suitable closed subset $M \subset X$, then $I \cong C_{0}(X\backslash M)$ and $C_{0}(X)/I \cong C_{0}(M)$ (cf.\ B-I).
It follows that given another closed ideal $J = \{f \in C_{0}(X) \colon f|_{N} = 0\}$ such that $I \subset J$ \ie $N \subset M$, then $J/I$ can be identified with $C_{0}(M\backslash N)$.

We do not use this concrete representation of $J/I$.
However, this shows that we stay within our setting of Banach spaces of continuous functions on locally compact spaces.
%% --
\begin{lemma}\label{lem:b3-2.8}
%%
%\index{Lemma!Spectral Bound}
%%
%\index{Spectral Bound!Properties}
Suppose $A$ is the generator of a positive semigroup $\TT$ such that the spectral bound $s(A)$ is a pole of the resolvent of order $k$.
Then there is a sequence
%% --
\begin{equation}\label{eq:b3-2.16}
    I_{-1} \coloneqq \{0\} \subset I_{0} \subsetneq I_{1} \subsetneq \ldots \subsetneq I_{k} \coloneqq E
\end{equation}
%% --
of $\TT$-invariant closed ideals satisfying the following. 
Denoting by $A_{n}$ $(n = 0,1,\ldots,k)$ the generator of the semigroup on $I_{n}/I_{n-1}$ which is induced by $(T(t))$ we have
%% --
\begin{enumerate}[\upshape (i)]
    \item 
    $s(A_{0}) < s(A)$\,,
    
    \item 
    If $n \geq 1$, then $s(A_{n}) = s(A)$ is a first order pole of the resolvent $R(\cdot,A_{n})$. The corresponding residue is a strictly positive operator.
\end{enumerate}
\end{lemma}
%% --
\begin{proof}
	We can assume that $s(A) = 0$ and we will denote the negative coefficients of the Laurent series of $R(\cdot,A)$ at $0$ by $Q_{n}$.
	Thus the following relations hold (see A-III,3.6),
	%% --
	\begin{equation}\label{eq:b3-2.17}
	\begin{split}
		Q_{n} &= \frac{1}{2\pi i}\int_{\gamma} z^{n-1}R(z,A)  \diff{z} \quad (n \in \N)\,, \\
		Q_{n} &\neq 0 \text{ if } n \leq k \text{ and } Q_{n} = 0 \text{ for } n > k\,, \\
		Q_{n} &= A^{n-1}Q_{1} \quad (n \in \N); \quad Q_{k} = \lim_{z \to 0}z^{k}\cdot R(z,A).
	\end{split}
	\end{equation}
	%% --
	We define $I_{n}$ as follows $(n = 0,1,\ldots,k-1)$.
	%% --
	\[
	I_{n} \coloneqq \{f \in E \colon Q_{n+1}|f| = Q_{n+2}|f| = \ldots = Q_{k}|f| = 0\}
	\]
	%% --
	At first we restrict our attention to $I_{k-1}$.
	
	Since $R(\lambda,A)$ is positive if $\lambda > 0$ (Corollary~\ref{cor:b3-1.3}), it follows from \eqref{eq:b3-2.17} that $Q_{k}$ is a positive bounded operator, hence $I_{k-1} = \{f \in E \colon Q_{k}|f| = 0\}$ is a closed ideal.
	Since $Q_{k}$ commutes with $R(\lambda,A)$ (see \eqref{eq:b3-2.17}, it follows that $I_{k-1}$ is a $T$-invariant ideal.
	By A-III,Corollary~4.3 the generators $A_{|I_{k-1}}$ and $A_{k}$ induced by $A$ on $I_{k-1}$ and $E/I_{k-1}$, respectively, have a pole at $0$.
	The coefficients of the Laurent series are the operators induced by $Q_{n}$ on $E/I_{k-1}$ and $I_{k-1}$, respectively.
	
	Suppose that the pole order of $R(\cdot,A_{k})$ is greater than $1$, say $m$.
	Then $Q_{m/} = \lim_{z \to 0}z^{m}R(z,A_{k})$ is a positive non-zero operator, hence we find for every $x \in E_{+}$ an element $y \in I_{k-1}$ such that $Q_{m}x + y \geq 0$.
	Then we have
	%% --
	\[
	0 \leq Q_{k}|Q_{m}x + y| = Q_{k}Q_{m}x + Q_{k}y = Q_{k+m-1}x + Q_{k}y = 0 + Q_{k}y \leq Q_{k}|y| = 0
	\]
	%% --
	hence $Q_{m}x = (Q_{m}x + y) - y \in I_{k-1}$ $(x \in E_{+})$.
	It follows that $Q_{m/} = 0$ which is a contradiction.
	
	So far we know that the resolvent of $A_{k}$ has a pole of order $\leq 1$.
	Moreover, since ${Q_{k}}_{|I_{k-1}} = 0$, 
	the resolvent of $A_{|I_{k-1}}$ has a pole of order $\leq k-1$.
	From A-III,Corollary~4.3 it follows that the pole order of $A_{k}$ and $A_{|I_{k-1}}$ is precisely $1$ and $k-1$, respectively.
	The residue ${Q_{1}}_{/I_{k-1}} = \lim_{z \to 0}zR(z,A_{k})$ is positive since $R(z,A_{k}) \geq 0$ for $z > 0$ (see Corollary~\ref{cor:b3-1.3}).
	It remains to prove that it is strictly positive. We assume $Q_{1/I_{k-1}}(|x + I_{k-1}|) = 0$ which means $Q_{1}|x| \in I_{k-1}$ hence $Q_{k}|x| = A^{k-1}Q_{1}|x| = 0$, that is, $x \in I_{k-1}$ or $x + I_{k-1} = 0$.
	
	Applying what we have proved so far to $I_{k-1}$ and $A_{|I_{k-1}}$ we obtain $I_{k-2}$, $A_{k-1}$, and so on.
	After $k$ steps $(n=1)$ we conclude that $I_{0}$ is $T$-invariant and that the order of the pole of $R(\cdot,A_{|I_{0}})$ is $0$, 
	implying that $0 \in \rho(A_{|I_{0}})$.
	Since $A_{|I_{0}}$ generates a positive semigroup and $R(\lambda,A_{|I_{0}}) = R(\lambda,A)_{|I_{0}}$ is positive for $\lambda > 0$, it follows from Corollary~\ref{cor:b3-1.3} that $s(A_{0}) = s(A_{|I_{0}}) < 0$.
\end{proof}
%% --
One can check the different steps of the proof by studying the following example
Consider this matrix as generator on $\C^{4}$
%% --
\[
\begin{pmatrix}
	-1 & a & b & c \\
	0 & 0 & d & e \\
	0 & 0 & 0 & f \\
	0 & 0 & 0 & 0
\end{pmatrix}
\quad \quad \text{where $ a,b,c,d,e,f \geq 0 $.} 
\]
%% --
The result is summarized in Table~\ref{tab:b3-table1} where 
$e_{j}$ are the unit vectors $(e_{j} \coloneqq (\delta_{jk}))$.
%The result is summarized in the following table $(e_{j} \coloneqq (\delta_{jk}))$.
%\marginpar{Die Tabelle auf p.14 sollte hierher}
%% --
\begin{table}[ht!]%\label{tab:b3-table1}
% 
\begin{center}
 \caption{Matrix Example}\label{tab:b3-table1}
\begin{tabular}{lll|c|cccc} % <--: 1st column left, 2nd middle and 3rd right, with vertical lines in between
		 &&& $ \overset{\text{\normalsize pole}}{\text{order}}$ & $I_0$ & $I_2$ & $I_2$ & $I_3$ \\[1ex]\hline
		$ d > 0 $, & $ f>0 $, &  $ e \ge 0 $ & 3 & $\langle e_1 \rangle$ &  $\langle e_1, e_2 \rangle$ &  $\langle e_1, e_2, e_3 \rangle$ &  $\C^4$\\
		$ d = 0 $, & $ f > 0 $, & $ e \ge 0 $ & 2 & $\langle e_1 \rangle$ &  $\langle e_1, e_2, e_3 \rangle$ &  $\C^4$ & \\	
		$ d = 0 $, & $ f = 0 $, & $ e > 0 $ & 2 & $\langle e_1 \rangle$ &  $\langle e_1, e_2, e_3 \rangle$ &  $\C^4$ & \\
		$ d > 0 $, & $ f = 0 $, & $ e > 0 $ & 2 & $\langle e_1 \rangle$ &  $\langle e_1, e_2 \rangle$ &  $\C^4$ & \\			
		$ d > 0 $, & $ f = 0 $, & $ e = 0 $ & 2 & $\langle e_1 \rangle$ &  $\langle e_1, e_2, e_4 \rangle$ &  $\C^4$ & \\	
\end{tabular}
\end{center}
\end{table}
%% --
This example also shows that the operators $Q_{k-1}, \ldots, Q_{1}$ are not necessarily positive (\eg $a>0$, $b=c=0$, $d=e=f=2$).

A more general (and more interesting) example is the following.
Suppose that $A_{i}$ $(i = 1,\ldots,n)$ are generators of positive semigroups on $C_{0}(X)$ such that $s(A_{i}) = 0$ is a first order pole of the resolvent.
And let $A_{ij}$ $(1 \leq i < j \leq n)$ be positive bounded operators on $C_{0}(X)$.
Then
%% --
\[
A \coloneqq \begin{pmatrix}
	A_{1} & A_{12} & \cdots & A_{1n} \\
	0 & A_{2} & \cdots & A_{2n} \\
	\vdots & \vdots & \ddots & \vdots \\
	0 & 0 & \cdots & A_{n}
\end{pmatrix}
\]
%% --
is the generator of a positive semigroup on 
%
\[
	C_{0}(X,\C^{n}) \cong C_{0}(X) \times C_{0}(X) \times \cdots \times C_{0}(X) , 
\]
%
and $s(A) = 0$ is a pole of the resolvent of order $k$ where $1 \leq k \leq n$.
%% --
\begin{theorem}\label{thm:b3-2.9}
%%
%\index{Theorem!Cyclic Boundary Spectrum}
%%
%\index{Boundary Spectrum!Cyclicity}
Suppose $A$ is the generator of a positive semigroup on $C_{0}(X)$ such that every point of $\sigma_{b}(A)$ is a pole of the resolvent
Then $P\sigma_{b}(A) = \sigma_{b}(A)$ is cyclic.
\end{theorem}
%% --
\begin{proof}
	If $\sigma(A) = \emptyset$, there is nothing to prove, thus we can assume that $s(A) = 0$.
	In view of the lemma and A-III, Proposition~4.3(i) we can assume that $s(A)$ is a first order pole with strictly positive residue, we call it $Q$.
	We have $AQ = QA = s(A)A = 0$ (see A-III, 3.6), hence
	%% --
	\begin{equation}\label{eq:b3-2.18}
		QT(t) = T(t)Q = Q \text{ for all } t \geq 0.
	\end{equation}
	%% --
	If $Ah = \im\alpha h$ for some $\alpha \in \R$, $h \neq 0$, then $T(t)h = \mathrm{e}^{\im\alpha t}h$ (by A-III, Corollary~6.4).
	Hence $|h| = |\mathrm{e}^{\im\alpha t}h| = |T(t)h| \leq T(t)|h|$, or equivalently, $T(t)|h| - |h| \geq 0$.
	By \eqref{eq:b3-2.18} we have $Q(T(t)|h| - |h|) = 0$.
	Since $Q$ is strictly positive, it follows that $T(t)|h| = |h|$ or $A|h| = 0$.
	Now we can apply Theorem~\ref{thm:b3-2.4} and obtain $Ah^{[n]} = \im n\alpha h^{[n]}$ for every $n \in \Z$.
	This shows that $P\sigma_{b}(A) = \sigma(A) \cap \im\R$ is cyclic.
\end{proof}
%% --
If $A$ has compact resolvent then every point of $\sigma(A)$ is a pole of the resolvent (see A-III,3.6) hence we have
%% --
\begin{corollary}\label{cor:b3-2.10}
%%
%\index{Corollary!Compact Resolvent}
%%
%\index{Resolvent!Compact Case}
If $A$ has compact resolvent, then $P\sigma_{b}(A) = \sigma_{b}(A)$ is cyclic.
\end{corollary}
%% --
If it is known that the boundary spectrum of a generator is cyclic and nonvoid, the following alternative holds.
%% --
\begin{equation}\label{eq:b3-2.19}
\text{Either $\sigma_{b}(A) = s(A)$ or else  $\sigma_{b}(A)$ is an infinite unbounded set.}
\end{equation}
%% --
If one can exclude the second alternative, then there is a unique spectral value having maximal real part.
A real spectral value $\lambda_{0}$ of a generator $A$ is called a \emph{dominant} provided that $\Re\lambda < \lambda_{0}$ for every $\lambda \in \sigma(A)$, it is called \emph{strictly dominant} if for some $\delta > 0$ one has $\Re\lambda \leq \lambda_{0} - \delta$ for every $\lambda \in \sigma(A)$, $\lambda \neq \lambda_{0}$.

The assumptions of Corollary~\ref{cor:b3-2.10} do not imply that $s(A)$ is dominant, the rotation semigroup (A-III,Example~5.6) is a counterexample.
%% --
\begin{corollary}\label{cor:b3-2.11}
%%
%\index{Corollary!Dominant Eigenvalue}
%%
%\index{Eigenvalue!Strict Dominance}	
	Assume that for some $t_{0} > 0$ (hence for all $t > 0$) one has 
    %% --
    \[
    r_{\text{ess}}(T(t_{0})) < r(T(t_{0})), 
    \]
    %% --
    \eg that $T(t_{0})$ is compact and $r(T(t_{0})) > 0$ (see A-III,3.7).
	Then $s(A)$ is a strictly dominant eigenvalue.
\end{corollary}
%% --
\begin{proof}
	If $s(A)$ is not strictly dominant, then we have by Theorem~\ref{thm:b3-2.9} and A-III,Corollary~6.5 that 
    %% --
    \[
    \{\lambda \in \sigma(A) \colon \Re\lambda > s(A) - r\}
    \]
    %% -- 
    contains infinitely many eigenvalues for every $r > 0$.
	From A-III,Corollary~6.4 it follows that $\{\lambda \in \sigma(T(t)) \colon |\lambda| > r\}$ contains infinitely many eigenvalues (counted according to their multiplicities) for every $r < \exp(s(A)t) = r(T(t))$.
	This contradicts the assumption $r_{\text{ess}}(T(t)) < r(T(t))$ (see A-III,3.7).
\end{proof}
%% --
\begin{corollary}\label{cor:b3-2.12}
%%
%\index{Corollary!Compact Resolvent}
%%
%\index{Resolvent!Norm Continuous Case}	
	Suppose $A$ has compact resolvent and non-empty spectrum.
	If the corresponding semigroup is eventually norm continuous (\eg if it is holomorphic or differentiable), then there is a strictly dominant eigenvalue admitting a positive eigenfunction.
\end{corollary}
%% --
\begin{proof}
	Since $(T(t))_{t \geq 0}$ is eventually norm continuous, 
    %% --
    \[
    \{\lambda \in \sigma(A) \colon \Re\lambda \geq s(A)-r\}
    \]
    %% --$$ 
    is compact for every $r > 0$ (see A-II, Theorem~1.20) and this set does not have accumulation points because $A$ has compact resolvent.
	In other words, it is a finite set.
	The assertion now follows from Theorem~\ref{thm:b3-2.9} and Corollary~\ref{cor:b3-1.4}.
\end{proof}
%% --
We now consider some examples. 
The first one shows that there are positive semigroups with $P\sigma_{b}(A)$ being not cyclic.
It is unknown whether there are positive semigroups where $\sigma_{b}(A)$ is not cyclic.
%% --
\begin{example}\label{ex:b3-2.13}
%
%\index{Spectral Theory on $ C_{0}(X)$!Examples!Non-cyclic Spectrum}
%% --
	Consider $E = C(\Gamma) \times C_{0}(\R) \, (\cong C_{0}(\Gamma\dot{\cup}\R))$.
	We fix a positive function $k \in C_{0}(\R)$ with compact support.
	The operator $A$ given by
	%% --
	\begin{equation}\label{eq:b3-2.20}
		\begin{aligned}
%\begin{empheq}[right={\empheqrbrace \,\text{(*)}}]{align*}
		A(f,g) &\coloneqq (f',g' + \frac{1}{2\pi}\int_{0}^{2\pi} f(\theta) \, \diff{\theta}\cdot k) \\
		D(A) &\coloneqq \{(f,g) \in E \colon f,g \in C^{1}, g' \in C_{0}(\R)\}
%\end{empheq}
		\end{aligned}
	\end{equation}
	%% --
	generates a semigroup $(T(t))_{t \geq 0}$ which is given by
	%% --
	\begin{equation}\label{eq:b3-2.21}
		\begin{aligned}
			T(t)(f,g) &\coloneq (f_{t},g_{t}) \text{ with }\\
		   f_{t}(\theta) &\coloneqq f(\theta+t), \\
			g_{t}(x) &\coloneqq g(x+t) + \frac{1}{2\pi}\int_{0}^{2\pi} f(\theta) \diff{\theta}\cdot\int_{x}^{x+t} k(u) \, \du .
		\end{aligned} 
	\end{equation}
	%% --
	Then $(T(t))_{t \geq 0}$ is a positive semigroup and $\|T(t)\| \leq (1 + \|k\|_{1})$.
	In particular, $s(A) \leq \omega_{0}(A) \leq 0$.
	It is easy to see that $0$ is not an eigenvalue of $A$, while all $\im k$, $k \in \Z$, $k \neq 0$ are eigenvalues, the corresponding eigenfunctions being $(e_{k},0)$ with $e_{k}(\theta) = \mathrm{e}^{\im k\theta}$.
\end{example}
%% --
\begin{example}\label{ex:b3-2.14}
%
%\index{Spectral Theory on $ C_{0}(X)$!Examples!Schrödinger Operator}


\begin{enumerate}[\upshape (i), wide, labelindent=.5em]
	\item \emph{One-dimensional Schrödinger operator.}

	Let $X = \R$, $E = C_{0}(X)$ and $V \colon \R \to \R$ be a continuous function such that $\inf_{x} V(x) > -\infty$.
	
	If we define
	%% --
	\begin{equation}\label{eq:b3-2.22}
		\begin{aligned}
			(Af)(x) &\coloneqq f''(x) - V(x)f(x), \\
			D(A) &\coloneqq \{f \in C_{0}(X) \colon f \in C^{2}, Af \in C_{0}(X)\}.
		\end{aligned}
	\end{equation}
	%% --
	then $A$ is the generator of a positive semigroup.
	
	In case $\lim_{|x| \to \infty}V(x) = \infty$, $A$ has compact resolvent.
	Then, by Corollary~\ref{cor:b3-2.10}, there exists a dominant real eigenvalue with corresponding positive eigenfunction.
	Actually, the eigenfunction $f$ is strictly positive. 
    (In fact, if $f \in C^{2}$, $f \geq 0$ and $f(x_{0}) = 0$ for some $x_{0}$, then $f'(x_{0}) = 0$.
	Therefore the uniqueness theorem for ordinary differential equations implies that $f$ is identically zero).
	
\item \emph{A retarded linear differential equation.} 	

	Consider $E = C[-1,0]$ and define $A_{m}$ and $A_{0}$ as follows,
	%% --
	\begin{equation}\label{eq:b3-2.23}
		A_{m}f \coloneqq f', \quad f \in D(A_{m}) = C^{1}[-1,0]\,,
	\end{equation}
	%% --
	%% --
	\begin{equation}\label{eq:b3-2.24}
		A_{0}f \coloneqq f', \quad f \in D(A_{0}) = \{f \in C^{1}[-1,0] \colon f'(0) = 0\}.
	\end{equation}
	%% --
	$A_{0}$ generates a contraction semigroup $(T_{0}(t))_{t \geq 0}$ which is given by
	%% --
	\begin{equation}\label{eq:b3-2.25}
		(T_{0}(t)f)(x) = \begin{cases}
			f(x+t) & \text{if } x+t \leq 0\,, \\
			f(0) & \text{if } x+t \geq 0.
		\end{cases}
	\end{equation}
	%% --
	This semigroup is positive, eventually norm continuous ($T_{0}(t) = \delta_{0}\otimes \1$ for $t \geq 1$) and has compact resolvent.
	Given a linear functional $\phi$ on $C[-1,0]$, we consider
	%% --
	\begin{equation}\label{eq:b3-2.26}
		A_{\phi} \coloneqq {A_{m}}_{|D(A_{\phi})} \text{ with } D(A_{\phi}) \coloneqq \{f \in C^{1}[-1,0] \colon f'(0) = \langle f,\phi \rangle\}.
	\end{equation}
	%% --
	Denoting the exponential function $x \mapsto \mathrm{e}^{\lambda x}$ by $e_{\lambda}$, we have for real $\lambda$ and $\lambda > \|\phi\|$.
	%% --
	 \begin{equation}\label{eq:b3-2.27}
	 \begin{aligned}
		\text{Id} - 1/\lambda \cdot \phi \otimes e_{\lambda} \text{ is a bijection of } D(A_{\phi}) \text{ onto } D(A_{0})   \\
		\text{ and }\quad \lambda - A_{\phi} = (\lambda - A_{0})(\text{Id} - 1/\lambda \cdot \phi \otimes e_{\lambda}).
	\end{aligned}
	\end{equation}
	%% --
	Using the Neumann series expansion of $(\text{Id} - 1/\lambda \cdot \phi \otimes e_{\lambda})^{-1}$ one obtains the following estimate.
	%% --
	\begin{equation}\label{eq:b3-2.28}
		\|(\text{Id} - 1/\lambda \cdot \phi \otimes e_{\lambda})^{-1}\| \leq \lambda/(\lambda - \|\phi\|) \quad \text{if } \lambda > \|\phi\|.
	\end{equation}
	%% --
It follows from \eqref{eq:b3-2.25} and \eqref{eq:b3-2.26} that for $\lambda > \|\phi\|$, the resolvent $R(\lambda,A_{\phi})$ exists and satisfies $\|R(\lambda,A_{\phi})\| \leq \lambda/(\lambda-\|\phi\|)\cdot 1/\lambda = 1/(\lambda-\|\phi\|)$.
Then the Hille-Yosida Theorem (A-II, Theorem~1.7) implies that $A_{\phi}$ generates a semigroup $(T(t))$ satisfying $\|T(t)\| \leq \exp(\|\phi\|t)$.
Moreover, this semigroup is eventually norm continuous (see B-IV,Corollary~3.3).

By B-II,Example~1.22 we have the following equivalence.
%% --
\begin{equation}\label{eq:b3-2.29}
	A_{\phi} \text{ generates a positive semigroup if and only if } \phi + r\delta_{0} \geq 0 \text{ for some } r \in \R.
\end{equation}
%% --
Thus Corollary~\ref{cor:b3-2.12} is applicable if $\phi + r\delta_{0} \geq 0$ for some $r \in \R$.
Since every eigenvalue of $A_{\phi}$ is an eigenvalue of $A_{m}$ and since $\Kern{\lambda-A_{m}} = \{\alpha e_{\lambda} \colon a \in \C\}$, the spectral bound $s(A_{\phi})$ is determined by the (unique) real $\lambda \in \R$ such that $e_{\lambda} \in D(A_{\phi})$ or equivalently, $\lambda$ is a solution of the so-called \emph{characteristic equation}
%% --
\begin{equation}\label{eq:b3-2.30}
	\lambda = \phi(e_{\lambda}), \quad \lambda \in \R.
\end{equation}
%% --
(The assumption $\phi + r\delta_{0} \geq 0$ implies that the function $\lambda \mapsto \phi(e_{\lambda})$ is strictly decreasing and $\lim_{\lambda \to \infty}\langle e_{\lambda},\phi \rangle > -\infty$, $\lim_{\lambda \to -\infty}\langle e_{\lambda},\phi \rangle = \infty$ unless $\phi = r_{0}\delta_{0}$ for some $r_{0} \in \R$.)
\end{enumerate}
%% --
\end{example}
%% --
We conclude this section with some additional remarks related to Theorem~\ref{thm:b3-2.9} and its corollaries.
%% --
\begin{remarks}\label{rem:b3-2.15}
%
%\index{Spectral Theory on $ C_{0}(X)$Remarks!Resolvent Poles}

	
	\begin{enumerate}[\upshape (i), wide, labelindent=.5em]
	\item
	If $s(A)$ is a pole of the resolvent, then for generators of positive semigroups one has the following equivalences.
	
	\begin{enumerate}[\upshape (a)]
		\item 
		$s(A)$ is a first order pole.
		
		\item 
		For every $0 < f \in \Kern{s(A) - A}$ there exists $0 \leq \phi \in \Kern{s(A) - A'}$ such that $\langle f,\phi \rangle > 0$.
		
		\item 
		For every $0 < \phi \in \Kern{s(A) - A'}$ there exists $0 \leq f \in \Kern{s(A) - A}$ such that $\langle f,\phi \rangle > 0$.
	\end{enumerate}
	In particular, if $\Kern{s(A) - A}$ contains a strictly positive function or if $\Kern{s(A) - A'}$ contains a strictly positive measure, then $s(A)$ is a first order pole.

	We sketch the proof of (a) $\Leftrightarrow$ (b) assuming that $s(A) = 0$.
	If $0$ is a first order pole, then the residue $P$ is a positive projection satisfying $PE = \Kern A$, $P'E' = \Kern A'$ (see A-III,3.6).
	Thus given $0 < f \in \Kern A$ and any $0 \leq \phi \in E'$ such that $\langle f,\phi \rangle > 0$, we have for $\phi \coloneqq P'\phi$: $\langle f,\phi \rangle = \langle f, P'\phi \rangle = \langle Pf,\phi \rangle = \langle f,\phi \rangle > 0$.
	To prove the reverse direction, we first observe that the highest coefficient $Q_{k}$ of the Laurent expansion is a positive operator.
	Thus if $0$ is a pole of order $k \geq 2$, we choose $0 < h \in E$ such that $f \coloneqq Q_{k}h > 0$.
	Then $Af = AQ_{k}h = 0$ and for every $\phi \in \Kern A'$ we have $\langle f,\phi \rangle = \langle Q_{k}h,\phi \rangle = \langle h,Q_{k}'\phi \rangle = \langle h,Q_{k-1}'A'\phi \rangle = 0$.
	%% --
	\item 
	If a linear operator $S$ on $C_{0}(X)$ is weakly compact, then $S^{2}$ is compact (see B-IV, Proposition~2.4(b)).
	Therefore every non-zero spectral value of a weakly compact operator is a pole of the resolvent.
	This shows that Theorem~\ref{thm:b3-2.9} is applicable if either $T(t_{0})$ is weakly compact for some $t_{0}$ or $R(\lambda,A)$ is weakly compact for some $\lambda \in \rho(A)$.
	We quote two criteria for weak compactness.
	%% --
    \begin{equation}\label{eq:b3-2.31}
		\begin{minipage}{.8\textwidth}
    	If   $T \in \L{C(K)}$, $K$ compact, is positive, then it is weakly compact  	if and only if its biadjoint 	$T''$ maps bounded Borel functions into $C(K)$ (see B-IV, Proposition~2.4).
    	\end{minipage}
	\end{equation}
%% --
	\begin{equation}\label{eq:b3-2.32}
        \begin{minipage}{.8\textwidth}
    		A positive operator   $T$ on $C_{0}(X)$ which is dominated by a finite rank 
    		operator, is weakly compact. 
    	\end{minipage}
    \end{equation}
%% --
    Actually, its adjoint  $T'$ is dominated by a finite rank operator as well, hence it maps the unit ball in an order interval. 
	It follows that $ T' $  is weakly compact hence so is $T$.		
	
		
	\item 
	Stronger results than Theorem~\ref{thm:b3-2.9} will be proved in Chapter C-III.
	Actually, assuming only that $s(A)$ is a pole of finite algebraic multiplicity one can show that $\sigma_{b}(A)$ contains only poles of finite multiplicity (C-III, Theorem~3.13).
	In C-III,Corollary~2.12 we will show that $\sigma_{b}(A)$ is cyclic whenever $s(A)$ is a pole of the resolvent.
	
	\item 
	Example~\ref{ex:b3-2.14}(ii) can be extended to systems of functional differential equations even the infinite dimensional case.
	For details we refer to Section~3 of Chapter B-IV.
\end{enumerate}	
\end{remarks}
%% --
\section{Irreducible Semigroups}%
%\index{Spectral Theory on $ C_{0}(X)$!Irreducible Semigroups}
%% --
In the case of matrices it is well known that considerably stronger results are available if one considers positive matrices which are irreducible. 
The concept of irreducibility can be extended to our setting and in many cases one can check easily whether a given semi­group has this property (see Example~\ref{ex:b3-3.4}). 
We will show that irreducible semigroups have many interesting properties. 
For example, the spectrum $\sigma(A)$ is always non-empty, positive eigenfunctions are strictly positive and if $s(A)$ is a pole, it is algebraically (and geometri­cally) simple (see Proposition~3.5). 
Moreover, in certain cases irreducibili­ty ensures that $\sigma_{b}(A)$ and $P\sigma_{b}(A)$ are not only cyclic subsets but \emph{subgroups} (see Theorem~\ref{thm:b3-3.6} and Theorem~\ref{thm:b3-3.11} for details). 

We start the discussion with several, mutually equivalent, definitions of irreducibility. 
%% --
\begin{definition}\label{def:b3-3.1}
%%
%\index{Semigroups!Irreducible}
%%
%\index{Irreducible Semigroups}
A positive semigroup $\TT = (T(t))$ on $E = C_{0}(X)$, $X$ locally compact, with generator $A$ is called \emph{irreducible} if one of the following, mutually equivalent conditions is satisfied.
%% --
\begin{enumerate}[\upshape (a)]

\item\label{item:b3-3.1-i}
There is no $T$-invariant closed ideal except $\{0\}$ and $E$.
	
\item\label{item:b3-3.1-ii}
Given $0 < f \in E$, $0 < \phi \in E'$, then $\langle T(t_{0})f,\phi \rangle > 0$ for some $t_{0} \geq 0$.
	
\item\label{item:b3-3.1-iii}
For every $f > 0$ we have 
$\bigcup_{t \geq 0}\{x \in X \colon (T(t)f)(x) > 0\} = X$.
	
\item\label{item:b3-3.1-iv}
For some (every) $\lambda > s(A)$ there exists no closed ideal which is invariant under $R(\lambda,A)$ except $\{0\}$ and $E$.
	
\item\label{item:b3-3.1-v}
For some (every) $\lambda > s(A)$, we have $R(\lambda,A)f$ is strictly positive whenever $f > 0$.
	
\item\label{item:b3-3.1-vi}
$\bigcup_{t \geq 0}\supp T(t)'\delta_{x}$ is dense in $X$ for every $x \in X$.

\end{enumerate}
\end{definition}
%% --
That these six conditions are actually equivalent can be seen as follows.

\ref{item:b3-3.1-i} $\implies$ \ref{item:b3-3.1-ii}: Suppose there are $0 < f \in E$, $0 < \phi \in E'$ such that $\langle T(t)f,\phi \rangle = 0$ for every $t \geq 0$.
Then the ideal $I$ generated by $\{T(t)f \colon t \geq 0\}$ satisfy $T \neq \{0\}$  and  $T \subset \{g \in E \colon \phi(|g|) = 0\} \neq E$.
Obviously $I$ is $\TT$-invariant.

\ref{item:b3-3.1-ii} $\implies$ \ref{item:b3-3.1-iii}: Given $0 < f \in E$, $x \in X$.
Then by (b) there exists $t_{0}$ such that $(T(t_{0})f)(x) = \langle T(t_{0})f,\delta_{x} \rangle > 0$.

\ref{item:b3-3.1-iii} $\implies$ \ref{item:b3-3.1-vi}: Suppose that $\bigcup_{t \geq 0}\supp T(t)'\delta_{y}$ is not dense for some $y \in X$.
Then there exists $f_{0} \in E$, $f_{0} > 0$ such that
$\supp f_{0} \cap \supp T(t)'\delta_{y} = \emptyset$ for every $t \geq 0$.
Hence $(T(t)f_{0})(y) = \langle f_{0},T(t)'\delta_{y} \rangle > 0$, \ie $y \notin \bigcup_{t\geq 0}\{x \in X \colon (T(t)f_{0})(x) > 0\}$.

\ref{item:b3-3.1-vi} $\implies$ \ref{item:b3-3.1-v}: Given $0 < f \in E$, $\lambda > \omega_{0}(A)$, $y \in X$, there exists $t_{0} \geq 0$ such that 
$\{x \colon f(x) > 0\} \cap \supp T(t_{0})'\delta_{y} \neq \emptyset$.
Hence, $(T(t_{0})(f)(y) = \langle f,T(t_{0})'\delta_{y} \rangle > 0$ and therefore
%% --
\[
(R(\lambda,A)f)(y) = \int_{0}^{\infty} \mathrm{e}^{-\lambda t}(T(t)f)(y) \dt > 0 .
\]
%% --
Since $\lambda \mapsto R(\lambda,A)f$ is decreasing in the interval $(s(A),\infty)$ (use the resolvent equation and the fact that $R(\lambda,A)$ is positive) we have $R(\lambda,A)f \gg 0$ for all $\lambda > s(A)$.
%% --

\ref{item:b3-3.1-v} $\implies$ \ref{item:b3-3.1-vi}: If $I$ is a $R(\lambda,A)$-invariant ideal and 0$ < f \in I$, then $g \coloneqq R(\lambda,A)f \in I$. 
By (e), $g$ is strictly positive, thus $I$ has to be dense (it contains all functions of compact support). 

\ref{item:b3-3.1-iv} $\implies$ \ref{item:b3-3.1-i}: At first we recall that a closed linear subspace which is invariant for $R(\lambda_0,A)\ (\lambda_0 \in \rho(A))$, is invariant for $R(\lambda ,A)$
whenever $\lambda$ and $\lambda_0$ belong to the same component of $\rho((A)$. 
Hence, if $\lambda_0 \in \rho_{+}(A)$ then by A-I,3.2 every $R(\lambda_0,A)$-invariant subspace is 
$\TT$-invariant and vice versa.
%% --
\begin{remark}\label{rem:b3-3.2}
%%
%\index{Semigroups!Irreducible!Conditions}
	Obviously, irreducibility of a semigroup $(T(t))_{t\geq 0}$ is implied by the following condition.
\begin{enumerate}[(g)]	
\item\label{item:b3-3.1-vii}
$T(t)f \gg 0$ whenever $f > 0$ and $t > 0$.
\end{enumerate}
%% --
The rotation semigroup (see A-I,2.5) is irreducible but it does not satisfy condition \ref{item:b3-3.1-vii}.
	However, assuming that the semigroup $(T(t))$ is holomorphic, then \ref{item:b3-3.1-vii}
	is equivalent to irreducibility.
	We will give a proof of this result in the more general situation of Banach lattices (see C-III, Theorem~3.2(b)).
\end{remark}
%% --
A semigroup $(T(t))_{t \ge 0}$ is irreducible if and only if $(\mathrm{e}^{-\alpha t}T(t))_{t \ge 0}$, $\alpha \in \R$  is. 
More generally, irreducibility is invariant under perturbations by multiplication operators. 
In fact, we have the following result. 
%% --
\begin{proposition}\label{prop:b3-3.3}
%
%\index{Semigroups!Irreducible!Perturbation}
	Suppose $A$ generates a positive semigroup $\TT$ on $C_{0}(X)$ and let $h$ be a continuous, bounded real-valued function on $X$.
	Then the semigroup $\mathcal{S}$ generated by $B \coloneqq A + M_{h}$ is irreducible if and only if\/ $\TT$ has this property.
\end{proposition}
%% --
\begin{proof}
	Since every closed ideal is of the form $\{f \in E \colon f|_{M} = 0\}$ where $M \subset X$ is a closed subset (cf.\ Section~1 of B-I), it is clear that all closed ideals are invariant under the multiplication operator $M_{h}$ and $M_{-h}$, respectively.
	Thus the assertion follows from the expansions which are true for $\lambda$ sufficiently large
	%% --
\begin{align*}
R(\lambda,B) &= (1 - R(\lambda,A)M_{h})^{-1}R(\lambda,A) 
	= \sum_{n=0}^{\infty} (R(\lambda,A)M_{h})^{n}R(\lambda,A) \\
R(\lambda,A) &= (1 - R(\lambda,B)M_{-h})^{-1}R(\lambda,B) = \sum_{n=0}^{\infty} (R(\lambda,B)M_{-h})^{n}R(\lambda,B)
\end{align*} 
	%% --
\end{proof}
%% --
\begin{examples}\label{ex:b3-3.4}
%
%\index{Semigroups!Irreducible!Examples}
	
\begin{enumerate}[\upshape (i), wide, labelindent=.5em]
	\item 
	(cf.\ B-II,Section~3). Suppose $(T(t))_{t\geq 0}$ is governed by a continuous semiflow $\phi \colon \R_{+} \times X \to X$, \ie $T(t)f = f\circ\phi_{t}$ $(f\in C_{0}(X))$.
	Then the following assertions are equivalent.
	\begin{enumerate}[\upshape (a)]
		\item 
		$(T(t))_{t\geq 0}$ is irreducible.
	
		\item 
		There is no closed subset of $X$ which is $\phi$-invariant except $\emptyset$ and $X$.
	
		\item 
		Every orbit $\{\phi(t,x) \colon t \in \R_{+}\}$ is dense in $X$.
	\end{enumerate}
	More generally, these equivalences are also valid if the semigroup $(T(t))$ is given by $T(t)f = h_{t}\cdot(f\circ\phi_{t})$ where $h_{t}$ are suitable continuous, strictly positive, bounded functions on $X$.
	
	\item 
	Suppose that the semigroup $(T(t))_{t\geq 0}$ has the following form. 
    There exist a positive measure $\mu$ on $X$ and a positive continuous function $k \colon (0,\infty) \times X \times X \to \R$ such that
%% --
	\begin{equation}\label{eq:b3-3.1}
		(T(t)f)(x) = \int_{X} k(t,x,y)f(y) \diff{\mu}(y) \quad (t > 0, f \in C_{0}(X), x \in X).		
	\end{equation}	
%% --
	Then $(T(t))_{t\geq 0}$ is irreducible if and only if $\bigcup_{t>0}\text{supp}\{k(t,x,.)\}$ is dense in $X$ for every $x \in X$.
	
	\item 
	We consider the first derivative $Af = f'$ (cf.\ A-I,2.4). 
    If $E = C_{0}(\R)$, then the corresponding semigroup $(T(t))$ is not irreducible.
	Note however, that there is no closed invariant ideal $I$ with $\{0\} \neq I \neq E$ which is invariant under the group $(T(t))_{t\in\R}$ generated by $A$.
	For $E = C_{0}[0,\infty)$ and $E = C_{0}(-\infty,0)$ the corresponding semigroups are reducible (\ie not irreducible) as well.
	If $E = C_{2\pi}(\R)$ (\ie the $2\pi$-periodic functions), then $Af = f'$ generates an irreducible semigroup on $E$.
	It is (isomorphic to) the semigroup of rotations on the unit circle.
    
	\item 
	(cf.\ Example~\ref{ex:b3-2.14}(ii)) Now we consider $Af = f'$ on $E = C[-1,0]$ with $D(A_{\phi}) = \{f \in C^1 \colon f'(0) = \phi(f)\}$ where the linear functional $\phi$ satisfies $\phi + \alpha\delta_{0} \geq 0$ for some $\alpha \in \R$ (see B-II,Example~1.22).
	The corresponding semigroup is irreducible if and only if $-1 \in \supp\phi$.
	
	\item 
	The second derivative $Af = f''$ generates an irreducible semigroup on $C_{0}(\R)$ and on $C_{0}(0,1)$ (cf.\ A-I,2.7).
	With Neumann boundary conditions (or more generally, $f'(0) = \alpha_{0}f(0)$, $f'(1) = \alpha_{1}f(1)$ where $\alpha_{0}, \alpha_{1} \in \R$) the second derivative generates an irreducible semigroup on $C[0,1]$ (cf.\ A-I,2.7).
	
	The operator $Af = f'' - Vf$ on $C_{0}(\R)$, where $V$ is continuous, real-valued with $\inf V(x) > -\infty$ (see Example~\ref{ex:b3-2.14}(i)) also generates an irreducible semigroup.
	This can be derived from the maximum principle as follows. 
    For $\lambda > -\inf V(x)$, $f \in C_{0}(\R)$, $g \coloneq R(\lambda,A)f$ we have $g \in C^2$ and $g'' - (\lambda + V)g = -f$.
	If $f > 0$, then $g > 0$, hence \citet[Chap.I, Theorem~3]{protterweinberger:1967} implies that $g$ is strictly positive.
	
	\item 
	The Laplacian $\Delta$ generates an irreducible semigroup on $C_{0}(\R^n)$ as can be seen easily from A-I,2.8.
	More general elliptic operators will be discussed below (see Example~\ref{ex:b3-3.10}(ii)).
\end{enumerate}
\end{examples}
%% --
We now return to the general situation and show that irreducible semigroups possess several interesting properties.
%% --
\begin{proposition}\label{prop:b3-3.5}
	Suppose $A$ is the generator of a strongly continuous semigroup on $C_{0}(X)$ which is irreducible.
	Then the following assertions are true.
	\begin{enumerate}[\upshape (i)]
		\item 
		$\sigma(A) \neq \emptyset$.
		
		\item 
		Every positive eigenfunction of $A$ is strictly positive.
		
		\item 
		Every positive eigenvector of $A'$ is strictly positive.
		
		\item 
		If\ $\Kern{s(A) - A'}$ contains a positive element (\eg if $X$ is compact (cf.\ Theorem~\ref{thm:b3-1.6}), then $\dim(\Kern{s(A) - A} \leq 1$.
		
		\item 
		if $s(A)$ is a pole of the resolvent, then it is algebraically simple.
		The residue has the form $P = \phi \otimes u$ where $\phi \in E'$ and $u \in E$ are strictly positive eigenvectors of $A'$ and $A$, respectively, satisfying $\langle u,\phi\rangle = 1$.
	\end{enumerate}
\end{proposition}
%% --
\begin{proof} 
\begin{enumerate}[\upshape (i), wide, labelindent=.5em]
	\item 
	Take any $f_{0} \in C_{0}(X)$ which is positive and has compact support.
	If $\lambda > s(A)$, then $R(\lambda,A)f_{0}$ is strictly positive (by Definition~3.1(e)), hence there exists $\epsilon > 0$ such that $R(\lambda,A)f_{0} \geq \epsilon f_{0}$.
	It follows that $R(\lambda,A)^nf_{0} \geq \epsilon^nf_{0} \geq 0$ for all $n \in \N$ and therefore
	$r(R(\lambda,A)) = \lim_{n \to \infty}\|R(\lambda,A)^{n}\|^{1/n} \geq \epsilon > 0 $.
	The assertion now follows from A-III, Proposition~2.5.	

	\item 
	Suppose $Ah = rh$ where $h \neq 0$ is positive.
	Then $r$ has to be real and we have $T(t)h = \mathrm{e}^{rt}h$ (A-III,Corollary~6.4).
	For $|f| \leq n\cdot h$ $(n \in \N)$ we have
	%% --
	\begin{equation}\label{eq:b3-3.2}
		|T(t)f| \leq T(t)|f| \leq n\cdot T(t)h = n\cdot \mathrm{e}^{rt}h.		
	\end{equation}
	%% --
		This shows that the ideal generated by $h$ is invariant, hence dense by irreducibility.
	This is true if and only if $h$ is strictly positive.
	
	\item 
	Suppose $A'\phi = r\phi$ for some $0 < \phi \in E'$.
	Again $r$ has to be real and $T(t)'\phi = \mathrm{e}^{rt}\phi$ $(t \geq 0)$.
	From
	%% --
	\begin{equation}\label{eq:b3-3.3}
	\langle|T(t)f|,\phi\rangle \leq \langle T(t)|f|,\phi\rangle = \langle|f|,\mathrm{e}^{rt}\phi\rangle, f \in E			
	\end{equation}
	%% --
	it follows that $I \coloneq \{f \in E \colon \phi(|f|) = 0\}$ is an invariant ideal.
	We have $I \neq E$ (because $\phi \neq 0$), hence the irreducibility implies $I = \{0\}$, \ie $\phi$ is strictly positive.
	
	\item 
	By (i) we know that $s(A) > -\infty$ hence we can assume without loss of generality that $s(A) = 0$.
	By (iii) there exists a strictly positive $\phi \in E'$ such that $A'\phi = 0$.
	It follows from \eqref{eq:b3-2.14} and \eqref{eq:b3-2.15} that
	%% --
	\begin{equation}\label{eq:b3-3.4}
	h \in \Kern A \text{ implies } |h| \in \Kern A
	\end{equation}
%% --
	Assuming $\dim(\Kern A) \geq 2$, then there is an eigenfunction $h \in \Kern A$, $h \neq 0$ which has at least one zero in $X$ $(h \coloneq h_{1}(x_{0})\cdot h_{2} - h_{2}(x_{0})\cdot h_{1}$, where $h_{1}$, $h_{2}$ are linearly independent, $x_{0} \in X)$.
	By \eqref{eq:b3-3.4} $|h|$ is a positive eigenfunction but not strictly positive.
	This is a contradiction with (ii).
	
	\item 
	If $s(A)$ is a pole, then there exists a corresponding positive eigenfunction (see the proof of Corollary~\ref{cor:b3-1.4}).
	By (ii) it is even strictly positive, thus $s(A)$ is a first order pole by Remark~\ref{rem:b3-2.15}(i).
	The residue $P$ is a positive operator satisfying $PE = \Kern{s(A) - A}$ and $P'E' = \Kern{s(A) - A'}$, therefore the remaining assertion follows from (ii) and (iv).
	\end{enumerate}
	\end{proof}
%% --ulgr ende
In the remainder of this section we focus our interest on the boundary spectrum of irreducible semigroups, more precisely, on the eigenvalues and the corresponding eigenfunctions of the boundary spectrum. 
In view of assertion (a) of Proposition~\ref{prop:b3-3.5} the assumption \enquote{$s(A) = 0$} is not crucial in the following theorem. 
However, it allows a simpler formulation. 

\begin{theorem}\label{thm:b3-3.6}
%
%\index{Semigroups!Irreducible!Eigenvalues}
	Suppose $\TT = (T(t))$ is an irreducible semigroup with generator $A$ and spectral bound $s(A) = 0$.
	Assume that there exists a positive linear form $\phi \neq 0$ such that $A'\phi = 0$. 
    (This is automatically satisfied whenever $X$ is compact (see Theorem~\ref{thm:b3-1.6}).)
	If $P\sigma(A)\cap \im\R$ is non-empty, then the following assertions are true.
	\begin{enumerate}[\upshape (i)]
		\item 
		$P\sigma(A)\cap \im\R$ is a (additive) subgroup of $\im \R$.
	
		\item 
		The eigenspaces corresponding to $\lambda \in P\sigma(A)\cap \im\R$ are one-dimensional.
	
		\item 
		If $Ah = \im\alpha h$ $(h \neq 0, \alpha \in \R)$, then $h$ has no zeros in $X$.
		In case $\alpha = 0$ then $h(x)/|h(x)|$ is constant; otherwise, $\{h(x)/|h(x)| \colon x \in X\}$ is a dense subset of $\Gamma$.
	
		\item 
		If $Ah = \im\alpha h$ $(h \neq 0, \alpha \in \R)$, then
		%% --
		\begin{equation}\label{eq:b3-3.5}
			S_{h}(D(A)) = D(A) \text{ and } S_{h}^{-1}\circ A \circ S_{h} = (A + \im\alpha).
		\end{equation}
		%% --
		In particular, spectrum and point spectrum of $A$ are invariant under translations by $\im \alpha$.
	
		\item 
		$0$ is the only eigenvalue admitting a positive eigenfunction.
	\end{enumerate}
\end{theorem}
%% --
\begin{proof}
	By Proposition~\ref{prop:b3-3.5}(iii) the invariant linear form $\phi$ is strictly positive and it satisfies $T(t)'\phi = \phi$ $(t \geq 0)$.
	\begin{enumerate}[wide, labelindent=.5em]
	\item[(iv)] 
	Supposing $Ah = \im\alpha h$ $(h \neq 0, \alpha \in \R)$ then $A|h| = 0$ by \eqref{eq:b3-2.14} and \eqref{eq:b3-2.15}.
	By Proposition~\ref{prop:b3-3.5}(ii) $|h|$ is strictly positive, thus Theorem~\ref{thm:b3-2.4}(ii) implies \eqref{eq:b3-3.5}.
	
	\item[(ii)] 
	Assertion (iv) implies that $S_{h}$ maps $\Kern{\im\alpha + A}$ onto $\Kern A$ whenever $\im \alpha \in P\sigma(A)\cap \im\R$.
	Moreover, we have seen in the proof of (iv) that $\Kern A \neq \{0\}$ hence it is one-dimensional by Proposition~\ref{prop:b3-3.5}(iv).
	
	\item[(i)] 
	Assume that $Ah = \im\alpha h$, $Ag = \im\beta g$ $(\alpha,\beta \in \R, h \neq 0, g \neq 0)$.
	By \eqref{eq:b3-3.5} we have $S_{\bar{g}}A S_{g} = A + \im\beta$ and $S_{h}A S_{\bar{h}} = A - \im\alpha$, therefore
	\begin{equation}\label{eq:b3-3.6}
	A + i(\beta - \alpha) = S_{h}(A + \im\beta)S_{\bar{h}} = S_{h}S_{\bar{g}}A S_{g}S_{h}^{-1}.
	\end{equation}
	It follows that $\Kern{A + i(\beta - \alpha)} = S_{h}S_{\bar{g}}(\Kern A) \neq \{0\}$, hence $\im (\beta - \alpha) \in P\sigma(A)$.
	
	\item[(v)] 
	If $Af = \lambda f$ where $f > 0$, then
	\begin{equation}\label{eq:b3-3.7}
	\lambda\langle f,\phi\rangle = \langle Af,\phi\rangle = \langle f,A'\phi\rangle = 0.
	\end{equation}
	Since $\phi$ is strictly positive we have $\langle f,\phi\rangle > 0$ hence $\lambda = 0$.
	
	\item[(iii)] 
	We already know that $Ah = \im\alpha h$ implies that $A|h| = 0$.
	It follows from Proposition~\ref{prop:b3-3.5}(ii) that $h$ is strictly positive, \ie $h$ has no zeros in $X$.
	By Proposition~\ref{prop:b3-3.5}(iv) $\Kern A$ is one-dimensional hence every
	eigenfunction corresponding to $0$ is the scalar multiple of a strictly positive function.
	If $Ah = \im\alpha h$, $h \neq 0$, $\alpha \neq 0$, we consider $\tilde{h}(x) \coloneq h(x)/|h(x)|$.
	Assuming that $\tilde{h}(X)$ is not dense in $\Gamma$, there exists a sequence of polynomials $(p_{n})_{n\in\N}$ such that
	%% --
	\begin{equation}\label{eq:b3-3.8}
		p_{n}(z) \to 1/z \text{ uniformly in } z \in \tilde{h}(X).
	\end{equation}
	%% --
	It follows that $h(x)\cdot p_{n}(\tilde{h}(x)) \to |h|(x)$ uniformly in $x \in X$.
	Obviously, $h\cdot p_{n}(h)$ is a linear combination of $h^{[1]}$, $h^{[2]}$, $h^{[3]}$, \ldots, that is, it is an element of
	%\[
	%\text{span }\{\bigcup_{k=1}^\infty\,\Kern{ \im k\alpha - A)\}} \mathrm{(cf.\ Theorem~\ref{thm:b3-2.4})}.
	%\]
	\[
	\text{span $\{\Kern{ \im k\alpha - A} \colon  k=0,1,2,\ldots \}$}, \  \text{(cf.\ Theorem~\ref{thm:b3-2.4})}.
	\]
	By \eqref{eq:b3-3.7} the linear form $\phi$ vanishes on $\Kern{\lambda - A}$ whenever $\lambda \neq 0$.
	Therefore $\langle h\cdot P_{n}(h),\phi\rangle = 0$ and we have $0 < \langle|h|,\phi\rangle = \lim_{n\to\infty}\langle h\cdot P_{n}(h),\phi\rangle = 0$ which is a contradiction.
	\end{enumerate}
%%
\end{proof}
%% --
The group $P\sigma(A) \cap \im\R$ need not be discrete. 
For example, the semigroup described in Example~\ref{ex:b3-2.6}(iv) satisfies the assumptions of Theorem~\ref{thm:b3-3.6} if $\frac{\alpha}{\beta}$ is irrational. 
In this case $P\sigma(A) = \im\alpha \Z + \im\beta \Z$ is a dense subgroup of $\im \R$. 
Actually one can show that for every subgroup $H$ of $\im \R$ there is an irreducible semigroup on $C(G), G \coloneqq (H_d)\,{\tilde{ }}$, such that $P\sigma(A) = H)$. 
Here $(H_d)\,{\tilde{ }}$ denotes the dual of the abelian group $H$ equipped with the discrete topology. 
For details see \citet[p.62]{greiner:1982}. 

An immediate consequence of assertion (iv) of Theorem~\ref{thm:b3-3.6} is the following corollary. 
%% --
\begin{corollary}\label{cor:b3-3.7}
%
%\index{Semigroups!Irreducible!Spectrum}
Suppose $\TT$ satisfies the hypotheses of Theorem~\ref{thm:b3-3.6} and let $A$ be its generator.
If $k$ is a bounded continuous real-valued function, $M_{k}$ the corresponding multiplication operator, then for $B \coloneq A + M_{k}$ we have $\sigma(B) + \left(P\sigma(A)\cap \im\R\right) = \sigma(B)$.
In particular, $s(B) + \left(P\sigma(A)\cap \im\R\right) \subset \sigma(B)$.
\end{corollary}
%% --
The next two corollaries are essentially consequences of assertion (iii) of Theorem~\ref{thm:b3-3.6}. 
The statement of the first one can be summarized as follows. 
In case there are non-real eigenvalues in the boundary spectrum, then 
the semigroup \emph{contains} the semigroup of rotations on $\Gamma$.


\begin{corollary}\label{cor:b3-3.8}%
%\index{Semigroups!Irreducible!Rotations}
Suppose that the hypotheses of Theorem~\ref{thm:b3-3.6} are satisfied and that there is an eigenvalue $\im \alpha$ of $A$ with $\alpha > 0$.
Let $\tau \coloneq 2\pi/\alpha$.
Then there exists a continuous injective lattice homomorphism $j \colon C(\Gamma) \to C_{0}(X)$ such that the diagram commutes. 
\[
\begin{tikzcd}
C(\Gamma) \arrow[r, "R_{\tau}(t)"] \arrow[d, "j"'] & C(\Gamma) \arrow[d, "j"] \\
C_{0}(X) \arrow[r, "T(t)"'] & C_{0}(X)
\end{tikzcd}
\]
%% --
Here, $(R_{\tau}(t))$ denotes the rotation semigroup of period $\tau$ (see A-I,2.5).
If $X$ is compact, then $j$ is a topological embedding.
\end{corollary}
%% --
\begin{proof}
	Assume that $Ah = \im\alpha h$, $\alpha > 0$, and let $\tilde{h}(x) \coloneq h(x)/|h(x)|$.
	Then we define $j$ by
	%% --
	\begin{equation}\label{eq:b3-3.9}
	j(f) \coloneq |h|\cdot f\circ\tilde{h} \quad \left(\text{\ie\ }(j(f))(x) = |h(x)|\cdot f(\tilde{h}(x))\right).
	\end{equation}
	%% --
	Obviously, $j$ is a lattice homomorphism and because $h$ has no zeros and $\tilde{h}$ has a dense image in $\Gamma$ (Theorem~\ref{thm:b3-3.6}(iii)), it follows that $j$ is injective.
	For the functions $e_{n} \in C(\Gamma)$ given by $e_{n}(z) = z^{n}$ $(n \in \Z)$ one has $j(e_{n}) = h^{[n]}$ $(n \in \Z)$ and therefore by Theorem~\ref{thm:b3-2.4}
	%% --
    \[ 
    T(t)\circ j(e_{n}) = T(t)h^{[n]} = \mathrm{e}^{\im n\alpha t}\cdot h^{[n]} 
    \ \text{ and } \ 
    j\circ R_{\tau}(t)(e_{n}) = j(\mathrm{e}^{\im n\alpha t}e_{n}) = \mathrm{e}^{\im n\alpha t}\cdot h^{[n]}.
    \]
    %% --
    Since $\{e_{n} \colon n \in \Z\}$ is a total subset of $C(\Gamma)$, we have $T(t)\circ j = j\circ R_{\tau}(t)$ for every $t > 0$.
	
	If $X$ is compact, then $\tilde{h}(X)$ is closed, hence $\tilde{h}$ is onto, moreover, $|h| \geq \epsilon$ for some $\epsilon > 0$, thus the definition of $j$ implies that $\|j(f)\| > \epsilon\|f\|$ for every $f \in C(\Gamma)$.
\end{proof}
%% --
A consequence of Corollary~\ref{cor:b3-3.8} is the following. 
There exist positive functions $f_{1}$ and $f_{2}$ such that $T(t)f_{1}$ and $T(t)f_{2}$ have disjoint support for every $t \geq 0$ (consider the images under $j$ of two disjoint functions on $C(\Gamma)$).
This observation proves the following corollary.
%% --
\begin{corollary}\label{cor:b3-3.9}
%%
%\index{Semigroups!Irreducible!Properties}
	Suppose that the hypotheses of Theorem~\ref{thm:b3-3.6} are satisfied and that for some $t_{0} > 0$ we have $T(t_{0})f \gg 0$ whenever $f > 0$.
	Then $P\sigma(A)\cap \im\R = \{0\}$.
\end{corollary}
%% --
Corollary~\ref{cor:b3-3.9} can be applied if $T(t_{0})$ is a kernel operator with strictly positive kernel.
We give some examples.

\begin{examples}\label{ex:b3-3.10}
%
%\index{Examples!Irreducible Semigroups}
	
	\begin{enumerate}[\upshape (i), wide, labelindent=.5em] 
	\item 
	We assume that the semigroup $(T(t))$ satisfies the hypotheses of Theorem~\ref{thm:b3-3.6} and that it is given by
	\[
	(T(t)f)(x) = \int_{X} k(t,x,y)f(y) \diff{\mu}(y)\,,
	\]
	where $\mu$ is a positive measure and $k$ is a positive continuous function (see Example~\ref{ex:b3-3.4}(ii)).
	We will show that $P\sigma(A)\cap(s(A) + \im\R) = \{s(A)\}$.
	
	Assuming the contrary, by Theorem~\ref{thm:b3-3.6}(iv) there exist $\alpha \neq 0$, $h \in C_{0}(X)$ such that
	%% --	
	\begin{equation}\label{eq:b3-3.10}
	S_{h}^{-1}T(t)\circ S_{h} = \mathrm{e}^{\im\alpha t}\cdot T(t) \text{ for all } t \geq 0.
	\end{equation}
	%% --
	This implies that $k$ satisfies
	%% --	
	\begin{equation}\label{eq:b3-3.11}
	\frac{\overline{h(x)}}{|h(x)|}\cdot\frac{h(y)}{|h(y)|}\cdot k(t,x,y) = \mathrm{e}^{\im\alpha t}k(t,x,y) \quad (t > 0, x,y \in X).
	\end{equation}
	%% --
	It follows that for $0 < |s-t| < 2\pi/\alpha$ the kernel functions  $k(t,.,.)$ and $k(s,.,.)$ have disjoint support.
	This is impossible if $k$ is continuous.
	
	\item 
	Let $\Omega$ be a domain in $\R^{n}$ and define $L_{0}$ as
	\[
	L_{0}f \coloneq \sum_{i,j=1}^{n} a_{ij}f_{ij}'' + \sum_{i=1}^{n}b_{i}f_{i}' + cf\,,
	\]
	with domain $D(L_{0}) \coloneq \{f \in C_{0}(\Omega) \colon f \text{ is } C^{\infty}, L_{0}f \in C_{0}(\Omega)\}$.
	(Here $f_{i}'$ stands for $\partial f/\partial x_{i}$, and $f_{ij}'' = \partial^{2}f/\partial x_{i}\partial x_{j}$).
	
	Suppose that $L_{0}$ is elliptic, $a_{ij}$, $b_{i}$, $c$ are real-valued $C^{\infty}$-functions and $\sup_x c(x) < \infty$, assume further that the closure $L$ of $L_{0}$ is the generator of a positive semigroup on $C_{0}(\Omega)$ which has compact resolvent.
	For example, this is true if the boundary  $\partial\Omega$ is $C^{\infty}$ and $a_{ij} \in C^{\infty}(\overline{\Omega})$ (cf.\ Theorem~4.8.3 of \citet{fattorini:1983}).
	We will show that $P\sigma(A)\cap(s(A) + \im\R) = \{s(A)\}$.
	
	In order to apply Theorem~\ref{thm:b3-3.6}, we have to show that the corresponding semigroup $(T(t))$ is irreducible.
	Given $0 < f \in C_{0}(X)$, then there is $g \in D(L_{0})$ such that $0 < g \leq f$.
	$h \coloneq R(\lambda,L)g$ is $C^{\infty}$ (Weyl's Lemma) and satisfies $L_{0}h - \lambda h = -g < 0$.
	Assuming that $\lambda > \lambda_{0} \coloneqq \sup_x c(x)$, then $h$ is positive, even strictly positive by the maximum principle \citet[Chapter~2, Theorem~6]{protterweinberger:1967}.
	It follows from $R(\lambda,L)f \geq R(\lambda,L)g = h \gg 0$ that $(T(t))$ is irreducible.
	
	Next we apply Theorem~\ref{thm:b3-3.6}(iv) in order to show that the spectral bound is a dominant eigenvalue.
	We can assume that $s(L) = 0$.
	If $s(L)$ is not dominant, then by Theorem~\ref{thm:b3-3.6}(iv) we have
	
	\begin{equation}\label{eq:b3-3.12}
		L_0h = i \alpha h, L_0|h| = 0, L_0\bar{h} = -\im\alpha \bar{h} 
		\text{ for some } h \ne 0, \alpha > 0.
	\end{equation}
%% --
If we define $u \coloneq |h|$ and $w \coloneq h/|h|$, then \eqref{eq:b3-3.12} reads
	 \begin{equation}\label{eq:b3-3.13}
	L_{0}(uw) = \im\alpha uw, \quad L_{0}(u) = 0, \quad L_{0}(u/w) = -\im\alpha\cdot u/w .
	\end{equation}
	Explicit calculation of $L_{0}(uw)$ and $L_{0}(u/w)$ using the product formula yields
		\begin{equation}\label{eq:b3-3.14}
		\begin{split}
		L_{0}(uw) &=  wL_{0}(u) + u\sum_{i,j} a_{ij}w_{ij}'' + \sum_{i}\left(ub_{i} + \sum_{j} a_{ij}u_{j}'\right)w_{i}'~,\\
		L_{0}(u/w) &= \frac{1}{w}L_{0}(u) + u\sum_{i,j}a_{ij}(1/w)_{ij}'' + \sum_{i}\left(ub_{i} + \sum_{j} a_{ij}u_{j}'\right)(1/w)_{i}'~.
		\end{split}
	\end{equation}
	Observing that $(1/w)_{i}' = -w^{-2}\cdot w_{i}'$ and $(1/w)_{ij}'' = w^{-3}\cdot(2w_{i}'w_{j}' - ww_{ij}'')$, we obtain
		\begin{equation}\label{eq:b3-3.15}
			L_{0}(uw) + w^{2}L_{0}(u/w) = 2wL_{0}(u) + 2u/w\cdot\sum_{ij} a_{ij}w_{i}'w_{j}'~.
		\end{equation}
	This identity and \eqref{eq:b3-3.13} implies that $2u/w\cdot\sum_{ij} a_{ij}w_{i}'w_{j}' = 0$.
	Since $u$ has no zeros and $(a_{ij})$ is positive definite, we have $\text{grad } w = (w_{i}') = 0$ in $\Omega$, hence $w = \text{const}$.
	Then, by \eqref{eq:b3-3.13}, we have $\im \alpha uw = L_{0}(uw) = wL_{0}(u) = 0$, a contradiction.

	The assumption that $L_{0}$ is elliptic, \ie that $(a_{ij})$ is positive definite, is essential in order to show that there is only one eigenvalue in the boundary spectrum.
	In the following example $(a_{ij})$ is positive semi-definite and $P\sigma_{b}(A) = s(A) + \im\alpha\Z$.
	
	\item 
	We consider $\Omega = \{(x,y) \in \R^{2} \colon 1 < (x^{2} + y^{2})^{1/2} < 2\}$, and the second order differential operator $L_{0}$ given by
	\[
	(L_{0}f)(x,y) = 1/(x^{2} + y^{2})\cdot(x^{2}f_{xx} + 2xyf_{xy} + y^{2}f_{yy}) + (xf_{y} - yf_{x}).
	\]
	The assertion concerning the boundary spectrum can be verified easily by using polar coordinates, $x = r\cdot\cos\omega$, $y = r\cdot\sin\omega$.
	Then $L_{0}$ becomes $L_{0}f = f_{rr} + f_{\omega}$ on the space $C_{0}(1,2) \otimes C_{2\pi}(\R)$.
	\end{enumerate}
\end{examples}
%% --
In this section we have seen that the eigenvalues in the boundary spectrum of an irreducible semigroup form a subgroup of $\im \R$ (provided that $s(A) = 0$).
We conclude this section mentioning an analogous statement for the whole boundary spectrum of Markov semigroups on $C(K)$, $K$ compact.
It seems to be unknown if this is true for irreducible semigroups in general.
To prove this result one uses the proof of the analogous result for a single operator (cf.\ \citet[Theorem~7]{schaefer:1968}) as a guideline.
%% --
\begin{theorem}\label{thm:b3-3.11}
%%
%\index{Semigroups!Irreducible!Markov}
	Suppose that $\TT$ is an irreducible semigroup of Markov operators on $C(K)$, $K$ compact.
	Then $\sigma_{b}(A)$ is a (closed) subgroup of $\im \R$.
	Hence either $\sigma_{b}(A) = \{0\}$ or $= \im\R$ or $= \im\alpha\Z$ for some $\alpha > 0$.
\end{theorem}

\section{Semigroups of Lattice Homomorphisms}%
%\index{Spectral Theory on $ C_{0}(X)$!Lattice Homomorphisms}
%% --
As we have seen in Section 2 the boundary spectrum of many positive semigroups is a cyclic set.
However, there are hardly any restrictions on the set 
\[
\lambda \in \sigma(A) \colon \Re\lambda < s(A)\}\,,
\] 
except that it is symmetric with respect to the real axis.
For semigroups of lattice homomorphisms the situation is quite different.
We will show that the whole spectrum is an imaginary additively cyclic subset of $\C$ (see Definition~2.5).
A complete proof of this results requires some facts of the theory of Banach lattices, therefore, we postpone it to Part C (see C-III, Theorem~4.2).
%% --
\begin{theorem}\label{thm:b3-4.1}
%
%\index{Semigroups!Lattice Homomorphisms!Spectrum}
	If $A$ is the generator of a semigroup of lattice homomorphisms, then $\sigma(A)$, $A\sigma(A)$ and $P\sigma(A)$ are cyclic subsets of $\C$.
\end{theorem}
%% --
\begin{proof}[$1^{st}$ part of the proof]
	We prove the assertion concerning $A\sigma(A)$ and $P\sigma(A)$.
	Assume that $Ah = (\alpha + \im\beta)h$, $\alpha,\beta \in \R$, $h \neq 0$, then $T(t)h = \mathrm{e}^{\alpha t}\mathrm{e}^{\im\beta  t}h$ for all $t \geq 0$ (A-III,Corollary~6.4).
	Since $T(t)$ is a lattice homomorphism we have $T(t)|h| = |T(t)h| = \mathrm{e}^{\alpha t}|h|$ $(t \geq 0)$ or $A|h| = \alpha|h|$, hence $Ah^{[n]} = (\alpha + \im n\beta)h^{[n]}$ for all $n \in \Z$ by Theorem~\ref{thm:b3-2.4}(ii).
	We have shown that $P\sigma(A)$ is cyclic.
	
	To prove that $A\sigma(A)$ is cyclic as well, one considers a semigroup $\mathcal{F}$-product $E_{\mathcal{F}}^{\TT}$ of $E$ (see A-III,4.5).
	It is easy to see that $E_{\mathcal{F}}^{\TT}$ is a Banach lattice 
	and $(T_{\mathcal{F}}(t))$ is a semigroup of lattice homomorphisms.
	The proposition in A-III,3.5 implies $A\sigma(A) = P\sigma(A_{\mathcal{F}})$.
	Thus the assertion follows from the cyclicity of point spectrum.
\end{proof}
%% --
Performing a similar construction as in Example~\ref{ex:b3-2.6}(vi) one can show that every closed cyclic subset of $\C$  which is contained in a left halfplane is the spectrum of a suitable semigroup of lattice homomorphisms. 
For details see \citet{derndingernagel:1979}. 
In the following we restrict ourselves to the case of compact spaces. 
Then a semigroup of lattice homomorphisms can be described explicitly by a semi-flow $\phi$, and real-valued functions $h$ and $p$ (see B-II, Thms.3.5 \& 3.6). The function $p$ has no influence on spectral properties (cf.\ B-II,(3.7)). 
Therefore we will assume that $(T(t))$ has the following form (cf.\ B-II, Theorem~3.5). 
%% --
\begin{equation}\label{eq:b3-4.1}
	T(t)f = h_{t}\cdot f\circ\phi_{t} \quad (t \geq 0, f \in C(K)) 
\end{equation}
where $\phi = (\phi_{t}): \R_{+} \times K \to K$ is a continuous semiflow and multipliers $h_{t}(x) \coloneq \exp\left(\int_{0}^{t} h(\phi(s,x)) \ds\right)$ $(t \geq 0, x \in K)$ for some continuous function $h \colon K \to \R$.
In the following we will describe the spectrum of the semigroup given by \eqref{eq:b3-4.1} in terms of $\phi$ and $h$.
At first we have to fix some notation.
Let $K$, $\phi$, $h$ be as in \eqref{eq:b3-4.1}. 
Then
\begin{equation}\label{eq:b3-4.2}
K_{t} \coloneq \phi_{t}(K) \quad (t < \infty), \quad K_{\infty} \coloneq \bigcap_{t < \infty} K_{t}.
\end{equation}
Some properties of the sets $K_t$  are listed in the following lemma.
The proof is not difficult and is left as an exercise. 
%% --
\begin{lemma}\label{lem:b3-4.2}
%
%\index{Semiflows!Properties}
	Every $K_{t}$ $(0 \leq t \leq \infty)$ is a non-empty closed subset of $K$ which is invariant under the semiflow $\phi$.
	Moreover, the following assertions are true.
	\begin{enumerate}[\upshape (i)]
		\item 
		For $s > t$ we have $K_{s} \subset K_{t}$.
		In case that $K_{s} = K_{t}$ then $K_{t} = K_{\infty}$.
		
		\item
		$\phi_{t}(K_{\infty}) = K_{\infty}$ for all $t \geq 0$.
		
		\item 
		If one partial mapping $\phi_{t}$, $t > 0$, is injective (surjective), then all mappings $\phi_{s}$ are injective (surjective).
	\end{enumerate}
\end{lemma}
%% --
We call a \emph{semiflow} $\phi$ \emph{injective (surjective)} if one and hence all of the partial mappings $\phi_{t}$ are injective (surjective).
Studying the spectrum of the semigroup given by \eqref{eq:b3-4.1} we divide the complex plane into these three sets
%% --
\begin{equation}\label{eq:b3-4.3}
	\begin{aligned}
		&\{\lambda \in \C \colon \Re\lambda < \underline{c}(h,\phi)\}\,, \\
		&\{\lambda \in \C \colon \underline{c}(h,\phi) \leq \Re\lambda \leq \overline{c}(h,\phi)\}\,, \\
		&\{\lambda \in \C \colon \overline{c}(h,\phi) < \Re\lambda\}.
	\end{aligned}
\end{equation}
%% --
The quantities $\underline{c}(h,\phi)$ and $\overline{c}(h,\phi)$ are defined as follows.
\begin{align}\label{eq:b3-4.4}
		&\overline{c}(h,\phi) \coloneq \lim_{t\to\infty} \overline{c}_{t}(h,\phi) = \inf_{t>0} \overline{c}_{t}(h,\phi) \text{ where } 
		\overline{c}_{t}(h,\phi) \coloneq \sup_{x\in K} \{\frac{1}{t}\int_{0}^{t} h(\phi(s,x)) \ds \},\notag
        \\& \\ \notag
		&\underline{c}(h,\phi) \coloneq \lim_{t\to\infty} \underline{c}_{t}(h,\phi) = \sup_{t>0} \underline{c}_{t}(h,\phi) \text{ where } 
		\underline{c}_{t}(h,\phi) \coloneq \inf_{x\in K} \{\frac{1}{t}\int_{0}^{t} h(\phi(s,x)) \ds.
\end{align}
%% --
It is easy to see that $\overline{c}_{t}(h,\phi) = 1/t\cdot\log\|T(t)\|$, hence in the definition of $\overline{c}(h,\phi)$, both the limit and the infimum exist and coincide with the growth bound (see A-I,(1.1)).
Furthermore, $\underline{c}_{t}(h,\phi) = -\overline{c}_{t}(-h,\phi)$.
Therefore, $\underline{c}(h,\phi)$ is well defined too.

First we will describe the part of $\sigma(A)$ which is contained in the left half-plane determined by $\underline{c}(h,\phi)$.
It turns out that either the whole half-plane is contained in $\sigma(A)$ or it has empty intersection with $\sigma(A)$.
This depends only on properties of $\phi$.
Essentially there 
are three different cases. 
Before we state the general result (see Theorem~\ref{thm:b3-4.4}) we give some typical examples.
%% 
\begin{examples}\label{ex:b3-4.3}%
%\index{Examples!Lattice Homomorphisms}
	\begin{enumerate}[\upshape (i), wide, labelindent=.5em]
	\item 
	Consider on $K = [0,\infty]$ the semiflow defined by $\phi(t,x) \coloneq x + t$ $(\infty + t = \infty)$.
	Then $K_{t} = [t,\infty]$ and $K_{\infty} = \{\infty\}$.
	The spectrum of the corresponding semigroup $T(t)f = f\circ\phi_{t}$ is $\sigma(A) = A\sigma(A) = \{\lambda \in \C \colon \Re\lambda \leq 0\}$.
	
	\item 
	Consider again $K = [0,\infty]$ and define a semiflow by
	\[
	\phi(t,x) \coloneq \begin{cases} 
		x - t & \text{if } x \geq t \,,\\
		0 & \text{if } x < t .
	\end{cases} \quad (\infty - t = \infty)
	\]
	Then $K_{t} = K$ for all $t$, hence $K_{\infty} = K$ and $\sigma(A) = \{\lambda \in \C \colon \Re\lambda \leq 0\}$, $R\sigma(A) = \{\lambda \in \C \colon \Re\lambda < 0\} \cup \{0\}$.
	
	\item 
	Consider on $K_{1} \coloneq [-1,\infty)$ the equivalence relation $\sim$ defined by \enquote{$x \sim y$ if and only if $x,y \geq 0$ and $x - y \in \Z$}.
	The semiflow $\phi_{1}$ on $K_{1}$ given by $\phi_{1}(t,x) = x + t$ induces a semiflow $\phi$ on $K \coloneq K_{1}/{\sim}$.
	For $0 < t < 1$ we have $K \neq K_{t} \neq K_{\infty}$ $(K_{\infty} = [0,1]/{\sim} \cong \Gamma)$.
	The spectrum is $\sigma(A) = 2\pi \im\Z$.
	
	\item 
	Consider on $K = [-1,1]$ the flow $\phi$ given by
	\[
	\phi(t,x)\coloneq \begin{cases}
		-1 & \text{if } x < 0 \text{ and } t > -\frac{x+1}{x} \,, \\
		\frac{x}{1+tx} & \text{otherwise} .
	\end{cases}
	\]
	Then $K_{t} = [-1,\frac{1}{1+t}]$, $K_{\infty} = [-1,0]$ and
	\[
	\sigma(A) = \{\lambda \in \C \colon \Re\lambda \leq 0\},\quad 
	\{\lambda \in \C \colon \Re\lambda < 0\} \subsetneq A\sigma(A)\cap R\sigma(A).
	\]
	\end{enumerate}
\end{examples}
%% --
\begin{theorem}\label{thm:b3-4.4}
	Suppose $\TT$ is a semigroup of lattice homomorphisms given by \eqref{eq:b3-4.1} with generator $A$.
	Considering $H \coloneq \{\lambda \in \C \colon \Re\lambda < \underline{c}(h,\phi)\}$, where $\underline{c}(h,\phi)$ is given by \eqref{eq:b3-4.4}, then we have.
	\begin{enumerate}[\upshape (i)]
		\item 
		If $K_{t} \neq K_{\infty}$ for every $t < \infty$, then $H \subseteq A\sigma(A)$.
	
		\item 
		If $\phi_{|K_{\infty}}$ is not injective, then $H \subseteq R\sigma(A)$.
	
		\item 
		If $K_{s} = K_{\infty}$ for some $s < \infty$ and $\phi_{|K_{\infty}}$ is injective, then $H\cap\sigma(A) = \emptyset$.
	\end{enumerate} 
\end{theorem}
%% --
\begin{proof}
	For $\epsilon > 0$ we define $H_{\epsilon} = \{\lambda \in \C \colon \Re\lambda < \underline{c}(h,\phi)-\epsilon\}$.
	Obviously it is enough to prove assertion (i), (ii) and (iii) respectively for $H_{2\epsilon}$, $\epsilon> 0$ arbitrary, instead of $H$.

	(i) By the definition given in \eqref{eq:b3-4.1} there exists a $\tau > 0$ such that $\underline{c}_{t}(h,\phi) \geq \underline{c}(h,\phi) - \epsilon$ for all $t \geq \tau$.
	It follows that
	%% --
	\begin{equation}\label{eq:b3-4.5}
		h_{t}(x) \geq \mathrm{e}^{(\alpha + \epsilon)t} \text{ whenever } t \geq \tau, x \in K, \alpha < \underline{c}(h,\phi)-2\epsilon .
	\end{equation}
	%% --
	Now we fix $\lambda = a + \im\beta \in H_{2\epsilon}$ $(a,\beta \in \R)$ and construct an approximate eigenvector $(g_{n})$ of $A$ corresponding to $\lambda$.
	For $n \leq \tau + 1$ we choose an arbitrary function $g_{n} \neq 0$.
	Now suppose $n > \tau + 1$.
	We choose $x_{n} \in K_{n+1/2} \setminus K_{n+1}$ (cf.\ Lemma~\ref{lem:b3-4.2}(i)), then there exists $y_{n} \in K$ such that $\phi(n+1/2,y_{n}) = x_{n}$.
	We have $\phi([0,n+1/2],y_{n}) \cap K_{n+1} = \emptyset$ and the mapping $t \mapsto \phi(t,y_{n})$ is a continuous injection, hence a homeomorphism from $[0,n+1/2]$ into $K$ (this is true because $\phi(n+1/2,y_{n}) \notin K_{n+1}$).
	By Tietze's Theorem there is $f_{n} \in C(K)$ such that
	%% --
	\begin{equation}\label{eq:b3-4.6}
		\begin{aligned}
			\|f_{n}\| \leq 1 &, \quad {f_{n}|}_{K_{n+1}} = 0\,, \\
			f_{n}(\phi(t,y_{n})) &= 0 \text{ for } 0 \leq t \leq n-(1+\delta) \text{ and } n+\delta \leq t \leq n+1\,, \\
			f_{n}(\phi(t,y_{n})) &= \mathrm{e}^{\im\beta  t} \text{ for } n-1 \leq t \leq n .
		\end{aligned}
	\end{equation}
	%% --
	The constant $\delta \in (0,1/2)$ will be determined later.
	
	Considering $g_{n}\coloneq \int_{0}^{n+1} \mathrm{e}^{-\lambda t} T(t)f_{n} \dt$, then $g_{n} \in D(A)$ and
	%% --
	\begin{equation}\label{eq:b3-4.7}
		(A - \lambda)g_{n} = (1 - \mathrm{e}^{-\lambda(n+1)}T(n+1))f_{n} = f_{n} - \mathrm{e}^{-\lambda(n+1)} \cdot h_{n+1} \cdot f_{n} \circ \phi_{n+1} = f_{n}.
	\end{equation}
	%% --
	Moreover,
	%% --
	\[\|g_{n}\| \geq |g_{n}(y_{n})| = \left|\int_{0}^{n+1} \mathrm{e}^{-\lambda t} h_{t}(y_{n})f_{n}(\phi(t,y_{n})) \dt \right| \geq\]
	%% --
	\[\left|\int_{n-1}^{n} \mathrm{e}^{-\lambda t} h_{t}(y_{n}) \mathrm{e}^{\im\beta  t} \dt\right| - \left[\int_{n-(1+\delta)}^{n-1} + \int_{n}^{n+\delta}\left| \mathrm{e}^{-\lambda t}|h_{t}(y_{n})f_{n}(\phi(t,y_{n}))\right| \dt\right]\]
	%% --
	\[\geq \int_{n-1}^{n} \mathrm{e}^{-at} \mathrm{e}^{(a+\epsilon)t} \dt - \left[\int_{n-(1+\delta)}^{n-1} + \int_{n}^{n+\delta} \mathrm{e}^{-at}|h_{t}(y_{n})| \dt\right]\]
	%% --
	\[= 1/\epsilon \cdot\left(\mathrm{e}^{\epsilon n} - \mathrm{e}^{\epsilon(n-1)}\right) - \left[\int_{n-(1+\delta)}^{n-1} + \int_{n}^{n+\delta} \mathrm{e}^{-at}|h_{t}(y_{n})| \dt\right].\]
	The constant $\delta$ can be chosen such that
	%% --
	\begin{equation}\label{eq:b3-4.8}
		\|g_{n}\| \geq 1/2\epsilon \cdot\left(\mathrm{e}^{\epsilon n} - \mathrm{e}^{\epsilon(n-1)})\right) \to \infty \text{ for } n \to \infty.
	\end{equation}
	%% --
	It follows from \eqref{eq:b3-4.8} and \eqref{eq:b3-4.7} that $ g_n / \|g_n\|$ is an approximate
	eigenvector of $A$ corresponding to $\lambda$. 
    Thus (i) is proved. 
	
	The 
	proofs of (ii) and (iii) will be handled simultaneously. 
    First we show that we can restrict attention to the case where $K = K_\infty$.
	
	Indeed, $K_{\infty}$ is a $\phi$-invariant subset, hence $I_{\infty}\coloneq \{f \in C(K) \colon f|_{K_{\infty}} = 0\}$ is a $T$-invariant ideal.
	Identifying $C(K)/I_{\infty}$ with $C(K_{\infty})$ (cf.\ B-I, Section~1), then $(T(t)_{/I_{\infty}})$ is the semigroup governed by $\phi_{|K_{\infty}}$ and $h_{|K_{\infty}}$.
	Since one always has $R{\sigma}(A_{/}) \subseteq R{\sigma}(A)$, assertion (ii) is proved.
	when we can show that $H_{2\epsilon} \subseteq R{\sigma}(A_{/})$.
	In case (iii) one has $K_{s} = K_{\infty}$ for some $s < \infty$, which implies $T(s)_{|I_{\infty}} = 0$.
	Hence we have $\sigma(A_{|I_{\infty}}) = \emptyset$ and therefore $\sigma(A) = \sigma(A_{/I_{\infty}})$ by A-III, Proposition~4.2.
	
	Henceforth we will assume that $K = K_{\infty}$, that is, $\phi$ is surjective (cf.\ Lemma~\ref{lem:b3-4.2}(ii)).
	
	We choose $\tau > 0$ such that \eqref{eq:b3-4.5} is true.
	Since $\phi$ is surjective, for every $f \in C(K)$ there is a $x_{f} \in K$ such that $\|f\| = \|f(\phi(\tau,x_{f}))\|$ and we obtain for $\lambda \in H_{2\epsilon}$, $\lambda = a + \im\beta$, $\alpha,\beta \in \R$
	%% --
	\begin{equation}\label{eq:b3-4.9}
		\begin{aligned}
			\|(\mathrm{e}^{\lambda\tau} - T(\tau))f\| &\geq |h_{\tau}(x_{f})f(\phi(\tau,x_{f})) - \mathrm{e}^{\lambda\tau}f(x_{f})| \\
			& \geq h_{\tau}(x_{f})\|f\| - \mathrm{e}^{a\tau}|f(x_{f})| 
			 \geq \mathrm{e}^{(a+\epsilon)\tau}\|f\| - \mathrm{e}^{a\tau}\|f\| \\
			&= \mathrm{e}^{a\tau}(\mathrm{e}^{\epsilon\tau} - 1)\|f\|.
		\end{aligned}
	\end{equation}
	%% --
	It follows that the disc $D \coloneq \{\lambda \in \C \colon |\lambda| < \exp(\underbar{c}(h,\phi)-2\epsilon)\}$ has an empty intersection with $A{\sigma}(T(\tau))$ and therefore $H_{2\epsilon}\cap A_{\sigma}(A) = \emptyset$ by A-III,6.2.
	Since every boundary point of the spectrum is an approximate eigenvalue (by A-III, Proposition~2.2(i)) we have the following alternative.
	%% --
	\begin{equation}\label{eq:b3-4.10}
		\begin{aligned}
		&\text{Either } D \subseteq \rho(T(\tau)) \text{ and } H_{2\epsilon} \subseteq \rho(A) \\
		&\text{ or } D \subseteq R{\sigma}(T(\tau)) \text{ and } H_{2\epsilon} \subseteq R{\sigma}(A).
	\end{aligned}
	\end{equation}
	%% --
	It is not difficult to see that $0 \in \rho(T(\tau))$ whenever $\phi_{t}$ is bijective and that $0$ is an eigenvalue of $T(\tau)'$ if $\phi_{\tau}$ is not injective.
	Since we assumed that $\phi$ is surjective, assertions (ii) and (iii) of the theorem are immediate consequences of \eqref{eq:b3-4.10}.
\end{proof}
%% --
The Examples~\ref{ex:b3-4.3}(i), (ii) and (iii) respectively are prototypes of the three different cases considered in Theorem~\ref{thm:b3-4.4}.
Example~\ref{ex:b3-4.3}(iii) also shows that there are semigroups whose spectrum is contained in a right half-plane, although they cannot be embedded in a group (compare Corollary~\ref{cor:b3-4.5} below).
Example~\ref{ex:b3-4.3}(iv) shows that (i) and (ii) do not exclude each other.
%% --
\begin{corollary}\label{cor:b3-4.5}
	If $\phi$ is injective or surjective, then the following assertions are equivalent.
	\begin{enumerate}[\upshape (a)]
		\item 
		$A$ is the generator of a strongly continuous group.
		
		\item 
		$\sigma(A)$ is contained in a right half-plane.
	\end{enumerate}
\end{corollary}
%% --
\begin{proof}
	$(a)\implies(b)$:\ \  holds true because $-A$ is a generator of a semigroup.
	
	$(b)\implies(a)$:\ \ 
	We have to show that one (hence each) operator $T(t)$, $t > 0$ is invertible.
	Obviously this is true if $\phi$ is bijective.
	At first we assume that $\phi$ is surjective, that is, $K = K_{\infty}$.
	By Theorem~\ref{thm:b3-4.4} we have that $\phi|_{K_{\infty}}$ is injective if (b) is true.
	Thus $\phi$ is bijective.
	Now we assume that $\phi$ is injective.
	We have to show that $K = K_{\infty}$.
	By Theorem~\ref{thm:b3-4.4} we have $K_{\infty} = K_{s}$ for some $s$, whenever (b) is true.
	Given $x \in K$ then by Lemma~\ref{lem:b3-4.2}(ii) there exists $y \in K_{\infty}$ such that $\phi(s,x) = \phi(s,y)$.
	If $\phi$ is injective we have $x = y \in K_{\infty}$.
\end{proof}
%% --
\begin{example}\label{ex:b3-4.6}
	Suppose $F \colon \R^n \to \R^n$ is continuously differentiable.
	We denote the maximal flow corresponding to the differential equation $y' = F(y)$ by $\phi_{0}$.
	In general, $\phi_{0}$ is only defined on an open subset of $\R \times \R^n$ which contains $\{0\} \times \R^n$.
	For $x \in \R^n$ there exist $\underline{t}_{x}$ and $\overline{t}_{x}$ such that
	%% --
	\begin{equation}\label{eq:b3-4.11}
		\begin{aligned}
			&-\infty \leq \underline{t}_{x} < 0 < \overline{t}_{x} \leq \infty\,, \\
			&\phi_{0}(t,x) \text{ is defined if } \underline{t}_{x} < t < \overline{t}_{x}\,, \\
			&\text{if } \overline{t}_{x} < \infty \text{ (}\underline{t}_{x} > -\infty\text{)} \text{ then } |\phi_{0}(t,x)| \to \infty \text{ as } t\uparrow\overline{t}_{x} \text{ (}t\downarrow\underline{t}_{x}\text{)}.
		\end{aligned}
	\end{equation}
	%% --
	For details see Sect.18.2 of \citet{dieudonne:1971}.
	\begin{enumerate}[\upshape (a), wide, labelindent=.5em]
	\item
	If $\phi_{0}$ is a global flow, \ie if $\phi_{0}$ is defined on $\R \times \R^n$, then one has a corresponding (semi-)group on $C_{0}(\R^n)$.
	If $F$ is differentiable, its generator is the closure of $A_{1}$ which is defined as follows (cf.\ B-II,Example~3.15).
	%% --
	\begin{equation}\label{eq:b3-4.12}
		\begin{aligned}
			A_{1}f = (F|\text{grad }f) \coloneq 
            {  \sum_i F_{i} \cdot \partial_{i}f} \,, \\
			D(A_{1}) \coloneq \{f \in C^1 \colon \supp(f)  \text{ is compact}\}.
		\end{aligned}
	\end{equation}
	%% --
	Then $\phi_{0}$ can be uniquely extended to a flow $\tilde{\phi}_{0}$ on $\R^n\cup\{\infty\}$ by defining $\tilde{\phi}_{0}(t,\infty) \coloneq \infty$ for all $t \in \R$.
	$\phi_{0}$ and $\tilde{\phi}_{0}$ satisfy condition (c) of Theorem~\ref{thm:b3-4.4}.

    A global flow exists if $F$ is globally Lipschitz continuous or uniformly bounded. 
    For $\{x \in \R^n \colon (x|F(x)) > 0\}$ bounded in $\R^n$, a global semiflow always exists (cf.\ \citet[Section~5.2]{deimling:1977}).
    
	\item 
	Now we do not asssume that  $\phi_0$ is globally defined.
	Instead we consider a bounded domain $\Omega \subset \R^n$ with smooth boundary $\partial\Omega$ such that $(F(x)|\nu(x)) > 0$ for every $x \in \partial\Omega$, where $\nu(x)$ denotes the outward normal vector.
	
	Then for $x \in \overline{\Omega}$ we have $\underline{t}_{x} = -\infty$.
	Moreover, either $\phi_{0}(t,x) \in \Omega$ for all $t \geq 0$ or else there exists a unique $s_{x}$ with $0 \leq s_{x} < \overline{t}_{x}$ such that $\phi_{0}(s_{x},x) \in \partial\Omega$.
	In the first case we write $s_{x} \coloneq \infty$.
	Then we define $\phi: \R_{+} \times \overline{\Omega} \to \overline{\Omega}$ as 
	%% --
	\[\phi(t,x) \coloneq \begin{cases}
		\phi_{0}(t,x) & \text{if } 0 \leq t < s_{x} \,, \\
		\phi_{0}(s_{x},x) & \text{if } t \geq s_{x} .
	\end{cases}\]
	%% --
	Then $\phi$ is a continuous semiflow on the compact set $K \coloneq \overline{\Omega}$.
	We have $K_{\infty} = K$ and $\phi_{|K_{\infty}}$ is not injective.
	
	In case $F$ is differentiable, the generator of the corresponding semigroup is the closure of the operator $A_{2}$ defined by
	%% --
	\[
	A_{2}f \coloneq (F|\text{grad }f), \quad D(A_{2}) \coloneq \{f \in C^1(\overline{\Omega}) \colon (F|\text{grad }f) = 0 \text{ on } \partial\Omega\}.	
	\]
	
	\item 
	We consider $\Omega$ as in (b) and assume that $(F(x)|\nu(x)) \leq 0$ for every $x \in \partial\Omega$.
	Then for every $x \in \overline{\Omega}$ we have $\overline{t}_{x} = \infty$.
	Thus $\phi \coloneq {\phi_{0}}_{|\R_{+}\times\overline{\Omega}}$ is a continuous semiflow on $K \coloneq \overline{\Omega}$.
	
	If $(F(x)|\nu(x)) < 0$ for some $x \in \partial\Omega$, we have $K_{t} \subsetneq K_{s}$ whenever $t > s$ and $\phi|_{K_{\infty}}$ is injective.
	For a differentiable vector field $F$, the generator of the corresponding semigroup is the closure of $A_{3}$ defined as
	%% --
	\[
	A_{3}f \coloneq (F|\text{grad }f), \quad D(A_{3}) \coloneq C^1(\overline{\Omega}).
	\]
	%% --
\end{enumerate}
\end{example}

We conclude the discussion of semi-flows associated with ordinary differential equations by remarking that the ideas of (b) and (c) can be combined to obtain semigroups for more general subsets $\Omega$.

We continue the discussion of the spectrum of semigroups of lattice homomorphisms on $C(K)$ given by \eqref{eq:b3-4.1}.
Theorem~\ref{thm:b3-4.4} gives a good description of the part contained in $\{\lambda \in \C \colon \Re\lambda < \underline{c}(h,\phi)\}$.

The half-plane $\{\lambda \in \C \colon \Re\lambda > \overline{c}(h,\phi)\}$ is always a subset of the resolvent set (see Proposition~\ref{prop:b3-4.8}(a) below).
The description of the remaining part $\{\lambda \in \sigma(A) \colon \underline{c}(h,\phi) \leq \Re\lambda \leq \overline{c}(h,\phi)\}$ is more difficult.
First we discuss some examples and then give a partial answer to this problem (see Proposition~\ref{prop:b3-4.8}(ii)-(v)).

\begin{example}\label{ex:b3-4.7}
	\begin{enumerate}[\upshape (i), wide, labelindent=.5em]
		\item 
		Consider the flow on $[-\frac{\pi}{2},\frac{\pi}{2}]$ defined by
		\[
		\phi(t,x)\coloneq \arctan(\tan x - t) , x \in [-{\pi}/{2},{\pi}/{2}], 
		t \in \R\
		\] 
		and a continuous function $h \colon [-\frac{\pi}{2},\frac{\pi}{2}] \to \R$ with $h(-\frac{\pi}{2}) \leq h(\frac{\pi}{2})$. The flow belongs to the differential equation $y' = -\cos^2y$. 
		Then we have $\underline{c}(h,\phi) = h(-\frac{\pi}{2})$ and $\overline{c}(h,\phi) = h(\frac{\pi}{2})$.
		The spectrum of the corresponding semigroup is given by 
		%% --
		\[
		\sigma(A) = \left\{\lambda \in \C \colon h(-\pi/2) \leq \Re\lambda \leq h(\pi/2)\right\}.
		\]
		%% --
		
		\item 
		Consider $K = \{z \in \C \colon 1 \leq |z| \leq 2\} = \{r\cdot \mathrm{e}^{\mathcal{\im\omega}} \colon \omega\in \R, 1 \leq r \leq 2\}$ and a continuous function $\kappa: [1,2] \to \R_+$.
		
		Let $\tilde{\phi}$ be the flow on $K$ governed by the differential equation $\dot{\omega} = \kappa(r)$, $\dot{r} = 0$ (hence $\tilde{\phi}(t,r\cdot \mathrm{e}^{\im\omega}) = r\cdot \mathrm{e}^{\im(\omega + \kappa(r)t)}$).
		
		For a continuous function $h: K \to \R$ let $\hat{h}(r) \coloneq \frac{1}{2\pi}\int_{0}^{2\pi} h(r\cdot \mathrm{e}^{\im t}) \dt$ $(1 \leq r \leq 2)$.
		The spectrum of the semigroup corresponding to $\phi$ and $h$ (cf.\ \eqref{eq:b3-4.1}) is given by
		%% --
		\[
		\sigma(A) = \{\hat{h}(r) +  \im k\kappa(r) \colon k \in \Z, 1 \leq r \leq 2\} \cup \{h(z) \colon \kappa(|z|) = 0\}.
		\]
	\end{enumerate}
\end{example}

\begin{proposition}\label{prop:b3-4.8}
	Suppose the semigroup $(T(t))_{t\geq 0}$ on $C(K)$ is given by \eqref{eq:b3-4.1} and let $\underline{c}(h,\phi)$, $\overline{c}(h,\phi)$ be defined as in \eqref{eq:b3-4.4}.
	Then the following assertions hold.
	\begin{enumerate}[\upshape (i)]
		\item 
		$\{\lambda \in \C \colon \Re\lambda > \overline{c}(h,\phi)\} \subset \rho(A)$\,,
		
		\item 
		$\overline{c}(h,\phi)$ and $\underline{c}(h,\phi)$ are spectral values\,,
		
		\item 
		If $\phi(t,x_{0}) = x_{0}$ for every $t \geq 0$, then $h(x_{0}) \in R{\sigma}(A)$\,,
		
		\item 
		Assume $x_{0}$ has a finite orbit (\ie $\phi(\R_{+},x_{0}) = \phi([0,T],x_{0})$ for some $T < \infty$) and $\tau \coloneq \inf\{t > 0 \colon \phi(T+t,x_{0}) = \phi(T,x_{0})\} > 0$, \\
		then $\hat{h}(x_{0}) + \frac{2\pi}{\tau}\im\Z \subset R{\sigma}(A)$ where $\hat{h}(x_{0}) \coloneq	\frac{1}{\tau}\int_{T}^{T+\tau}h(\phi(s,x_{0}))\ds $\,,
		
		\item 
		If $x_{0}$ has an infinite orbit and $\hat{h} \coloneq  \lim_{t\to\infty}h(\phi(t,x_{0}))$ exists, then $\hat{h} + \im\R \subset \sigma(A)$.
	\end{enumerate}
\end{proposition}
%% --
\begin{proof}	
	(i) and (ii): By \eqref{eq:b3-4.4} we have $\overline{c}_{t}(h,\phi) = 1/t\cdot\log\|T(t)\|$ hence $\overline{c}(h,\phi) = \omega_{0}(A)$ (cf.\ A-I, (1.1)).  
	Consequently, we have $\{\lambda \in \C \colon \Re\lambda > \overline{c}(h,\phi)\} \subset \rho(A)$ and $\overline{c}(h,\phi) \in \sigma(A)$ by Theorem~\ref{thm:b3-1.6}.
	
	To prove $\underline{c}(h,\phi) \in \sigma(A)$, we can assume by Theorem~\ref{thm:b3-4.4} that $K_{\infty} = K_{s}$ for some $s$ and that $\phi_{|K_{\infty}}$ is injective.
	It is easy to see that $\underline{c}(h,\phi) = \underline{c}(h_{|K_{\infty}},\phi_{|K_{\infty}})$, moreover, we have $\sigma(A_{|I_{\infty}}) = \emptyset$ hence $\sigma(A) = \sigma(A_{/I_{\infty}})$ by A-III, Proposition~4.2.
	This shows that we also can assume that $K = K_{\infty}$, \ie $\phi$ is bijective or $A$ is the generator of a group.
	Now the assertion follows from $\underline{c}(h,\phi) = \underline{c}(h,\phi^{-1}) = -\overline{c}(-h,\phi^{-1}) = -s(-A)$.
	
	(iii) and (iv): One can check easily that in case of (iii) the Dirac functional $\delta_{x_{0}}$ is an eigenvector of $A'$ corresponding to $h(x_{0})$.
    A little bit more calculation is necessary to check that in case of (iv) the functional $\phi_{k}$ defined by
%% --
\[
	\phi_{k}(f) \coloneq \int_{T}^{T+\tau} \exp\left(-\im\cdot\frac{2\pi k}{\tau}\cdot t\right)\cdot h_{t}(x_{0})\cdot f(\phi(t,x_{0})) \dt  \ (k \in \Z, f \in C(K))
\]
%% --
    is an eigenvector of $A'$ corresponding to $\hat{h}(x_{0}) + \im\cdot\frac{2\pi k}{\tau}$.

    (v) Given $\beta \in \R$ we will show that $\hat{h} + \im\beta \in A\sigma(A') \subseteq \sigma(A)$.
    For $n,m \in \N$ we define a linear functional $\phi_{nm}$ as 
%% --
\[  
		\phi_{nm}(f) \coloneq \frac{1}{n}\int_{0}^{n} \exp(-(h^{\star}+\im\beta )t)\cdot h_{t}(\phi(m,x_{0}))\cdot f(\phi(m+t,x_{0})) \text{ d}t, \quad f\in C(K).
\]
For $f \in D(A)$ we have 
%% --
	\begin{align*}
	&\langle(\hat{h}+\im\beta -A)f,\phi_{nm}\rangle = \\
	&  \frac{1}{n}\cdot\left(f(\phi(m,x_{0})) - \mathrm{e}^{-\im\beta  n}\cdot % \exp(-\im\beta  n)
	\exp\left(\int_{m}^{m+n} (h(\phi(s,x_{0}))-\hat{h})\ds \right)f(\phi(m+n,x_{0}))\right).
	\end{align*}
%% --
It follows that $\phi_{nm} \in D(A')$ and, since $\lim_{t\to\infty} h(\phi(t,x_{0})) = \hat{h}$,
%% --
\begin{equation}\label{eq:b3-4.13}
	\lim \sup_{m\to\infty} \|(\hat{h}+\im\beta -A')\phi_{nm}\| \leq 1/n \text{ for every } n \in \N.
\end{equation}
%% --
Because the orbit is infinite we have
%% --
	\begin{align*}
	\|\phi_{nm}\| & {= \frac{1}{n}\int_{0}^{n} \left|\mathrm{e}^{-(\hat{h}+\im\beta )t} h_{t}(\phi(m,x_{0}))\right| \dt} \\
	& {= \frac{1}{n}\int_{0}^{n} \exp\left(\int_{m}^{m+t} (h(\phi(s,x_{0}))-\hat{h})\ds \right)\dt}
	\end{align*}
%% --
which shows that
%% --
\begin{equation}\label{eq:b3-4.14}
	\lim_{m\to\infty} \|\phi_{nm}\| = 1 \text{ for every } n \in \N.
\end{equation}
%% --
In view of \eqref{eq:b3-4.13} and \eqref{eq:b3-4.14} it is not difficult to find a subsequence $k(n)$ of $\N$ such that $(\phi_{n,k(n)})$ is an approximate eigenvector of $A'$ corresponding to $\hat{h} + \im\beta$.
\end{proof}
%% --
We are now going to apply the results obtained so far to the special case where $h = 0$, \ie we consider semigroups of lattice homomorphisms which are Markov operators.
%% --
\begin{theorem}\label{thm:b3-4.9}
	Suppose $\TT$ is a semigroup of Markov lattice homomorphisms on $C(K)$ governed by the semiflow $\phi$.
	\begin{enumerate}[\upshape (i)]
		\item 
		If $\phi_{|K_{\infty}}$ is not injective or if $K_{t} \neq K_{\infty}$ for every $t < \infty$, then $\sigma(A) = \{\lambda \in \C \colon \Re\lambda \leq 0\}$.
		
		\item 
		If $K_{\infty} = K_{s}$ for some $s$ and $\phi_{|K_{\infty}}$ is injective, then $\sigma(A)$ is a cyclic closed subset of $\im \R$.
		Moreover, we have $\sigma(A) \neq \im\R$ if and only if there is a $T < \infty$ such that every orbit of $\phi$ has length less than $T$ (\ie $\phi(\R_{+},x) = \phi([0,T],x)$ for every $x \in K$).
	\end{enumerate}
\end{theorem}
%% --
\begin{proof}
\begin{enumerate}[\upshape (i), wide, labelindent=.5em]%	
\item 
	This is an immediate consequence of Theorem~\ref{thm:b3-4.4} and Proposition~\ref{prop:b3-4.8}.
	
\item 
	The first assertion follows from Theorem~\ref{thm:b3-4.4} and Theorem~\ref{thm:b3-4.1}.
	Moreover, as in the proof of Theorem~\ref{thm:b3-4.4}(ii) and (iii) we can assume without loss of generality that $K = K_{\infty}$, hence $\phi$ is bijective.
	If there is no upper bound for the length of the orbits, then $\sigma(A) = \im\R$ by assertions (iv) and (v) of Proposition~\ref{prop:b3-4.8}.
	
	Now we assume that the lengths of the orbits are bounded by $T$.
	Because $\phi$ is bijective, for every $x \in K$ there exists a $r = r_{x}$ with $T/2 \leq r \leq T$ such that 
	%% --
	\[
	\phi(t,x) = \phi(t+r,x) = \phi(t+2r,x) = \cdots = \phi(t+kr,x) \quad (t \in \R_{+}, k \in \N).
	\]
	
	Therefore we have for $\lambda \in \C$, $\Re\lambda > 0$, $f \in C(K)$, $x \in K$
	%% --
	\begin{equation}\label{eq:b3-4.15}
		\begin{aligned}
			&(R(\lambda,A)f)(x) = \int_{0}^{\infty} \mathrm{e}^{-\lambda t}f(\phi(t,x)) \dt = \\
			&= \sum_{k=0}^{\infty} \mathrm{e}^{-\lambda kr}\int_{kr}^{(k+1)r} \mathrm{e}^{-\lambda(t-kr)} f(\phi(t-kr,x)) \dt =  \\
			&= (1 - \mathrm{e}^{-\lambda r})^{-1}\cdot\int_{0}^{r} \mathrm{e}^{-\lambda t} f(\phi(t,x)) \dt.
		\end{aligned}
	\end{equation}
	%% --
	If $0 < \beta < 2\pi/T$, then the assumption $T/2 \leq r \leq T$ implies that there exists a neighborhood $U$ of $\lambda_{0} \coloneq \im\beta$ such that the functions $\lambda \mapsto (1 - \exp(-\lambda r_{x}))^{-1}$ are uniformly bounded on $U$, by $M$ say.
    Then from \eqref{eq:b3-4.15} we can conclude that $\|R(\lambda,A)f\| \leq M\left(\int_{0}^{r} \left|\mathrm{e}^{-\lambda t}\right|\dt\right)\|f\|$ for $\lambda \in U$, $\Re\lambda > 0$, therefore $\lambda_{0} = \im\beta \in \rho(A)$.
\end{enumerate}
\end{proof}
%% --
\begin{remark}\label{rem:b3-4.10}
	In case $\sigma(A) \neq \im\R$, then $\phi_{|K_{\infty}}$ is bijective and has only finite orbits.
	Therefore every $x \in K_{\infty}$ has a well-defined period $\tau_{x} \coloneq \inf\{\tau > 0 \colon \phi(\tau,x) = x\}$.
	A more detailed analysis yields the following description 
	%% --
	\begin{equation}\label{eq:b3-4.16}
		\sigma(A) = \overline{\{\im \cdot 2\pi k/\tau_{x} \colon k \in \Z, x \in K_{\infty}, \tau_{x} > 0\}} \cup \{0\}.
	\end{equation}
	%% --
	The inclusion \enquote{$\subseteq$} can be derived from Theorem~\ref{thm:b3-4.11} which is stated below.
	The reverse inclusion follows from Proposition~\ref{prop:b3-4.8}(iv).
\end{remark}
	
	In our detailed discussion of the spectrum of lattice homomorphisms we restricted ourselves to the case where the space $K$ is compact.
	The main reason is that there is no description as given in \eqref{eq:b3-4.1} of the semigroups for locally compact spaces $X$.
	In general, it is difficult to define a semiflow on $X$ because points may tend to infinity in a finite time.
	But even if one can find a flow on a suitable compactification of $C$, it may be impossible to find a multiplicator.
	This can be seen by studying the following example.
	
	Suppose $\phi_{1}$ is a semiflow on a compact space $K_{1}$ and $K_{0}$ is a closed $\phi_{1}$-invariant subset, $h$ a continuous function on $K_{1}$.

    The semigroup $(T_{1}(t))$ on $C(K_{1})$ corresponding to $\phi_{1}$ and $h$ leaves the ideal $I \coloneq \{f \in C(K_{1}) \colon f|_{K_{0}} = 0\}$ invariant and induces via restriction a semigroup $(T(t))$ on $I = C_{0}(X)$, where $X = K_{1} \setminus K_{0}$.
    In this case one can construct semi-flows associated with $(T(t))$ on $X\cup\{\infty\}$ or on $\overline{X}$ (closure in $K_{1}$), but in general one cannot find a corresponding multiplicator which is defined on one of these compactifications.

The situation is much nicer when groups of lattice homomorphisms instead of semigroups are considered.
In this case there is an analogue of \eqref{eq:b3-4.1} (cf.\ B-II, Theorem~3.14) and the spectrum can be described completely.
For more details and the proof of the following result we refer to \citet{arendtgreiner:1984}.
%% --
\begin{theorem}\label{thm:b3-4.11}
	Suppose $X$ is a locally compact space and $(T(t))_{t\in\R}$ is a group of lattice homomorphisms governed by the flow $\phi$ and the multiplicator $h$.
	Then we have
	\begin{enumerate}[\upshape (i)]
		\item 
		$\sigma(A) = \sigma_{1}\cup\sigma_{2}\cup\sigma_{3}$ where the sets $\sigma_{i}$ are defined as follows.
		%% --
		\[\begin{aligned}
			\sigma_{1} &\coloneq \overline{\{\hat{h}(x) + \im\cdot 2\pi k/\tau_{x} \colon x \in X, 0 < \tau_{x} < \infty\}}\,,\\
			\sigma_{2} &\coloneq \overline{\{h(x) \colon x \in X, \tau_{x} = 0\}}\,, \\
			\sigma_{3} &\coloneq \{\lambda \in \C \colon \lambda + \im\R \subseteq \sigma(A)\}
		\end{aligned}\]
		
		\item 
		$\sigma(T(t)) = \overline{\exp(t\sigma(A))}$ for every $t \geq 0$.
		
		\item 
		Every isolated point of $\sigma(A)$ is a first order pole of the resolvent.
	\end{enumerate}
\end{theorem}

%% --
%\clearpage
\section*{Notes}
\addcontentsline{toc}{section}{Notes}
%% --
Spectral theory for a single positive operator is an essential cornerstone for spectral theory of positive one-parameter semigroups.
Many of the results we have presented in this chapter have analogues in the discrete case (\ie for a single operator).
This relation may serve as a guide.
However, only in few cases can the continuous version be deduced directly from its discrete analogue.
Therefore we will not try to trace back the roots of every result to the discrete version.
Instead we refer to \citet{schaefer:1974} and the notes and references given there.

Many of the results we have presented in this chapter can be extended (more or less easily) to the more general setting of Banach lattices, which include the very important examples of $L^p$-spaces.
Others are typical for $C_{0}(X)$ and allow no extension.
We will discuss this fact in more detail in Chapter C-III.
The more general setting considered there also allows us to prove results for $C_{0}(X)$ which we could not obtain staying within the framework of spaces of continuous functions.

\begin{enumerate}[label=\emph{Section \arabic*:}, wide]

\item
Theorem~\ref{thm:b3-1.1} was stated by \citet{karlin:1959}, but a complete proof is given in \citet{derndinger:1980}.
Proposition~\ref{prop:b3-1.5}  is taken from \citet{greiner:1982} and Theorem~\ref{thm:b3-1.6} is (implicitely) contained in \citet{derndingernagel:1979}.
A generalization to (non-lattice) ordered Banach spaces can be found in Section~2.4 of \citet{battyrobinson:1984}.

\item
Lemma~\ref{lem:b3-2.3} dates back to Rota (see \citet{schaefer:1974}).
Our approach follows \citet{greiner:1981}.
The notion \enquote{(imaginary) additively cyclic} was introduced by \citet{derndinger:1980} (and {Schaefer (1980)} respectively).
Derndinger proves some cyclicity results for the boundary spectrum.
A result similar to Proposition~\ref{prop:b3-2.7} is given in Section~7.4 of \citet{davies:1980}.
Lemma~\ref{lem:b3-2.8} in combination with C-III, Lemma 3.13 can be used to characterize semigroups whose spectral bound is a pole of finite algebraic multiplicity (see C-III,(3.19)).
The hypothesis of Theorem~\ref{thm:b3-2.9} can be weakened, one only needs that $s(A)$ is a pole of the resolvent (see C-III,Corollary~2.12).
Further results on the cyclicity of the boundary spectrum will be given in Chapter C-III.
In particular we refer to C-III, Theorems.2.10, 3.11 and 3.13.
The dichotomy stated in (\ref{eq:b3-2.19})  is probably the most interesting consequence of cyclicity results.
It has far reaching consequences on the asymptotic behavior of positive semigroups.
Example~\ref{ex:b3-2.13} is due to Davies (unpublished note).
Example~\ref{ex:b3-3.4}(ii) will be discussed in more detail and more generality in Section 3 of Chapter B-IV.
We return to Remark~\ref{rem:b3-2.15}(ii) in Section 2 of B-IV.

\item
The concept of irreducibility as defined in \ref{def:b3-3.1} closely related to various other notions:
In topological dynamics flows inducing irreducible semigroups are called \enquote{minimal flows} (cf.\ Example~\ref{ex:b3-3.4}(i)).
Moreover, \enquote{ergodicity} and \enquote{unique ergodicity} are closely related to irreducibility (see\citet{cornfeldetal:1982} or \citet{krengel:1985}).
Irreducible semigroups are discussed to some extent in \citet{davies:1980}.
E.g. he proves a special case of Theorem~\ref{thm:b3-3.6}.
Proposition~\ref{prop:b3-3.3} will be generalized in C-III, Proposition~3.3.
Assertion (a) of Proposition~\ref{prop:b3-3.5} was proven by \citet{schaefer:1982} while Theorem~\ref{thm:b3-3.6} is taken from \citet{greiner:1982}.
Elliptic operators (more general than Example~\ref{ex:b3-3.4}(ii)) as generators on spaces of continuous functions, were investigated by many people, \eg \citet{bonyetal:1968}, \citet{kuhn:1984}, \citet{roth:1976}, \citet{roth:1978} and \citet{stewart:1974}.

\item
Theorem~\ref{thm:b3-4.1} is due to \citet{derndinger:1984}.
The spectrum of semigroups of Markov lattice homomorphisms is investigated by \citet{derndingernagel:1979}.
In particular they prove Theorem~\ref{thm:b3-4.4} for Markov semigroups.
Earlier results are due to \citet{scarpellini:1974}.
We indicated briefly in Example~\ref{ex:b3-4.6} that there is a relationship between spectral properties of lattice semigroups and differentiable dynamics.
For more details we refer to \citet{chiconeswanson:1981} and \citet{sackersell:1978}.
E.g., the \enquote{annular hull theorem} is a special case of Theorem~\ref{thm:b3-4.11}(ii).
The general result 4.11 was proven by \citet{arendtgreiner:1984}.

\end{enumerate}


%% -- Literatur
%% --
\section*{References}
\addcontentsline{toc}{section}{References}
{\RaggedRight
\renewcommand{\bibsection}{}
\bibliographystyle{abbrvnat}
\bibliography{bib/ln-references}}
 
 