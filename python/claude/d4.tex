% !TeX program = pdfLaTeX
% !TeX encoding = UTF-8
\documentclass{article}
%% Pakete laden
\usepackage[utf8]{inputenc}  
\usepackage[ngerman]{babel}
\usepackage{amsthm}
\usepackage[inline,shortlabels]{enumitem}
\usepackage{amsmath}

%% Umgebungen definieren
\newtheorem{theorem}{Theorem}
\newtheorem{definition}[theorem]{Definition}

\begin{document}

\section{Einleitung}
\label{sec:intro}

Sei $f: A \to B$ eine stetige Abbildung zwischen zwei topologischen Räumen $A$ und $B$.

\begin{definition}
Die Abbildung $f: A \to B$ heißt \emph{homöomorph}, falls $f$ bijektiv ist und die Umkehrabbildung $f{-1}: B \to A$ ebenfalls stetig ist.
\end{definition}

In diesem Fall nennt man die Räume $A$ und $B$ \emph{homöomorph}.

\begin{theorem}
Seien $f: A \to B$ und $g: B \to C$ zwei homöomorphe Abbildungen.
Dann ist die Verkettung 
%% --
\[
	g \circ f: A \to C, x \mapsto g(f(x))
\]
%% --
ebenfalls ein Homöomorphismus.
\label{th:composition}
\end{theorem}

\begin{proof}
Da $f$ und $g$ nach Voraussetzung bijektiv und stetig sind mit stetiger Umkehrabbildung, ist die Verkettung $g \circ f$ ebenfalls
\begin{enumerate*}[label=(\roman*)]
\item bijektiv,
\item stetig und
\item die Umkehrabbildung $(g \circ f){-1} = f{-1} \circ g{-1}$ ist stetig,
\end{enumerate*}
da die Verkettung stetiger Funktionen wieder stetig ist nach einem bekannten Resultat (\autoref{sec:intro}).
\end{proof}

\end{document}