\documentclass[10pt]{scrartcl}
\usepackage{ablatt-ug}

\begin{document}
\pagestyle{empty}
%% --
\section*{Index Seite 4 }
\begin{longtable}{>{\bfseries}p{4cm}p{4cm}p{4cm}p{4cm}}

Population equation 	& 229ff, 344f, 354ff, 364ff \\
Positive part 	& 235 \\
Positive minimum principle 	& 126ff, 133ff, 263ff, 268 \\
Positive subeigenvector 	& 261 \\
Positivity 	& 138ff, 238, 242, 244, 370 \\
	& n- 	& 370, 403 \\
	& strict 	& 238, 242, 310, 316 \\
Predual 	& 329 \\
Projection 	& 72, 209ff, 343ff, 410ff, 423 \\
	& ergodic 	& 410ff, 424 \\
	& recurrent 	& 407 \\
	& semigroup 	& 209ff, 310, 343ff, 411 \\
	& spectral 	& 86ff \\
Pseudo-resolvent 	& 289ff, 314ff, 372ff, 353ff, 392ff, 419ff \\
	& positive 	& 299ff \\
	& \\
Quasi-compact 	& 214ff, 343ff \\
Quasi-interior point 	& 238, 306 \\
	& \\
Range condition 	& 55ff, 146ff, 249, 270 \\
Regular mapping 	& 242, 272, 279ff \\
Regularity 	& 242, 272, 279ff \\
Residue 	& 67f, 72ff, 309ff, 385ff \\
Resolvent 	& 63ff, 370 \\
	& compact 	& 40, 73, 130, 166, 177, 305, 315, 336 \\
	& positive 	& 123ff \\
	& pseudo 	& 298ff, 314ff, 372ff, 353ff, 392ff, 419ff \\
	& slowly growing 	& 301ff \\
Resolvent 	& 63ff, 370 \\
	& equation 	& 127, 298 \\
	& integral representation 	& 6, 293ff \\
	& positive 	& 127 \\
	& set 	& 63ff, 75 \\
Retarded 	& \\
	& differential equation 	& 219ff \\
	& equation 	& 356ff \\
Riesz Decomposition theorem 	& 237 \\
Riesz Schauder theory 	& 72ff \\
	& \\
Schrödinger operator 	& 273f, 278f, 336 \\
Schwarz map 	& 370ff, 379ff, 381ff, 407ff \\
	& identity preserving 	& 370ff, 379ff, 381ff, 408ff \\
Schwarz inequality 	& 370 \\
Schwartz space 	& 19, 260 \\
Self-adjoint part 	& 369 \\
Semiflow 	& 143ff, 332ff \\
	& continuous 	& 144ff, 192 \\
	& injective 	& 193 \\
	& surjective 	& 193 \\
Semigroup 	& 1ff \\
	& adjoint 	& 20ff, 77, 437 \\
	& analytic 	& 83ff \\
%	& bounded 	& 8 \\
	& bounded holomorphic (of angle α) 	& 83ff, 116 \\
	& compact 	& 40ff \\
	& commuting 	& 94 \\
	& contraction 	& 47ff, 247ff, 297f, 307 \\
%	& convolution 	& 12 \\
	& differentiable 	& 37f, 41 \\
	& diffusion 	& 15ff \\
	& disjointness preserving 	& 281ff \\
	& eventually compact 	& 40ff, 209, 211, 214 \\
	& eventually differentiable 	& 37, 41 \\
	& eventually norm continuous 	& 38ff, 41, 87ff, 106, 176, 304f, 318, 337, 345 \\
	& $\mathcal{F}$-product 	& 25ff, 74ff, 192 \\
	& holomorphic (of angle α) 	& 33ff, 41, 100, 163, 305ff, 311ff \\
	& identity preserving 	& 370ff, 379ff, 381ff, 408ff, 424f \\
	& implemented 	& 403 \\
	& induced 	& 74ff, 298, 374 \\
	& irreducible 	& 162ff, 210, 315ff, 338ff, 409ff \\
	& lattice homomorphism 	& 136ff, 143ff, 194ff, 235, 320ff \\
	& Markovian 	& 144ff, 191 \\
%	& matrix 	& 7 \\
	& mean-ergodic 	& 346 \\
	& modulus 	& 278ff, 292ff \\
	& multiplication 	& 7f, 42ff, 65f, 287ff \\
	& nilpotent 	& 15, 41f, 74ff \\
	& norm continuous 	& 38ff, 41 \\
	& of Schwarz type 	& 370ff, 379ff, 409ff, 424f \\
	& one-parameter 	& 1 \\
	& partially periodic 	& 359ff, 410ff \\
	& periodic 	& 79ff, 85, 313, 416 \\
	& positive 	& 123ff \\
	& preadjoint 	& 447 \\
	& quasi-compact 	& 214ff, 343ff \\
	& quotient 	& 19, 74 \\
	& reduced 	& 374, 407 \\
	& rescaled 	& 18 \\
	& rotation 	& 14, 69, 189, 313, 352ff \\
	& similar 	& 18ff \\
	& Sobolev 	& 23ff \\
	& strongly continuous 	& 2ff \\
	& strongly ergodic 	& 406, 408ff, 424f \\
	& subspace 	& 18f, 74 \\

\end{longtable}

\end{document}