\documentclass[10pt]{scrartcl}
\usepackage{longtable}
\usepackage{array}
\begin{document}

\section*{Subject Index (Fortsetzung)}

\begin{longtable}{>{\bfseries}p{6cm}p{8cm}}
\textbf{Term} & \textbf{Page Numbers} \\
\hline
\endhead

\textbf{Population equation} & 229ff, 344ff, 354ff, 364ff \\
\\
\textbf{Positive part} & 235 \\
\\
\textbf{Positive minimum principle} & 185ff, 133ff, 253ff, 368 \\
\\
\textbf{Positive subeigenvector} & 207 \\
\\
\textbf{Positivity} & 119, 119, 120, 125ff, 233, 243, 244, 370 \\
\quad n- & 370, 403 \\
\quad strict & 119, 119, 120, 233, 242, 310, 316 \\
\\
\textbf{Predual} & 329 \\
\\
\textbf{Projection} & 72, 209ff, 343ff, 410ff, 423 \\
\quad ergodic & 410ff, 424 \\
\quad recurrent & 407 \\
\quad semigroup & 209ff, 310, 343ff, 411 \\
\quad spectral & 86ff \\
\\
\textbf{Pseudo-resolvent} & 208ff, 314ff, 372ff, 383ff, 392ff, 419ff \\
\quad positive & 299ff \\
\\
\textbf{Quasi-compact} & 214ff, 343ff \\
\\
\textbf{Quasi-interior point} & 238, 306 \\
\\
\textbf{Range condition} & 53ff, 146ff, 249, 270 \\
\\
\textbf{Regular mapping} & 242, 272, 279ff \\
\\
\textbf{Regularity} & 372, 273, 279ff \\
\\
\textbf{Residue} & 67ff, 72ff, 306ff, 385ff \\
\\
\textbf{Resolvent} & 63ff, 370 \\
\quad compact & 40, 73, 130, 166, 177, 305, 315, 336 \\
\quad positive & 133ff \\
\quad pseudo & 298ff, 314ff, 372ff, 383ff, 392ff, 419ff \\
\quad slowly growing & 301ff \\
\\
\textbf{Resolvent} & 6, 63ff, 370 \\
\quad equation & 137, 299 \\
\quad integral representation & 6, 293ff \\
\quad positive & 137 \\
\quad set & 63ff, 75 \\
\\
\textbf{Retarded} & \\
\quad differential equation & 219ff \\
\quad equation & 356ff \\
\\
\textbf{Riesz Decomposition theorem} & 237 \\
\\
\textbf{Riesz Schauder theory} & 72ff \\
\\
\textbf{Schrödinger operator} & 273ff, 278ff, 336 \\
\\
\textbf{Schwarz map} & 370ff, 379ff, 381ff, 407ff \\
\quad identity preserving & 370ff, 379ff, 381ff, 408ff \\
\\
\textbf{Schwarz inequality} & 370 \\
\\
\textbf{Schwartz space} & 19, 250 \\
\\
\textbf{Self-adjoint part} & 369 \\
\\
\textbf{Semiflow} & 143ff, 323ff \\
\quad continuous & 144ff, 192 \\
\quad injective & 193 \\
\quad surjective & 193 \\
\\
\textbf{Semigroup} & 1ff \\
\quad adjoint & 16ff, 77, 400 \\
\quad analytic & 83ff \\
\quad bounded & 3 \\
\quad bounded holomorphic (of angle α) & 83ff, 110 \\
\quad compact & 40ff \\
\quad commuting & 54 \\
\quad contraction & 3, 47ff, 247ff, 287ff, 397 \\
\quad convolution & 12 \\
\quad differentiable & 37ff, 41 \\
\quad diffusion & 11ff \\
\quad disjointness preserving & 281ff \\
\quad eventually compact & 40ff, 209, 211, 214 \\
\quad eventually differentiable & 37, 41 \\
\quad eventually norm continuous & 38ff, 41, 87ff, 106, 178, 304ff, 318, 337, 345 \\
\\
\textbf{F-product} & 20ff, 74ff, 192 \\
\quad holomorphic (of angle α) & 33ff, 41, 100, 163, 305ff, 311ff \\
\quad identity preserving & 370ff, 379ff, 381ff, 408ff, 424 \\
\quad implemented & 403 \\
\quad induced & 14ff, 74ff, 206, 374 \\
\quad irreducible & 182ff, 210, 315ff, 386ff, 409ff \\
\quad lattice homomorphism & 136ff, 143ff, 194ff, 235, 380ff \\
\quad Markovian & 144ff, 191 \\
\quad matrix & 7 \\
\quad mean-ergodic & 346 \\
\quad modulus & 276ff, 382ff \\
\quad multiplication & 7ff, 42ff, 65ff, 287ff \\
\quad nilpotent & 11, 41ff, 74ff \\
\quad norm continuous & 38ff, 41 \\
\quad of Schwarz type & 370ff, 379ff, 408ff, 424 \\
\quad one-parameter & 1 \\
\quad partially periodic & 352ff, 410ff \\
\quad periodic & 79ff, 85, 313, 416 \\
\quad positive & 123ff \\
\quad preadjoint & 414 \\
\quad quasi-compact & 214ff, 343ff \\
\quad quotient & 15, 74 \\
\quad reduced & 374, 407 \\
\quad rescaled & 14 \\
\quad rotation & 10, 69, 189, 313, 352ff \\
\quad similar & 15ff \\
\quad Sobolev & 19ff \\
\quad strongly continuous & 2ff \\
\quad strongly ergodic & 406, 408ff, 424 \\
\quad subspace & 14ff, 74 \\

\end{longtable}

\end{document}