%% --
%% -- Updated Notes D1-D4
%% -- Stand 2025-10-11
%% -- ulgr
%% --
\section{Updated Notes D-I}
%% --
\section{Updated Notes D-I}
%% --
An overview of positive operators on operator algebras can be found in \mycite{zbMATH06128372}, but there seems to be no systematic reference for general positive 
$C_{0}$-semigroups on operator algebras.
Many papers deal with Markov semigroups (see, \eg \mycite{zbMATH03762672}) or with so-called E-semigroups
(see \mycite{zbMATH01949821})
%% --
%% --
\section{Updated Notes D-II}
%% -- 
\begin{enumerate}
\item 
As we have seen in Chapter A-II, Section 3, strongly continuous semigroups on commutative \WA-algebras, that is, on $L^{\infty}$, are already norm-continuous. 
The proof depends heavily on the Grothendieck property (GP) and the Dunford-Pettis property (DPP) of these Banach spaces. 
It was shown by \mycite{zbMATH00537321} that every \WA-algebra has (GP) (see also \mycite{zbMATH00125315} for an alternative proof).
On the other hand, for a \WA-algebra $M$ the following are equivalent.
%% --
\begin{enumerate}[\upshape(a)]
    \item The \WA-algebra $M$ is of Type I finite.
    \item The \WA-algebra $M$ has (DPP).
    \item The predual $M_{*}$ has (DPP).
\end{enumerate}
%% --
(see \mycite{zbMATH00125315} and \mycite{zbMATH00097659}).
Thus, as a consequence of A-II Theorem 3.5, strongly continuous semigroups on Type I finite \WA-algebras have a bounded generator. 
\item
\mycite{zbMATH01495754} has shown that strong and norm convergence for sequences of completely positive maps are equivalent on \WA-algebras (even on the larger class of \AW-algebras).
Therefore a strongly continuous $C_{0}$-semigroup of completely positive operators on a \WA-algebra is norm continuous.
\item 
In \mycite{zbMATH00097715} it is shown that the equivalence of strong and norm convergence of semigroups on $C(K)$ implies the \CA-algebra $C(K)$ is a Grothendieck space.
This is also true for a noncommutative \CA-algebra $\BA$.
To prove this, one uses the fact that a \CA-algebra is a Grothendieck space if and only if $c_{0}$ is not a complemented subspace (see \mycite[Prop. 3.1.13 \& 4.2.1]{zbMATH07458830}).
Indeed, suppose that every strongly convergent $C_{0}$-semigroup is uniformly continuous on $\BA$.
Then $c_{0}$ can't be a complemented subspace of $\BA$.
For if $\BA = c_{0} \oplus F$ for a closed subspace $F$ of $\BA$, then take a strongly convergent but not uniformly convergent semigroup ${T(t)}$ on $c_{0}$ and let $S(t) = T(t) \oplus Id_{F}$ on $\BA$. 
Then ${S(t)}$ is strongly but not uniformly convergent on $\BA$. 
\end{enumerate}
%% --
%% --
\section{Updated Notes D-III}
%% -- 
\begin{enumerate}
\item 
Since the positive cone of the self-adjoint elements of a \CA-algebra is a normal cone, one can derive properties such as $s(A) \in \sigma(A)$ or $s(A) = \omega_{0}$ from the theory of semigroups of positive operators on ordered Banach spaces with such cones; see \mycite{zbMATH03883065} or \mycite{zbMATH03736445}.
%But to remain in the category of \CA-algebras, we have summarized this using \CA-algebra methods in a preprint.
\item 
The ultraproduct construction for spectral theory on \WA-algebras is not stable; see \mycite[p. 79]{zbMATH03640303} or \mycite{zbMATH03467832}.
Fortunately, the preduals of \WA-algebras are stable under ultraproducts and can be used for such investigations.
More on ultraproducts of \WA-algebras can be found in \mycite{zbMATH06326930}.
\item 
Another approach to the spectral theory on \WA-algebras is in \mycite{zbMATH06031834}, where a Jacobs-de Leeuw-Glicksberg decomposition is constructed.
This leads to a noncommutative version of the Perron-Frobenius theorem for \WA-algebras and is applied to the asymptotics of \WA-dynamical systems.
A similar approach is in \mycite{zbMATH06728793}.
\end{enumerate}
%% --
%% --
\section{Updated Notes D-IV}
%% --
As in D-III, results from \citeEN{zbMATH07964911} on semigroups on ordered Banach spaces with a normal cone and order unit can be applied. 
More precise results on \WA-algebras and their preduals are in \mycite{zbMATH02244715}.
Specific investigations focus on \emph{Quantum Markov semigroups and decoherence} (see, \eg, \mycite{zbMATH08072228}) or on \emph{Spectral gaps and convergence to equilibrium} (see, \eg \mycite{zbMATH06737571}).
%% --