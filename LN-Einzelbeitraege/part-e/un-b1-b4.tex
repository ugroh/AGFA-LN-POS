%% --
%% -- Updated Notes B1-B4
%% -- Stand 2025-09-30
%% -- ulgr
%% --
\section{Updated Notes B-I}
%% --
For the abstract chacterization of spaces of continuous functions as commutative \CA-algbras, \ie the Gelfand-Naimark theorem, see \mycite[Chapter I-3]{zbMATH01692441}. 
The other concepts as ideals, their connections with closed sets, the representation of lattice or algebraic homomorpisms we refer to \mycite{zbMATH03357434}. 
The various types of positive operators on these algebras are discussed in \mycite[Chapter 4]{zbMATH06423122}. 

Semigroups on spaces of continuous functions generated by elliptic
operators in non-divergence form are treated in the monograph
\mycite{zbMATH00732330}
%% --
%% --
\section{Updated Notes B-II}
%% --
Today many examples of positive  semigroups on spaces of continuous functions are known. For elliptic operators in divergence form with Dirichlet boundary conditions we refer to \mycite{zbMATH01241400}, for Robin boundary conditions to \mycite{zbMATH02051543} and \mycite{zbMATH05925852}. 
In contrast to the situation in $L^p$, in spaces of continuous functions irreducibility is not so easy to prove, we refer to \mycite{zbMATH07232945} for a Banach lattice argument which works for elliptic operators in divergence form. 
Elliptic operators in non-divergence form generate an irreducible, positive, holomorphic semigroup on $C_{0}(\Omega)$ if $\Omega$ is connected and satisfies the uniform exterior cone condition, see \mycite{zbMATH06317650}.

The Dirichlet-to-Neumann operator is an example of a non-local operator generating a positive semigroup on $C(\partial\Omega)$, whenever $\Omega$ is a  bounded,  open set with Lipschitz boundary, see \mycite{zbMATH07372898}, where also very general elliptic operators are considered. The semigroup is irreducible whenever $\Omega$ is connected. This is surprising since the boundary may not be connected (think of a ring). Thus, the notion of irreducibility reflects the non-local character of the Dirichlet-to-Neumann operator.
So far it is unknown whether Lipschitz continuity of the boundary suffices for the semigroup generated by the Dirichlet-to-Neumann operator on  $C(\partial\Omega)$ to be holomorphic. However, for slightly better boundary it is, see  \mycite{zbMATH07062560}.


It was discovered by \mycite{zbMATH06347363} that the Dirichlet-to-Neumann operator with respect to the  Laplace operator perturbed by a potential has unexpected properties concerning positivity. In fact, there are cases where the semigroup is merely eventually positive but not positive. 
This discovery was the origin of a systematic investigation of semigoups which are merely positive after some time (called eventually positive semigroups) and similar concepts, see \mycite{zbMATH06487326}, \mycite{zbMATH07497413}, \mycite{zbMATH06897364}, \mycite{zbMATH06723334} and \mycite{zbMATH07830511} for some recent  results in this direction.



There are also non-local versions of Dirichlet boundary conditions \mycite{zbMATH07220470}  and of Robin and Wentzell boundary conditions \mycite{arXiv:2502.03216} which lead to positive semigroups. 

Feller semigroups are positive contractive semigroups acting on spaces of continuous functions. They are of great importance for stochastic processes. 
As an example for  the rich litterature we mention the monographs \mycite{zbMATH06314001}, \mycite{zbMATH05714362}, \mycite{zbMATH01707584}, \mycite{zbMATH01807482}  and  \mycite{zbMATH02189175} . 

Perturbation results for Feller semgiroups are obtained in \mycite{zbMATH06286612} and \mycite{zbMATH07606117}, convergence of Feller semigroups is studied in  \mycite{zbMATH07694938}.



In B-II, Example 3.15 the solution flow of a nonlinear differential
equation on $\R^{n}$ leads to a $C_{0}$-(semi-)group of
positive operators  on a Banach lattice of continuous functions. 
Its generator is a linear differential operator given by formula (3.12).
Such "Markov lattice semigroups", see B-II, Definition 3.3, are now
frequently called \enquote{Koopman semigroups}. 
This kind of a linearization of
nonlinear pdes became popular, e.g, by the work of I. Mezic \citeEN{zbMATH05045798}
in the context of numerical problems using the "dynamical mode
decomposition". A solid mathematical setting for such Koopman semigroups
on $C_{0}(X)$, $X$ not locally compact,  needed for the solution flow of a
pde, is proposed by Kreidler-Farkas \mycite{MR4176386} . 
An introduction to Koopman semigroups is given in Chapter 16  of \mycite{zbMATH06695787}.


 Finally we mention a recent perturbation theory for generators of positive semigroups on AM- and AL-spaces presented in \mycite{zbMATH07971662}.

%% --
%% --
\section{Updated Notes B-III}
Noch nichts vorhanden
%% --
%% --
\section{Updated Notes B-IV}
Noch nichts vorhanden
