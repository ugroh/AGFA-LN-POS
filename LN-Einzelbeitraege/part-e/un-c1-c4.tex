%% --
%% -- Updated Notes C1-C4
%% -- Stand 2025-09-30
%% -- ulgr
%% --
\section{Updated Notes C-I}
%% --
Our main source for the theory of Banach lattices and positive operators is \mycite{zbMATH03464348}.
Other useful references are \mycite{zbMATH03983937}, \mycite{zbMATH00051953}, and \mycite{zbMATH00967648}.

A gentle introduction to the theory of semigroups of positive operators is \mycite{zbMATH06695787}, starting from finite dimensions and leading to many concrete applications.

Motivated by concrete PDEs, see \eg \mycite{zbMATH06487326}, \enquote{eventually positive} semigroups form another very active research area.
We refer to the survey article by \mycite{zbMATH07497413}.

\subsection*{Für Abschnitt C-III und C-IV}


 \mycite{zbMATH07124949}, \\
 \mycite{zbMATH06591946}, \\
 \mycite{zbMATH07024808}, \\
 \mycite{zbMATH07423250}, \\
 \mycite{zbMATH07201923}
%% --
%% --
\section{Updated Notes C-II}
%% --
\begin{enumerate}[$\bullet$, wide, labelindent=.75em]


\item 
It is interesting that some properties of semigroups are preserved by domination.
An important result by \mycite{zbMATH07735826} says the following.
%% --
\begin{quote}
\textit{%
Let $(T(t))_{t\geq 0}$ and $(S(t))_{t\geq 0}$ be positive semigroups on a Banach lattice $E$ such that $S(t) \leq T(t)$ for all $t\geq 0$. If the semigroup  $(T(t))_{t\geq 0}$ is holomorphic, then so is the semigroup $ (S(t))_{ t\geq 0 } $.} %% so ist es richtig
\end{quote}
%% --

The proof uses an interesting result by \mycite{zbMATH01080532} about the preservation of spectral and asymptotic behaviour of semigroups under domination.

Also mean ergodicity is preserved under domination  if the underlying Banach lattice $E$ has order continuous norm, see \mycite{zbMATH00031435}. Specifically, this is valid for complex Banach lattices and even if the semigroup S is not necessarily positive (which is needed for the preceding result, though). Thus the weaker domination property  
$ \lvert S(t)f \rvert \leq T(t)\lvert f\rvert $ for all $ t\geq 0 $, $ f\in E $ suffices. However, on a space of type $C(K)$ mean ergodicity is not necessarily inherited from a dominating semigroup, see Section 3 in \mycite{zbMATH00031435}.

Another interesting result is proved by \mycite{zbMATH00537201}:
%% --
\begin{quote}
\textit{%
Let $(T(t))_{t\geq 0}$ and $(S(t))_{t\geq 0}$ be positive semigroups on a Banach lattice $E$ with order continuous norm such that $S(t) \leq T(t)$ for all $t\geq 0$.
If $T(t)f$ converges to $Pf$ as $t\rightarrow 0$ for all $f\in E$ and $P$ has finite rank, then also $S(t)f$ converges as $t\rightarrow 0$ for all $f\in E$.}
\end{quote}
%% --
For further properties inherited by domination we refer to \mycite{zbMATH00561284} and the literature mentioned there.

\item
Kato's classical inequality is frequently used to prove uniqueness results. 
 %For instance, it can be used to prove an %interesting maximum principle for the Dirichlet %problem. We refer to the article \mycite{MR3965216}.
An interesting generalisation of Kato's inequality has been proved by \mycite{zbMATH02093937}. 
The Kato inequality (K) in C-II, Theorem 3.8 for generators of positive semigroups has interesting applications to semi-linear evolution equations, see \mycite{zbMATH07978278}.


\item 
Form  methods are important for generation of holomorphic semigroups on a Hilbert space.
The Beurling-Deny criterion is a most efficient tool to characterise positivity of a semigroup on $L^{2}$ which is associated with a form. 
\mycite{zbMATH00001168} extended this criterion to describe invariance of arbitrary closed convex sets in the underlying Hilbert space. This allows him to characterize irreducibility of the associated  semigroups in a very simple way. 
We refer to Ouhabaz' monograph \mycite{zbMATH02168554} for this and a comprehensive theory of forms. 
In particular, semigroups generated by elliptic operators under diverse boundary conditions on $L^{2}$ can be described in a very efficient way by form methods. 

\item 

Domination can be proved most conveniently for semigoups associated with a form, see e.g., \mycite{zbMATH05059635}, \mycite{zbMATH02168554}. 
More general criteria for domination, valid in ordered Banach spaces, are given by \mycite{zbMATH05119932} .
The modulus semigroups has been determined in a series of interesting concrete cases, see \mycite{zbMATH05262326},      \mycite{zbMATH02106405}, \mycite{zbMATH02167818} 

\item 
Kernel estimates for positive semigroups, and in particular Gaussian estimates, play an important role. 
They imply that a semigroup defined and holomorphic on $L^{2}$ extends to all $L^{p}$-spaces and is holomorphic on each of these spaces (and in particular on $L^{1}$), see \mycite{zbMATH00788452}. 
Even the spectrum of the generator is independent on p in this case, see \mycite{zbMATH01344331}. 
In \mycite{zbMATH01538134} Gaussian estimates are proved for semigroups generated by elliptic opertors with measurable coefficients under several boundary conditions. A comprehensive account is given in \mycite{zbMATH02168554}.


\item 
It is most remarkable that a positive contractive semigroup on $L^{p}$ for $1 < p < \infty$ enjoys \emph{maximal regularity}, an important property much studied in the past two decades. This result is due to \mycite{zbMATH01661089}.
We refer to Chapter 17 in the monograph \mycite{zbMATH07784626} for a comprehensive treatment of maximal regularity. 

%For the existence of a continuous kernel associated with the 
%semigroup we refer to \mycite{zbMATH07123546}

\item 
The Dirichlet-to-Neumann operator generates a holomorphic positive irreducible semigroup on $L^{p}(\partial\Omega)$ whenever $ \Omega$  is a bounded, connected  Lipschitz domain. 
This can be proved by form methods. 
Kernel estimates are obtained in \mycite{zbMATH07063341}

\item 
 Important research has been done on Ornstein-Uhlenbeck semigroups. These are positive semigroups which are explicitly given by a Gaussian kernel. 
 Such a semigroup acts on all $L^{p}$-spaces with respect to the Lebesgue measure and also with respect to the invariant measure $\mu$ when the drift matrix $A$ is real with eigenvalues in the open left halfplane. 
 Using perturbation methods the domain of its generator can be described explicitely, see \mycite{zbMATH02217252}. 
For regularity properties and the spectrum of Ornstein-Uhlenbeck operators we refer the reader to the survey article \mycite{MR4176390} and the monograph \mycite{zbMATH06593873}. 
For quantitative and qualitative properties of more general Kolmogorov operators we refer to \mycite{zbMATH06593873}, \mycite{zbMATH01837428} and \mycite{zbMATH01789518}.



\item
An elliptic operator with Robin boundary conditions (also called boundary conditions of the third kind) generate a positive semigroup for very general functions defining the Robin boundary, see \mycite{zbMATH05551221} . 

Also  non-local boundary conditions lead to positive semigroups.  
An interesting example is treated in \mycite{arXiv:2502.03216}. 
 
\item 
The survey article \mycite{zbMATH07402424} shows which role positivity plays in models and also gives some new perturbation results (in Section 6). Further results can be found in \mycite{zbMATH08056303}. 

\item 
Semigroups of lattice homomorphisms from the Koopman point of view 
on $L^{p}$-spaces are the subject of \mycite{zbMATH07036266}; see also the extended notes of Chapter B-II concerning Koopman semigroups.


\item
In C-II Proposition 5.16 it is shown that any strongly continuous group in the center of a real Banach lattice has a bounded generator. 
In this context it is interesting to mention the \emph{Markov conjecture}: \textit{Any generator of a strongly continuous positive semigroup on $\ell^{1}$ which is isometric on the positive cone and which extends to a group has a bounded generator.} 
This conjecture is still open, but a special case has been proved by \mycite{zbMATH07367032}.

\item

Perturbation of positive semigroups is systematically studied in the monograph \mycite{zbMATH05030445}.

\end{enumerate}
%% --
%% --
\section{Updated Notes C-III}
\begin{enumerate}
    \item 
The research over the last 30 years has progressively expanded our understanding of when the peripheral spectrum of positive semigroups is additively cyclic. 
While the general question remains open, J. Glück and other researchers have identified numerous sufficient conditions (growth conditions, boundedness assumptions, irreducibility properties) and extended classical results to broader classes of operators including semigroups and eventually positive operators. 
Glück's counterexample for irreducible operators  (see below) shows that some classical results cannot be fully generalized without additional assumptions.

\item
In \mycite{zbMATH06591946} it has been proven that under appropriate growth and regularity conditions, the peripheral point spectrum of a positive semigroup is cyclic. 
In addition, dimension estimates for the corresponding eigenspaces are given. In \mycite{zbMATH07024808}, Glück identifies several growth conditions involving eigenvectors or the resolvent that provide new sufficient criteria for cyclicity of the peripheral spectrum. He gives an alternative proof that every (WS)-bounded positive semigroup has cyclic peripheral spectrum, and shows that for irreducible (WS)-bounded semigroups, every peripheral eigenvalue is algebraically simple. In \mycite{zbMATH07423250} the author conctructs irreducible stochastic semigroups whose whose peripheral spectrum equals a finite union of discrete subgroup. Thus Theorem\ref{c3:thm-3.8} which states that the peripheral point spectrum is a subgroup does not hold for the whole peripheral spectrum.

\end{enumerate}
%% --
%% --
\section{Updated Notes C-IV}
Mögliche Literatur 

\mycite{zbMATH07124949}:
\\
author	= {Gerlach, Moritz and Gl{\"u}ck, Jochen},
\\
  title		= {Convergence of positive operator semigroups},

\mycite{zbMATH07201923}:
\\
  author	= {Gl{\"u}ck, Jochen and Haase, Markus},
\\
title		= {Asymptotics of operator semigroups via the semigroup at
		  infinity},

\mycite{zbMATH05262335}:
\\
author = {Keicher, Vera and Nagel, Rainer},
\\
title = {Positive semigroups behave asymptotically as rotation groups},

\mycite{zbMATH05491771}: 
\\
author = {Eisner, Tanja and Farkas, B{\'a}lint and Nagel, Rainer and Ser{\'e}ny, Andr{\'a}s},
\\
title = {Weakly and almost weakly stable {{\(C_0\)}}-semigroups}
 
Und das hat Claude daraus gemacht: 

Ich habe nun umfangreiche Suchen durchgeführt und kann dir eine englische Zusammenfassung der wichtigsten Ergebnisse erstellen.
 
Hier ist die Zusammenfassung in LaTeX:



\textbf{Summary of Main Results on Asymptotic Theory of Operator Semigroups}




\subsection*{The Four Papers}

\paragraph{Eisner, Farkas, Nagel, Serény (2007): Weakly and Almost Weakly Stable $C_0$-Semigroups}

This paper investigates stability notions weaker than strong stability for $C_0$-semigroups. The work distinguishes between weak stability and almost weak stability, providing characterizations and examples that separate these concepts from classical stability notions.

\paragraph{Gerlach \& Glück (2019): Convergence of Positive Operator Semigroups}

Gerlach and Glück present new conditions for semigroups of positive operators to converge strongly as time tends to infinity. Their proofs are based on a novel approach combining the Jacobs--de Leeuw--Glicksberg splitting theorem with a purely algebraic result about positive group representations. This allows them to obtain convergence theorems not only for one-parameter semigroups but for a much larger class of semigroup representations, dropping any continuity or regularity assumptions with respect to the time parameter.

The main assumption is that the semigroup representation contains an AM-compact operator (which includes kernel operators and compact operators). This framework unifies various theorems from the literature and generalizes them in several respects.

\paragraph{Glück \& Haase (2019): Asymptotics of Operator Semigroups via the Semigroup at Infinity}

Glück and Haase systematize and generalize recent results on strong convergence and spectral theory of bounded positive operator semigroups on Banach spaces and lattices. They introduce the ``semigroup at infinity'' and give useful criteria ensuring that the Jacobs--de Leeuw--Glicksberg splitting theory can be applied to it. For positive semigroups on Banach lattices with a quasi-interior point, these criteria are intimately linked to AM-compact operators, and they imply that the original semigroup asymptotically embeds into a compact group of positive invertible operators on an atomic Banach lattice.

\paragraph{Keicher \& Nagel (2008): Positive Semigroups Behave Asymptotically as Rotation Groups}

This paper establishes that under appropriate conditions, positive semigroups exhibit asymptotic behavior resembling rotation groups, connecting the peripheral spectrum structure to the long-term dynamics of the semigroup.

\subsection*{Related Work from the Last 30 Years}

\paragraph{The Jacobs--de Leeuw--Glicksberg Decomposition}

The JdLG decomposition is a fundamental tool that splits a dynamical system into a ``reversible'' part (which can be embedded into a compact group action) and a ``stable'' part. This classical result by Jacobs, de Leeuw, and Glicksberg has been extended to various settings including operator semigroups on Banach spaces.

Recent work (2024) provides new characterizations of the reversible and almost weakly stable parts of the JdLG decomposition, showing that the reversible part is weakly equivalent to a unitary representation on a Hilbert space decomposing into finite dimensional representations.

\paragraph{Eventually Positive Semigroups}

Arora and Glück (2021) extended convergence characterizations to eventually positive semigroups---semigroups that become positive only after sufficient time. They prove that similar theorems as for positive semigroups remain true for this larger class, including a version of the Niiro--Sawashima theorem for eventually positive operators. However, several proofs for positive operators do not work in this setting, necessitating different arguments.

Earlier work (2018) developed the theory of eventually positive $C_0$-semigroups on Banach lattices, giving characterizations by means of spectral and resolvent properties of the generators. This enables treatment of examples including the square of the Laplacian with Dirichlet boundary conditions and the bi-Laplacian on $L^p$-spaces.

Recent work (2024) investigates spectral and asymptotic properties of eventually positive semigroups, particularly in the persistently irreducible case, using ultrapower arguments to overcome challenges where standard techniques for positive operators fail.

\paragraph{Locally Eventually Positive Semigroups}

Glück and others (2021--2023) initiated a theory of locally eventually positive operator semigroups, where solutions become positive only in part of the domain after sufficient time. This weakens requirements on the leading eigenvalue of the semigroup generator and establishes spectral and convergence properties under minimal additional assumptions.

\paragraph{Kernel Operators}

Arendt (2008) showed that the peripheral point spectrum of the generator of a positive bounded $C_0$-semigroup of kernel operators on $L^p$ is reduced to $0$, implying convergence to equilibrium if the semigroup is irreducible with non-trivial fixed space.

Kernel operators play a central role in convergence theory. Semigroups containing or dominating kernel operators satisfy strong convergence properties under appropriate conditions.

\paragraph{Applications and Extensions}

Recent doctoral research (2022--2024) extends these methods to semigroups lacking strong continuity, using the concept of the semigroup at infinity alongside the JdLG decomposition to prove convergence results with respect to operator norm. Applications include transport processes on infinite networks and coupled parabolic PDEs.

Glück and Wolff (2019) analyze positive operator semigroups via asymptotic domination, providing tools for long-term analysis on ordered Banach spaces beyond the classical Banach lattice setting.

\subsection*{Summary}

The last 30 years have seen substantial progress in understanding the asymptotic behavior of positive operator semigroups through:

\begin{enumerate}
\item \textbf{Unified frameworks}: The work of Gerlach and Glück provides a unified algebraic approach using the JdLG decomposition and AM-compact operators that encompasses many previous results.

\item \textbf{The semigroup at infinity}: This concept by Glück and Haase enables analysis of convergence without regularity assumptions on time.

\item \textbf{Extension to weaker positivity}: Eventually and locally eventually positive semigroups extend the theory beyond classical positive semigroups.

\item \textbf{Spectral characterizations}: Deep connections between peripheral spectrum properties (cyclicity, triviality of peripheral point spectrum) and long-term behavior.

\item \textbf{Applications}: These abstract results apply to concrete differential equations, transport processes, and stochastic processes.
\end{enumerate}