%% --
%% -- Updated Notes
%% -- Stand 2025-10-07
%% --
\setlist{$\bullet$, wide, labelindent=.5em,itemsep=.25em}
\section{Updated Notes C-II}
%% --
\begin{enumerate} 

\item 
It is interesting that some properties of semigroups are preserved by domination.
We mention the following  result by \mycite{zbmath07735826}.
%% --
\begin{quote}
\textit{%
Let $(T(t))_{t\geq 0}$ and $(S(t))_{t\geq 0}$ be positive semigroups on a Banach lattice $E$ such that $S(t) \leq T(t)$ for all $t\geq 0$. If the semigroup  $(T(t))_{t\geq 0}$ is holomorphic, then so is the semigroup $ (S(t))_{ t\geq 0 } $.} %% so ist es richtig
\end{quote}
%% --

The proof uses a result by \mycite{zbmath01080532} about the preservation of spectral and asymptotic behaviour of semigroups under domination.

Also mean ergodicity is preserved under domination  if the underlying Banach lattice $E$ has order continuous norm, see \mycite{zbmath00031435}. Specifically, this is valid for complex Banach lattices and even if the semigroup S is not necessarily positive (which is needed for the preceding result, though). Thus the weaker domination property  
$ \lvert S(t)f \rvert \leq T(t)\lvert f\rvert $ for all $ t\geq 0 $, $ f\in E $ suffices. However, on a space of type $C(K)$ mean ergodicity is not necessarily inherited from a dominating semigroup, see Section 3 in \mycite{zbmath00031435}.

Another interesting result is proved by \mycite{zbmath00537201}:
%% --
\begin{quote}
\textit{%
Let $(T(t))_{t\geq 0}$ and $(S(t))_{t\geq 0}$ be positive semigroups on a Banach lattice $E$ with order continuous norm such that $S(t) \leq T(t)$ for all $t\geq 0$.
If $T(t)f$ converges to $Pf$ as $t\rightarrow 0$ for all $f\in E$ and $P$ has finite rank, then also $S(t)f$ converges as $t\rightarrow 0$ for all $f\in E$.}
\end{quote}
%% --
For further properties inherited by domination we refer to \mycite{zbmath00561284} and the literature mentioned there.

\item
Kato's classical inequality is frequently used to prove uniqueness results. 
 %For instance, it can be used to prove an %interesting maximum principle for the Dirichlet %problem. We refer to the article \mycite{MR3965216}.
A generalisation of Kato's inequality has been proved by \mycite{zbmath02093937}. 
The abstract Kato inequality (K) in C-II, Theorem 3.8 for generators of positive semigroups has interesting applications to semi-linear evolution equations, see \mycite{zbmath07978278}.


\item 
Form  methods are important for generation of holomorphic semigroups on a Hilbert space.
The Beurling-Deny criterion is a most efficient tool to characterise positivity of a semigroup on $L^{2}$ which is associated with a form. 
\mycite{zbmath00001168} extended this criterion to describe invariance of arbitrary closed convex sets in the underlying Hilbert space. This allows him to characterize irreducibility of the associated  semigroups in a very simple way. 
We refer to Ouhabaz' monograph \mycite{zbmath02168554} for this and a comprehensive theory of forms. 
In particular, semigroups generated by elliptic operators under diverse boundary conditions on $L^{2}$ can be described efficiently  by form methods. 

\item 

Domination can be proved most conveniently for semigroups associated with a form, see e.g., \mycite{zbmath05059635}, \mycite{zbmath02168554}. 
More general criteria for domination, valid in ordered Banach spaces, are given by \mycite{zbmath05119932} .
The modulus semigroups has been determined in a series of concrete cases, see \mycite{zbmath05262326},      \mycite{zbmath02106405}, \mycite{zbmath02167818} 

\item 
Kernel estimates for positive semigroups, and in particular Gaussian estimates, play an important role. 
They imply that a semigroup defined and holomorphic on $L^{2}$ extends to all $L^{p}$-spaces and is holomorphic on each of these spaces (and in particular on $L^{1}$), see \mycite{zbmath00788452}. 
Even the spectrum of the generator is independent on p in this case, see \mycite{zbmath01344331}. 
In \mycite{zbmath01538134} Gaussian estimates are proved for semigroups generated by elliptic opertors with measurable coefficients under several boundary conditions. A comprehensive account is given in \mycite{zbmath02168554}.


\item 
It is most remarkable that a positive contractive semigroup on $L^{p}$ for $1 < p < \infty$ enjoys \emph{maximal regularity}, an important property much studied in the past two decades. This result is due to \mycite{zbmath01661089}.
We refer to Chapter 17 in the monograph \mycite{zbmath07784626} for a comprehensive treatment of maximal regularity. 

%For the existence of a continuous kernel associated with the 
%semigroup we refer to \mycite{zbmath07123546}

\item 
The Dirichlet-to-Neumann operator generates a holomorphic positive irreducible semigroup on $L^{2}(\partial\Omega)$ whenever $ \Omega$  is a bounded, connected  Lipschitz domain (see \mycite{zbMATH06171001} and the Updated Notes of B-II). 
This can be proved by form methods. 
Kernel estimates are obtained in \mycite{zbmath07063341}

\item 
 Important research has been done on so-called Ornstein-Uhlenbeck semigroups which are  explicitly given by a Gaussian kernel. 
 Such a semigroup acts on all $L^{p}$-spaces with respect to the Lebesgue measure and also with respect to the invariant measure $\mu$ when the drift matrix $A$ is real with eigenvalues in the open left halfplane. 
 The domain of its generator can be described explicitely, see \mycite{zbmath02217252}. 
For regularity properties and the spectrum of Ornstein-Uhlenbeck operators we refer the reader to the survey article \mycite{MR4176390} and the monograph \mycite{zbmath06593873}. 
For quantitative and qualitative properties of more general Kolmogorov operators we refer to \mycite{zbmath06593873}, \mycite{zbmath01837428} and \mycite{zbmath01789518}.



\item
An elliptic operator with Robin boundary conditions (also called boundary conditions of the third kind) generates a positive semigroup for very general functions defining the Robin boundary, see \mycite{zbmath05551221} . 

Also  non-local boundary conditions lead to positive semigroups,  see e.g.
\mycite{arXiv:2502.03216}. 
 
\item 
The survey article \mycite{zbmath07402424} shows which role positivity plays in models and also gives some new perturbation results (in Section 6). Further results can be found in \mycite{zbmath08056303}. 

\item 
Semigroups of lattice homomorphisms from the Koopman point of view 
on $L^{p}$-spaces are the subject of \mycite{zbmath07036266}; see also the extended notes of Chapter B-II concerning Koopman semigroups.


\item
In C-II, Proposition 5.16 it is shown that any strongly continuous group in the center of a real Banach lattice has a bounded generator. 
In this context it is interesting to mention the \emph{Markov conjecture}: \textit{Any generator of a strongly continuous positive semigroup on $\ell^{1}$ which is norm-preserving on the positive cone and which extends to a group has a bounded generator.} 
This conjecture is still open, but a special case has been proved by \mycite{zbmath07367032}.

\item

Perturbation of positive semigroups is systematically studied in the monograph \mycite{zbmath05030445}.

\end{enumerate}


%\section*{References}
\addcontentsline{toc}{subsection}{Updated Notes: References}
%% --


