%% --
%% -- Updated Notes
%% -- Stand 2025-10-07
%% --
\setlist{$\bullet$, wide, labelindent=.5em,itemsep=.25em}
\section*{Updated Notes C-IV}
\addcontentsline{toc}{section}{Updated Notes}
\begin{enumerate}
\item 
The problem formulated after C-IV, Theorem 1.1
has been solved by \mycite{zbmath00868262}:
The growth bound and spectral bound coincide for positive semigroups on all $L^{p}$-spaces, for $1 \leq p \ < \infty$. 
This proof is reproduced with more details in the monograph \mycite{zbmath00921898}, and a different proof is given in \mycite[Theorem 5.3.6]{zbmath05842872}.  
Recently a short and elegant proof of Weis' Theorem has been found by \mycite{zbmath07574448}, which is even valid for eventually positive semigroups.

\item 
A survey on the asymptotic behaviour of positive semigroups can be found in \mycite{mr4176382}, where also the condition of a countable boundary spectrum is discussed.


\item 
In C-IV, Corollary 2.12 non-spectral conditions imply strong convergence of a semigroup as $t \rightarrow \infty$. The essential property is that one operator $T(t_0)$ is a kernel operator. This surprising phenomenon has been systematically studied in \mycite{zbmath07124949}, where various generalizations and different arguments are given. The main hypothesis is that one of the semigroup operators $T(t_0)$ is AM-compact (which includes kernel operators and compact operators).
These ideas are developed further in \mycite{zbmath07201923}. 
\item 
In C-IV, Theorem 2.14, conditions are given implying that  a positive, irreducible, bounded semigroup converges strongly to a periodic group. 
Further results of alalogous asymptotic behaviour are given in \mycite{zbmath05262335}.
Also, certain flows on a network converge to a periodic flow as shown in \mycite{zbmath02146799}.
Many of such results are based on the Jacobs-DeLeeuw-Glicksberg Theorem, see  \mycite[Theorem  V.2.8]{zbmath01354832} for an introduction taylored for one-parameter semigroups, and see \mycite{zbmath06423122} for a more complete presentation of this theorem. 
The strongest results of this sort are obtained if the semigroup has strongly compact orbits, see  \mycite[Theorem V.2.14]{zbmath01354832}. 
A different approach to such a decomposition in a group part and a part which converges to $0$ is given in \mycite[Chapter V]{zbmath05842872}, see also the notes to Section 5.4 in that book.
\mycite{zbmath07201923} introduce the notion \emph{semigoup at infinity} which allows them not only to generalize the results by \mycite{zbmath07124949} mentioned above, but also to give structure theorems for positive groups. 


\end{enumerate}


%% --
\addcontentsline{toc}{section}{Updated Notes: References}
%% --