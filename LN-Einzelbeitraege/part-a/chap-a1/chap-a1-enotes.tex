%% --
%% -- Updated Notes
%% -- Stand 2025-10-07
%% --
\setlist{$\bullet$, wide, labelindent=.5em,itemsep=.25em}
\section{Updated Notes A-I}
%% --
Among recent books on $C_{0}$-semigroups on Banach spaces we mention:
\begin{enumerate}
\item 
\mycite{zbmath05842872} approaches semigroups via the Laplace transform and the resolvent of the generator. 
\item 
The first part of \mycite{zbmath01354832} contains the theory of $C_{0}$-semigroups via generation results, perturbation and approximation, spectral theory, and asymptotic behavior; hence, it is very analogous to the present lecture notes. 
The second half, under the headline \enquote{Semigroups Everywhere} and partly written by other authors, shows how different evolution equations can be treated using the theory of semigroups (see \mycite[Chap. VI]{zbmath01354832}).
\item 
Operator semigroups on Hilbert spaces can also be studied via the theory of forms.
We refer to the monographs of \mycite{zbmath02168554} and W. Arendt, H. Vogt, and J. Voigt: \emph{Form Methods for Evolution Equations} (Birkhäuser, to appear).
\item
The role of one-parameter semigroups in the theory of dynamical systems is studied in great detail in the monograph \mycite{zbmath01329917}.
\item
As textbooks suited for graduate courses, we recommend, \eg 
\mycite{zbmath05051510}, \mycite{zbmath07066876}, and
\mycite{zbmath06706234}.
\item
While all semigroups in these texts are assumed to be strongly continuous, in many situations semigroups appear---under various names---that are continuous only for some weaker topology. 
The concept of \enquote{bicontinuous semigroups}, covering these different notions, is proposed in \mycite{zbmath02051541}. 
\item
The $\F$-product in Section 3.7 on page \pageref{subsec:a1-3.7} and the corresponding extension of a $C_{0}$-semigroup is a special case of the so-called \emph{ultraproduct construction} of Banach spaces (see, \eg \mycite{zbmath03640303} or \mycite{zbmath03987965}).
This technique is useful for spectral theory, converting the approximate point spectrum into point spectrum. 
Its application to the spectral theory of $C_{0}$-semigroups, as in 
\mbox{A-III.6.6}, was started with the aforementioned work of \mycite{DerndingerEN:1980} and extended in \mycite{zbmath069217}.
\end{enumerate}

%%% GG:  \eg,  durch \eg  ersetzt
% **Vorgenommene Korrekturen:**
% - Item 3: Komma vor "and J. Voigt" hinzugefügt (Oxford comma im amerikanischen Englisch)
% - Item 5: Komma nach "courses" hinzugefügt
% - Item 7: "and converts" zurück zu "converting" geändert (Parallelstruktur mit "useful for")
% - Item 7: "is extended" zurück zu "extended" geändert (konsistent mit "was started")
\addcontentsline{toc}{section}{Updated Notes: References}

