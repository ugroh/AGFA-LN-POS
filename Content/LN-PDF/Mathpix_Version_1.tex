% This LaTeX document needs to be compiled with XeLaTeX.
\documentclass[10pt]{article}
\usepackage[utf8]{inputenc}
\usepackage{ucharclasses}
\usepackage{graphicx}
\usepackage[export]{adjustbox}
\graphicspath{ {./images/} }
\usepackage{amsmath}
\usepackage{amsfonts}
\usepackage{amssymb}
\usepackage[version=4]{mhchem}
\usepackage{extpfeil}
\usepackage{stmaryrd}
\usepackage{bbold}
\usepackage[fallback]{xeCJK}
\usepackage{polyglossia}
\usepackage{fontspec}
\usepackage{eurosym}
\usepackage{newunicodechar}
\setCJKmainfont{Noto Serif CJK KR}
\setCJKfallbackfamilyfont{\CJKrmdefault}{
{Noto Serif CJK JP}
}
\begin{document}
\setmainlanguage{english}
\setotherlanguages{thai, hindi}
\newfontfamily\thaifont{Noto Serif Thai}
\newfontfamily\hindifont{Noto Serif Devanagari}
\newfontfamily\lgcfont{CMU Serif}
\setDefaultTransitions{\lgcfont}{}
\setTransitionsFor{Thai}{\thaifont}{\lgcfont}
\setTransitionsFor{Hindi}{\hindifont}{\lgcfont}

\title{Lecture Notes in Mathematics }

\author{by\\
Ulrich Groh *)}
\date{}


%New command to display footnote whose markers will always be hidden
\let\svthefootnote\thefootnote
\newcommand\blfootnotetext[1]{%
  \let\thefootnote\relax\footnote{#1}%
  \addtocounter{footnote}{-1}%
  \let\thefootnote\svthefootnote%
}

%Overriding the \footnotetext command to hide the marker if its value is `0`
\let\svfootnotetext\footnotetext
\renewcommand\footnotetext[2][?]{%
  \if\relax#1\relax%
    \ifnum\value{footnote}=0\blfootnotetext{#2}\else\svfootnotetext{#2}\fi%
  \else%
    \if?#1\ifnum\value{footnote}=0\blfootnotetext{#2}\else\svfootnotetext{#2}\fi%
    \else\svfootnotetext[#1]{#2}\fi%
  \fi
}

\newunicodechar{€}{\ifmmode\text{\euro}\else\euro\fi}
\newunicodechar{×}{\ifmmode\times\else{$\times$}\fi}

\begin{document}
\maketitle
Lecture Notes in Mathematics

\section*{Shi Jian-Yi}
\section*{The Kazhdan-Lusztig Cells in Certain Affine Weyl Groups}


\section*{1184}


\section*{One-parameter Semigroups of Positive Operators}
Edited by R. Nagel\\
\includegraphics[max width=\textwidth, center]{2024_12_23_c6487cc0859199a15bd9g-002}

Springer-Verlag\\
Berlin Heidelberg New York Tokyo

\section*{Editor}
\section*{Rainer Nagel}
Mathematisches Institut, Universität Tübingen\\
Auf der Morgenstelle 10, 7400 Tübingen, Federal Republic of Germany

\section*{Authors}
\section*{Wolfgang Arendt}
Annette Grabosch\\
Günther Greiner\\
Ulrich Moustakas\\
Rainer Nagel\\
Ulf Schlotterbeck\\
Mathematisches Institut, Universität Tübingen\\
Auf der Morgenstelle 10, 7400 Tübingen, Federal Republic of Germany

Ulrich Groh\\
IBM-Deutschland, NM FB Lehre und Forschung\\
Poccistr. 11, 8000 München 2, Federal Republic of Germany

Heinrich P. Lotz\\
Department of Mathematics, University of Ilinois\\
Urbana, Illinois 61801, USA

Frank Neubrander\\
Department of Mathematics, Georgetown University\\
Washington, D. C. 20057, USA

Mathematics Subject Classification (1980): 34 G 10 , $34 \mathrm{~K} 30,35 \mathrm{~B} 40,35 \mathrm{~F} 10$, 35K22, 35P05, 35R 15, 47A 10, 47B38, 47B55, 47C 15, 47D05

ISBN 3-540-16454-5 Springer-Verlag Berlin Heidelberg New York Tokyo ISBN 0-387-16454-5 Springer-Verlag New York Heidelberg Berlin Tokyo

\footnotetext{This work is subject to copyright. All rights are reserved, whether the whole or part of the material is concerned, specifically those of translation, reprinting, re-use of illustrations, broadcasting, reproduction by photocopying machine or similar means, and storage in data banks. Under $\S 54$ of the German Copyright Law where copies are made for other than private use, a fee is payable to "Verwertungsgesellschaft Wort", Munich.\\
(c) by Springer-Verlag Berlin Heidelberg 1986

Printed in Germany\\
Printing and binding: Beltz Offsetdruck, Hemsbach/Bergstr.\\
2146/3140-543210
}\section*{PREFACE}
As early as 1948 in the first edition of his fundamental treatise on Semigroups and Functional Analysis, E. Hille expressed the need for "developing an adequate theory of transformation semigroups operating in partially ordered spaces" (1.c., Foreword). In the meantime the theory of one-parameter semigroups of positive linear operators has grown continuously. Motivated by problems in probability theory and partial differential equations W. Feller (1952) and R. S. Phillips (1962) laid the first cornerstones by characterizing the generators of special positive semigroups. In the $60^{\prime} \mathrm{s}$ and $70^{\prime}$ s the theory of positive operators on ordered Banach spaces was built systematically and is well documented in the monographs of H. H. Schaefer (1974) and A.C. zaanen (1983). But in this process the original ties with the applications and, in particular, with initial value problems were at times obscured. Only in recent years an adequate and up-to-date theory emerged, largely based on the techniques developed for positive operators and thus recombining the functional analytic theory with the investigation of Cauchy problems having positive solutions to each positive initial value. Even though this development - in particular with respect to applications to concrete evolution equations in transport theory, mathematical biology, and physics - is far from being complete, the present volume is a first attempt to shape the multitude of available results into a coherent theory of one-parameter semigroups of positive linear operators on ordered Banach spaces.

The book is organized as follows.\\
We concentrate our attention on three subjects of semigroup theory: characterization, spectral theory and asymptotic behavior. By characterization, we understand the problem of describing special properties of a semigroup, such as positivity, through the generator. By spectral theory we mean the investigation of the spectrum of a generator. Asymptotic behavior refers to the orbits of the initial values under a given semigroup and phenomena such as stability.

This program (characterization, spectral theory, asymptotic behavior) is worked out on four different types of underlying spaces:\\
A. On Banach spaces. Here we present the background for the subsequent discussions related to order.\\
B. On spaces $C_{0}(\mathrm{X})$ ( X locally compact), which constitute an important class of ordered Banach spaces and where our results can be presented in a form which makes them accessible also for the non-expert in order-theory.\\
C. On Banach lattices, which admit a rich theory and are still sufficiently general as to include many concrete spaces appearing in analyis; e.g., $C_{0}(x), \mathrm{L}^{\mathrm{P}}(\mu)$ or $\mathrm{I}^{\mathrm{P}}$.\\
D. On non-commutative operator algebras such as $\mathrm{C}^{*}$ - or W*-algebras, which are not lattice ordered but still possess an interesting order structure of great importance in mathematical physics.

In each of these cases we start with a short collection of basic results and notations, so that the contents of the book may be visualized in the form of a $4 \times 4$ matrix in a way which will allow "row readers" (interested in semigroups on certain types of spaces) and "column readers" (interested in certain aspects) to find a path through the book corresponding to their interest.

We display this matrix, together with the names of the authors contributing to the subjects defined through this scheme:

\begin{center}
\begin{tabular}{|c|c|c|c|c|}
\hline
 & \( \begin{gathered} \text { I } \\ \text { Basic } \\ \text { Theory } \end{gathered} \) & \begin{tabular}{l}
II \\
Character- \\
ization \\
\end{tabular} & \texttt{III Spectral Theory} & \begin{tabular}{l}
IV \\
Asymptotics \\
\end{tabular} \\
\hline
\begin{tabular}{l}
Banach \\
A. \\
Spaces \\
\end{tabular} & \begin{tabular}{l}
R. Nagel \\
U. Schlotterbeck \\
\end{tabular} & \begin{tabular}{l}
w. Arendt \\
H. P. Lotz \\
\end{tabular} & \begin{tabular}{l}
G. Greiner \\
R. Nagel \\
\end{tabular} & F. Neubrander \\
\hline
B. $\mathrm{C}_{\mathrm{O}}(\mathrm{X})$ & \begin{tabular}{l}
R. Nagel \\
u. Schlotterbeck \\
\end{tabular} & w. Arendt & G. Greiner & \begin{tabular}{l}
A. Grabosch \\
G. Greiner \\
U. Moustakas \\
F. Neubrander \\
\end{tabular} \\
\hline
\begin{tabular}{l}
Banach \\
c. \\
Lattices \\
\end{tabular} & \begin{tabular}{l}
R. Nagel \\
u. Schlotterbeck \\
\end{tabular} & w. Arendt & G. Greiner & \begin{tabular}{l}
A. Grabosch \\
G. Greiner \\
U. Moustakas \\
R. Nagel \\
F. Neubrander \\
\end{tabular} \\
\hline
\begin{tabular}{l}
Operator \\
D. \\
Algebras \\
\end{tabular} & u. Groh & u. Groh & U. Groh & u. Groh \\
\hline
\end{tabular}
\end{center}

This "matrix of contents" has been an indispensable guide line in our discussions on the scope and the spirit of the various contributions. However, we would not have succeeded in completing this manuscript, as a collection of independent contributions (personally accounted for by the authors), under less favorable conditions than we have actually met. For one thing, Rainer Nagel was an unfaltering and undisputed spiritus rector from the very beginning of the project. On the other hand we gratefully acknowledge the influence of Helmut H. Schaefer and his pioneering work on order structures in analysis. It was the team spirit produced by this common mathematical background which, with a little help from our friends, made it possible to overcome most difficulties.

We have prepared the manuscript with the aid of a word processor, but we confess that without the assistance of Klaus Kuhn the pitfalls of such a system would have been greater than its benefits.

\section*{Part A One-parameter Semigroups on Banach Spaces}
A-I Basic Results on Semigroups on Banach Spaces ..... 1\\
by Rainer Nagel and Ulf Schlotterbeck

\begin{enumerate}
  \item Standard Definitions and Results ..... 1
  \item Standard Examples ..... 7
  \item Standard Constructions ..... 13\\
Notes ..... 24\\
A-II Characterization of One-parameter Semigroup on Banach Spaces ..... 25
  \item The Abstract Cauchy Problem, Special Semigroups and Perturbation ..... 26\\
by Wolfgang Arendt
  \item Contraction Semigroups and Dissipative Operators ..... 47\\
by Wolfgang Arendt
  \item Semigroups on $\mathrm{I}^{\infty}$ and $\mathrm{H}^{\infty}$ ..... 54\\
by H. P. Lotz\\
Notes ..... 58\\
A-III Spectral Theory of Semigroups on Banach Spaces ..... 60\\
by Günther Greiner and Rainer Nagel
  \item Introduction ..... 60
  \item The Fine Structure of the Spectrum ..... 63
  \item Spectral Decompositions ..... 68
  \item The Spectrum of Induced Semigroups ..... 74
  \item The Spectrum of Periodic Semigroups ..... 79
  \item Spectral Mapping Theorems ..... 82
  \item Weak Spectral Mapping Theorems ..... 89\\
Notes ..... 96\\
A-IV Asymptotics of Semigroups on Banach Spaces ..... 98\\
by Frank Neubrander
  \item Stability: Homogeneous Case ..... 98
  \item Stability: Inhomogeneous Case ..... 112\\
Notes ..... 115
\end{enumerate}

\section*{Part $B$ positive Semigroups on Spaces $C_{0}(X)$}
B-I Basic Results on C (X) ..... 117by Rainer Nagel and Ulf Schlotterbeck

\begin{enumerate}
  \item Algebraic and Order Structure; Ideals and Quotients ..... 117
  \item Linear Forms and Duality ..... 118
  \item Linear Operators ..... 120\\
B-II Characterization of Positive Semigroups on $\mathrm{C}_{\mathrm{O}}(\mathrm{X})$ ..... 122by Wolfgang Arendt
  \item Generators of Positive Semigroups on C(K) ..... 123
  \item Lattice Semigroups on $C_{O}(X)$ ..... 135
  \item Semiflows, Flows and Positive Groups ..... 143\\
Notes ..... 162\\
B-III Spectral Theory of Positive Semigroups on $\mathrm{C}_{\mathrm{O}}(\mathrm{X})$ ..... 163\\
by Günther Greiner
  \item The Spectral Bound ..... 163
  \item The Boundary Spectrum ..... 169
  \item Irreducible Semigroups ..... 182
  \item Semigroups of Lattice Homomorphisms ..... 192\\
Notes ..... 202\\
B-IV Asymptotics of Positive Semigroups on $C_{0}(X)$ ..... 204
  \item Stability of Positive Semigroups on $C_{0}(x)$ ..... 204\\
by Frank Neubrander
  \item Compact and Quasi-compact Semigroups ..... 209\\
by Günther Greiner
  \item A Semigroup Approach to Retarded Differential Equations ..... 219\\
by Annette Grabosch and Ulrich Moustakas\\
Notes ..... 231
\end{enumerate}

\section*{Part C Positive Semigroups on Banach Lattices}
C-I Basic Results on Banach Lattices and Positive Operators ..... 233\\
by Rainer Nagel and Ulf Schlotterbeck

\begin{enumerate}
  \item Sublattices, Ideals, Bands ..... 236
  \item Order Units, Weak Order Units, Quasi-interior Points ..... 238
  \item Linear Forms and Duality ..... 238
  \item AM- and AL-Spaces ..... 239
  \item Special Connections between Norm and Order ..... 241
  \item Positive Operators, Lattice Homomorphisms ..... 242
  \item Complex Banach Lattices ..... 243
  \item The Signum Operator ..... 245
  \item The Center of $L(E)$ ..... 246\\
C-II Characterization of Positive Semigroups on Banach Lattices ..... 247\\
by Wolfgang Arendt
  \item Positive Contraction Semigroups and Bounded Generators ..... 248
  \item Kato's Inequality ..... 256
  \item A Characterization of Generators of Positive Semigroups ..... 260
  \item Domination of Semigroups ..... 269
  \item Semigroups of Disjointness Preserving Operators ..... 281\\
Notes ..... 290\\
C-III Spectral Theory of Positive Semigroups on Banach Iattices ..... 292\\
by Günther Greiner
  \item The Spectral Bound ..... 292
  \item The Boundary Spectum ..... 296
  \item Irreducible Semigroups ..... 306
  \item Semigroups of Lattice Homomorphisms ..... 320\\
Notes ..... 331\\
C-IV Asymptotics of Positive Semigroups on Banach Lattices ..... 333
  \item Stability of Positive Semigroups on Banach Lattices ..... 334\\
by Günther Greiner and Frank Neubrander
  \item Convergence of Positive Semigroups ..... 342\\
by Günther Greiner and Rainer Nagel
  \item A Semigroup Approach to Retarded Equations ..... 356\\
by Annette Grabosch and Ulrich Moustakas ..... 367Notes\\
Part D Positive Semigroups on $C^{*}$ - and $\mathrm{W}^{*}$-Algebras\\
by Ulrich Groh\\
D-I Basic Results on Semigroups and Operator Algebras ..... 369
  \item Notations ..... 369
  \item A Fundamental Inequality for the Resolvent ..... 370
  \item Induction and Reduction ..... 374\\
D-II Characterization of Positive Semigroups on W*-Algebras ..... 376
  \item Positive Semigroups on Properly Infinite W*-Algebras ..... 376\\
D-III Spectral Theory of Positive Semigroups on W*-Algebras and ..... 379\\
their Preduals
  \item Spectral Theory for Positive Semigroups on Preduals ..... 379
  \item Spectral Properties of Uniformly Ergodic Semigroups ..... 391\\
Notes ..... 398\\
D-IV Asymptotics of Positive Semigroups on $C^{*}$ - and $\mathrm{W}^{*}$-Algebras ..... 400
  \item Stability of Positive Semigroups ..... 400
  \item Stability of Implemented Semigroups ..... 403
  \item Convergence of Positive Semigroups ..... 406
  \item Uniform Ergodic Theorems ..... 419\\
Notes ..... 425\\
Bibliography ..... 427\\
Table of symbols ..... 453\\
Subject Index ..... 456
\end{enumerate}

\section*{ONE-PARAMETER SEMIGROUPS ON BANACH SPACES }
CHAPTER A-I

B A S I C R S U L T O N SEMIGROUPS

ON B A NACA $\quad$ A P A C E S\\
by\\
Rainer Nagel and Ulf Schlotterbeck

Since the basic theory of one-parameter semigroups can be found in several excellent books (e.g. Davies (1980), Goldstein (1985a), Pazy (1983) or Hille-Phillips (1957)) we do not want to give a self-contained introduction to this subject here. It may however be useful to fix our notation, to collect briefly some important definitions and results (Section 1), to present a list of standard examples (Section 2) and to discuss standard constructions of new semigroups from a given one (Section 3).\\
In the entire chapter we denote by $E$ a (real or) complex Banach space and consider one - parameter semigroups of bounded linear operators $T(t)$ on $E$. By this we understand a subset $\left\{T(t): t \in \mathbb{R}_{+}\right\}$ of $L(E)$, usually written as $(T(t))_{t \geq 0}$, such that\\
$\mathrm{T}(0)=\mathrm{Id}$,\\
$T(s+t)=T(s) \cdot T(t) \quad$ for all $s, t \in \mathbb{R}_{+} \cdot$\\
In more abstract terms this means that the map $t \rightarrow T(t)$ is a homomorphism from the additive semigroup $\left(\mathbb{R}_{+},+\right)$into the multiplicative semigroup $(L(E), \cdot)$. Similarly, a one-parameter group $(T(t))_{t \in \mathbb{R}}$ will be a homomorphic image of the group $(\mathbb{R},+)$ in $(L(E), \cdot)$.

\section*{1. STANDARD DEFINITIONS AND RESULTS}
We consider a one-parameter semigroup $(T(t))_{t \geq 0}$ on a Banach space $E$ and observe that the domain $\mathbb{R}_{+}$and the range $L(E)$ of the (semi-\\
group) homomorphism $\tau: t \rightarrow T(t)$ are topological semigroups for the natural topology on $\mathbb{R}_{+}$and any one of the standard operator topologies on L(E) . We single out the strong operator topology on $L(E)$ and require $\tau$ to be continuous.

Definition 1.1. A one-parameter semigroup $(T(t))_{t \geqq 0}$ is called strongly continuous if the map $t \rightarrow T(t)$ is continuous for the strong operator topology on $L(E)$, i.e. $\lim _{t \rightarrow t_{0}}\left\|T(t) f-T\left(t_{0}\right) f\right\|=0$ for every $f \in E$ and $t, t_{0} \geq 0$.

Clearly one defines in a similar way weakly continuous, resp. uniformly continuous (compare A-II, Def..1.19) semigroups, but since we concentrate on the strongly continuous case we agree on the following terminology:

If not stated otherwise, a s e m i g r o u p is a strongly continuous one-parameter semigroup of bounded linear operators.

Next we collect a few elementary facts on the continuity and boundedness of one-parameter semigroups.

Remarks 1.2. (1) A one-parameter semigroup (T)( $t)_{t \geq 0}$ on a Banach space $E$ is strongly continuous if and only if for any $f \in E$ it is true that $T(t) f \rightarrow f$ as $t \rightarrow 0$.\\
(2) For every strongly continuous semigroup ( $T(t){ }^{\prime}{ }_{t \geqq 0}$ there exist constants $M \geqq 1, w \in \mathbb{R}$ such that $\|T(t)\| \leqq M \cdot e^{w t}$ for every $t \geqq 0$.\\
(3) If $(T(t))_{t \geq 0}$ is a one-parameter semigroup such that $\|T(t)\|$ is bounded for $0<t \leqq \delta$ then it is strongly continuous if and only if $\lim _{t \rightarrow 0} T(t) f=f$ for every $f$ in a total subset of $E$.

The exponential estimate from Remark $1.2,(2)$ for the growth of $\|T(t)\|$ can be used to define an important characteristic of the semigroup.

Definition 1.3. By the growth bound (or type) of the semigroup $(T(t))_{t \geq 0}$ we understand the number


\begin{align*}
& w:=\inf \left\{w \in \mathbb{R}: \text { There exists } M \in \mathbb{R}_{+} \text {such that }\|T(t)\| \leqq M e^{w t}\right. \\
&\text { for } t \geqq 0\}
\end{align*}


Particularily important are semigroups such that for every $t \geqq 0$ we have $\|T(t)\| \leq M \quad$ (bounded semigroups) or $\|T(t)\| \leq 1$ (contraction semigroups). In both cases we have $\omega \leq 0$.

It follows from the subsequent examples and from 3.1 that $w$ may be any number $-\infty \leqq \omega<+^{\infty}$. Moreover the reader should observe that the infimum in (1.1) need not be attained and that $M$ may be larger than 1 even for bounded semigroups.

Examples 1.4. (i) Take $E=\mathbb{C}^{2}, A=\left(\begin{array}{cc}0 & 1 \\ 0 & 0\end{array}\right)$ and $T(t)=e^{t A}=\left(\begin{array}{ll}1 & t \\ 0 & 1\end{array}\right)$. Then for the 1 -norm on $E$ we obtain $\|T(t)\|=1+t$, hence $(T(t))_{t \geqq 0}$ is an unbounded semigroup having growth bound $\omega=0$. (ii) Take $E=L^{1}(\mathbb{R})$ and for $f \in E, t \geqq 0$ define

$$
T(t) f(x):= \begin{cases}2 \cdot f(x+t) & \text { if } x \in[-t, 0] \\ f(x+t) & \text { otherwise. }\end{cases}
$$

Each $T(t), t>0$, satisfies $\|T(t)\|=2$ as can be seen by taking $f:=I_{[0, t]}$. Therefore $(T(t))_{t \geqq 0}$ is a strongly continuous semigroup which is bounded, hence has $\omega=0$, but the constant $M$ in (1.1) cannot be choosen to be 1 .

The most important object associated to a strongly continuous semigroup $(T(t))_{t \geq 0}$ is its 'generator' which is obtained as the (right) derivative of the map $t \rightarrow T(t)$ at $t=0$. Since for strongly continuous semigroups the functions $t \rightarrow T(t) f, f \in E$, are continuous but not always differentiable we have to restrict our attention to those $f \in E$ for which the desired derivative exists. We then obtain the 'generator' as a not necessarily everywhere defined operator.

Definition 1.5. To every semigroup $\left(T(t){ }_{t \geqq 0}\right.$ there belongs an operator $(A, D(A))$, called the generator and defined on the domain

$$
D(A):=\left\{f \in E: \lim _{h \rightarrow 0} \frac{T(h) f-f}{h} \text { exists in } E\right\}
$$

by Af $:=\lim _{h \rightarrow 0} \frac{T(h) f-f}{h}$ for $f \in D(A)$.

Clearly, $D(A)$ is a linear subspace of $E$ and $A$ is linear from $D(A)$ into E. Only in certain special cases (see 2.1) the generator\\
is everywhere defined and therefore bounded (use prop.1.9(i)). In general the precise extent of the domain D(A) is essential for the characterization of the generator. But since the domain is canonically associated to the generator of a semigroup we shall write in most cases A instead of (A,D(A)).\\
As a first result we collect some information on the domain of the generator.

Proposition 1.6. For the generator $A$ of a semigroup ( $T(t))_{t \geqq 0}$ on a Banach space $E$ the following assertions hold: If $f \in D(A)$ then $T(t) \pounds \in D(A)$ for every $t \geqq 0$.\\
(ii) The map $t+T(t) f$ is differentiable on $\mathbb{R}_{+}$if and only if $f \in D(A)$. In that case one has $\frac{d}{d t} T(t) f=A T(t) f=T(t) A f$.\\
(iii) For every $f \in E$ and $t>0$ the element $\int_{0}^{t} T(s) f d s$ belongs to $D(A)$ and one has\\
(1.3) $\quad A \int_{0}^{t} T(s) f d s=T(t) f-f$.\\
(iv) If $f \in D(A)$ then\\
(1.4) $\quad \int_{0}^{t} T(s) A f d s=T(t) f-f$.\\
(v) The domain $D(A)$ is dense in E.

The identity (1.2) is of great importance and shows how semigroups are related to certain Cauchy problems. We state this explicitely in the following theorem.

Theorem 1.7. Let (A,D(A)) be the generator of a strongly continuous semigroup $(T(t))_{t \geq 0}$ on the Banach space $E$. Then the 'abstract Cauchy problem'\\
$(\mathrm{ACP}) \quad \frac{\mathrm{d}}{\mathrm{dt}} \xi(t)=\mathrm{A} \xi(t), \xi(0)=\mathrm{f}_{0}$,\\
has a unique solution $\xi: \mathbb{R}_{+} \rightarrow D(A)$ in $C^{1}\left(\mathbb{R}_{+}, E\right)$ for every $f_{O} \in D(A)$. In fact, this solution is given by $\xi(t):=T(t) f_{0}$.

For the important relation of semigroups to abstract Cauchy problems we refer to A-II, Section 1. Here we only point out that the above theorem implies that a semigroup is uniquely determined by its generator.\\
While the generator is bounded only for uniformly continuous semigroups (see 2.1 below), it always enjoys a weaker but useful property.

Definition 1.8. An operator $B$ with domain $D(B)$ on a Banach space $E$ is called closed if $D(B)$ endowed with the graph norm

$$
\|f\|_{B}:=\|f\|+\|B E\|
$$

becomes a Banach space. Equivalently, (B,D(B)) is closed if and only if its graph $\{(f, B f): f \in D(B)\}$ is closed in $E \times E$, i.e.\\
(1.5) $\quad f_{n} \in D(B), f_{n} \rightarrow f$ and $B f_{n} \rightarrow g$ implies $f \in D(B)$ and $B f=g$.

It is clear from this definition that the 'closedness' of an operator $B$ depends very much on the size of the domain $D(B)$. For example, a bounded and densely defined operator (B,D(B)) is closed if and only if $D(B)=E$.\\
On the other hand it may happen that $(B, D(B))$ is not closed but has a closed extension (C,D(C)), i.e. $D(B) \subset D(C)$ and $B f=C f$ for every $f \in D(B)$. In that case, $B$ is called closable, a property which is equivalent to the following:\\
(1.6) $\quad \mathrm{f}_{\mathrm{n}} \in \mathrm{D}(\mathrm{B}), \mathrm{f}_{\mathrm{n}} \rightarrow 0$ and $B \mathrm{f}_{\mathrm{n}} \rightarrow \mathrm{g}$ implies $\mathrm{g}=0$.

The smallest closed extension of (B,D(B)) will be called the closure $\bar{B}$ with domain $D(\bar{B})$. In other words, the graph of $\bar{B}$ is the closure of $\{(f, B f): f \in D(B)\}$ in $E \times E$.\\
Finally we call a subset $D_{0}$ of $D(B)$ a core for $B$ if $D_{0}$ is $\|\cdot\|_{B}$-dense in $D(B)$. This means that a closed operator is determined (via closure) by its restriction to a core in its domain.

We now collect the fundamental topological properties of semigroup generators, their domains (see also $\mathrm{A}-\mathrm{II}, \mathrm{Cor.1.34)}$ and their resolvents.

Proposition 1.9. For the generator A of a strongly continuous semigroup ( $T(t))_{t \geqq 0}$ the following holds:\\
(i) The generator A is a closed operator.\\
(ii) If a subspace $D_{O}$ of the domain $D(A)$ is dense in $E$ and (T(t))-invariant, then it is a core for A.\\
(iii) Define $D\left(A^{n}\right):=\left\{f \in D\left(A^{n-1}\right): A f \in D\left(A^{n-1}\right)\right\}, D\left(A^{1}\right)=D(A)$. Then $D\left(A^{\infty}\right):=n_{n \in \mathbb{N}} D\left(A^{n}\right)$ is dense in $E$ and a core for $A$.

Example 1.10. Property (iii) above does not hold for general densely defined closed operators. Take $E=C[0,1], D(B)=C 1[0,1]$ and $B f=q \cdot f^{\prime}$ for some nowhere differentiable function $q \in C[0,1]$. Then $B$ is closed, but $D\left(B^{2}\right)=(0)$.

Proposition 1.11. For the generator A of a strongly continous semigroup $(T(t))_{t \geq 0}$ on a Banach space $E$ the following holds. If $\int_{0}^{\infty} e^{-\lambda t} T(t) f d t$ exists for every $\pounds \in E$ and some $\lambda \in \mathbb{C}$, then $\lambda \in \rho(A)$ and $R(\lambda, A) f=\int_{0}^{\infty} e^{-\lambda t} T(t) f d t$. In particular,\\
\includegraphics[max width=\textwidth]{2024_12_23_c6487cc0859199a15bd9g-016} for every $f \in E, n \geqq 0$ and $\lambda \in \mathbb{C}$ with $\operatorname{Re} \lambda>\omega$.

Remarks 1.12. (1) For continuous Banach space valued functions such as $t \rightarrow T(t) f$ we consider the Riemann integral and define $\int_{0}^{\infty} T(t) f d t$ as $l_{t \rightarrow \infty} \int_{0}^{t} T(s) f d s$. Sometimes such integrals for strongly continuous semigroups $(T(t))_{t \geqslant 0}$ are written as $\int_{a}^{b} T(t) d t$ and understood in the strong sense.\\
(2) Since the generator ( $A, D(A)$ ) determines the semigroup ( $T(t)$ ) $t \geq 0$ uniquely, we will speak occasionally of the growth bound of the generator instead of the semigroup, i.e. we write $\omega=\omega(A)=\omega\left((T(t)){ }_{t \geqslant 0}\right)$. (3) For one-parameter groups it might seem to be more natural to define the generator as the 'derivative' rather than just the 'right derivative' at $t=0$. This yields the same operator as the following result shows:\\
The strongly continuous semigroup $(T(t))_{t \geqq 0}$ with generator $A$ can be extended to a strongly continuous one-parameter group (U(t)) $t \in \mathbb{R}$ if and only if -A generates a semigroup $(S(t))_{t \geqq 0} \cdot$

In that case $(U(t))_{t \in \mathbb{R}}$ is obtained as

$$
U(t):= \begin{cases}T(t) & \text { for } t \geqq 0 \\ S(-t) & \text { for } t \leqq 0\end{cases}
$$

We refer to [Davies (1980), Prop.1.14] for the details.

\section*{2. STANDARD EXAMPLES}
In this section we list and discuss briefly the most basic examples of semigroups together with their generators. These semigroups will reappear throughout this book and will be used to illustrate the theory. We start with the class of semigroups mentioned after Definition 1.1 .

\subsection*{2.1. Uniformly Continuous Semigroups}
It follows from elementary operator theory that for every bounded operator $A \in L(E)$ the sum

$$
\sum_{n=0}^{\infty} t^{n} A^{n} / n!=: e^{t A}
$$

exists and determines a unique uniformly continuous (semi)group ( $\left.e^{t A}\right)_{t \in \mathbb{R}}$ having $A$ as its generator.\\
Conversely, any uniformly continuous semigroup is of this form: If the semigroup $(T(t))_{t \geq 0}$ is uniformly continuous, then $\frac{1}{t} \int_{0}^{t} T(s) d s$ uniformly converges to $T(0)=I d$ as $t \rightarrow 0$. Therefore for some $t^{*}$ $>0$ the operator $\frac{1}{t} \cdot \int_{0}^{t^{\prime}} \mathrm{T}(\mathrm{s}) \mathrm{ds}$ is invertible and every $f \in E$ is of the form $f=\frac{1}{t}, \int_{0}^{t} T(s) g d s$ for some $g \in E$. But these elements belong to $D(A)$ by (1.3), hence $D(A)=E$. Since the generator $A$ is closed and everywhere defined it must be bounded.\\
Remark that bounded operators are always generators of groups, not just semigroups. Moreover the growth bound $\omega$ satisfies $|\omega| \leq\|A\|$ in this situation.

The above characterization of the generators of uniformly continuous semigroups as the bounded operators shows that these semigroups are at least in many aspects - rather simple objects.

\subsection*{2.2. Matrix Semigroups}
The above considerations expecially apply in the situation $E=\mathbb{C}^{n}$. If $n=2$ and $A=\left(a_{i j}\right)_{2 \times 2}$ the following explicit formulas for $e^{t A}$ might be of interest:\\
Set $s:=\operatorname{trace} A, d:=\operatorname{det} A$ and $D:=\left(s^{2}-4 d\right)^{1 / 2}$. Then\\
$e^{t A}=e^{t s / 2} \cdot\left[D^{-1} 2 \sinh (t D / 2) \cdot A+\left(\cosh (t D / 2)-s D^{-1} \sinh (t D / 2)\right) \cdot I d\right]$\\
if $\mathrm{D} \neq 0$,\\
$e^{t A}=e^{t s / 2} \cdot[t A+(1-t s / 2) \cdot \operatorname{ld}]$ if $D=0$, resp. .

\subsection*{2.3. Multiplication Semigroups}
Many Banach spaces appearing in applications are Banach spaces of (real or) complex valued functions over a set X. As the most\\
standard of these "function spaces", we mention the space $C_{0}(X)$ of all continuous complex valued functions vanishing at infinity on a locally compact space $X$, or the spaces $L^{p}(X, \Sigma, \mu), 1 \leqq \mathrm{p} \leqq \infty$, of all (equivalence classes of) p-integrable functions on a o-finite measure space $(X, \Sigma, \mu)$.

On these function spaces $E=C_{O}(X)$, resp. $E=L^{P}(x, \Sigma, \mu)$, there is a simple way to define "multiplication operators" : Take a continuous, resp. measurable function $q: X \rightarrow \mathbb{C}$ and define

$$
M_{q} \pounds:=q \cdot f, \text { i.e. } \quad M_{q} \pounds(x):=q(x) \cdot f(x) \text { for } x \in X \text {, }
$$

for every $f$ in the "maximal" domain $D\left(M_{q}\right):=\{g \in E: q \cdot g \in E\}$. This natural domain is a dense subspace of $\mathrm{C}_{\mathrm{O}}(\mathrm{X}), \operatorname{resp}, \mathrm{L}^{\mathrm{P}}(\mathrm{X}, \mathrm{\Sigma}, \mu)$, for $1 \leqq \mathrm{p}<\infty$. Moreover, $\left(\mathrm{M}_{\mathrm{q}}, D\left(\mathrm{M}_{\mathrm{q}}\right)\right)$ is a closed operator. This is easy in case $E=C_{0}(X)$. For $E=L^{P}(\mu), 1 \leqq p<\infty$, we consider a sequence $\left(f_{n}\right) \subset E$ such that $\lim _{n \rightarrow \infty} f_{n}=f \in E$ and $l i m_{n \rightarrow \infty} q f_{n}=: g$ $€ E$. Choose a subsequence $\left(f_{n(k)}\right)_{k \in N}$ such that $\lim _{k+\infty} f_{n(k)}(x)=$ $f(x)$ and $\lim _{k \rightarrow \infty} q(x) f_{n(k)}(x)=g(x)$ for $\mu$-almost every $x \in x$. Then $g=q \cdot f$ and $f \in D\left(M_{q}\right)$, i.e. ${ }^{M_{q}}$ is closed.\\
For such multiplication operators many properties can be checked quite directly. For example, the following statements are equivalent:\\
(a) $\mathrm{M}_{\mathrm{q}}$ is bounded.\\
(b) $q$ is (u-essentially) bounded.

One has $\left\|M_{q}\right\|=\|q\|_{\infty}$ in this situation.

Observe that on spaces $C(K)$, $K$ compact, there are no densely defined, unbounded multiplication operators.

By defining the multiplication semigroups

$$
T(t) f(x):=\exp (t \cdot q(x)) f(x) \quad, \quad x \in X, f \in E
$$

one obtains the following characterizations.

Proposition. Let $M_{q}$ be a multiplication operator on $E=C_{0}(X)$ or $\mathrm{E}=\mathrm{L}^{\mathrm{P}}(\mathrm{X}, \Sigma, \mu), 1 \leqq \mathrm{p}<\infty$. Then the properties (a) and (b), resp. (a') and ( $\left.b^{\circ}\right)$, are equivalent:\\
(a) $\mathrm{M}_{\mathrm{q}}$ generates a strongly\\
continuous semigroup.\\
(b) $\sup \{\operatorname{Re} q(x): x \in X\}<\infty$.\\
(a') $M_{q}$ generates a uniformly continuous semigroup.\\
(b') $\sup \{|q(x)|: x \in x\}<\infty$.

As a consequence one computes the growth bound of a multiplication semigroup as follows:

\begin{verbatim}
$\omega=\sup \{\operatorname{Re} q(x): x \in X\}$
$\omega=\mu-e s s-\sup \{\operatorname{Re} q(x): x \in X\}$
\end{verbatim}

\begin{verbatim}
in the case E = C CO (X),
in the case E = L P
\end{verbatim}

It is a nice exercise to characterize those multiplication operators which generate strongly continuous groups.\\
We point out that the above results cover the cases of sequence spaces such as $\mathrm{c}_{0}$ or $1^{\mathrm{P}}, 1 \leqq \mathrm{p}<\infty$. An abstract characterization of generators of multiplication semigroups will be given in C-II, Thm.5.13.

\subsection*{2.4. Translation (Semi) Groups}
Let $E$ to be one of the following function spaces $C_{0}\left(\mathbb{R}_{+}\right), C_{0}(\mathbb{R})$ or $L^{P}\left(\mathbb{R}_{+}\right), L^{P}(\mathbb{R})$ for $1 \leqq p<\infty$. Define $T(t)$ to be the (left) translation operator

$$
T(t) f(x):=f(x+t)
$$

for $x, t \in \mathbb{R}_{+}$, resp. $x, t \in \mathbb{R}$ and $f \in E$. Then $(T(t)){ }_{t \geqq 0}$ is a strongly continous semigroup, resp. group of contractions on $E$ and its generator is the first derivative $\frac{d}{d x}$ with 'maximal' domain. In order to be more precise we have to distinguish the cases $E=C_{0}$ and $\mathrm{E}=\mathrm{L}^{\mathrm{P}}$ :\\
(i) The generator of the translation (semi)group on $E=C_{0}\left(\mathbb{R}_{+}\right)$is

$$
\begin{aligned}
& \text { Af }:=\frac{d}{d x} f=f^{\prime}, \\
& D(A):=\left\{f \in E: f \text { differentiable and } f^{\prime} \in E\right\} .
\end{aligned}
$$

Proof. For $f \in D(A)$ it follows that for every $x \in \mathbb{R}_{(+)}$

$$
\lim _{h \rightarrow 0} \frac{T(h) f(x)-f(x)}{h}=\lim _{h \rightarrow 0} \frac{f(x+h)-f(x)}{h} \text { exists }
$$

(uniformly in $x$ ) and coincides with $A f(x)$. Therefore $f$ is differentiable and f' $\epsilon$ E.\\
On the other hand, take $f \in E$ differentiable such that $f^{\prime} \in E$. Then

$$
\left|\frac{f(x+h)-f(x)}{h}-f^{\prime}(x)\right| \leqq \frac{1}{h} \int_{x}^{x+h}\left|f^{\prime}(y)-f^{\prime}(x)\right| d y
$$

where the last expression tends to zero uniformly in x as $\mathrm{h} \rightarrow 0$. Thus $f \in D(A)$ and $f^{\prime}=A f$.\\
(ii) The generator of the translation (semi)group on $E=L^{P}(\mathbb{R}(+))$, $1 \leqq \mathrm{p}<\infty$, is

$$
\begin{aligned}
& \text { Af }:=\frac{d}{d x} f=f^{\prime}, \\
& D(A):=\left\{f \in E: f \text { absolutely continuous, } f^{\prime} \in E\right\} .
\end{aligned}
$$

Proof. Take $f \in D(A)$ such that $\lim _{h \rightarrow 0} \frac{1}{h}(T(h) f-f)=g \in E$. Since integration is continuous we obtain for every $a, b \in \mathbb{R}(+)$ that\\
(\textit{) $\frac{1}{h} \int_{b}^{b+h} f(x) d x-\frac{1}{h} \int_{a}^{a+h} f(x) d x=\int_{a}^{b} \frac{f(x+h)-f(x)}{h} d x$\\
converges to $\int_{a}^{b} g(x) d x$ as $h \rightarrow 0+$. But for almost all a, b the left hand side of (}) converges to $f(b)-f(a)$. By redefining $f$ on a nullset we obtain

$$
f(y)=\int_{a}^{y} g(x) d x+f(a), \quad y \in \mathbb{R}(+)
$$

which is an absolutely continuous function whose derivative is (almost everywhere) equal to $g$.

On the other hand, let $f$ be absolutely continuous such that $f^{\prime} \in L^{p}$. Then

$$
\begin{aligned}
& \lim _{h \rightarrow 0} \int \frac{f(x+h)-f(x)}{h}-\left.f^{\prime}(x)\right|^{p} d x \\
= & \lim _{h \rightarrow 0} \int \frac{1}{h}\left|\int_{0}^{h}\left(f^{\prime}(x+s)-f^{\prime}(x)\right) d s\right|^{p} d x \\
= & \lim _{h \rightarrow 0} \int\left|\int_{0}^{1}\left(f^{\prime}(x+u h)-f^{\prime}(x)\right) d u\right|^{p} d x \\
\leq & \lim _{h \rightarrow 0} \iint_{0}^{1}\left|f^{\prime}(x+u h)-f^{\prime}(x)\right|^{p} d u d x \\
= & \iint_{0}^{1} \lim _{h \rightarrow 0} \int\left|f^{\prime}(x+u h)-f^{\prime}(x)\right|^{p} d x d u=0 \text {, hence } f \in D(A) .
\end{aligned}
$$

\subsection*{2.5. Rotation Groups}
On $E=C(\Gamma)$, resp. $E=L^{\mathrm{P}}(\mathrm{I}, \mathrm{m}), 1 \leqq \mathrm{p}<\infty, \mathrm{m}$ Lebesgue measure we have canonical groups defined by rotations of the unit circle $\Gamma$ with a certain period, i.e. for $0<\tau \in \mathbb{R}$ the operators

$$
R_{\tau}(t) f(z):=f\left(e^{2 \pi i t / \tau} \cdot z\right)
$$

yield a group $\left(R_{\tau}(t)\right)_{t \in \mathbb{R}}$ having period $\tau$, i.e. $R_{\tau}(\tau)=I d$. As in Example 2.4 one shows that its generator has the form

\begin{verbatim}
$D(A)=\left\{f \in E ; f\right.$ absolutely continuous, $\left.f^{\prime} \in E\right\}$,
$A f(z)=(2 \pi i / \tau) \cdot z \cdot f '(z)$.
\end{verbatim}

An isomorphic copy of the group $\left(R_{1}(t)\right)_{t \in \mathbb{R}}$ is obtained if we consider $E=\{f \in C[0,1]: f(0)=f(1)\}$, resp. $E=L^{p}([0,1])$ and the group of 'periodic translations'

$$
T(t) f(x):=f(y) \quad \text { for } y \in[0,1], y=x+t \bmod 1
$$

with generator

\begin{verbatim}
D(A) := {f E E : f absolutely continuous, f' € E},
Af := f'.
\end{verbatim}

\subsection*{2.6. Nilpotent Translation Semigroups}
Take $E=L^{P}([0, \tau], m)$ for $1 \leqq p<\infty$ and define

$$
T(t) f(x):=\left\{\begin{array}{cl}
f(x+t) & \text { if } x+t \leq \tau \\
0 & \text { otherwise }
\end{array}\right.
$$

Then (T(t)) $t \geqq 0$ is a semigroup satisfying $T(t)=0$ for $t \geqq \tau$. Its generator is still the first derivative $A=\frac{d}{d x}$, but its domain is $D(A)=\left\{f \in E: f\right.$ absolutely continuous, $\left.\mathrm{f}^{\prime} \in \mathrm{E}, \mathrm{f}(\mathrm{f})=0\right\}$. In fact, if $f \in D(A)$ then $f$ is absolutely continuous with $\pounds^{\prime} \in E$. By Prop.1.6.i it follows that $T(t) f$ is absolutely continuous and hence $f(\tau)=0$.

\subsection*{2.7. One-dimensional Diffusion Semigroup}
For the second derivative

$$
B f(x):=\frac{d^{2}}{d x^{2}} f(x)=f^{\prime}(x)
$$

we take the domain

$$
D(B):=\left\{f \in C^{2}[0,1]: f^{\prime}(0)=f^{\prime}(1)=0\right\}
$$

in the Banach space $E=C[0,1]$. Then $D(B)$ is dense in $C[0,1]$, but closed for the graph norm. Obviously, each function

$$
e_{n}(x):=\cos \pi n x, \quad n \in \mathbb{Z} \text {, }
$$

is contained in $D(B)$ and an eigenfunction of $B$ pertaining to the eigenvalue $\lambda_{n}:=-\pi^{2} n^{2}$. The linear hull\\
$\operatorname{span}\left\{e_{n}: n \in \mathbb{Z}\right\}=: E_{o}$\\
forms a subalgebra of $D(B)$ which by the Stone-Weierstrass theorem is dense in E .\\
We now use $e_{n}$ to define bounded linear operators

$$
e_{n} \otimes e_{n}: f \rightarrow\left(\int_{0}^{1} f(x) e_{n}(x) d x\right) e_{n}=\left\langle f, e_{n}>e_{n}\right.
$$

satisfying $\left\|e_{n} * e_{n}\right\| \leqslant 1$ and

$$
\left(e_{n} \otimes e_{n}\right)\left(e_{m} \otimes e_{m}\right)=\delta_{n, m}\left(e_{n} \otimes e_{n}\right) \text { for } n \in z
$$

For $t>0$ we define

$$
\begin{aligned}
T(t) & =\sum_{n \in \mathbb{Z}} \exp \left(-\pi^{2} n^{2} t\right) \cdot e_{n} e_{n} \\
& =e_{0} \otimes e_{0}+2 \sum_{n=1}^{\infty} \exp \left(-\pi^{2} n^{2} t\right) \cdot e_{n} \otimes e_{n}
\end{aligned}
$$

or

$$
\begin{aligned}
& T(t) f(x)=\int_{0}^{1} k_{t}(x, y) f(x) d y \\
& \quad \text { where } k_{t}(x, y)=1+2 \sum_{n=1}^{\infty} \exp \left(-\pi^{2} n^{2} t\right) \cos \pi n x \cos \pi n y
\end{aligned}
$$

The Jacobi identity

$$
\begin{aligned}
w_{t}(x): & =1 /(4 \pi t)^{\frac{1}{2}} \quad \sum_{m \in Z} \exp \left(-(x+2 m)^{2} / 4 t\right) \\
& =\frac{1}{2}+\sum_{n \in \mathbb{N}} \exp \left(-\pi^{2} n^{2} t\right) \cos \pi n x
\end{aligned}
$$

and trigonometric relations show that

$$
k_{t}(x, y)=w_{t}(x+y)+w_{t}(x-y)
$$

which is a positive function on $[0,1]^{2}$. Therefore $T(t)$ is a bounded operator on $C[0,1]$ with

$$
\|T(t)\|=\|T(t) 1\|=\sup _{x \in[0,1]} \int_{0}^{1} k_{t}(x, y) d y=1
$$

From the behavior of $T(t)$ on the dense subspace $E_{0}$ it follows that $(T(t))_{t \geq 0}$ with $T(0)=I d$ is a strongly continuous semigroup on $E$ and its generator $A$ coincides with $B$ on $E_{0}$. Finally we observe that $E_{0}$ is a core for ( $A_{r} D(A)$ by Prop.1.9(ii).\\
Consequently $(T(t))$ t 1 is the semigroup generated by the closure of the second derivative with domain $\mathrm{D}(\mathrm{B})$.

\section*{2.8. n-dimensional Diffusion Semigroup}
On $E=L^{P}\left(\mathbb{R}^{\mathrm{n}}\right), 1 \leqq \mathrm{P}<\infty$, the operators

$$
\begin{aligned}
T(t) f(x) & :=(4 \pi t)^{-n / 2} \int_{\mathbb{R}^{n}} \exp \left(-|x-y|^{2 / 4 t) f(y)} d y\right. \\
& :=\mu_{t}^{*} f(x)
\end{aligned}
$$

for $x \in \mathbb{R}^{n}, t>0$ and $\mu_{t}(x):=(4 \pi t)^{-n / 2} \exp \left(-|x|^{2 / 4 t)}\right.$ form a strongly continuous semigroup:\\
In fact the integral exists for every $f \in L^{p}\left(\mathbb{R}^{n}\right)$, since $\mu_{t}$ is an element of the schwartz space $S\left(\mathbb{R}^{n}\right)$ of all rapidly decreasing smooth functions on $\mathbb{R}^{n}$.\\
Moreover,

$$
\|T(t) f\|_{p} \leqq\left\|\mu_{t}\right\|_{I}\|f\|_{p}=\|f\|_{p}
$$

by Young's inequality [Reed-Simon (1975), p.28], hence $\|T(t)\| \leq 1$ for every $t>0$. Next we observe that $S\left(\mathbb{R}^{n}\right)$ is dense in $E$ and invariant under each $T(t)$. Therefore we can apply the fourier trans-\\
formation $F$ which leaves $S\left(\mathbb{R}^{n}\right)$ invariant and yields

$$
F\left(\mu_{t}{ }^{*} f\right)=(2 \pi)^{n / 2} F\left(\mu_{t}\right) \cdot F(f)=(2 \pi)^{n / 2} \vec{\mu}_{t} \cdot \vec{F}
$$

where $\pounds \in S\left(\mathbb{R}^{n}\right)$ ,点 $=F \mathrm{f} \in S\left(\mathbb{R}^{n}\right)$.\\
In other words, $F$ transforms $\left(\left.T(t)\right|_{S\left(\mathbb{R}^{n}\right)^{\prime}}{ }_{t \geqslant 0}\right.$ into a multiplication semigroup on $S\left(\mathbb{R}^{n}\right)$ which is pointwise continuous for the usual topology of $S\left(\mathbb{R}^{n}\right)$. The generator, i.e. the right derivative at 0 , of this semigroup is the multiplication operator

$$
B \vec{f}(x):=-|x|^{2} \vec{f}(x)
$$

for every $f \in S\left(\mathbb{R}^{n}\right)$.\\
Applying the inverse Fourier transformation and observing that the topology of $S\left(\mathbb{R}^{n}\right)$ is finer than the topology induced from $L^{p}\left(\mathbb{R}^{n}\right)$, we obtain that $(T(t))_{t \geqq 0}$ is a semigroup which is strongly continuous (use Remark 1.2,(3)) and its generator A coincides with

$$
\Delta f(x)=\sum_{i=1}^{n} \frac{\delta^{2}}{\delta x^{2}} \quad f\left(x_{1}, \ldots, x_{n}\right)
$$

for every $f \in S\left(\mathbb{R}^{n}\right)$.\\
Since $S\left(\mathbb{R}^{n}\right)$ is (T(t))-invariant we have determined the generator on a core of its domain (see prop.1.9.ii).\\
In particular the above semigroup 'solves' the initial value problem for the 'heat equation'

$$
\frac{\delta}{\delta t} f(x, t)=\Delta f(x, t), f(x, 0)=f_{0}(x), x \in \mathbb{R}^{n}
$$

For the analogous discussion of the unitary group on $L^{2}\left(\mathbb{R}^{\mathrm{n}}\right)$ generated by

$$
\mathrm{C}:=\mathrm{i} \Delta
$$

we refer to Section IX. 7 in Reed-Simon (1975).\\
Analogous examples to 2.7 are valid in $\mathrm{L}^{\mathrm{P}}[0,1]$, resp. to 2.8 in $c_{0}\left(R^{n}\right)$.

\section*{3. STANDARD CONSTRUCTIONS}
Starting with a semigroup $(T(t))_{t \geqq 0}$ on a Banach space $E$ it is possible to construct new semigroups on spaces naturally associated with E. Such constructions will be important technical devices in many of the subsequent proofs. Although most of these constructions are rather routine, we present in the sequel a systematic account of them for the convenience of the reader.\\
We always start with a semigroup $(T(t))_{t \geqq 0}$ on a Banach space $E$, and denote its generator by $A$ on the domain $D(A)$.

\subsection*{3.0. Similar Semigroups}
There is an easy way how to obtain different (but isomorphic) semi-\\
groups out of a given semigroup $(T(t))_{t \geqslant 0}$ on a Banach space E . Let $V$ be an isomorphism from $E$ onto $E$. Then $s(t):=V T(t) \mathrm{V}^{-1}$, $t \geqq 0$, defines a strongly continuous semigroup. If $A$ is the generator of $(T(t))_{t \geqq 0}$ then\\
B. := $\operatorname{VAV}^{-1}$ with domain $D(B):=\left\{f \in E: V^{-1} f \in D(A)\right\}$\\
is the generator of $\left(S(t) t_{t \geqslant 0}\right.$.

\subsection*{3.1. The Rescaled Semigroup}
For fixed $\lambda \in \mathbb{C}$ and $\alpha>0$ the operators

$$
S(t):=\exp (\lambda t) T(\alpha t)
$$

yield a new semigroup having generator

$$
B:=a A+\lambda I d \text { with } D(B)=D(A) \text {. }
$$

This 'rescaled semigroup' enjoys most of the properties of the original semigroup and the same is true for the corresponding generators. However, by using this procedure certain constants associated with $(T(t))_{t \geqq 0}$ and $A$ can be normalized. For example, by this rescaling we may in many cases suppose without loss of generality that the growth bound $\omega$ is zero.\\
Another application is the following: For $\lambda \in \mathbb{C}$ and\\
$S(t):=\exp (-\lambda t) T(t)$ the formulas (1.3) and (1.4) yield:


\begin{align*}
& e^{-\lambda t} T(t) f-f=(A-\lambda) \int_{0}^{t} e^{-\lambda s_{T}} T(s) f d s \\
& \text { or }  \tag{3.1}\\
& \left(e^{\lambda t}-T(t)\right) f=(\lambda-A) \int_{0}^{t} e^{\lambda(t-s)} T(s) f d s \quad \text { for } f \in E,
\end{align*}


and


\begin{align*}
& e^{-\lambda t} T(t) f-f=\int_{0}^{t} e^{-\lambda s} T(s)(A-\lambda) f d s \\
& \text { or }  \tag{3.2}\\
& \left(e^{\lambda t}-T(t)\right) f=\int_{0}^{t} e^{\lambda(t-s)} T(s)(\lambda-A) f d s \quad \text { for } f \in D(A) .
\end{align*}


\subsection*{3.2. The Subspace Semigroup}
Assume $F$ to be a closed ( $T(t)$ )-invariant or, equivalently, $R(\lambda, A)$-invariant $(\lambda \in \mathbb{C}$, Re $\lambda \omega)$ subspace of $E$. Then the semigroup ( $\left.T(t)\right|^{\prime} t \geqq 0$ of all restrictions $T(t)|:=T(t)| F$ is strongly continuous on F. If (A,D(A)) denotes the generator of $(T(t))_{t \geqq 0}$ it follows from the (T(t))-invariance and closedness of $F$ that $A$ maps $D(A) \cap F$ into $F$. Therefore\\
$A_{\mid}:=A \mid D(A) \cap F$ with domain $D(A \mid):=D(A) \cap F$\\
is the generator of $(T(t) \mid)$.

Conversely, if $F$ is a closed 'linear subspace of $E$ with $A(D(A) \cap F) \subset F$ such that $A$ is a generator on $F$, then $F$ is (T(t))-invariant.\\
An A-invariant subspace need not necessarily be (T(t))-invariant: Take for example the translation group with $\mathrm{T}(\mathrm{t}) \mathrm{f}(\mathrm{x})=\mathrm{f}(\mathrm{x}+\mathrm{t})$ on $E=C_{0}(\mathbb{R})$ and $F:=\{f \in E: f(x)=0$ for $x \leqq 0\}$.

\subsection*{3.3. The Quotient Semigroup}
Let $F$ be a closed ( $T(t)$ )-invariant subspace of $E$ and consider the quotient space $\mathrm{E}_{/}:=\mathrm{E}_{/ \mathrm{F}}$ with quotient map $\mathrm{q}: \mathrm{E} \rightarrow \mathrm{E}_{/}$. The quotient operators

$$
T(t), q(f):=q(T(t) f), f \in E,
$$

are well defined and form a strongly continuous semigroup

$$
\left(T(t), \rho^{\prime} t \geq 0\right.
$$

on $E /$. For the generator $(A, D(A /))$ of $\left(T(t),{ }^{\prime} t \geq 0\right.$ the following holds:

$$
D(A,)=q(D(A)) \quad \text { and } \quad A, q(f)=q(A f)
$$

for every $f \in D(A)$. Here we use the fact that every $\hat{f}:=q(f) \epsilon$ $D(A$,$) can be written as$\\
$\hat{f}=\int_{0}^{\infty} e^{-\lambda s} \hat{T}(s), \hat{g} d s=\int_{0}^{\infty} e^{-\lambda s} q(T(s) g) d s=q\left(\int_{0}^{\infty} e^{-\lambda s} T(s) g d s\right)=q(h)$\\
where $h \in D(A)$ and $\lambda>\omega$ (see(1.6)). In particular we point out that for every $\hat{\mathbf{f}} \in D(A$,$) there exist representatives f \in \hat{\mathbf{f}}$ belonging to $D(A)$.

Example. We start with the Banach space $E=L^{1}(\mathbb{R})$ and the translation semigroup $(T(t))_{t \geqq 0}$ where $T(t) f(x):=f(x+t)$ (see Example 2.4). Then $L^{1}((-\infty, 1])$ can be identified with the closed, ( $\left.T(t)\right)$-invariant subspace

\begin{verbatim}
J :={f { E : f(x) = 0 for l < x < \ }
\end{verbatim}

and we obtain the subspace semigroup

$$
T(t) f^{f(x)}=I_{(-\infty, 1]}(x) \cdot f(x+t),
$$

where $f \in L^{1}((-\infty, 1]),-\infty<x \leqq 1$ and $t \geqq 0$.\\
By 2.4 and 3.2 its generator is\\
$A \mid f:=f^{\prime}$\\
for $f^{\prime} \in D(A):,=\left\{f \in E: f \in A C\right.$ with $f^{\prime} \in E$ and $f(x)=0$ for $\left.x \geqq 1\right\}$.\\
Next we identify $L^{l}([0,1])$ with the quotient space $L^{1}((-\infty, 1])$ Next we identify $\mathrm{L}^{1}([0,1])$ with the quotient space $\mathrm{L}^{1}((-\infty, 1]) / I$ where\\
$I:=\left\{f \in \mathrm{~L}^{1}((-\infty, 1\}): f(x)=0\right.$ for $\left.0 \leqq x \leq 1\right\}$.\\
Again $I$ is invariant for the restricted semigroup $(T(t)$, and the\\
quotient semigroup $(T(t) \mid /)$ on $L^{1}([0,1])$ is the nilpotent translation semigroup as in Example 2.6. In particular it follows that the domain of the generator is\\
$D(A \mid /)=\left\{f \in L^{1}([0,1]): f \in A C\right.$ with $f^{\prime} \in L^{1}([0,1])$ and $\left.f(1)=0\right\}$.

\subsection*{3.4. The Adjoint Semigroup}
The adjoint operators $\left(T(t)^{\prime}{ }_{t \geqslant 0}\right.$ of a strongly continuous semigroup ( $\mathrm{T}(\mathrm{t}))_{t} \geqslant 0$ on a Banach space $E$ form a semigroup on $E^{\prime}$ which need, however, not be strongly continuous.

Example. Take the translation operators $\mathrm{T}(\mathrm{t}) \mathrm{f}(\mathrm{x})=\mathrm{f}(\mathrm{x}+\mathrm{t}) \quad$ on $E=I^{1}(\mathbb{R}) \quad$ (see Example 2.4) and their adjoints

$$
T(t)^{\prime} f(x)=f(x-t)
$$

on $E^{\prime}=L^{\infty}(\mathbb{R})$. Then $\left(T(t)^{\prime}\right)_{\infty} t \in \mathbb{R}$ is a one-parameter group which is not strongly continuous on $L^{\infty}(\mathbb{R})$ (take any non-trivial characteristic function).

Since the semigroup $\left(T(t)^{\prime}\right)_{t \geq 0}$ is obviously weak*-continuous in the sense that $\lim _{t \rightarrow s}\left\langle f,\left(T(t)^{\prime}-T(s)^{\prime}\right) \phi>=0\right.$ for every $f \in E, \phi \in E^{\prime}$ and $s, t \geqq 0$, it is natural to associate $\left(T(t)^{\prime}\right)_{t} \geq_{0}$ its a weak*\_ generator

$$
\begin{aligned}
& A^{\prime} \phi:=\sigma\left(E^{\prime}, E\right)-1 i m \frac{1}{h}\left(T(h)^{\prime} \phi-\phi\right) \quad \text { for every } \phi \text { in the domain } \\
& D\left(A^{\prime}\right):=\left\{\phi \in E^{\prime}: \sigma\left(E^{\prime}, E\right)-\lim \frac{1}{h}\left(T(h)^{\prime} \phi-\phi\right) \text { exists }\right\} .
\end{aligned}
$$

This operator coincides with the adjoint of the generator $(A, D(A))$, i.e.\\
$D\left(A^{\prime}\right)=\left\{\phi \in E^{\prime}\right.$; there exists $\psi \in E^{\prime}$ such that $\langle f, \psi\rangle=\langle A f, \phi\rangle$ for all $E \in D(A)\}$\\
and $A^{\prime} \phi=\Psi$.\\
In particular, A' is a closed and $\sigma\left(E^{\prime}, E\right)$-densely defined operator in $\mathrm{E}^{\prime}$.

It follows from Thm.III.5.30 in Kato (1966) that the resolvent $R\left(\lambda, A^{\prime}\right)$ of $A^{\prime}$ is $R(\lambda, A)^{\prime}$. In particular, the spectra $\sigma(A)$ and $\sigma\left(A^{\prime}\right)$ coincide. But it is still necessary in some situations to have strong continuity for the adjoint semigroup. In order to achieve this we restrict $T(t)$ to an appropriate subspace of $E^{\prime}$.

Definition (Phillips, 1955). The semigroup dual of the Banach space E with respect to the strongly continuous semigroup $(T)(t){ }_{t \geq 0}$ is

$$
E^{*}:=\left\{\phi \in E^{\prime}:\|\cdot\|-\lim _{t \rightarrow 0} T(t)^{\prime} \phi=\phi\right\} .
$$

The adjoint semigroup on $\mathrm{E}^{*}$ is given by the operators\\
$T(t)$ * $:=T(t)^{\prime} \mid E^{*}, t \geqq 0$.\\
since $\left(T(t)^{*}\right)_{t \geqq 0}$ is strongly continuous on $E^{*}$ we call its generator (A*,D(A*)) the adjoint generator.

The above definition makes sense since $E^{*}$ is norm-closed in $E^{\prime}$ and (T(t)')-invariant . The main point is that E* is still reasonably large. In fact, since $\int_{0}^{t} \mathrm{~T}(s)^{\prime} \phi \mathrm{ds}$, understood in the weak sense, is contained in $E^{*}$ for every $\phi \in E^{\prime}, t \geqq 0$ it follows that $\sup \left\{\langle f, \phi\rangle: \phi \in E^{*},\|\phi\| \leqq 1\right\} \leqq\|f\| \leqq M \cdot \sup \left\{\langle f, \phi\rangle: \phi \subseteq E^{*},\|\phi\| \leqq 1\right\}$ where $M:=\lim \sup _{t \rightarrow 0}\|T(t)\|$ In particular, $E^{*}$ separates $E$, i.e. $E^{*}$ is $\sigma\left(E^{\prime}, E\right)$-dense in $E^{\prime}$. In addition the estimate of $\|\cdot\|$ given above yields

$$
\|T(t) *\| \leqq\|T(t)\| \leqq M\|T(t) *\| \quad \text { for all } t \geqq 0
$$

In the following proposition we describe the relation between A* and $A^{\prime}$.

Proposition. For the adjoint generator $A *$ of a strongly continuous semigroup $(T(t))_{t \geq 0}$ on $E$ the following assertions hold:\\
$E^{*}$ is the $\|\cdot\|-c l o s u r e ~ o f ~ D\left(A^{\prime}\right)$.\\
$D\left(A^{*}\right)=\left\{\phi \in D\left(A^{\prime}\right): A^{\prime} \phi \in E^{*}\right\}$.\\
$A^{*}$ and $A^{\prime}$ coincide on $D\left(A^{*}\right)$.

Proof. (i) Take $\phi \in D\left(A^{\prime}\right)$ fixed. For every $f \in D(A)$ with $\|f\| \leqq 1$ we define a continuously differentiable function

$$
t \rightarrow \xi_{f}(t):=\langle T(t) f, \phi\rangle
$$

on $[0,1]$ with dexivative $\xi_{f}^{\prime}(t)=\langle T(t) A f, \phi\rangle=\left\langle T(t) f, A^{\prime} \phi\right\rangle$.\\
Since $\left\{\xi_{f}^{\prime}(t): t \in[0,1], f \in D(A),\|f\| \leqq 1\right\}$ is bounded it follows that the set\\
$\left\{\xi_{\pounds}: f \in D(A),\|f\| \leqq 1\right\}$\\
is equicontinuous at 0 , i.e. for every $\varepsilon>0$ there exists $0<t_{0}<1$ such that\\
$\left|\xi_{f}(s)-\xi_{f}(0)\right|=\left|<f, T(s)^{\prime} \phi-\phi>\right|<\varepsilon$\\
for every $0 \leqq s \leqq t_{0}$ and $f \in D(A),\|f\| \leqq 1$. But this implies $\left\|\mathrm{T}(\mathrm{s})^{\prime} \phi^{\prime}-\phi^{\prime}\right\|<\varepsilon$ and hence $\phi \in \mathrm{E}^{*}$. Conversely take $\psi \in E^{*}$. Then $\frac{1}{t} \int_{0}^{t} T(s)^{\prime} \psi d s, t>0$, belongs to $D$ (A') and norm converges to $\psi$ as $t \rightarrow 0$, i.e. $\psi$ belongs to the norm closure of D(A') .\\
(ii) and (iii): since the weak* topology on $E^{\prime}$ is weaker than the norm topology it follows that $A^{\prime}$ is an extension of $A^{*}$.\\
Now take $\phi \in D\left(A^{\prime}\right)$ such that $A^{\prime} \phi \in E^{*}$. As above define the func-\\
tions $\xi_{f}$. The assumption on $\phi$ implies the set of all derivatives $\left\{\xi_{f}^{\prime}: f \in D(A),\|f\| \leq 1\right\}$\\
to be equicontinuous at $t=0$. This means that for every $\varepsilon>0$ there exists $0<t_{o}<1$ such that $\left|\xi_{f}^{\prime}(0)-\xi_{f}^{\prime}(s)\right|<\varepsilon$ for every $\mathrm{f} \in \mathrm{D}(\mathrm{A}),\|\mathrm{f}\| \leq 1$ and $0<s<t_{0}$.\\
In particular,

$$
\varepsilon>\left|\xi_{f}^{\prime}(0)-\frac{1}{s}\left(\xi_{f}(s)-\xi_{f}(0)\right)\right|=\left\lvert\,\left\langle f, A^{\prime} \phi-\frac{1}{s}\left(T(s)^{\prime} \phi-\phi\right)\right\rangle 1\right.,
$$

hence\\
$\varepsilon>\left\|A^{\prime} \phi-\frac{1}{s}(T(s) ' \phi-\phi)\right\|$\\
for all $0 \leq s \leq t_{0}$. From this it follows that $\phi \in D\left(A^{*}\right)$.

On reflexive Banach spaces we have $A *=A$ by the above proposition. In other cases this construction is more interesting.

Example(continued). The adjoints of the (left) translation $T(t)$ on $E=L^{1}(\mathbb{R})$ are the (right) translations $T(t)$ on $E^{\prime}=L^{\infty}(\mathbb{R})$. The largest subspace of $I^{\infty}(\mathbb{R})$ on which these translations form a semigroup which is strongly continuous with respect to the sup-norm, is the space of all bounded uniformly continuous functions on $\mathbb{R}$, i.e. $E^{*}=c_{b u}(\mathbb{R})$.\\
Calculating $D\left(A^{\prime}\right)$ and $D\left(A^{*}\right)$ respectively, one obtains\\
$D\left(A^{\prime}\right)=\left\{f \in L^{\infty}(\mathbb{R}): f \in A C, f^{\prime} \in L^{\infty}(\mathbb{R})\right\}$,\\
$D\left(A^{*}\right)=\left\{f \in L^{\infty}(\mathbb{R}): f \in C^{1}(\mathbb{R}), f^{\prime} \in C_{b u}(\mathbb{R})\right\}$.\\
Obviously, the function $x \rightarrow|\sin x|$ belongs to $D\left(A^{\prime}\right)$ but not to $\mathrm{D}\left(\mathrm{A}^{*}\right)$.

\subsection*{3.5. The Associated Sobolev Semigroups}
Since the generator $A$ of a strongly continuous semigroup ( $T(t))_{t \geq 0}$ is closed, its domain $D(A)$ becomes a Banach space for the graph norm $\|\mathrm{f}\|_{1}:=\|\mathrm{f}\|+\|\mathrm{Af}\|$.\\
We denote this Banach space by $E_{1}$ and the continuous injection from $E_{1}$ into $E$ by $i_{1}$. Since $E_{1}$ is invariant under $(T(t))_{t \geqslant 0}$ - apply Prop.1.6.i - it makes sense to consider the semigroup $\left(T_{1}(t)\right)_{t \geq 0}$ of all restrictions $T_{1}(t):=T(t) \mid E_{1}$. The results of Prop. 1.6 imply that $T_{1}(t) \in L\left(E_{1}\right)$ and $\left\|T_{1}(t) f-f\right\|_{1} \rightarrow 0$ as $t \rightarrow 0$ for every $f \in E_{1}$. Thus $\left(T_{1}(t)\right)_{t \geq 0}$ is a strongly continuous semigroup on $\mathrm{E}_{1}$ and has a generator denoted by ( $A_{1}, \mathrm{D}\left(\mathrm{A}_{1}\right)$ ) . Using Prop.1.6 again we see that $A_{1}$ is the restriction of $A$ to $E_{1}$ with maximal domain, i.e.

\begin{verbatim}
D(A)
A
\end{verbatim}

It is now possible to repeat this construction in order to obtain Banach spaces $E_{n}$ and semigroups $\left(T_{n}(t)\right)_{t \geqq 0}$ with generators $\left(A_{n}, D\left(A_{n}\right)\right.$ which are related as visualized in the following diagram:\\
\includegraphics[max width=\textwidth, center]{2024_12_23_c6487cc0859199a15bd9g-029(1)}

For the translation semigroup on $L^{P}(\mathbb{R})$ (see 2.3 ) the above construction leads to the usual 'Sobolev spaces'. Therefore we might call $\mathrm{E}_{\mathrm{n}}$ the $n-t h$ sobolev space and $\left(T_{n}(t)\right)_{t \geq 0}$ the $n-t h$ sobolev semigroup associated to $E$ and $(T(t))_{t \geqq 0}$.

Remarks: 1. For $\lambda \in \rho(A)$ the operator $(\lambda-A)$ and the resolvent $R(\lambda, A)$ are isomorphisms from $E_{1}$ onto $E$, resp. from $E$ onto $E_{1}$ (show that $\|\cdot\|_{1}$ and $\|\cdot\|_{\lambda}$ with $\|\cdot\|_{\lambda}:=\|(\lambda-A) \cdot\|$ are equivalent). In addition, the diagram\\
\includegraphics[max width=\textwidth, center]{2024_12_23_c6487cc0859199a15bd9g-029}\\
commutes. Therefore all Sobolev semigroups $\left(E_{n},\left(T_{n}(t)\right)_{t \geqslant 0}\right), n \in \mathbb{N}$, are isomorphic.\\
2. For $\lambda \in \rho(A)$ consider the norm\\
$\|f\|_{-1}:=\|R(\lambda, A) f\|$\\
for every $f \in E$ and define $E_{-1}$ as the completion of $E$ for $\left\|_{-1} \cdot\right\|_{-1}$.

Then $(T(t))_{t \geq 0}$ extends continuously to a strongly continuous semigroup $\left(T_{-1}(t)\right)_{t \geqq 0}$ on $E_{-1}$ and the above diagram can be extended to the negative integers.

\subsection*{3.6. The F-product Semigroup}
It is a very successful mathematical method to consider a sequence of points in a certain space as a point in a new and larger space. In particular such a method can serve to convert an approximate eigenvector of a linear operator into an eigenvector. Occasionally we will need such a construction and refer to section V.I of Schaefer (1974) for the details.\\
If we try to adapt this construction to strongly continuous semigroups we encounter the difficulty that the semigroup extended to the larger space will not remain strongly continuous. An idea already used in 3.4 will help to overcome this difficulty.

Let $T=(T(t))_{t \geqq 0}$ be a strongly continuous semigroup on the Banach space $E$. Denote by $m(E)$ the Banach space of all bounded E-valued sequences endowed with the norm

$$
\left\|\left(f_{n}\right)_{n \in N}\right\|:=\sup \left\{\left\|f_{n}\right\|: n \in N\right\} .
$$

It is clear that every $T(t)$ extends canonically to a bounded linear operator

$$
\hat{T}(t)\left(f_{n}\right):=\left(T(t) f_{n}\right)
$$

on $m(E)$, but the semigroup $(\hat{T}(t))_{t \geqq 0}$ is strongly continuous if and only if $T$ has a bounded generator. Therefore we restrict our attention to the closed, ( $\hat{T}(t))$-invariant subspace\\
$m^{\top}(E):=\left\{\left(f_{n}\right) \in m(E): \lim _{t \rightarrow 0}\left\|T(t) f_{n}-f_{n}\right\|=0\right.$ uniformly for $\left.n \in \mathbb{N}\right\}$. Then the restricted semigroup

$$
\tilde{T}(t)\left(f_{n}\right)=\left(T(t) f_{n}\right),\left(f_{n}\right) \in m^{\top}(E),
$$

is strongly continuous and we denote its generator by ( $\tilde{A}, \bar{D}(\tilde{A})$ ) . The following lemma shows that $A$ is obtained canonically from A .

Lemma. For the generator $\tilde{\sim} \tilde{A}$ of $(\tilde{T}(t))_{t \geq 0}$ on $m^{\top}$ (E) one has:


\begin{align*}
& D(\tilde{A})=\left\{\left(f_{n}\right) \in m^{\top}(E): f_{n} \in D(A) \text { and }\left(A f_{n}\right) \in m^{\top}(E)\right\},  \tag{i}\\
& \tilde{A}\left(f_{n}\right)=\left(A f_{n}\right) \text { for }\left(f_{n}\right) \in D(A) \text {. } \tag{ii}
\end{align*}


For the proof we refer to Lemma 1.4. of Derndinger (1980).

Now let $F$ be any filter on $N$ finer than the Frechet filter (i.e. the filter of sets with finite complement). (In most cases $F$ will be either the Frechet filter or some free ultra filter.) Then the sub-\\
space of all F-null sequences in m(E)

$$
c_{F}(E):=\left\{\left(f_{n}\right) \in m(E): F-1 i m\left\|f_{n}\right\|=0\right\}
$$

is closed in $m(E)$ and invariant under $(\hat{T}(t))_{t \geqslant 0}$. We call the quotient spaces

$$
E_{F}:=m(E) / c_{F}(E) \quad \text { and } \quad E_{F}^{T}:=m^{T}(E) / c_{F}(E) n m^{T}(E)
$$

the F-product of $E$ and the F-product of $E$ with respect to the semigroup $T$, respectively. Thus $E_{F}^{T}$ can be considered as a closed linear subspace of $E_{F}$. We have $E_{F}^{T}=E_{F}$ if (and only if) $T$ has a bounded generator.\\
The canonical quotient norm on $\mathrm{E}_{F}$ is given by

$$
\left\|\left(f_{n}\right)+c_{F}(E)\right\|=F-1 \text { im } \sup \left\|f_{n}\right\| .
$$

We can apply 3.3 in order to define the F-product semigroup $\left(T_{F}(t)\right)_{t \geq 0}$ on $E_{F}^{T}$ by

$$
T_{F}(t)\left(\left(f_{n}\right)+c_{F}(E)\right):=\left(T(t) f_{n}\right)+c_{F}(E) \cap m^{T}(E)
$$

Thus $T_{F}(t)$ is the restriction of $T(t){ }_{F}$ where $T(t)_{F}$ denotes the canonical extension of $T(t)$ to the F-product $\mathrm{E}_{F}$. (Note that (T( $t)_{F}{ }_{t \geqslant 0}$ is not strongly continuous unless $T$ has a bounded generator.)\\
With the canonical injection $j: f \rightarrow(f, f, f, \ldots)+c_{F}(E)$ from $E$ into $E_{F}^{T}$ the operators $T_{F}(t)$ are extensions of $T(t)$ satisfying $\left\|\mathrm{T}_{F}(t)\right\|=\|\mathrm{T}(t)\|$. The basic facts about the generator ( $A_{F}, D\left(A_{F}\right)$ ) of\\
\includegraphics[max width=\textwidth]{2024_12_23_c6487cc0859199a15bd9g-031} position.

Proposition. For the generator ( $A_{F}, D\left(A_{F}\right)$ ) of the F-product semigroup the following holds:


\begin{align*}
& D\left(A_{F}\right)=\left\{\left(f_{n}\right)+c_{F}(E): f_{n} \in D(A) ;\left(f_{n}\right),\left(A f_{n}\right) \in m^{T}(E)\right\},  \tag{i}\\
& A_{F}\left(\left(f_{n}\right)+c_{F}(E)\right)=\left(A f_{n}\right)+c_{F}(E) . \tag{ii}
\end{align*}


In case $A$ is a bounded operator then $D\left(A_{F}\right)=E_{F}^{\top}=E_{F}$ and $A_{F}$ is the canonical extension of $A$ to $E_{F}$.\\
We will show in A-III, 4.5 that the above construction preserves and even improves many spectral properties of the semigroup and its generator.

\subsection*{3.7. The Tensor Product Semigroup}
Real- or complex-valued functions of two variables $x$, $y$ are often limits of functions of the form $\sum_{i=1}^{n} f_{i}(x) g_{i}(y)$, which to some extent allows one to consider the variables $x$ and $y$ separately.

Since algebraic manipulation with these latter functions is governed by the formal rules of a tensor product, it is customary to identify (for example) the function

$$
(x, y) \rightarrow f(x) g(y)
$$

with the tensor product $f$ g and to consider limits of linear combinations of such functions as elements of a completed tensor product. To be more precise, we briefly present the most important examples for this situation.

Examples. 1. Let $(X, \Sigma, \mu)$ and $(X, \Omega, v)$ be measure spaces. Identifying for $f_{i} \in L^{P}(\mu), g_{i} \in L^{P}(v)$ the elements $\sum_{i=1}^{n} f_{i} 8 g_{i}$ of the tensor product

$$
\mathrm{L}^{\mathrm{P}}(\mu) \otimes \mathrm{L}^{\mathrm{P}}(v)
$$

with the (class of $\mu \times v$-a.e.-defined) functions

$$
(x, y) \rightarrow \sum_{i=1}^{n} f_{i}(x) g_{i}(y),
$$

$\mathrm{L}^{\mathrm{P}}(\mu) \mathrm{L}^{\mathrm{P}}(\nu)$ becomes a dense subspace of $\mathrm{L}^{\mathrm{P}}(\mathrm{X} \times \mathrm{Y}, \Sigma \times \Omega, \mu \times \nu)$ for $1 \leq \mathrm{p}<\infty$.\\
2. Similarly, let $X, Y$ be compact spaces. Then\\
$C(X) \otimes C(Y)$\\
becomes a dense subspace of $C(X \times Y)$ by identifying, for $\mathbf{f} \in \mathrm{C}(\mathrm{X})$ and $g \in C(Y), f \approx g$ with the function\\
$(x, y) \rightarrow f(x) g(y)$.

We do not intend to go into a deeper investigation of the quite sophisticated problems related to normed tensor products of general Banach spaces, but will rather confine ourselves to the discussion of certain special cases. These will always be related to one of the following standard methods to define a norm on the tensor product of two Banach spaces E, F :\\
Let $u:=\sum_{i=1}^{n} f_{i} \otimes g_{i}$ be an element of $E \otimes F$. Then\\
(i) $\quad\|u\|_{\pi}:=\inf \left\{\sum_{i=1}^{m}\left\|h_{i}\right\|\left\|k_{i}\right\|: u=\sum_{i=1}^{m} h_{i} \otimes k_{i}, h_{i} \in E, k_{i} \in F\right\}$ defines the "greatest cross norm $\pi$ " on $\mathrm{E} \otimes \mathrm{F}$.\\
(ii) $\|u\|_{\varepsilon}:=\sup \left\{\langle u, \phi \psi\rangle: \phi \in E^{\prime}, \psi \in F^{\prime},\|\phi\|,\|\psi\| \leqq 1\right\}$ defines the "least cross norm $\varepsilon$ " on $\mathrm{E} \times \mathrm{F}$. Here, <u, $\phi$. $\psi>$ denotes the canonical bilinear form on ( $E F) \times\left(E^{\prime} \otimes F^{\prime}\right)$, i.e. $\left\langle\sum_{i=1}^{n} f_{i} \otimes g_{i}, \phi \otimes \psi\right\rangle=\sum_{i=1}^{n}\left\langle f_{i}, \phi\right\rangle\left\langle g_{i}, \psi\right\rangle$;\\
(iii) if $E$ and $F$ are Hilbert spaces, $\|u\|_{h}=(u \mid u)_{h} / 2$, where the scalar product $(\cdot \mid \cdot)_{h}$ is defined as in (ii), defines the "Hilbert norm $h$ " on $E \otimes F$.

In the following we write $E{ }_{\alpha} F$ for the tensor product of $E$ and F endowed - if applicable - with one of the norms $\pi, \varepsilon, h$ just defined. In each case one has $\|f \propto g\|=\|f\|\|g\|$ for $f \in E, g \in \mathbf{F}$. By $E \tilde{\theta}_{\alpha} F$ we mean the completion of $E \theta_{\alpha} F$. Moreover we recall how examples 1 and 2 above fit into this pattern:

$$
\begin{aligned}
& L^{1}(\mu \otimes v)=L^{1}(\mu) \tilde{\theta}_{\pi} L^{1}(v), L L^{2}(\mu \otimes v)=L^{2}(\mu) \tilde{\theta}_{\mathrm{h}} L^{2}(v) \\
& C(X \otimes Y)=C(X)
\end{aligned}
$$

Finally we point out that for any $s \in L(E), T \in L(F)$, the mapping

$$
\sum_{i=1}^{n} f_{i} \otimes g_{i} \rightarrow \sum_{i=1}^{n} S f_{i} \odot T g_{i}
$$

defined on $E \otimes F$ is linear and continuous on $E{ }_{\alpha} F$, hence has a continuous extension to $\mathrm{E}=\mathrm{F}$. This operator, as well as its continuous extension, will be denoted by $\mathrm{S} \otimes \mathrm{T}$ and satisfies $\|\mathrm{S} \otimes \mathrm{T}\|=\|\mathrm{S}\|\|\mathrm{T}\|$. The notation $\mathrm{A} \otimes \mathrm{B}$ will also be used in the obvious way if $A$ and $B$ are not necessarily bounded operators on $E$ and $F$. We are now ready to consider semigroups induced on tensor product.

Proposition. Let $(S(t))_{t \geq 0}$ and $(T(t))_{t \geqq 0}$ be strongly continuous semigroups on Banach spaces $E, F$, and let $A$, B be their generators. Then the family\\
$(S(t) \otimes T(t))_{t \geq 0}$\\
is a strongly continuous semigroup on $E \tilde{\theta}_{\alpha} F$.\\
The closure of\\
A $\operatorname{Id}+\mathrm{Id}$ ) B ,\\
defined on the core $D(A)$ D(B) is its generator.

Proof. It is immediately verified that ( $\mathrm{S}(\mathrm{t}) \mathrm{T}(\mathrm{t}))_{t \geq 0}$ is in fact a semigroup of operators on $\mathrm{E} \tilde{\hat{\theta}}_{\alpha} \mathrm{F}$. The strong continuity need only be verified at $t=0$ and on elements of the form $u=f \otimes g \in E \otimes F$. This verification being straightforward, there remains to show that the generator of $(S(t) \otimes T(t))_{t \geq 0} \quad i s$ obtained as the closure of $(A \otimes I d+I d \otimes B, D(A) \otimes D(B))$. To this end, let $f \in D(A)$ and $g \in D(B)$. Then $\lim _{h \rightarrow 0} \frac{1}{h}(T(h) \& S(h)(f \otimes g)-f \otimes g)$

$$
\begin{aligned}
& =1 i m_{h \rightarrow 0} \frac{1}{h}(T(h) f(S(h) g-g)+(T(h) f-f) \otimes g) \\
& =(f \otimes B g)+(A f \otimes g) .
\end{aligned}
$$

Since the elements of the form $\pounds g, f \in D(A), g \in D(B)$, generate the linear subspace $D(A) O D(B)$ of $E \tilde{\theta}_{\alpha} F$, this subspace belongs\\
to the domain of the generator. Moreover, $D(A)$ D(B) is dense in $E \tilde{\theta}_{\alpha} F$ and invariant under $(S(t) \otimes T(t))_{t \geq 0}$, hence it is a core of $\mathrm{A} \otimes \mathrm{Id}+\mathrm{Id} \otimes \mathrm{B}$ by Prop.1.9.ii .

\subsection*{3.8. The Product of Commuting Semigroups}
Let $\left(S(t) t_{t \geqq 0}\right.$ and $(T(t))_{t \geq 0}$ be semigroups with generators $A$ and B, respectively on some Banach space E. It is not difficult to see that the following assertions are equivalent.\\
(i) $S(t) T(t)=S(t) T(t)$ for all $t \geqq 0$.\\
(ii) $R(\mu, A) R(\mu, B)=R(\mu, B) R(\mu, A)$ for some $\mu \in \rho(A) \cap \rho(B)$.\\
(iii) $\mathrm{R}(\mu, \mathrm{A}) \mathrm{R}(\mu, \mathrm{B})=\mathrm{R}(\mu, \mathrm{B}) \mathrm{R}(\mu, \mathrm{A})$ for all $\mu \in \rho(\mathrm{A}) \cap \rho(B)$.

In that case $U(t)=s(t) T(t)(t \geqq 0)$ defines a semigroup $(U(t))_{t \geqq 0}$. Using Prop.1.9(ii) one easily shows that $D_{O}:=D(A) \cap D(B)$ is a core for its generator $C$ and $C f=A f+B f$ for all $f \in D_{0}$.

\section*{NOTES.}
For a more complete information on semigroup theory we refer the reader to HillePhillips (1957), to the recent monographs by Davies (1980), Goldstein (1985a) and Pazy (1983), to the survey article by Krein-Khazan (1985) and to the bibllography by Goldstein (1985b).

\section*{CHARACTERIZATION OF SEMIGROUPS }
ON BANACH SPACES

In this chapter two different problems are treated:

\begin{enumerate}
  \item to characterize generators of strongly continuous semigroups;
  \item to characterize various properties of strongly continuous semigroups in terms of their generators.
\end{enumerate}

In Section 1 the first problem is solved by finding conditions on the cauchy problem associated with $A$ and also by finding conditions on the resolvent of A. The second problem is treated for a hierarchy of smoothness properties of the semigroup.

Contraction semigroups are considered in Section 2. Here, the first problem has a simple and extremely useful solution: A densely defined operator A is generator of a contraction semigroup if and only if A is dissipative and satisfies a range condition.\\
Our approach is quite general. We do not only consider contractions with respect to the norm but also with respect to "half-norms". This will allow us to obtain results on positive contraction semigroups simultaneously by choosing a suitable half-norm (cf. C-II, Sec.1).

The last section contains a surprising result: on certain Banach spaces (e.g., $L^{\infty}$ ) only bounded operators are generators of strongly continuous semigroups.

\section*{1. THE ABSTRACT CAUCHY PROBLEM, SPECIAL SEMIGROUPS AND PERTURBATION}
\section*{by}
Wolfgang Arendt

Linear differential equations in Banach spaces are intimately connected with the theory of one-parameter semigroups. In fact, given a closed linear operator A with dense domain D(A) the following statement is true (with some reservation regarding a technical detail): The abstract Cauchy problem

$$
\begin{aligned}
& \dot{u}(t)=A u(t) \quad(t \geqq 0) \\
& u(0)=f
\end{aligned}
$$

has a unique solution for every $f \in D(A)$ if and only if $A$ is the generator of a strongly continuous semigroup.

This is one characterization of generators which illustrates their important role for applications. The fundamental Hille-Yosida theorem gives a different characterization in terms of the resolvent and yields a powerful tool for actually proving that a given operator is the generator of a semigroup.

Another problem we will treat here is how diverse properties of a semigroup can be described in terms of its generator. This is a reasonable question from the theoretical point of view (since the generator uniquely determines the semigroup). It is of interest from the practical point of view as well: the generator is the given object, defined by the differential equation. It is useful to dispose of conditions of the generator itself giving information on the solutions (which might not be known explicitely). We discuss smoothness properties such as analyticity, differentiability, norm continuity and compactness of the semigroup.

A frequent method to obtain new generators out of a given one is by perturbation. We will have a brief look at this circle of problems at the end of this section.

The results are explained and illustrated by examples. Proofs are only given when new aspects are presented which are not yet contained in the literature, otherwise we refer to the recent monographs Davies (1980), Goldstein (1985a), Pazy (1983).

\section*{The Abstract Cauchy Problem}
Let $A$ be a closed operator on a Banach space $E$ and consider the abstract Cauchy problem\\
(ACP) $\quad\left\{\begin{array}{l}\dot{u}(t)=A u(t) \quad(t \geqq 0) \\ u(0)=f .\end{array}\right.$\\
By a solution of (ACP) for the initial value $f \in D(A)$ we understand a continuously differentiable function u : $[0, \infty) \rightarrow$ E satisfying $u(0)=f$ and $u(t) \in D(A)$ for all $t \geqq 0$ such that $\dot{u}(t) \Rightarrow A u(t)$ for $t \geqq 0$.

By A-I,Thm.1.7 there exists a unique solution of (ACP) for all initial values in the domain $D(A)$ whenever $A$ is the generator of a strongly continuous semigroup. The converse does not hold (see Example 1.4. below). However, for the operator $A_{1}$ on the Banach space $E_{1}=D(A)$ (see $\left.A-I, 3.5\right)$ with domain $D\left(A_{1}\right)=D\left(A^{2}\right)$ given by $A_{1} f=A f\left(f \in D\left(A_{1}\right)\right)$ the following holds.

Theorem 1.1. The following assertions are equivalent.\\
(i) For every $f \in D(A)$ there exists a unique solution of (ACP).\\
(ii) $A_{1}$ is the generator of a strongly continuous semigroup.

Proof. (i) implies (ii).\\
Assume that (i) holds; i.e., for every $f \in D(A)$ there exists a unique solution $u(\cdot, f) \in C^{1}([0, \infty), E)$ of (ACP). For $f \in E_{1}$ define $T_{1}(t) f:=u(t, f)(t \geq 0)$. By the uniqueness of the solutions it follows that $\mathbb{T}_{1}(t)$ is a linear operator on $E_{1}$ and $T_{1}(s+t)=$ $T_{1}(s) T_{1}(t)$. Moreover, since $u(\cdot, f) \in C^{1}$, it follows that $t \rightarrow T_{1}(t) f$ is continuous from $[0, \infty)$ into $E_{1}$. We show that $\mathrm{T}_{1}(t)$ is a continuous operator for all $t>0$.\\
Let $t>0$. Consider the mapping $n: E_{1} \rightarrow C\left([0, t], E_{1}\right)$ given by $n(f)=T_{1}(\cdot) f=u(\cdot, f)$. We show that $n$ has a closed graph. In fact, let $f_{n} \rightarrow f$ in $E_{1}$ and $n\left(f_{n}\right)=u\left(\cdot, f_{n}\right) \rightarrow v$ in $c\left([0, t], E_{1}\right)$. Then $u\left(s, f_{n}\right)=f_{n}+\int_{0}^{S} A u\left(r, f_{n}\right) d r$. Letting $n \rightarrow \infty$ we obtain $v(s)=E+\int_{0}^{s} A v(r) d x$ for $0 \leqq s \leqq t$. Let\\
$\tilde{v}(s)=T_{1}(s-t) v(t)$ for $s>t$, and $\tilde{v}(s)=v(s)$ for $0 \leq s \leq t$.

Then $\vec{v}$ is a solution of (ACP). It follows that $\tilde{v}(s)=T_{1}(s) f$ for all $s \geqq 0$. Hence $v=n$ (f). We have shown that $n$ has a closed graph and so $\eta$ is continuous. This implies the continuity of $T_{1}(t)$. It remains to show that $A_{1}$ is the generator of $\left(T_{1}(t)\right)_{t \geqslant 0^{*}}$

We first show that for $f \in D\left(A^{2}\right)$ one has


\begin{equation*}
\mathrm{AT}_{1}(t) \mathrm{f}=\mathrm{T}_{1}(t) \mathrm{Af} \tag{1.1}
\end{equation*}


In fact, let $v(t)=f+\int_{0}^{t} u(s, A f) d s$. Then\\
$\dot{v}(t)=u(t, A f)=A f+\int_{0}^{t} A u(s, A f) d s=A\left(f+\int_{0}^{t} u(s, A f) d s\right)=A v(t)$. Since $v(0)=f$, it follows that $v(t)=u(t, f)$.\\
Hence $A u(t, f)=A v(t)=\dot{v}(t)=u(t, A f)$. This is (1.1).\\
Now denote by $B$ the generator of $\left(T_{1}(t)\right)_{t \geqq 0^{*}}$\\
For $f \in D\left(A^{2}\right)$ we have

$$
\lim _{t \rightarrow 0} \frac{T_{1}(t) f-f}{t}=A f
$$

and by (1.1),\\
$\lim _{t \rightarrow 0} A \frac{T_{1}(t) f-f}{t}=\lim _{t \rightarrow 0} \frac{T_{1}(t) A f-A f}{t}=A^{2} f$ in the norm of $E$.\\
Hence $\lim _{t \rightarrow 0} \frac{T_{1}(t) f-f}{t}=$ Af in the norm of $E_{1}$.\\
This shows that $A_{1} \subset B$. In order to show the converse, let $f \in D(B)$. Then $\lim _{t \rightarrow 0} A \frac{T_{1}(t) f-f}{t}$ exists in the norm of $E$.\\
since $\lim _{t \rightarrow 0} \frac{\mathrm{~T}_{1}(t) f-f}{t}=$ Af $\quad$ in the norm of $E$, it follows that $A f \in D(A)$, since $A$ is closed. Thus $f \in D\left(A^{2}\right)=D\left(A_{1}\right)$. We have shown that $B=A_{1}$.\\
(ii) implies (i).

Assume that $\mathrm{A}_{1}$ is the generator of a strongly continuous semigroup $\left(T_{1}(t)\right)_{t \geq 0}$ on $E_{1}$. Let $f \in D(A)$ and $\operatorname{set} u(t)=T_{1}(t) f$. Then $u \in \mathrm{C}([0, \infty), \mathrm{E})$ and $\mathrm{Au}(\cdot) \in \mathrm{C}([0, \infty), E)$.\\
Moreover, $\int_{0}^{t} u(s) d s=\int_{0}^{t} T_{1}(s) f d s \in D\left(A_{1}\right)=D\left(A^{2}\right)$ and $A \int_{0}^{t} u(s) d s=$ $u(t)-u(0)=u(t)-f \quad$ (by $A-I,(1.3))$.\\
Consequently, $u(t)=f+A \int_{0}^{t} u(s) d s=f+\int_{0}^{t} A u(s) d s$.\\
Hence $u \in C^{1}([0, \infty), E)$ and $\dot{u}(t)=A u(t)$. Thus $u$ is a solution of (ACP). We have shown existence.

In order to show uniqueness, assume that $u$ is a solution of (ACP) with initial value 0 . We have to show that $u \equiv 0$. Let $v(t)=\int_{0}^{t} u(s) d s$. Then $v(t) \in D(A)$ and $A v(t)=\int_{0}^{t} A u(s) d s=$ $\int_{0}^{t} \dot{u}(s) d s=u(t) \in D(A)$. Consequently, $v(t) \in D\left(A^{2}\right)$ for all $t \geqq 0$. Moreover, $\quad \dot{v}(t)=u(t)=\operatorname{Av}(t)$ and $\frac{d}{d t} \operatorname{Av}(t)=A u(t)=A \dot{v}(t)=$ $A^{2} v(t)$. Thus $v \in C^{I}\left([0, \infty), E_{1}\right)$ and $\dot{v}(t) \stackrel{=}{=} A_{1} v(t)$. since $v(0)=0$, it follows that $\mathrm{v} \equiv 0$. Thus $\mathrm{u} \equiv \mathrm{v} \equiv 0$.

If (ACP) has a unique solution for every initial value in $D(A)$, then A is the generator of a strongly continuous semigroup only if some additional assumptions on the solutions (continuous dependence from the initial value) or on $A(\rho(A) \neq \varnothing)$ are made.

Corollary 1.2. Let $A$ be a closed operator. Consider the following existence and uniqueness condition.\\
(EU) For every $f \in D(A)$ there exists a unique solution $u(\cdot, f) \in \mathrm{C}^{1}([0, \infty), \mathrm{E})$ of the Cauchy problem associated with A having the initial value $\mathrm{u}(0, \mathrm{f})=\mathrm{f}$.

The following assertions are equivalent.\\
(i) A is the generator of a strongly continuous semigroup.\\
(ii) A satisfies (EU) and $\rho(A) \neq \varnothing$.\\
(iii) A satisfies (EU) and for every $\mu \in \mathbb{R}$ there exists $\lambda>\mu$ such that $(\lambda-A) D(A)=E$.\\
(iv) A satisfies (EU), has dense domain and for every sequence ( $f_{n}$ ) in $D(A)$ satisfying $\lim _{n \rightarrow \infty} f_{n}=0$ one has $\lim _{n \rightarrow \infty} u\left(t, f_{n}\right)=0$ uniformly in $t \in[0,1]$.

Proof. It is clear that (i) implies the remaining assertions. So assume that A satisfy (EU). Then by Theorem 1.1., $\mathrm{A}_{1}$ is a generator. If there exists $\lambda \in \rho(A)$, then $(\lambda-A)$ is an isomorphism from $E_{1}$ onto $E$ and $A$ is similar to $A_{1}$ via this isomorphism [since $D\left(A_{1}\right)=\left\{(\lambda-A)^{-1} f: f \in D(A)\right\} \quad$ and $A f=(\lambda-A) A_{1}(\lambda-A)^{-1} f$ for all $f \in D(A)$, see $A-I, 3.0]$. Thus $A$ is a generator on $E$ and we have shown that (ii) implies (i).\\
If (iii) holds, then there exists $\lambda>s\left(A_{1}\right)$ such that $(\lambda-A) D(A)=E$. We show that $(\lambda-A)$ is injective. Then $\lambda \in \rho(A)$ since $A$ is closed. Assume that $A f=\lambda f$ for some $f \in D(A)$. Then $f \in D\left(A^{2}\right)=D\left(A_{1}\right)$, and so $f=0$ since $\lambda \in \rho\left(A_{1}\right)$. This proves that (iii) implies (ii).

It remains to show that (iv) implies (i). Assertion (iv) implies that for all $t \geq 0$ there exist bounded operators $T(t) \in L(E)$ such that $u(t, f)=T(t) f$ if $\pounds \in D(A)$. Moreover, $\sup _{0 \leq t \leq 1}\|T(t)\|<\infty$. It follows that $T(\cdot) \mathrm{f}$ is strongly continuous for all $\mathrm{f} \in \mathrm{E}$ (since it is so for $f \in D(A)$ and $D(A)$ is dense). Let $t>1$. There exist unique $n \in \mathbb{N}$ and $s \in[0,1)$ such that $t=n+s$. Let $T(t):=$ $\mathrm{T}(1)^{\mathrm{n}} \mathrm{T}(\mathrm{s})$. From the uniqueness of the solutions it follows that $T(t) f=u(t, f)$ for all $t \geqq 0$ as well as $T(t+s) f=T(s) T(t) f$ for all $f \in D(A)$ and $s, t \geq 0$. Thus $(T(t))_{t \geq 0}$ is a semigroup. Denote by $B$ its generator. It follows from the definition that $A \subset B$. Moreover, $D(A)$ is invariant under the semigroup. So by A-I,Prop.1.9.(ii) D(A) is a core of B. Since A is closed this implies that $\mathrm{A}=\mathrm{B}$.

Remark 1.3. It is surprising that from condition (ii) and (iii) in the corollary it follows automatically that $D$ (A) is dense.\\
On the other hand this condition cannot be omitted in (iv). In fact, consider any generator $A$ and its restriction $A$ with domain $D(A)=\{0\}$. Then $A$ satisfies the remaining conditions in (iv) but is not a generator (if dim E > ).

Example 1.4. We give a densely defined closed operator A, such that there exists a unique solution of (ACP) for all initial values in $D(A)$, but $A$ is not a generator.\\
Let $B$ be a densely defined unbounded closed operator on a Banach space $F$. Consider $E=F \oplus F$ and $A$ on $E$ given by

$$
A=\left(\begin{array}{ll}
0 & B \\
0 & 0
\end{array}\right)
$$

with domain $\mathrm{F} \times \mathrm{D}(\mathrm{B})$.\\
Then $D\left(A^{2}\right)=\{(f, g) \in F \times D(B): B g \in F\}=D(A)$ and so $A_{1} \in L\left(E_{1}\right)$. In particular, $A_{1}$ is a generator. But $A$ is not. For instance condition (ii) in Corollary 1.2. does not hold, since for each $\lambda \in \mathbb{C}$, $(\lambda-A) D(A)=\{(\lambda f-B g, \lambda g): \pounds \in F, g \in D(B)\}$

$$
C F \times D(B) \neq F \times F=E .
$$

So $\rho(A)=\varnothing$.\\
As a further illustration, we note that the solution of the corresponding abstract Cauchy problem for the initial value (f,g) $\epsilon$ $F \times D(B)$ is given by $u(t)=(f+t B g, g)$. Since $B$ is unbounded, condition (iv) of Corollary 1.2. is clearly violated.

Remark. Frequently a generator A can be extended to a closed operator B. Then one can consider the abstract Cauchy problem ACP (B) associated with B. It also has a solution for every initial value in $D(B)$, but none of the solutions is unique unless $A=B$.

IIn fact, denote by $(T(t))_{t \geq 0}$ the semigroup generated by $A$. Let $f \in D(B)$. Let $\lambda>\omega(A)$. Then there exists $g \in D(A)$ such that $(\lambda-B) f=(\lambda-A) g$. Let $h=f-g$. Then $h \in \operatorname{ker}(\lambda-B)$. Define $u$ by $\mathrm{u}(\mathrm{t})=\mathrm{e}^{\lambda t} \mathrm{~h}+\mathrm{T}(\mathrm{t}) \mathrm{g}$. Then u is a solution ACP (B) with initial value f. It follows from Cor.1.2 that there exists a non-trivial solution for the initial value 0 .]

\section*{One-parameter groups}
Generators of one-parameter groups can be characterized similarly by existence and uniqueness of the solutions of the associated Cauchy problem. However, here the assumption of continuous dependence on the initial values can be relaxed (in fact, one has no longer to assume that the continuous dependence is uniform in t).

Theorem 1.6. Let A be a closed densely defined operator. The following assertions are equivalent.\\
(i) A is generator of a strongly continuous one-parameter group.\\
(ii) For every $f \in D(A)$ there exists a unique function\\
$u(\cdot, f) \in C^{1}(\mathbb{R})$ satisfying $u(t, f) \in D(A)$ for all $t \in \mathbb{R}$ and $u(0, f)=f$ such that $\frac{d}{d t} u(t, f)=A u(t, f)$; and if $f_{n} \in D(A)$ such that $\lim _{n \rightarrow \infty} f_{n}=0$, then $\lim _{n \rightarrow \infty} u\left(t, f_{n}\right)=0$ for all $t \in \mathbb{R}$.

Proof. It is clear that (i) implies (ii). If (ii) holds then there exist operators $T(t) \in L(E)$ such that $u(t, f)=T(t) f(t \in \mathbb{R}$ , $f \in D(A))$. It follows from the uniqueness of the solutions that $T(t+s)=T(t) T(s) \quad(s, t \in \mathbb{R})$. Let $f \in E$. There exist $f_{n} \in D(A)$ such that $\lim _{n \rightarrow \infty} f_{n}=f$. Then $\lim _{n \rightarrow \infty} T(t) f_{n}=T(t) f$ for all $t \in \mathbb{R}$. since $T(\cdot) f$ is continuous, it follows that $T(\cdot) \pounds$ is measureable. Hence by [Hille-Phillips (1975),10.2.1] $\sup _{t \in J}\|\mathrm{~T}(t)\|<\infty$ for every compact interval $J \subset(0, \infty)$. Because of the group property this implies that T(-) is norm bounded on bounded subsets of $\mathbb{R}$. T(-)f is continuous if $\mathrm{E} \in \mathrm{D}(\mathrm{A})$. Since $D(A)$ is dense this implies the strong continuity of $(T(t))_{t \in \mathbb{R}^{\prime}}$

\section*{The Hille-Yosida theorem}
Given an operator A frequently it is easier to obtain information about its resolvent than to solve the cauchy problem. Therefore the following theorem is central in the theory of one-parameter semigroups.

Theorem 1.7 (Hille-Yosida). Let $A$ be an operator on a Banach space E. The following conditions are equivalent.\\
(i) A is the generator of a strongly continuous semigroup.\\
(ii) There exist $w, M \in \mathbb{R}$ such that $(w, \infty) \subset \rho(A)$ and $\left\|(\lambda-w)^{n} R(\lambda, A){ }^{n}\right\| \leqq M$ for all $\lambda>w$ and $n \in \mathbb{N}$.

In general it is not easy to give an estimate for the powers of the resolvent which enables one to apply Theorem 1.7. However, there is an important case when it suffices to consider merely the resolvent.

Corollary 1.8. For an operator $A$ on a Banach space $E$ the following assertions are equivalent.\\
(i) A is the generator of a strongly continuous contraction semigroup.\\
(ii) $(0, \infty) \subset \rho(A)$ and $\|\lambda R(\lambda, A)\| \leqq 1$ for all $\lambda>0$.

The difficult part in the proof of Theorem 1.7. is to show that (ii) implies (i). One has to construct the semigroup out of the resolvent. We mention two formulas which can be used for the proof.

Proposition 1.9. Let $A$ be the generator of a strongly continuous semigroup $(T(t))_{t \geqq 0}$. For $\lambda>0 \operatorname{let} A(\lambda)=\lambda^{2} R(\lambda, A)-\lambda \operatorname{Id}(=\lambda \operatorname{AR}(\lambda, A))$. Then\\
(1.2) $T(t) f=\lim _{\lambda \rightarrow \infty} e^{t A(\lambda)} f$ for all $f \in E$ and $t \geq 0$.

Yosida's proof consists in showing that the limit in (1.2) exists under the hypothesis (ii) of Theorem 1.2. (see [Davies (1980)], [Goldstein (1985b)] or [Pazy (1982)]).\\[0pt]
The proof of Hille (see [Kato (1966)]) is inspired by the following formula.

Proposition 1.10. Let $A$ be the generator of a strongly continuous semigroup $(T(t))_{t \geqq 0}$. Then\\
(1.3) $T(t) f=\lim _{n \rightarrow \infty}(I d-t / n A)^{-n} f=\lim _{n \rightarrow \infty}(n / t \cdot R(n / t, A))^{n} f$ for all $\pounds \in E$ and $t \geqq 0$.

\section*{Holomorphic semigroups}
We now describe a hierarchy of smoothness conditions on the semigroup, starting with the most restrictive class; namely, holomorphic semigroups. The generators of these semigroups can be characterized by a particularly simple condition.\\
For $\alpha \in(0, \pi]$ we define the sector $S(\alpha)$ in the complex plane by $s(\alpha)=\left\{r e^{i \theta}: r \geqq 0, \theta \in(-\alpha, \alpha)\right\}$.

Definition 1.11. Let $\alpha \in(0, \pi / 2]$. A strongly continuous semigroup $(T(t))_{t \geqq 0}$ is called a bounded holomorphic semigroup of angle $\alpha$ if T(*) is the restriction of a holomorphic function

$$
T: S(\alpha) \rightarrow L(E)
$$

satisfying\\
(1.4) $T(z) T\left(z^{\prime}\right)=T\left(z+z^{\prime}\right) \quad\left(z, z^{\prime} \in S(\alpha)\right)$\\
(1.5) For each $\alpha_{1} \in(0, \alpha)$ the set $\left\{T(z): z \in S\left(\alpha_{1}\right)\right\}$ is uniformly bounded and $\lim _{n \rightarrow \infty} T\left(z_{n}\right) f=f$ for every null-sequence ( $z_{n}$ ) in $S\left(\alpha_{1}\right)$ and every $f \in E$.

Remark. A function $T: S(\alpha) \rightarrow L(E)$ is holomorphic with respect to the operator norm if and only if it is strongly holomorphic if and only if it is weakly holomorphic [Yosida (1965); v. 3].

Theorem 1.12. Let $A$ be a densely defined operator on a Banach space $E$ and $a \in(0, \pi / 2]$. Then $A$ is the generator of a bounded holomorphic semigroup of angle $\alpha$ if and only if

$$
S(\alpha+\pi / 2) \subset \rho(A)
$$

and for every $\alpha_{1} \in(0, \alpha)$ there exists a constant $M$ such that (1.6) $\|R(\lambda, A)\| \leqq M /|\lambda| \quad\left(\lambda \in S\left(\alpha_{1}+\pi / 2\right)\right)$.

For the proof we refer to [Davies (1980)], for example.

Remark. Let $A$ be the generator of a bounded holomorphic semigroup $(T(t))_{t \geq 0}$ of angle $\alpha$, and let $z_{0} \in S(\alpha)$. Then $z_{0}{ }^{A}$ generates a\\
bounded semigroup $(S(t))_{t \geq 0}$ given by $S(t)=T\left(z_{0} t\right) \quad(t \geq 0)$ (where again we denote by $T$ the holomorphic extension of $(T(t))_{t \geq 0}$ on $s(\alpha)$ ).

As an application of Theorem 1.12. we prove the following.

Corollary 1.13. Let $A$ be the generator of a bounded group. Then $\mathrm{A}^{2}$ generates a bounded holomorphic semigroup of angle $\pi / 2$.

Proof. Let $0<\alpha_{1}<\pi / 2 ; \lambda \in S\left(\alpha_{1}+\pi / 2\right)$. There exist $I_{\geqq} \geqq 0$ and $\beta \in\left(-\beta_{1}, \beta_{1}\right)$, where $\beta_{1}:=\left(\alpha_{1}+\pi / 2\right) / 2$, such that $\lambda=r^{2} e^{i 2 \beta}$. Then $\left(\lambda-A^{2}\right)=\left(r e^{i \beta}-A\right)\left(r e^{i \beta}+A\right)$; so it follows that $\lambda \in \rho\left(A^{2}\right)$ and (1.7) $R\left(\lambda, A^{2}\right)=R\left(r e^{i \beta}, A\right) R\left(r e^{i B},-A\right)$.

Since A generates a bounded group, there exists a constant $\mathrm{N} \geq 0$ such that $\|R(\mu, A)\| \leq N / \operatorname{Re} \mu,\|R(\mu,-A)\| \leq N / R e \mu$ for all $\mu \in S(\pi / 2)$. Consequently, $\left\|R\left(\lambda, A^{2}\right)\right\|_{i} \leqq N^{2} / r^{2}(\cos B)^{2} \leqq 1 / r^{2} \cdot\left[N / \cos \beta_{1}\right]^{2}=M /|\lambda|$.

The corollary will be extended below. We first consider an example.

Example (The Laplacian on $\mathrm{E}=\mathrm{C}_{0}\left(\mathbb{R}^{\mathrm{n}}\right)$ or $\mathrm{L}^{\mathrm{P}}\left(\mathbb{R}^{\mathrm{n}}\right)(1 \leq \mathrm{p}<\infty)$ ). a) Let $n=1$. Then $(U(t) f)(x)=f(x+t) \quad(t \in \mathbb{R}, x \in \mathbb{R})$ defines an isometric group on E. Its generator $A$ is given by $A f=f^{\prime}$ with $D(A)=\left\{f \in C^{1}(\mathbb{R}) \cap C_{0}(\mathbb{R}): \mathbb{E}^{\prime} \in C_{O}(\mathbb{R})\right\}$ in the case $E=C_{0}(\mathbb{R})$ and $D(A)=\left\{f \in E \cap A C(\mathbb{R}): f^{\prime} \in E\right\}$ in the case $E=L^{P}$ (see $A-I, 2,4$ ). Thus $A^{2}$ generates a bounded holomorphic semigroup by Cor.1.13.\\
b) Let $E=C_{0}\left(\mathbb{R}^{n}\right)$ or $L^{p}\left(\mathbb{R}^{n}\right) \quad(1 \leqq p<\infty)$. For $i \in\{1, \ldots, n\}$ denote by $\left(U_{i}(t)\right)_{t \in \mathbb{R}}$ the group given by $\left(U_{i}(t) f\right)(x)=f\left(x_{1}, \ldots, x_{i-1}, x_{i}+t, \ldots, x_{n}\right) \quad\left(x \in \mathbb{R}^{n}, t \in \mathbb{R}\right) \quad$ and by $A_{i}$ its generator. since $U_{i}(t) U_{j}(s)=U_{j}(s) U_{i}(t) \quad(s, t \in \mathbb{R}, i, j \in\{1, \ldots, n\})$ it follows that the resolvents of $A_{i}$ commute. So the same is true for the resolvents of $\mathrm{A}_{i}{ }^{2}$ (cf.(1.7) and $\mathrm{A}-\mathrm{I}, 3.8$ ).\\
Denote by $\left(T_{i}(t)\right)_{t \geq 0}$ the semigroup generated by $A_{i}{ }^{2}(i=1, \ldots, n)$. Then for $z_{i} z^{\prime} \in S(\pi / 2)$ one has $T_{i}(z) T_{j}\left(z^{\prime}\right)=T_{j}\left(z^{\prime}\right) T_{i}(z)$ $(i, j=1, \ldots, n)$. Consequently, $T(t):=T_{1}(t) \circ \ldots \circ T_{n}(t) \quad(t \geq 0)$ defines a holomorphic semigroup of angle $\pi / 2$. According to $A-I, 3.8$ the domain of its generator $A$ contains $D\left(A_{1}{ }^{2}\right)$ n... $\mathrm{n}\left(\mathrm{A}_{n}{ }^{2}\right)$; in particular $D_{0}=\left\{f \in E \cap C^{2}\left(\mathbb{R}^{n}\right): D^{\alpha} f \in E\right.$ for every multiindex $\alpha$ with $|\alpha| \leqq 2\} \subset D(A)$ and the generator is given by\\
$A f=\left(A_{1}^{2}+\ldots+A_{n}^{2}\right) f=\sum_{i=1}^{n} \frac{\partial^{2}}{\left(\partial x_{i}\right)^{2}} f=\Delta f \quad$ for all $f \in D_{0}$.

Let $a \in(0, \pi / 2]$. A semigroup $(T(t))_{t \geqq 0}$ is called holomorphic of angle $\alpha$ if it possesses an extension $T: S(\alpha)+L(E)$ for some $\alpha \in(0, \pi / 2]$ which satisfies all the requirements of Definition 1.11 except that it is not required to be bounded on any sector $\mathrm{s}\left(\alpha_{1}\right)$.

Theorem 1.14. A densely defined operator A is the generator of a holomorphic semigroup if and only if there exist $M>0$ and $r \geqq 0$ such that $\lambda \in \rho(A)$ and $\|R(\lambda, A)\| \leqq M /|\lambda|$ whenever $\operatorname{Re} \lambda>0$, $|\lambda| \geqq r$.

Proof. It is not difficult to show that $A$ generates a holomorphic semigroup of angle $\alpha$ if and only if for every $\alpha_{1} \epsilon(0, \alpha)$ there exists $w \in \mathbb{R}$ such that $A-w$ generates a bounded holomorphic semigroup of angle $\alpha_{1}$ (cf.[Reed-Simon (1978b), p.252]). As a consequence one obtains the following. A densely defined operator A generates a holomorphic semigroup of angle $a \in(0, \pi / 2]$ if and only if for every $\alpha_{1} \in[0, \alpha)$ there exist a constant $M \geqq 0$ and $r \geqq 0$ such that

$$
S\left(a_{1}+\pi / 2\right) \backslash B(r) \subset \rho(A) \quad(\text { where } B(r)=\{z \in \mathbb{C}:|z| \leqq r\})
$$

and

$$
\|R(\lambda, A)\| \leqq M /|\lambda| \quad \text { for all } \quad \lambda \in S\left(\alpha_{1}\right) \backslash B(r)
$$

This shows that the condition of the theorem is necessary. Conversely, assume that the condition holds. Since $\|R(\lambda, A)\| \rightarrow \infty$ when $\lambda$ approaches $\sigma(A)$ (cf. Lemma 1.21 below), it follows that $\lambda \in \rho(A)$ and $\|R(\lambda, A)\| \leqq M /|\lambda|$ if $\operatorname{Re} \lambda=0$ and $|\lambda|>r$ as well.\\
Let $c=1 / 2 \mathrm{~m}$. If $\xi, n \in \mathbb{R}$ satisfy $|\xi| \leqq c_{n}|,|\eta| \geqq r$, then $\left\|\xi R\left(i_{n}, A\right)\right\| \leqq \xi \cdot M /|n| \leqq c \cdot M=1 / 2$.\\
Hence $R\left(\xi+i_{\eta}, A\right)=\sum_{n=0}^{\infty}(-\xi)^{n_{R}\left(i_{n}, A\right)^{n+1}}$ exists and

$$
\begin{aligned}
\left\|R\left(\xi+i_{n}, A\right)\right\| & \leqq\left(\left|\xi+i_{n}\right|\right)^{-1} \cdot\left|\xi+i_{n}\right| \cdot \sum_{n=0}^{\infty}|\xi|^{n_{M}}{ }^{n+1} /|n|^{n+1} \\
& \left.\leqq\left.\left(\left|\xi+i_{n}\right|\right)^{-1} \cdot M \cdot| | \xi\right|^{2}+|n|^{2}\right)^{-1 / 2} /|n| \cdot \sum_{n=0}^{\infty} M_{c}^{n} n^{n} \\
& \leqq\left(2 M \cdot\left(c^{2}+1\right)^{-1 / 2}\right) /\left|\xi+i_{n}\right| \\
& =N /\left|\xi+i_{n}\right| .
\end{aligned}
$$

This together with the assumption implies that there exist $\mathrm{N}^{\prime} \geqq 0$ and $\alpha \in(0, \pi / 2]$ such that $\lambda \in \rho(A)$ and $\|R(\lambda, A)\| \leq N^{\prime} /|\lambda|$ for all $\lambda \in S(\alpha+\pi / 2)$.

Compared with the Hille-Yosida theorem, Theorem 1.14 gives a very simple criterion for an operator to be the generator of a (holomorphic) semigroup. Merely the resolvent and not its powers have to be\\
estimated. However, the resolvent has to be known in the right halfplane instead of a right half-line.

On the other hand, given a strongly continuous semigroup, merely an estimate on a vertical line implies that the semigroup is holomorphic. More precisely, the following holds.

Corollary. A strongly continuous semigroup with generator A is holomorphic if and only if there exist $w>\omega(A)$ and $M \geqq 0$ such that one has $\|R(w+i n, A)\| \leqq M /|n|$ for all $n \in \mathbb{R}$.

Proof. In fact, assume that the condition holds. Since A-w is the generator of a bounded semigroup one has $\|R(\lambda, A-w)\|_{i} \leqq N / \operatorname{Re} \lambda$ for some $N>0$ and all $\lambda \in \mathbb{C}$ satisfying $\mathrm{Re}^{\lambda}>0$. Consequently, for every $\quad \alpha \in(0, \pi / 2), \quad\|R(\lambda, A-w)\|_{i} \leqq(|\lambda| / \operatorname{Re} \lambda) \cdot N /|\lambda| \leqq \mathrm{N}(\cos \alpha)^{-1} /|\lambda|$ for all $\lambda \in S(\alpha)$. The remaining estimate for a sector around the imaginary axis is given by the proof of Thm.1.14 and shows that A-w generates a holomorphic semigroup. The reverse implication is clear.

We now prove the following extension of cor.1.13.

Theorem 1.15. Let $A$ be the generator of a strongly continuous group. Then $A^{2}$ generates a holomorphic semigroup of angle $\pi / 2$.

Proof. There exists $w \geqq 0$ such that ( $\pm \mathrm{A}-\mathrm{w}$ ) generates a bounded semigroup. Consequently, there exists $M \geqq 0$ such that $\|R(\mu, \pm A-w)\| \leqq M / \operatorname{Re} \mu$ whenever $\operatorname{Re} \mu>0$.\\
Let $\alpha \in(0, \pi / 2)$. There exist $r_{0} \geqq 0$ and $\beta \in(0, \pi / 2)$ such that $S(\alpha+\pi / 2) \backslash B\left(r_{0}\right) \subset\left\{z^{2}: z \in S(B)+w\right\}$.\\[0pt]
[In fact, the line $\{w+r(\cos \beta+i \sin \beta): r \geqq 0\}$ can be parameterized by $z(t)=w+t+i \cdot t / \varepsilon \quad(t \geq 0)$ (where $\varepsilon>0$ depends on $\beta$ ). Then $z(t)^{2}=(w+t)^{2}-t^{2} / \varepsilon^{2}+i 2 t(w+t) / \varepsilon$.\\
Thus $\lim _{t \rightarrow \infty} \operatorname{Imz}(t)^{2} / \operatorname{Rez}(t)^{2}=2 \varepsilon /\left(\varepsilon^{2}-1\right)$. Choose $\beta \in(\pi / 4, \pi / 2)$ such that $\left.\tan (\alpha+\pi / 2)>2 \varepsilon /\left(\varepsilon^{2}-1\right).\right]$\\
Now let $\lambda \in \mathrm{S}(\alpha+\pi / 2) \backslash B\left(r_{0}\right)$. Then there exist $\theta \in(-\beta, \beta)$ and $r \geqq 0$ such that $\lambda=\left(r e^{i \theta}+w\right)^{2}$. Thus $\left(\lambda-A^{2}\right)=\left(r e^{i \theta}+w-A\right)\left(r e^{i \theta}+w-A\right)$. Hence $\lambda \in \rho\left(A^{2}\right)$ and $R\left(\lambda, A^{2}\right)=R\left(r e^{i \theta}, A-w\right) R\left(r e^{i \theta},-A-w\right)$. We conclude that $|\lambda| \cdot\left\|R\left(\lambda, A^{2}\right)\right\| \leqq|\lambda| \cdot M^{2} /(\cos \theta)^{2} r^{2} \leqq\left(|\lambda| / r^{2}\right) \cdot M^{2} /(\cos \beta)^{2}$. Thus $|\lambda| \cdot R\left(\lambda, A^{2}\right)$ is uniformly bounded for $\lambda \in S(\alpha+\pi / 2) \backslash B\left(r_{0}\right)$.

Remark. In Theorem 1.15 the assumption that $\pm \mathrm{A}$ are generators can be relaxed. In fact, the proof shows the following. If A is a densely defined operator such that $\{\lambda \in \mathbb{C} \operatorname{Re} \lambda>0\} \subset \rho( \pm A-w)$ and $\|R(\lambda, \pm A-w)\| M / \operatorname{Re} \lambda$ for some $M \geqq 0, w \geqq 0$, then $A^{2}$ generates a holomorphic semigroup of angle $\pi / 2$.

Next we consider semigroups satisfying a less restrictive smoothness condition.

\section*{Differentiable semigroups}
Let $(T(t))_{t \geqq 0}$ be a strongly contimuous semigroup with generator A. Let $t_{0} \geqq 0$ and $f \in E$. Then the function $t \rightarrow T(t) f$ is right sided differentiable in $t_{0}$ if and only if $T\left(t_{0}\right) f \in D(A)$; and in that case it is differentiable in every $s>t_{0}$ and the derivative is AT(s)f (this follows from A-I, Prop.1.6(ii)).

Definition 1.16. A strongly continuous semigroup $(T(t))_{t \geqslant 0}$ on a Banach space $E$ is called eventually differentiable if there exists $t_{0} \geq 0$ such that the function $t \rightarrow T(t) \mathrm{f}$ from $\left(t_{0}, \infty\right)$ into $E$ is differentiable for every $f \in$ E. The semigroup is called differentiable if $t_{0}$ can be chosen 0 .

It is not difficult to see that if $(T(t))_{t \geqq 0}$ is differentiable for $t>t_{0}$ then it is n-times differentiable in all $s>n t_{0}$ and $T(s) E \subset D\left(A^{n}\right)$ (n $\left.\in \mathbb{N}\right)$. If $(T(t))_{t \geq 0}$ is differentiable, then the function $t \rightarrow T(t) f$ from $(0, \infty)$ into $E$ is infinitely often differentiable for every $f \in$ E.

Generators of (eventually) differentiable semigroups can be characterized similarly as those of holomorphic semigroups by the spectral behavior of the resolvent. Whereas the spectrum of the generator of a holomorphic semigroup is included in a sector, the spectrum of the generator of an eventually differentiable semigroup is limited by a function of exponential growth (instead of a line).

Theorem 1.17. A strongly continuous semigroup $(T(t))_{t \geqq 0}$ is eventually differentiable if and only if its generator A satisfies the following: there exist constants $c>0, b>0, M>0$ such that

$$
\Sigma:=\left\{\lambda \in \mathbb{C}: \mathrm{ce}^{-\mathrm{b} \cdot \operatorname{Re} \lambda} \leqq|\operatorname{Im} \lambda|\right\} \subset \rho(\mathrm{A})
$$

and $\|R(\lambda, A)\| \leqq M .|\operatorname{Im}|$ for all $\lambda \in \Sigma \quad$ satisfying $\operatorname{Re} \lambda \leqq \omega(A)$.


Theorem 1.18. A strongly continuous semigroup (T) $(t){ }_{t \geq 0}$ is differentiable if and only if its generator A satisfies the following: for $a l l \quad b>0$ there exist $c>0, M>0$ such that

$$
\Sigma:=\left\{\lambda \in \mathbb{C}: \mathrm{ce}^{-\mathrm{b} \cdot \operatorname{Re} \lambda} \leqq|\operatorname{Im} \lambda|\right\} \subset \rho(\mathrm{A})
$$

and $\|R(\lambda, A)\| \leqq M \cdot|\operatorname{Im}|$ for all $\lambda \in \Sigma \quad$ satisfying $\operatorname{Re} \lambda \leqq \omega(A)$.

For the proofs of these two theorems we refer to [Pazy (1983), Chap.3, Theorem 4.7 and 4.87.

\section*{Norm continuous semigroups}
Let $(T(t))_{t \geq 0}$ be a strongly continuous semigroup and $t^{*}>0$.\\
If $\lim _{t+t^{\prime}}\left\|\mathrm{T}(t)-\mathrm{T}\left(\mathrm{t}^{\prime}\right)\right\|=0$, then it follows from the semigroup property, that the function $t \rightarrow T(t)$ is norm continuous on the whole half line ( $\left.t^{\prime}, \infty\right)$.

Definition 1.19. A semigroup $(T(t))_{t \geqq 0}$ is called eventually norm continuous if there exists $t^{\prime} \geqq 0$ such that the function $t \rightarrow T(t)$ from (t', $\infty$ ) into $L(E)$ is norm continuous. The semigroup is called norm continuous if $t^{\prime}$ can be chosen equal to 0 .

The spectrum of generators of eventually norm continuous semigroups still is compact in every right half-plane.

Theorem 1.20. Let A be the generator of an eventually norm continuous semigroup. Then for every $b \in \mathbb{R}$ the set\\
$\{\lambda \in \sigma(A): \operatorname{Re} \lambda \geqq b\}$\\
is bounded.

For the proof of Theorem 1.20 we use the following lemma.\\
Lemma 1.21. Let $A$ be an operator and $\lambda \in \rho(A)$. Then

$$
\operatorname{dist}(\lambda, \sigma(A))=r(R(\lambda, A))^{-1} .
$$

Proof. One has $\{0\} U \sigma(R(\lambda, A))=\{0\} \cup\left\{(\lambda-\mu)^{-1}: \mu \in \sigma(A)\right\}$ [Davies (1980), Lemma 2.11]. Hence $r(R(\lambda, A))=\sup \left\{|\lambda-\mu|^{-1}: \mu \in \sigma(A)\right\}=$ $[\inf \{|\lambda-\mu|: \mu \in \sigma(A)\}]^{-1}=\operatorname{dist}(\lambda, \sigma(A))^{-1}$.

Proof of Thm.1.20. It is enough to show the following. Let a $>$. A . . Then for every $\varepsilon>0$ there exist $n \in \mathbb{N}$ and $r_{0} \geqq 0$ such that $\left\|R(a+i r, A)^{n}\right\|^{1 / n}<\varepsilon$ for all $r \in \mathbb{R}$ satisfying $|r| \geqq r_{0}$.\\[0pt]
[In fact, then we have by the lemma,\\
$\operatorname{dist}(a+i r, \sigma(A))=r(R(a+i r, A))^{-1} \geqq\left\|R(a+i r, A)^{n}\right\|^{-1 / n}>1 / \varepsilon \quad$ whenever $\left.|r| \geqq r_{0}\right]$.\\
So let $\varepsilon>0$. If Re $\lambda>\omega(A)$, then by $A-I$, Prop.1.11,\\
$R(\lambda, A)^{n+1}=1 / n!\int_{0}^{\infty} e^{-\lambda t} t^{n_{r}} \mathrm{~T}(t) d t \quad(n \in \mathbb{N})$. Let $t^{\prime}>0$ such that $t \rightarrow T(t)$ is norm continuous on $\left[t^{\prime}, \infty\right)$. Let $w \in(\omega(A), a)$. There exists $M \geqq 1$ such that $\|\mathrm{T}(\mathrm{t})\| \leqq \mathrm{Me}^{\mathrm{wt}}$ for all $t \geqq 0$. Let $N:=M \cdot \int_{0}^{t '} e^{-a t} e^{w t} d t$. Since $\lim _{n \rightarrow \infty} c^{n} / n!=0$ for all $c>0$, there exists $n \in \mathbb{N}$ such that $N \cdot\left(t^{\prime}\right)^{n} / n!<\varepsilon^{n+1} / 3$. Choose $T \geqq t^{\prime}$ such that $1 / n!\cdot \int_{T}^{\infty} t^{n} e^{-a t}\|T(t)\| d t<\varepsilon^{n+1} / 3$.\\
Since $(T(t))_{t \geqq 0}$ is norm continuous for $t \geqq t^{\prime}$, it follows from the Riemann-Lebesgue lemma that there exists $r_{0} \geqq 0$ such that $\left\|1 / n!\cdot \int_{t}^{T}, t^{n} e^{-i r t} e^{-a t_{T}}(t) d t\right\|<\varepsilon^{n+1} / 3$ whenever $|r| \geqq r_{0}$.\\
All together we obtain for $|r| \geq r_{0}$, $\left\|R(a+i r, A)^{n+1}\right\|=1 / n!\cdot \| \int_{0}^{\infty} e^{-(a+i r) t_{t}{ }^{n} T(t) d t \|}$

$$
\begin{aligned}
& \leqq 1 / n!\cdot \int_{0}^{t^{\prime}} e^{-a t^{n}} t^{n}\|(t)\| d t \\
& +1 / n!\cdot\left\|\int_{t}^{T} \cdot t^{n} e^{-i r t} e^{-a t_{T}}(t) d t\right\| \\
& +1 / n!\cdot \int_{T}^{\infty} e^{-a t^{n}}\|T(t)\| d t \\
& \leqq 1 / n!\cdot\left(t^{\prime}\right)^{n} \int_{0}^{t^{\prime}} e^{-a t} M e^{W t} d t+2 / 3 \cdot \varepsilon^{n+1} \\
& \leqq N \cdot\left(t^{\prime}\right)^{n} / n!+2 / 3 \cdot \varepsilon^{n+1} \leqq \varepsilon^{n+1} .
\end{aligned}
$$

A complete characterization of eventually norm continuous semigroups in terms of their generator seems not to be known.\\
Eventually norm continuous semigroups are of particular interest in spectral theory (cf. A-III,Thm.6.6). Moreover their asymptotic behavior is easy to describe (see A-IV,(1.8)).

Next we describe special norm continuous semigroups.

\section*{Compact semigroups}
Let $(T(t))_{t \geq 0}$ be a strongly continuous semigroup and $t_{0}>0$. If $\mathrm{T}\left(\mathrm{t}_{\mathrm{o}}\right)$ is compact, then it follows from the semigroup property that $T(t)$ is compact for all $t \geqq t_{0}$ Moreover, $t \rightarrow T(t)$ is norm continuous in every $t>t_{0}$\\[0pt]
[In fact, since $T(h) \rightarrow$ Id strongly with h $\psi 0$, it follows that $\lim _{h+0} T(h) \pounds=f$ uniformly on every compact subset $k$ of E. Now let $t \geqq t_{0}$. Then $K=\bar{T} \bar{T} \bar{U}$ is compact (where $U$ denotes the unit ball of E). Hence $\lim _{h+0} T(h+t) f=1 i m_{h+0} T(h) T(t) f$ uniformly for $f \in U$. So the semigroup is right-sided norm continuous on $\left[t_{0}, \infty\right)$ and so norm continuous on $\left(t_{0}, \infty\right) .1$

Definition 1.22. A strongly continuous semigroup ( $\mathbb{T}^{( }(t){ }_{t \geq 0}$ is called compact if $T(t)$ is compact for all $t>0$; the semigroup is called eventually compact if there exists $t_{0}>0$ such that $T\left(t_{0}\right)$ is compact (and hence $T(t)$ is compact for all $t \geqq t_{0}$ ).

We want to find a relation between the compactness of the semigroup and the compactness of the resolvent of its generator.

Definition 1.23. Let $A$ be an operator and $\rho(A) \neq \emptyset$. We say, $A$ has a compact resolvent if $R(\lambda, A)$ is compact for one (and hence all) $\lambda \in \rho(A)$.

Proposition 1.24. Let (T(t)) $t \geq 0$ be a strongly continuous semigroup and assume that its generator has a compact resolvent. If $t \rightarrow T(t)$ is norm continuous in $t_{0}$ then $T(t)$ is compact for all $t \geqq t_{0}$.

Proof. Considering $\left(e^{-w t} T(t)\right) t \geqq 0$ for some $w>0$ if necessary, we can assume that $s(A)<0$. Let $S(t) \in L(E)$ be given by $S(t) f=\int_{0}^{t} T(s) f d s(t \geq 0)$. Then $A S(t) f=T(t) f-f$ for all $f \in E$, and so $S(t)=R(0, A)(I d-T(t))$ is compact for all $t \geqq 0$. Since $t \rightarrow T(t)$ is norm continuous for $t \geqq t_{0}$, one has $\lim _{h \downarrow 0} \frac{1}{h}\left(S\left(t_{0}+h\right)-S\left(t_{0}\right)\right)=T\left(t_{0}\right)$ in the operator norm. Thus $T\left(t_{0}\right)$ is compact as limit of compact operators.

Theorem 1.25. A strongly continuous semigroup ( $T(t){ }_{t \geq 0}$ is compact if and only if it is norm continuous and its generator A has compact resolvent.

Proof. Assume that $(T(t)){ }_{t \geq 0}$ is compact. Then $T(\cdot)$ is norm continuous on $(0, \infty)$, and so $\int_{0}^{t} e^{-W s} \mathrm{~T}(s) d s$ is compact as the norm limit of linear combinations of compact operators, where $w>\omega(A)$. since $R(w, A)=\lim _{t \rightarrow \infty} \int_{0}^{t} e^{-w s} T(s) d s$ in the operator norm, it follows that $R(w, A)$ is compact. This proves one implication. The other follows from Proposition 1.24.

Remark 1.26.a) Generators of eventually compact semigroups do not necessarily have compact resolvent. Consider the nilpotent translation semigroup $(T(t))_{t \geq 0}$ on $F:=L^{1}[0,1]$ (see $\left.A-I, E x, 2.6\right)$. Let $E=F \otimes_{\pi} F=I^{1}([0, I] \times[0,1])$ and $S(t)=T(t)$ Id ( $\left.t \geq 0\right)$. Then $(S(t))_{t \geq 0}$ is a strongly continuous semigroup (see $A-I, 3.7$ ). Denote by $B$ its generator. $(S(t))_{t \geqslant 0}$ is a nilpotent semigroup (so it is eventually compact), but $R(\lambda, B)=R(\lambda, A)$ Id is not compact. b) It is obvious that a group $(T(t))_{t \in \mathbb{R}}$ is eventually norm continuous if and only if it is norm continuous in 0 ; i.e., its generator is bounded.\\
On the other hand, the generator of the rotation group (A-I,Ex, 2.5) has a compact resolvent. Hence this condition does not imply any smoothness property of the semigroup.

Positive eventually compact semigroups have remarkable properties in the setting of the Perron-Frobenius theory (see e.g., B-III, Cor.2.12).

The following scheme indicates the relation between the different classes of semigroups defined so fax.

\begin{verbatim}
holomorphic -> differentiable }->\mathrm{ eventually differentiable
    t \
    norm continuous \ eventually norm continuous
    +
    compact \ eventually compact
\end{verbatim}

All these classes are different. This is shown by the following examples.

Example 1.27. The nilpotent shift semigroup ( $A-I, 2.6$ ) is obviously eventually differentiable, eventually compact and eventually norm continuous. But it is not norm continuous and consequently not differentiable or compact.

Example 1.28. We consider multiplication semigroups (see A-I,2.3).\\
Let $E=C_{0}(X)$, where $X$ is a locally compact space, or\\
$\mathrm{E}=\mathrm{L}^{\mathrm{P}}(\mathrm{x}, \Sigma, \mu)$, where $1 \leqq \mathrm{p}<\infty$ and $(\mathrm{X}, \Sigma, \mu)$ is a $\sigma$-finite measure space. Let $m: X \rightarrow \mathbb{R}$ be continuous [resp.,measurable] such that $[\operatorname{ess}]-\sup _{x \in X} \operatorname{Re}(\mathrm{~m}(\mathrm{x}))<\infty$.\\
Then $A f=m \cdot f$ with domain $D(A)=\{f \in E: m \cdot f \in E\}$ is the generator of the semigroup $(T(t)) t \geq 0$ given by $(T(t) f)(x)=e^{t m(x)} f(x)$ $(t \geqq 0, x \in X, f \in E)$. Observe that $\sigma(A)=\overline{m(X)}$ in case $E=C_{0}(X)$ and $\sigma(A)=[$ ess $]$-image $(m):=\{\lambda \in \mathbb{C}: \mu((\{x \in X:|m(x)-\lambda|<\varepsilon\}) \neq 0$ for all $\varepsilon>0\}$ if $E=L^{p}$ (see A-II, 2.3). Consequently, $s(A)=\omega(A)=[\operatorname{ess}]-\sup _{x \in x} \operatorname{Re}(m(x))$.\\
a) The semigroup is norm continuous for $t>0$ if and only if it is eventually norm continuous if and only if $\{\lambda \in \sigma(A): \operatorname{Re} \lambda \geqq b\}$ is bounded for every $b \in \mathbb{R}$. Thus the property proved in Theorem 1.20 characterizes generators of eventually norm continuous semigroups in the case that the semigroup consists of multiplication operators.

Proof. Assume that $\{\lambda \in \sigma(A): \operatorname{Re} \lambda \geq b\}$ is bounded for every $b \in \mathbb{R}$. Let $t^{\prime}>0$. We show that the semigroup is norm continuous in $t^{\prime}$. Let $\varepsilon>0$. Let $b \in \mathbb{R}$ such that $2 e^{\left(t^{\prime}+1\right) b}<\varepsilon$.\\
If $\operatorname{Re}(m(x)) \leqq b$, then\\
$\left|e^{t m(x)}-e^{t^{\prime} m(x)}\right| \leqq e^{t \cdot \operatorname{Re}(m(x))}+e^{t^{\prime} \cdot \operatorname{Re}(m(x))} \leqq 2 e^{\left(t^{\prime}+1\right) b}<\varepsilon$\\
whenever $\mid t-)^{\prime} \mid \leqq 1$.\\
By hypothesis, the set $H:=\{m(x): x \in X, \operatorname{Re}(m(x)) \geqq b\}$ in the case $E=C_{0}(X)$ and $H:=\{m(x): R e \lambda \geq b$ and for all $n>0$ $\mu(\{x \in X:|m(x)-\lambda|<n\} \neq 0\}$ in the case $E=L^{P}(X, \Sigma, \mu)$ is a bounded subset of $\mathbb{C}$.\\
Thus $\lim _{t \rightarrow t^{\prime}}\left|e^{t z}-e^{t^{\prime} z}\right|=0$ uniformly for $z \in H$. Hence there exists $\delta \in(0,1]$ such that\\[0pt]
[ess]-sup $\left\{\left|e^{t m(x)}-e^{t^{\prime} m(x)}\right|: x \in X, \operatorname{Re}(m(x))>b\right\}<\varepsilon \quad$ whenever $\left|t-t^{\prime}\right|<\delta$. Together with the inequality above, we obtain that |T(t) - $T\left(t^{\prime}\right) \mid=[e s s]-\sup \left\{\left|e^{t m(x)}-e^{t^{\prime} m(x)}\right|: x \in x\right\}<\varepsilon \quad$ whenever $\left|t-t^{\prime}\right|<\delta$. We have shown that the semigroup is norm continuous for $t>0$ whenever $\{\lambda \in \sigma(A): \operatorname{Re} \lambda b\}$ is bounded for $a 11 \mathrm{~b} \in \mathbb{R}$.\\
b) The semigroup is right-sided differentiable in a fixed point $t>0$ if and only if there exists $c>0$ such that $\{\lambda \in \mathbb{C}$ : $\left.|\operatorname{Im} \lambda|>c \cdot e^{-t \operatorname{Re} \lambda}\right\} \rho \rho(\mathrm{A})$.

Proof. The semigroup is right-sided differentiable in $t$ if and only if $T(t) E \subset D(A)$ if and only if $e^{t m} \cdot f \cdot m \in E$ for all $f \in E$ if and only if $e^{t m} \cdot m$ is [essentially] bounded if and only if $e^{\text {tRe } \mathrm{m}} \cdot \mathrm{Im} \mathrm{m}$ is [essentially] bounded if and only if there exists c > 0 such that [ess]-image (m) $\subset\left\{\lambda \in \mathbb{C}: e^{t \operatorname{Re} \lambda}|\operatorname{Im} \lambda| \leqq c\right\}$ if and only if there exists $c>0$ such that $\left\{\lambda \in \mathbb{C}:|\operatorname{Im} \lambda|>c \cdot e^{-t \operatorname{Re} \lambda}\right\} c \rho(A)$.\\
c) $(T(t))_{t \geq 0}$ is a bounded holomorphic semigroup of angle $\theta$ if and only if $S(\theta+\pi / 2) \subset \rho(A)$.

Proof. The condition is necessary by Theorem 1.12.\\
Conversely, if $S(\theta+\pi / 2) \subset \rho(A)$, then one verifies directly that $(T(z) f)(x)=e^{z \cdot m(x)} f(x) \quad(f \in E, x \in X)$ defines a family $(T(z)) z \in S(\theta)$ of bounded operators satisfying conditions (1.4) and (1.5).\\
d) Choosing $X=\mathbb{N}$ and $\mu$ the counting measure we have $E=c_{0}$ or $\ell^{\mathrm{P}}(1 \leqq \mathrm{p}<\infty)$. Then $A$ has a compact resolvent if $\lim _{n \rightarrow \infty}|m(n)|=\infty$. [In fact, let $\lambda>s(A)$. Then $(R(\lambda, A) f)(n)=(\lambda-m(n))^{-1} f(n)$. Hence $R(\lambda, A)$ is compact if and only if $\left(\left(\lambda-(m(n))^{-1)} n \in \mathbb{N} \in c_{0}\right]\right.$\\
The semigroup is compact if and only if it is eventually compact if and only if $\lim _{n \rightarrow \infty} \operatorname{Re}(m(n))=-\infty$.\\
e) Now it is easy to give concrete examples. Again let $\mathrm{X}=\mathbb{N}$, so that $E=C_{0}$ or $\ell^{p}(1 \leq p<\infty)$. Let $m(n)=-n+i \cdot \exp \left(n^{2}\right)$. Then the semigroup is compact and (consequently) norm continuous for $t>0$, but it is not eventually differentiable. Let $m(n)=-n+i e^{t ' n}$. Then the semigroup is differentiable for $t>t^{\prime}$ but not differentiable in $t \in\left[0, t^{\prime}\right)$. If $m(n)=-n+i \cdot n^{2}$, then the semigroup is differentiable but not holomorphic.

\section*{Perturbation of Generators}
A useful way to construct new semigroups out of a given one is by additive perturbation.

Theorem 1.29. Let $A$ be the generator of a strongly continuous semigroup $(T(t))_{t \geqq 0}$ and let $B \in L(E)$. Then $A+B$ with domain $D(A+B)=D(A)$ is the generator of a strongly continuous semigroup $(S(t))_{t \geq 0}$.

It is possible to express the new semigroup $(S(t))_{t \geqslant 0}$ by known objects. The product formula\\
(1.8) $S(t) f=\lim _{n \rightarrow \infty}\left(T(t / n) e^{t / n \cdot B}\right)_{f}$\\
holds for all $t \geqq 0$ and $f \in E$.\\
Moreover, $S(t)$ is the solution of the following integral equation\\
(1.9) $S(t) f=T(t) f+\int_{0}^{t} T(t-s) B S(s) f d s \quad(t \geqq 0, f \in E)$.

Let $S_{0}(t)=T(t)$ and\\
(1.10) $\quad S_{n}(t) f=\int_{0}^{t} T(t-s) B S_{n-1}(s) f d s \quad(f \in E)$ for $n \in \mathbb{N}$. Then\\
(1.11)\\
$s(t)=\sum_{n=0}^{\infty} s_{n}(t)$,\\[0pt]
where the series converges in the operator norm uniformly on bounded intervals. We refer to [Davies (1980), III.1], [Goldstein (1985a), I. 6] or [Pazy (1983), Chap.3] for these results.

Several special properties discussed above are preserved by bounded perturbations.

Theorem 1.30 . Let $(\mathrm{T}(\mathrm{t}))_{t \geq 0}$ be a strongly continuous semigroup with generator $A$. Let $B \in L(E)$. If $(T(t))_{t \geq 0}$ is holomorphic or norm continuous or compact, then so is the semigroup $(S(t))_{t} \geq 0$ generated by $A+B$.\\
If $A$ has a compact resolvent then so has $A+B$.\\
Let $t_{0} \geqq 0$. If $(T(t))_{t \geqslant 0}$ is norm continuous for $t>t_{0}$ and if $B$ is compact, then $(S(t))_{t \geqq 0}$ is also norm continuous for $t>t_{0}$ *

Proof. If (T(t)) ${ }_{t \geqq 0}$ is norm continuous for $t>0$, then $S_{n}(t)$ in (1.10) is norm continuous in $t>0$ for every $n$. Thus (S(t)) $t \geq 0$ is norm continuous in $t>0$ by (1.11). There exists $\lambda_{0} \in \mathbb{R}$ such that $\|\mathrm{R}(\lambda, \mathrm{A})\| \leqq(2\|\mathrm{~B}\|)^{-1}$ for $\operatorname{Re} \lambda \geqq \lambda_{0} \cdot \operatorname{Hence}(I d-\operatorname{BR}(\lambda, \mathrm{A}))^{-1}$ exists for $\operatorname{Re} \lambda \geqq \lambda_{0} \cdot$ since $(\lambda-(A+B)) f=(I d-B R(\lambda, A))(\lambda-A) f$ for all $f \in D(A)$ it follows that $(\lambda-(A+B))^{-1}$ exists and is given by


\begin{equation*}
R(\lambda, A+B)=R(\lambda, A)(\operatorname{Id}-B R(\lambda, A))^{-1} \tag{1.12}
\end{equation*}


whenever $\operatorname{Re} \lambda \geqq \lambda_{0}$. Now if $A$ generates a holomorphic semigroup,\\
there exists $M \geqq 0$ such that $\left\|R\left(\lambda_{0}+i n, A\right)\right\| \leqq M /|\eta|$ for all $n \in \mathbb{R}$. Consequently, $\quad\left\|R\left(\lambda_{0}+i n, A+B\right)\right\| \leqq \|\left(I d-B R\left(\lambda_{0}+i n, A\right)^{-1} \| \cdot M /|n| \leq 2 M /|n|\right.$ for all $\eta \in \mathbb{R}$. Thus $A+B$ generates a holomorphic semigroup by the corollary of Thm.1.14. Moreover, it follows from (1.12) that $R(\lambda, A+B)$ is compact whenever $R(\lambda, A)$ is compact. Consequently by Theorem 1.25 and the assertion proved above, $(S(t))_{t \geqslant 0}$ is compact whenever $(T(t))_{t \geqq 0}$ is compact.\\
Finally assume that $B$ is compact and $t_{0} \geqq 0$ such that $\left.(T)(t)\right)_{t \geqq 0}$ is norm continuous for $t>t_{0}$. Fix $t>t_{0}$. Denote by $u$ the unit ball of $E$ and fix $s \in(0, t]$. Then $\lim _{h \rightarrow 0}(T(t+s-h)-T(t-s)) f=0$ for all $\mathrm{f} \in \overline{\mathrm{B}} \overline{\mathrm{S}} \overline{( } \overline{\mathrm{s}}) \bar{U}=: K$.\\
Since $K$ is compact it follows that the limit exists uniformly in f $\in$ K; i.e. $\quad \lim _{h \rightarrow 0}\|(T(t+s-h)-T(t-s)) B S(s)\|=0$. It follows from the dominated convergence theorem that

\begin{verbatim}
(1.13) $\quad \lim _{h \rightarrow 0} \int_{0}^{t}\|(T(t+s-h)-T(t-s)) B S(s)\| \mathrm{ds}=0$.
Using (1.9) we obtain $\|s(t+h)-s(t)\|$
$\leqq\|T(t+h)-T(t)\|+\left\|\int_{0}^{t+h} T(t+h-s) B S(s) d s-\int_{0}^{t} T(t-s) B S(s) d s\right\|$
$\leqq\|\mathrm{T}(t+h)=T(t)\|+\left\|\int_{t}^{t+h} \mathrm{~T}(t+h-s) B S(s) d s\right\|$
    $+\int_{0}^{t}\|(T(t+h-s)-T(t-s)) B S(s)\| d s \rightarrow 0$ when $h \rightarrow 0$.
\end{verbatim}

In C-IV, Ex.2.15 a generator $A$ of an eventually differentiable and eventually compact semigroup and a bounded operator B will be given such that the semigroup generated by $A+B$ is not eventually norm continuous.

Using Theorem 1.29 we now prove a perturbation result due to DeschSchappacher(1984). Instead of assuming that $B \in L(E)$ we assume that $B \in L(D(A))$. The short proof given below is due to G. Greiner.

Theorem 1.31. Let $(T(t))_{t \geq 0}$ be a strongly continuous semigroup with generator $A$. Assume that $B: D(A) \rightarrow D(A)$ is linear and continuous for the graph norm on $D(A)$.\\
Then $A+B$ with domain $D(A+B)=D(A)$ is the generator of a strongly continuous semigroup. Moreover, there exists a bounded operator $C$ on $E$ such that $A+B$ is similar to $A+C$.

Proof. We first show that (Id - BR $(\lambda, A)$ ) is invertible for some $\lambda$ $€ \mathbb{C}$. Choose $\lambda_{0} \in \rho(A)$. Then $S:=\left(\lambda_{0}-A\right) B R\left(\lambda_{0}, A\right) \in L(E)$. Let $\lambda>$ $s(A)$ be so large such that $\|\operatorname{SR}(\lambda, A)\|<1$.

Then $\left.\left(1-\left(\lambda_{0}-A\right) B R\left(\lambda_{0}, A\right) R(\lambda, A)\right)\right)^{-1}=\left(1-S R\left(\lambda_{1}, A\right)\right)$ is invertible. Consequently, also $(1-\operatorname{BR}(\lambda, A))^{-1}$ exists (since $\sigma\left(\operatorname{TR}\left(\lambda_{0}, A\right)\right) \backslash\{0\}=$ $\left.\sigma\left(R\left(\lambda_{0}, A\right) T\right) \backslash\{0\}, T=\left(\lambda_{0}-A\right) B R(\lambda, A)\right)$.\\
Let $C=(A-\lambda) B(A-\lambda)^{-1} \in L(E)$. Then $A+C$ is the generator of a strongly continuous semigroup by Theorem 1.29 . We show that $A+B$ is similar to $A+C$. In fact, let $U=(1-B R(\lambda, A))$. Then $U$ is an isomorphism on $E$ such that $U(D(A))=D(A)$.\\
Moreover, $U(A+C) U^{-1}=U(A-\lambda+C) U^{-1}+\lambda=U\left[(A-\lambda-(A-\lambda) B R(\lambda, A)] U^{-1}+\lambda\right.$ $=U(A-\lambda)[1-B R(\lambda, A)] U^{-1}+\lambda=U(A-\lambda)+\lambda=A-\lambda+B+\lambda=A+B$.

Corollary 1.32. Keeping the hypotheses and notations of Theorem 1.31 denote by $(S(t))_{t \geqq 0}$ the semigroup generated by $A+B$. If $(T(t))_{t \geqslant 0}$ is norm continuous or compact or holomorphic, then $(s(t))_{t \geqq 0}$ has the corresponding properties. If $B$ is compact as an operator on D(A) endowed with the graph norm and if $(T(t))^{t \geqq 0}$ is eventually norm continuous then so is $(S(t))_{t \geqslant 0^{*}}$

Proof. This follows from Theorem 1.30 since $\left(U^{\prime}(t) U^{-1}\right) t \geqq 0$ has $A+C$ as generator.

\section*{Domains of Uniqueness}
Given a semigroup (T( $t$ ) $t \geq 0$ frequently it is frequently difficult to determine the precise domain of its generator A. So it is important to know which (possibly strict) subspaces of D(A) determine the semigroup uniquely. This can be formulated more precisely in the following way. Let $D_{0}$ be a subspace of $D(A)$ and consider the restriction $A_{0}$ of $A$ to $D_{0}$. Under which condition on $D_{0}$ is $A$ the only extension of $A_{o}$ which is a generator? One obvious condition is that $D_{0}$ is a core. [In fact, in that case, $A$ is the closure of $A_{0}$. Since every generator $B$ extending $A_{0}$ is closed, it follows that $A \subset B$ and hence $A=B$ since $\rho(A) \cap \rho(B) \neq \emptyset]$.

We now show that cores are the only domains of uniqueness.

Theorem 1.33. Let $A$ be the generator of a semigroup and $D_{0} a$ subspace of $D(A)$. Consider the restriction $A_{0}$ of $A$ to $D_{0}$. If $D_{0}$ is not a core of $A$, then there exists an infinite number of extensions of $A_{0}$ which are generators.

Proof. If $D_{0}$ is not dense in $D(A)$ with respect to the graph norm, then there exists a non-zero linear form $\phi$ on $D(A)$ which is continuous for the graph norm such that $\phi(f)=0$ for all $f \in D_{0}$. Let $u \in D(A)$ and $B: D(A) \rightarrow D(A)$ be given by $B f=\phi(f) u$ for all $f \in D(A)$. Then $B$ is continuous for the graph norm. So by Theorem 1.31 the operator $A+B$ with domain $D(A)$ is a generator. Clearly, $A+B \neq A$ if $u \neq 0$ but $A f+B f=A f$ for all $f \in D_{0}$. It is obvious that an infinite number of generators can be constructed in that way.

Corollary 1.34. Let $(T(t))_{t \geq 0}$ be a strongly continuous semigroup with generator $A$. Let $D_{0}$ be a dense subspace of E. Assume that $D_{0} \subset D(A)$ and $T(t) D_{0} c D_{0}$ for all $t \geq 0$. Then $D_{0}$ is a core.

Proof. Let $(S(t))_{t \geq 0}$ be a semigroup with generator $B$ such that $\left.{ }^{B}\right|_{D_{0}}=A_{D O^{\circ}}$ Let $f \in D_{0}$. Then $u(t):=T(t) f$ satisfies $u(0)=f$ and $\quad \dot{u}(t)=\operatorname{AT}(t) f=B T(t) f=B u(t) \quad(t \geq 0)$. Since $\quad v(t)=s(t) f$ ( $t \geqq 0$ ) also is a solution of the Cauchy problem defined by $B$ with initial value $f$ it follows that $S(t) f=T(t) f(t \geqq 0)$. Since $D_{0}$ is dense in $E$, it follows that $s(t)=T(t) \quad(t \geqq 0)$.\\
2. CONTRACTION SEMIGROUPS AND DISSIPATIVE OPERATORS\\
by\\
Wolfgang Arendt

The Hille-Yosida theorem gives a characterization of generators in terms of the resolvent of the operator. However, given an operator A, frequently it is difficult to compute the resolvent (and its powers). So it is desirable to find conditions more immanent on A . This is possible for generators of contraction semigroups. For later purposes (see B-II and C-II) it will be useful not only to consider semigroups which are contractive with respect to the norm but to consider more general sublinear functionals than the norm as well.\\
So our setting is the following. By $E$ we denote a real Banach space throughout, and $p: E \rightarrow \mathbb{R}$ is a continuous sublinear function; i.e., p satisfies\\
(2.1) $\mathrm{p}(\mathrm{f}+\mathrm{g}) \leqq \mathrm{p}(\mathrm{f})+\mathrm{p}(\mathrm{g})$\\
(f, $g \in E)$\\
(2.2) $p(\lambda f)=\lambda p(f)$\\
$(f \in E, \lambda \geqq 0)$.

The continuity of $p$ implies that there exists a constant $c>0$ such that\\
(2.3) $|p(f)| \leqq c\|f\| \quad(f \in E)$.

Moreover, it follows from (2.1) and (2.2) that\\
(2.4) $p(f)+p(-f) \geqq p(0)=0 \quad(f \in E)$.

A bounded operator $T$ on $E$ is called p-contractive if p(Tf) $\leqq$ $p(f)$ for all $f \in E$. Similarly, a semigroup (T(t)) $t_{\geq 00}$ is called p-contractive if $T(t)$ is p-contractive for all $t \geqq 0$.\\
Of course, the most important case we have in mind in this section is the case when $p$ is the norm function $N$ given by $N(f)=\|f\|$ ( $f \in \mathrm{E}$ ). An N-contractive operator is just a contraction in the usual sense.

Remark. However in Chapter B-II and C-II it will be important to dispose of a variety of sublinear functionals other than N . For example, we will consider $\mathrm{N}^{+}$on $\mathrm{C}[0,1]$ given by $\mathrm{N}^{+}$(f) = $\sup _{x \in[0,1]} f(x)$. Then a bounded operator $T$ is $\mathrm{N}^{+}$-contractive if and only if T is positive and $\|\mathrm{T}\| \leqq 1$.

We first want to solve the following problem. Given the generator A of a semigroup $(T(t))_{t \geqq 0}$ find a condition on $A$ which is equivalent to $T(t)$ being $p$-contractive for all $t \geqq 0$.

The subdifferential dp of $p$ in $f$ is defined by\\
(2.5) $d p(f)=\left\{\phi \in E^{\prime}:\langle g, \phi\rangle \leqq p(g)\right.$ for all $g \in E$, $\langle f, \phi\rangle=p(f)\}$.

It follows from the Hahn-Banach theorem that $d p(f) \neq \varnothing$ for all $f \in E$.

Definition 2.1. An operator $A$ on $E$ is called p-dissipative if for all $f \in D(A)$ there exists $\phi \in d p(f)$ such that $\langle A f, \phi>\leqq 0$; A is called strictly p-dissipative if for all $f \in \mathrm{f}(\mathrm{A})$ the inequality <Af, $\phi \leqq 0$ holds for all $\phi \in \operatorname{dp}(f)$.

For convenience we want to have a distinctive name for the norm function. So we denote by $N: E \rightarrow \mathbb{R}$ the function given by $N(f)=$ $\|f\|$ throughout. Then (2.5) can be written in the form\\
(2.6) $\mathrm{dN}(\mathrm{f})=\left\{\phi \in \mathrm{E}^{\prime}:\|\phi\| \leq 1,\langle\mathrm{f}, \phi\rangle=\|\mathrm{f}\|\right\}$.

A [strictly] N -dissipative operator is simply called [strictly] dissipative (which is in accordance with the usual nomenclature).

Example 2.2. a) Let $E=C[0,1], \pounds \in E$.\\
Then there exists $x \in[0,1]$ such that $|f(x)|=\|f\|_{\infty}$. Define $\phi \in E^{\prime}$ by $\langle g, \phi\rangle=(\operatorname{sign} f(x)) g(x)$. Then $\phi \in d N(f)$. Note that dN(f) may be an infinite set.\\
b) Let $H$ be a Hilbert space, $f \in H, f \neq 0$. Then $d N(f)=\left\{\phi_{f}\right\}$ where $<g, \phi_{f}>=1 /\|f\|(g \mid f)$.\\
c) A - |Alid is strictly dissipative for every bounded operator A.

Proposition 2.3. Let $A$ be an operator on E.\\
Then $A$ is p-dissipative if and only if\\
(2.7) $p(f) \leqq p(f-t A f) \quad$ for all $f \in D(A), t>0$.

If in particular $(w, \infty) \subset \rho(A)$ for some $w \in \mathbb{R}$, then $A$ is p-dissipative if and only if


\begin{equation*}
p(\lambda R(\lambda, A) f) \leqq p(f) \quad \text { for } a l l \quad f \in E, \lambda>w \tag{2.8}
\end{equation*}


Proof. Assume that $A$ is $P$-dissipative, Let $f \in D(A), t>0$. There exists $\phi \in \mathrm{dp}(\mathrm{f})$ such that $\langle\mathrm{Af}, \phi\rangle \leqq 0$. Hence, $p(f)=\langle f, \phi\rangle=\langle f-t A f+t A f, \phi\rangle \leqq\langle f-t A f, \phi\rangle \leqq p(f-t A f)$. So (2.7) holds.

Converse, let $f \in D(A)$. For every $t>0$ choose $\phi_{t} \Leftrightarrow \operatorname{dp}(f-t A f)$. Then $\pm<g, \phi_{t}>\leqq p( \pm g) \leqq c\|g\|$ for $a l l g \in E, t>0$. Thus the net $\left(\phi_{t}\right)_{t>0}$ is bounded. Consequently it posseses a $\sigma\left(E^{\prime}, E\right)$ - limit point $\phi$ as $t \rightarrow 0$. We show that $\phi \in d p(f)$ and $<$ Af, $\phi>0$. Since $\left\langle g, \phi_{t}\right\rangle \leqq p(g)$ for all $t>0$ it follows that $\langle g, \phi\rangle \leqq p(g)$ $(g \in E)$. Moreover, <f, $\left.\phi_{t}\right\rangle-t\left\langle A f, \phi_{t}\right\rangle=p(f-t A f)(t>0)$. Letting $t \rightarrow 0$ yields $\langle f, \phi\rangle=p(f)$.\\
We have proved that $\phi \in d p(f)$. By hypothesis we have for all $t>0$, $p(f) \leqq p(f-t A f)=<\pounds-t A f, \phi_{t}>=<f, \phi_{t}>-t<A f, \phi_{t}>\leqq p(f)-t<A f, \phi_{t}>$. Consequently <Af, $\phi_{t}>\leqq 0$ for all $t>0$. Thus <Af, $\phi^{\prime} \leqq 0$.

Remark 2.4. The function $p$ is convex. So the one-sided Gateauxderivatives\\
$D_{g}^{+} p(f)=\lim _{t \downarrow 0} 1 / t(p(f+t g)-p(f)) \quad$ and\\
$\mathrm{D}_{\mathrm{g}}^{-} \mathrm{p}(\mathrm{f})=\lim _{t \uparrow 0} 1 / \mathrm{t}(\mathrm{p}(\mathrm{f}+\mathrm{tg})-\mathrm{p}(f))$\\
exist and satisfy $\mathrm{D}_{\mathrm{g}}^{-} \mathrm{p}(\mathrm{f}) \leqq \mathrm{D}_{\mathrm{g}}^{+} p(f)$ for all $\mathrm{f}, \mathrm{g} \in \mathrm{E}$\\
(cf. Moreau (1966)). Moreover,


\begin{align*}
& D_{g}^{+} p(f)=\sup \{\langle g, \phi\rangle: \phi \in d p(f)\}  \tag{2,9}\\
& D_{g}^{-} p(f)=\inf \{\langle g, \phi\rangle: \phi \in d p(f)\} . \tag{2.10}
\end{align*}


Thus $A$ is p-dissipative if and only if $D_{A f}^{-} p(f) \leqq 0$, and $A$ is strictly p-dissipative if and only if $D_{A f}^{+} P(f) \leqq 0$ for all $f \in$ $\mathrm{D}(\mathrm{A})$.

Corollary 2.5. Let $A$ be a closable operator. If $A$ is p-dissipative, then so is its closure.

Theorem 2.6. Let $p$ be a continuous sublinear functional on a real Banach space E . Let A be the generator of a strongly continuous semigroup ( $T(t){ }_{t \geqq 0}$. The following assertions are equivalent.\\
(i) $p(T(t) f) \leqq p$ (f) for all $t \geqq 0, f \in E$.\\
(ii) A is strictly p-dissipative.\\
(iii) There exists a core $D$ of $A$ such that $A \mid D$ is p-dissipative.

Proof. Assume that (i) holds. Let $f \in D(A), \phi \in \operatorname{dp}(f)$. Then $\langle A f, \phi\rangle=\lim _{t \rightarrow 0} 1 / t(\langle T(t) f, \phi\rangle-\langle f, \phi\rangle)$\\
$=\lim _{t \rightarrow 0} 1 / t(\langle T(t) f, \phi\rangle-p(f))$\\
$\leq \limsup _{t \rightarrow 0} 1 / t(p(T(t) f)-p(f)) \leq 0$.\\
This proves (ii).\\
It is trivial that (ii) implies (iii). So let us assume (iii). Then it follows from Cor. 2.5 that $A$ is p-dissipative. Hence by (2.8) $p(\lambda R(\lambda, A) g) \leqq p(g)$ for all $g \in E, \lambda>\omega(A)$. Hence $\lambda R(\lambda, A)$ is p-contractive for $\lambda>\omega(A)$. This implies that $T(t)$ is p-contractive by the formula (1.3)\\
$T(t)=\lim _{t \rightarrow 0}(n / t R(n / t, A))^{n} \quad($ strongly) for $t \geqq 0$.

We have shown that for generators, p-dissipativity is equivalent to p-contractivity of the semigroup. Now we will consider a p-dissipative operator A (which is not a generator a priori) and investigate under which additional hypotheses $A$ is the generator of a\\
(necessarily contractive) semigroup. At first we present some consequences of p-dissipativity.

Theorem 2.7. Let $A$ be a p-dissipative operator. If $D(A)$ is dense, then $A$ is strictly p-dissipative.

Proof. Let $f \in D(A), \phi \in \operatorname{dp}(f)$. Then for every $t>0$ and $g \in D(A)$ we have\\
$\langle A f, \phi\rangle=1 / t(\langle f+t A f, \phi\rangle-\langle f, \phi\rangle) \leqq 1 / t(p(f+t A f)-p(f))$\\
$\leqq 1 / t(p(f+t g)+t p(A f-g)-p(f))$\\
$\leqq 1 / t(p((I d-t A)(f+t g))+\operatorname{tp}(A f-g)-p(f))$ (by (2.7))\\
$\leqq 1 / t\left(p(f)+\operatorname{tp}(g-A f)+t^{2} p(-A g)+\operatorname{tp}(A f-g)-p(f)\right)$\\
$\leq 1 / t\left(2 t c\|g-A f\|+t^{2} c\|A g\|\right) \quad(b y$ (2.3))\\
$=2 c\|g-A f\|+t c\|A g\|$.

Letting $t \rightarrow 0$ we obtain 〈Af, $\rangle \leq 2 c\|g-A f\|$ for all $g \in D(A)$. since $D(A)$ is dense in $E$, this implies that $<A f, \phi\rangle \leqq 0$

We now impose stronger conditions on p . A continuous sublinear function $p: E \rightarrow \mathbb{R}$ is called half-norm if\\
(2.11) $p(f)+p(-f)>0 \quad$ whenever $f \neq 0$;\\
and $p$ is called a strict half-norm if in addition there exists some constant $d>0$ such that\\
(2.12) $\mathrm{P}(\mathrm{f})+\mathrm{p}(-\mathrm{f}) \geqq \mathrm{d}\|f\| \quad$ for all $\mathrm{f} \in \mathrm{E}$.

If $p$ is a half-norm, then\\
(2.13) $\quad\|f\|_{p}=p(f)+p(-f) \quad(f \in E)$\\
defines a norm on $E$ which is equivalent to the given norm if and only if $p$ is strict.

Remark 2.8. Every half-norm p induces a closed proper cone $\mathrm{E}_{\mathrm{p}}:=$ $\{f \in E: p(-f) \leqq 0\}$ on $E$. Any p-contractive operator $T$ on $E$ leaves the cone $\mathrm{E}_{\mathrm{p}}$ invariant (i.e. T is positive for the corresponding ordering).\\
Conversely, given a closed proper cone $\mathrm{E}_{+}$on E , then $\mathrm{p}(\mathrm{f}):=$ dist $\left(-\mathrm{f}_{,} \mathrm{E}_{+}\right)=\inf \left\{\|f+g\|: g \in E_{+}\right\}$defines a half-norm on $E$ such that $E_{+}=E_{p}$. This half-norm is called the canonical half-norm on the ordered Banach space $\left(E, E_{+}\right)$. The canonical half-norm is strict if and only if the cone $E_{+}$is normal (this is equivalent to the fact that for every $\phi \in \mathrm{E}^{\prime}$ there exist positive linear forms $\phi_{1}$ and $\phi_{2}$ on E such that $\phi=\phi_{1}-\phi_{2}$ (see [Batty-Robinson (1984)] and [Schaefer (1966), Chap.V]).

Proposition 2.9. Let $A$ be a p-dissipative operator where $p$ is a half-norm. If $D(A)$ is dense, then $A$ is closable (and the closure of A is p-dissipative as well (by Cor. 2.5)).

Proof. Let $f_{n} \in D(A), \lim _{n \rightarrow \infty} f_{n}=0, \lim _{n \rightarrow \infty} A f_{n}=g$. We have to show that $g=0$. To this end let $h \in D(A)$. Then (2.7) gives $p\left(f_{n}+t h\right) \leqq p\left(f_{n}+t h-t A\left(f_{f n}+t h\right)\right) \quad(t>0)$. Letting $n \rightarrow \infty$ we obtain $p(t h) \leqq p\left(t h-t g-t^{2} A h\right) \quad(t>0)$.\\
Hence $p(h) \leqq p((h-g)-t A h)(t>0)$ by positive homogeneity. Letting $t+0$ finally we obtain $p(h) \leqq p(h-g)$ for all $h \in D(A)$. since $D(A)$ is dense by hypothesis, we can approximate $g$ by $h \in$ $D(A)$ and conclude that $p(g) \leqq p(0)=0$. since $\lim _{n \rightarrow \infty} A\left(-f_{n}\right)=-g$, we have $\mathrm{p}(-\mathrm{g}) \leqq 0$ by symmetry. Hence $\mathrm{p}(\mathrm{g})+\mathrm{p}(-\mathrm{g}) \leqq 0$ which implies $g=0$ by (2.11).

Lema 2.10. Let $p$ be a half-norm and $A$ a p-dissipative operator. Then


\begin{equation*}
\lambda\|f\|_{p} \leqq\|(\lambda-A) f\|_{p} \quad \text { for all } f \in D(A), \lambda>0 . \tag{2.14}
\end{equation*}


In particular, $(\lambda-A)$ is injective for all $\lambda>0$.\\
If $p$ is strict and $A$ is closed, then im(A - A) is closed for all $\lambda>0$.

Proof. Let $\lambda>0, f \in D(A)$. Then by (2.7), $\lambda p( \pm f) \leqq p((\lambda-A)( \pm f))$. Hence $\quad \lambda\|f\|_{p}=\lambda p(f)+\lambda p(-f) \leqq p((\lambda-A) f)+p(-(\lambda-A) f)=\|(\lambda-A) f\|_{p}$. Thus (2.14) is proved. Now suppose that p is strict. Then $\left\|\|_{p}\right.$ is equivalent to the given norm. Let $\lambda>0$ and $g \in(i m(\lambda-A))^{-}$. Then $g=\lim _{n \rightarrow \infty}(\lambda-A) f_{n}$ for some sequence $\left(f_{n}\right){ }_{n} \in \mathbb{N} \subset \mathrm{D}(A)$. It follows from (2.14) that $\left(f_{n}^{\prime}{ }_{n \in \mathbb{N}}\right.$ is a Cauchy sequence. Let $f=1 i m_{n \rightarrow \infty} f_{n}$. Then $\lim _{n \rightarrow \infty} A f_{n}=\lambda \lim _{n \rightarrow \infty} f_{n}-\lim _{n \rightarrow \infty}(\lambda-A) f_{n}=\lambda f-g$ exists. If $A$ is closed, this implies that $f \in D(A)$ and $A f=\lambda f-g$. Hence $g=(\lambda-A) f \in \operatorname{im}(\lambda-A)$. We have shown that im( $\lambda-\mathrm{A})$ is closed.

The following is the main theorem of this section.

Theorem 2.11. Let $p$ be a strict half-norm and $A$ an operator on E . The following assertions are equivalent.\\
(i) A is the generator of a p-contraction semigroup.\\
(ii) $D(A)$ is dense, $A$ is p-dissipative and im $(\lambda-A)=E$ for some $\lambda>0$.

Proof. Since $p$ is a strict half-norm we can assume that $\|f\|=\|f\|_{p}$ for all $\pounds \in E$. (i) implies (ii) by Theorem 2.6.\\
Now suppose that (ii) holds. Then it follows from Lemma 2.10 that $\mu \in \rho(A)$ and $\|\mu R(\mu, A)\| \leqq 1$ whenever $\mu>0$ such that im $(\mu-A)=E$. So by hypothesis $\lambda \in \rho(A)$ and dist $(\lambda, \sigma(A)) \geqq\|R(\lambda, A)\|^{-1} \geqq \lambda$. Hence $(0,2 \lambda) \subset \rho(A)$. Iterating this argument we see that $(0, \infty) \subset \rho(A)$. It follows from the Hille-Yosida theorem that A generates a contraction semigroup $(\mathbb{T}(t))_{t \geqq 0}$. Finally, from Thm. 2.6 it follows that $(T(t))_{t \geq 0}$ is p-contractive.

Of course, the norm function $N$ given by $N(f)=\|f\|$ is a strict half-norm. In the case when $\mathrm{p}=\mathrm{N}$, Theorem 2.11 is due to Lumer and Phillips (1961). It turns out to be extremely useful in showing that a concrete operator is a generator. Because of its importance we state this special case explicitly below (including the complex case). Before that let us formulate Theorem 2.11 for the case when the operator is merely given on a core.

Corollary 2.12. Let $p$ be a strict half-norm and $A$ be a densely defined operator. If $A$ is p-dissipative and ( $\lambda-A$ ) has dense range for some $\lambda>0$, then $A$ is closable and the closure $\bar{A}$ of $A$ generates a p-contraction semigroup.

Proof. It follows from Prop. 2.9 that $A$ is closable and the closure $\bar{A}$ is p-dissipative. Lemma 2.10 implies that $(\lambda-\bar{A}) D(\bar{A})=E$. So Thm. 2.11 yields the desired conclusion.

We conclude this section indicating the results for the complex case.

Let $E$ be a complex Banach space and $p: E \rightarrow \mathbb{R}_{+}$be a seminorm on $E$ (i.e., $p(f+g) \leqq p(f)+p(g)$ and $p(\lambda f)=|\lambda| p(f)$ holds for all $f, g \in E, \lambda \in \mathbb{C}$ ). The subdifferential $d p(f)$ of $p$ in $f \in E$ is defined by\\
(2.15) $d p(f)=\left\{\phi \in E^{\prime}: \operatorname{Re}\langle g, \phi\rangle \leqq p(g)\right.$ for all $g \in E$ and $\langle f, \phi\rangle=p(f)\}$.

We assume in addition that $p$ is continuous. Then it follows from the Hahn-Banach theorem that $\operatorname{dp}(f) \neq \varnothing$ for any $f \in E$.

A linear operator $A$ on $E$ is called p-dissipative if for all\\
$f \in D(A)$ there exists $\phi \in \operatorname{dp}(f)$ such that Re<Af, $\phi>\leq 0$.\\
The arguments given above show that also in the situation considered here A is p-dissipative if and only if

$$
p((1-t A) f) \geqq p(f)
$$

for all $f \in D(A), t \geqq 0$.

The results of this section carry over if they are appropriately modified. We explicitly state the most important result for the case when $p$ is the norm. A linear operator A is simply called dissipative if it is N-dissipative where $N(f)=\|f\| \quad$ (f $\in$ E).

Theorem 2.13 (Lumer-Phillips). Let A be a densely defined operator on a complex Banach space E . The following assertions are equivalent.\\
(i) A is closable and the closure of $A$ is the generator of a contraction semigroup.\\
(ii) A is dissipative and $(\lambda-\mathrm{A})$ has dense range for some $\lambda>0$.

\section*{3. SEMIGROUPS ON $\mathrm{I}^{\infty}$ AND $\mathrm{H}^{\infty}$}
\section*{by}
Heinrich P. Lotz

In this section we shall prove that on $L^{\infty}$, on $H^{\infty}$ (D), and on some other classical Banach spaces every strongly continuous semigroup of operators is uniformly continuous.

Lemma 3.1. Let $T=(T(t))_{t \geq 0}$ be a one-parameter semigroup of operators on a Banach space $E$. Suppose that $s=\lim \sup _{t}, 0\|T(t)-I d\|$ is finite. If $\lim _{t \rightarrow 0}\left\|(T(t)-I d)^{2}\right\|=0$, then $T$ is uniformly continuous.

Proof. The identity $2(T(t)-I d)=T(2 t)-I d-(T(t)-I d)^{2}$ shows\\
that $2\|T(t)-I d\|-\left\|(T(t)-I d)^{2}\right\| \leqq\|T(2 t)-I d\|$. Hence\\
$2 \mathrm{~s} \leqslant \lim \sup _{t \downarrow 0} \| \mathrm{T}(2 \mathrm{t})$ - Id|. Obviously, lim sup ${ }_{t+0} \| \mathrm{T}(2 \mathrm{t})$-Id|=s and so, $2 \mathrm{~s} \leq \mathrm{s}$. Consequently, $\mathrm{s}=0$.

Remarks. 1. If in Lemma 3.1 $T=(T(t))_{t \geq 0}$ is strongly continuous, in which case $s<\infty$, one can replace $\lim _{t \rightarrow 0}\left\|(\mathrm{~T}(t)-\mathrm{Id})^{2}\right\|=0$ by the weaker condition $\lim \sup r(T(t)-I d)<1$ [Lotz (1985), Lemma 2] where $r$ denotes the spectral radius.\\
2. The condition $s<\infty$ in Lemma 3.1 is essential as the following example shows:

Let $f: \mathbb{R} \rightarrow \mathbb{R}$ be non-continuous with $f(s+t)=f(s)+f(t)$ for all $s, t \in \mathbb{R}$ (see[Hamel (1905)]). Then $(t, f(t)): t \in \mathbb{R}$ ) is dense in $\mathbb{R}^{2}$. Hence for the semigroups $T=(T(t))_{t \geq 0}$ on $\mathbb{R}^{2}$ with

$$
T(t)=\left(\begin{array}{cc}
1 & f(t) \\
0 & 1
\end{array}\right) \quad \text { for } \quad t \geq 0
$$

we have $s=\infty$. Therefore $T$ is not uniformly continuous. However, $(T(t)-I d)^{2}=0$ for all $t \geqq 0$.

Lemma 3.2. Let $T=(T(t))_{t \geqq 0}$ be a one-parameter semigroup of operators on a Banach space E. Then the following assertions are equivalent:\\
(a) $T^{\prime}=\left(T(t)^{\prime}{ }_{t \geq 0}\right.$ is a strongly continuous semigroup on the dual E'.\\
(b) ((T(t) - Id) $x_{n}$ ) converges weakly to zero for every bounded sequence $\left(x_{n}\right)$ in $E$ and every sequence ( $t_{n}$ ) in $[0, \infty)$ with $\lim t_{n}=0$.

Moreover, (a) implies\\
(c) $T$ is strongly continuous.

Proof, Let $x^{\prime} \in E^{\prime}$ and $t_{n} \geqq 0$ be given. Then 1 im $\|\left(T\left(t_{n}\right)-\right.$ Id)' $x^{\prime} \|=0$ if and only if $\lim \left\langle x_{n},\left(T\left(t_{n}\right)-I d\right)^{\prime} x^{\prime}\right\rangle=0$ for every bounded sequence $\left(x_{n}\right)$ in $E$. This easily implies the equivalence of (a) and (b). In particular, (a) implies that ( $\left(t_{n}\right.$ ) - Id) $x$ ) converges weakly to zero for every sequence $\left(t_{n}\right)$ in $[0, \infty)$ with $\lim t_{n}=0$ and every $x \in E$. Hence $T$ is strongly continuous by Propositon 1.23 in [Davies (1980)].

We recall that a Banach space $E$ is called a Grothendieck space if every weak* convergent sequence in $E^{\prime}$ converges weakly.

Theorem 3.3. Let $E$ be a Grothendieck space. If $T=(T(t))_{t \geqq 0}$ is a strongly continuous semigroup in E , then $T^{\prime \prime}=\left(\mathrm{T}(\mathrm{t})^{\prime \prime}\right) \mathrm{t} \geq 0$ is strongly continuous in $\mathrm{E}^{\prime \prime}$.

Proof. Suppose that $\left(x_{n}^{\prime}\right)$ is a bounded sequence in $E^{\prime}$ and that $t_{n} \geqq 0$ with 1 im $t_{n}=0$. Put $v_{n}=T\left(t_{n}\right)-I d$. Then Iim $\left\|v_{n} x\right\|=0$ and therefore $\lim \left\langle x_{,} V_{n}^{\prime} x_{n}^{\prime}\right\rangle=0$ for every $x \in E$. Hence $\left(V_{n}^{\prime} x_{n}^{\prime}\right)$ $w^{*}$-converges to zero. Since $E$ is a Grothendieck space ( $V_{n}^{\prime} x_{n}^{\prime}$ ) converges weakly to zero. Now Lemma 3.2 implies that (T(t)") is strongly continuous.

Recall now that a Banach space $E$ is said to have the Dunford-Pettis property if $\lim \left\langle x_{n}, x_{n}^{\prime}\right\rangle=0$ whenever $\left(x_{n}\right)$ in $E$ and $\left(x_{n}^{\prime}\right)$ in $E^{\prime}$ converge weakly to zero.

Theorem 3.4. Let $E$ be a Banach space with the Dunford-Pettis property and let $T=(T(t))_{t \geq 0}$ be a one-parameter semigroup of operators on E . If $T "=\left(T(t)^{\prime \prime}\right)_{t \geqq 0}$ is strongly continuous in $\mathrm{E}^{\prime \prime}$, then $T$ is uniformly continuous.

Proof. Suppose that $T^{\prime \prime}$ is a strongly continuous semigroup. Then Lemma 3.2 implies that $T$ and $T$ are strongly continuous. Hence by the uniform boundedness principle, $\lim \sup _{t \rightarrow 0} \|(T(t)-I d)$ is finite. By Leman 3.1 it suffices to show that $\lim _{t \rightarrow 0}\left\|(T(t)-I d)^{2}\right\|=$ 0 . Let $t_{n} \geqq 0$ with $\lim t_{n}=0$ be given. Then there exists a bounded sequence $\left(x_{n}\right)$ in $E$ and a bounded sequence $\left(x_{n}^{\prime}\right)$ in $E^{\prime}$ such that $\left\|\left(T\left(t_{n}\right)-I d\right)^{2}\right\|=\left\langle\left(T\left(t_{n}\right)-I d\right) x_{n},\left(T\left(t_{n}\right)-I d\right)^{\prime} x_{n}^{\prime}\right\rangle$.\\
since $T$, and $T^{\prime \prime}$ are strongly continuous, Lemma 3.2 implies that $\left(\left(T\left(t_{n}\right)-I d\right) x_{n}\right)$ and $\left.\left(T\left(t_{n}\right)-I d\right)^{\prime} x_{n}^{\prime}\right)$ converge weakly to zero. Since $E$ has the Dunford-Pettis property, $\lim \left\|\left(T\left(t_{n}\right)-I d\right)^{2}\right\|=0$. Consequently, $\quad \lim _{t \rightarrow 0}\left\|(\mathrm{~T}(\mathrm{t})-\mathrm{Id})^{2}\right\|=0$.

An imediate consequence of Theorem 3.3 and Thoerem 3.4 is the following.

Theorem 3.5. Let $E$ be a Grothendieck space with the Dunford-Pettis property. Then every strongly continuous semigroup of operators on E is uniformly continuous.

A compact Hausdorff space is called an F-space if the closures of two disjoint open $F_{\sigma}$-sets are disjoint and is called a Stonean (res., $\underline{\sigma}$-Stonean) space if the closure of every open set (res., open $F_{\sigma}$-set) is open. Every o-Stonean space is an F-space.

Theorem 3.6. Every strongly continuous semigroups of operators on one of the following Banach spaces is uniformly continuous:

\begin{enumerate}
  \item $C(K)$, where $K$ is a compact F-space.
  \item $L^{\infty}(S, \Sigma, \mu)$ for any measure space $(S, \Sigma, \mu)$.
  \item The Banach space $B(S, \Sigma)$ of bounded $\Sigma$-measurable functions on $S$ if $\Sigma$ is a $\sigma$-algebra of subsets of $S$.
  \item The Banach space $H(O)$ of bounded continuous solutions of
\end{enumerate}

$$
\Sigma_{1 \leqq i \leq n}\left(\partial^{2} f / \partial x_{i}^{2}\right)=0
$$

on an open subset 0 of $\mathbb{R}^{n}$.\\
5) The Banach space $W(0)$ of bounded continuous solutions of

$$
\sum_{1 \leqq i \leqq n}\left(\partial^{2} f / \partial x_{i}^{2}\right)=\left(\partial f / \partial x_{n+1}\right)
$$

on an open subset 0 of $\mathbb{R}^{n+1}$.\\
6) The Banach space $H^{\infty}(0)$ of bounded analytic functions on a finitely connected domain 0 of the complex plane.

Proof. By Theorem 3.5 it suffices to show that the space listed above are Grothendieck spaces with the Dunford-Pettis property.

\begin{enumerate}
  \item If $K$ is compact, then $C(K)$ has the Dunford-Pettis property [Grothendieck (1953) Theorème 4]. If K is a compact F-space, then C(K) is a Grothendieck space [Seever (1958) Theorem 2.5]; the special cases for stonean and o-Stonean spaces are due to [Grothendieck (1953), Thérème 9] and [Ando (1961)] respectively.
  \item and 3) It is well known that every o-order complete AM-space with unit is isometric to a space $C(K)$ where $K$ is a compact o-Stonean space. Obviously, the spaces under 2) and 3) are o-order complete AM-spaces with unit and therefore by 1) Grothendieck spaces with the Dunford-Pettis property.
  \item and 5) These spaces are order complete vector lattices. This follows from [Bauer (1966) pp.18-22, Standardbeispiele 1 and 2 p.55]. Since these spaces contain the constant functions on $O$ they are complete for the supremum-norm. Indeed if ( $f_{n}$ ) is a cauchy-sequence for this norm, it is easily seen that ( $f_{n}$ ) converges in norm to $\inf { }_{n} \sup \left(f_{k}: n<k\right.$ ). Therefore these spaces are $\sigma$-order complete AM-spaces with unit and so as before Grothendieck spaces with the Dunford-Pettis property.
  \item Bourgain [(1980), Cor. 3] proves that $H^{\infty}$ (D) has the DunfordPettis property and in [(1984), Proposition III.1], that $H^{\infty}$ (D) is a
\end{enumerate}

Grothendieck space, where $D$ is the open unit disc $\{z:|z|<1\}$. If $O$ is a finitely connected domain and $\mathrm{H}^{\infty}$ does not only contain the constant functions, then $H^{\infty}(0)$ is isomorphic to a finite direct sum of copies of $H^{\infty}$ (D). (Note that $H^{\infty}$ (D) is isomorphic to $\left\{f \in H^{\infty}(D): f(0)=0\right\}$ via the map $f \rightarrow z \pounds$. Then use [Grothendieck (1953),p.77 and Prop.4.4.1J). Hence $H^{\infty}(0)$ is a Grothendieck space with the Dunford-Pettis property.

Final Remark. It follows from Theorem 3.6 that on $L^{\infty}$ the infinitesimal generator of a strongly continuous semigroup is necessarily bounded. It is not obvious that on $L^{\infty}([0,1])$ there exist closed densely defined unbounded operators. To see this let $A$ be a closed densely defined unbounded operator form $\ell^{2}$ into $L^{\infty}([0,1])$ with domain D (such operators can easily be constructed). By the Khintchine inequality, the map $R:\left(a_{n}\right) \rightarrow \sum a_{n} r_{n}$ where $r_{n}$ denotes the $\mathrm{n}^{\text {th }}$ Rademacher function, from $\ell^{2}$ into $\mathrm{L}^{1}([0,1])$ is a topological isomorphism. Hence $T=R^{\prime} \operatorname{maps} L^{\infty}([0,1])$ onto $2^{2}$. Banach's homomorphism theorem implies that $\mathrm{T}^{-1}(\mathrm{D})$ is dense in $\mathrm{L}^{\infty}([0,1])$ and that AT is a closed densely defined unbounded operator on $\mathrm{L}^{\infty}([0,1])$ with domain $\mathrm{T}^{-1}(\mathrm{D})$. This solves a problem raised by R.Kaufman.\\
H. Porta and the author have shown that if a Banach space $E$ has an infinite dimensional separable quotient space and $F$ is an infinite dimensional Banach space then there always exists a closed densely defined unbounded operator from E into F .

NOTES.

Section 1. The abstract Cauchy problem is treated systematically in the monographs of Krein (1971) and Fattorini (1983). We refer to these books for more details and historical notes. One implication of Theorem 1.1 is proved in Krein (1971) (Thm.2.11).\\
The Hille-Yosida theorem has been proved independently by Hille (1948) and Yosida (1948) for contraction semigroups. The extension to arbitrary strongly continuous semigroups is independently due to Feller (1953), Miyadera (1952) and Phillips (1953). Thus our terminology is slightly incorrect, some authors refer to the general version as the Hille-Yosida-Phillips theorem which is slightly more correct. Holomorphic semigroups belong to the standard material of the theory of oneparameter semigroups. Our Theorem 1.14 deviates from the usual presentation since the condition on the resolvent is merely required on a half-plane.\\
Differentiable semigroups are treated in detail in the book of Pazy (1983) who discovered Theorem 1.17 and 1.18 .

The spectral property of eventually norm continuous semigroups given in Theorem 1.20 is contained in Hille-Phillips (1957) (Thm.16.4.2) with a proof depending on Gelfand theory. For norm continuous semigroups it is contained in Pazy (1983) with a simpler proof. The elementary proof we give here is due to G. Greiner. Theorem 1.29 on the perturbation by bounded operators is due to Phillips (1953) who also investigated permanence of smoothness properties by this kind of perturbation. We also refer to Pazy (1983) (Sec.3.1).\\
The observation that eventually norm continuity is preserved by perturbation by a compact operator (see Thm. 1.30) seems to be new.\\
The perturbation by continuous operators on the graph of the generator is due to Desch-Schappacher (1984). The short proof we give here is due to G. Greiner and has the advantage to yield the same permanence for smoothness properties as in the classical case (Cor.l.32).\\
The characterization of a core as "domain of uniqueness" (Thm.1.33) seems to be new. In this section we have presented part of the standard theory of one-parameter semigroups including some new aspects. A very elegant brief introduction to oneparameter semigroups is given in the treatise of Kato (1966) where one can also find all the results on perturbation theory going beyond the elementary facts we discuss here. A complete information on the general theory can be obtained by consulting the books of Davies (1980), Goldstein (1985a) and Pazy (1983). The monograph of Goldstein (1985a) in particular contains a variety of examples and applications.

Section 2. Dissipative operators were introduced by Lumer-Phillips (1961). The analogous notion of dispersiveness is due to Phillips (1962). Our approach follows closely Arendt-Chernoff-Kato (1982) where half-norms were introduced. Related previous results were obtained by Calvert (1971), Hasegawa (1966), Sato (1968), Bénilan-Picard (1979) and Picard (1972), where the two last consider non-1inear semigroups. A further investigation of half-norms can be found in Batty-Robinson (1983) who consider ordered Banach spaces other than Banach lattices in great detail. We also refer to the historical notes given there.

Section 3. It had been proved by Kishimoto-Robinson (1981) that every generator of $a_{\infty}$ positive semigroup on L is bounded. That every strongly continuous semigroup on $\mathrm{L}^{\infty}$ is uniformly continuous was first shown by Lotz (1982), (1984), (1985). The proof of Lemma 3.1 was commicated to the author by T. Coulhon, who independently obtained a particular case (Coulhon (1984)).

\section*{SPECTRALTHEORY }
by\\
Günther Greiner and Rainer Nagel

\section*{1. INTRODUCTION}
In this chapter we start a systematic analysis of the spectrum of a strongly continuous semigroup $T=(T(t))_{t \geqslant 0}$ on a complex Banach space E . By the spectrum of the semigroup we understand the spectrum $\sigma(A)$ of the generator $A$ of $T$. In particular we are interested in precise relations between $\sigma(A)$ and $\sigma(T(t))$. The heuristic formula

$$
" T(t)=e^{t A} "
$$

serves as a leitmotiv and suggests relations of the form

$$
" \sigma(T(t))=e^{t^{\sigma}(A)}=\left\{e^{t^{\lambda}}: \lambda \in \sigma(A)\right\} n,
$$

called 'spectral mapping theorem'. These - or similar - relations will be of great use in Chapter IV and enable us to determine the asymptotic behavior of the semigroup $T$ by the spectrum of the generator. As a motivation as well as a preliminary step we concentrate here on the spectral radius


\begin{equation*}
r(T(t)):=\sup \{|\lambda|: \lambda \in \sigma(T(t))\}, t \geqq 0 \tag{1.1}
\end{equation*}


and show how it is related to the spectral bound\\
(1.2) $s(A):=\sup \{\operatorname{Re} \lambda: \lambda \in \sigma(A)\}$\\
of the generator A and to the growth bound\\
(1.3) $w:=\inf \left\{w \in \mathbb{R}:\|w(t)\| \leqq M_{w} \cdot e^{w t}\right.$ for all $t \geqq 0$ and suitable $\mathrm{M}_{\mathrm{W}}$ \}\\
of the semigroup $T=(T(t))_{t \geq 0}$ * (Recall that we sometimes write $\omega(T)$ or $\omega(A)$ instead of $\omega)$. The Examples 1.3 and 1.4 below illustrate the main difficulties to be encountered.

Proposition 1.1. Let $\omega$ be the growth bound of the strongly continuous semigroup $T=(T(t))_{t \geq 0}$. Then


\begin{equation*}
r(T(t))=e^{\omega t} \tag{1.4}
\end{equation*}


for every $t \geqq 0$.

Proof. From A-I, (1.1) we know that

$$
\omega(T)=\lim _{t \rightarrow \infty} \frac{1}{t} \log \|\mathrm{~T}(t)\|
$$

Since the spectral radius of $T(t)$ is given as

$$
r(T(t))=\lim _{n \rightarrow \infty}\|T(n t)\|^{1 / n}
$$

we obtain for $t>0$

$$
\begin{aligned}
r(T(t)) & =\lim _{n \rightarrow \infty} \exp \left(t(n t)^{-1} \log \|T(n t)\|\right) \\
& =e^{\omega t} .
\end{aligned}
$$

It was shown in A-I,Prop. 1.11 that the spectral bound $s(A)$ is always dominated by the growth bound $a$ and therefore $e^{s(A) t} \leqq r(T(t))$. If the above mentioned spectral mapping theorem holds - as is the case for bounded generators (e.g., see Thm. VII. 3.11 of Dunford-Schwartz (1958)) we obtain the equality

$$
e^{s(A) t}=r(T(t))=e^{\omega t}
$$

hence $s(A)=\omega$. Therefore the following corollary is a consequence of the definitions of $s(A)$ and $\omega$.

Corollary 1.2. Consider the semigroup $T=(T(t))_{t \geq 0}$ generated by some bounded linear operator $A \in L(E)$. If $\operatorname{Re} \lambda<0$ for each $\lambda \in \sigma(A)$ then $\lim _{t \rightarrow \infty}\|T(t)\|=0$.

Through this corollary we have re-established a famous result of Liapunov which assures that the solutions of the linear Cauchy problem

$$
\dot{x}(t)=A x(t), x(0)=x_{0} \in \mathbb{C}^{n} \text { and } A=\left(a_{i j}\right)_{n \times n}
$$

are 'stable' ; i.e., they converge to zero as $t \rightarrow \infty$, if the real parts of all eigenvalues of the matrix A are smaller than zero.

For unbounded generators the situation is much more difficult and s(A) may differ drastically from $\omega$.

Example 1.3. (Banach function space, Greiner-Voigt-Wolff (1981)) Consider the Banach space $E$ of all complex valued continuous functions on $\mathbb{R}_{+}$which vanish at infinity and are integrable for $e^{x} d x$, i.e.

$$
E:=c_{0}\left(\mathbb{R}_{+}\right) \cap L^{1}\left(\mathbb{R}_{+}, e^{x} d x\right)
$$

endowed with the norm

$$
\|f\|:=\|f\|_{\infty}+\|f\|_{1}=\sup \left\{|f(x)|: x \in \mathbb{R}_{+}\right\}+\int_{0}^{\infty}|f(x)| e^{x} d x .
$$

The translation semigroup

$$
T(t) f(x):=f(x+t)
$$

is strongly continuous on $E$ and one shows as in $A-I, 2.4$ that its generator is given by

$$
A f=f^{\prime}, D(A)=\left\{f \in E: f \in C^{1}\left(\mathbb{R}_{+}\right), f^{\prime} \in E\right\}
$$

First we observe that $\|T(t)\|=1$ for every $t \geqq 0$, hence $\omega(T)=0$. Moreover it is clear that $\lambda$ is an eigenvalue of $A$ as soon as Re $<-1$ (in fact : the function

$$
x+\varepsilon_{\lambda}(x):=e^{\lambda x}
$$

belongs to $\mathrm{D}(\mathrm{A})$ and is an eigenvector of A , hence $s(\mathrm{~A}) \geqq-1$, For $f \in E, \operatorname{Re} \lambda>-1$,

$$
\|\cdot\|_{1}-\lim _{t \rightarrow \infty} \int_{0}^{t} e^{-\lambda s} T(s) f d s
$$

exists since $\|T(s) f\|_{1} \leqq e^{-s}\|f\|_{1}, s \geq 0$, and

$$
\|\cdot\|_{\infty}-1 i m_{t \rightarrow \infty} \int_{0}^{t} e^{-\lambda s} T(s) f d s
$$

exists since $\int_{0}^{\infty} e^{x}|f(x)| d x<\infty$. Therefore $\int_{0}^{\infty} e^{-\lambda s} T(s) f d s$ exists in $E$ for every $f \in E, \operatorname{Re\lambda }>-1$. As we observed in A-I,Prop.1.11 this implies $\lambda \in \rho(A)$. Therefore $T=(T(t))_{t \geq 0}$ is a semigroup having $s(A)=-1$ but $\omega(T)=0$.

Example 1.4. (Hilbert space, Zabczyk (1975)) For every n $\in \mathbb{N}$ consider the n-dimensional Hilbert space $E_{n}:=\mathbb{C}^{n}$ and operators $A_{n} \in L\left(E_{n}\right)$ defined by the matrices

$$
A_{n}=\left(\begin{array}{ccccc}
0 & 1 & \cdot & \cdot & 0 \\
\cdot & 0 & 1 & & \cdot \\
& \cdot & \cdot & \cdot & \\
\cdot & & & \cdot & 1 \\
0 & \cdot & & \cdot & 0
\end{array}\right)_{n \times n}
$$

These matrices are nilpotent and therefore $\sigma\left(A_{n}\right)=\{0\}$. The elements $x_{n}:=n^{-1 / 2}(1, \ldots, 1) \in E_{n}$ satisfy the following properties :\\
(i) $\left\|x_{n}\right\|=1$ for every $n \in \mathbb{N}$,\\
(ii) $\quad \lim _{n \rightarrow \infty}\left\|A_{n} x_{n}-x_{n}\right\|=0$,\\
(iii) $\quad \lim _{n \rightarrow \infty}\left\|\exp \left(t A_{n}\right) x_{n}-e^{t} x_{n}\right\|=0$.

Consider now the Hilbert space $E:=\oplus_{n \in N} E_{n}$ and the operator $A:=\left(A_{n}+2 \pi i n\right)_{n \in N}$ with maximal domain in $E$. Analogously we define a semigroup $T=(T(t))_{t \geq 0}$ by

$$
T(t):=\left(e^{2 \pi i n t} \exp \left(t A_{n}\right)\right){ }_{n \in N}
$$

Since $\left\|\exp \left(t A_{n}\right)\right\| e^{t}$ for every $n \in \mathbb{N}, t \geqq 0$, and since $t \rightarrow T(t) x$ is continuous on each component $E_{n}$ it follows that $T$ is strongly continuous. Its generator is the operator $A$ as defined above.\\
For $\lambda \in \mathbb{C}, \operatorname{Re} \lambda>0$, we have $\lim _{n \rightarrow \infty}\left\|\mathrm{R}\left(\lambda-2 \pi i n, A_{n}\right)\right\|=0$, hence

$$
\left(R\left(\lambda, A_{n}+2 \pi i n\right)\right)_{n \in \mathbb{N}}=\left(R\left(\lambda-2 \pi i n, A_{n}\right)\right)_{n \in \mathbb{N}}
$$

is a bounded operator on $E$ representing the resolvent $R(\lambda, A)$. Therefore we obtain $s(A) \leqq 0$. On the other hand, each $2 \pi$ in is an eigenvalue of $A$, hence $s(A)=0$.\\
Take now $x_{n} \in E_{n}$ as above and consider the sequence ( $x_{n}$ ' $n \in \mathbb{N}$. From (iii) it follows that for $t>0$ the number $e^{t}$ is an approximate eigenvalue of $T(t)$ with approximate eigenvector ( $\left.x_{n}\right)_{n} \in \mathbb{N}$ (see Def. 2.1 below). Therefore $e^{t} \leqq r(T(t)) \leqq\|T(t)\|$ and hence $\omega(T) \geqq 1$. On the other hand, it is easy to see that $\|T(t)\|=e^{t}$, hence $\omega(T)=1$.\\
Finally if we take $S(t):=e^{-t / 2} \cdot \mathbb{T}(t)$ we obtain a semigroup having spectral bound $-\frac{1}{2}$ but such that $\lim _{t \rightarrow \infty}\|S(t)\|=\infty$ in contrast with Cor. 1.2 •

These examples show that neither the conclusion of Cor.1.2, i.e. 's $(A)<0$ implies stability', nor the 'spectral mapping theorem'

$$
\sigma(T(t))=\exp (t \cdot \sigma(A))
$$

is valid for arbitrary strongly continuous semigroups. A careful analysis of the general situation will be given in section 6 below, but we will first develop systematically the necessary spectral theoretic tools for unbounded operators.

\section*{2. THE FINE STRUCTURE OF THE SPECTRUM}
As usual, with a closed linear operator A with dense domain D(A) in a Banach space $E$, we associate its spectrum $\sigma(\mathrm{A})$, its resolvent set $\rho(A)$ and its resolvent

$$
\lambda \rightarrow \mathrm{R}(\lambda, \mathrm{~A}):=(\lambda-\mathrm{A})^{-1}
$$

which is a holomorphic map from $\rho(A)$ into $L(E)$. In contrast to the finite dimensional situation, where a linear operator fails to be surjective if and only if it fails to be injective, we now have to distinguish different cases of 'non-invertibility' of $\lambda$ - A . This distinction gives rise to a subdivision of $\sigma(\mathrm{A})$ into different subsets. We point out that these subsets need not be disjoint, but our defini-\\
tion seems to be justified by the fact that for each of the following subsets of $\sigma(A)$ there exist canonical constructions converting the corresponding spectral values into eigenvalues (see prop. 2.2.ii and Prop. 4.5 below).

Definition 2.1. For a closed, densely defined, Iinear operator A with domain $D(A)$ in the Banach space $E$ denote by the\\
(i) point spectrum $P \sigma(A)$ the set of all $\lambda \in \mathbb{C}$ such that $\lambda-\mathbb{A}$ is not injective.\\
(ii) approximate point spectrum $A \sigma(A)$ the set of all $\lambda \in \mathbb{C}$ such that $\lambda-A$ is not injective or $(\lambda-A) D(A)$ is not closed in E.\\
(iii) residual spectrum Ro(A) the set of all $\lambda \in \mathbb{C}$ such that $(\lambda-A) D(A)$ is not dense in E.

From these definitions it follows that $\lambda \in \mathbb{C}$ is an eigenvalue of $A$, i.e. $\lambda \in P \sigma(A)$, if and only if there exists an eigenvector $0 \neq f \in D(A)$ such that $A f=\lambda f$. It follows from the open Mapping Theorem that $\lambda \in A \sigma(A)$ if and only if $\lambda$ is an approximate eigenvalue, i.e. there exists a sequence $\left(f_{n}\right)_{n \in \mathbb{N}} \subset \mathrm{D}(\mathrm{A})$, called an approximate eigenvector, such that $\left\|f_{n}\right\|=1$ and $\lim _{n \rightarrow \infty}\left\|A f_{n}-\lambda f_{n}\right\|$ $=0$.\\
Clearly we have $P \sigma(A) \subset A \sigma(A)$ and $\sigma(A)=A \sigma(A) \cup \operatorname{Ro}(A)$ where the union need not be disjoint.

The following proposition is a first indication that the subdivision we made implies nice properties.

Proposition 2.2. For a closed, densely defined, lineax operator (A, D(A)) in a Banach space $E$ the following holds:\\
(i) The topological boundary $\partial \sigma(A)$ of $\sigma(A)$ is contained in $A \sigma(A)$.\\
(ii) $\quad R \sigma(A)=P \sigma\left(A^{\prime}\right)$ for the adjoint operator $A^{\prime}$ on $E^{\prime}$.

Proof. (i) Take $\lambda_{0} \in \partial \sigma(A)$ and $\lambda_{n} \in \rho(A)$ such that $\lambda_{n} \rightarrow \lambda_{0}$. since $\left\|R\left(\lambda_{n}, A\right)\right\| \geqq\left(\text { dist }\left(\lambda_{n}, \sigma(A)\right)\right)^{-1} \quad($ see Prop. $2.5 .(i j))$, by the uniform boundedness principle we find $f \in E$ such that

$$
\lim _{n \rightarrow \infty}\left\|R\left(\lambda_{n}, A\right) f\right\|=\infty .
$$

Define $\quad g_{n} \in D(A)$ by

$$
g_{n}:=\left\|R\left(\lambda_{n}, A\right) f\right\|^{-1} R\left(\lambda_{n}, A\right) f
$$

and use the identity

$$
\left(\lambda_{0}-A\right) g_{n}=\left(\lambda_{0}-\lambda_{n}\right) g_{n}+\left(\lambda_{n}-A\right) g_{n}
$$

to show that $\left(g_{n}\right)_{n \in \mathbb{N}}$ is an approximate eigenvector corresponding to $\lambda_{0}$.\\
(ii) This is a simple consequence of the Hahn-Banach theorem.

In order to illuminate the above definitons we now return to the Standard Examples introduced in Section 2 of A-I and discuss the fine structure of the spectrum of these strongly continuous semigroups, i.e. of their generators and their semigroup operators.

\subsection*{2.3 The Spectrum of Multiplication Semigroups.}
Take $E=C_{0}(X)$ for some locally compact space $X$ and take a continuous function $q: X \rightarrow \mathbb{C}$ whose real part is bounded above. As observed in A-I,2.3 the multiplication operator

$$
M_{q}: E \rightarrow q \cdot f
$$

with maximal domain $D\left(M_{q}\right)$ generates the multiplication semigroup

$$
T(t) f:=e^{t q} \cdot f \quad, f \in E
$$

Since $M_{q}$ is bounded if and only if $q$ is bounded we conclude that ${ }^{M} q$ is invertible (with bounded inverse $M_{1 / q}$ ) if and only if

$$
0 \notin T q(x): x \in x\}
$$

Therefore we obtain

$$
\left.\sigma\left(M_{q}\right)=\overline{q(x)}=\overline{q(x)}: x \in x\right\}
$$

and

$$
\sigma(T(t))=\overline{\exp (\operatorname{tq}(x)): x \in X]}
$$

In particular the following 'weak spectral mapping theorem' is valid:

$$
\sigma(T(t))=\overline{\exp \left(t \sigma\left(M_{G}\right)\right)} .
$$

In addition we observe that to each spectral value of A (resp. of $\mathrm{T}(\mathrm{t})$ ) there exists an approximate eigenvector and hence

$$
\sigma(A)=A \sigma(A) \quad \text { and } \quad \sigma(T(t))=A \sigma(T(t))
$$

Since each Dirac functional is an eigenvector for the adjoint multiplication operator we obtain

$$
q(x) \subset \operatorname{Ro}\left(M_{q}\right) \quad \text { and } e^{t q(x)} \subset \operatorname{Ro}(T(t))
$$

The eigenvalues of $M_{q}$ can be characterized as follows:\\
$\lambda \in \operatorname{Po}\left(\mathrm{M}_{\mathrm{q}}\right)$ if and only if the set $\{\mathrm{x} \in \mathrm{X}: \mathrm{q}(\mathrm{x})=\lambda\}$ has non empty interior (analogously for $\mathrm{Po}_{\sigma}(\mathrm{T}(\mathrm{t})$ ) ) For example, it follows that $\operatorname{Po}\left(M_{q}\right)=\emptyset$ for $E=C_{0}\left(\mathbb{R}_{+}\right)$and $q(x)=-x, x \in \mathbb{R}_{+}$.

On $\mathrm{E}=\mathrm{L}^{\mathrm{P}}(\mathrm{X}, \Sigma, \mu)$ analogous results are valid, but their exact formulation - using the notion 'essential range', see Goldstein (1985a) is left to the reader.

\subsection*{2.4 The Spectrum of Translation Semigroups.}
First we consider the translation semigroup

$$
T(t) f(x):=f(x+t)
$$

on $E=C_{0}\left(\mathbb{R}_{+}\right)$( or $L^{P_{1}}\left(\mathbb{R}_{+}\right)$, see $A-I, 2.4$ ). Its generator $A$ is the first derivative and for every $\lambda \in \mathbb{C}, \operatorname{Re} \lambda<0$, the function $\varepsilon_{\lambda}: x \rightarrow e^{\lambda x}$ belongs to $D(A)$ and satisfies

$$
A \varepsilon_{\lambda}=\lambda \varepsilon_{\lambda}
$$

hence $\lambda \in \operatorname{Po}(A)$. Since $T=(T(t))_{t \geq 0}$ is a contraction semigroup it follows that $\sigma(A)=\{\lambda \in \mathbb{C}: \operatorname{Re} \lesssim \leqq 0\}$ and $i \mathbb{R} \subset \mathrm{~A} \sigma(\mathrm{~A})$ (use Prop. 2.2. (i) or show directly that $f_{n}(x)=e^{i \alpha x} \cdot e^{-x / n}$ defines an approximate eigenvector for $i \alpha, \alpha \in \mathbb{R}$, Using the same functions one obtains

$$
\operatorname{Po}(T(t))=\left\{e^{\lambda t}: \operatorname{Re} \lambda<0\right\}=\{z \in \mathbb{C}:|z|<1\}
$$

and $\sigma(T(t))=\{z \in \mathbb{C}:|z| \leqq 1\}$ for every $t>0$.\\
In the case of the translation group on $E=C_{0}(\mathbb{R})$ one has that $\sigma(A) \subset i \mathbb{R}$. As above one obtains approximate eigenvectors for every $\alpha \in \mathbb{R}$ from $f_{n}(x)=e^{j \alpha x} \cdot e^{-|x| / n}$, hence

$$
\sigma(A)=A \sigma(A)=i \mathbb{R}
$$

The generator A of the nilpotent translation semigroup A-I, 2.6 has empty spectrum by A-I, Prop.1.11. The resolvent is given by\\
$R(\lambda, A) f(x)=e^{\lambda x} \int_{x}^{\tau} e^{-\lambda s_{f}} f(s) d s \quad\left(f \in L^{p}([0, \tau], \lambda \in \mathbb{C})\right.$.\\
Finally the generator of the periodic translation group from $A-I, 2.5$ on $E=\{f \in C[0,1]: f(0)=f(1)\}$ has point spectrum

$$
P \sigma(A)=2 \pi i \mathbb{Z}
$$

with eigenfunctions $\varepsilon_{n}(x):=\exp (2 \pi i n x)$. In Section 5 we show that

$$
\sigma(A)=2 \pi i \mathbf{Z} .
$$

We now return to the general theory and recall from Corollary 1.2 that it is very useful (e.g., for stability theory) to be able to convert\\
spectral values of the generator A into spectral values of the semigroup operator $\mathrm{T}(\mathrm{t})$ and vice versa. As shown in Examples 1.3 and 1.4 this is not possible in general. Therefore we tackle first a much easier 'spectral mapping theorem': the relation between $\sigma(A)$ and $\sigma\left(R\left(\lambda_{0}\right)\right)$, where $R\left(\lambda_{0}\right):=R\left(\lambda_{0}, A\right)$ for some $\lambda_{0} \in \rho(A)$.

Proposition 2.5. Let (A,D(A)) be a densely defined closed linear operator with non-empty resolvent set $\rho(A)$. For each $\lambda_{0} \in \rho(A)$ the following assertions hold :\\
$\sigma\left(R\left(\lambda_{0}\right)\right) \backslash\{0\}=\left(\lambda_{0}-\sigma(A)\right)^{-1}$. In particular, $r\left(R\left(\lambda_{0}\right)\right)=\left(\operatorname{dist}\left(\lambda_{0}, \sigma(A)\right)\right)^{-1}$.\\
(ii) Analogous statements hold for the point-, approximate point-, residual spectra of $A$ and $R\left(\lambda_{0}, A\right)$.\\
(iii) The point $\alpha$ is isolated in $\sigma(A)$ if and only if $\left(\lambda_{0}^{-\alpha}\right)^{-1}$ is isolated in $\sigma\left(\mathrm{R}\left(\lambda_{0}\right)\right)$. In that case the residues (resp., the pole orders) in $\alpha$ and in $\left(\lambda_{0}{ }^{-\alpha}\right)^{-1}$ coincide.

Proof. (i) is well known. It can be found for example in [DunfordSchwartz (1958), VII.9.2J.\\
(ii) We show that $\alpha \in \operatorname{A\sigma }(A)$ if $\left(\lambda_{0}-\alpha\right)^{-1} \in A \sigma\left(R\left(\lambda_{0}\right)\right)$ and leave the proof of the remaining statements to the reader. Take $\left(f_{n}\right)_{n \in \mathbb{N}} \subset E$ such that $\left\|f_{n}\right\|=1,\left\|\left(\lambda_{0}-\alpha\right)^{-1} f_{n}-R\left(\lambda_{0}, A\right) f_{n}\right\| \rightarrow 0$ and $\left\|R\left(\lambda_{0}, A\right) f_{n}\right\| \geqq \frac{1}{2}\left|\lambda_{0}-\alpha\right|^{-1}$. Define

$$
g_{n}:=\left\|R\left(\lambda_{0}, A\right) f_{n}\right\|^{-1} \cdot R\left(\lambda_{0}, A\right) f_{n} \in D(A)
$$

and deduce from

$$
\begin{aligned}
(\alpha-A) g_{n} & =\left\|R\left(\lambda_{0}, A\right) f_{n}\right\|^{-1} \cdot\left[\left(\lambda_{0}-A\right)-\left(\lambda_{0}-\alpha\right)\right] R\left(\lambda_{0}, A\right) f_{n} \\
& =\left\|R\left(\lambda_{0}, A\right) f_{n}\right\|^{-1} \cdot\left(\lambda_{0}-\alpha\right)\left[\left(\lambda_{0}-\alpha\right)^{-1}-R\left(\lambda_{0}, A\right)\right] f_{n}
\end{aligned}
$$

that $\left(g_{n}\right)$ is an approximate eigenvector of $A$ to the eigenvalue $\alpha$. (iii) Take a circle $r$ with centex $\alpha$ and sufficiently small radius. Then the residue $P$ of $R(., \mathrm{A})$ at $\alpha$ is

$$
\begin{aligned}
P & =\frac{1}{2 \pi i} \int_{\Gamma} R(z, A) d z \\
& =\frac{1}{2 \pi i} \int_{\Gamma}\left(\lambda_{O}-z\right)^{-2} R\left(\left(\lambda_{0}-z\right)^{-1}, R\left(\lambda_{0}, A\right)\right) d z \\
& \left.\quad-\frac{1}{2 \pi i} \int_{\Gamma}\left(\lambda_{0}-z\right)^{-1} d z, \text { (use } \$ \$\right) .
\end{aligned}
$$

If $\lambda_{0}$ lies in the exterior of $\Gamma$ the second integral is zero. The\\
substitution $\tilde{z}:=\left(\lambda_{0}-z\right)^{-1}$ yields a path $\tilde{I}^{2}$ around $\left(\lambda_{0}^{-\alpha}\right)^{-1}$ and we obtain

$$
P=\frac{1}{2 \pi \dot{i}} \int_{\Gamma} R\left(\tilde{z}, R\left(\lambda_{0}, A\right)\right) d z^{2},
$$

which is the residue of $R\left(\ldots, R\left(\lambda_{0}, A\right)\right)$ at $\left(\lambda_{0}{ }^{-\alpha}\right)^{-1}$. The final assertion on the pole order follows from the identities

$$
v_{-n}=\left(\left(\lambda_{0}-\alpha\right)^{-1} R\left(\lambda_{0}, A\right)\right)^{n-1} u_{-n}, \quad n \in N,
$$

where $U_{n}$, resp. $V_{n}$ stand for the $n$-th coefficient in the Laurent series of $R(., A)$, resp. $R\left(., R\left(\lambda_{0}, A\right)\right)$ at $\alpha$, resp. $\left(\lambda_{0}-\alpha\right)^{-1}$. This has already been proved for $n=1$ and follows for $n>1$ by induction, using the relations

$$
U_{-n-1}=(A-\alpha) U_{-n} \text { and } \quad V_{-n-1}=\left(R\left(\lambda_{0}, A\right)-\left(\lambda_{0}-\alpha\right)^{-1}\right) V_{-n}
$$

\section*{3. SPECTRAL DECOMPOSITION}
In the next two sections we develop some important techniques for our further investigation of semigroups and their generators. Even though these methods are well known (compare, e.g. Section VII. 3 of DunfordSchwartz (1958)) or rather technical, it is useful to present them in a coherent way.\\
Our interest in this section is the following : Let E be a Banach space and $T=(T(t))_{t \geqq 0}$ a strongly continuous semigroup with generator A . Suppose that the spectrum $\sigma(\mathrm{A})$ splits into the disjoint union of two closed subsets $\sigma_{1}$ and $\sigma_{2}$. Does there exist a corresponding decomposition of the space $E$ and the semigroup $T$ ?

In the following definition we explain what we understand by "corresponding decomposition".

Definition 3.1. Assume that $\sigma(\mathrm{A})$ is the disjoint union

$$
\sigma(A)=\sigma_{1} \cup \sigma_{2}
$$

of two non-empty closed subsets $\sigma_{1}, \sigma_{2}$. A decomposition

$$
\mathrm{E}=\mathrm{E}_{1} \oplus \mathrm{E}_{2}
$$

of $E$ into the direct sum of two non-trivial closed T-invariant subspaces is called a spectral decomposition corresponding to $\sigma_{1} U \sigma_{2}$ if the spectrum $\sigma\left(A_{i}\right)$ of the generator $A_{i}$ of $T_{i}:=\left(T(t) \mid E_{i}{ }^{\prime} t \geqslant 0\right.$ coincides with $\sigma_{i}$ for $i=1,2$.

For a better understanding of the above definition we recall that to every direct sum decomposition $\mathrm{E}=\mathrm{E}_{1} \oplus \mathrm{E}_{2}$ there corresponds a continuous projection $P \in L(E)$ such that $P E=E_{1}$ and $P^{-1}(0)=E_{2}$. Moreover, the subspaces $E_{1}, E_{2}$ are T-invariant if and only if $P$ commutes with the semigroup $T$, i.e. $T(t) P=\operatorname{PT}(t)$ for every $t \geqq 0$. In this case it follows that the domain $D(A)$ of the generator $A$ splits analogously and $D(A) \cap E_{i}$ is the domain $D\left(A_{i}\right)$ of the generator $A_{i}$ of the restricted semigroup $T_{i}, i=1,2$, We write

$$
A=A_{1} \oplus A_{2},
$$

say that " A commutes with P " and call P a spectral projection. In terms of the generator $A$ this means that for $f \in D(A)$ we have $\operatorname{Pf} \in D(A)$ and $A P f=$ PAf.\\
The existence of such projections is very helpful since it reduces the semigroup $T$ into two (possibly simpler) semigroups $T_{1}, T_{2}$ such that

$$
\sigma(A)=\sigma\left(A_{1}\right) \cup \sigma\left(A_{2}\right) \quad \text { and } \quad \sigma(T(t))=\sigma\left(T_{1}(t)\right) \cup \sigma\left(T_{2}(t)\right)
$$

For example, in some cases (see Theorem 3.3 below) it can be shown that one of the reduced semigroups has additional properties.

In order to achieve such decompositions we will assume that o(A) decomposes into sets $\sigma_{1}$ and $\sigma_{2}$ and will then try to find a corresponding spectral projection. Unfortunately such spectral decompositions do not exist in general.

Example 3.2. Take the rotation semigroup from $A-I, 2.4$ on the Banach space $L^{\mathrm{P}}(\Gamma), 1 \leqq \mathrm{p}<\infty, \tau=2 \pi$. It was stated in 2.4 and will be proved in Section 5 that its generator A has spectrum

$$
\sigma(A)=P_{\sigma}(A)=i \mathbb{Z},
$$

where $\varepsilon_{k}(z):=z^{k}$ spans the eigenspace corresponding to $i k, k \in \mathbf{Z}$. Now, $\sigma(A)$ is the disjoint union of $\sigma_{1}:=\{0, i, 2 i, \ldots\}$ and $\sigma_{2}:=\{-i,-2 i, \ldots\}$. By a result of M. Riesz there is no projection $P^{2} \in L\left(L^{1}(r)\right)$ satisfying $P_{\varepsilon_{k}}=\varepsilon_{k}$ for $k \geqq 0, P \varepsilon_{k}=0$ for $k<0$ (see Lindenstrauss-Tzafriri (1979), p.165), hence there is no spectral decomposition of $\mathrm{L}^{1}(\mathrm{r})$ corresponding to $\sigma_{1}, \sigma_{2}$. On the other hand, for $\mathrm{L}^{\mathrm{p}}(\Gamma), 1<\mathrm{p}<\infty$, such a spectral projection exists (l.c., 2.c.15). As long as $p \neq 2$ we can always decompose o(A) into suitable subsets admitting no spectral decomposition (1.c., remark before 2.c.15). Clearly, for $p=2$ such spectral decompositions always exist.

In the above example both subsets $\sigma_{1}, \sigma_{2}$ of $\sigma(A)$ are unbounded. But as soon as one of these sets is bounded a corresponding spectral decomposition can always be found.

Theorem 3.3. Let $T$ be a strongly continuous semigroup on a Banach space $E$ and assume that the spectrum $\sigma(A)$ of the generator $A$ can be decomposed into the disjoint union of two non-empty closed subsets $\sigma_{1}, \sigma_{2}$. If $\sigma_{1}$ is compact then there exists a unique corresponding spectral decomposition $E=E_{1} \oplus E_{2}$ such that the restricted semigroup $T_{1}$ has a bounded generator.

Proof. We assume the reader to be familiar with the spectral decomposition theorem for bounded operators (see e.g. [Dunford-Schwartz (1958), p.5721) and apply the "spectral mapping theorem" for the resolvent (A-III,Thm.2.5) in order to decompose $R(\lambda, A)$ instead of $A$ : For $\lambda_{0}>\omega(T)$ it follows from $A-I I I$, Thm. 2.5 that $\sigma\left(R\left(\lambda_{0}, A\right)\right) \backslash\{0\}$ $=\left(\lambda_{0}-\sigma(A)\right)^{-1}$. From $\sigma(A)=\sigma_{1} U \sigma_{2}$ we obtain a decomposition of $\sigma\left(R\left(\lambda_{0}, A\right)\right) \backslash\{0\}$ into

$$
\mathrm{t}_{1}:=\left(\lambda_{0}-\sigma_{1}\right)^{-1}, \quad \mathrm{~T}_{2}:=\left(\lambda_{0}-\sigma_{2}\right)^{-1} .
$$

Since $\sigma_{1}$ is compact the set ${ }^{\tau}{ }_{1}$ is compact and does not contain 0 . only in the case that $\sigma_{2}$ is unbounded the point 0 will be an accumulation point of $\tau_{2}$. Therefore $\sigma\left(R\left(\lambda_{0}, A\right)\right) U\{0\}$ is the disjoint union of the closed sets ${ }^{\tau}{ }_{1}$ and ${ }^{\tau}{ }_{2} U\{0\}$.\\
Take now $P$ to be the spectral projection of $R\left(\lambda_{0}, A\right)$ corresponding to this decomposition. Then $P$ commutes with $R\left(\lambda_{0}, A\right)$ (by definition), with $\mathrm{R}(\lambda, A)$ for every $\lambda>\omega(T)$ (use the series representation of the resolvent), with $T(t)$ for each $t \geqq 0$ (use $A-I I$, Prop.1.10) and therefore with the generator $A$ (in the sense explained above). In particular, we obtain

$$
R\left(\lambda_{0}, A\right) P=R\left(\lambda_{0}, A_{1}\right), R\left(\lambda_{0}, A\right)(I d-P)=R\left(\lambda_{0}, A_{2}\right)
$$

for the generator $A_{1}$ of $T_{1}=(T(t) P)_{t \geqq 0}$ and $A_{2}$ of $T_{2}=(T(t)(I d-P))_{t \geqq 0}$. Applying the Spectral Mapping Theorem 2.5 we conclude

$$
\sigma\left(A_{1}\right)=\sigma_{1} \quad \text { and } \quad \sigma\left(A_{2}\right)=\sigma_{2}
$$

i.e., $P$ is a spectral projection corresponding to ${ }_{1}{ }_{1}$, ${ }_{2}$. Finally, the above spectral decomposition of $R\left(\lambda_{0}, A\right)$ is unique and satisfies $0 \neq \sigma\left(\mathrm{R}\left(\lambda_{0}, A_{1}\right)\right)$. Therefore $R\left(\lambda_{0}, A_{1}\right)^{-1}=\left(\lambda_{0}-A_{1}\right)$ is bounded.

Example. If we do not require $T_{1}$ to be uniformly continuous the above spectral decomposition need not be unique :\\
Consider a decomposition $\mathrm{E}=\mathrm{E}_{1} \oplus \mathrm{E}_{2}$ and add a direct summand $\mathrm{E}_{3}$ with a strongly continuous semigroup $T_{3}$ whose generator $\mathrm{A}_{3}$ has empty spectrum (e.g. A-I, Example 2.6). Then still $\sigma(A)=\sigma_{1} U \sigma_{2}$ but $E_{1} \oplus\left(E_{2} \oplus E_{3}\right)$ and $\left(E_{1} \oplus E_{3}\right) \oplus E_{2}$ are two different spectral decompositions corresponding to $\sigma_{1}, \sigma_{2}$.

The importance of the above theorem stems from the fact that $T_{1}$ has a bounded generator and therefore is easy to deal with. In particular the asymptotic behavior of $T_{1}$ can be deduced from the location of $\sigma_{1}$.

Corollary 3.4. Assume that $\sigma(A)$ splits into non-empty closed sets $\sigma_{1}, \sigma_{2}$ where $\sigma_{1}$ is compact and consider the corresponding spectral decomposition $\mathrm{E}=\mathrm{E}_{1} \oplus \mathrm{E}_{2}$ for which $T_{1}$ is uniformly continuous. For all constants $v, w \in \mathbb{R}$ satisfying\\
$\mathrm{v}<\inf \left\{\operatorname{Re} \lambda: \lambda \in \sigma_{1}\right\}$ and $\sup \left\{\operatorname{Re} \lambda: \lambda \in \sigma_{1}\right\}<w$ there exist $m, M \geqq 1$ such that

$$
m \cdot e^{v t}\|f\| \leqq\left\|\mathrm{T}_{1}(t) f\right\| \leqq M \cdot e^{w t}\|f\|
$$

for every $f \in E_{1}, t \geqq 0$.

Proof. Since the generator $A_{1}$ of $T_{1}$ is bounded we have $T_{1}(t)=$ $=\exp \left(t A_{1}\right)$ and $\sigma\left(I_{1}(t)\right)=\exp \left(t_{\sigma}\left(A_{1}\right)\right)$. Therefore by the remark following Prop.1.1 the spectral bound $s\left(A_{1}\right)$ coincides with the growth bound $w\left(T_{1}\right)$ and we have the upper estimate. The lower estimate is obtained by applying the same reasoning to $-\mathrm{A}_{1}$ which generates the semigroup $\left(\mathrm{T}_{1}(t)^{-1}\right)_{t \geqq 0}$ on $\mathrm{E}_{1}$.

It is clear from Examples 1.3. 1.4 that no norm estimates for $\left(\mathrm{T}_{2}(t)\right)_{t \geq 0}$ can be obtained from the location of $\sigma_{2}$. Only by adding appropriate hypotheses we will achieve spectral decompositions admitting norm estimates on both components (see A-III, 6.6).\\
Another way of obtaining such norm estimates is by constructing spectral decompositions starting from a semigroup operator $T\left(t_{0}\right)$ (instead of A resp. $R(\lambda, A)$, as in Thm.3.3).

Corollary 3.5. If $\sigma\left(\mathrm{T}\left(\mathrm{t}_{0}\right)\right)=\tau_{1} U \tau_{2}$ for two non-empty, closed, disjoint sets ${ }^{\tau_{1}},{ }^{\tau_{2}}$ and if $P$ is the spectral projection correspon-\\
ding to $T\left(t_{0}\right)$ and ${ }^{\tau}{ }_{1},{ }^{\tau}{ }_{2}$, then $\sigma(A)$ splits into closed subsets $\sigma_{1}, \sigma_{2}$ and $P$ is the corresponding spectral projection for $T$ and $\sigma_{1}, \sigma_{2}$.

Proof. The spectral projection $P$ of $T\left(t_{0}\right)$ is obtained by integrating $R\left(\lambda, T\left(t_{0}\right)\right)$ (see e.g. IDunford-Schwartz (1958), Section VII. 3l). Since every $T(t), t \geqq 0$, commutes with $T\left(t_{0}\right)$ it must commute with $R\left(\lambda, T\left(t_{0}\right)\right)$, hence with $P$. The statement on the decomposition $\sigma(A)=\sigma_{1} \cup \sigma_{2}$ follows from the spectral Inclusion Theorem 6.2 below.

This decomposition can be applied as follows to the study of the asymptotic behavior of $T$ : In the situation of Cor.3.5 assume

$$
\sup \left\{|\lambda|: \lambda \in \tau_{2}\right\}<\alpha<\inf \left\{|\lambda|: \lambda \in{ }_{T}{ }_{1}\right\}
$$

If we set $\beta:=(\log \alpha) / t_{0}$ and use [Pazy (1984), Chap. I, Thm. 6.5] we obtain $\omega\left(T_{2}\right)<\beta$ and $\left.\omega^{0} T_{1}^{-1}\right)<8$ by Prop.1.1. Therefore we have constants $\mathrm{m}, \mathrm{M} \geqq 1$ such that

$$
\begin{array}{ll}
\|T(t) f\| \leqq M \cdot e^{B t}\|f\| & \text { for } f \in E_{2}, \\
\|T(t) f\| \geqq m \cdot e^{B}\|f\| & \text { for } f \in E_{1} .
\end{array}
$$

As nice as they might look results of this type are unsatisfactory : we need information on the semigroup in order to estimate its asymptotic behavior. In Chapter IV we will try to obtain such results by exploiting information about the generator only.

\subsection*{3.6 Isolated singularities and poles.}
In case that $\lambda_{0}$ is an isolated point of $\sigma(A)$ the holomorphic function $\lambda \rightarrow R(\lambda, A)$ can be expanded as a Laurent series\\
$R(\lambda, A)=\sum_{n=-\infty}^{+\infty} U_{n}\left(\lambda-\lambda_{0}\right)^{n}$ for $0<\left|\lambda-\lambda_{0}\right|<\delta$ and some $\delta>0$. The coefficients $U_{n}$ are bounded linear operators given by\\
(3.1) $U_{n}=\frac{1}{2 \pi i} \int_{\Gamma}\left(z-\lambda_{0}\right)^{-(n+1)} R(z, A) d z, n \in \mathbb{Z}$,\\
where $\Gamma=\left\{z \in \mathbb{C}:\left|z-\lambda_{0}\right|=\delta / 2\right\}$.\\
The coefficient $U_{-1}$ is the spectral projection corresponding to the spectral set $\left\{\lambda_{0}\right\}$ (see Def.3.1), it is called the residue of $R(\cdot, A)$ at $\lambda_{0}$, and will be denoted by $P$. From (3.1) one deduces\\
(3.2) $U_{-(n+1)}=\left(A-\lambda_{0}\right)^{n} \cdot p$ and\\
$U_{-(n+1)}{ }^{\circ} U_{-(m+1)}=U_{-}(n+m+1)$ for $n, m \geq 0$.

If there exists $k>0$ such that $U_{-k} \neq 0$ while $U_{-n}=0$ for all $n>k$ the point $\lambda_{0}$ is called a pole of $\mathrm{R}(\cdot, \mathrm{A})$ of order $k$. In view of $(3.2)$ this is true if $U_{-k} \neq 0$ and $U_{-}(k+1)=0$. In this case one can retrieve $\mathrm{U}_{-\mathrm{k}}$ as\\
(3.3) $U_{-k}=\lim _{\lambda \rightarrow \lambda_{0}}\left(\lambda-\lambda_{0}\right)^{k_{R}}(\lambda, A)$.

The dimension of PE (i.e., the dimension of the spectral subspace corresponding to $\left\{\lambda_{0}\right\}$, is called algebraic multiplicity $m_{a}$ of $\lambda_{0}$, while the geometric multiplicity is $\mathrm{m}_{\mathrm{g}}:=\operatorname{dim} k e r\left(\lambda_{0}-\mathrm{A}\right)$. In case $m_{a}=1$ we call $\lambda_{0}$ an algebraically simple pole.\\
If $k$ is the pole order $(k=\infty$ in case of an essential singularity) we have\\
(3.4) $\max \left(m_{g}, k\right\} \leqq m_{a} \leqq k \cdot m_{g}$ ,\\
where $\infty=0=\infty$. These inequalities yield the following implications:\\
$-m_{a}<\infty$ if and only if $\lambda_{0}$ is a pole with $\mathrm{m}_{g}<\infty$,

\begin{itemize}
  \item if $\lambda_{0}$ is a pole with order $k$, then $\lambda_{0} \in \operatorname{Po}(A)$ and $\mathrm{PE}=\operatorname{ker}\left(\lambda_{0}-A\right)^{k}$.
\end{itemize}

If A has compact resolvent then every point of $\sigma(\mathrm{A})$ is a pole of finite algebraic multiplicity. This is a consequence of Prop.2.5(iii) and the well-known Riesz-Schauder Theory for compact operators (see [Dunford-Schwartz (1958),VII.4.5]).

\subsection*{3.7. The essential spectrum.}
For $T \in L(E)$ the Fredholm domain $\rho_{F}(T)$ is\\
(3.5) $\rho_{F}(T):=\{\lambda \in \mathbb{C}: \lambda-T$ is a Fredholm operator $\}$\\
$=\{\lambda \in \mathbb{C}: \operatorname{ker}(\lambda-\mathrm{T})$ and $\mathrm{E} / \mathrm{im}(\lambda-\mathrm{T})$ are finite dimensional\} .

An equivalent characterization of $\rho_{F}(T)$ is obtained through the Calkin algebra $L(E) / K(E)$, where $K(E)$ stands for the closed ideal of all compact operators. In fact, $\rho_{F}(T)$ coincides with the resolvent set of the canonical image of $T$ in the Calkin algebra. The complement of $\rho_{\mathrm{F}}(\mathrm{T})$ is called essential spectrum of T and denoted by $\sigma_{e s s}(T)$. The corresponding spectral radius, called essential spectral radius, satisfies


\begin{equation*}
r_{\text {ess }}(T):=\sup \left\{|\lambda|: \lambda \in \sigma_{\text {ess }}(T)\right\}=\lim _{n \rightarrow \infty}\left\|T^{n}\right\|_{\text {ess }} / n \tag{3,6}
\end{equation*}


where $\|T\|_{\text {ess }}=\operatorname{dist}(T, K(E)):=\inf \{\|T-K\|: K \in K(E)\}$ is the norm of $T$ in $L(E) / K(E)$.\\
For every compact operator $K$ we have $\|\mathrm{T}-\mathrm{K}\|_{\text {ess }}=\|\mathrm{T}\|_{\text {ess }}$, hence


\begin{equation*}
r_{\text {ess }}(T-K)=r_{\text {ess }}(T) \tag{3.7}
\end{equation*}


A detailed analysis of $\rho_{F}(T)$ can be found in Section IV. 5.6 of Kato (1966). In particular we recall that the poles of $\mathrm{R}(\cdot, \mathrm{T})$ with finite algebraic multiplicity belong to $\rho_{\mathrm{F}}(\mathrm{T})$. Conversely, an element of the unbounded component of $\rho_{F}(T)$ either belongs to $\rho(T)$ or is a pole of finite algebraic multiplicity. Thus $r_{\text {ess }}(\mathrm{T})$ can be characterized as follows\\
(3.8) $r_{\text {ess }}(T)$ is the smallest $r \in \mathbb{R}_{+}$such that every $\lambda \in \sigma(T)$,\\
$|\lambda|>r$ is a pole of finite algebraic multiplicity.\\
Now, if $T=(T(t))_{t \geqq 0}$ is a strongly continuous semigroup then VIII.1, Lemma 4 of Dunford-Schwartz (1958) applied to the function $t \rightarrow \log \|\mathrm{~T}(\mathrm{t})\|_{\text {ess }}$ ensures that\\
(3.9) $\operatorname{less}(\mathrm{T}):=\lim _{t \rightarrow \infty} \frac{1}{t} \log \|\mathrm{~T}(t)\|_{\text {ess }}=\inf \left\{\frac{1}{t} \log \|\mathrm{~T}(t)\|_{\text {ess }}: t>0\right\}$ is well defined (possibly $-\infty$ ). By the definition of $\omega_{\text {ess }}(T)$ and by (3.6) we have\\
(3.10) $r_{\text {ess }}(T(t))=\exp \left(t \cdot \omega_{\text {ess }}(T)\right), t \geqq 0$.

Obviously, wess $\leqq \omega$ and equality occurs if and only if $r_{\text {ess }}(T(t))=r(T(t))$ for $t \geqq 0$.\\
If $\omega_{\text {ess }}<\omega$ there exists an eigenvalue $\lambda$ of $T(t)$ satisfying $|\lambda|$ $=r(T(t))$, hence by Theorem 6.3 below there exists $\lambda_{1} \in \operatorname{Po}(A)$ such that $\operatorname{Re} \lambda_{1}=\omega$. Thus $\omega_{\text {ess }}<\omega$ implies $s(A)=\omega(T)$, i.e., we have


\begin{equation*}
\omega(T)=\max \left\{\omega_{\text {ess }}(T), s(A)\right\} \tag{3.11}
\end{equation*}


As a final observation we point out that


\begin{equation*}
\omega_{\operatorname{ess}}(T)=\omega_{\operatorname{ess}}(S) \tag{3.12}
\end{equation*}


whenever $T$ is generated by $A$ and $S$ is generated by $A+K$ for some compact operator K (see Prop.2.8 and Prop.2.9 of B-IV).

\section*{4. THE SPECTRUM OF INDUCED SEMIGROUPS}
In the previous section we tried to decompose a semigroup into the direct sum of two, hopefully simpler objects. Here we present other methods to reduce the complexity of a semigroup and its generator. Forming subspace or quotient semigroups as in $\mathrm{A}-1,3.2, \mathrm{~A}-\mathrm{I}, 3.3$ are such methods. But also the constructions of new semigroups on canonically associated spaces such as the dual space, see $A-I, 3.4$, or the F-product, see $\mathrm{A}-\mathrm{I}, 3.6$, might be helpful. We review these construc-\\
tions under the spectral theoretical point of view and collect a number of technical properties for later use.

We start by studying the spectrum of subspace and quotient semigroups. To that purpose assume that the strongly continuous semigroup $T=(T(t))_{t \geqq 0}$ leaves invariant some closed subspace $N$ of the Banach space $E$. There are canonically induced semigroups $T_{1}$ on $N$, resp. $T$, on ${ }^{E / N}$ and their generators ${ }^{A} /$, resp. A / are canonically obtained from the generator $A$ of $T$ (see $A-I$, section 3). The following example shows that the spectra of $\mathrm{A}, \mathrm{A} /$ and $\mathrm{A} /$ may differ quite drastically.

Example 4.1. As in the example in $A-I, 3.3$ we consider the translation semigroup on $E=L^{1}(\mathbb{R})$ and the invariant subspace $\mathrm{N}:=\{f \in \mathrm{E}: \mathrm{f}(\mathrm{x})=0$ for $\mathrm{x} \geqq 1\}$. Then $\sigma(\mathrm{A})=\mathrm{i} \mathbb{R}$ but $\sigma\left(A_{1}\right)=\{\lambda \in \mathbb{C}: \operatorname{Re}(\lambda) \leq 0\}$. Next we take the translation invariant subspace $M:=\{f \notin N: f(x)=0$ for $0 \leqq x \leqq 1\}$ and obtain $\sigma\left(A_{\mid},\right)=\emptyset$ for the generator $A_{\mid /}$of the quotient semigroup $T_{1 /}$ (use the fact that $T_{\mid /}$is nilpotent).

In the next proposition we collect the information on $\sigma(\mathrm{A})$ which in general can be obtained from the 'subspace spectrum' $\sigma\left(A_{1}\right)$ and the 'quotient spectrum' $\sigma\left(A_{j}\right)$.

Proposition 4.2. Using the standard notations the following inclusions hold:

$$
\rho+(A) \quad \subset\left[\rho\left(A_{1}\right) \cap \rho(A,)\right] \subset \rho(A) \subset[\rho(A \mid) \cap \rho(A,)] \cup[\sigma(A,) \cap \sigma(A,)]
$$

(iii) (ii) (i)\\
where $\rho_{+}(A)$ denotes the connected component of $\rho(A)$ which is unbounded to the right.

Proof. (i) Assume $\lambda \in \rho(A)$, i.e. $(\lambda-A)$ is a bijection from $D(A)$ onto $E$. Since $N$ is T-invariant we have $D\left(A_{1}\right)=D(A) \cap N$ and $(\lambda-A) D(A \mid) \subset N$. If $(\lambda-A) D\left(A_{\mid}\right)=N$ then $R(\lambda, A) N=D(A \mid)$ and the induced operators $\mathrm{R}(\lambda, \mathrm{A})$, resp. $\mathrm{R}(\lambda, \mathrm{A})$, are the inverses of $\left(\lambda-A_{1}\right)$, resp. $\left(\lambda-A_{1}\right)$. If $(\lambda-A) D\left(A_{\mid}\right) \neq N$ then $\lambda \in \sigma\left(A_{\mid}\right)$. In addition there exists $f \in D(A) \backslash N$ such that $g:=(\lambda-A) f \in N$. Hence for $\hat{f}:=f+N, \hat{g}:=g+N \in E$, it follows that $(\lambda-A,) \hat{f}=\hat{g}=0$, i.e. $\lambda \in \sigma(A).$,\\
(ii) Take $\lambda \in \rho\left(A_{\mid}\right) \cap \rho(A$,$) . Then (\lambda-A)$ is injective: $(\lambda-A) f=0$ implies $(\lambda-A,) \hat{f}=0$, hence $\hat{f}=0$, i.e. $f \in N$ and therefore $f=0$.

In addition, ( $\lambda-\mathrm{A})$ is surjective: For $g \in \mathrm{E}$ there exists $\hat{f} \in \mathrm{E}$, such that $(\lambda-A) E=$,$g , i.e. there exists h \in N$ such that $(\lambda-A) f-g=h=(\lambda-A) k$ for some $k \in D(A)$.\\
Therefore we obtain $(\lambda-A)(f-k)=9$.\\
(iii) The integral representation of the resolvent for $\lambda>\omega(T)$ (see A-I, Prop.1.11) shows that $R(\lambda, A) N \subset N$. By the power series expansion for holomorphic functions this extends to all $\lambda \in \rho_{+}(A)$. Therefore the restriction $R(\lambda, A)$ coincides with the resolvent $R\left(\lambda, A_{1}\right)$, On the other hand $R(\lambda, A)$ is well defined on $E$, and satisfies

$$
R(\lambda, A),(f+N)=R(\lambda, A) E+\mathbb{N}
$$

(use again the integral representation). This proves that

$$
\mathbb{R}(\lambda, A),=R(\lambda, A,)
$$

Corollary 4.3. Under the above assumptions take a point $\mu$ in the closure of $\rho_{+}(A)$. Then\\
(i) $\mu \in \sigma(\mathrm{A})$ if and only if $\mu \in \sigma(\mathrm{A} \mid)$ or $\mu \in \sigma(\mathrm{A}$, ).\\
(ii) $\mu$ is a pole of $R(\cdot, A)$ if and only if $\mu$ is a pole of $\mathrm{R}(\cdot, \mathrm{A}$, ) and of $\mathrm{R}(\cdot, \mathrm{A}$, ). In that case,\\
\includegraphics[max width=\textwidth, center]{2024_12_23_c6487cc0859199a15bd9g-086}\\
for the respective pole orders.

Proof. (i) follows from prop.4.2, inclusions (ii) and (iii). (ii) By the previous assertion we may assume that for some $\delta>0$ the pointed disc

$$
\{\lambda \in \mathbb{C}: 0<|\lambda-\mu|<\delta\}
$$

is contained in $\rho(A) \cap \rho\left(A_{\mid}\right) \cap \rho\left(A_{/}\right)$. Call $U_{n}$ the coefficients of the Laurent expansion of $R(\cdot, A)$. Since $N$ is $R(\lambda, A)$-invariant for $\lambda \in \rho_{+}(A)$ the same holds for each $U_{n}$. With the obvious notations we have

$$
R(\lambda, A)=\sum U_{n}(\lambda-\mu)^{n}, R(\lambda, A)\left|=\sum U_{n}\right|^{(\lambda-\mu)^{n}} \text { and } R(\lambda, A) /=\sum U_{n} /(\lambda-\mu)^{n}
$$

which shows $\max \left(k_{\mid}, k,\right) \leqq \mathrm{k}$. If $\mathrm{R}(\cdot, \mathrm{A}) \mid$ has a pole in $\mu$ of order $\ell$, then $\left.\mathrm{U}_{-(\ell+1)}\right|^{-}=0$, i.e. $U_{-(\ell+1)^{N}}=\{0\}$. similarly it follows that $U_{-}(m+1) E \subset \mathrm{E}$ if $R(., \mathrm{A})$, has a pole in $\mu$ of order $m$.

Therefore $U_{-(\ell+1)}{ }^{\circ} U_{-(m+1)}=0$. The relations (3.2) imply $\mathrm{U}_{-(m+1+1)}=0$, hence the pole order of $R(., A)$ is dominated by l + m .\\
4.4. Spectrum of the adjoint semigroup.

We recall from A-I,3.4 that to every strongly continuous semigroup $T=(T(t))_{t \geqq 0}$ there corresponds a strongly continuous adjoint semigroup $T^{*}=\left(T(t)^{*}\right){ }_{t \geqslant 0}$ on the semigroup dual

$$
E^{\star}=\left\{\phi \in E^{\prime}: \lim _{t \rightarrow \infty}\|T(t) ' \phi-\phi\|=0\right\} .
$$

Its generator $A^{*}$ is the maximal restriction of the adjoint $A^{\prime}$ to E* . For these operators the spectra coincide, or more precisely\\
(i) $\sigma(T(t))=\sigma(T(t) ')=\sigma(T(t) *)$,

$$
R_{\sigma}(T(t))=P_{\sigma}(T(t) ')=P_{\sigma}(T(t) *) .
$$


\begin{equation*}
\sigma\left(A^{\prime}\right)=\sigma\left(A^{\prime}\right)=\sigma\left(A^{*}\right), R_{\sigma}(A)=P_{\sigma}\left(A^{\prime}\right)=P_{\sigma}\left(A^{*}\right) . \tag{ii}
\end{equation*}



\begin{equation*}
s(A)=s\left(A^{*}\right), \omega(A)=\omega\left(A^{*}\right) . \tag{iii}
\end{equation*}


The left part of these equalities is either well known or has been stated in Prop.2.2(ii). The first statment of (iii) follows from (ii), while the second is an immediate consequence of the estimate $\|\mathrm{T}(\mathrm{t}) *\| \leqq$ $\|T(t)\| \leqq M \cdot\|T(t) *\|$ given in $A-I, 3.4$. As a sample for the remaining assertions we show that $0 \notin \sigma(A)$ if and only if 0$\} \sigma\left(A^{*}\right): I f A$ and therefore $A^{\prime}$ is invertible it follows from $A-I, 3.4$ that $A *$ is a bijection from D(A*) onto E* . Conversely assume that $\mathrm{A}^{*}$ is invertible. Then $A^{\prime}$ must be injective by the Proposition in A-I, 3.4. Moreover $A^{\prime}\left(D\left(A^{\prime}\right)\right)$ contains $A^{*}\left(D\left(A^{*}\right)\right)=E^{*}$ and is $\sigma\left(E^{\prime}, E\right)$-dense in $E^{\prime}$. By standard duality arguments follows that $A$ is injective with dense image. We show that $A(D(A))$ is closed: For $f \in D(A)$ choose $\phi \in D\left(A^{\prime}\right)$ such that $\|\phi\| \leqq 1$ and $|\langle f, \phi\rangle| \geqq \frac{1}{2}\|f\|$. Then

$$
\begin{aligned}
\left\|\left(A^{*}\right)^{-1}\right\|\|A f\| & \left.\geqq\left|\left(A^{*}\right)^{-1} \|\right|<A f, \phi\right\rangle\left|\geqq\left|<A f,\left(A^{*}\right)^{-1} \phi>\right|\right. \\
& =|\langle\mathrm{f}, \phi\rangle| \geqq \frac{1}{2}\|\mathrm{f}\|
\end{aligned}
$$

hence

$$
\|A f\| \geq \frac{1}{2}\left\|\left(\mathrm{~A}^{*}\right)^{-1}\right\|^{-1}\|\mathrm{f}\|,
$$

and $A(D(A))$ is closed since $A$ is closed.

\subsection*{4.5 Spectrum of the F-product semigroup.}
As stated in $A-I, 3.6$ the $F$-product semigroup $T_{F}=\left(T_{F}(t)\right)_{t \geq 0}$ on $E_{F}^{T}$ of a strongly continuous semigroup $T$ on $E$ serves to convert sequences in E into points in $\mathrm{E}_{F}^{T}$. In particular it can be used to\\
convert approximate eigenvectors of the generator A into eigenvectors of $A_{F}$.

Proposition. Let $A$ be the generator of a strongly continuous semigroup. Then the generator $A_{F}$ of the $F$-product semigroup satisfies\\
(i) $A \sigma(A)=A \sigma\left(A_{F}\right)=\operatorname{P\sigma }\left(A_{F}\right)$.\\
(ii) $\sigma(A)=\sigma\left(A_{F}\right)$.

Remark: In case $A$ is bounded then $A$ is a generator and $E_{F}^{T}=E_{F}$ (cf. A-I, 3.6). Thus the proposition applies to bounded linear operators and their canonical extensions to the $F$-product $E_{F}$.

Proof of the proposition. (i) The inclusion $\operatorname{Po}\left(A_{F}\right) \subset A \sigma\left(A_{F}\right)$ holds trivially. We show that $A \sigma\left(A_{F}\right) \subset A \sigma(A)$ : Take $\lambda \in A \sigma\left(A_{F}\right)$ and an associated approximate eigenvector $\left(\hat{f}^{m}\right)_{n \in \mathbb{N}}$, i.e. $\hat{f}^{m} \in D\left(A_{F}\right)$, $\left\|\hat{f}^{m}\right\|=1$ and $\left(\lambda-A_{F}\right) \hat{f}^{m} \rightarrow 0$ as $m \rightarrow \infty$. By the considerations in $A-I, 3.6$ we can represent each $\hat{f}^{m}$ as a normalized sequence $\left(f_{n}^{m}\right), N$ in $D(A)$ such that

$$
\lim _{m \rightarrow \infty} \lim \sup _{n \rightarrow \infty}\left\|(\lambda-A) f_{n}^{m}\right\|=0
$$

Therefore we can find a sequence $g_{k}=f_{k}^{m(k)}$ satisfying

$$
1 i m_{k \rightarrow \infty}\left\|(\lambda-A) g_{k}\right\|=0
$$

i.e. $\lambda \in A \sigma(A)$.

Finally we show $A \sigma(A) \subset \operatorname{Po}\left(A_{F}\right)$ : For $\lambda \in A \sigma(A)$ take a corresponding approximate eigenvector ( $f_{n}$ ). By $A-I,(3.2)$ we have

$$
\begin{aligned}
\left\|T(t) f_{n}-f_{n}\right\| & \leqq\left\|T(t) f_{n}-e^{\lambda t_{f}}\right\|+\left|e^{\lambda t}-1\right| \\
& =\left\|\int_{0}^{t} e^{\lambda(t-s)} T(s)(\lambda-A) f_{n} d s\right\|+\left|e^{\lambda t}-1\right|
\end{aligned}
$$

which converges to zero uniformly in $n$ as $t \rightarrow 0$, i.e. $\left(E_{n}\right) \in m^{T}(E)$. By the characterization of $D\left(A_{F}\right)$ given in $A-I, 3.6$ it follows that

$$
\hat{f}:=\left(f_{n}\right)+c_{F}(E) \in D\left(A_{F}\right),
$$

and $\hat{A}_{F} \hat{f}=\lambda \hat{f}$, i.e. $\lambda \in \operatorname{Po}\left(A_{F}\right)$.\\
(ii) The inclusion $\sigma(A) \subset \sigma\left(A_{F}\right)$ follows from (i) and the inclusion $R \sigma(A) \subset R \sigma\left(A_{F}\right): F o r \quad \lambda \in \operatorname{RO}(A)$ choose $f \in E$ such that $\|(\lambda-A) g-f\| \geq 1$ for every $g \in D(A)$. Then $\left\|\left(\lambda-A_{F}\right) \hat{g}-\hat{f}\right\| \geqq 1$ for every $\hat{g} \in D\left(A_{F}\right)$ and $\hat{E}=(E, f, \ldots)+c_{F}(E)$. Therefore $\lambda \in R \sigma\left(A_{F}\right)$. We now show $\rho(A) \subset \rho\left(A_{F}\right)$ : Assume $\lambda \in \rho(A)$, By (i) $\left(\lambda-A_{F}\right)$ has to be injective. Choose $\hat{f}={ }_{T}\left(f_{1}, f_{2}, \ldots\right)+c_{F}(E)$ such that $\left(f_{n}\right) \in m^{\top}(E)$. Then $\left(R(\lambda, A) f_{n}\right) \in m^{\top}(E)$ and $\left(\lambda-A_{F}\right)\left(\left(R(\lambda, A) f_{n}\right)+c_{F}(E)\right)$ $=\left(f_{n}\right)+c_{F}(E)$, i.e., $\left(\lambda-A_{F}\right)$ is surjective and $\lambda \in \rho\left(A_{F}\right)$.

Applying the proposition to a single operator $T(t)$ we obtain $\mathrm{A} \mathrm{\sigma}(\mathrm{~T}(\mathrm{t}))=\operatorname{P\sigma }\left(\mathrm{T}(\mathrm{t})_{F}\right)$. Note that in general $\mathrm{A} \mathrm{\sigma}(\mathrm{~T}(\mathrm{t})) \neq \operatorname{Po}\left(\mathrm{T}_{F}(t)\right)$ (see the Examples 1.3 and 1.4 in combination with Theorem 6.3).

\section*{5. THE SPECTRUM OF PERIODIC SEMIGROUPS}
In this section we determine the spectrum of a particularly simple class of strongly continuous semigroups and thereby achieve a rather complete description of the semigroup itself. Besides being nice and simple these semigroups gain their importance as building blocks for the general theory.

Definition 5.1. A strongly continuous semigroup $T=(T(t))_{t \geqq 0}$ on a Banach space $E$ is called periodic if $T\left(t_{0}\right)=I d$ for some $t_{0}>0$. The period $\tau$ of $\tau$ is obtained as $\tau:=\inf \left\{t_{0}>0: T\left(t_{0}\right)=I d\right\}$.

We immediately observe that periodic semigroups are groups with inverses $T(t)^{-1}=T(n t-t)$ for $0 \leqq t \leqq n T, \tau$ the period of $T$. Moreover, they are bounded, hence the growth bound is zero and $\sigma(\mathrm{A}) \subset \mathrm{i} \mathbb{R}$.

Lemma 5.2. Let $T$ be a strongly continuous semigroup with period $\tau>0$ and generator A . Then


\begin{equation*}
\sigma(A) \subset 2 \pi i / \tau \cdot \mathbb{Z}, \text { and } \tag{5.1}
\end{equation*}


for $\mu \$ 2 \pi i / \tau \cdot \mathbb{Z}$.

Proof. From the basic identities $A-I,(3.1)$ and $A-I,(3.2)$ for $t=\tau$, it follows that ( $\mu-\mathrm{A})$ has a left and right inverse if $\mu \neq 2 \pi \mathrm{in} / \tau$, $\mathrm{n} \in \mathbf{Z}$, and that the inverse is given by the above expression.

The representation of $\mathrm{R}(1, \mathrm{~A})$ given in $\mathrm{A}-\mathrm{I}, \mathrm{Prop} .1 .11$ shows that the resolvent of the generator of a periodic semigroup is a meromorphic function having only poles of order one and the residues

$$
P_{n}:=\lim _{\mu \rightarrow \mu_{n}}\left(\mu-\mu_{n}\right) R(\mu, A) \quad \text { in } \quad \mu_{n}:=2 \pi i n / \tau, n \in \mathbb{Z} \text {, are }
$$


\begin{equation*}
P_{n}=\tau^{-1} \int_{0}^{\tau} \exp \left(-\mu_{n} s\right) T(s) d s . \tag{5.2}
\end{equation*}


Moreover, it follows that the spectrum of $A$ consists of eigenvalues only and each $P_{n}$ is the spectral projection belonging to $\mu_{n}$ (see\\
3.6). Another way of looking at $P_{n}$ is given by saying that $P_{n}$ is the n-th Fourier coefficient of the r -periodic function $\mathrm{s} \rightarrow \mathrm{T}(\mathrm{s})$. From this it follows that no non-zero $\phi \in E^{\prime}$ vanishes on all $P_{n} E^{\prime}$ simultaneously. By the Hahn-Banach theorem we conclude that $\operatorname{spann} U_{n \in \mathbb{Z}} P_{n} E$ is dense in $E$.\\
since $P_{n} E \subset D(A)$ we obtain from $A-I,(3.1)$ that


\begin{equation*}
A P_{n} f=\mu_{n} P_{n} f \tag{5.3}
\end{equation*}


for every $f \in E, n \in \mathbb{Z}$. This and $A-I,(3.2)$ imply


\begin{equation*}
T(t) P_{n} \mathcal{E}=\exp \left(\mu_{n} t\right) \cdot P_{n} E \tag{5.4}
\end{equation*}


for every $t \geqq 0$. Therefore $\mu_{n}$ is an eigenvalue of $A$ and $\exp \left(\mu_{n} t\right)$ is an eigenvalue of $T(t)$ if and only if $\mathrm{P}_{n} \neq 0$. In that case, $P_{n} E$ is the corresponding eigenspace and we have the following lemma.

Lemma 5.3. For a $\tau$-periodic semigroup $T$ we take $\mu_{n}:=2 \pi i n / \tau$, $n \in \mathbb{Z}$ and consider

$$
P_{n}:=\tau^{-1} \cdot \int_{0}^{\tau} \exp \left(-\mu_{n} s\right) T(s) d s .
$$

Then the following assertions are equivalent:\\
(a) $P_{n} \neq 0$\\
(b) $\mu_{n} \in P_{\sigma}(A)$\\
(c) $\exp \left(\mu_{n} t\right) \in P_{o}(T(t))$ for every $t>0$.

The action of $A$, resp. $T(t)$ on the subspaces $P_{n} E, n \in \mathbb{Z}$, is determined by (5.3), resp. (5.4). Moreover,

$$
\begin{aligned}
P_{m} P_{n} f & =\tau^{-1} \cdot \int_{0}^{\tau} \exp \left(-\mu_{m} s\right) T(s) P_{n} f d s= \\
& =\tau^{-1} \cdot \int_{0}^{\tau} \exp \left(\left(\mu_{n}-\mu_{m}\right) s\right) P_{n} f d s=0
\end{aligned}
$$

for $n \neq m$, i.e. the subspaces $P_{n} E$ are "orthogonal". Since their union is total in E one expects to be able to extend the representations (5.3) and (5.4) of $A$ and $T(t)$. This is possible if

$$
\sum_{-\infty}^{+\infty} P_{n}=I d,
$$

where the series should be summable for the strong operator topology. Unfortunately this is false in general since the family of projections

$$
Q_{H}:=\sum_{n \in H} P_{n}
$$

where $H$ runs through all finite subsets of $\mathbf{Z}$, may be unbounded (see the example below) . Nevertheless the following is true.

Theorem 5.4. Let $T=(T(t))_{t \geq 0}$ be a $\tau$-periodic semigroup on a Banach space E with generator A and associated spectral projections\\
$\mathrm{P}_{\mathrm{n}}:=\mathrm{T}^{-1} \cdot \int_{0}^{\tau} \exp \left(-\mu_{n} \mathrm{~s}\right) \mathrm{T}(\mathrm{s}) \mathrm{ds}, \mu_{\mathrm{n}}:=2 \pi i n / \tau, \mathrm{n} \in \mathbb{Z}$.\\
For every $f \in D(A)$ one has $f=\sum_{-\infty}^{+\infty} P_{n} f$ and therefore\\
(i)

$$
\begin{array}{rlrl}
T(t) f & =\sum_{-\infty}^{+\infty} \exp \left(\mu_{n} t\right) P_{n} f & \text { if } f \in D(A), \\
A f & =\sum_{-\infty}^{+\infty} \mu_{n} p_{n} f & & \text { if } \pounds \in D\left(A^{2}\right) .
\end{array}
$$

Proof. It suffices to prove the first statement. Then (i) and (ii) follow by $(5.3)$ and (5.4).\\
We assume $\tau=2 \pi$ and show first that $\sum_{-\infty}^{+\infty} P_{n} f$ is summable for $f \in D(A)$ : For $g:=A f$ we obtain $P_{n} g=P_{n} A f=A P_{n} f=i n P_{n} f$. Take $H$ to be a finite subset of $\mathbb{Z} \backslash\{0\}$ and $\phi \in E \cdot$. Then

$$
\begin{aligned}
\left|\Sigma_{n \in H}\left\langle p_{n} f, \phi\right\rangle\right| & =\left|\sum_{n \in H}(i n)^{-1}\left\langle p_{n} g, \phi\right\rangle\right| \\
& \leqq\left(\sum_{n \in H} n^{-2}\right)^{1 / 2}\left(\sum_{n \in H}\left|<p_{n} g, \phi>\right|^{2}\right)^{1 / 2}
\end{aligned}
$$

From Bessel's inequality we obtain for the second factor

$$
\begin{aligned}
\Sigma_{n \in H}\left|<p_{n} g, \phi>\right|^{2} & \leqq 1 / 2 \pi \cdot \int_{0}^{2 \pi}|<T(s) g, \phi>|^{2} \mathrm{ds} \\
& \leqq\|\phi\|^{2} \cdot 1 / 2 \pi \cdot \int_{0}^{2 \pi}\|T(s) g\|^{2} \mathrm{ds} .
\end{aligned}
$$

With the constant $c:=\left(1 / 2 \pi \cdot \int_{0}^{2 \pi}\|T(s) g\|^{2} \mathrm{ds}\right)^{1 / 2}$ we obtain

$$
\left\|\sum_{n \in H} P_{n} f\right\| \leqq c\left(\sum_{n \in H} n^{-2}\right)^{1 / 2}
$$

for every finite subset $H$ of $\mathbb{Z}$, i.e. $\sum_{-\infty}^{+\infty} P_{n} f$ is sumable.\\
Next we set $h:=\sum_{-\infty}^{+\infty} p_{n} f$ and observe that for every $\phi^{\prime} \in E^{\prime}$ the Fourier coefficients of the continuous, t-periodic functions

$$
s \rightarrow\langle T(s) h, \phi\rangle \quad \text { and } \quad s \rightarrow\langle T(s) f, \phi\rangle
$$

coincide. Therefore these functions are identical for $s \geqq 0$ and in particular for $s=0$, i.e. $\langle h, \phi\rangle=\langle\mathrm{f}, \phi\rangle$. By the Hahn-Banach Theorem we obtain $\mathrm{f}=\mathrm{h}$.

The above theorem contains rather precise information on periodic semigroups. In particular, it characterizes periodic semigroups by the fact that $\sigma(A)$ is contained in $i \alpha \mathbb{Z}$ for some $\alpha \in \mathbb{R}$ and the eigenfunctions of A form a total subset of E.

If we suppose in addition that a periodic semigroup has a bounded generator it follows that the spectrum of its generator is bounded.

Therefore only a finite number of spectral projections $P_{n}$ are distinct from 0 and we have the following characterization.

Corollary 5.5. Let $T=(T(t))_{t \geqq 0}$ be a semigroup with bounded generator on some Banach space E . This semigroup has period $\tau / k$ for some $k \in \mathbb{N}$ if and only if there exist finitely many pairwise orthogonal projections $P_{n},-m \leq n \leq m, P_{-m} \neq 0$ or $P_{m} \neq 0$, such that\\
(i) $\quad \sum_{-m}^{+m} p_{n}=I d$,\\
(ii) $\quad T(t)=\Sigma_{-m}^{+m} \exp (2 \pi i n t / \tau) P_{n}$,\\
(iii) $A=\Sigma_{-m}^{+m}(2 \pi i n / \tau) P_{n}$.

Example 5.6. From A-I,2.5 we recall briefly the rotation group $R_{\tau}(t) f(z):=f(\exp (2 \pi i n t / \tau) \cdot z) \quad$ on $E=C(\Gamma), \quad r e s p . \quad E=L^{p}(\Gamma, m)$ for $1 \leqq p<\infty$. The spectrum of the generator

$$
\begin{aligned}
A f(z) & =(2 \pi i / \tau) z \cdot f^{\prime}(z) \\
\sigma(A) & =(2 \pi i / \tau) \cdot \mathbb{Z} .
\end{aligned}
$$

The eigenfunctions $\varepsilon_{n}(z):=z^{n}$ yield the projections

$$
\begin{aligned}
P_{n} & =(1 / 2 \pi i) \cdot \varepsilon-(n+1) \varepsilon_{n}, i \cdot e \\
P_{n} f(z) & =(1 / 2 \pi i) \cdot\left(\int_{\Gamma} f(w) w^{-(n+1)} d w\right) \cdot z^{n} .
\end{aligned}
$$

It is left as an exercise to compute the norms of $Q_{m}:=\sum_{-m}^{+m} P_{n}$ in $\mathrm{L}^{\mathrm{P}}(\mathrm{F})$ for various p and then check the assertions of Theorem 5.4. Clearly, this proves some classical convergence theorems for Fourier series (compare Davies (1980), Chap.8.1).

\section*{6. SPECTRAL MAPPING THEOREMS}
We now return to the question posed in the introduction to this chapter: In which form and under which conditions is it true that the spectrum $\sigma(T(t))$ of the semigroup operators is obtained - via the exponential map - from the spectrum $\sigma(\mathrm{A})$ of the generator, or briefly

$$
\sigma(T(t))=\exp \left(t_{0}(A)\right) ?
$$

This and similar statements will be called spectral mapping theorems for the semigroup $T=(T(t))_{t \geq 0}$ and its generator $A$.

In addition, we saw in Prop.1.1 that the validity of such a spectral mapping theorem implies

$$
s(A)=\omega(A)
$$

for the spectral- and growth bounds and therefore guarantees that the location of the spectrum of A determines the asymptotic behavior of $T$. As we have seen in Examples 1.3 and 1.4 the last statement does not hold in general. We therefore present a detailed analysis, where and why it fails and what additional assumptions are needed for its validity. Before doing so we have another look at the examples.

\subsection*{6.1 The counterexamples revisited.}
(i) Take the nilpotent translation semigroup from $A-I, 2,6$. Then $\sigma(A)=\emptyset$ and $\sigma(T(t))=0$ for every $t>0$. By this trivial example and since $e^{z} \neq 0$ for every $z \in \mathbb{C}$, it is natural to read the 'spectral mapping theorem' modulo the addition of $\{0\}$, i.e.

$$
\sigma(T(t)) \cup\{0\}=\exp (t \sigma(A)) \cup\{0\} \text { for } t \geqq 0
$$

(ii) The spectrum of the generator $A$ of the $\tau$-periodic rotation group $\left(R_{\tau}(t)\right)_{t \geqq 0}$ on $C(\Gamma)$ is $\sigma(A)=2 \pi i / \tau \cdot \mathbb{Z}$ and $\exp (2 \pi i n t / \tau), n \in \mathbb{Z}$, is an eigenvalue of $R_{\tau}(t)$ for every $t \geqq 0$ (see Example 5.6). If $t / \tau$ is irrational these eigenvalues form a dense subset of $\Gamma$. Since the spectrum is closed we obtain $\sigma(T(t))=\Gamma$ for these $t$. Therefore in this example the spectral mapping theorem is valid only in the following 'weak' form:

$$
\sigma(T(t))=\overline{\exp (t \sigma(A))}, \quad t \geqq 0 .
$$

(iii) By Example 1.3 there exists a semigroup $T=(T(t))_{t \geq 0}$ with generator $A$ such that $s(A)=-1$ and $\omega(T)=0$. This implies that for preassigned real numbers $\alpha<B$ there exists a semigroup $S=(S(t))_{t \geq 0}$ with generator $B$ such that $s(B)=\alpha$ and $\omega(S)=\beta$ : Take $S(t):=e^{B t_{T}((B-\alpha) t)}$ and observe that $B=(B-\alpha) A+B I d$. In that case $\exp (\operatorname{to}(B))$ is contained in the circle about 0 with radius $e^{a t}$ by Lemma 1.1; hence there must be points in $\sigma(S(t))$ which are not in the closure of $\exp (\operatorname{to}(B))$.\\
(iv) The Example 1.3 can be strengthend in order to yield a semigroup $T=(T(t))_{t \geqq 0}$ with generator $A$ such that $\sigma(A)=\varnothing$ but $\|T(t)\|=r(T(t))=1$ for $t \geq 0$, i.e. $s(A)=-\infty$, $\omega=0$ and $s(A)<\omega$ :

Take the translation semigroup on the Banach space $E:=C_{0}\left(\mathbb{R}_{+}\right) \cap L^{1}\left(\mathbb{R}_{+}, e^{\mathrm{X}^{2}} \mathrm{dx}\right)$ with $\|f\|:=\sup \left\{|f(x)|: x \in \mathbb{R}_{+}\right\}+\int_{0}^{\infty}|f(x)| e^{x^{2}} d x$ (see Greiner-Voigt-Wolff (1981)).\\
(v) Another modification of Example 1.3 yields a group $T=(T(t))_{t \in \mathbb{R}}$ satisfying $s(A)<\omega$. Therefore the spectral mapping theorem does not hold (see Wolff (1981)).

The next few theorems form the core of this chapter. We show that only one part of the spectrum and one inclusion is responsible for the failure of the spectral mapping theorem. The usefulness of this detailed analysis will become clear in the subsequent chapter on stability and asymptotics.\\
6.2. Spectral Inclusion Theorem. Let $A$ be the generator of a strongly continuous semigroup $T=(T(t))_{t \geq 0}$ on some Banach space $E$. Then

$$
\exp (t \sigma(A)) \subset \sigma(T(t)) \text { for } t \geqq 0
$$

More precisely we have the following inclusions:

\begin{center}
\begin{tabular}{lll}
(6.1) & $\exp (t \cdot \operatorname{Po}(A))$ & $\subset \operatorname{P\sigma }(T(t))$, \\
$(6.2)$ & $\exp (t \cdot A \sigma(A))$ & $\subset \quad A \sigma(T(t))$, \\
$(6.3)$ & $\exp (t \cdot \operatorname{R} \sigma(A))$ & $\subset \quad \operatorname{R} \sigma(T(t))$. \\
\end{tabular}
\end{center}

Proof. Since $e^{\lambda t}-T(t)=(\lambda-A) \int_{0}^{t} e^{\lambda(t-s)} T(s)$ ds (see $A-I,(3.1)$ ) it follows that $\left(e^{\lambda t}-T(t)\right)$ is not bijective if $(\lambda-A)$ fails to be bijective, which proves the main inclusion.\\
The inclusion (6.1) becomes evident from the following proof of (6.2): Take $\lambda \in A \sigma(A)$ and an associated approximate eigenvector $\left(f_{n}\right) \subset D(A)$. Then

$$
g_{n}:=e^{\lambda t_{f}} f_{n}-T(t) f_{n}=\int_{0}^{t} e^{\lambda(t-s)} T(s)(\lambda-A) f_{n} d s
$$

converges to zero as $n \rightarrow \infty$. Consequently, $e^{\lambda t} \epsilon A \sigma(T(t))$ and in fact, the same approximate eigenvector ( $\mathrm{f}_{n}$ ) does the job for all $t \geq 0$.\\
For the proof of (6.3) we take $\lambda \in R \sigma(A)$ and observe that\\
$\left(e^{\lambda t}-T(t)\right) f=(\lambda-A)\left(\int_{0}^{t} e^{\lambda(t-s)} T(s) f d s\right) \epsilon(\lambda-A) D(A)$\\
for every $f \in \mathrm{E}$.

As we know from the Examples 6.I, the converse inclusions do not hold in general, i.e., not every spectral value of a semigroup operator $T(t)$ comes - via the exponential map - from a spectral value of the generator. But at least this is true for some important parts of the spectrum.\\
6.3 Spectral Mapping Theorem for Point and Residual Spectrum. Let A be the generator of a strongly continuous semigroup $T=(T(t))_{t \geq 0}$. Then\\
(6.4) $\exp (t \cdot \operatorname{Po}(A))=\operatorname{P\sigma }(T(t)) \backslash\{0\}$,\\
(6.5) $\exp (t \cdot \operatorname{Ro}(A))=\operatorname{Ro}(T(t)) \backslash\{0\}$, for $t \geqq 0$.

Proof. For the proof of (6.4) take $t_{0}>0$ and $0 \neq \lambda \in \operatorname{Po}\left(T\left(t_{0}\right)\right.$ ). After rescaling the semigroup $T=(T(t))_{t \geqq 0}$ to the semigroup $\left(\exp \left(-t \cdot \log \lambda / t_{0}\right) T(t)\right)_{t \geq 0}$ we may assume $\lambda=1$. Then the closed, T-invariant subspace

$$
G:=\left\{g \in E: T\left(t_{0}\right) g=g\right\}
$$

is non trivial. The restricted semigroup $T_{\mid}$is periodic with period $\tau \leqq t_{0}$ and the spectrum of its generator $A_{\mid}$contains at least one eigenvalue $\mu=2 \pi i n / t_{0}$ for some $n \in \mathbf{Z}$ (see Lemma 5.3). Since every eigenvalue of $A_{\mid}$is an eigenvalue of $A$ we obtain that $1 \in \exp \left(t_{0} \cdot P \sigma(A)\right)$. The converse inclusion has been proved in (6.1). In fact, even more can be said: Let $g \in G$ be an eigenvector of $T\left(t_{0}\right)$ corresponding to the eigenvalue $\lambda=1$. For each $n \in \mathbb{Z}$ define

$$
g_{n}:=P_{n} g=1 / t_{0} \cdot \int_{0}^{t_{0}} \exp \left(-2 \pi i n s / t_{0}\right) T(s) g d s \in G
$$

as in section 5. Then $g_{n}$ is an eigenvector of $A_{1}$, hence of $A$ with eigenvalue $2 \pi i n / t_{0}$ as soon as $g_{n}$ is distinct from zero. Since $D(A \mid)$ is dense in $G$ it follows from theorem 5.4 that this holds for at least one $n \in \mathbb{Z}$. From the proof of (6.1) we know that this $g_{n}$ is in fact an eigenvector for each $T(t), t \geqq 0$. Since $\operatorname{Ro}(A)=\operatorname{Po}\left(A^{*}\right)$ and $\operatorname{Ro}(T(t))=\operatorname{Po}(T(t) *)$ (see 4.4) the assertion (6.5) follows from (6.4).

Note that the proof is essentially an application of the structure theorem for periodic semigroups as given in Thm.5.4. The information gained there can be reformulated into statements on the eigenspaces of $A$ and $T(t)$.

Corollary 6.4. For the eigenspaces of the generator A, resp. of the semigroup operators $\mathrm{T}(\mathrm{t}), t>0$, the following holds:\\
(ii) $\quad \operatorname{ker}\left(e^{\mu t}-T(t)\right)=\overline{\operatorname{span}}_{n \in Z} \operatorname{ker}(\mu+2 \pi i n / t-\bar{A}), \mu \in \mathbb{C}$.

Remark that analogous statements are valid for ker(p- A') and ker( $\left.e^{\mu t}-T(t)^{\prime}\right)$ if we take in (ii) the $\sigma\left(E^{\prime}, E\right)$-closure.

Without proof (see Greiner (1981), Prop.1.10) we add another corollary showing that poles of the resolvent of $T(t)$ correspond necessarily to poles of the resolvent of the generator. Again the converse is not true as shown by Example 5.6 .

Corollary 6.5. Assume that $e^{\mu t}$ is a pole of order $k$ of $R(\cdot, T(t))$ with residue $P$ and $Q$ as the k-th coefficient of the Laurent series. Then\\
(i) $\mu+2 \pi i n / t$ is a pole of $R(\cdot, A)$ of order $\leqq k$ for every $n \in \boldsymbol{Z}$,\\
(ii) the residues $\mathrm{P}_{\mathrm{n}}$ in $\mu+2 \pi i n / t$ yield $\mathrm{PE}=\overline{\operatorname{span}}{ }_{n \in \mathbf{Z}} \mathrm{P}_{\mathrm{n}} \mathrm{E}$,\\
(iii) the k-th coefficient of the Laurent series of $R(\cdot, A)$ at\\
$\mu+2 \pi i n / t$ is

$$
Q_{n}=\left(t \cdot e^{\mu t}\right)^{1-k} \cdot Q \cdot(1 / t) \int_{0}^{t} e^{-(\mu+2 \pi i n / t) s_{T}(s)} d s .
$$

From Theorem 6.2 and 6.3 it follows that the approximate point spectrum is the trouble maker in the sense that not every approximate eigenvalue of $T(t)$ corresponds to an approximate eigenvalue of the generator A . Since nothing more can be said in general we now look for additional hypotheses on the semigroup implying the spectral mapping theorem.\\
As a simple example we assume $T\left(t_{0}\right)$ to be compact for some $t_{0}>0$. Then $\sigma(T(t)) \backslash\{0\}=P_{\sigma}(T(t)) \backslash\{0\}$ for $t \geqq t_{0}$ and the spectral mapping theorem is valid by (6.4). A different class of semigroups verifying the spectral mapping theorem is given by the uniformly continuous semigroups (compare Cor.1.2).\\
Both cases, and many more, are included in the following result.\\
6.6 Spectral Mapping Theorem for Eventually Norm Continuous Semigroups.\\
Let $T=\left(T(t) t_{t \geq 0}\right.$ be an eventually norm continuous semigroup with generator A . Then the spectral mapping theorem is valid, i.e.\\
(6.6) $\quad \sigma(T(t)) \backslash\{0\}=e^{t \cdot \sigma(A)} \quad$ for every $t \geqq 0$.

Proof. By the previous considerations it suffices to show that $\overline{A \sigma(T(t)}) \backslash\{0\} c e^{t \cdot \sigma(A)}$ for $t>0$. This will be done by converting approximate eigenvectors into eigenvectors in the semigroup F-product (see 4.5). The assertion then follows from (6.4) and assertion (ii) of the Proposition in 4.5.\\
Assume $t \rightarrow T(t)$ to be norm continuous for $t \geqq t_{0}$. Moreover it suffices to consider $1 \in \operatorname{A\sigma }\left(\mathrm{~T}\left(\mathrm{t}_{1}\right)\right)$ for some $t_{1}>0$, i,e. we have a normalized sequence $\left(f_{n}\right)_{n \in N} \subset E$ such that

$$
\lim _{n \rightarrow \infty}\left\|T\left(t_{1}\right) f_{n}-f_{n}\right\|=0
$$

Choose $k \in \mathbb{N}$ such that $k t_{1}>t_{0}$ and define $g_{n}:=T\left(k t_{1}\right) f_{n}$. Then

$$
\lim _{n \rightarrow \infty}\left\|g_{n}\right\|=\lim _{n \rightarrow \infty}\left\|T\left(t_{1}\right) k_{f_{n}}\right\|=\lim _{n \rightarrow \infty}\left\|f_{n}\right\|=1
$$

and

$$
\lim _{n \rightarrow \infty}\left\|T\left(t_{1}\right) g_{n}-g_{n}\right\|=0,
$$

i.e. $\left(g_{n}\right)_{n \in \mathbb{N}}$ yields an approximate eigenvector of $T\left(t_{1}\right)$ with approximate eigenvalue 1 . But the semigroup $T$ is uniformly continuous on sets of the form $T\left(t_{0}\right) \mathrm{V}, \mathrm{V}$ bounded in E . In particular, it is uniformly continuous on the sequence $\left(g_{n}\right){ }_{n \in N_{T}}$, which therefore defines an element $g$ in the semigroup $F$-product $E_{F} \tau_{F}$.\\
Obviously, $\hat{g}$ is an eigenvector of $\mathbb{T}_{F}\left(t_{1}\right)$ with eigenvalue 1 and by (6.4) we obtain an eigenvalue $2 \pi i n / t_{1}$ of $A_{F}$ for some $n \in \mathbb{Z}$. The coincidence of $\sigma(A)$ and $\sigma\left(A_{F}\right)$ proves the assertion.

We point out that the above spectral mapping theorem implies the coincidence of spectral bound and growth bound for eventually norm continuous semigroups, hence we have generalized the liapunov Stability Theorem 1.2 to a much larger class of semigroups. As mentioned before this will be of great use in many applications. Therefore we state explicitly the spectral mapping theorem for several important classes of semigroups all of which are eventually norm continuous ( $c f$. the diagram preceding A-İI,Ex.1.27).

Corollary 6.7. The spectral mapping theorem


\begin{equation*}
\sigma(T(t)) \backslash\{0\}=e^{t \cdot \sigma(A)}, \quad t \geq 0 \tag{6.6}
\end{equation*}


holds for each of the following classes of strongly continuous semigroups:\\
(i) eventually compact semigroups,\\
(ii) eventually differentiable semigroups,\\
(iii) holomorphic semigroups,\\
(iv) uniformly continuous semigroups.

Another application of the above spectral mapping theorem can be made to tensor product semigroups (see $A-I, 3.7)$. Let $T_{1}=\left(T_{1}(t)\right)_{t \geq 0}$, $T_{2}=\left(T_{2}(t)\right)_{t \geqq 0}$ be strongly continuous semigroups on Banach spaces $E_{1}, E_{2}$ with generator $A_{1}, A_{2}$. The tensor product semigroup $T=T_{1} \otimes T_{2}$ on some (appropriate) tensor product $\mathrm{E}:=\mathrm{E}_{1} \tilde{\mathcal{O}}_{2}$ has the generator $\mathrm{A}=\mathrm{A}_{1} \otimes \mathrm{Id}+\mathrm{Id} \mathrm{A}_{2}$, but in general the spectrum of A is not determined by the spectra of $A_{1}, A_{2}$. But with an additional hypothesis the following can be proved.

Corollary 6.8. If $T_{1}$ and $T_{2}$ are eventually norm continuous then

$$
\sigma(A)=\sigma\left(A_{1}\right)+\sigma\left(A_{2}\right)
$$

where $A$ is the generator of the tensor product semigroup

$$
T_{1} \otimes T_{2}=\left(T_{1}(t) \otimes T_{2}(t)\right)_{t \geq 0}
$$

Proof. Clearly, the tensor product semigroup is eventually norm continuous and hence the spectral mapping theorem 6.6 is valid for all three semigroups $T_{1}, T_{2}$ and $T$. Moreover the spectrum of the tensor product of bounded operators is the product of the spectra [Reed-Simon (1978), XIII.9]. Therefore

$$
\sigma\left(\mathrm{T}_{1}(t) \otimes \mathrm{T}_{2}(t)\right)=\sigma\left(\mathrm{T}_{1}(t)\right) \cdot \sigma\left(\mathrm{T}_{2}(t)\right), t \geqq 0 .
$$

Consequently we have the following identity for every $t \geq 0$ :

$$
\begin{aligned}
e^{t \cdot \sigma(A)} & =\sigma\left(\mathrm{T}_{1}(t) \Leftrightarrow \mathrm{T}_{2}(t)\right) \backslash\{0\} \\
& =\sigma\left(\mathrm{T}_{1}(t)\right) \cdot \sigma\left(\mathrm{T}_{2}(t)\right) \backslash\{0\} \\
& =e^{t \cdot \sigma\left(A_{1}\right) \cdot e^{t \cdot \sigma\left(A_{2}\right)}} \\
& =e^{t\left(\sigma\left(A_{1}\right)+\sigma\left(A_{2}\right)\right)}
\end{aligned}
$$

From this identity we want to deduce $\sigma(\mathrm{A})=\sigma\left(\mathrm{A}_{1}\right)+\sigma\left(\mathrm{A}_{2}\right)$.\\
$" \subset$ ": Take $\xi \in \sigma(A)$. Then for every $t>0$ there exist $\mu_{t} \in \sigma\left(A_{1}\right)$, $\lambda_{t} \in \sigma\left(A_{2}\right)$ and $n_{t} \in \mathbb{Z}$ such that $\xi=\mu_{t}+\lambda_{t}+2 \pi i n_{t} / t$. Since the real parts of $\mu_{t}, \lambda_{t}$ are bounded above, they lie in some interval [a,b] . But

$$
\sigma\left(A_{i}\right) \cap([a, b]+i R)
$$

is compact for $i=1,2$, since $A_{i}$ is the generator of an eventually norm continuous semigroup (see A-II, Thm.1.20). By taking $t$\\
sufficiently small we conclude that $n_{t}=0$ for some $t^{\prime}>0$, i.e. $\xi=\mu_{t}+\lambda_{t}$,\\
$" \supset "$ Choose $\mu \in \sigma\left(A_{1}\right), \lambda \in \sigma\left(A_{2}\right)$. For every $t>0$ there exist $n_{t} \in \sigma(A), m_{t} \in \mathbf{Z}$ such that $\mu+\lambda=n_{t}+2 \pi i m_{t} / t$.\\
since $\operatorname{Re} \mu+\operatorname{Re} \lambda=\operatorname{Re} \eta_{t}$ and $\left\{\operatorname{Im} \eta_{t}: t>0\right\}$ is bounded $-T=\left(T_{1}(t) \otimes T_{2}(t)\right)_{t \geqq 0}$ is eventually norm continuous - it follows that $m_{t},=0$ for some $t^{\prime}>0$.

\section*{7. WEAK SPECTRAL MAPPING THEOREMS}
In the previous section we showed under which hypotheses a spectral mapping theorem of the form


\begin{equation*}
\sigma(T(t)) \backslash\{0\}=e^{t \cdot \sigma(A)}, t \geqq 0 \tag{7.1}
\end{equation*}


is valid for the generator $A$ of a strongly continuous semigroup $(T(t))_{t \geqq 0}$.\\
Among the various examples showing that (7.1) does not hold in general we recall the following.\\
Take the Banach space $E=c_{0}$, the multiplication operator $A\left(x_{n}{ }_{n} \in \mathbb{N}\right.$ $=\left(i n x_{n}\right)_{n \in \mathbb{N}}$ with maximal domain and the corresponding semigroup $T(t)\left(x_{n}\right)_{n \in \mathbb{N}}=\left(e^{i n t} x_{n}{ }_{n \in \mathbb{N}}\right.$. Then $\sigma(A)=\{$ in $: n \in \mathbb{N}\}$ and the spectral mapping theorem is valid only in the following weak form:


\begin{equation*}
\sigma(T(t))=\overline{\exp (t \cdot \sigma(A))}, t \geqq 0 . \tag{7,2}
\end{equation*}


In this section we prove similar weak spectral mapping theorems. We start with a generalization of the above example.\\
Consider the Banach space $E=C_{0}\left(X, \mathbb{C}^{n}\right)$ of all continuous $\mathbb{C}^{n}$-valued functions vanishing at infinity on some locally compact space X . In analogy to $A-I, 2.3$ we associate to every continuous function $q: X \rightarrow M(n)$, where $M(n)$ denotes the space of all complex $n \times n-$ matrices, a "multiplication operator"

$$
M_{q}: f \rightarrow q \cdot f \text { such that }(q \cdot f)(x)=q(x) \cdot f(x), x \in X
$$

on the maximal domain $D\left(M_{q}\right)=\{f \in E: q \cdot f \in E\}$. If $\left\|e^{t q(x)}\right\|$ is uniformly bounded for $0 \leqslant t \leqslant 1$ and $x \in x$ it follows that $M_{q}$ generates the multiplication semigroup

$$
(T(t) f)(x)=e^{t q(x)}(f(x)), f \in E, x \in X, t \geqq 0
$$

Since $M_{q}$ has a bounded inverse if and only if $q(x)^{-1}$ exists and is uniformly bounded for $x \in x$ it follows that the eigenvalues of each matrix $q(x)$ are always contained in $\sigma\left(M_{q}\right)$. In fact, much more can be sald in case the function is bounded.

Lemma 7.1. The spectrum of the matrix valued multiplication operator $M_{p}$ where $p: X \rightarrow M(n)$ is bounded is given by $\sigma\left(M_{p}\right)=\overline{U_{x \in X} \sigma(p(x))}$. Proof. It remains to show that $0 \notin \overline{U_{x \in X} \sigma(p(x))}$ implies $0 \& \sigma\left(M_{p}\right)$. since det $\mathrm{p}(\mathrm{x})$ is the product of n eigenvalues (according to their multiplicity) of $p(x)$ the hypothesis implies that\\
$\mathrm{d}:=\inf \{|\operatorname{det} \mathrm{p}(\mathrm{x})|: \mathrm{x} \in \mathrm{x}\}>0$. By Formula 4.12 in Chapter I of Kato (1966) we obtain

$$
\left\|p(x)^{-1}\right\| \leq \gamma \cdot\|p(x)\|^{n-1} \cdot|\operatorname{det} p(x)|^{-1} \leqq y / \alpha \cdot\left\|m_{p}\right\|^{n-1}
$$

for every $x \in X$ and a constant $\gamma$ depending only on the norm chosen on $\mathbb{C}^{n}$. Therefore $x \rightarrow p(x)^{-1}$ defines a bounded continuous function on $x$ which obviously yields the inverse of $M_{p}$, i.e., $0 \notin \sigma\left(M_{p}\right)$.

Theorem 7.2. Let $A=M_{q}$ be a matrix multiplication operator on $c_{0}\left(X, C^{n}\right)$ generating a strongly continuous semigroup $(T(t))$, $\left(e^{t q(\cdot)}\right)_{t \geq 0}$. Then the Weak Spectral Mapping Theorem of the form


\begin{equation*}
\sigma(T(t))=\overline{\exp (t \cdot \sigma(A)}) \tag{7.2}
\end{equation*}


is valid.

Proof. By the spectral Inclusion Theorem 6.2 we always have $\exp \left(t_{\sigma}(A)\right) \quad C \sigma(T(t))$. Since $T(t)$ is a matrix multiplication operator with a bounded function we obtain from Lemma 7.1\\
$\sigma(T(t))=\overline{U_{x \in X} \sigma(\exp (\operatorname{tg}(x)))}=\overline{U_{x \in X} \exp (t \sigma(q(x)))} \subset \overline{\exp \left(t_{\sigma}(A)\right)}$,\\
which proves the assertion.

Corollary 7.3. The growth bound $w(A)$ and the spectral bound $s(A)$ coincide for matrix multiplication semigroups.

Remark. The above results remain valid for other Banach spaces of $\mathbb{c}^{\mathrm{n}}$-valued functions such as $\mathrm{L}^{\mathrm{p}}\left(\mathrm{X}, \mathrm{c}^{\mathrm{n}}\right), 1 \leqq \mathrm{p}<\infty$.

The example given at the beginning of this section can be generalized in a different way. In fact, $A\left(x_{n}\right):=\left(i n x_{n}\right)$ on $E=c_{0}$ generates a bounded group, and we will show that this property too ensures that the Weak Spectral Mapping Theorem (7.2) holds. Without any boundedness assumption on $\left(T(t){ }_{t \in \mathbb{R}}\right.$ this result cannot be true (see [HillePhillips (1957), Sec.23.16] or [Wolff(1981)]).

Theorem 7.4. Let $T=(T(t))_{t \in \mathbb{R}}$ be a strongly continuous group on some Banach space $E$ such that $\|T(t)\| p(t)$ for some polynomial $p$ and all $t \in \mathbb{R}$. Then the Weak Spectral Mapping Theorem holds, i.e.,\\
(7.2) $\quad \sigma(T(t))=\overline{\exp (t \cdot \sigma(A))}$ for all $t \in \mathbb{R}$.

From the proof we isolate a series of lemmas for which we always assume the hypothesis made in Thm.7.4. Moreover we recall from Fourier analysis that the Fourier transformation $\phi \rightarrow \vec{\phi}$,\\
$\Phi(\alpha):=\int_{-\infty}^{\infty} \phi(x) e^{-i \alpha x} d x$, and its inverse $\psi \rightarrow \Psi$,\\
$\Psi(x):=1 / 2 \pi \cdot \int_{-\infty}^{\infty} \Psi(\alpha) \cdot e^{i \alpha x}$ da are topological isomorphisms of the Schwartz space $S(=S(\mathbb{R}))$. Since the subspace $D$ of all functions having compact support is dense in $S$ it follows that $\{\phi \in S: \overleftarrow{\phi} \in D\}$ is dense in $S$.

Lemma 7.5. For every function $\phi \in S$ we obtain an operator $T(\phi) \in L(E) \quad b y$

$$
T(\phi) \pounds:=\int_{-\infty}^{\infty} \phi(s) T(s) \pounds d s, E \in E E .
$$

This operator can be represented as\\
$T(\phi) f=\lim _{\varepsilon \rightarrow 0} 1 / 2 \pi \cdot \int_{-\infty}^{\infty} \overleftarrow{\phi}(\alpha)\lceil R(\varepsilon-i \alpha, A) f-R(-\varepsilon-i \alpha, A) f\rceil d \alpha, f \in E$.

Proof. That $T(\phi)$ is well-defined follows from the polynomial boundedness of $(T(t))_{t \in \mathbb{R}}$. In fact, $\phi \rightarrow T(\phi)$ is continuous from $S$ into $(L(E),\|\cdot\|)$.\\
We obtain

$$
\begin{aligned}
& \mathrm{T}(\phi) \mathrm{f}=\lim _{\varepsilon \rightarrow 0} \int_{-\infty}^{\infty} \phi(s) e^{-\varepsilon|s|} \mathrm{T}(s) f \mathrm{ds} \\
& =\lim _{\varepsilon \rightarrow 0} \int_{-\infty}^{\infty} 1 / 2 \pi \int_{-\infty}^{\infty} \stackrel{\leftarrow}{\phi}(\alpha) e^{i \alpha s} e^{-\varepsilon|s|} \mathrm{T}(s) f \mathrm{~d} \alpha \mathrm{ds} \\
& =\lim _{\varepsilon \rightarrow 0} 1 / 2 \pi \int_{-\infty}^{\infty} \overleftarrow{\phi}(\alpha) \int_{-\infty}^{\infty} e^{i \alpha s^{-\varepsilon}|s|} \mathrm{T}(\mathrm{~s}) \pounds \mathrm{ds} \mathrm{~d} \alpha \\
& =\lim _{\varepsilon \rightarrow 0} 1 / 2 \pi \int_{-\infty}^{\infty} \stackrel{+}{\phi}(\alpha)[R(\varepsilon-i \alpha, A) f-R(-\varepsilon-i \alpha, A) f] d \alpha .
\end{aligned}
$$

Here we used that polynomially bounded semigroups have growth bound 0 , hence $\omega(A)=\omega(-A)=0$. Hence the integral representation of $R(\varepsilon-i \alpha, A) \quad(c f . A-I, P r o p .1 .11)$ exists for $\varepsilon \neq 0$.

Lemma 7.6. If $E \neq\{0\}$, then $\sigma(A) \neq \emptyset$.

Proof. If $\sigma(A)=\varnothing$ then (7.3) implies $T(\phi)=0$ whenever $\overleftarrow{\phi}$ has compact support. Since these functions form a dense subspace of $S$ we conclude that $T(\phi)=0$ for all $\phi \in S$.

Choosing an approximate identity $\left(\Psi_{n}\right)_{n \in N} \subset D$ we obtain

$$
f=T(0) f=\lim _{n \rightarrow N} T\left(\psi_{n}\right) f=0
$$

for every $\mathrm{f} \in \mathrm{E}$.

Proof of Theorem 7.4 (1 ${ }^{\text {st }}$ part). By the Spectral Inclusion Theorem 6.2 we have to show that every spectral value of $T(t)$ can be approximated by exponentials of spectral values of A . In view of the rescaling procedure it suffices to prove this when $-1 \in \rho(T(\pi))$, provided that the following condition is satisfied.\\
(7.4) There exists $\varepsilon>0$ such that $U_{k \in \mathbb{Z}} i[2 \mathrm{k}+1-2 \varepsilon, 2 \mathrm{k}+1+2 \varepsilon] \subset \rho(A)$. Assume now that (7.4) holds. Then each of the sets $\sigma_{k}:=\left\{i_{\alpha} \in \sigma(A): \alpha \in[2 k-1,2 k+1]\right\}$ is a spectral set of A with corresponding spectral projection $\mathrm{P}_{\mathrm{k}}$. If we choose $\phi_{0} \in D$ such that supp $\phi_{O} \subset[-1+\varepsilon, 1-\varepsilon]$ and $\phi_{O}(x)=1$ for $x \in[-1+2 \varepsilon, 1-2 \varepsilon]$ it follows from $(7.3)$ and the integral representation of $p_{k}$ (cf.(3.1)) that $P_{0}=T\left(\overleftarrow{\phi}_{0}\right)$. More generally, since $\left(e^{i 2 k} \cdot \overleftarrow{\phi}_{0}\right) \rightarrow(\alpha)=\phi_{0}(\alpha-2 k)$, the assertions (7.3) and (7.4) imply


\begin{equation*}
P_{k}=\int_{-\infty}^{\infty} e^{i 2 k s} \overleftarrow{\phi}_{O}(s) \mathrm{T}(s) \text { ds for } k \in \mathbb{Z} \text {. } \tag{7.5}
\end{equation*}


At this point we isolate another lemma.

Lemma 7.7. span $U_{k \in \mathbb{Z}} P_{k} \mathrm{E}$ is dense in $E$.

Proof. The closure of span $U_{k \in \mathbb{Z}} P_{k} E$ is a T-invariant subspace $G$ of E. Consider the quotient group $\left(T(t) /{ }^{\prime} t \in \mathbb{R}\right.$ induced on $E / G$. The spectrum of its generator A , is contained in $\sigma$ (A) by Prop.4.2.iii. Moreover the spectral projection corresponding to $\sigma(A,) \cap \sigma_{k}$ is the quotient operator $\mathrm{P}_{\mathrm{k} /}$. Obviously $\mathrm{P}_{\mathrm{k} / \mathrm{F}}=0$, therefore $\sigma\left(\mathrm{A}\right.$, ) $\cap \sigma_{k}=\emptyset$ for every $k \in \mathbb{Z}$ and $\sigma(A)=,\emptyset$. By Lemma 7.6 this implies $\mathrm{E} / \mathrm{G}=\{0\}$, i.e. $\mathrm{G}=\mathrm{E}$.

Proof of Theorem 7.4 ( $2^{\text {nd }}$ part). We return to the situation of the first part. Using (7.5) the spectral projection $\mathrm{P}_{\mathrm{k}}$ can be transformed into

$$
\begin{aligned}
P_{k} & =\int_{-\infty}^{\infty} e^{i 2 k s} \overleftarrow{\phi}_{O}(s) T(s) d s \\
& =\sum_{m \in \mathbb{Z}} \int \begin{array}{c}
(\mathrm{m}+1 / 2) \pi \\
(\mathrm{m}-1 / 2)
\end{array} e^{i 2 k s} \overleftarrow{\phi}_{O}(s) T(s) d s \\
& =\int_{-\pi}^{\pi / 2} e^{i 2 k s} \sum_{m \in \mathbb{Z}} \overleftarrow{\phi}_{O}\left(s+m_{\pi}\right) T\left(s+m_{\pi}\right) d s,
\end{aligned}
$$

i.e., $P_{k} f$ is the $k-t h$ Fourier coefficient of the $\pi$-periodic, continuous function $\xi_{f}: s \rightarrow \sum_{m \in \mathbb{Z}} \bar{\phi}_{\mathrm{O}}(s+m \pi) \mathrm{T}(s+\mathrm{m} \pi) \mathrm{f}, \mathrm{f} \in \mathrm{E}$. Since the projections $P_{k}$ are mutually orthogonal, i.e. $P_{k} P_{m}=0$ for $k \neq m$, it follows that $g=\sum_{n \in \mathbb{Z}} P_{n} g$ for every $g \in \operatorname{span} U_{k \in \mathbb{Z}} P_{k} E$. In particular, the Fourier coefficients of the function $\xi_{g}$ are absolutely summable, hence the fourier series of $\xi_{\mathrm{g}}$ converges to $\xi$. For $s=0$ we obtain\\
$g=\sum_{n \in \mathbb{Z}} P_{n} g \cdot e^{-i n 0}=\sum_{m \in \mathbb{Z}} \overleftarrow{\phi}_{0}(0+m \pi) \quad T(0+m \pi) g \quad\left(g \in \operatorname{span} U_{k \in \mathbb{Z}} P_{k} E\right)$. since span $U_{k \in \mathbb{Z}} P_{k} E$ is dense (Lemma 7.7 ) we conclude that


\begin{equation*}
\sum_{\mathrm{m} \in \mathbb{Z}} \overleftarrow{\phi}_{\mathrm{O}}(\mathrm{~m} \pi) \mathrm{T}(\mathrm{~m} \pi)=\mathrm{Id} \tag{7.6}
\end{equation*}


As the final step we construct the inverse operator of Id $+\mathrm{T}(\pi)$ showing that $-1 \in \rho(T(\pi))$. We define $\Psi_{0}(\alpha):=\phi_{0}(\alpha) \cdot\left(1+e^{i \pi \alpha}\right)^{-1}$, $\alpha \in \mathbb{R}$. Then we have $\psi_{0} \in S$ and $\psi_{0}\left(1+e^{i \pi}\right)=\phi_{0}$, hence $\Psi_{O}(x)+\Psi_{O}(x+\pi)=\Phi_{O}(x)$ for all $x \in \mathbb{R}$. Then (7.6) implies $\mathrm{Id}=\sum_{\mathrm{m} \in \mathbb{Z}} \Phi_{O}(\mathrm{mH}) \mathrm{T}(\mathrm{mH})$\\
$=\sum_{m \in \mathbb{Z}}\left(\Psi_{0}(m \pi)+\Psi_{O}((m+1) \pi)\right) T(m \pi)$\\
$=\left[\sum_{m \in \mathbb{Z}} \Psi_{O}(\mathrm{~m} \pi) \mathrm{T}(\mathrm{m} \pi)\right](I \mathrm{~d}+\mathrm{T}(\pi))$.

In the rest of this section we discuss the behavior of the single spectral values $\lambda$ of $T(t), t>0$. The aim is a characterization of $\sigma(T(t))$ involving only properties of the generator. By the rescaling procedure $A-I, 3.1$ we may assume $\lambda=1$ and $t=2 \pi$. From the Spectral Inclusion Theorem 6.2 we know that $1 \in \rho(T(2 \pi))$ implies iz $\subset \rho(A)$. As seen in many examples the converse does not hold and we are now looking for additional conditions. Henceforth we assume $i \mathbb{Z} \subset \rho(\mathrm{~A})$ and define for $k \in \mathbb{Z}$


\begin{equation*}
Q_{k}:=1 / 2 \pi \int_{0}^{2 \pi} e^{-i k s} T(s) d s=1 / 2 \pi(1-T(2 \pi)) R(i k, A), \tag{7.7}
\end{equation*}


(cf. Formula A-I,(3.1)).

Our approach is based on Fejér's Theorem (for Banach space valued functions). Let us recall this result. Suppose $\xi$ : $[0,2 \pi] \rightarrow E$ is a continuous function and let $\xi_{k}:=1 / 2 \pi \int_{0}^{2 \pi} e^{-i k s} \xi(s)$ ds be its k-th Fourier coefficient. Then the Fourier series is Césaro summable to $\xi$ in every point $t \in(0,2 \pi)$. Moreover one has


\begin{equation*}
1 / 2(\xi(0)+\xi(2 \pi))=c_{1}-\sum_{k \in Z} \xi_{n}:=\operatorname{Iim}_{\mathrm{N} \rightarrow \infty} 1 / \mathrm{N} \cdot \sum_{\mathrm{n}=0}^{\mathrm{N}-1}\left(\sum_{\mathrm{k}=-\mathrm{n}}^{\mathrm{n}} \xi_{\mathrm{k}}\right) . \tag{7.8}
\end{equation*}


This result enables us to prove the following proposition:

Proposition 7.8. Let $(T(t))_{t \geqq 0}$ be a semigroup on a Banach space $E$ and denote its generator by A . Then the following conditions are equivalent:\\
(a) $1 \in \rho(\mathbb{T}(2 \pi))$,\\
(b) $\quad i \mathbb{Z} \subset \rho(A)$ and the series $\sum_{k \in \mathbb{Z}} R(i k, A) f$ is Césaro-summable for every $f \in E$,\\
(c) iZ $\subset$ (A) and the series $\sum_{k \in \mathbb{Z}} R(i k, A) Q_{k} f$ is Césaro-summable for every $f \in \mathrm{E}$.

Proof. (a) $\rightarrow$ (b): The Spectral Inclusion Theorem implies iZ $\subset$ o (A) . By $(7.7)$ we have $R(i k, A)=2 \pi \cdot(1-T(2 \pi))^{-1} Q_{k}$. Since $\sum_{k \in \mathbb{Z}} Q_{k} f$ is Cesaro-summable (towards $1 / 2(f+T(2 \pi) f)$ (see (7.8)) it follows that $\sum_{k \in \mathbb{Z}} R(i k, A) f$ is Césaro-summable as well.\\
(b) \&= (c): If we use A-I, (3.1) and integrate by parts, we obtain $R(i k, A) Q_{k} f=1 / 2 \pi \int_{0}^{2 \pi} e^{-i k s} T(s) R(i k, A) \pounds d s$\\
$=1 / 2 \pi \int_{0}^{2 \pi}\left[R(i k, A) f-\int_{0}^{s} e^{-i k t} T(t) f d t\right] d s$

$$
=R(i k, A) f-I / 2 \pi \int_{0}^{2 \pi} e^{-i k t}(2 \pi-t) T(t) f d t .
$$

Fejer's theorem ensures that $\sum_{k \in \mathbb{Z}}(1 / 2 \pi) \int_{0}^{2 \pi} e^{-i k t}(2 \pi-t) T(t) f d t$ is Césaro summable. Hence $\sum_{k \in \mathbb{Z}} R(i k, A) Q_{k} f$ is Césaro-summable if and only if $\sum_{k \in \mathbb{Z}} R(i k, A) f$ is.\\
(b) $\rightarrow$ (a): We have $Q_{k}=\frac{1}{2 \pi}(1-T(2 \pi)) R(i k, A)$. Furthermore we know by (7.7) and (7.8) that $\sum_{k \in \mathbb{Z}} Q_{k} f$ is Cesaro-summable towards $\frac{1}{2}(f+T(2 \pi) f)$.\\
If we define $\mathrm{S}: \mathrm{E} \rightarrow \mathrm{E}$ by $\mathrm{Sf}:=\frac{\mathrm{f}}{2}+\frac{1}{2 \pi} \cdot \mathrm{C}_{1}-\sum \mathrm{R}(i k, A) \mathrm{f}$ then we have $(1-T(2 \pi)) S f=\frac{1}{2}(f-T(2 \pi) f)+\frac{1}{2 \pi} \cdot C_{2}-\sum(1-T(2 \pi)) R(i k, A) f=$ $=\frac{1}{2}(f-T(2 \pi) f)+\frac{1}{2}(f+T(2 \pi) f)=f$.\\
Since $S$ commutes with $T(2 \pi)$ it follows that $S$ is the inverse of $(1-\mathrm{T}(2 \pi)$ ) thus $1 \in \rho(\mathrm{~T}(2 \pi))$.

Based on the equivalence of (a) and (b), one can state a characterization of the spectrum of $T(t)$ in terms of the generator and its resolvent only. However, in general it is difficult to verify the summability condition stated in (b).\\
In Hilbert spaces the boundedness of the resolvents will suffice (see Thm. 7. 10 below).

Lemma 7.9. Let $(\mathrm{T}(\mathrm{t}))_{t \geq 0}$ be a semigroup on some Hilbert space $H$ and assume $i \mathbb{Z} \subset \rho(A)$ for the generator $A$. Then we have\\
(i) $\quad\left(Q_{k} f\right)_{k \in \mathbb{Z}} \subset 2^{2}(H)$ for every $f \in H$, and\\
(ii) if $\sup _{k \in \mathbb{Z}}\|R(i k, A)\|<\infty$, then $\sum_{k \in \mathbb{Z}} R(i k, A) f_{k}$ is summable whenever $\left(f_{k}\right)_{k \in \mathbb{Z}} \in \ell^{2}(H)$.

Proof. (i) We consider the Hilbert space $\mathrm{L}^{2}([0,2 \pi], H)$ and obtain $0 \leqq\left\|\mathrm{~T}(\cdot) \mathrm{f}-\sum_{k=-n}^{n} Q_{k} f \cdot e^{i k \cdot}\right\|^{2}$\\
$=\int_{0}^{2 \pi}\|T(s) E\|^{2} d s-\int_{0}^{2 \pi} \sum_{k=-n}^{n}\left(T(s) f \mid e^{i k s} Q_{Q_{k}} f\right) d s-$ $\int_{0}^{2 \pi} \sum_{k=-n}^{n}\left(e^{i k s} Q_{k} f \mid T(s) f\right) d s+\int_{0}^{2 \pi}\left(\sum_{k=-n}^{n} e^{i k s} Q_{k} f \mid \sum_{\ell-n}^{n} e^{i \ell s} Q_{\ell} f\right) d s$ $=\int_{0}^{2 \pi}\|\mathrm{~T}(\mathrm{~s}) \mathrm{f}\|^{2} \mathrm{ds}-2 \pi \sum_{\mathrm{k}=-n}^{\mathrm{n}}\left\|\mathrm{e}_{\mathrm{k}} f\right\|^{2}$, (use (7.5)).\\
It follows that $\sum_{\mathrm{k} \in \mathbb{Z}}\left\|\mathrm{Q}_{\mathrm{k}} \mathrm{E}\right\|^{2} \leqq \frac{1}{2 \pi} \cdot \int_{0}^{2 \pi}\|\mathrm{~T}(\mathrm{~s}) \mathrm{f}\|^{2} \mathrm{ds}<\infty$.\\
(ii) Fix $\lambda>0$ sufficiently large and set

$$
g_{k}:=(1+\lambda R(i k, A)) f_{k^{\prime}} k \in \mathbb{Z}
$$

Using the resolvent equation and then $(A-I,(3.1))$ we obtain\\
$R(i k, A) f_{k}=R(\lambda+i k, A) g_{k}=\left[1-e^{-2 \pi \lambda} T(2 \pi)\right]^{-1} \int_{0}^{2 \pi} e^{-\lambda s} e^{-i k s_{T}}(s) g_{k} d s$. This yields for every finite subset $N$ of $\mathbf{Z}$ $\left\|\Sigma_{k \in \mathbb{N}} R(i k, A) f_{k}\right\| \leqq\left\|\left(1-e^{-2 \pi \lambda} T(2 \pi)\right)^{-1}\right\| \cdot \int_{0}^{2 \pi}\|T(s)\|\left\|\Sigma_{k \in N} e^{-i k s} g_{k}\right\| d s \leqq$ $\leqq\left\|\left(1-e^{-2 \pi \lambda} \mathrm{~T}(2 \pi)\right)^{-1}\right\| \cdot\left(\int_{0}^{2 \pi}\|\mathrm{~T}(\mathrm{~s})\|_{i}^{2} \mathrm{ds}\right)^{1 / 2} \cdot\left(\int_{0}^{2 \pi}\left\|\Sigma_{\mathrm{k} \in \mathrm{N}} e^{-i k s} \mathrm{~g}_{\mathrm{k}}\right\|^{2} \mathrm{dx}\right)^{1 / 2}$ $=c\left(\sum_{k \in N}\left\|g_{k}\right\|^{2}\right)^{1 / 2} \leqq c(1+\lambda M)\left(\sum_{k \in N}\left\|f_{k}\right\|^{2}\right)^{1 / 2}$.\\
Here $c:=\left\|\left(1-e^{-2 \pi \lambda} \mathrm{~T}(2 \pi)\right)^{-1}\right\| \cdot\left(\int_{0}^{2 \pi}\|\mathrm{~T}(\mathrm{~s})\|^{2} \mathrm{ds}\right)^{1 / 2}$ and $M:=\sup _{k \in \mathbb{Z}}\|\mathrm{R}(i k, A)\|$.

Theorem 7.10. Let $A$ be the generator of a semigroup $(T(t))_{t \geqq 0}$ on some Hilbert space $H$. Then the following form of the spectral mapping theorem is valid\\
$\sigma(T(t)) \backslash\{0\}=\left\{e^{\lambda t}:\right.$ either $\mu_{k}:=\lambda+2 \pi i k / t \in \sigma(A)$ for some $k \in \mathbb{Z}$ or $\left(\left\|R\left(\mu_{k}, A\right)\right\|\right)_{k \in Z}$ is unbounded $\}$.

Proof. If $e^{\lambda t}(\sigma(T(t))$ it follows from the spectral inclusion theorem that $\mu_{k} \neq \sigma(A)$ for every $k \in \mathbb{Z}$ and from $A-I, 3.1$, Formula (3.1), that $\left\|R\left(\mu_{k}, A\right)\right\| i$ is bounded. For the converse inclusion it suffices to assume $t=2 \pi$ and $\lambda=0$ (use the rescaling procedure $A-I, 3.1)$. Assuming that $i \mathbb{Z} \subset \rho(A)$ and $\|R(i k, A)\|$ is bounded then $\Sigma_{k \in \mathbb{Z}} R(i k, A) Q_{k} f$ is summable by Lemma $7.9 \%$ Since every summable series is Césaro-summable condition (c) of Prop. 7.8 is satisfied and we conclude $1 \in \rho(\mathrm{~T}(2 \pi))$.

As an immediate consequence we obtain an interesting characterization of the growth bound $\omega$ of semigroups on Hilbert spaces.

Corollary 7.11. The growth bound of a semigroup ( $T(t))_{t \geq 0}$ on a Hilbert space H satisfies\\
(7.9) $\quad \omega=\inf \{\lambda \in \mathbb{R}: \lambda+i \mathbb{R} \in \rho(A)$ and $\|R(\lambda+i \mu, A)\|$ is bounded for $\mu \in \mathbb{R}\}$.

The Example 1.3 above in combination with C-III, Cor.1.3 shows that (7.9) is not valid in arbitrary Banach spaces.

\section*{NOTES.}
Section 1. It was already known to Hille-Phillips (1957) that for strongly continuous semigroups $(T(t))$ with generator A the spectral mapping theorem " $\sigma(\mathrm{T}(t)$ ) $=\exp (t \sigma(A))^{\prime \prime}$ may be víolated in various ways [1.c., Sec.23.16]. The simple Examples 1.3 and 1.4 are due to Wolff (see Greiner-Voigt-Wolff (1981)) and Zabczyk (1975). A more sophisticated example of a positive group with "s(A) < $\omega(A)$ " is given in Wolff (1981).

Section 2. In Definition 2.1 we define the residual spectrum of A in such a way that it coincides with the point spectrum of the adjoint $A^{\prime}$ (see Prop. 2.2.(ii)). It therefore differs slightly from the one used, e.g., by schaefer (1974). The spectral mapping theorem for the resolvent (Thm.2.5) is well known and can, e.g., be deduced from Lemma 9.2 and Thm. 3.11 of Chap.VII in Dunford-Schwartz (1958).

Section 3. The general theory of spectral decompositions can be found in [Kato (1966), Chap.III, § 6.4]. Applications to isolated singularities like 3.6 are discussed extensively in [1.c., Chap.III, \$6.5] and [Yosida (I965), Chap.VIII, Sec.8]. There are many ways to introduce an "essential spectrum" (see the footnote on page 243 of Kato (1966)). However each one yields the same "essential spectral radius".

Section 4. These results are taken from Derndinger (1980) and Greiner (1981).\\
Section 5. Periodic semigroups are studied explicitely in Bart (1977) but most of the results of this section seem to be well known.

Section 6. The Spectral Inclusion Theorem 6.2 and the Spectral Mapping Theorem 6.6 for eventually norm continuous semigroups date back to Hille-Phillips (1957). Davies (1980) gives an elegant proof using Banach algebra techniques. Tensor products of operators and their spectral theory have been studied by Ichinose and others (see Chap. XIII. 9 of Reed-Simon (1978)). The spectral mapping theorem in Corollary 6.8 generalizes Thm.XIII. 35 of Reed-Simon (1978) (see also Herbst (1982)).

Section 7. Matrix valued multiplication semigroups appear as solution, via Fourier transformation, of systems of partial differential equations. Kreiss initiated a systematic investigation (see, e.g., Kreiss (1958), Kreiss (1959), Miller-Strang\\[0pt]
(1966) ) and the Weak Spectral Mapping Theorem 7.2 must have been known to him. The direct proof of the Weak Spectral Mapping Theorem 7.4 for polynomially bounded groups seems to be new. The result can also be deduced from the theory of spectral subspaces of group representations (see, e.g., Combes-Delaroche (1978)), since the Arveson spectrum of a strongly continuous one-parameter group can be identifled with the spectrum of the generator (see Evans (1976)). The final part of this section is taken from Greiner (1985) and yields a new approach to Gearhart's characterlzation of the spectrum of semigroups on Hilbert spaces [Gearhart (1978)]. Different proofs can be found in Herbst (1983), How1and (1984) and Prüß (1984).

\section*{by}
Frank Neubrander

In this chapter we study the asymptotic behavior of the solutions of the initial value problem


\begin{equation*}
\dot{u}(t)=A u(t)+F(t), u(0)=\pounds \tag{*}
\end{equation*}


with respect to time $t \geqq 0$. Here $A$ will be a generator of a strongly continuous semigroup $(T(t))_{t \geqq 0}$ on a Banach space $E$ and $F(\cdot)$ is a function from $\mathbb{R}_{+}$with values in $E$.\\
In section 1 we investigate whether and how fast a solution $T(\cdot) f$ of the homogeneous problem tends to the zero solution as $t \rightarrow \infty$; in Section 2 we consider the long term behavior of the solutions of (*) for different classes of forcing terms $F$.

\section*{1. STABILITY : HOMOGENEOUS CASE}
Let $(T(t))_{t \geq 0}$ be a semigroup on $E$ with generator $A$. An initial value $f \in D(A)$ is called stable if the solution $t \rightarrow T(t) f$ of\\
(ACP) $\quad \dot{u}(t)=A u(t), u(0)=f$\\
converges to zero as $t$ tends to infinity. The semigroup is called stable if every solution converges to zero; i.e., if every initial value $f \in D(A)$ is stable.\\
If the space $E$ is finite dimensional, then the stability of the semigroup implies that the decay is exponential. More precisely, the statements\\
(a)\\
$\|T(t) f\| \rightarrow 0$ for every $f \in \mathbb{C}^{n}$,\\
(b) $\|T(t)\| \leqq M e^{-\omega t}$ for some $\omega>0$\\
are equivalent. Moreover, these statements hold if and only if

$$
s(A)=\sup \{\operatorname{Re} \lambda: \lambda \in \sigma(A)\}<0,
$$

see A-III, Cor.1.2.

As already discussed in chapter A-III the situation is far more difficult in the infinite dimensional case. Here, and for unbounded generators, we have to distinguish between strong and generalized (mild) solutions of $\dot{u}(t)=A u(t)$ and between various notions of stability, Recall that for $f \in D(A)$ the function $T(\cdot) f$ is a strong solution of (ACP) (see A-II,Cor.1.2.); for arbitrary $f \in E$ the function T(•)f is called a generalized or mild solution of (ACP). Next we introduce several constants characterizing the growth of the solutions of (ACP).

Definition 1.1 ( $1^{\text {st }}$ part). Let $A$ be the generator of a strongly continuous semigroup $(T(t))_{t \geqslant 0}$ on a Banach space $E$.\\
Then\\
(i) $\quad \omega$ (f):= inf\{w:|T(t)f|ฏMemt for some $M$ and every $t \geq 0\}$ is called the (exponential) growth bound of T(.)f .\\
(ii) $\quad \omega_{1}(A):=\sup \{\omega(f): f \in D(A)\}$ is called the (exponential) growth bound of the solutions of the cauchy problem $\dot{u}(t)=A u(t)$.\\
(iii) $\omega(A)=\sup \{\omega(f): f \in E\}$ is called the (exponential) growth bound of the mild solutions of the Cauchy problem $\dot{u}(t)=A u(t)$.

Note that, by the Principle of Uniform Boundedness, sup\{w(f): $f \in E\}$ $=\inf \left\{\omega:\|T(t)\| \leqq e^{w t}\right.$ for some $M$ and every $\left.t \geqq 0\right\}$. Hence $\omega(A)$ coincides with the growth bound of the semigroup ( $\mathrm{T}(\mathrm{t})^{\mathrm{I}}{ }_{t \geqq 0}$ as defined in A-I,1.3. With the constants defined above we obtain the following stability concepts.

Definition 1.1 (2 $2^{\text {nd }}$ part). The semigroup $(T(t))_{t \geqq 0}$ is called\\
(iv) uniformly exponentially stable if $\omega(A)<0$;\\
(v) exponentially stable if $\omega_{1}(A)<0$;\\
(vi) uniformly stable if $\|\mathrm{T}(t) f\| \rightarrow 0$ (as $t \rightarrow \infty$ ) for every $\mathbf{f} \in \mathrm{E}$; (vii) stable if $\|\mathrm{T}(t) \mathrm{f}\| \rightarrow 0 \quad(\mathrm{as} \quad t \rightarrow \infty)$ for every $f \in \mathrm{D}(\mathrm{A})$.

The interrelation between these stability concepts is given by\\
\includegraphics[max width=\textwidth, center]{2024_12_23_c6487cc0859199a15bd9g-110}

If $A$ is a bounded operator, i.e., if $D(A)=E$, then (iv) $<=>(v)$ and (vi) <=> (vii). If $A$ is unbounded then the stability notions may differ as we will see in the following examples.

Examples 1.2. (a) Let $E=c_{0}$. Then $A:\left(x_{n}\right)_{n \in \mathbb{N}} \rightarrow\left(-1 / n \cdot x_{n}\right)_{n \in \mathbb{N}}$ generates the semigroup $T(t)\left(x_{n}\right)=\left(e^{-t / n_{n}}\right)_{n}{ }_{n} \in$. It is easy to see that $\|T(t)\|=1$ and $\|T(t) \pm\| \rightarrow 0$ for every $f \in c_{0}$. Moreover, $A$ is a bounded operator, hence $D(A)=E$. This gives an example for a (uniformly) stable but not exponentially stable semigroup. The translation semigroups generated by the first derivative on $C_{0}\left(\mathbb{R}_{+}\right)$or $L^{\mathrm{P}}\left(\mathbb{R}_{+}\right)$for $1<\mathrm{p}<\infty$ give further examples for (uniformly) stable but not exponentially stable semjgroups. Moreover, as seen in A-II, Ex.1.14, the Laplacian $\Delta$ on $C_{0}\left(\mathbb{R}^{n}\right)$ generates a bounded holomorphic semigroup given by

$$
T(t) f(x)=(4 \pi t)^{-n / 2} \cdot \int_{\mathbb{R}^{n}} e^{-(x-y)^{2} / 4 t} f(y) d y
$$

which cannot be exponentially stable because $0 \in \sigma(\Delta)\left(i m \Delta \neq c_{0}\left(\mathbb{R}^{n}\right)\right)$, see Cor.1.5 below, By a straightforward (2-E)-argument using $(4 \pi t)^{-n / 2} \int_{\mathbb{R}^{n}} \exp \left(-y^{2} / 4 t\right) d y=1$ one can easily show that $\|T(t) E\| \rightarrow 0$ for all $f \in C_{0}\left(\mathbb{R}^{n}\right)$ (see also B-III, Ex.1.7).\\
Therefore, the Laplacian on $C_{0}\left(\mathbb{R}^{n}\right)$ (and also on $L^{\mathrm{P}}\left(\mathbb{R}^{\mathrm{n}}\right)$ for $1<\mathrm{p}<\infty$, see Ex. 1.15 below) generates a (uniformly) stable but not exponentially stable semigroup.\\
(b) Note that the condition $0 \leqq \omega(A)=\inf (\omega:\|T(t)\| \leqq$ Me for all $t \geqq 0\}$ does not exclude that the semigroup $(T(t))_{t \geqq 0}$ is exponentially stable. In fact, as shown in A-III,1.3 the translation semigroup $(T(t))_{t \geqq 0}$ on $E:=C_{0}\left(\mathbb{R}_{+}\right) \cap I^{1}\left(\mathbb{R}_{+}, e^{\left.x_{d x}\right)}\right.$ satisfies $\|T(t)\|$ $=1$, hence $\omega(A)=0$, and for every $\lambda \in \mathbb{C}$ with $\operatorname{Re} \lambda>-1$ the resolvent of the generator is given as $R(\lambda, A) f=\int_{0}^{\infty} e^{\lambda t} T(t) f d t$ for every $f \in E$. From the equation $A-I, 3.2$

$$
T(t) f=e^{\lambda t}\left(f-\int_{0}^{t} e^{-\lambda s} T(s)(\lambda-A) f d s\right)
$$

and the existence of $\lim _{t \rightarrow \infty} \int_{0}^{t} e^{-\lambda s} T(s)(\lambda-A) f d s$ it follows that $\|T(t) f\| \leqq M \cdot e^{\lambda t}$ for every $f \in D(A)$ and some constant $M$ depending\\
on $f$. This yields $w_{1}(A) \leqq-1<0=w(A)$. Thus we have a semigroup which is exponentially, but not uniformly exponentially stable.\\
(c) Rescaling this semigroup (see $A-I, 3.1$ ) we obtain a semigroup with $-1 / 2=\omega_{1}(A)$ and $1 / 2=\omega(A)$. Therefore there are exponentially stable and hence stable semigroups which are not bounded and hence not uniformly stable. We conclude from this example that there may be an essential difference between the long term behavior of the semigroup $\left(T(t){ }_{t \geqq 0}\right.$ (i.e. of the set of all mild solutions) and the long term behavior of the strong solutions $\{T(\cdot) f: f \in D(A)\}$ of (ACP).

In the following we characterize the exponential growth bounds $\omega(f)$, $\omega_{1}(A)$ and $\omega(A)$ by certain abscissas of simple or absolute convergence of the Laplace transform of $\mathrm{T}(\cdot) \mathrm{I}$. These characterizations will be the basic tool in showing that for certain semigroups the growth bounds $w(A)$ and/or $\omega_{1}(A)$ coincide with the spectral bound $s(A)=\sup \{\operatorname{Re} \lambda: \lambda \in \sigma(A)\}$.

We remark first that $s(A)$ can be regarded as the abscissa of holomorphy of the Laplace transform $\lambda \rightarrow \int_{0}^{\infty} e^{\lambda t} \mathrm{~T}(t) d t$ of the semigroup $(T(t))_{t \geqslant 0}$.\\
Next we recall that the Laplace transform exists for every $\lambda \in \mathbb{C}$ with $\operatorname{Re} \lambda>\operatorname{Re} \mu$ as soon as it exists for $\mu$. This follows from the equation


\begin{align*}
\int_{0}^{t} e^{-\lambda s} f(s) d s= & e^{-(\lambda-\mu) t} \cdot \int_{0}^{t} e^{-\mu s} f(s) d s  \tag{1.1}\\
& +(\lambda-\mu) \int_{0}^{t} e^{-(\lambda-\mu) s} \int_{0}^{s} e^{-\mu r} f(r) d r d s
\end{align*}


Note that even boundedness of $\int_{0}^{t} e^{-\mu s} f(s)$ ds implies the existence of the Laplace transform for Re $\lambda>\operatorname{Re} \mu$. Therefore the subset of $\mathbb{C}$ for which the Laplace transform exists is always a half-plane $\{\lambda \in \mathbb{C}$ : $\operatorname{Re} \lambda>\gamma\} U H$, where $H$ is a subset of the line $\{\lambda \in \mathbb{C}: \operatorname{Re} \lambda=\gamma\}$.

In the subsequent theorem we show that the bound of the half-plane for which the Laplace transform of $T(\cdot) f$ (f $\in$ E) exists absolutely and the bound of the half-plane for which the Laplace transform of T(•)Af ( $f \in D(A)$ ) exists strongly coincide with the growth bound $\omega(f)=$ $\inf \left\{\omega:\|T(t) f\| \leqq M^{\omega t}\right.$ for all $\left.t \geqq 0\right\}$.

Theorem 1.3. Let $A$ be the generator of a strongly continuous semigroup $(T(t))_{t \geqslant 0}$ on a Banach space $E$. Then, for every $f \in E$,


\begin{equation*}
w(f)=1 \mathrm{im} \sup _{t \rightarrow \infty} 1 / t \cdot \log \|T(t) f\|, \tag{1.2}
\end{equation*}


and\\
(i) $\quad \omega(f)=\inf \left\{\operatorname{Re} \lambda: \int_{0}^{\infty}\left\|e^{-\lambda t} T(t) f\right\| d t\right.$ exists $\}$.

If $\operatorname{ker} A=\{0\}$, then for every $f \in D(A)$ we have\\
(ii) $\quad \omega(f)=\inf \left\{\operatorname{Re} \lambda: \int_{0}^{\infty} e^{-\lambda t} \mathrm{~T}(t) A f d t\right.$ exists as an improper Riemann integral) .

Proof. The proof of (1.2) is omitted (see Hille-Phillips (1957), p.306). In order to prove (i) and (ii) we need the following lemma.

Lemma. Let $F \in C\left(\mathbb{R}^{+}, \mathbb{R}^{+}\right)$be such that $\int_{0}^{\infty} F(t)$ dt exists. If there is a positive number $m$ and an interval $[0, n]$ such that $F(t+s) \leqq$ $m \cdot F(s)$ for $a l l s \geq 0$ and $t \in[0, n]$, then $\lim _{s \rightarrow \infty} F(s)=0$.

Proof of the lemma. For all $\varepsilon>0$ there exists a $>0$ such that $A(a):=\int_{a}^{\infty} F(s) d s \leqq \frac{n}{m} \cdot \varepsilon$. For all $t>a+n$ there exists $r \in[t-n, t]$ such that $F(x) \leqq \frac{1}{n} \cdot A(a)$.\\
Therefore, $F(t)=F(t-r+x) \leqq m \cdot F(x) \leqq \frac{m}{n} \cdot A(a) \leqq \varepsilon$.

In order to prove (i) we define $b:=\inf \left\{R e \lambda: \int_{0}^{\infty}\left\|e^{-\lambda t} T(t) f\right\| d t\right.$ exists\}. A straightforward application of the lemma shows that $\omega(\pounds) \leqq$ b. The definition of $\omega(f)$ gives the reverse inequality. It remains to prove statement (ii) of Thm.1.3.\\
Assume that ker $A=\{0\}$ and let $f \in D(A), \lambda \in \mathbb{C}$ with $\operatorname{Re} \lambda>\omega(f)$. From the equation

$$
\int_{0}^{t} e^{-\lambda s} T(s) A f d s=e^{-\lambda t} T(t) E-f+\lambda \int_{0}^{t} e^{-\lambda s} T(s) f d s
$$

it follows that $\int_{0}^{\infty} e^{-\lambda s} T(s) A f d s$ exists.\\
Therefore $b:=\inf \left\{\operatorname{Re} \lambda: \int_{0}^{\infty} e^{-\lambda t} T(t) A f d t\right.$ exists\} $\leqq w(\pounds)$.

Next we show that $b<0$ implies $b \leqq \omega(f)$. Suppose $b<0$. Then, by (1.1), $\int_{0}^{\infty} T(s) A f d s$ exists. By $\int_{0}^{r} T(s) A f d s=T(r) f-f$ we see that $\lim _{r \rightarrow \infty} \mathrm{~T}(r) f=: g$ exists. But, for every $t \geqq 0, T(t) g=g$ and therefore $g \in$ Ker $A$ or $g=0$. Hence $\int_{t}^{\infty} T(s) A f d s=-T(t) f$. Then choosing $r, b<r<0$, and integrating by parts we obtain

$$
\begin{aligned}
-T(t) f= & \lim _{u \rightarrow \infty} \int_{t}^{u} e^{r s} e^{-r s} T(s) A f d s \\
= & \lim _{u \rightarrow \infty}\left(e^{r u} \int_{0}^{u} e^{-r s} T(s) A f d s-e^{r t} \int_{0}^{t} e^{-r s} T(s) A f d s\right. \\
& \left.\quad-r \int_{t}^{u} e^{r s} \int_{0}^{s} e^{-r v} T(v) A f d v d s\right) \\
= & -e^{r t} \int_{0}^{t} e^{-r s} T(s) A f d s-r \int_{t}^{\infty} e^{r s} \int_{0}^{s} e^{-r v} T(v) A f d v d s .
\end{aligned}
$$

From $\left\|\int_{0}^{t} e^{-r s} T(s) A f d s\right\| \leq M$ for some $M$ and every $t \geqq 0$ we conclude that $\|T(t) f\| \leqq \tilde{M e} r$ for $a l l \quad t \geqq 0$ and some constant $\tilde{M}$. Hence $\omega(f) \leqq r$ For every $b<r<0$, i.e., $\omega(f) \leqq b$.\\
If $b \geqq 0$ and $w>b$, then $\left\|\int_{0}^{t} e^{-w s} T(s) A f d s\right\| \leqq M$ for all $t \geqq 0$.\\
By $\quad T(t) f-f=\int_{0}^{t} e^{W s} e^{-W s} T(s) A f d s$

$$
=e^{w t} \int_{0}^{t} e^{-w s} T(s) A f d s-w \int_{0}^{t} e^{w s} \int_{0}^{s} e^{-W r} T(r) A f d r d s
$$

we obtain $\|T(t) \pounds-f\| \leqq M e^{W t}+M\left(e^{W t}-1\right) \leqq 2 M e^{W t}$. Hence $w(f) \leqq w$ for every $w>b$, i.e., $\omega(f) \leqq b$.

From (1.2) and the Uniform Boundedness Principle it follows that the growth bound $\omega_{1}(A)=\sup \{\omega(f): \pounds \in D(A)\}$ satisfies\\
(1.3) $\omega_{1}(A)=$ infiw $=$ for every $f \in D(A)$ there exists a constant $M$ such that $\|\mathrm{T}(t) \mathrm{f}\| \leqq \mathrm{Me}^{\mathrm{wt}}$ for every $t \geqq 0$ \} $=\lim \sup _{t \rightarrow \infty} 1 / t \cdot \log \|T(t) R(\lambda, A)\| \quad(\lambda \in \rho(A))$.

The subsequent theorem will be of particular importance in the stability theory of positive semigroups. We show that the constant $\omega_{1}(A)$ coincides with the abscissa of simple convergence of the Laplace transform of the semigroup and with the abscissa of absolute convergence of the Laplace transform of the strong solutions of (ACP).

Theorem 1.4. Let $A$ be the generator of a strongly continuous semigroup $(\mathbb{P}(t))_{t \geqq 0}$ on a Banach space $E$. Then\\
(1.4) $\quad \omega_{1}(A)=\inf \left\{\operatorname{Re} \lambda: \int_{0}^{\infty} e^{-\lambda t} \mathrm{~T}(t) f d t\right.$ exists as an improper Riemann integral for every $f \in \mathbb{E}\}$\\
$=\inf \left\{\operatorname{Re} \lambda: \int_{0}^{\infty}\left\|e^{-\lambda t} \mathrm{~T}(t) f\right\| d t\right.$ exists for every $f \in D(A)$.

Remarks. (a) one can show that the abscissas of uniform, strong and weak convergence of the Laplace transform coincide (see C-III, Thm.1.2, last part of the proof). Therefore, by Thm.1.4,\\
(1.5) $\quad \omega_{1}(A)=\inf \left\{\operatorname{Re} \lambda\right.$ : weak-lim ${ }_{t \rightarrow \infty} \int_{0}^{t} e^{-\lambda s} \mathrm{~T}(\mathrm{~s})$ ds exists\}\\
$=\inf \left\{\right.$ Re $\lambda$ : uniform-lim ${ }_{t \rightarrow \infty} \int_{0}^{t} e^{-\lambda s} \mathrm{~T}(s) \mathrm{ds}$ exists\}.\\
(b) In the equations (1.4) and (1.5) the term "Re $\lambda$ " may be replaced by " $\lambda \in \mathbb{R}^{\prime}$ (use (1.1)).

Proof of Thm.1.4. The equality $\omega_{1}(A)=\inf \left\{\operatorname{Re} \lambda: \int_{0}^{\infty}\left\|e^{-\lambda t_{T}} \mathrm{~T}(t) \mathrm{f}\right\| \mathrm{dt}\right.$ exists for all $f \in D(A)\}$ follows from the definition of $\omega_{1}(A)$ and the lemma used in the proof of Thm.1.3.\\
We prove $\omega_{1}(A)=$ inf\{Re $\lambda: \int_{0}^{\infty} e^{-\lambda s} T(s) f$ ds exists for every $\left.f \in E\right\}$. The identity

$$
T(t) f=e^{\lambda t}\left(f-\int_{0}^{t} e^{-\lambda s} T(s)(\lambda-A) f d s\right)
$$

yields\\
$w_{1}(A) \leqq \inf \left\{\operatorname{Re} \lambda: \int_{0}^{\infty} e^{-\lambda t} T(t) \pounds d t\right.$ exists for every $\left.f \in \operatorname{im}(\lambda-A)\right\}$. Therefore\\
$\omega_{1}(A) \leqq \inf \left\{\operatorname{Re} \lambda: \int_{0}^{\infty} e^{-\lambda t} \mathrm{~T}(t) f d t\right.$ exists for every $\left.f \in \mathrm{E}\right\}=$ : $b$. Take $\lambda \in \mathbb{C}$ with $\operatorname{Re} \lambda>\omega_{1}(A)$. Then $\int_{0}^{\infty} e^{-\lambda t} T(t) f$ dt exists for every $f \in D(A)$. Define $g:=\int_{0}^{1} e^{-\lambda t} T(t) f d t$. Then $g \in D(A)$ and $\int_{0}^{n} e^{-\lambda t} \mathrm{~T}(t) f d t=\sum_{k=0}^{n-1} e^{-\lambda k} T(k) g$. Since $\operatorname{Re} \lambda>\omega_{1}(A)$ it follows that the sum converges for every $g \in D(A)$. Therefore the integral converges as $n \rightarrow \infty(n \in \mathbb{N})$ for every $f \in \mathbb{E}$. For every $t \in \mathbb{R}_{+}$ define a bounded operator $T_{t}$ by $f \rightarrow \int_{0}^{t} e^{-\lambda s} T(s) f d s$. As seen above, $T_{n} \pounds$ converges as $n \rightarrow \infty \quad(n \in \mathbb{N}$ ) for every $f \in \mathrm{E}$. It follows from the Uniform Boundedness Principle that the family $\left(T_{n}\right.$ ) $n \in \mathbb{N}$ is uniformly bounded.\\
For every $t \in \mathbb{R}_{+}$there exist $n \in \mathbb{N}$ and $t^{\prime} \in[0,1)$ such that $\mathrm{T}_{t}=\mathrm{T}_{t^{\prime}}+\mathrm{e}^{-\lambda t^{\prime}} \mathrm{T}\left(\mathrm{t}^{\prime}\right) \mathrm{T}_{\mathrm{n}}$. Since the operator families on the right side of the equation are uniformly bounded the same is true for $\left(T_{t}\right)_{t \geq 0}$. Since $\left(T_{t} f\right)_{t \geq 0}$ converges for every $f \in D(A)$ it follows that $\left(T_{t}\right)_{t \geqq 0}$ converges for every $\pounds \in E$. Thus $b \leqq \omega_{1}(A)$.

The inequality\\
$\omega(A) \geqq \inf \left\{\operatorname{Re} \lambda: \int_{0}^{\infty}\left\|e^{-\lambda t} T(t) f\right\| d t\right.$ exists for every $\left.\pounds \in E\right\}$\\
in combination with the lemma of Thm. 1.3 implies that the growth bound $\omega$ (A) coincides with the abscissa of absolute convergence of the Laplace transform of $(T(t))_{t \geqq 0}$; i.e.,\\
(1.6) $\omega(A)=\inf \left\{\operatorname{Re} \lambda: \int_{0}^{\infty}\left\|e^{-\lambda t} T(t) f\right\| d t\right.$ exists for every $\left.f \in E\right\}$.

As seen in $A-I$, Prop.1.11, if $\int_{0}^{\infty} e^{-\lambda t} T(t) f d t$ exists for every $f \in E$, then $\lambda \in \rho(A)$ and $R(\lambda, A) f=\int_{0}^{\infty} e^{-\lambda t} T(t) f d t$. This and Thm. 1.4 yield the following corollary.

Corollary 1.5. Let A be the generator of a strongly continuous semigroup $(T(t))_{t \geqq 0}$ on a Banach space $E$. Then

$$
s(A) \leqq \omega_{1}(A) \leqq \omega(A)
$$

Example 1.2.(2) shows that the uniform exponential stability of the semigroup is not equivalent to $\sigma(A) \subset\{\lambda \in \mathbb{C}: \operatorname{Re} \lambda \leqq q<0\}$. In the following example we will see that not only the semigroup (i.e., all. generalized solutions of (ACP)), but also the strong solutions can be unstable even when $s(A)<0$. In fact, we will give an example of a semigroup with $s(A)<\omega_{1}(A)<\omega(A)$.

Example 1.6. In A-III, Ex.1.4 it was shown that the semigroup (T(t)) ${ }_{t \geq 0}$ on the Hilbert space $E=\left\{\left(x^{1}, x^{2}, \ldots\right), x^{n} \in \mathbb{C}^{n}: \sum_{i=1}^{\infty}\left\|x^{i}\right\|^{2}<\infty\right\}$ given by $T(t):=\left(e^{2 \pi i n t} \cdot \exp \left(t A_{n}\right)\right){ }_{n \in \mathbb{N}}$ with

$$
A_{n}=\left(\left.\begin{array}{ccccc}
0 & 1 & 0 & \ldots & 0 \\
\cdot & \cdot & & & 0 \\
\cdot & & \cdot & & 1 \\
\dot{0} & \ldots & \ldots & \ldots & 0
\end{array}\right|_{n \times n}\right.
$$

has growth $e^{t}\left(\|T(t)\|=e^{t}\right)$. Thus $\omega(A)=1$ whereas the generator $A=\left(2 \pi i n+A_{n}\right)_{n \in \mathbb{N}}$ has spectral bound 0 . We will show first that for this semigroup $\omega_{1}(A)=\omega(A)$ holds (we will use this to construct a semigroup with $\left.s(A)<\omega_{1}(A)<\omega(A)\right)$.\\
Let $e_{n}=n^{-1 / 2} \cdot(1, \ldots, 1) \in \mathbb{C}^{n}$. Then we have $\left\|\exp \left(t A_{n}\right) \cdot e_{n}\right\|^{2}=$

$$
\begin{aligned}
& =\frac{1}{n} \cdot\left\|\left(1+t+\ldots+\frac{t^{n-1}}{(n-1)!}, 1+t+\ldots+\frac{t^{n-2}}{(n-2)!}, \cdots, 1+t, 1\right)\right\|^{2}= \\
& =\frac{1}{n} \cdot \sum_{r=0}^{n-1}\left(\sum_{j=0}^{r} \frac{1}{j!} \cdot t^{j}\right)^{2}= \\
& =\frac{1}{n} \cdot \sum_{r=0}^{n-1}\left(\sum_{j, s=0}^{r} \frac{1}{j!s!} \cdot t^{j+s}\right)= \\
& =\frac{1}{n} \cdot \sum_{r=0}^{n-1} \sum_{i=0}^{2 r} t^{i} \sum_{j+s=i} \frac{1}{j!s!}= \\
& =\frac{1}{n} \cdot \sum_{r=0}^{n-1} \sum_{i=0}^{2 r} \frac{t^{i}}{i!} \sum_{j=0}^{i}\left(\frac{i}{j}\right)= \\
& =\frac{1}{n} \cdot \sum_{r=0}^{n-1} \sum_{i=0}^{2 r} \frac{1}{i!}(2 t)^{i} \geqq \\
& \geqq \frac{1}{n^{2}} \cdot \sum_{j=0}^{n-1} \frac{1}{i!}(2 t)^{i} .
\end{aligned}
$$

For $0<q<1$ we consider $x_{q} \in E$ defined as follows

$$
x_{q}:=\left(q \cdot e_{1}, 2 q^{2} e_{2}, \ldots, n q^{n} e_{n}, \ldots\right)
$$

Then $x_{q} \in D(A)$ and

$$
\begin{aligned}
\left\|T(t) x_{q}\right\|^{2} & =\sum_{n=1}^{\infty} n^{2} q^{2 n}\left\|\exp \left(t A_{n}\right) e_{n}\right\|^{2} \geqq \\
& \geqq \sum_{n=1}^{\infty} n^{2} q^{2 n}\left(\frac{1}{n^{2}} \cdot \sum_{i=0}^{n-1} \frac{1}{i!}(2 t)^{i}\right) \\
& =\sum_{i=0}^{\infty} \sum_{n=i+1}^{\infty}\left(q^{2 n} \cdot \frac{1}{i!}(2 t)^{i}\right) \\
& =\sum_{i=0}^{\infty} q^{2 i+2} \cdot\left(1-q^{2}\right)^{-1} \cdot \frac{1}{i!}(2 t)^{i}= \\
& =\frac{q^{2}}{1-q^{2}} \cdot \sum_{i=0}^{\infty} \frac{1}{i!}\left(2 t q^{2}\right)^{i}=\frac{q^{2}}{1-q^{2}} \cdot e^{2 t q^{2}} .
\end{aligned}
$$

It follows that $w\left(x_{q}\right) \geq q^{2}$. Thus

$$
I=\sup \left\{\omega\left(x_{q}\right): 0<q<1\right\} \leqq \omega_{1}(A) \leqq \omega(A)=1 .
$$

Rescaling the semigroup (i.e. looking at $e^{-3 / 2 \cdot t} T(t)$ ) we obtain a semigroup generator $A$ on the Hilbert space $E$ with $-3 / 2=s(A)$ and $\omega_{1}(A)=\omega(A)=-1 / 2$. On the other hand, Example 1.2.(2) yields a semigroup on a Banach space $F$ with generator $B$ such that $-1=s(B)=\omega_{1}(B)$ while $\omega(B)=0$. Now the operator $C:=A \oplus B$ on the Banach space $E \oplus F$ is a semigroup generator for which\\
$s(C)=\max \{s(A), s(B)\}=-1, \omega_{1}(C)=\max \left\{\omega_{1}(A), \omega_{1}(B)\right\}=-1 / 2$ and $\omega(C)=\max \{\omega(A), \omega(B)\}=0$.\\
(1.7) Important remark: For eventually norm continuous semigroups, in particular for compact, differentiable, holomorphic or nilpotent semigroups the spectral mapping theorem $\sigma(T(t)) \backslash\{0\}=e^{t \sigma(A)}$ holds, and therefore


\begin{equation*}
s(A)=\omega_{1}(A)=w(A) \tag{1.8}
\end{equation*}


is valid (Cor.1.5 and A-III, Cor. 6.7).\\
Hence, if A is the generator of an eventually norm continuous semigroup, then the exponential growth bounds of the strong and the mild solutions of $\dot{u}(t)=A u(t)$ are determined by the spectral bound $s(A)=\sup \{\operatorname{Re} \lambda: \lambda \in \sigma(A)\}$.

In general, the growth bound $w(A)$ can be obtained through the Hille-Yosida theorem (see A-II,Thm.1.7) as\\
(1.9) $\omega(A)=\inf \left(w:\left\|R(\lambda, A)^{n}\right\| \leqq M \cdot(\operatorname{Re} \lambda-w)^{-n}\right.$ for some $M$ and every $n \in \mathbb{N}$ and $\lambda \in \mathbb{C}$ with $\operatorname{Re} \lambda>w\}$.

In view of the difficulties in estimating all powers of the resolvent this equation is of little practical interest. If A is the generator\\
of a semigroup on a Hilbert space $H$, then it is shown in A-III, Cor.7.11 that


\begin{equation*}
w(A)=\inf \left\{w:\|R(\lambda, A)\| \leq M_{w} \text { for } \operatorname{Re} \lambda>w\right\} \tag{1.10}
\end{equation*}


Unfortunately, the identity (1.10) does not hold on arbitrary Banach spaces, but we will see in Section 1 of C-IV that for every positive semigroup on a Banach lattice the identity


\begin{equation*}
s(A)=\omega_{1}(A)=\inf \left\{w:\|R(\lambda, A)\| \leqq M_{w} \text { for Re } \lambda>w\right\} \tag{1.11}
\end{equation*}


is valid. Therefore, for positive semigroups with $s(A)=\omega_{1}(A)<\omega(A)$ (see Ex.1.2.(2)) the equation (1.10) is not true. However, we can prove the following theorem.

Theorem 1.9. Let $A$ be the generator of a strongly continuous semigroup $(T(t))_{t \geq 0}$ on a Banach space $E$. If there are constants $a \geqq 0$ and $q \geqq s(A)$ and if there are $C \in \mathbb{R}_{+}, n \in \mathbb{N}$ such that, for every $\lambda \in \mathbb{C}$ with $\operatorname{Re} \lambda>q$ and $|\operatorname{Im} \lambda|>a$ we have $\|R(\lambda, A)\| \leqq C|\lambda|^{n-2}$, then $\sup \left\{\omega(f), f \in D\left(A^{n}\right)\right\} \leqq q$.

Proof. The hypothesis $\|\mathrm{R}(\lambda, \mathrm{A})\| \leq \mathrm{C}|\lambda|^{\mathrm{n}-2}$ is invariant under rescaling; i.e., the resolvent $\mathrm{R}(\lambda,-\mathrm{b}+\mathrm{A})$ of the generator $-b+\mathrm{A}$ of the rescaled semigroup $e^{-b t} T(t)$ satisfies $\|R(\lambda,-b+A)\| \leqq \tilde{c}|\lambda|^{n-2}$ for every $\lambda \in \mathbb{C}$ with $\operatorname{Re} \lambda>q-b$ and $|\operatorname{Im} \lambda|>a+2 b$ and a suitable constant $\tilde{C}$. Therefore we may assume that $b:=\max (\omega), q)<0$. Let $\omega(A)<\mathrm{p}<0$. Then, by the inversion formula for the Laplace transform for every $f \in D(A)$ and $p^{\prime}=\max \{p, q\}<0$,


\begin{equation*}
T(t) f=\frac{1}{2 \pi i} \cdot \int_{p^{\prime}-i \infty}^{p^{\prime}+i \infty} e^{\lambda t} R(\lambda, A) f d_{\lambda} . \tag{1.12}
\end{equation*}


(For a proof of the vector valued version of the inversion formula one may follow [Widder(1946), p. 66]; also see the notes to this section.) By the resolvent equation we obtain

$$
R(\lambda, A) R(0, A)^{n}=\sum_{k=1}^{n}(-1)^{k+1} \cdot \lambda^{-k} R(0, A)^{n+1-k}+(-1)^{n} \cdot \lambda^{-n} R(\lambda, A)
$$

Using that $\frac{1}{2 \pi i} \cdot \int_{p^{\prime}-i \infty}^{p^{\prime}+i \infty} e^{\lambda t} \cdot \lambda^{-k} d \lambda=0$ for $k \geqslant 1, p^{\prime}<0$ and $t>0$ we obtain


\begin{equation*}
T(t) R(0, A)^{n} f=(-1)^{n} \cdot \frac{1}{2 \pi i} \cdot \int_{p^{\prime}-i \infty}^{p^{\prime}+i \infty} e^{\lambda t} \cdot \lambda^{-n} R(\lambda, A) f d \lambda \tag{1.13}
\end{equation*}


for every $f \in E$ and $t>0$.\\
If $q<p^{\prime}$, then, by Cauchy's Integral Theorem and since $\|R(\lambda, A)\| \leqq$ $C \cdot|\lambda|^{n-2}$ we see that the path of integration can be shifted to $\operatorname{Re}_{\lambda}=\mathrm{q}$;\\
i.e., $T(t) R(0, A)^{n} f=(-1)^{n} \cdot \frac{1}{2 \pi i} \cdot \int_{q-i \infty}^{q+i \infty} e^{\lambda t} \cdot \lambda^{-n} R(\lambda, A) f d \lambda$.

Therefore $\left\|T(t) R(0, A)^{n} f\right\| \leqq c^{\prime} e^{q t}\|f\| \int_{-\infty}^{\infty}\left(q^{2}+s^{2}\right)^{-1} d s=m \cdot e^{q t} \cdot\|f\|$ or $\|T(t) f\| \leq M \cdot e^{q t}\left\|A^{n} f\right\|$ for $f \in D\left(A^{n}\right)$.

In view of the characterizations given in section 1 of A-II, the semigroups occuring in the theorem are holomorphic if $n=1$. In this case one may apply (1.7) to obtain the stronger statement (1.8).

Instead of making assumptions on the resolvent of $A$ we now take a different view and characterize the property "w(A) < 0 " in terms of the semigroup $(T(t)) t \geqq 0$ directly.

Proposition 1.10. Let $A$ be the generator of the strongly continuous semigroup $(T(t))_{t \geqq 0}$. Then the following statements are equivalent:\\
(a) $\omega(A)<0$\\
(b) $\quad \lim _{t \rightarrow \infty}\|\mathrm{~T}(t)\|=0$\\
(c) $\left\|T\left(t^{\prime}\right)\right\|<1$ for some $t^{\prime}>0$.

Proof. The only nontrivial implication $(c) \rightarrow(a)$ follows from $\omega(A)=\lim _{t \rightarrow \infty} \frac{1}{t} \log \|T(t)\| \quad($ see $A-I,(1.1))$ and $\frac{\log \|\mathrm{T}(t)\|}{t} \leqq \frac{\log \left\|\mathrm{~T}\left(t^{\prime}\right)\right\|}{t^{\prime}}+\frac{\log \|T(t)\|}{n t^{\prime}+s}$ for $t=n t^{\prime}+s, s \in\left[0, t^{\prime}\right)$.

Other less obvious characterizations of the property " $\omega(\mathrm{A})$ < 0 " are given in the next theorem. The equivalence of (a) and (c) is known as Datko's Theorem.

Theorem 1.11. Let $A$ be the generator of a strongly continuous semigroup $(T(t))_{t \geq 0}$ on a Banach space $E$. Then the following statements are equivalent:\\
(a) $w(A)<0$.\\
(b) $s(A)<0$ and there is $t_{0}>0$ such that\\
$|\lambda|<1$ for every $\lambda \in \operatorname{Ao}\left(T\left(t_{0}\right)\right)$.\\
(c) For every (some) $p \geq 1$ exists $\int_{0}^{\infty}\|T(t) f\|^{p} d t$ for every $f \in E$.

Proof. The implication " $(a) \rightarrow(b) "$ follows from $x(T(t))=e^{W(A) t}<1$ and $s(A) \leqq \omega(A)<0$. For the point and residual spectrum the spectral mapping theorem is valid (see A-III, Thm.6.3). The approximate point spectrum is closed, hence the additional information in (ii)\\
implies $|\lambda| \leqq r<1$ for some $r$ and each $\lambda \in$ Ao(T( $\left.t_{0}\right)$. Consequently, $\exp \left(\omega(A) \cdot t_{0}=r\left(T\left(t_{0}\right)\right) \leq \max \left\{\exp \left(t_{0} \cdot s(A)\right), r\right)\right\}<1$ or $\omega(A)<0$. This proves " $(b) \rightarrow$ (a)". For a proof of the equivalence of (a) and (c) we refer to Datko (1972) or Pazy (1983),Thm.4.4.1 .

Rescaling a given semigroup $(T(t))_{t \geqslant 0}$ one obtains the following corollary from (1.1) and statement (c) of the above theorem.

Corollary 1.12. Let $(\mathrm{T}(\mathrm{t}))_{t \geqq 0}$ be a strongly continuous semigroup on a Banach space E . Then the set of complex numbers $\lambda$ for which $\int_{0}^{\infty}\left\|e^{-\lambda t} T(t) f\right\| d t$ exists for every $f \in E$ is an open right halfplane.

In the next theorem we give two necessary conditions for stability of (T(t)) $t_{\geqq 0}$ in terms of the generator $A$. We will see in Chapter C-IV that for positive semigroups a condition similar to statement (ii) below is even sufficient for stability of the semigroup. We emphasize that stable semigroups need not be uniformly bounded (see Ex.1.2(3)) and that $s(A)=\omega(A)=0$ does not imply boundedness or even stability of the semigroup (see also A-I, Ex.1.4.(i)).

Theorem 1.13. Let $A$ be the generator of a stable semigroup $\left(T(t){ }_{t} \geqq 0\right.$ on a Banach space $E$. Then the following assertions hold:\\
(i) $\quad s(A) \leqq 0$ and $\operatorname{Re} \lambda<0$ for every $\lambda \in \operatorname{P\sigma }(A) \cup \operatorname{R\sigma }(A)$.\\
(ii) $\lim _{\lambda \rightarrow 0^{+}} \lambda R(\lambda, A) f$ exists for every $f \in D(A)$.

Proof. (i) If (T(t)) ${ }_{t \geqq 0}$ is stable, then $\|\mathrm{T}(t) \mathrm{f}\| \leqq \mathrm{M}_{\mathrm{f}}$ for every $E \in D(A)$. Therefore $s(A) \leqq \omega_{1}(A) \leqq 0$.\\
Assume there is $\lambda \in \mathrm{P} \mathrm{\sigma}(\mathrm{~A})$ with $\operatorname{Re} \lambda=0$. Then by $\mathrm{A}-\mathrm{III}, \mathrm{Cor} \cdot 6,4$ there is $g \neq 0$ such that $T(t) g=e^{\lambda t} g$ for all $t \geqq 0$. Since $\left|e^{\lambda t}\right|=1$ this contradicts the stability of the semigroup.\\
Assume there is $\lambda \in \operatorname{Ro}(A)=\operatorname{po}\left(A^{\prime}\right)$ with $\operatorname{Re} \lambda=0$. Then there is $0 \neq \Phi \in E^{\prime}$ with $T(t)^{\prime} \Phi=\exp (\lambda t) \cdot \Phi$ for all $t \geqq 0$. Choose $f \in D(A)$ such that $\langle\mathcal{E}, \Phi\rangle \neq 0$. Then $|\langle T(t) \pounds, \Phi\rangle|=|\langle f, \phi\rangle|>0$ for every $t \geqq 0$ which contradicts the stability of the semigroup.\\
(ii) From the stability of the semigroup and the identity\\
$\int_{0}^{t} T(s) A f d s=T(t) f-f$ we see that $\int_{0}^{\infty} T(s) A f d s$ exists for every $f \in D(A)$. But $\omega_{1}(A) \leqq 0$ and hence $R(\lambda, A) A f=\int_{0}^{\infty} e^{-\lambda s} T(s) A f d s$ for every $\lambda>0$ (see Thm.1.4). By a classical theorem of Laplace transform theory (for a proof of the vector valued version one may\\
follow Widder (1971), p.196) we conclude that $\lim _{\lambda \rightarrow 0+} R(\lambda, A) A f$ exists and is equal to $\int_{0}^{\infty} T(s) A f$ ds. The identity $R(\lambda, A) A f=\lambda R(\lambda, A) f-f$ yields the existence of $\lim _{\lambda \rightarrow 0+} \lambda R(\lambda, A) f$ for every $f \in D(A)$.

Bounded holomorphic semigroups (see A-II, Def.1.11) satisfy\\
$\|A T(t)\| m \cdot t^{-1}$ [Goldstein (1985a), p.33], hence $T(t) f \rightarrow 0$ as $t \rightarrow \infty$ whenever $f \in \operatorname{im} A$. If im $A$ is dense (i.e., $0 \& \operatorname{Ro}(A)$ ) we obtain uniform stability and the following corollary.

Corollary 1.14. Let $A$ be the generator of a bounded holomorphic semigroup (T(t)) $t \geq 0$ on a Banach space $E$. Then the following statements are equivalent.\\
(a) $0 \notin P_{\sigma}(A) \cup R_{\sigma}(A)$.\\
(b) $(T(t))_{t \geq 0}$ is uniformly stable.

Example 1.15 The Laplacian $\Delta$ generates a bounded holomorphic semigroups on $L^{P}\left(\mathbb{R}^{n}\right)$ for $1 \leqq p<\infty$ (see the example proceeding Cor.1.13 of Chap. A-II). All solutions of $\Delta f=0$ are either constant or unbounded, therefore $0 \& \operatorname{Po}(\Delta)$. If $1<\mathrm{p}<\infty$, then the adjoint of the Laplacian on $L^{p}\left(\mathbb{R}^{n}\right)$ is the Laplacian on $L^{q}\left(\mathbb{R}^{n}\right)$ where $\frac{1}{p}+\frac{1}{q}=1$. Therefore $0 \notin \operatorname{Ro}(\Delta) \cup \operatorname{Po}\left(\Delta^{\prime}\right)$ and we obtain by Cor. 1.14 that $\Delta$ generates uniformly stable semigroups on the space $L^{p}\left(\mathbb{R}^{n}\right)$ for $1<p<\infty$ which are, by $i m \Delta \neq L^{P}\left(\mathbb{R}^{n}\right)$ and cor.1.5, not exponentially stable.

As seen in Thr.1.4, exponential stability can be defined by saying that the abscissa of convergence of the Laplace transform of (T) $(t) t_{t \geq 0}$ is less than zero. This should be compared to the assertion of our final theorem.

Theorem 1.16. Let $A$ be the generator of a strongly continuous semigroup $(T(t))_{t \geq 0}$ on a Banach space $E$. The following assertions are equivalent:\\
(a) $\quad(\mathrm{T}(\mathrm{t}))_{t \geq 0} \quad$ is stable.\\
(b) $\quad \operatorname{ker} A=\{0\}$ and $\int_{0}^{\infty} T(t) f d t$ exists for all $f \in$ im $A$.

Furthermore the following statements are equivalent:\\
(a') $(T(t)) t \geq 0$ is stable and bounded.\\
(b') $(T(t)) t \geqq 0$ is uniformly stable.\\
$\left(c^{\prime}\right) \quad(T(t))_{t \geqq 0}$ is bounded and there is a dense subspace $D$ such

that $\quad \int_{0}^{\infty} T(t) f d t \quad$ exists for every $f \in D$.

Proof. If $(T(t)\}_{t \geq 0}$ is stable, then, by Thm.1.12, ker $A=\{0\}$ and $\int_{0}^{t} \mathrm{~T}(s) A f d s=T(t) f-f \rightarrow-f$ as $t \rightarrow \infty$. Therefore (a) implies (b). on the other hand, if $\int_{0}^{t} \mathrm{~T}(\mathrm{~s}) \mathrm{Af}$ ds converges as $t \rightarrow \infty$, then, by the equation above, $g:=\lim _{t \rightarrow \infty} T(t) \pounds$ exists. But ker $A=\{0\}$ and therefore $g=0$. This proves " $(b) \rightarrow(a) "$.\\
The implication "(a') $+\left(b^{\prime}\right) "$ is obvious. If $T(t) f \rightarrow 0$ for every $\mathrm{f} \in \mathrm{E}$, then $\|\mathrm{T}(\mathrm{t})\| \leq \mathrm{M}$ and $0 \notin \operatorname{Ro}(\mathrm{~A})$ (Thm.1.12). Therefore $D:=\operatorname{im} A$ is dense and $\int_{0}^{\infty} T(t) f d t$ exists for every $f \in D$. This proves "(b') $\rightarrow$ ( $\left.c^{\prime}\right) "$ " We have to show that ( $c^{\prime}$ ) implies ( $a^{\prime}$ ). Define $G:=\left\{h \in E: h=\int_{0}^{\infty} T(t) g\right.$ dt for some $\left.g \in D\right\}$. We will show that $G$ is dense in $E$. First we notice that $g-T(s) g \in D$ whenever $g \in D$ and $s \in \mathbb{R}_{+}$.\\
Define $h_{s}=\frac{1}{s} \cdot \int_{0}^{\infty} T(t)(g-T(s) g) d t=\frac{1}{s} \cdot \int_{0}^{s} T(t) g d t$. Then $h_{S} \in G$ and $h_{s}+g$ as $s \rightarrow 0$. Therefore $D \subset \bar{G}$ or $\bar{G}=E$. Now let $h \in G$. Then $T(t) h=T(t) \int_{0}^{\infty} T(s) g d s=\int_{t}^{\infty} T(s) g d s \rightarrow 0$ as $t \rightarrow \infty$. But $\|T(t)\| \leqq M$ and therefore $T(t) E \rightarrow 0$ for every $f \in E$.

Remarks 1.17. (a) If $A$ is the generator of a stable semigroup $(T(t))_{t \geq 0}$ on a Banach space $E$, then, by the previous theorem, im $A \subset\left\{f \in E: \int_{O}^{\infty} T(t) f d t\right.$ exists\} $=$ : A .\\
If $g \in H$, then $\int_{0}^{\infty} T(t) g d t \in D(A)$ and $A \int_{0}^{\infty} T(t) g d t=-g$. Therefore $g \in$ im $A$ and we obtain that the dense subspace im $A$ is given by


\begin{equation*}
\operatorname{im} A=\left\{f \in E: \int_{0}^{\infty} T(t) f d t \text { exists }\right\} \tag{1.14}
\end{equation*}


in case that $A$ is the generator of a stable semigroup $(T(t))_{t \geqq 0}$.\\
(b) If $\omega(f)<0$ for every $f \in D(A)$, then (T(t)) is stable (but might not be exponentially stable if\\
$0=\omega_{1}(A)=\sup \{\omega(f): f \in D(A)\}$. In this case it can be seen by a proof similar to the one of Thm.1.4, that $\sigma(\mathrm{A})$ has to be contained in the open left half-plane; i.e. $\operatorname{Re} \lambda<0$ for $\lambda \in \sigma(A)$.\\
(c) If one defines a semigroup (T( $t)_{t \geqq 0}$ to be weakly stable if $\langle T(t) \pounds, \Phi\rangle \rightarrow 0$ as $t \rightarrow \infty$ for every $f \in D(A)$ and $\Phi \in E^{\prime}$ or as weakly uniformly stable if $\langle T(t) f, \phi\rangle \rightarrow 0$ as $t \rightarrow \infty$ for every $f \in E$ and $\Phi \in \mathrm{E}^{\prime}$, then Theorem 1.13 and 1.16 can be reformulated in a\\
weak form (i.e.; we replace stable by weakly stable and 'lim' by 'weak-lim'). The proofs require only some obvious modifications.\\
If $A$ has a compact resolvent or if $A$ is the generator of a bounded holomorphic semigroup, then weak stability implies stability. In general, this is no longer true; e.g., the translation semigroup on $\mathrm{L}^{2}(\mathbb{R})$ is weakly uniformly stable but not stable (see also B-IV,EX. 1.2).

\section*{2. STABILITY: INHOMOGENEOUS CASE}
Using the results of the first section, we now investigate the long term behavior of the solutions of the inhomogeneous initial value problem


\begin{equation*}
\dot{u}(t)=A u(t)+F(t), \quad u(0)=f \tag{2,1}
\end{equation*}


where $A$ is the generator of a strongly continuous semigroup on a Banach space $E$ and $F(\cdot)$ is a locally integrable function from $\mathbb{R}_{+}$ into E called forcing term. A function u(•) is called a (strong) solution of (2.1) if $u(\cdot): \mathbb{R}_{+} \rightarrow D(A), u(\cdot) \in C^{1}\left(\mathbb{R}_{+}, E\right)$ and (2.1) is satisfied for $t \geqq 0$.\\
The assumption that $A$ is the generator of a semigroup $(T(t))_{t \geq 0}$ yields the uniqueness of the solution of (2.1). If u(•) is a solution of $(2.1)$, then the function $v(s):=T(t-s) u(s), 0 \leqq s \leqq t$, is differentiable and $\mathrm{V}^{\prime}(\mathrm{s})=\mathrm{T}(\mathrm{t}-\mathrm{s}) \mathrm{F}(\mathrm{s})$. But $\mathrm{F}(\cdot)$ is locally integrable, and by $\int_{0}^{t} T(t-s) F(s) d s=v(t)-v(0)=u(t)-T(t) f$ we see that the solution $u(t)$ of (2.1) is given by


\begin{equation*}
u(t)=T(t) \pounds+\int_{0}^{t} T(t-s) F(s) d s . \tag{2.2}
\end{equation*}


Example. Let (T( $)_{t \geq 0}$ be not eventually differentiable. Then there exists $g \in E$ such that $t \rightarrow T(t) g$ is not differentiable on ( $0, \infty$. The initial value problem $\dot{u}(t)=A u(t)+T(t) g, u(0)=0$ has no (strong) solution u(•) because otherwise

$$
u(t)=\int_{0}^{t} T(t-s) T(s) g d s=t T(t) g
$$

has to be differentiable on $\mathbb{R}_{+}$.\\
Whenever the expression (2.2) makes sense we call it a generalized (or mild) solution of (2.1). If $F(\cdot)$ is continuous and $f \in D(A)$, then the generalized solution of (2.1) is a strong solution if and only if $v(t):=\int_{0}^{t} T(t-s) F(s) d s$ is differentiable (see pazy (1983) Chap.4,

Thm.2.4). There are several sufficient conditions on the generator A, the forcing term $F(\cdot)$ or the space $E$ such that every mild solution is a strong solution of (2.1) (see Travis (1979) or Pazy (1983) Sec.4.2).

It is our aim in this section to study the asymptotic behavior of the solutions of (2.1) as $t \rightarrow \infty$. To that purpose we consider absolutely integrable or periodic forcing terms $F(\cdot)$, and assume the semigroup to be uniformly stable.\\
Similar results for integrable and convergent forcing terms F(.) can be obtained if the semigroup is supposed to be uniformly exponentially stable (see Pazy (1983), p. 119 or Neubrander (1985b)). However, if the semigroup is positive, these results even hold for stable semigroups (see Section C-IV). From Theorem 1.13.(i) we recall that for stable semigroups im $A$ is dense in E.

Theorem 2.1. Let $A$ be the generator of a uniformly stable semigroup $(T(t))_{t \geq 0}$ on a Banach space $E$. If there is $g \in$ im $A$ such that $\int_{0}^{\infty}\|F(s)-g\|^{\prime} d s$ exists, then every generalized solution u(-) of (2.1) converges as $t \rightarrow \infty$ and $\lim _{t \rightarrow \infty} u(t)=-h$ where $h \in D(A)$ with $\mathrm{Ah}=\mathrm{g}$.

Proof. If $u(\cdot)$ is a generalized solution of (2.1), then, by (2.2), $\overline{u(t)}=\mathrm{T}(t) \mathrm{f}+\int_{0}^{t} \mathrm{~T}(\mathrm{~s}) \mathrm{g} \mathrm{ds}+\int_{0}^{t} \mathrm{~T}(t-\mathrm{s})(\mathrm{F}(\mathrm{s})-\mathrm{g}) \mathrm{ds}$. By the uniform stability and Thm.1.14 we see that the first term converges to zero and that the second one converges to -h . We have to show that the third term converges to zero. Take $\varepsilon>0$ and $G(s):=F(s)-g$. Then

$$
\begin{aligned}
\left\|\int_{0}^{t} T(t-s) G(s) d s\right\| & \leqq\left\|\int_{0}^{t^{\prime}} T\left(t-t^{\prime}+t^{\prime}-s\right) G(s) d s\right\|+\left\|\int_{t^{\prime}}^{t} T(t-s) G(s) d s\right\| \\
& \leqq\left\|T\left(t-t^{\prime}\right) \int_{0}^{t^{\prime}} T\left(t^{\prime}-s\right) G(s) d s\right\|+M \int_{t^{\prime}}^{\infty}\|G(s)\| d s .
\end{aligned}
$$

Since the semigroup is uniformly stable we obtain\\
$T\left(t-t^{\prime}\right) \int_{0}^{t^{\prime}} T\left(t^{\prime}-s\right) G(s) d s \rightarrow 0$ as $t \rightarrow 0$ for every $t^{\prime} \geqq 0$. Therefore $\left\|\int_{0}^{t} T(t-s) G(s) d s\right\| \leq \varepsilon$ for all sufficiently large $t$.

In the subsequent theorem we see that if $A$ is the generator of a uniformly stable semigroup, if the forcing term $F(\cdot)$ is p-periodic and if $\int_{0}^{p} T(p-s) F(s) d s \in i m$ (Id - $T(p)$ ) (notice that, by Thm.1.13. (i) and A-III, Lemma 5.3, $\overline{\text { im(Id - T(p) }}=\mathrm{E}$ ), then (2.1) admits a unique p-periodic, asymptotically stable mild solution.

Lemma 2.2. Let $A$ be the generator of a strongly continuous semigroup $\left(T(t) t_{t \geqq 0}\right.$ on a Banach space $E$ and let $F(\cdot)$ be a p-periodic, locally integrable function, $p>0$. Then the following statements are equivalent:\\
(a) $\dot{\mathrm{u}}(t)=\mathrm{Au}(t)+\mathrm{F}(t)$ admits a (unique) generalized p-periodic solution.\\
(b) There exists a (unique) $f \in E$ such that $(I d-T(p)) f=$ $\int_{0}^{p} T(p-s) F(s) d s$.

Proof. " $(a) \rightarrow(b) "$ Let $f:=u(0)$ be the initial value for which (2.1) has the p-periodic solution. Then we have

$$
\begin{aligned}
u(t)=u(t+p) & =T(t) T(p) f+\int_{0}^{p} T(t+p-s) F(s) d s+\int_{p}^{t+p} T(t+p-s) F(s) d s \\
& =T(t)\left[T(p) f+\int_{0}^{P} T(p-s) F(s) d s\right]+\int_{0}^{t} T(t-s) F(s) d s
\end{aligned}
$$

for every $t \geqq 0$. Therefore $f=u(0)=T(p) f+\int_{0}^{p} T(p-s) F(s) d s$. Clearly, if $u(\cdot)$ is a unique periodic solution with $u(0)=f$, then $f$ is the unique element for which $f=T(p) f+\int_{0}^{p} T(p-s) F(s) d s$ holds.\\
$"(b) \rightarrow(a) "$ Define u(.) as in (2.2). Then\\
$u(t+p)=T(t)\left[T(p) f+\int_{0}^{p} T(p-s) F(s) d s\right]+\int_{0}^{t} T(t-s) F(s) d s=u(t)$.

If $f$ is unique, then, by the considerations above, the solution is unique.

Remark 2.3. Let $A$ be the generator of a strongly continuous semigroup (T(t)) $t \geqq 0$ for which the spectral mapping theorem (see A-III, sec. 6 ) is valid and let $F$ be a p-periodic forcing term.\\
If $\frac{2 \pi i n}{p} \in \rho(A)$ for every $n \in \mathbf{Z}$, then, by Lemma 2.2 , $\dot{u}(t)=A u(t)+F(t)$ has a unique p-periodic solution with initial value $(I d-T(p))^{-1}\left(\int_{0}^{p} T(p-s) F(s) d s\right)$.

As a consequence of Thm.1.13 and A-III, Cor.6.4, for a uniformly stable semigroup there exists at most one $f \in E$ such that $(I d-T(p)) E=\int_{0}^{p} T(p-s) F(s) d s$. This and Lemma 2.2 is used to prove the following theorem.

Theorem 2.4. Let A be the generator of a uniformly stable semigroup $(T(t))_{t \geq 0}$ and let $F(\cdot)$ be a p-periodic locally integrable function such that $(I d-T(p)) f=\int_{O}^{P} T(p-s) F(s) d s$ for some $f \in E$. Then the\\
unique p-periodic generalized solution

$$
u(t)=T(t) f+\int_{0}^{p} T(p-s) F(s) d s
$$

is asymptotically stable; i.e., for every generalized solution v(.)\\
of (2.1) we have $\lim _{t \rightarrow \infty}\|v(t)-u(t)\|=0$.

Example 2.5. Let $E$ be the Banach space $C_{O}\left(\mathbb{R}_{+}\right) ; A=\frac{d}{d x}$ with $D(A)=\left\{f \in E: E^{\prime} \in C^{1}\right.$ and $\left.f^{\prime} \in E\right\}$ is the generator of the uniformly stable translation semigroup $T(t) f(x):=f(t+x)$. Applying (1.14) we obtain that im $A=\left\{f: \int_{0}^{\infty} f(x) d x\right.$ exists $\}$ is dense in $C_{0}\left(\mathbb{R}_{+}\right)$. Let $r \in$ im $A$ and let $F(\cdot)$ be a p-periodic real-valued function.\\
We apply Theorem 2.4 to the initial value problem


\begin{equation*}
\frac{d}{d t} u(t, x)=\frac{d}{d x} u(t, x)+r(x) F(x+t), u(0, \cdot) \in D(A) \tag{*}
\end{equation*}


We may rewxite (*) as


\begin{equation*}
\dot{v}(t)=A v(t)+G(t) \tag{**}
\end{equation*}


where $v(t)=u(t, \cdot)$ and $G: \mathbb{R}_{+} \rightarrow E$ is defined by

$$
G(t)(x)=r(x) F(x+t)
$$

$G$ is p-periodic with values in $E$ and $h_{0}:=\int_{0}^{p} T(p-t) G(t) d t$ is the function $x \rightarrow\left[\int_{0}^{p} T(p-t) G(t) d t\right](x)=F(x) \int_{x}^{x+p} r(s) d s$. For the function $f=\sum_{k=0}^{\infty} T(k p) h_{0}$, which is given by $x \rightarrow F(x) \int_{x}^{\infty} r(s) d s$, we clearly have (Id - T(p)) $\mathbf{f}=h_{0}$. Therefore (**) has a unique p-periodic generalized solution (Thm.2.4) although iR $\in \sigma(A)$ (compare with Remark 2.3).\\
The unique p-periodic generalized solution u(t,.) is given by $u(t, x)=F(x+t) \int_{x+t}^{\infty} r(s) d s+F(x+t) \int_{x}^{x+t} r(s) d s=F(x+t) \int_{x}^{\infty} r(s) d s$. For every solution $v(t, \cdot)$ of (*) we have, by Thm.2.4:\\
$\sup \left\{\left|v(t, x)-F(x+t) \int_{x}^{\infty} r(s) d s\right|: x \in \mathbb{R}_{+}\right\} \rightarrow 0$ as $t \rightarrow 0$.

NOTES.\\
Section 1. The exponential growth bounds $\omega(f)$ and $\omega(\mathrm{A})$ as well as the characterizations (1.2), (1.6) and Theorem 1.3 (i) can be found in Hille-Phillips (1957). Growth bounds similar to $\omega_{1}$ (A) were considered first in [D'Jacenko (1976)] and in [Zabczyk (1979), Prop.2]. Example 1.2.(2) is taken from Wolff (1981); other 'counterexamples' can be found in Hille-Phillips (1957), Foias (1973), Triggiani (1975), Zabczyk (1975) and Greiner-Voigt-Wolff (1981). The statements (1.2), (1.6) and Theorem 1.3.(i) are semigroup versions of results of classical Laplace transform\\
theory, see Hille-Phillips (1957) and Widder (1946). Theorem 1.3. (ii) is a semigroup version of Theorem 1.2 .7 and 1.2 .8 in Doetsch (1950). The lema in the proof of Thm. 1.3 is taken from Mil'stein (1975). Theorem 1.4 and Corollary 1.5 can be found in Neubrander (1985a). Example 1.6 follows Remark 2 in Zabczyk (1975). Statement ( 1.8 ) is sometimes called the 'spectrum determined growth assumption', see, for example, Triggiani (1975b). Theorem 1.9 is due to Slemrod (1976). The proof given here is based on a much sharper version of the inversion formula for the Laplace transform, than the one given by Hille-Phillips (1957), p.349. Using Widder (1946), p. 66 or Doetsch (1950), p. 212 one can show the following theorem (see Neubrander (1984b)).

Theorem. Let $A$ be the generator of a strongly continuous semigroup $(T(t)) t \geq 0$ on a Banach space E. For every $f \in D(A)$ and $p>\omega_{1}(A)$ we have

$$
T(t) f=\frac{1}{2 \pi i} \int_{p+i^{\infty}}^{p+i^{\infty}} e^{\mu t} R(\mu, A) f d \mu .
$$

The equivalence of the statements (1.12), (1.13) and ' $\omega$ (A) < 0 ' were observed by many authors, see for example, Balakrishnan (1976), p.178 or Benchimol (1978). Theorem 1.11 is due to Datko (1970) and Delfour (1974); for a proof see Pazy (1983), p. 116. Theorems 1.13 and 1.16 can be found in Neubrander (1985b) and Corollary 1.14 is due to Komatsu (1969). An example of an unstable semigroup generator A with Re $\mu<0$ for all $\mu \in \sigma(\mathrm{A})$ is given in Datko (1983).

Section 2. For a discussion of well-posedness of inhomogeneous Cauchy problems we refer to Goldstein (1985a), p.83 and Pazy (1983), p.105. Further results on the asymptotic behavior of the solutions of the inhomogeneous problem can be found in Rao-Hengartner (1974), Zaidman (1979), Pazy (1983), and Neubrander (1985b). Results similar to Lemma 2.2 and Theorem 2.4 are due to Priiß (1984). For a discussion of the asymptotic behavior of the solutions of $u^{\prime}(t)=A(t) u(t)+F(t)$ see Datko (1972) and Pazy (1983), p. 172 .

\section*{POSITIVE SEMIGROUPS ON $C_{0}(X)$ }
CHAPTER B-I\\
\includegraphics[max width=\textwidth, center]{2024_12_23_c6487cc0859199a15bd9g-127}\\
by\\
Rainer Nagel and Ulf Schlotterbeck

This part, of the book is devoted to a study of one-parameter semigroups of operators on spaces of continuous functions of type $C_{0}(X)$, spaces which are Banach lattices of a very special kind. We treat this case separately since we feel that an intermingling with the abstract Banach lattice situation considered in Part $C$ would have made it difficult to appreciate the easy accessibility and the pilot function of methods and results available here. In this chapter we want to fix the notation we are going to use and to collect some basic facts about the spaces we are considering.

If $X$ is a locally compact topological space, then $C_{0}(X)$ denotes the space of all continuous complex-valued functions on x which vanish at infinity, endowed with the supremum-norm. If X is compact, then any continuous function on $X$ "vanishes at infinity" and $C_{0}(X)$ is the space of all continuous functions on $X$. We often write $c(X)$ instead of $C_{0}(X)$ in this situation. We sometimes study real-valued functions and write the corresponding real spaces as $C_{0}(X, \mathbb{R})$ and $C(X, \mathbb{R})$, and the notations $C_{0}(x, \mathbb{C})$ and $C(x, \mathbb{C})$ are used if there is the possibility of confusion between both cases.

\section*{1. ALGEBRAIC AND ORDER-STRUCTURE; IDEALS AND QUOTIENTS}
Any space $C_{0}(X)$ is a commutative $C^{*}$-algebra under the sup-norm and the pointwise multiplication, and by the Gelfand Representation Theorem any commutative c*-algebra can, on the other hand, be canonically represented as an algebra $C_{0}(x)$ on a suitable locally compact space x . The algebraic notions of ideal, quotient, homomorphism are\\
well known and need not be explained further. Another natural and important structure of $C_{0}(x)$ is the pointwise ordering, under which $C_{0}(X, \mathbb{R})$ is a (real) Banach lattice and $C_{0}(X, \mathbb{C})$ a complex Banach lattice in the sense explained in Chapter C-I. Concerning the order structure of $C_{0}(x)$ we use the following notations: For a function $f$ in $C_{0}(x, \mathbb{R})$

\begin{verbatim}
$f \geqq 0$ means $f(t) \geqq 0$ for all $t \in X$ ( $f$ is then called
    positivel,
$\mathbf{f}>0$ means that E is positive but does not vanish iden-
    tically,
$f \gg 0$ means that $f(t)>0$ for all $t$ in $X$ ( $f$ is then
    called strictly positive).
\end{verbatim}

The notion of an order ideal explained in Chapter $\mathrm{C}-\mathrm{I}$ applies of course to the Banach lattices $C_{0}(X)$ and might give rise to confusion with the corresponding algebraic notion. However, since we are mainly considering closed ideals and since a closed linear subspace I of $C_{0}(X)$ is a lattice ideal if and only if $I$ is an algebraic ideal, we may and will simply speak of closed ideals without specifying whether we think of the algebraic or the order theoretic meaning of this word. An important and frequently used characterization of these objects is the following: A subspace $I$ of $C_{O}(X)$ is a closed ideal if and only if there exists a closed subset $A$ of $X$ such that a function $f$ belongs to I if and only if $f$ vanishes on A . A is of course uniquely determined by I and is called the support of I . If $I=I_{A}$ is a closed ideal with support $A$ then $I_{A}$ is naturally isomorphic to $C_{0}(X \backslash A)$ and the quotient $C_{O}(X) / I$ is lunder the natural quotient structurel again a Banach algebra and a Banach lattice that can be identified canonically (via the map $f+I \rightarrow f_{\mid A}$ ) with $C_{O}(A)$.

\section*{2 . LINEAR FORMS AND DUALITY}
The Riesz Representation Theorem asserts that the dual of $C_{0}(X)$ can be identified in a natural way with the space of bounded regular Borel measures on $X$. While there is no natural algebra structure on this dual, the dual ordering (see $\mathrm{C}-\mathrm{I}$ ) makes $\mathrm{C}_{\mathrm{O}}(\mathrm{X})^{\text {' into a Banach lat- }}$ tice. We will occasionally make use of the order structure of $C_{0}(\mathrm{X})^{\prime}$ but since at least its measure theoretic interpretation is supposed to be well-known, it may suffice here to refer to Chapter C-I, Sections 3\\
and 7 , for a more detailed discussion and to recall some basic notations here: If $\mu$ is a linear form on $\mathrm{C}_{\mathrm{O}}(\mathrm{X}, \mathbb{R})$ then

\begin{verbatim}
\mu\geqq0 means \mu(f) \ 0 for all f \geqq \ ; \ is then called
    positive (positivity automatically implies continuity),
\mu> 0 means that }\mu\geq0\mathrm{ , but }\mu\mathrm{ does not vanish identically,
\mu >> 0 means that }\mu\mathrm{ (f) > 0 for any f > 0 ; , is then called
    strictly positive.
\end{verbatim}

If $\mu$ is a linear form on $C_{O}(x, \mathbb{C})$, then $\mu$ can be written uniquely as $\mu=\mu_{1}+i \mu_{2}$ where $\mu_{1}$ and $\mu_{2}$ map $C_{0}(X, \mathbb{R})$ into $\mathbb{R}$ (decomposition into real and imaginary parts). We call $\mu$ positive (strictly positive) and use the above notations if $\mu_{2}=0$ and $\mu_{1}$ is positive (strictly positive). We point out that strictly positive linear forms need not exist on $C_{0}(X)$, but if $X$ is separable then a strictly positive linear form is obtained by suming a suitable sequence of point measures.

The coincidence of the notions of a closed algebraic and a closed lattice ideal in $C_{0}(X)$ has of course its effect on the algebraic and the lattice theoretic notions of a homomorphism. The case of a homomorphism into another space $C_{0}(Y)$ will be discussed below. As for homomorphisms into the scalar field, we have essentially coincidence between the algebraic and the order theoretic meaning of this word, more exactly: A linear form $\mu \neq 0$ on $C_{0}(x)$ is a lattice homomorphism if and only if $\mu$ is, up to normalization, an algebra homomorphism (algebra homomorphisms $\neq 0$ must necessarily have norm 1 ). Since the algebra homomorphisms $\neq 0$ on $c_{0}(x)$ are known to be the point measures (denoted by $\delta_{t}$ ) on $X$ and since on the other hand $\mu$ is a lattice homomorphism if and only if $|\mu(f)|$ equals $\mu(|f|$ ) for all f , it follows that this latter condition on $\mu$ is equivalent to $\mu=\alpha \delta_{t}$ for a suitable $t$ in $X$ and a positive real number $\alpha$. This can be summarized by saying that $X$ can be canonically identified, via the map $t \rightarrow \delta_{t}$, with the subset of the dual $C_{0}(x){ }^{\prime}$ consisting of the non-zero algebra homomorphisms, which is also the set of all normalized lattice homomorphisms. This identification is a topological isomorphism with respect to the weak*-topology of $\mathrm{C}_{\mathrm{O}}(\mathrm{X})^{\prime}$.

\section*{3. IINEAR OPERATORS}
A linear mapping $T$ from $C_{0}(X, \mathbb{R})$ into $C_{0}(Y, \mathbb{R})$ is called\\
positive (notation: $T \geqq 0$ ), if $T f$ is a positive function whenever f is positive,\\
a lattice homomorphism if $|\mathrm{Tf}|=\mathrm{T}|\mathrm{f}|$ for all f ,\\
a Markov-operator if X and Y are compact and T is a positive operator mapping $1_{X}$ to $I_{Y}$.

In the case of complex scalars $T$ can be decomposed into real and imaginary parts. We call $T$ positive in this situation if the imaginary part of T is $=0$ and the real part is positive. The terms "Markov operator" and "lattice homomorphism" are defined formally in the same way as above. Note that a complex lattice homomorphism is necessarily positive, and that the complexification of a real lattice homomorphism is a complex lattice homomorphism. Positive Operators are always continuous.

Since the adjoint of a Markov operator $T$ maps positive normalized measures into positive normalized measures while the adjoint of an algebra homomorphism (lattice homomorphism) maps point measures into (multiples of) point measures, the adjoint of a Markov lattice homomorphism as well as the adjoint of an algebra homomorphism induces a continuous map $\phi$ from $Y$ (viewed as a subset of the weak dual $C(Y)^{\prime}$ ) into $X$ (viewed as a subset of $C(X)^{\prime}$ ). This mapping $\phi$ determines $T$ in a natural and unique way, so that the following are equivalent assertions on a linear mapping $T$ from a space $C(X)$ into a space $C(Y)$ :\\
(a) T is a Markov lattice homomorphism\\
(b) T is a Markov algebra homomorphism\\
(c) There exists a continuous map $\phi$ from Y into X such $T f=f \circ \phi$ for all $f \in C(x)$.

If $T$ is in addition bijective, then the mapping $\phi$ in (c) is a homeomorhism from X onto X . This characterization of homomorphisms carries over mutatis mutandis to situations where the conditions on $\mathrm{X}, \mathrm{Y}$ or T are less restrictive. For later reference we explicitly state:\\
(i) Let $K$ be compact. Then $T \in L(C(K))$ is a lattice homomorphism if and only if there is a mapping $\phi$ from $K$ into K and a function\\
$h \in C(K)$ such that $T f(s)=h(s) f(\phi(s))$ holds for all $s \in K$. $\phi$ is continuous in every point $t$ with $h(t) \neq 0$.\\
(ii) Let X be locally compact, $\mathrm{T} \in L\left(\mathrm{C}_{\mathrm{O}}(\mathrm{X})\right)$. T is a lattice isomorphism if and only if there is a homeomorhism $\phi$ from X onto X and a bounded continuous function h on X such that $h(s) \geq \delta>0$ for all $s$ and $T f(s)=h(s) f(\phi(s))(s \in X$ ).T is an algebraic *-isomorphism if and only if $T$ is a lattice isomorphism and the function $h$ above is $\equiv 1$.

\section*{CHARACTERIZATIONOF POSITIVE }
SEMI GROURS ON C $\mathrm{C}_{\mathrm{O}}(\mathrm{X})$\\
by\\
Wolfgang Arendt

It lies in the very nature of the theory of one-parameter semigroups that frequently an operator $A$ is known to be a generator but the semigroup is not known explicitly. Thus, since the semigroup is uniquely determined by the generator, it is a central task in the theory to express properties of the semigroup in terms of its generator. In this chapter we do this for two properties. We characterize generators of positive semigroups and generators of lattice semigroups.

In section $I$ we consider a semigroup $(T(t))_{t \geq 0}$ on the real space C(K) (K compact). It is shown that the semigroup consists of positive operators if and only if its generator satisfies a positive minimum principle (P). Even without assuming a priori that $A$ is a generator the positive minimum principle has strong consequences. Together with a range condition it implies that A is a generator (of a positive semigroup). Moreover, we show that for a densely defined operator A to be the generator of a positive semigroup it is already sufficient that the resolvent $\mathrm{R}(\lambda, \mathrm{A})$ of A exists and is positive for all sufficiently large real $\lambda$. For all these results it is essential to assume that K is compact. Concerning the characterization of positive semigroups on $C_{0}(X)$ ( $X$ locally compact, non-compact) we follow a completely different line and will treat this case in the context of general Banach lattices in Chapter C-II.

A special class of positive semigroups are lattice semigroups; i.e., semigroups of lattice homomorphisms. We show in section 2 that $(T(t))_{t \geqq 0}$ is a lattice semigroup if and only if its generator A satisfies an identity (K), the so-called Kato's equality (Theorem 2.5). We refer to Chapter C-II for a discussion of this identity and classical analogs for the Laplacian due to Kato (1973).

After the abstract characterization in section 2 we show that every continuous semiflow on X together with a cocycle defines a lattice semigroup in a canonical way, and on $C(K)$, every lattice semigroup can be so represented. This furnishes a wide class of examples. Furthermore, positive one-parameter groups on $C_{0}(x)$ (which form a particular type of lattice semigroups) are discussed. Their generators are similar to a derivation perturbated by a multiplication operator (Section 3).

\section*{1. Generators of Positive Semigroups on $C(K)$.}
Let $X$ be a locally compact space. Throughout this section we denote by $c_{0}(x)$ the space of all real-valued continuous functions on $c_{0}(X)$ which vanish in infinity. Recall that a semigroup $\left(T(t){ }_{t \geq 0}\right.$ on $c_{0}(\mathrm{X})$ is called positive if $\mathrm{T}(t) \geqq 0$ for all $t \geqq 0$. It is easy to describe the positivity of $(T(t))_{t \geqq 0}$ in terms of the resolvent $R(\lambda, A)$ of its generator A because of the close relation between these two objects. In fact, the resolvent is expressed by the semigroup by\\
(1.1) $R(\lambda, A)=\int_{0}^{\infty} e^{-\lambda t} T(t) d t \quad(\lambda>\omega(A))$;\\
and conversely, the semigroup by the resolvent via the formula\\
(1.2) $T(t)=\lim _{n \rightarrow \infty}(n / t R(n / t, A))^{n} \quad$ strongly\\
(cf. A-II, Prop.1.10). So we obtain the following.

Proposition 1.1. Let $(T(t))_{t \geq 0}$ be a semigroup with generator $A$. The semigroup is positive if and only if $R(\lambda, A) \geq 0$ for all sufficiently large real $\lambda$.

It is more difficult and more interesting to characterize the positivity of the semigroup by intrinsic conditions on the generator. This is the purpose of this section. As a first orientation we consider bounded generators. We need the following lemma.

Lemma 1.2. Let $X$ be a locally compact space, $x \in X$ and $\mu$ a regular bounded Borel measure on $X$ such that $\mu(\{x\})=0$. Then $\mu \geqq 0$ if and only if $\langle f, \mu\rangle \geqq 0$ for all $\pounds \in C_{O}(X){ }_{+}$satisfying $\mathrm{f}(\mathrm{x})=0$.

We omit the easy proof.

Theorem 1.3. Let $X$ be locally compact and $A$ be a bounded operator on $c_{0}(\mathrm{X})$. The following assertions are equivalent.\\
(i) $e^{t A} \geqq 0 \quad(t \geqq 0)$.\\
(ii) For $0 \leqq f \in C_{0}(x)$ and $x \in X$, $f(x)=0$ implies $(A f)(x) \geqq 0$.\\
(iii) $A+\|A\| I d \geqq 0$.

Proof. (i) implies (ii). Let $f \in C_{0}(X)+$ and $x \in X$ such that $f(x)=0$. Then

$$
\begin{aligned}
(\operatorname{Af})(x) & =\lim _{t \rightarrow 0} 1 / t\left(\left(e^{t A_{f}}(x)-f(x)\right)\right. \\
& =\lim _{t \rightarrow 0} 1 / t\left(\left(e^{t A_{f}}(x)\right) \geqslant 0\right.
\end{aligned}
$$

(ii) implies (iii). Let $x \in X$. We have to show that (Af)(x) + $\|A\| f(x) \geqq 0$ for all $f \in C_{0}(x)$. Let $A^{\prime} \delta_{x}=\mu+c_{x}^{\delta}$ where $\mu \in M(X)$ such that $\mu(\{x\})=0$ and $c \in \mathbb{R}$. We claim that $\mu \geqq 0$. Let $0 \leqq$ $f \in C_{O}(X)$ such that $f(x)=0$. Then $\langle f, \mu\rangle=\left\langle f, A^{\prime} \delta_{x}\right\rangle=(A f)(x) \geqq 0$ by (ii). Thus $\mu \geqq 0$ by Lemma 1.1. Moreover, $|c|=\left\|c^{\delta}{ }_{x}\right\|\left\|_{c}{ }_{x}+\mu\right\|$ $=\left\|A^{\prime} \delta_{x}\right\| \leqq\|A\|$. Hence, for $f \in C_{0}(X)+, \quad(A f)(x)+\|A\| f(x)=<f$, $\left.A^{\prime} \delta_{x}+\|A\| \delta_{x}\right\rangle=\left\langle f, \mu+(c+\|A\|) \delta_{x} \gg 0\right.$. This shows (ii) to hold. (iii) implies (i). We have $e^{t A}=e^{-t\|A\|} e^{t(A+\|A\|)} \geqq e^{-t\|A\|}$ Id for all $t \geqq 0$.

Example 1.4. a) Let $B$ be a positive operator on $C_{0}(X)$ and $m$ : $X$ $\rightarrow \mathbb{R}$ be a continuous and bounded mapping. Let $A f=B f-m \cdot f \quad(f \in$ $\left.c_{0}(x)\right)$. Then $e^{t A} \geq 0$ for all $t \geqq 0$.\\
b) Let $A$ be a nxn - matrix. Then $e^{t A} \geq 0$ for $a l l \quad t \geqq 0$ if and only if $a_{i j} \geq 0$ for $i \neq j$. This is the linear version of Kamke's theorem (see Kamke (1932)).

Now we come to the actual subject of this section, the characterization of strongly continuous positive semigroups on $C(K)$. Here K\\
denotes a compact space and $C(K)$ the space of all real-valued continuous functions on $K$. It will be essential that $K$ is compact for all what follows since it will be needed that the positive cone of $\mathrm{C}(\mathrm{K})$ has interior points.

We reformulate condition (ii) of Theorem 1.3 for unbounded operators.

Definition 1.5. An (unbounded) operator $A$ on $C(K)$ is said to satisfy the positive minimum principle if


\begin{align*}
& \text { for every } 0 \leqq \pounds \in D(A) \text { and } x \in K, \\
& f(x)=0 \quad \text { implies }(A f)(x) \geqq 0 \tag{P}
\end{align*}


Our next theorem shows that the positive minimum principle characterizes the positivity of the semigroup; and in fact, the proof is very elementary. Using more involved arguments we will later prove a much stronger result (Theorem 1.13).

Theorem 1.6. Let A be the generator of a strongly continuous semigroup on $C(K)$. Then the semigroup is positive if and only if the generator A satisfies the positive minimum principle (P).

Proof. The necessity of the condition is proved as "(i) implies (ii)" in Theorem 1.3. Assume that (P) holds. We claim that $\mathrm{R}(\lambda, \mathrm{A}) \geqq 0$ for sufficiently large real $\lambda$. (This implies the positivity of the semigroup by Prop. 1.1). Let $s:=\inf \{\lambda \in \mathbb{R}:[\lambda, \infty) \subset \rho(A)\}$. Then $s \leqq$ $\omega(A)<\infty$. Let $0 \ll u \in C(K)$. Then $\lambda_{0}:=\inf \{\lambda>s: R(\mu, A) u>0$ for all $\mu \in(\lambda, \infty)\}<\infty$ since $\lim _{\mu \rightarrow \infty} \mu R(\mu, A) u=u$.\\
We claim that $\lambda_{0}=s$.\\
In fact, if this is not true, then $\left[\lambda_{0}, \infty\right) \subset \rho(A)$ and $R\left(\lambda_{0}, A\right) u \geqq 0$ but $R\left(\lambda_{0}, A\right) u$ is not strictly positive. Consequently there exists $x \in K$ such that $\left(R\left(\lambda_{0}, A\right) u\right)(x)=0$. Then $(P)$ implies that $A\left(R\left(\lambda_{0}, A\right) u\right)(x) \geqq 0$. Hence, $0<u(x)=\lambda_{0}\left(R\left(\lambda_{0}, A\right) u\right)(x)-$ $A\left(R\left(\lambda_{0}, A\right) u\right)(x) \leq 0$, a contradiction. We have shown that $R(\lambda, A) u \gg 0$ for all $u \gg 0$ and $\lambda>\mathrm{s}$. Since $\{u \in C(K): u \gg 0\}$ is dense in $\mathrm{C}(\mathrm{K})_{+}$, it follows that $\mathrm{R}(\lambda, \mathrm{A}) \geqq 0$ for all $\lambda>\mathrm{s}$.

Remark 1.7. The proof of Theorem 1.6 shows that for the generator A of a positive semigroup on $C(K), R(\lambda, A) u>0$ whenever $0 \ll u \in C(K)$ and $[\lambda, \infty) \subset \rho(A)$. In particular, $R(\lambda, A) \geqq 0$ whenever $[\lambda, \infty) \subset \rho(A)$.

If A is a generator, then the positivity of the resolvent $R(\lambda, A)$ for large real $\lambda$ implies the positivity of the semigroup (by Prop. 1.1). On $C(K)$ much more is true. Even if $A$ is not supposed to be a generator, the existence and positivity of $R(\lambda, A)$ for large real $\lambda$ implies that $A$ is a generator (of a positive semigroup). This is surprising, because it means that in the case when the resolvent is positive, the norm condition on the resolvent sup $\left\{\left\|(\lambda-w){ }^{n} R(\lambda, A){ }^{n}\right\|\right.$ : $n \in \mathbb{N}, \lambda \geqq 0\}<\infty$ which appears in the Hille-Yosida theorem (A-II, Thm.1.7) is automatically fulfilled.

Theorem 1.8. Let K be compact and A be a densely defined operator on $C(K)$. Suppose that there exists $w \in \mathbb{R}$ such that $[w, \infty) \subset \rho(A)$ and $R(\lambda, A) \geqq 0$ for all $\lambda \geqq w$. Then $A$ is the generator of a strongly continuous positive semigroup. Moreover,\\
(1.3) $\omega(A) \leqq w$.

Proof. a) Assume that $w<0$. Denote by 1 the constant-1-function. Let $u=R(0, A) 1$. We claim that $u>0$. If not, then there exists $x \in K$ such that $u(x)=0$. Let $f \in C(K)$. Then $|f| \leqq\|f\| 1$. Consequently, $|R(0, A) f| \leqq R(0, A)|f| \leqq\|f\| R(0, A) 1=\|f\| u$. Hence $(R(0, A) f)(x)=0$ for all $f \in C(K)$. Since $D(A)=R(0, A) C(K)$, it follows that $D(A)$ is not dense, a contradiction. Define $\|f\|_{0}=$ $\inf \{\lambda>0:|f| \leqq \lambda u\}=\|f / u\|_{\infty}$. Then $\left\|\|_{0}\right.$ is an equivalent norm on $\mathrm{C}(\mathrm{K})$. Moreover, $\|f\|_{0} \leqq 1$ if and only if $\mathrm{f} \in[-u, u]$. By the resolvent equation we have\\
$\lambda R(\lambda, A) u=\lambda R(\lambda, A) R(0, A) 1=R(0, A) 1-R(\lambda, A) 1 \leqq R(0, A) 1=u$ for all $\lambda \geqq 0$. This implies that $\lambda R(\lambda, A)$ is contractive for the norm $\left\|\|_{0}{ }^{0}\right.$ Thus by the Hille-Yosida Theorem $A$ is the generator of a semigroup which is contractive with respect to the norm $\left\|\|_{0}\right.$, and so is bounded with respect to the supremum norm on $C(K)$.\\
b) If $w$ is arbitrary, let $\lambda>\mathrm{w}$ and consider $\mathrm{A}-\lambda$. Then $[w-\lambda, \infty) \subset \rho(A-\lambda)$ and $R(\mu, A-\lambda)=R(\mu+\lambda, A) \geqq 0$ for all $\mu \in[w-\lambda, \infty)$, Thus by a), $A-\lambda$ is the generator of a bounded positive semigroup. Consequently, $A$ is a generator as well and $\omega(A) \leqq \lambda$.

In Theorem 1.8 it is enough to assume that $R\left(\lambda_{n}, A\right) \geqq 0$ for some sequence $\left(\lambda_{n}\right) \in \rho(A) \cap \mathbb{R}$ satisfying $\lim _{n \rightarrow \infty} \lambda_{n}=\infty$. This follows from the following lemma.

Lemma 1.9. Let $B$ be an operator on $C(K)$ (more generally, on a Banach lattice). If $\mu_{1}, \mu_{2} \in \rho(B) \cap \mathbb{R}$ such that $0 \leqq R\left(\mu_{1}, B\right)$, $0 \leqq \mathrm{R}\left(\mu_{2}, \mathrm{~B}\right)$ and $\mu_{1}<\mu_{2}$, then $\left[\mu_{1}, \mu_{2}\right] \subset \rho(B)$ and

$$
0 \leqq R\left(\mu_{2}, B\right) \leqq R(\mu, B) \leqq R\left(\mu_{1}, B\right) \quad \text { for all } \mu \in\left[\mu_{1}, \mu_{2}\right] \text {. }
$$

Proof. Let $M:=\left\{\mu \in \rho(B) \cap\left[\mu_{1}, \mu_{2}\right]:\left[\mu, \mu_{2}\right] \subset \rho(B)\right.$ and $R(\lambda, B) \geqq 0$ for all $\left.\lambda \in\left[\mu, \mu_{2}\right]\right\}$.\\
a) The set $M$ is open. In fact, let $\mu \in M$. Then for small $h>0$ one has $R(\mu-h, B)=\sum_{n=0}^{\infty} h^{n} R(\mu, B)^{n+1} \geqq 0$.\\
b) $M$ is closed. In fact, by the resolvent equation one has for $\mu \in \mathrm{M}, \mathrm{R}\left(\mu_{1}, \mathrm{~B}\right)-\mathrm{R}(\mu, \mathrm{B})=\left(\mu-\mu_{1}\right) \mathrm{R}\left(\mu_{1}, \mathrm{~B}\right) \mathrm{R}(\mu, \mathrm{B}) \geqq 0$, hence $\mathrm{R}(\mu, \mathrm{B}) \leqq \mathrm{R}\left(\mu_{1}, \mathrm{~B}\right)$. Consequently, dist $(\mu, \sigma(\mathrm{B})) \geqq 1 /\|\mathrm{R}(\mu, \mathrm{B})\| \geqq$ $1 /\left\|\mathrm{R}\left(\mu_{1}, B\right)\right\|$ for all $\mu \in \mathrm{M}$. This implies that $M$ is closed. The assertions a) and b) imply that $M=\left[\mu_{1}, \mu_{2}\right]$.

Remark. a) The lemma shows in particular that the resolvent of the generator $A$ of a positive semigroup is decreasing on (s $(A), \infty$ ).\\
b) There exists a linear operator $B$ on $\mathbb{R}^{n}$ such that $R(\mu, B) \geqslant 0$ on some interval $\left[\mu_{1}, \mu_{2}\right] \subset \rho(B) \cap \mathbb{R}$ but $\left(e^{t B}\right) t \geqslant 0$ is not positive (see Greiner-Voigt-Wolff (1981)).

Remark. Theorem 1.8 does not hold in $C_{0}(X)$, in general. In fact, the operator $A$ on $C_{0}(0,1]$ given by $A f(x)=f^{\prime}(x)+\alpha / x f(x)$ $(x \in(0,1])$ with domain $D(A)=\left\{f \in C^{1}[0,1]: f^{\prime}(0)=f(0)=0\right\}$ where $\alpha \in(0,1)$ satisfies the following: $\rho(A)=\mathbb{C}, \mathbb{R}(\lambda, A) \geq 0$ for all $\lambda \in \mathbb{R}$. But $A$ is not the generator of a semigroup (even if more general classes than $C_{0}$-semigroups are admitted). See Arendt (1985b) for this example and a general theory of resolvent positive operators. Another example is given by Batty-Davies (1983).

Next we investigate consequences of the positive minimum principle for a densely defined operator which is not a priori assumed to be a generator. For that we will make use of the theory of half-norms developed in A-II, Sec. 2.

For $0 \ll u \in C(K)$ let


\begin{equation*}
P_{u}(f)=\inf \left\{\lambda \in \mathbb{R}_{+}: f \leqq \lambda u\right\}=\sup _{x \in K} f^{+}(x) / u(x) . \tag{1.4}
\end{equation*}


Then $\mathrm{P}_{\mathrm{u}}$ is a strict half-norm on $\mathrm{C}(\mathrm{K})$ (see $\mathrm{A}-\mathrm{II}, \mathrm{sec}$. 2). Note that


\begin{equation*}
p_{u}(f) u-f \geqq 0 \quad(f \in C(K)) \tag{1.5}
\end{equation*}


For $x \in K$, define $\phi_{x} \in C(K)^{\prime}$ by $<f, \phi_{x}>=f(x) / u(x)$.\\
Let $f \in C(K)$ such that $-f$ is not strictly positive. Then there exists $x \in K$ such that $f(x) / u(x)=p_{u}(f)$. For such an $x$ we have (1.6) $\quad \phi_{x} \in d p_{u}$ (f)\\
(see A-II, Sec. 2 for the definition of the subdifferential $d_{u}$ ).\\
Note that for $\pounds \in C(K)$ one has $\pounds \geqq 0$ if and only if $P_{u}(-f) \leqq 0$ (i.e., the half-norm $p_{u}$ induces the given ordering on $C(K)$ (cf. A-II,Rem.2.8)). As a consequence, every $p_{u}$-contractive bounded operator T on $\mathrm{C}(\mathrm{K})$ is positive.

Proposition 1.10. Let $A$ be a densely defined operator on $C(K)$. Then there exists a strictly positive $u \in D(A)$. For any such $u$ the following assertions are equivalent.\\
(i) A is $\mathrm{p}_{\mathrm{u}}$-dissipative.\\
(ii) $A u \leqq 0$ and $A$ satisfies (P).

Proof. Since $\{u \in C(K): u>0\}$ is open and non-empty and $D(A)$ is dense, there exists $0 \ll u \in D(A)$.\\
(i) implies (ii). One has $p_{u}(u)=1$. Let $x \in K$. It follows from (1.6) that $\phi_{x} \in d p_{u}(u)$. Since $D(A)$ is dense, it follows from $A-I I$, Thm. 2.7 that $A$ is strictly $P_{u}$-dissipative. Hence $<A u, \phi_{x}>0$. Thus $(A u)(x) \leqq 0$. We now show $(P)$. Let $0 \leqq f \in D(A)$ and $x \in K$ such that $f(x)=0$. We have to show that $(A f)(x) \geqq 0$. Since $f(x)$ $=0$ and $p_{u}(-f)=0$ we have by (1.6) $\phi_{x} \in d p_{\mu}(-f)$. Since $A$ is strictly $p_{u}$-dissipative we conclude that $-u(x)^{-1}(A f)(x)=\left\langle A(-f), \phi_{x}\right\rangle$ $\leqq 0$. Hence (Af) (x) $\geqq 0$.\\
(ii) implies (i). Let $f \in D(A)$. If $P_{u}(f)=0$, then $\phi:=0 \in$ $d p_{u}$ (f) and $\left\langle A f, \phi>\leqq 0\right.$. If $p_{u}$ (f) > 0 , then there exists $x \in K$ such that $\phi_{x} \in d p_{u}(f)$. Hence, $0 \leqq p_{u}(f) u-f$ and $\left(p_{u}(f) u-f\right)(x)=0$. It follows from (P) that $P_{u}(f)(A u)(x)-(A f)(x) \geqq 0$. Hence (Af) (x) $\leqq P_{u}(f)(A u)(x) \leq 0$ (by (ii)); i.e., $<A f, \phi_{x} \leq 0$.

Corollary 1.11. Let $A$ be a densely defined operator on C(K) . If A satisfies (P) then A is closable and the closure of A satisfies (P) as well.

Proof. Let $u \in D(A)$ be strictly positive. Then there exists $\lambda \in \mathbb{R}$ such that $A u \leqq \lambda u$. The operator $B=A-\lambda$ satisfies ( B ) as well and $\mathrm{Bu} \leq 0$. Then by Prop. $1.10, B$ is $p_{u}$-dissipative. Hence $B$ is closable and the closure $\bar{B}$ of $B$ is $p_{u}$-dissipative as well (by A-II Prop. 2.9). Then by Prop. 1.10 $\bar{B}$ satisfies (P). Thus A is closable and its closure $\overline{\mathrm{A}}=\overline{\mathrm{B}}+\lambda$ satisfies (P) as well.

Corollary 1.12. Let $A: C(K) \rightarrow C(K)$ be linear. If $A$ satisfies (P) then $A$ is bounded and $A+\|A\| I d \geqq 0$.

Proof. It follows from Corollary 1.11 that A is closed. Hence A is bounded. Since A satisfies (P), it follows from Thm. 1.3 that A $+\|A\| I d \geqq 0$.

The next result is a strengthened form of Theorem 2.6. It is somewhat similar to the Lumer-Phillips theorem (A-II, Thm. 2.13). Note that, however, in contrast with the condition of dissipativity, A + w satisfies (P) for any $w \in \mathbb{R}$ whenever (P) holds for $A$. Thus (P) is not a "metric" condition; that is, it does not imply any norm estimate for the semigroup. We also point out that, if (T(t)) ${ }_{t \geqq 0}$ is a positive semigroup on $C(K)$, then in general none of the semigroups $\left(e^{-w t} T(t)\right)_{t \geq 0} \quad(w \in \mathbb{R})$ is contractive (see Batty-Davies (1983) or Derndinger (1983)).

Theorem 1.13. Let $A$ be a densely defined operator on $C(K)$ which satisfies (P). Then

$$
\lambda_{0}:=\inf \{\lambda \in \mathbb{R}: A u \leqq \lambda u \text { for some } 0 \ll \mathrm{u} \in D(A)\}<\infty
$$

(a) If $(\lambda-A) D(A)$ is dense for some $\lambda>\lambda_{0}$, then $A$ is closable and the closure $\bar{A}$ of $A$ is the generator of a positive semigroup. (b) If $\lambda$ - $A$ is surjective for some $\lambda>\lambda_{0}$, then $A$ is the generator of a positive semigroup.

Proof. It follows from Prop.1.10 that $\lambda_{0}<\infty$.\\
Assume that $(\lambda-A) D(A)$ is dense, where $\lambda>\lambda_{0}$. Let $\lambda_{0}<\mu<\lambda$ and $\mathrm{B}=\mathrm{A}-\mu$. Then B satisfies $(P)$ and $\mathrm{Bu} \leqq 0$ for some strictly positive $u \in D(B)=D(A)$. Thus $B$ is $p_{u}$-dissipative by Prop.1.10. Moreover, $((\lambda-\mu)-B) D(B)$ is dense. Thus by A-II, Cor. 2.12 the closure $\bar{B}$ of $B$ generates a $\mathrm{p}_{\mathrm{u}}$-contraction semigroup. Hence the\\
closure $\bar{A}=\bar{B}+\mu$ of $A$ generates a positive semigroup of type $w(\bar{A}) \leqq \lambda$.\\
If $(\lambda-A)$ is surjective, then $A=\bar{A}$.

The proof of Theorem 1.13 yields estimates for the growth bound of a positive semigroup (see A-III, (1,3)) which we state explicitly in the next corollary.

Corollary 1.14. Let $A$ be the generator of a strongly continuous positive semigroup on $C(K)$. Then\\
(1.7) $-\infty<s(A)=\omega(A) \in \sigma(A)$ Moreover,\\
(1.8) $s(A)=\inf \{\lambda \in \mathbb{R}: A u \leqq \lambda u$ for some $0 \ll \mathrm{u} \in \mathrm{D}(\mathrm{A})\}$; and\\
(1.9) $s(A) \geq \sup \{\mu \in \mathbb{R}: A f \geq \mu f$ for some $0<f \in D(A)\}$.

Proof. Let $s=\inf \{\lambda \in \mathbb{R}:[\lambda, \infty) \subset \rho(A)\}$. Clearly, $s \leqq s(A)$. Moreover, by Remark 1.7, $\mathrm{R}(\lambda, \mathrm{A}) \geqq 0$ for all $\lambda>\mathrm{s}$. Hence it follows from ( 1.3 ) that $w(A) \leqq s$. Consequently, $s=s(A)=w(A)$. Next we prove (1.9). Let $0<1 \in D(A)$ such that $A \& \geqq \mu f$. Assume that $\mu>s(A)$. Then $R(\mu, A) \geqq 0$. Hence, $f=R(\mu, A)(\mu-A) \pounds \leqq$, a contradiction.\\
Since $D(A)$ is dense, there exists a strictly positive $u \in D(A)$. Then $A u \geqq \mu u$ for some $\mu \in \mathbb{R}$. Hence, $-\infty<\mu \leqq \mathrm{s}(\mathrm{A})$ by (1.9). Since $s(A)=s$ it is clear that $s(A) \in \sigma(A)$. It remains to show (1.8). Let $\lambda>s(A)$ and $u=R(\lambda, A) 1$. Then $u$ is strictly positive (by Rem. 1.7) and $\mathrm{Au}=\lambda \mathrm{u}-1 \leqq \lambda u$. This proves one inequality in (1.8). Assume now that $u \in D(A)$ is strictly positive such that $A u \leqq \lambda u$. Then by the proof of Thm. 1.13 we have $\omega(A) \leqq \lambda$. This proves the other inequality in (1.8).

Remark 1.15. If $A$ has compact resolvent, then by the Krein-Rutmann theorem there exists a positive eigenvector $u$ of A corresponding to a real eigenvalue. So the equality is valid in (1.9) and the supremum is a maximum. If in addition the semigroup is irreducible (see B-III, sec. 3 below), then $u$ is strictly positive and in (1.8) the infimum is attained as well.\\
Conversely, if in (1.8) the infimum is attained, then $s(A)$ is an eigenvalue.

Example 1.16. Let $A=\left(a_{i j}\right)$ be an $n \times n$-matrix such that $a_{i j} \geqq 0$ whenever $i \neq j$ (see Example I. 4 b). Then by Corollary 1.13, $s(A)=\inf \{\lambda \in \mathbb{R}: A u \leqq \lambda u$ for some strictly positive $u\}=$ $\inf { }_{u>>0} \inf \{\lambda \in \mathbb{R}: A u \leq \lambda u\}=\inf \left\{\sup _{i} \sum_{j=1}^{n} a_{i j} u_{j} / u_{i}: u>>0\right\}$. This formula is due to Collatz (1942) (see also [Schaefer (1974), Chap, Exercise 20] and Wielandt (1950)).

Corollary 1.17. Let $(T(t))_{t \geq 0}$ be a strongly continuous positive semigroup on $C(K)$. Then $T(t) u \gg 0$ for all $u \gg 0, t \geqq 0$.

Proof. Denote by A the generator of $(T(t))_{t \geq 0}$. Then by the proof of Thm. 1.13 there exist $u>>0$ and $\lambda \in \mathbb{R}$ such that $A-\lambda$ is $p_{u}$-dissipative. This implies that $p_{u}(T(t) f) \leqq e^{\lambda t} p_{u}(f)$. Observing that $f \gg 0$ if and only if $P_{u}(-f)<0$ the claim follows.

Remark 1.18. Corollaries 1.14 and 1.17 do not hold on $C_{0}(X)$. For example, let $\mathrm{X}=[0,1)$ and

$$
(T(t) f)(x)=\left\{\begin{array}{ll}
f(x+t) & \text { if } x+t \leqq 1 \\
0 & \text { if } x+t>0
\end{array} .\right.
$$

Then $(T(t))_{t \geqq 0}$ is a positive semigroup on $C_{0}(X)$ and $T(t)=0$ for all $t \geqq 1$. The generator $A$ of $(T(t))_{t \geqq 0}$ has empty spectrum, so that (1.7) is violated. However, it is still true that $\mathrm{s}(\mathrm{A})=w(\mathrm{~A})$ for generators of positive semigroups on $C_{0}(X)$ (see B-IV,Thm, 1.4).

Remark 1.19. So far, the results of this section do not depend on the lattice structure of $\mathrm{C}(\mathrm{K})$. They also hold on an ordered Banach space E with normal cone $E_{+}$which has non-empty interior. We refer to Arendt-Chernoff-Kato (1982) and to Batty-Robinson (1984) for this more general setting.

Next we apply Theorem. 1.13 to prove a result on the multiplicative perturbation of a generator $A$ which is due to Dorroh (1966) in the case when A is dissipative.

Theorem 1.20. Let $A$ be the generator of a positive semigroup on $C(K)$ and $m \in C(K)$ be strictly positive. Then the operator $m \cdot A$ given by $(m \cdot A) f=m \cdot(A f)$ on the domain $D(m \cdot A)=D(A)$ is the generator of a positive semigroup. Moreover,\\
(1.10) $\left\|m^{-1}\right\|_{\infty}^{-1} \omega(A) \leqq \omega(m \cdot A) \leqq\|m\|_{\infty} \omega(A)$.

Proof. We can assume that $\|m\|_{\infty} \leqq 1$ (in fact, if $\mathrm{B}:=\left(\mathrm{m} /\|\mathrm{m}\|_{\infty}\right) \cdot \mathrm{A}$ is the generator of a positive semigroup, then by A-I, 3.1 $\mathrm{m} \cdot \mathrm{A}=\|\mathrm{m}\|_{\infty} \mathrm{B}$ also generates a positive semigroup). The assertion of the theorem holds for A if and only if it is valid for A - w (w $\in \mathbb{R}$ ). So by the proof of Thm. 1.13 we can assume that there exists $0 \ll \mathrm{u} \in \mathrm{C}(\mathrm{K})$ such that A is $\mathrm{P}_{\mathrm{u}}$-dissipative. We first show,\\
(1.11) if $B$ is a $\mathrm{P}_{\mathrm{u}}$-dissipative operator and $0 \ll q \in C(K)$, then $q \cdot B$ is $p_{u}$-dissipative.

Let $f \in D(q \cdot B)=D(B)$. There exists $x \in K$ such that $\phi_{x} \in d p_{u}(f)$\\
(by (1.6)). Hence $\left\langle B f, \phi_{x} \gg 0\right.$. Consequently, $<q \cdot B f, \phi_{x}>=q(x)$ $\left\langle B f, \phi_{x}>\leq 0\right.$.

Next we show,\\
if $B$ is the generator of a $\mathrm{P}_{\mathrm{u}}$-contraction semigroup and (1.12) $I \geqq q \in C(K)_{+}$is such that $\|I-q\|_{\infty}<1 / 2$, then $q \cdot B$ generates a $\mathrm{p}_{\mathrm{u}}$-contraction semigroup.

Because of (1.11) we only have to show that ( $\mathrm{I}-\mathrm{q} \cdot \mathrm{B}$ ) is surjective. Note that $1 \in \rho(B)$. We have $(I d-q \cdot B)=(I d-B-(q-1) B)=$ (Id - (q-1)BR(1,B)) (Id - B) . Thus it suffices to show that\\
Id - $(q-1) B R(1, B)$ is invertible. The norm $\|f\|_{u}=\max \left\{p_{u}(f), p_{u}(-f)\right\}$ $=\sup _{x \in K}|f(x)| / u(x)$ is equivalent to the supremum norm. Denote by $\|\mathrm{T}\|_{\mathrm{u}}$ the operator norm corresponding to $\left\|\|_{\mathrm{u}}\right.$ ( $\left.\mathrm{T} \in L(\mathrm{E})\right)$. Then it is enough to show that $\|(q-1) B R(1, B)\|_{u}=\|(q-1)(R(1, B)-I)\|_{u}<1$. For $r \in C(K)_{+}$denote by $M_{r}$ the multiplication operator given by $M_{r} f=r \cdot f$. Then $\left\|M_{r}\right\|_{u}=\sup \left\{\|r \cdot f\|_{u}:\|f\|_{u} \leqq 1\right\}=\sup \left\{\sup _{x \in K}\right.$ $\left.r(x)|f(x)| / u(x):\|f\|_{u} \leqq I\right\} \leqq\|r\|_{\infty}$. Since $B$ is $\mathrm{p}_{\mathrm{u}}$-dissipative we have $\|R(1, B)\|_{u} \leqq 1$ (by A-II,Lemma 2.10). This gives $\|(q-1)(R(1, B)-I)\|_{u} \leqq\left\|_{(1-q)}\right\|_{u}\left(\|R(1, B)\|_{u}+1\right) \leq 2\|1-q\|_{\infty}<1$. The proof of (1.12) is complete.

There exists $k \in \mathbb{N}$ such that $\left\|1-\mathrm{m}^{1 / k}\right\|_{\infty}<1 / 2$. Applying now (1.12) succesively to $\mathrm{B}=\mathrm{m}^{1 / \mathrm{k}} \cdot \mathrm{A}$ and $\mathrm{q}=\mathrm{m}^{1 / \mathrm{k}}$ (1=1, .., k-1) we obtain that $m \cdot A$ generates a $p_{u}$-contraction semigroup (which in particular is positive).\\
Finally we show (1.10) to hold.\\
Let $0 \ll u \in D(A)=D(m \cdot A)$ and $A u \leqq \lambda u$. Then $m \cdot A u \leqq\|m\|_{\infty} \lambda u$. So (1.8) implies that $\omega(m \cdot A) \leqq\|m\|_{\infty} \omega(A)$. This is one part of (1.10).

The other part follows from this since $w(A)=\omega\left(m^{-1} \cdot m \cdot A\right)$ $\leqq\left\|\mathrm{m}^{-1}\right\|_{\infty} w(\mathrm{~m} \cdot \mathrm{~A})$.

In the following lemma a condition ( $\mathrm{P}^{\prime}$ ) is introduced which is dual to the positive minimum principle.

Lemma 1.21. Let $A$ be the generator of a strongly continuous positive semigroup on $C(K)$. Then for $f \in C(K)_{+}, 0 \leqq \mu \in D\left(A^{\prime}\right)$\\
( $\left.P^{\prime}\right) \quad\langle f, \mu\rangle=0$ implies $\left\langle f, A^{\prime} \mu\right\rangle \geqq 0$.

Proof. $\left\langle f, A^{\prime} \mu\right\rangle=\lim _{t \rightarrow 0} \frac{1}{\mathrm{t}}\langle T(t) f-f, \mu\rangle=\lim _{t \rightarrow 0} \frac{1}{t}\langle T(t) f, \mu\rangle \geq 0$.

Example 1.22. Let $K=[-1,0]$. Let $\alpha \in \mathbb{R}$ and $\mu$ be a measure on $[-1,0]$ such that $\mu(\{0))=0$. Desine the operator $A$ on $\mathrm{C}[-1,0]$ by $\mathrm{Af}=f^{\prime}$ with domain $D(A)=\left\{f \in C^{1}[-1,0]: f^{\prime}(0)=\alpha f(0)+\right.$ $<\mathrm{f}, \mu>\}$.\\
Claim: A is the generator of a positive semigroup if and only if $\mu \geq 0$.

Proof of the claim. Assume that A generates a positive semigroup. By the definition of $A$ one has $\delta_{0} \in D\left(A^{\prime}\right)$ and $A^{\prime} \delta_{0}=\alpha \delta_{0}+\mu$. So it follows from ( $P^{\prime}$ ) that $\langle f, \mu\rangle=\left\langle f, A^{\prime} \delta_{0}\right\rangle \geq 0$ for all $f \in C[-1,0]_{+}$ such that $f(0)=0$. By Lemma 1.2 this implies that $\mu \geq 0$.\\
In order to show the converse assume that $\mu \geqq 0$.\\
a) We show that $A$ is densely defined. Consider the normed space $F=$ $C^{1}[-1,0]$ with the supremum norm. Then $\psi: F \rightarrow \mathbb{R}$ given by $\psi(f)=$ $f^{\prime}(0)-\alpha f(0)-\langle f, \mu>$ is a discontinuous linear form on $F$. Consequently $D(A)=$ ker $\psi$ is dense in $F$. Since $F$ is dense in $C[-1,0], D(A)$ is dense in $C[-1,0]$ as well.\\
b) A satisfies (P) (see Def. 1.5). In fact, let $f \in D(A)+$ and $x \in[-1,0]$ such that $f(x)=0$. It is clear that $A f(x)=f^{\prime}(x) \geqq 0$ if $x<0$. But if $f(0)=0$, then $\operatorname{Af}(0)=f^{\prime}(0)=\langle f, \mu\rangle \geqq 0$ since $f \in D(A)$.\\
c) We show that $(\lambda-A)$ is bijective for $\lambda>\alpha+\|\mu\|$. Let $g \in C[-1,0]$. The solutions of the equation $\lambda f-f^{\prime}=g$ ( $f \in C[-1,0]$ ) are given by $f(x)=e^{\lambda x}\left[\int_{x}^{0} e^{-\lambda y} g(y) d y+c\right]$ where $c \in \mathbb{R}$. Moreover, $f \in D(A)$ if and only if\\
(2.8) $c\left(\lambda-\alpha-\int_{-1}^{0} e^{\lambda x} d \mu(x)=g(0)+\int_{-1}^{0} e^{\lambda x} \int_{x}^{0} e^{-\lambda y} g(y) d y d \mu(x)\right.$.

If $\lambda>\alpha+\|\mu\|$, then $\lambda-\alpha-\int_{-1}^{0} e^{\lambda x} d \mu(x) \neq 0$ and there exists exactly one $c \in \mathbb{R}$ satisfying (2.8). We have shown that\\
( $\lambda$ - A) is bijective for $\lambda>\alpha+\|\mu\|$.\\
By Thm.1.13, it follows from a),b) and c) that A generates a positive semigroup.

Let us mention in addition that it follows from a) in the proof that $(\alpha+\|\mu\|, \infty) \subset \rho(A)$, since $A$ is closed. By Remark 1.7 we thus have


\begin{equation*}
\mathrm{s}(\mathrm{~A}) \leqq \alpha+\|\mu\| . \tag{2,9}
\end{equation*}


Example 1.23. Let $E=C\left([-1,0], \mathbb{R}^{n}\right)$. Then $u \in E$ is given by $u=$ $\left(u_{1}, \ldots, u_{n}\right)$ where $u_{i} \in C[-1,0](i=1, \ldots, n)$. Let $A$ be defined by $A u=u^{\prime}=\left(u_{1}^{\prime}, \ldots, u_{n}^{\prime}\right)$ with domain $D(A)=\left\{u \in C^{1}\left([-1,0], \mathbb{R}^{n}\right)\right.$ : $\left.u^{\prime}(0)=L u\right\}$.\\
Here $L$ is defined by

$$
L u=\left(\begin{array}{c}
L_{11} u_{1}+\ldots+L_{1 n} u_{n} \\
\vdots \\
L_{n 1} u_{1}+\ldots+L_{n n} u_{n}
\end{array}\right)
$$

where $L_{i j} \in C[-1,0]$ (I $\leqq$ i,j $\left.\leqq n\right)$.\\
Let $L_{i i}=c_{i} \delta_{0}+\mu_{i}$ with $\mu_{i}(\{0\})=0 \quad(i=1, \ldots, n)$.\\
Then A generates a positive semigroup if and only if

$$
L_{i j} \geqq 0 \text { for } i \neq j \text { and } \mu_{i} \geqq 0 \quad(i, j=1, \ldots, n) \text {. }
$$

This can be proved in a similar way as the claim in Example 1.21 (see Arendt (1984a)).

Example 1.24. Let $A$ on $C[0,1]$ be given by $A f=f^{\prime \prime}$ with domain $D(A)=\left\{f \in C^{2}[0,1]: f^{\prime}(0)+\alpha f(0)=0, f^{\prime}(1)+B f(1)=0\right\}$, where $\alpha, \beta \in \mathbb{R}$. Then $A$ is the generator of positive semigroup.

Proof. The operator A satisfies (P). In fact, let $0 \leqq f \in D(A)$ and $f(a)=0$ where $a \in[0,1]$. If $a \in(0,1)$ then $f^{\prime \prime}(a) \geqq 0$ since $f$ has a minimum in a . If $a=0$, then $\mathrm{f}^{\prime}(0)=\mathrm{f}^{\prime}(0)+\operatorname{af}(0)=0$ since $f \in D(A)$. Consequently, $f(x)=\int_{0}^{x}(x-y) f^{\prime \prime}(y) d y \geqq 0$ for all $\mathbf{x} \geq 0$. This implies $\mathrm{f}^{\prime \prime}(0) \geqq 0$. The argument for $a=1$ is analoguous.

It remains to show that $\mu-\mathrm{A}$ is surjective for large real $\mu$. Let $g \in c[0,1]$. Let $\lambda>0$ and $k=I / 2 \lambda\left\lceil e^{\lambda x} \int_{x}^{1} e^{-\lambda y} g(y) d y-\right.$ $\left.e^{-\lambda x} \int_{x}^{1} e^{\lambda} \mathrm{g}(y) d y\right]$. Then $k \in c^{2}[0,1]$ and ${ }^{\lambda^{2}} k-k^{\prime \prime}=g$. Let $h=$ $a e^{\lambda x}+e^{-\lambda x}$, where $a, b \in \mathbb{R}$. Then $h \in C^{2}[0,1]$ and $\lambda^{2} h-h^{\prime \prime}=0$. Let $\mathbf{f}=\mathrm{k}+\mathrm{h}$. Then $\lambda^{2} \mathbf{f}-\mathbf{f}^{\prime \prime}=\mathrm{g}$. The condition that $\mathrm{f} \in \mathrm{D}(\mathrm{A})$ leads to two linear equations in $a$ and $b$, and it is easy to see that they have a solution $(a, b) \in \mathbb{R}^{2}$ if $(\lambda+\alpha)(\beta-\lambda)+(\lambda-\alpha)(\lambda+\beta) \exp \left(\lambda^{2}\right) \neq 0$. Thus there exists a solution if $\lambda$ is large enough, and $\left(\lambda^{2}-A\right)$ is surjective.

\section*{2. Lattice Semigroups on C (X)}
Throughout this section $X$ denotes a locally compact space and $C_{0}(X, R) \quad$ (resp., $C_{0}(X, C)$ the space of all real-valued (resp., complex-valued) continuous functions on $x$ which vanish at infinity. If we do not want to specify the field we simply write $\mathrm{C}_{\mathrm{O}}(\mathrm{X})$. Recall from $B-I, S e c .3$ that a linear bounded operator $T$ on $C_{0}(X)$ is positive if and only if\\
(2.1) $|T f| \leqq T|f|$ for al1 $f \in C_{O}(X)$.

The operator T is a lattice homomorphism if and only if in (2.1) equality holds; i.e.,\\
(2.2) $|\mathrm{Tf}|=\mathrm{T}|\mathrm{f}|$ for all $f \in C_{0}(\mathrm{X})$.

Remark 2.1. If $T$ is a lattice homomorphism on $C_{0}(X, \mathbb{C})$, then $T$ leaves $C_{0}(X, \mathbb{R})$ invariant and the restriction $\mathrm{T}_{\mathbb{R}}$ of $T$ to $C_{0}(X, \mathbb{R})$ is a lattice homomorphism. Conversely, the linear extension $T$ of a lattice homomorphism $\mathbb{T}_{\mathbb{R}}$ on $C_{O}(\mathrm{X}, \mathbb{R})$ to $C_{0}(\mathrm{X}, \mathbb{C})$ is a lattice homomorphism (see B-I,Sec.3).

A semigroup (T(t)) $t \geq 0$ is called lattice semigroup if $T(t)$ is a lattice homomorphism for all $t \geqq 0$. In Section 3 we will give a concrete representation of lattice-semigroups which shows that there is a large variety of examples. This section is devoted to the characterization of lattice semigroups in terms of their generators.

The heuristic idea is the following. Let $(T(t))_{t \geqq 0}$ be a lattice semigroup with generator $A$. Let $\pounds \in D(A)$ and assume that the\\
modulus function $\theta$ given by $\theta(g)=|g|$ is differentiable in $f$ (in some sense which has to be made precise). Then one expects that a chain rule holds so that $\theta(T(t) f)=|T(t)|$ is differentiable at $t=0$. Since $|T(t) f|=T(t)|f|$, this implies $|f| \in D(A)$ and $A|f|$ $=d /\left.d t\right|_{t=0} \theta(T(t) f)=D_{A f} \theta(f) d / d t{ }_{t=0} T(t) f=\left(D_{A f} \theta(f)\right.$ Af) (where the precise meaning of ( $\mathrm{Af}_{\mathrm{Af}} \theta$ (f))Af depends on the chain rule which we will have to establish). So we obtain an identity for the generator A which corresponds exactly to the lattice property $|T(t) f|=$ $T(t)|f|$ of the semigroup. We will see in C-II, Sec. 5 that in a Banach lattice with order continuous norm the above argument is rigorous (for all $\mathrm{f} \in \mathrm{D}(\mathrm{A})$ ). On $\mathrm{C}_{\mathrm{O}}(\mathrm{X})$ we have to use a weak form of the argument and $|f| \in D(A)$ only holds for special $f \in D(A)$ (see Cor. 2.8).

We start by investigating differentiability of the modulus and by establishing a chain rule. For later use we formulate the following definition and proposition for a general Banach space $G$ even though only $G=\mathbb{C}$ will be considered in this section.

Definition 2.2. Let $G$ be a Banach space and $\theta: G \rightarrow G$ a mapping. Let $f \in G, u \in G$. Then $\theta$ is called right-sided Gateaux differentiable in f in direction u if\\
(2.3) $D_{u} \Theta(f):=\lim _{t \downarrow 0} 1 / t(\theta(f+t u)-\theta(f))$ exists.

The mapping $\theta$ is right-sided Gateaux differentiable in $f$ if $D_{u} \theta(f)$ exists for all directions $u \in G$; and if $\theta$ is right-sided Gateaux-differentiable in every point $\pounds \epsilon G$, then we call $\theta$ right-sided Gateaux differentiable.

Proposition 2.3 (chain rule). Let $G$ be a Banach space and $k: \mathbb{R} \rightarrow G$ be right-sided differentiable in $a \in \mathbb{R}$ (with right derivative $\left.k^{\prime}(a)\right)$. Suppose that $\theta: G \rightarrow G$ is a Lipschitz continuous mapping. If $\theta$ is right-sided Gateaux-differentiable in $k(a)$ in the direction of $k^{\prime}(a)$, then $\theta \circ k: \mathbb{R} \rightarrow G$ is right-sided differentiable in a and has a right derivative\\
(2.4) $(\theta \circ k)^{\prime}(a)=D_{k^{\prime}(a)} \theta(k(a))$.

Proof. There exists $L \geqq 0$ such that $\|\theta(x)-\theta(y)\| L\|x-y\|$ for all $x, y \in G$. Then\\
$\lim _{t \nmid 0}\left\|1 / t(\theta(k(a+t))-\theta(k(a)))-D_{k^{\prime}(a)} \theta(k(a))\right\| \leqq$ $\operatorname{lim-sup}_{t+0} \| 1 / t\left(\theta(k(a+t))-\theta\left(k(a)+t k^{\prime}(a)\right) \|+\right.$\\
$\operatorname{lim-sup}_{t \downarrow 0} \| 1 / \mathrm{t}\left[\theta\left(k(a)+t k^{\prime}(a)\right)-\theta(k(a))-D_{k^{\prime}}(a)^{\theta(k(a))] \|_{\|} \leqq}\right.$ $\operatorname{lim-sup}_{t+0} \mathrm{~L} \cdot 1 / \mathrm{t}\left(k(a+t)-k(a)-t k^{\prime}(a)\right)+0=0$.

For $z \in \mathbb{C}$ we let\\
(2.5) $\operatorname{sign} z= \begin{cases}z /|z| & \text { if } z \neq 0 \\ 0 & \text { if } z=0 .\end{cases}$

Lemma 2.4. The function $\theta: \mathbb{C} \rightarrow \mathbb{C}$ given by $\theta(z)=|z|$ is rightsided Gateaux differentiable and

\[
D_{u} \theta(z)= \begin{cases}\operatorname{Re}[(\operatorname{sign} \bar{z}) \cdot u] & \text { if } z \neq 0  \tag{2.6}\\ |u| & \text { if } z=0\end{cases}
\]

Proof. If $z=0$, relation (2.6) is obvious from the definiton. Let $z=\left(x_{0}+i y_{0}\right) \neq 0$. We identify $\mathbb{C}$ and $\mathbb{R}^{2}$. Then $\theta(\mathrm{x}, \mathrm{y})^{0}=\left(\mathrm{x}^{\circ}+\mathrm{y}^{2}\right)^{1 / 2}$ is differentiable in $z$ and $D_{u} \theta(z)=\left(\operatorname{grad} \theta\left(x_{0}, y_{0}\right) \mid u\right)=1 /|z|\left(\left(x_{0}, y_{0}\right) \mid\left(u_{1}, u_{2}\right)\right)=$ $1 /|z|\left(x_{0} u_{1}+y_{0} u_{2}\right)=1 /|z| \operatorname{Re}\left(\left(x_{0}-i y_{0}\right) \cdot\left(u_{1}+i u_{2}\right)\right)=\operatorname{Re}[(\operatorname{sign} \bar{z}) \cdot u]$, where $u=u_{1}+i u_{2}=\left(u_{1}, u_{2}\right) \in \mathbb{C}=\mathbb{R}^{2}$ and (v|u) denotes the canonical scalar product of $v, u \in \mathbb{R}^{2}$.

Let $f, g \in C_{0}(X)$. We denote by (sign f) $(g)$ the bounded Borel function given by\\
(2.7) $[(\operatorname{sign} f)(g)](x)= \begin{cases}(\operatorname{sign} f(x)) \cdot g(x) & \text { if } f(x) \neq 0 \\ |g(x)| & \text { if } f(x)=0 .\end{cases}$

Similarly, (sign f) (g) is defined by


\begin{equation*}
[(\operatorname{sign} f)(g)](x)=(\operatorname{sign} f(x)) \cdot g(x) \cdot \tag{2.8}
\end{equation*}


We identify the dual space of $C_{0}(X)$ with $M(X)$, the space of all bounded regular Borel measures on $X$. We extend the duality by setting

$$
\langle h, \phi\rangle=\int h(x) d \phi(x)
$$

for every bounded Borel function $h$ on $X$ and every $\phi \in M(X)$.

After these preparations we now can show that the lattice property $|T(t) f|=T(t)|f|$ of the semigroup corresponds to the identity (2.9) below for the generator, which we call Kato's equality (cf. Remark 2.7).

Theorem 2.5. A strongly continuous semigroup (T(t) $t \geqslant 0$ on $C_{0}(X)$ is a lattice semigroup if and only if its generator A satisfies


\begin{equation*}
\langle\operatorname{Re}[(\operatorname{sign} \bar{f})(A f)], \phi\rangle=\langle | f\left|, A^{\prime} \phi\right\rangle \tag{2.9}
\end{equation*}


for all $f \in D(A), \phi \in D\left(A^{\prime}\right)$\\
(Kato's equality).

From the proof of the theorem we isolate the following lemma.

Lemma 2.6. Let $(T(t))_{t \geqslant 0}$ be a semigroup on $C_{O}(X)$ with generator $A$. Then for every $f \in D(A), \phi \in M(X)$,


\begin{equation*}
d / d t_{\mid t=0}\langle | T(t) f|, \phi\rangle=\langle\operatorname{Re}[(\operatorname{sign} \bar{f})(A f)], \phi\rangle . \tag{2.10}
\end{equation*}


Proof. Let $f \in D(A)$ and $x \in X$. Define the function $k(t)=$ (T(t)f)(x)(t $\geqq 0$ ). Then $k$ is right-sided differentiable in 0 with derivative $k^{\prime}(0)=(A f)(x)$. It follows from the chain rule Prop. 2.3 that


\begin{equation*}
d / d t_{\mid t=0}|(T(t) f)(x)|=\operatorname{Re}[(\operatorname{sig} n \bar{f})(A f)](x) \tag{2.11}
\end{equation*}


Moreover, $1 / t| | T(t) f|-|f|| \leqq 1 / t|T(t) f-f|$. Thus sup $1 \geq t>01 / t$ $\||\mathrm{T}(t) \mathrm{f}|-|f|\|<\infty$; i.e., the functions $k_{t} \in C_{0}(X)$ given by


\begin{equation*}
k_{t}(x)=1 / t(|(T(t) f)(x)|-|f(x)|) \quad(x \in X) \tag{2.12}
\end{equation*}


$(t>0)$ are uniformly dominated by a constant. The dominated convergence theorem and (2.11) imply that $d / d t_{\mid t=0}\langle | T(t) f|, \phi\rangle=\lim _{t \downarrow 0}\left\langle k_{t, \phi}\right\rangle=\langle\operatorname{Re}[(\operatorname{sign} \bar{f})(A f)], \phi\rangle$.

Proof of Theorem 2.5. Assume that $(T(t))_{t \geqq 0}$ is a lattice semigroup. Let $f \in D(A), \phi \in D\left(A^{\prime}\right)$. It follows from the preceding lemma that\\
$\langle\operatorname{Re}[(\operatorname{signn} \bar{f})(A f)], \phi\rangle=d / d t{ }_{t=0}\langle | T(t) f|, \phi\rangle=d /\left.d t\right|_{t=0}\langle T(t)| f|, \phi\rangle=$ $\langle | f|, A ' \phi\rangle$.\\
conversely, assume that (2.9) holds. Let $t>0, \pounds \in C_{0}(x)$. We have to show that $|T(t) f|=T(t)|f|$. Since $D(A)$ is dense in $C_{0}(X)$, we can assume that $\pounds \in D(A)$. Moreover, since $D\left(A^{\prime}\right)$ is $\sigma\left(M(X), C_{O}(X)\right)$-dense in $M(X)$, it suffices to show that


\begin{equation*}
\langle | T(t) f|, \phi\rangle=\langle T(t)| f|, \phi\rangle \tag{2.13}
\end{equation*}


for all $\phi \in D\left(A^{\prime}\right)$.\\
Let $\phi \in D\left(A^{\prime}\right)$ and define the function $k(s)=\langle T(t-s)| T(s) f|, \phi\rangle$\\
( $s \in[0, t]$ ). We claim that $k$ is right-sided differentiable with derivative $k^{\prime}(s)=0$ for all $s \in[0, t]$. This implies that $k(0)=$ $k(t)$ which is (2.13). Since $\phi \in D\left(A^{\prime}\right)$ we have\\
(2.14) $1 i m_{h+0} 1 / h\left\langle g,(T(t-(s+h))-T(t-s))^{\prime} \phi\right\rangle=-\left\langle g, A^{\prime} T(t-s)^{\prime} \phi\right\rangle$\\
for all $g \in C_{O}(X)$. Consequently,

$$
\overline{I i m}_{h+0} 1 / h\left\|(T(t-(s+h))-T(t-s))^{\prime} \phi\right\|<\infty
$$

by the uniform boundedness principle. Hence, since\\
$\lim _{h+0}|\mathrm{~T}(\mathrm{~s}+\mathrm{h}) \mathrm{f}|=|\mathrm{T}(\mathrm{s}) \mathrm{f}|$, (2.14) implies that\\
(2.15)

$$
\begin{aligned}
& \left.\lim _{h+0} 1 / h<|T(s+h) f|,(T(t-(s+h))-T(t-s))^{\prime} \phi\right\rangle= \\
& -\langle | T(s) f\left|, A^{\prime} T(t-s)^{\prime} \phi\right\rangle .
\end{aligned}
$$

Using this we obtain\\
$\lim _{h+0} 1 / \mathrm{h}(k(s+h)-k(s))$\\
$=\lim _{h+0} 1 / \mathrm{h}(\langle T(t-(s+h))| \mathrm{T}(\mathrm{s}+\mathrm{h}) \mathrm{f}|, \phi\rangle-\langle T(t-s)| \mathrm{T}(\mathrm{s}+\mathrm{h}) \mathrm{f}|, \phi\rangle+$

$$
\left.\lim _{h \downarrow 0} 1 / h(\ll T(t-s))|T(s+h) f|-T(t-s)|T(s) f|, \phi\right\rangle
$$

$=-\langle | T(s) f\left|, A^{\prime} T(t-s)^{\prime} \phi\right\rangle+\lim _{h+0} 1 / h\left\langle\left(|T(s+h) f|-|T(s) f|, T(t-s)^{\prime} \phi\right\rangle\right.$.\\
By Lemma 2.6 the last term is\\
$-\langle | T(s) f\left|, A^{\prime} T(t-s)^{\prime} \phi\right\rangle+\left\langle\operatorname{Re}[(\operatorname{sign} \mathrm{T}(s) f)(A T(s) f)], T(t-s)^{\prime} \phi\right\rangle$,\\
and this is 0 by hypothesis.

Remark 2.7. We will see in Chapter C-II that the inequality $|T(t) f| \leqq \mathrm{T}(t)|f|$, which holds precisely for positive semigroups, implies the inequality corresponding to (2.9). For $\mathrm{A}=\Delta$ (the Laplacian) this is a version of the classical Kato's inequality.

Corollary 2.8. Let $(\mathbb{T}(t))_{t \geqq 0}$ be a lattice semigroup on $C_{0}(X)$ with generator $A$. If $f \in D(A)$ and $f(x) \neq 0$ for all $x \in X$, then $|f| \in D(A)$ and $\operatorname{Re}[(\operatorname{sign} \bar{f})(A f)]=A|f|$.

Proof. If $f \in D(A)$ and $f(x) \neq 0$ for all $x \in X$, then $(\operatorname{sign} \bar{f})(A f)=(\operatorname{sign} \bar{f})(A f) \in C_{0}(x)$. Hence by (2.9), $\langle\operatorname{Re}[(\operatorname{sign} \overline{\mathrm{f}})(\mathrm{Af})\urcorner, \phi\rangle=\langle | f\left|, A^{\prime} \phi\right\rangle$ for all $\left.\phi \in D^{\prime} A^{\prime}\right)$. So the assertion follows from the following lemma.

Lemma 2.9. Let $A$ be a densely defined closed operator on a (real or complex) Banach space $G$. Let $f, g \in G$ such that $\langle f, \phi\rangle=\left\langle g, A^{\prime} \phi\right\rangle$ for all $\phi \in D\left(A^{\prime}\right)$. Then $g \in D(A)$ and $A g=f$.

Proof. Denote by $G(A):=\{(h, A h): h \in D(A)\} C G \times G$ the graph of A. Assume that $(g, f) \& G(A)$. Since $G(A)$ is closed, it follows from the Hahn-Banach theorem that there exists $\left(\psi_{1}, \psi_{2}\right) \in G^{\prime} \times G^{\prime}$ such that $\left\langle h, \psi_{1}\right\rangle+\left\langle A h, \psi_{2}\right\rangle=0$ for all $h \in D(A)$ and $\left\langle g, \psi_{1}\right\rangle+\left\langle f, \psi_{2}\right\rangle \neq 0$. By the definition of $A^{\prime}$ this implies that $\left.\psi_{2} \in D^{\prime} A^{\prime}\right)$ and $A^{\prime} \psi_{2}=-\psi_{1}$. Hence $\left\langle f, \psi_{2}\right\rangle \neq-\left\langle g, \psi_{1}\right\rangle=\left\langle g, A^{\prime} \psi_{2}\right\rangle$. So the condition in the lemma is violated.

Next we prove a converse of Corollary 2.8.

Theorem 2.10. Let $A$ be the generator of a real semigroup (T) $(t){ }_{t \geqq 0}$ on $C(K, \mathbb{C})$, where $K$ is compact. Then $(T(t))_{t \geqq 0}$ is a lattice semigroup if and only if\\
$\mathrm{f} \in \mathrm{D}(\mathrm{A}), \mathrm{f}(\mathrm{x}) \neq 0$ for all $\mathrm{x} \in \mathrm{K}$ implies $|\mathrm{f}| \in D(A)$ and $A|f|=\operatorname{Re}((\operatorname{sign} \bar{f}) A f)$.

Remark. Although we assume that $(T(t))_{t \geqq 0}$ is a real semigroup (i.e., $T(t) C(K, \mathbb{R}) \subset C(K, \mathbb{R})$ for all $t \geqq 0$ ), it is important for the proof that we consider the space of all complex-valued continuous functions on $K$. In fact, if $K$ is connected, the condition in the theorem is always trivially satisfied for all $\mathrm{f} \in \mathrm{C}(\mathrm{K}, \mathbb{R})$.

Proof. It follows from cor. 2.8 that the condition is necessary. So assume that the condition is satisfied. Since $\left(T(t)_{t \geqq 0}\right.$ is real, the restriction $T_{\mathbb{R}}(t)$ of $T(t)$ to $C(K, \mathbb{R}) \quad(t \geqq 0)$ defines a strongly continuous semigroup. Its generator $A_{\mathbb{R}}$ is a restriction of A . Since $D\left(A_{\mathbb{R}}\right)$ is dense in $C(K, \mathbb{R})$, there exists a strictly positive\\
$u \in D\left(A_{\mathbb{R}}\right)$. Moreover, $\lim _{t \rightarrow 0} T(t) u=u$ uniformly. Thus there exists $t_{0}>0$ such that $T(t) u$ is strictly positive for all $t \in\left[0, t_{0}\right]$.\\
Let $f \in D\left(A_{R}\right)$. For $\varepsilon>0$ let $f_{\varepsilon}=f+i \varepsilon u$. Then $T(t) f_{\varepsilon} \in D(A)$ and $\left|T(t) f_{\varepsilon}\right|$ is strictly positive for all $t \in\left[0, t_{0}\right]$. By hypothesis, $\left|T(t) f_{\varepsilon}\right| \in D(A)$ and $\operatorname{Re}\left(\left(\operatorname{sign}\left(T(t) \bar{f}_{\varepsilon}\right)\right) A T(t) f_{\varepsilon}\right)=A\left|T(t) f_{E}\right|$ for all $t \in\left[0, t_{0}\right]$. One sees as in the proof of Thm. 2.5 that this implies that $\left|T(t) f_{\varepsilon}\right|=T(t)\left|f_{\varepsilon}\right|$ for all $t \in\left[0, t_{0}\right]$. Letting $\varepsilon \rightarrow 0$ one obtains that $|T(t) f|=T(t)|f| \quad\left(t \in\left[0, t_{0}\right]\right)$. Since $\quad D(A)$ is dense in $C(K, \mathbb{R})$ we conclude that $|T(t) f|=T(t)|E|$ for all $f \in C(K, \mathbb{R})$ and all $t \in\left[0, t_{0}\right]$. Let $s>t_{0}$. Then $s / n \leqq t_{0}$ for some $n \in \mathbb{N}$. Hence $|T(s) f|=\left|T(s / n)^{n} f\right|=T(s / n)^{n}|f|=T(s)|f|$ for all $f \in C(K, \mathbb{R})$. We have shown that $\mathbb{T}_{\mathbb{R}}(t)$ is a lattice homomorphism for all $t \geqq 0$; hence $\mathrm{T}(\mathrm{t})$ is so as well (cf. Rem. 2.1).

Corollary 2.11. Let $A$ be the generator of a lattice semigroup on $C(K, \mathbb{C}) \quad(K$ compact). Assume that $m \in C(K)$ is strictly positive. Then $m \cdot A$ with domain $D(m \cdot A)=D(A)$ generates a lattice semigroup.

Proof. By Theorem $1.20 \mathrm{~m} \cdot \mathrm{~A}$ is the generator of a strongly continuous semigroup. It remains to show that it is a lattice semigroup. We use Theorem 2.10. Let $f \in D(m \cdot A)=D(A)$ such that $f(x) \neq 0$ for all $x \in K$. Then $\operatorname{Re}[(\operatorname{sign} \bar{E}) m \cdot A f]=m \cdot \operatorname{Re}[(\operatorname{sign} \bar{f}) A f]=m \cdot A|f|$.

Example 2.12. The operator $A_{\max }$ on the (real or complex space) $C[-1,0]$ given by $A_{\text {max }} f=f^{\prime}$ with domain $D\left(A_{\max }\right)=C^{1}[-1,0]$ satisfies Kato's equality; i.e.,


\begin{gather*}
\left.<\operatorname{Re}\left[(\operatorname{sign} \mathrm{f})\left(A_{\max } f\right)\right], \phi\right\rangle=\langle | f\left|, A_{\max } \phi\right\rangle  \tag{2.16}\\
\left(f \in D\left(A_{\max }\right), \phi \in D\left(A_{\max }\right)\right) .
\end{gather*}


Moreover, $\left(\lambda-A_{\text {max }}\right)$ is surjective for $\lambda \geqq 0$ (cf. Ex. 1.22). Thus, since $\operatorname{ker}\left(\lambda-A_{\max }\right)=\mathbb{C e}_{\lambda} \quad\left(e_{\lambda}(x)=e^{\lambda x}\right)$, Kato's equality does not have as strong consequences as the positive minimum principle (which by Thm. 1.13 would imply that $A_{\max }$ is a generator).

Proof. It is not difficult to prove that the adjoint A' of $A_{\text {max }}$ is given by


\begin{equation*}
A_{\max }^{\prime} \phi=\phi(0) \delta_{0}-\phi(-1) \delta_{-1}-d \phi \tag{2.17}
\end{equation*}


with domain $D\left(A_{\text {max }}^{\prime}\right)=B V[-1,0]$ (the space of all functions of bounded variation on $[-1,0]$ ). Here we identify $B V[-1,0] \subset L^{1}[-1,0]$ with a subspace of $C[-1,0]$ by setting $\langle f, \phi\rangle=\int_{-1}^{0} f(x) \phi(x) d x$ (f $\in \mathrm{C}[-1,0], \phi \in \mathrm{BV}[-1,0]$ ). For $\phi \in \operatorname{BV}[-1,0]$, d $\phi$ denotes the linear form on $C[-1,0]$ given by $f \rightarrow \int_{-1}^{0} f(x) d \phi(x)$.\\
We now show $(2.16)$. Let $f \in D\left(A_{\max }\right)^{-1}=C^{1}[-1,0], \phi \in D\left(A_{\max }^{\prime}\right)=$ BV[-1,0]. By Lemma 2.4 and the chain rule (Prop. 2.3) we have\\
$|f(x)| '\left(:=d^{+} / d t|t=x| f(t) \mid=\operatorname{Re}(\operatorname{sign} \bar{f}) f^{\prime}\right](x) \quad$ (where $f^{\prime}(x)=$ $(\operatorname{Re} f)^{\prime}(x)+i(\operatorname{Imf})^{\prime}(x)$ in the complex case). Thus\\
$\langle\operatorname{Re}[(\operatorname{sign} \bar{f}) \mathrm{Af}], \phi\rangle=\int_{-1}^{0}|f(x)|, \phi(x) d x=\int_{-1}^{0} \phi(x) d|f(x)|=$ $\phi(0)|f(0)|-\phi(-1)|f(-1)|-\int_{-1}^{0}|f(x)| d \phi(x)=\langle | f\left|, A_{\max }^{\prime} \phi\right\rangle$.

Example 2.13. Let $A$ on (the real or complex) space $\mathrm{C}[-1,0]$ be given by $A f=f^{\prime}$ with domain $D(A)=\left\{f \in C^{1}[-1,0]: f^{\prime}(0)=\right.$ Lf $\}$ where $L \in M[-1,0]=C[-1,0]^{\prime}$. Then $A$ is the generator of a lattice semigroup if and only if $L=\alpha \delta_{0}$ for some $\alpha \geqq 0$.

Proof. Assume that $A$ is the generator of a lattice semigroup $(T(t))_{t \geq 0}$. There exists $\mu \in M[-1,0]$ satisfying $\mu(\{0\})=0$ and $\alpha \in \mathbb{R}$ such that $L=\alpha \delta_{0}+\mu$. We claim that\\
(2.18) $|\langle f, \mu\rangle|=\langle | f|, \mu\rangle$ for all $f \in D(A)$ satisfying $f(0)=0$.

In fact, by the definition of $A$ we have\\
(2.19) $\delta_{0} \in D\left(A^{\prime}\right)$ and $A^{\prime} \delta_{0}=L$.

Moreover, by Thm. 2.5, A satisfies Kato's inequality (2.9). Since $f(0)=0$ this implies\\
$\left|<f, \mu>\left|=\left|f^{\prime}(0)\right|=\operatorname{Re}\left[(\operatorname{sign} f)\left(f^{\prime}\right)\right](0)\right.\right.$

$$
=\left\langle\operatorname{Re}[(\operatorname{sign} n f)(A f)], \delta_{0}\right\rangle=\langle | f\left|, A^{\prime} \delta_{0}\right\rangle \quad(b y \quad(2.9))
$$

$$
=\langle | \pounds|, \mu\rangle .
$$

Since $\phi(f)=f^{\prime}(0)-<f, \mu>$ defines a linear form on the space $F=$ $\left\{f \in C^{1}[-1,0]: f(0)=0\right\}$ which is discontinuous for the supremum norm, the space $D(A)=$ ker $\phi$ is dense in $F$ and consequently dense in $c_{0}[-1,0)$. It follows that $(2.18)$ holds for all $f \in c_{0}[-1,0)$. So by $B-I, \sec .2$, there exist $\beta \geqq 0$ and $x \in[-1,0)$ such that $\mu=\beta \delta_{\mathrm{r}}$. Assume that $\beta \frac{4}{7} 0$. It is easy to see that there exists a real function $f \in C^{1}[-1,0]$ satisfying $f^{\prime}(0)=\alpha f(0)+\theta f(x)$ and $f(0) f(x)<0$. Hence $f \in D(A)$ but <Re[(sign f) $(A f)], \delta_{0}>=$ $(\operatorname{sign} f(0)) f^{\prime}(0)=(\operatorname{sign} f(0))(\alpha f(0)+\beta f(x))=$\\
$\alpha|f(0)|+\beta(\operatorname{sign} f(0)) f(x) \neq \alpha|f(0)|+\beta|f(x)|=\langle | f\left|, \alpha \delta_{0}+\beta \delta_{x}\right\rangle=$ $\langle | f\left|, A^{\prime} \delta_{0}\right\rangle$. This contradicts (2.9). We have shown that $\beta=0$; i.e., $L=\alpha \delta_{0}$.\\
The converse can be shown by using Thm. 2.5 again. However, if $L=\alpha \delta_{0}$, then it is easy to see that $A$ generates the semigroup $(T(t))_{t \geqq 0}$ given by

$$
(T(t) f)(x)= \begin{cases}f(x+t) & \text { if } x+t<0 ; \\ e^{t \alpha} f(0) & \text { if } x+t \geqq 0\end{cases}
$$

So $(T(t))_{t \geq 0}$ is clearly a lattice semigroup.

\section*{3. SEMIFLOWS, FLOWS AND POSITIVE GROUPS}
In this section we establish a relation between generators of lattice homomorphisms and derivations. On the space $C_{0}(\mathbb{R})$, for example, this will enable us, to give a detailed description of all generators of positive groups.

At first we consider a compact space $K$ and denote by $C(K)=C(K, \mathbb{R})$ the space of all real valued continuous functions on K . The extension of the subsequent results to the complex spaceis obvious.

A lattice homomorphism $T$ on $C(K)$ is an algebra homomorphism if and only if $T 1=1$ (see B-I, Sec.3). We start by describing semigroups of algebra homomorphisms on $C(K)$.

Definition 3.1. A mapping $\phi:[0, \infty) \times K \rightarrow K$ is called semiflow if for each $t \geqq 0$ the mapping $\phi_{t}$ given by $\phi_{t}(x)=\phi(t, x)$ is continuous and\\
(3.1) $\quad \phi_{s}{ }^{\circ \phi_{t}}=\phi_{s+t}$\\
for all $s, t \geq 0$\\
(3.2) $\phi_{O}(x)=x \quad(x \in K)$.

A semiflow $\phi$ on K induces a family $(T(t))_{t \geq 0}$ of algebra homomorphisms on $\mathrm{c}(\mathrm{K})$ by\\
(3.3) $T(t) f=f \circ \phi_{t}$.

Then clearly $T(t) T(s)=T(t+s)(t, s \geq 0) ;$ i.e., $(T(t))_{t \geqq 0}$ has the\\
semigroup property. Conditions for strong continuity are given in the following lemma.

Lemma 3.2. The following assertions are equivalent:\\
(i) The mapping $\phi: \mathbb{R}_{+} \times \mathrm{K} \rightarrow \mathrm{K}$ is continuous (where $\mathbb{R} \times \mathrm{K}$ carries the product topology).\\
(ii) The mapping $\phi$ is separately continuous.\\
(iii) (T(t)) $t_{t \geqslant 0}$ is a strongly continuous semigroup on $C(K)$.

Proof. (i) trivially implies (ii).\\
If (ii) holds, then $t \rightarrow T(t) f$ is weakly continuous for every $f \in C(K)$ (by the theorem of dominated convergence). This implies strong continuity (see for example [Davies (1980); Prop. 1.23]).\\
It remains to show that (iii) implies (i). Because of (3.1) it suffices to show that the restriction $\phi_{O}$ of $\phi$ to $[0,1] \times \mathrm{K}$ is continuous. By hypothesis, the mapping $W: f \rightarrow(t \rightarrow T(t) f)$ from $C(K)$ into $C([0,1], C(K))$ is continuous. Identifying $C([0,1], C(K))$ canonically with $C([0,1] \times K)$ the mapping $W$ obtains the form $f \rightarrow f \circ \phi_{O} \cdot$ Since $W$ is continuous, $\phi_{O}$ is continuous as well.

A semiflow is called continuous if it satisfies the equivalent conditions of Lemma 3.2.

Definition 3.3. An operator $\delta$ on $C(K)$ is called derivation if $D(\delta)$ is a subalgebra of $C(K)$ such that\\
(3.4) $\delta(f \cdot g)=(\delta f) g+f(\delta g)$ for all $f, g \in D(\delta)$.\\
(3.5) $1 \in D(\delta)$

Note that (3.4) implies $\delta 1=0$.\\
A lattice semigroup $(T(t))_{t \geq 0}$ on $C(K)$ is called Markovian if $T(t) 1=1$ for all $t \geqq 0$.

Theorem 3.4. Let ( $T(t))_{t \geqq 0}$ be a semigroup on $C(K)$ with generator A . The following assertions are equivalent.\\
(i) $\quad(T(t))_{t \geqq 0}$ is a Markovian lattice semigroup.\\
(ii) $T(t)$ is an algebra homomorphism for every $t \geqq 0$.\\
(iii) There exists a continuous semiflow $\phi$ on K such that $T(t) f=f \circ \phi_{t} \quad(t \geqq 0)$.\\
(iv) A is a derivation.

Proof. (i) and (ii) are equivalent by the remark at the beginning of this section.

Assume that (ii) holds. Then there exists a continuous mapping $\phi_{t}: K \rightarrow K$ such that $T(t) f=f \circ \phi_{t}$ for all $f \in C(K)$ (see B-I, Sec. 3 ). The semigroup property implies that $\left(\phi_{t}\right)_{t \geqq 0}$ is a continuous semiflow. This shows (iii) to hold.\\
If (iii) holds, then $T(t) 1=1$ for all $t \geqq 0$ hence $1 \in D(A)$ and $A 1=0$. Let $f, g \in D(A)$. Then $d /\left.d t\right|_{t=0} T(t)(f \cdot g)=$ $d /\left.d t\right|_{t=0}(T(t) f) \cdot(T(t) g)=(A f) \cdot g+f \cdot(A g)$. Thus $f \cdot g \in D(A)$ and (3.4) holds. Hence A is a derivation.

Finally assume that (iv) holds. We prove (ii), i.e., we have to show that $T(t)(f \cdot g)=T(t) f \cdot T(t) g$ for $t>0$. Since $D(A)$ is a dense subalgebra, we can assume that $f, g \in D(A)$. Define $n:[0, t] \rightarrow C(K)$ by $n(s):=T(t-s)[T(s) f \cdot T(s) g]$. Then $n(0)=T(t)(f \cdot g)$ and $\eta(t)=T(t) f \cdot T(t) g$. Since $A$ is a derivation, $\eta^{\prime}(s)=0$ for $s \in[0, t]$. Hence $\eta(0)=\eta(t)$. This shows (ii) to hold.

If $\delta$ is the generator of a semigroup $(T(t))_{t \geqq 0}$ given by $T(t) f=f \circ \phi_{t}$, then we call $\phi$ given by $\phi(t, x)=\phi_{t}(x)$ the semiflow associated with (T(t)) $t_{\geq 0}$ (or associated with $\delta$ ). We now can describe the generator of any lattice semigroup as a perturbation of a derivation. If 1 is in the domain of the generator, an additive perturbation (by a multiplication operator) suffices; in general a similarity transformation has to be applied in addition. This is the assertion of the following two theorems.

Theorem 3.5. Let $A$ be a generator of a semigroup (T( $t)_{t \geqq 0}$ on $C(K)$. Suppose that $1 \in D(A)$. Then the following assertions are equivalent.\\
(i) (T(t)) $t \geqq 0$ is a lattice semigroup.\\
(ii) There exist a derivation $\delta$ (generating a semigroup of algebra homomorphisms) and a multiplier $h \in C(K)$ such that $A=\delta+h$ (i.e., $D(A)=D(\delta)$ and $A f=\delta f+h \cdot f$ for $f \in D(A)$ ).

Moreover, if (ii) holds, then $(T(t))_{t \geqq 0}$ is given by\\
(3.6) $(T(t) f)(x)=\exp \left(\int_{0}^{t} h(\phi(s, x)) d s\right) \cdot f(\phi(t, x))$\\
where $\phi$ is the semiflow associated with $\delta$.

Proof. Let $h=A 1$ and $\delta=A-h$. Then the semigroup $\left(T_{0}{ }^{(t)}\right)_{t \geq 0}$ generated by $\delta$ is a lattice semigroup if and only if ( $T(t))_{t \geq 0}$ is a lattice semigroup [since $T_{0}(t) f=\lim _{n+\infty}\left(e^{-t / n \cdot h} \cdot T\left(\frac{t}{n}\right)\right)^{n_{f}}$ and $T(t) f=\lim _{n \rightarrow \infty}\left(e^{t / n \cdot h} \cdot T_{0}\left(\frac{t}{n}\right)\right)^{n} f$ for $a l l$, $t \geqq c(k) \quad$ by

A-II, (1.8). Since $\delta 1=0$ one has $T_{0}(t) 1=1$ for all $t \geqq 0$ and the equivalence of (i) and (ii) follows with the help of Theorem 3.4. Now assume that (i) and (ii) hold. Let

$$
(S(t) f)(x)=\exp \left(\int_{0}^{t} h(\phi(s, x)) d s\right) \cdot f(\phi(t, x))
$$

for all $x \in K, f \in C(K), t \geqq 0$. Then one easily shows that $(S(t))_{t \geq 0}$ is a strongly continuous semigroup. Denote by $B$ its generator. For $f \in D(\delta),\left.\frac{d}{d t}\right|_{t=0} S(t) f=h \cdot f+\delta f$. Hence $\delta+h \subset B$. Since $\delta+\mathrm{h}$ also is a generator, it follows that $\delta+\mathrm{h}=\mathrm{B}$.

Theorem 3.6. An operator $A$ is generator of a lattice semigroup on $C(K)$ if and only if there exists a derivation $\delta$ which is a generator, a function $h \in C(K)$ and a strictly positive function $\mathrm{P} \in \mathrm{C}(\mathrm{K})$ such that\\
(3.7) $\quad \mathrm{A}=\mathrm{MoM}^{-1}+\mathrm{h}$\\
where $M \in L(C(K))$ is given by $M f=p \cdot f$.

Proof. In order to show the non-trivial implication assume that A generates a lattice semigroup. Since $D(A)$ is dense in $C(K)$, there exists $0 \ll p \in D(A)$. Let $h(x)=(A p)(x) / p(x)$. The operator given by Mf $=f \cdot p$ is a lattice isomorphism. Thus $\delta:=\mathrm{M}^{-1}(\mathrm{~A}-\mathrm{h}) \mathrm{M}$ generates a lattice semigroup. Since $M 1=p \in D(A)$ one has $1 \in D(\delta)$ and $\delta 1=\mathrm{M}^{-1}(\mathrm{~A}-\mathrm{h}) \mathrm{P}=0$. Thus $\delta$ is the generator of a semigroup of algebra homomorphisms, hence a derivation by Theorem 3.4.

At the end of this section we will show that any derivation on c[0,1] which generates a group is similar to a differential operator of first order. This in connection with Theorem 3.6. will enable us to describe all generators of positive groups as perturbations of a differential operator.\\
In Section 1 we had obtained a very simple condition describing generators of positive semigroups on $C(K)$ by the positive minimum principle and a range condition. This result yields a characterization of generators of automorphism groups by "locality" and a range condition. By an automorphism we understand an algebra isomorphism of $C(K)$ onto itself.

Theorem 3.7. Let $A$ be a densely defined operator on $C(K)$. The following assertions are equivalent.\\
(i) A is the generator of an automorphism group.\\
(ii) $1 \in D(A)$ and $A 1=0 ;( \pm 1-A) D(A)=C(K)$ and $A$ is local, in the sense that for $0 \leqq f \in D(A), f(x)=0$ implies (Af) $(x)=0 \quad(x \in K)$.

Proof. An invertible operator $T$ such that $T \geqq 0$ and $T^{-1} \geqq 0$ is an automorphism if and only if $T l=1$. Hence $A$ is the generator of an automorphism group if and only if $A$ and -A generate a positive group, $1 \in D(A)$ and $A 1=0$. Thus Theorem 3.7. follows from Theorem 1.13.

Remark. It is remarkable that from locality, the range condition and $1 \in D(A), A 1=0$ it follows that $D(A)$ actually is a subalgebra of $C(K)$ and $A$ is a derivation. The "order-theoretical" property of locality is in some aspects stronger than the algebraic property of being a derivation. For example a local, densely defined operator is closable (by Prop.1.11); but there exist derivations on c[0,1] which are not closable (see Bratteli-Robinson (1975)).

Remark (an excursion to C*-algebras).\\
Theorem 3.7 also holds for non-commutative $c^{*}$-algebras. More precisely: Let $A$ be a $C^{*}$-algebra with unit 1 and let $A_{h}$ be the real Banach space of all hermitian elements in $A$. Then $A_{h}$ is a real ordered Banach space and 1 is an interior point of $\left(A_{h}\right)+$. Let $A$ be a densely defined operator on $A_{h}$.\\
Then $A$ is the generator of an automorphism group if and only if $1 \in D(A)$ and $A 1=0 ;( \pm 1-A)(D(A))=A_{h}$ and $A$ is local in the sense that for $0 \leqq x \in D(A), 0 \leqq \phi \in\left(A_{h}\right)^{\prime}, \phi(x)=0$ implies $\phi(A x)=0$.\\
The proof of Theorem 3.7 can be carried over to this case if one notices the following. A strongly continuous group $T(t), \in \mathbb{R}$ on $A_{h}$ is an automorphism group if and only if it is positive and $T(t) 1=1$ for all $t \in \mathbb{R}$ [see Bratteli-Robinson (1979), Cor. 3.2.21].

Now we let X be a locally compact space and consider positive groups on $C_{0}(X)=C_{0}(X, \mathbb{R})$, the space of all continuous real-valued functions on $x$ which vanish at infinity. Our aim is to describe their generators as perturbations of generators of automorphism groups; i.e., we will extend Theorem 3.6 by allowing X to be noncompact but\\
restrict ourselves to positive groups (or equivalently semigroups of lattice isomorphisms). And in fact, it is not difficult to show by an example that the corresponding result is wrong for lattice semigroups in general.

A mapping $\phi: \mathbb{R} \times \mathrm{X} \rightarrow \mathrm{X}$ is called a flow on X if the partial maps $\phi_{t}: X \rightarrow X$ given by $\phi_{t}(x)=\phi(t, x)$ are continuous and satisfy\\
$\begin{array}{lll}(3.8) & \phi_{0}(x)=x & (x \in x) \\ (3.9) & \phi_{s}{ }^{\circ} \phi_{t}=\phi_{s+t} & (s, t \in \mathbb{R}) .\end{array}$

It follows from the definition that each $\phi_{t}$ is a homeomorphism on $X$ and $\phi_{-t}=\left(\phi_{t}\right)^{-1}$.\\
A flow $\phi$ is called continuous if it is continuous with respect to the product topology on $\mathbb{R} \times \mathrm{X}$.\\
Given a flow $\phi$ a family $\left(h_{t}\right)_{t \in \mathbb{R}} \subset C^{b}(x)$ is called a cocycle of $\phi$ if\\
(3.10) $\quad h_{0}=1$\\
(3.11) $\quad h_{t+s}=h_{t} \cdot\left(h_{s} \circ \phi_{t}\right) \quad(s, t \in \mathbb{R})$.

It follows from (3.10) and (3.11) that $h_{t}(x) \neq 0$ for all $x \in x$ and $1 / h_{t}(x)=h_{-t}\left(\phi_{t}(x)\right) \quad(t \in \mathbb{R})$. The cocycle is called continuous if the mapping $(t, x) \rightarrow h_{t}(x)$ from $\mathbb{R} \times x$ into $\mathbb{R}$ is continuous with respect to the product topology on $\mathbb{R} \times \mathrm{x}$.

Let $\$$ be a flow and ( $\left.h_{t}\right)_{t \in R}$ a cocycle of $\phi$. Then\\
(3.12) $T(t) \pounds=h_{t} \cdot f \circ \phi_{t}$\\
defines a bounded operator $T(t)$ on $C_{O}(X) \quad(t \in \mathbb{R})$. Clearly $T(s+t)=T(s) T(t)$ for all $s, t \in \mathbb{R}$.

Proposition 3.8. Let $\phi: \mathbb{R} \times X \rightarrow X$ be a flow and ( $h_{t}$ ) $t \in \mathbb{R}$ a cocycle of $\phi$. If for every $x \in X$ the mappings $t \rightarrow \phi_{t}(x)$ and $t \rightarrow h_{t}(x)$ are continuous, then (3.12) defines a strongly continuous group.

Proof. We first note that $\|\mathrm{T}(t)\|$ is bounded on compact intervals of $\mathbb{R}$. This follows from [Hille-Phillips (1957), 7.4.1] since\\
$q(t)=\log \|\mathrm{T}(t)\|$ defines a subadditive, measurable function from $\mathbb{R}$ into $\mathbb{R}$. [In fact, $\|\mathrm{T}(t)\|=\sup _{x \in X}\left|\mathrm{~h}_{t}(\mathrm{x})\right|$ for $t \in \mathbb{R}$. so it follows from the assumption that $t \rightarrow\|\mathrm{f}(t)\|$ is lower semicontinuous and hence measurable]. If $f \in C_{O}(X)$, then by hypothesis the function\\
$t \rightarrow h_{t}(x) f(\phi(t, x))=(T(t) f)(x)$ is continuous on $\mathbb{R}$. It follows from the dominated convergence theorem that $T()$.$f is weakly continuous.$ Hence (T(t)) $t_{t \in \mathbb{R}}$ is strongly continuous (see e.g., [Davies (1980), Prop. 1.23]).

The group defined by (3.12) is positive whenever $\left(h_{t}\right)_{t \in \mathbb{R}} \subset \mathrm{C}^{\mathrm{b}}(\mathrm{X}){ }_{+}$. We now show that every positive group on $C_{0}(x)$ is of the form (3.12).

Proposition 3.9. Let $(T(t))_{t \in \mathbb{R}}$ be a strongly continuous group of positive operators on $C_{0}(X)$. Then there exist a continuous flow on $X$ and a continuous cocycle $\left(h_{t}\right) t_{t \in R}$ of $\phi$ such that (3.12) holds.

Proof. Since $T(t)$ and $T(t)^{-1}=T(-t)$ are positive operators, T(t) actually is a lattice isomorphism. Then there exist a homeomorphism $\phi_{t}$ on $X$ and $h_{t} \in C^{b}(X){ }_{+}$such that $T(t) f=h_{t} \cdot f \circ \phi_{t}$ for all $f \in C_{0}(x) \quad(t \in \mathbb{R})$. The group property of $(T(t)) t \in \mathbb{R}$ then implies that $\phi(t, x):=\phi_{t}(x)$ defines a flow on $x$ and that $\left(h_{t}\right)_{t \in R}$ is a cocycle of $\phi$. It remains to show that $\phi$ and $\left(h_{t}\right)_{t \in \mathbb{R}}$ are continuous.\\
At first we consider the flow. Since we have $\phi_{t+s}=\phi_{t}{ }^{\circ} \phi_{s}$ and every $\phi_{t}$ is a homeomorphism on $X$, it is enough to establish continuity of $\phi$ at points $\left(0, x_{0}\right) \in \mathbb{R} \times x$. Given a compact neighbourhood $V$ of $x_{0}=\phi\left(0, x_{0}\right)$, there exists a continuous function $f: x \rightarrow[0,1]$ satisfying $f\left(x_{0}\right)=1$ and supp $f \in V$. There exists $t_{0}>0$ such that $\|\mathrm{T}(t) f-\mathrm{f}\|<\frac{1}{2}$ for $|t| \leq t_{0}$. Let $W:=\left\{\mathrm{x} \in \mathrm{X}:|\mathrm{f}(\mathrm{x})|>\frac{1}{2}\right\}$; then for $|t| \leqq t_{0}$ and $x \in W$ we have $\left|h_{t}(x) \cdot f(\phi(t, x))-f(x)\right|<\frac{1}{2}$ and $|f(x)|>\frac{1}{2}$; hence $f(\phi(t, x)>0$. This implies that $\phi(t, x) \in V$ whenever $|t| \leqq t_{0}$ and $x \in W$.\\
To prove continuity of the cocycle we first remark that by strong continuity of $(T(t))_{t \in \mathbb{R}}$ the mapping $(t, x) \rightarrow(T(t) f)(x)$ is continuous on $\mathbb{R} \times X$ for every fixed $f \in C_{0}(X)$. Given compact subsets $A \subset \mathbb{R}, B \subset X$, the set $C:=\phi(A \times B)$ is compact; hence there exists $f \in C_{0}(X)$ such that $\left.f\right|_{C}=1$. For $(t, x) \in A \times B$ we have $h_{t}(x)=$ $(T(t) f)(x)$. Thus $(t, x) \rightarrow h_{t}(x)$ is continuous on $A \times B$.

Corollary 3.10. Let $\phi$ be a separately continuous flow. Then $\phi$ is continuous. If $\left(h_{t}\right)_{t \in R}$ is a cocycle of $\phi$ such that $t \rightarrow h_{t}(x)$ is continuous for every $x \in X$, then $\left(h_{t}\right){ }_{t \in \mathbb{R}}$ is continuous.

This follows from Prop.3.8 and Prop.3.9.

Example 3.11. Let $\phi$ be a continuous flow on X .\\
a) Let $p$ be a continuous function on $x$ such that inf $x \in x \quad p(x)>0$ and $\sup _{\mathrm{x} \in \mathrm{X}} \mathrm{p}(\mathrm{x})<\infty$. Then\\
(3.13) $\quad \mathrm{p}_{\mathrm{t}}:=\mathrm{p} / \mathrm{po} \phi_{t} \quad(t \in \mathbb{R})$\\
defines a continuous cocycle of $\phi$.\\
b) For $h \in c^{b}(x)$ define\\
(3.14) $h_{t}(x):=\exp \left(\int_{0}^{t} h(\phi(s, x)) d s\right)$.

Then ( $\left.h_{t}\right)_{t \in R}$ is a continuous cocycle of $\$$ (compare (3.6)).

Cocycles as defined by (3.13) are always globally bounded. In general. this is false for cocycles of the second type. On the other hand, a cocycle described by (3.14) is differentiable with respect to $t$. This is not satisfied by cocycles of the first type in general. Thus neither (3.13) nor (3.14) gives a description of arbitrary cocycles. However every positive cocyle is a product of cocycles of the form (3.13) and (3.14) . More precisely, we have the following lemma.

Lemma 3.12. Let $\phi$ be a continuous flow on $x$ and $\left(k_{t}\right)_{t \in \mathbb{R}} \subset c^{b}(x)+$ a continuous cocycle of $\phi$. Then there exist $p \in C^{b}(X)$ satisfying $\inf _{x \in x} p(x)>0$ and $h \in C^{b}(x)$ such that\\
(3.15) $k_{t}(x)=\left(p(x) / p(\phi(t, x)) \cdot \exp \left(\int_{0}^{t} h(\phi(s, x)) d s\right.\right.$ for all $t \in \mathbb{R}, \mathbf{x} \in \mathbf{X}$.

Proof. We first note that there exist constants $M, w \geqq 1$ such that (3.16) $\left(M e^{(w-1)|t|}\right)^{-1} \leq k_{t}(x) \leqq M e^{(w-1)|t|}$ for all $t \in \mathbb{R}, x \in x$. In fact, let (T(t) $)_{t \in \mathbb{R}}$ be the group given by $T(t) f=k_{t} \cdot f_{0} \phi_{t}$ $\left(t \in \mathbb{R}, f \in C_{0}(X)\right)$. Then there exist constants $M, w \geqq 1$ such that (3.17) $\|\mathrm{T}(t)\| \leqq \mathrm{Me}^{(\mathrm{w}-1)|t|}$\\
for all $t \in \mathbb{R}$. since $\|T(t)\|=\sup _{x \in X} k_{t}(x)$ the right inequality of\\
(3.16) follows. Moreover, $k_{-t}=1 / k_{t}{ }^{\circ \phi}{ }_{-t}$. Hence $\|\mathrm{T}(-t)\|$ $=\sup _{x \in X} 1 / k_{t}(x)=\left[i n f_{x \in X} k_{t}(x)\right]^{-1}$. So (3.17) also implies the first inequality in (3.16).\\
Now we define $p$ and $h$ by

$$
p(x):=\int_{0}^{\infty} e^{-w s_{s}}(x) d s, h(x)=w-1 / p(x) \quad(x \in x)
$$

Then $p$ is a continuous function and we have $(M(2 w-1))^{-1}=$\\
$=\int_{0}^{\infty} e^{-w s}\left(M e^{(w-1) s}\right)^{-1} d s \leqq p(x) \leqq \int_{0}^{\infty} e^{-w s} M e^{(w-1) s} d s=m \quad$ for all\\
$\mathrm{x} \in \mathrm{X}$. In particular, it follows that $h \in \mathrm{C}^{\mathrm{b}}(\mathrm{X})$.\\
For all $x \in x, t \in \mathbb{R}$ we have\\
$k_{t}(x) \cdot p(\phi(t, x))=\int_{0}^{\infty} e^{-w s_{k_{t+s}}}(x) d s=e^{w t} \int_{t}^{\infty} e^{-w s_{k}}(x) d s$.\\
Now fix $x \in X$ and define $f: \mathbb{R} \rightarrow \mathbb{R}$ by\\
$f(t):=k_{t}(x) p\left(\phi(t, x) / p(x)=\left[e^{w t} / p(x)\right] \cdot \int_{t}^{\infty} e^{-w s_{k}}(x) d s\right.$.\\
The function f is differentiable and satisfies the following differential equation $f^{\prime}(t)=w f(t)-k_{t}(x) / p(x)=h(\phi(t, x)) f(t)$. Moreover $f(0)=1$. Hence $f(t)=\exp \left(\int_{0}^{t} h(\phi(s, x)) d s\right)$ for every $t \in \mathbb{R}$. This is (3.15).

As before we call a group $\left(\mathrm{T}_{0}(t)\right)_{t \in \mathbb{R}}$ on $C_{0}(X)$ an automorphism group if each $\mathrm{T}_{0}(t)$ is an algebra isomorphism on $\mathrm{C}_{0}(\mathrm{X})$.\\
Analogously an operator $\delta$ on $C_{0}(X)$ is called a derivation if $D(\delta)$ is a subalgebra of $C_{O}(X)$ and $\delta(f \cdot g)=(\delta f) \cdot g+f \cdot(\delta g)$ for all $f, g \in D(\delta)$.

Proposition 3.13. Let $\left(T_{0}(t)\right)_{t \in \mathbb{R}}$ be a group on $C_{0}(x)$. The following assertions are equivalent.\\
(i) $\left(T_{0}(t)\right)_{t \in \mathbb{R}}$ is an automorphism group.\\
(ii) There exists a continuous flow $\phi$ on X such that $T_{0}(t) f=f \circ \phi_{t} \quad\left(f \in C_{0}(x), t \in \mathbb{R}\right)$.\\
(iii) The generator of $\left(\mathrm{T}_{O}(t)\right)_{t \in \mathbb{R}}$ is a derivation.

Proof. Every automorphism group is positive. So by Prop. 3.9 it is defined via (3.12) by some continuous flow and cocycle. It is easy to see that the cocycle is identically 1 . Thus (i) implies (ii). One shows as in Theorem 3.4 that (ii) implies (iii) and (iii) implies (i).

If $\left(T_{0}(t)\right)_{t \in \mathbb{R}}$ is an automorphism group with generator $\delta$ we call $\phi$ in (ii) of Prop.3.13 the flow associated with (T $\mathrm{T}_{0}(t){ }_{t \in R}$ (or associated with $\delta$ ).\\
Now we can show that every generator of a positive group is a perturbation of a derivation.

Theorem 3.14. An operator $A$ on $C_{0}(X)$ is the generator of a positive group $(T(t))_{t \in \mathbb{R}}$ if and only if there exist a derivation $\delta$ on $c_{0}(x)$ which is the generator of a group, a function $h \in c^{b}(x)$ and $p \in C^{b}(x)$ satisfying inf ${ }_{x \in X} p(x)>0$ such that\\
(3.18) $\quad A=V \delta V^{-1}+h$\\
where $V: c_{0}(X) \rightarrow c_{0}(X)$ is given by Vf $=p \cdot f$. In that case one has (3.19) $(\mathbb{T}(t) f)(x)=\left[p(x) / p\left(\phi_{t}(x)\right)\right] \cdot\left(\exp \int_{0}^{t} h(\phi(s, x)) d s\right) \cdot f\left(\phi_{t}(x)\right)$ for all $f \in c_{0}(x), t \in \mathbb{R}, x \in X$.

Note: (3.18) means that $D(A)=\left\{f: V^{-1} f \Leftrightarrow D(\delta)\right\}$ and $A f=V \delta V^{-1} f+h f$.

Proof. Assume that $A$ is given by (3.18). Since V is a lattice isomorphism, it is clear that $\mathrm{V}^{-1} \delta \mathrm{~V}$ generates a positive group; and consequently, A does so as well (cf. the proof of Theorem 3.5). Conversely, let $\left(T(t){ }_{t \in R}\right.$ be a positive group with generator $A$. By Prop. 3.9 and Lemma 3.12 we know that there exist a continuous flow $\phi, 0 \ll p \in C^{b}(x)$ and $h \in C^{b}(X)$ such that (3.19) holds. Let $\delta$ be the generator of the automorphism group defined by $\phi$. We have to show that (3.18) holds. As in Theorem 3.5 one sees that $\delta+\mathrm{h}$ generates the group $(S(t))_{t \in \mathbb{R}}$ given by $(S(t) f)(x)=\exp \left(\int_{0}^{t} h(\phi(s, x)) d s\right) \cdot f\left(\phi_{t}(x)\right)$. Hence $V \delta V^{-1}+h=V(\delta+h) V^{-1}$ generates $\left(V S(t) V^{-1}\right)_{t \in \mathbb{R}}=(T(t))_{t \in \mathbb{R}}$. This is (3.18).

Since every generator of a positive group is the perturbation of a derivation, we now look for examples of derivations which generate a group.

Example 3.15. Let $x=\mathbb{R}^{n}$. Consider a function $F \in C^{1}\left(\mathbb{R}^{n}, \mathbb{R}^{n}\right)$ such that $\sup _{x \in \mathbb{R}^{n}}\|D F(x)\|<\infty$ where $D F(x) \in L\left(\mathbb{R}^{n}\right)$ denotes the derivative of $F$ in $x$. Then there exists a continuous flow $\phi$ on $\mathbb{R}^{\text {n }}$ such that\\
(3.20) $\frac{\partial}{\partial t} \phi(t, x)=F(\phi(t, x)) \quad$ for all $t \in \mathbb{R}, x \in \mathbb{R}^{n}$.

Consider the automorphism group $\left(T_{0}(t)\right)_{t \in \mathbb{R}}$ given by $T_{0}(t) f=f \circ \phi_{t}$\\
and denote by $\delta$ its generator.\\
Then $D_{0}=\left\{f \in C_{0}\left(\mathbb{R}^{n}\right) \cap C^{1}\left(\mathbb{R}^{n}\right): \lim \|x\| \rightarrow \infty\|(g r a d f)(x)\|=0\right\}$ is a core of $\delta$ and\\
(3.21) $(\delta f)(x)=((\operatorname{grad} f)(x) \mid F(x))$ for all $f \in D_{0}, x \in \mathbb{R}^{n}$,\\
where $(\cdot \mid \cdot)$ denotes the scalar product in $\mathbb{R}^{n}$

Proof. Let $f \in D_{0}$. Then $g=f-($ grad $f \mid F) \in C_{0}\left(\mathbb{R}^{n}\right)$ and\\
$(R(1, \delta) g)(x)=\int_{0}^{\infty} e^{-t} f(\phi(x, t)) d t-\int_{0}^{\infty} e^{-t}((g r a d f)(\phi(t, x)) \mid F(\phi(t, x))) d t$ $=f(x)$ by integrating by parts. Hence $f \in D(\delta)$ and $f-\delta f=g$; i.e. $\delta f=(g r a d f \mid F)$. This proves (3.21).

Next we show $T_{0}(t) D_{0} \subset D_{0}$ for all $t \geqq 0$, which implies that $D_{0}$ is a core of $\delta$ by A-I,Thm.1.9 (or A-II, Cor.1.34).\\
Since $F \in C^{1}\left(\mathbb{R}^{n}, \mathbb{R}^{n}\right)$, it follows that $\phi \in C^{1}\left(\mathbb{R} \times \mathbb{R}^{n}, \mathbb{R}^{n}\right.$ ) (see e.g., [Hirsch-smale (1974), 15.21). Moreover for each $x \in \mathbb{R}^{n}, \frac{d}{d t}\left(D \phi_{t}(x)\right)$ $=D F\left(\phi_{t}(x)\right) \cdot D \phi_{t}(x)$ and $D \phi_{0}(x)=I d$, (see [Hirsch-Smale (1974). p. 300] ; here $I d \in L\left(\mathbb{R}^{n}\right)$ denotes the identity operator. Hence $D \phi_{t}(x)=I d+\int_{0}^{t} D F\left(\phi_{s}(x)\right) \cdot D \phi_{s}(x) d s \quad$. Consequently $\left\|D \phi_{t}(x)\right\| \leq 1+\int_{0}^{t} M \cdot\left\|D \phi_{s}(x)\right\| d s$ for all $t \geqq 0$ and $x \in \mathbb{R}^{n}$; where $M:=\sup _{x \in \mathbb{R}^{n}}\|D F(x)\|_{1}^{\prime}<\infty$ by hypothesis. Hence by Gronwall's inequality, $\left\|D \phi_{t}(x)\right\| \leq e^{M t} \quad(t \geq 0)$ for all $x \in \mathbb{R}^{n}$. Now let $f \in D_{0}$, $t \geqq 0$. Then [grad (fo申 $\left.\left.{ }_{t}\right)\right](x)=\left[\left(g r a d\right.\right.$ f) $\left.\left(\phi_{t}(x)\right)\right] \cdot D \phi_{t}(x)$. Hence $\left\|\left[\operatorname{grad}\left(f \circ \phi_{t}\right)\right](x)\right\| \leqq e^{M t}\left\|(g r a d f)\left(\phi_{t}(x)\right)\right\|$, and so $\lim _{\|x\| \rightarrow \infty}\left\|\left[g r a d\left(f \circ \phi_{t}\right)\right](x)\right\| \leqq e^{M t} \lim \|x\| \rightarrow \infty\left\|(g r a d f)\left(\phi_{t}(x)\right)\right\|=0$. Thus $f \circ \phi_{t} \in D_{0}$ for all $t \geqq 0$.

As a second class of examples we consider derivations on $c_{0}(a, b)$. Eventually we will determine all derivations on $c_{0}(a, b)$, which are generators of a group. We start by looking at differential operators of first order. Let $-\infty \leqq a<b \leq \infty$ and let $m:(a, b)+\mathbb{R}$ be a continuous function. We consider the operator $\delta_{m}$ on $c_{0}(a, b)$ given by $\left(\delta_{m} f\right)(x)= \begin{cases}m(x) f^{\prime}(x) & \text { if m(x) } \neq 0 \\ 0 & \text { otherwise }\end{cases}$\\
with domain $D\left(\delta_{m}\right)=\left\{f \in C_{0}(a, b): f\right.$ is differentiable in $x$ if $m(x) \neq 0$ and $\left.\delta_{m} \pounds \in C_{0}(a, b)\right\}$.\\
Note that $\delta_{\mathrm{m}}$ is a derivation on $C_{0}(a, b)$.

Definition 3.16. A function $m:(a, b) \rightarrow \mathbb{R}$ is admissible if it is continuous and the following holds.\\
Whenever $a \leqq c<d \leqq b$ such that $m(x) \neq 0$ for $x \in(c, d)$ and $m(c)=0$ or $c=a=-\infty$ and $m(d)=0$ or $d=b=+\infty$, then $\int_{c}^{z} 1 /|m(x)| d x=\int_{z}^{d} 1 /|m(x)| d x=\infty$ for $z \in(c, d)$.

Note: If $m$ is admissible and $a>-\infty$, then $m(a)=0$; similary, if $b<\infty$, then $m(b)=0$. Moreover every Lipschitz continuous function is admissible.

Theorem 3.17. Let $\mathrm{m}:(a, b) \rightarrow \mathbb{R}$ be a continuous function. The operator $\delta_{m}$ is generator of an automorphism group on $c_{0}(a, b)$ if and only if $m$ is admissible.\\
In that case $D_{0}\left(\delta_{m}\right):=\left\{f \in D\left(\delta_{m}\right): f\right.$ is differentiable on $\left.(a, b)\right\}$ is a core of $\delta_{\mathrm{m}}$.\\
Additional properties. If $m$ is admissible, then the flow $\phi$ defining the group generated by $\delta_{\mathrm{m}}$ can be described explicitely:\\
The set $\{x \in(a, b): m(x) \neq 0\}$ is the union of a finite or countable number of disjoint intervals $\left(a_{n}, b_{n}\right)(n \in J)$. Let\\
$c_{n} \in\left(a_{n}, b_{n}\right) \quad$ and $q_{n}(x):=\int_{c_{n}}^{x} 1 / m(y) d y \quad\left(x \in\left(a_{n}, b_{n}\right), n \in J\right)$. Since $m$ is admissible, $q_{n}$ is a homeomorphism from ( $a_{n}, b_{n}$ ) onto iR . Now the flow $\phi$ is defined by\\
(3.22) $\phi(t, x)=\left\{\begin{array}{l}x \\ q_{n}^{-}\end{array}\right.$

$$
q_{n}^{-1}\left(q_{n}(x)+t\right) \quad \text { if } x \in\left(a_{n}, b_{n}\right)
$$

for all $t \in \mathbb{R}$.

We first prove a special case of Theorem 3.17.

Proposition 3.18. Suppose that $m(x) \neq 0$ for all $x \in(a, b)$. Then $\delta_{m}$ is the generator of a group on $c_{0}(a, b)$ if and only if $m$ is admissible. In that case the group generated by $\delta_{m}$ is similar to the translation group on $C_{0}(\mathbb{R})$.\\
proof. Let $q \in C^{1}(a, b)$ such that $q^{\prime}(x)=1 / m(x)$ for all $x \in(a, b)$. Then $q$ is a $c^{1}$-diffeomorphism from $(a, b)$ onto an interval ( $\left.a^{\prime}, b^{\prime}\right)$. By $V f=f \circ q$ one defines an isomorphism from $c_{0}\left(a^{\prime}, b^{\prime}\right)$ onto $c_{0}(a, b)$. Let $B$ on $c_{0}\left(a^{\prime}, b^{\prime}\right)$ be given by $B=V^{-1} \delta_{m} V$. Then $D(B)=\left\{g \in C_{0}\left(a^{\prime}, b^{\prime}\right): g \circ q \in D\left(\delta_{m}\right)\right\}$ $=\left\{g \in C_{0}\left(a^{\prime}, b^{\prime}\right) \cap c^{1}\left(a^{\prime}, b^{\prime}\right): g^{\prime} \circ q=m \cdot\left(g^{\circ} q\right)^{\prime} \in c_{0}(a, b)\right\}$ $=\left\{g \in C_{0}^{0}\left(a^{\prime}, b^{\prime}\right) \cap C^{1}\left(a^{\prime}, b^{\prime}\right): g^{\prime} \in C_{0}\left(a^{\prime}, b^{\prime}\right)\right\}$ and $\mathrm{Bg}=\mathrm{V}^{-1} \delta_{\mathrm{m}}^{\mathrm{O}} \mathrm{V}^{-1}\left(\mathrm{~m}\left(\mathrm{~g}^{\circ} \mathrm{q}\right)^{\prime}\right)=\mathrm{V}^{-1}\left(\mathrm{~g}^{\prime} \circ \mathrm{q}\right)=\mathrm{g}^{\prime}$.

Now observe that $m$ is admissible if and only if $a^{\prime}=-\infty$ and $b^{\prime}=\infty$.\\
If $a^{\prime}=-\infty$ and $b^{\prime}=\infty$, then $B$ is the generator of the translation group on $c_{0}(\mathbb{R})$. Hence also $\delta_{m}$ is the generator of a group $(T(t))_{t \in \mathbb{R}}$ on $C_{0}(a, b)$.\\
Conversely, assume that $B$ generates a group (T(t)) $t \in \mathbb{R}$. Assume that $a^{\prime}>-\infty$. Then $C_{0}\left(a^{\prime}, b^{\prime}\right)$ is a closed subspace of $c_{0}\left[a^{\prime}, b^{\prime}\right)$. Let\\
$\left(T_{1}(t) f\right)(x)= \begin{cases}f(x+t) & \text { for } x+t<b^{\prime} \\ 0 & \text { for } x+t \geqq b^{\prime}\end{cases}$\\
for all $f \in c_{0}\left[a^{\prime}, b^{\prime}\right), x \in\left[a^{\prime}, b^{\prime}\right), t \geqq 0$. Then $\left(T_{1}(t)\right)_{t \geq 0}$ is a semigroup on $C_{0}\left[a^{\prime}, b^{\prime}\right)$ with generator $B_{1}$ given by $B_{1} E=f^{\prime}$ with domain $\mathrm{D}\left(\mathrm{B}_{1}\right)=\left\{f \in c_{0}\left[a^{\prime}, b^{\prime}\right) \cap c^{1}\left(a^{\prime}, b\right): \lim _{x^{\prime} b^{\prime}} f^{\prime}(\mathrm{x})=0\right\}$. If we consider $B$ as an operator on $C_{o}\left[a^{\prime}, b^{\prime}\right)$, then $B C^{B} B_{1}$. Let $f \in D(B)$. Then $u(t):=T(t) f \in D(B) \subset D\left(B_{1}\right)$ for all $t \geqq 0$; and $u(t)=B u(t)=B_{1} u(t) ; u(0)=f$. It follows from A-I, Thm.1.7. (or A-II, Corl.2.) that $T_{1}(t) f=u(t)$. Hence $T_{1}(t) \pounds c_{0}\left(a^{\prime}, b^{\prime}\right)$ for all $t \geqq 0$ and $f \in D(B)$. This is impossible since $a^{\prime}>{ }^{\infty}$. Similary one shows that $b^{\prime}=\infty$.

Proof of Theorem 3.17. Suppose that $m$ is admissible. It is easy to see that (3.22) then defines a continuous flow on ( $a, b$. Moreover, for every $x \in(a, b)$ the function $\phi(\cdot, x)$ is differentiable and satisfies\\
(3.23) $\frac{\partial}{\partial t} \phi(t, x)=m(\phi(t, x)) \quad(x \in(a, b), t \in \mathbb{R})$.

Denote by $(T(t))_{t \in \mathbb{R}}$ the group on $C_{0}(a, b)$ given by $T(t) f=f \circ t_{t}$ $\left(t \in \mathbb{R}, f \in C_{0}(a, b)\right)$ and let $A$ be its generator. Take $g \in C_{0}(a, b)$ and $f=R(1, A) g$. Then $f(x)=\int_{0}^{\infty} e^{-t} g(\phi(t, x)) d t, x \in(a, b)$. If $m(x)=0$ then $f(x)=\int_{0}^{\infty} e^{-t} g(x) d t=g(x)$. If $x \in\left(a_{n}, b_{n}\right)(n \in J)$, then $f(x)=\int_{0}^{\infty} e^{-t} g\left(q_{n}^{-1}\left(q_{n}(x)+t\right)\right) d t=e^{q_{n}(x)} \int_{q_{n}}^{\infty}(x) e^{-s_{n}^{n}}\left(q_{n}^{-1}(s)\right) d s$. Thus $f$ is differentiable in $x$ and $f^{\prime}(x)=(1 / m(x))(f(x)-g(x))$. Consequently $\pounds \in D\left(\delta_{m}\right)$ and $\delta_{m} \pounds=f-g$. This shows that $A \subset \delta_{m}$. In order to show the converse inclusion, let $f \in D\left(\delta_{m}\right)$ and $g=f-$ $\delta_{m}(f) \in C_{0}(a, b)$ Then $R(1, A) g(x)=f(x)$ if $m(x)=0$ and $R(1, A) g(x)=\int_{0}^{\infty} e^{-t} f(\phi(t, x)) d t-\int_{0}^{\infty} e^{-t} m(\phi(t, x)) f^{\prime}(\phi(t, x)) d t$ $=\int_{0}^{\infty} e^{-t} f(\phi(t, x)) d t-\int_{0}^{\infty} e^{-t} \frac{\partial}{\partial t} f(\phi(t, x)) d t \quad$ (by (3.23)) $=f(x)$ by integrating by parts. This shows that $f=R(1, A) g \in D(A)$ and that $\delta_{m}$ is the generator of the group $(T(t))_{t \in \mathbb{R}}$. Finally we show that $T(t) D_{0}\left(\delta_{m}\right) \subset D_{0}\left(\delta_{m}\right)$, which implies that $D_{0}\left(\delta_{m}\right)$ is a core (by A-II, Cor 1.34). Let $t \in \mathbb{P}$. The function $\phi_{t}{ }^{(\cdot)}$ is\\
differentiable on $(a, b)$ and $m(x) \frac{\partial}{\partial x} \phi(t, x)=m(\phi(t, x))$ for all $x \in$ $(a, b)$. Let $f \in D_{0}\left(\delta_{m}\right)=D\left(\delta_{m}\right) \cap C^{1}, t \in \mathbb{R}$. Then $T(t) f=f \circ \phi_{t}$ is differentiable and so in $D_{0}\left(\delta_{m}\right)$.

Conversely, assume that $\delta_{m}$ is generator of a group $(T(t))_{t \in \mathbb{R}}$ on $c_{0}(a, b)$. Since $\delta_{\mathrm{m}}$ is a derivation, there exists a continuous flow $\left(\phi_{t}\right)_{t \in R}$ on ( $a, b$ ) such that $T(t) f=f 0 \phi_{t}$ for all $f \in c_{0}(a, b)$, $t \in \mathbb{R}$. In order to show that $m$ is admissible let $a \leqq c<d \leqq b$ such that $m(x) \neq 0$ for all $x \in(c, d)$ and $m(c)=0$ or $a=c=-\infty$ and $\mathrm{m}(\mathrm{d})=0$ or $\mathrm{d}=\mathrm{b}=\infty$.\\
If $a<c$ then $m(c)=0$; consequently $\left(\delta_{m} f\right)(c)=0$ for aJ.l\\
$f \in D\left(\delta_{m}\right)$. Thus $(T(t) f)(c)=f(c)$ for all $f \in D\left(\delta_{m}\right)$ and $t \in \mathbb{R}$. This shows that $\phi(t, c)=c$ for $a l l \in \mathbb{R}$. Consequently $\phi_{t}(a, c) \subset(a, c)$ for all $t \in \mathbb{R}$. Similary $\phi_{t}(a, b) \subset(d, b)$ for all $t \in \mathbb{R}$. Thus the space $E_{0}:=\left\{f \in C_{0}(a, b): \pounds\right.$ vanishes off $\left.(c, d)\right\}$ is invariant under the group $(\mathrm{T}(\mathrm{t}))_{t \in \mathbb{R}}$. We denote the group restricted to $E_{0}$ by $\left(T_{0}(t)\right)_{t \in \mathbb{R}}$ and by $A_{0}$ its generator. Then $D\left(A_{0}\right)=$ $=\left\{f \in E_{0} \cap D\left(\delta_{m}\right): \delta_{m} f \in E_{O}\right\}$. Identifying $E_{0}$ with $C_{0}(c, d)$ we obtain $A_{0}=\delta_{m}$, where $m^{\prime}$ denotes the restriction of $m$ to (c,d) . So it follows from Prop. 3.18 that $\mathrm{m}^{\prime}$ is admissible.

Remark 3.19. If $\phi$ is a flow on $(a, b)$, a point $x \in(a, b)$ is called stationary if $\phi(t, x)=x$ for all $t \in \mathbb{R}$. Let $\delta$ be the generator of the group $(\mathrm{T}(t))_{t \in \mathbb{R}}$ associated with $\phi$. Then $x \in(a, b)$ is a stationary point if and only if $(\delta f)(x)=0$ for all $f \in D(\delta)$. If $m$ is an admissible function on ( $a, b$ ) then we have seen that $x \in(a, b)$ is a stationary point of the flow associated with $\delta_{m}$ if and only if $\mathrm{m}(\mathrm{x})=0$. This does no longer hold for functions which are not admissible as the following example shows.

Example 3.20. Consider the flow $\phi(t, x)=\left(x^{1 / 3}+t\right)^{3}$ on $\mathbb{R}$ and the group $(T(t))_{t \in \mathbb{R}}$ induced by this flow on $C_{0}(\mathbb{R})$. One can easily see that the generator $\delta$ of $(T(t))_{t \in \mathbb{R}}$ is the following operator. Let $m(x)=3 x^{2 / 3}$. Then (ff) $(x)=m(x) f^{\prime}(x)$ for $x \neq 0$ and\\
$D(\delta)=\left\{f \in C_{0}(\mathbb{R}): E\right.$ is differentiable in $x \neq 0$ and $m(x) f^{\prime}(x)$ has a continuous extension in $\left.C_{0}(\mathbb{R})\right\}$. However the function $m$ is not admissible. And in fact $\mathrm{m}(0)=0$ but 0 is not a stationary point of $\phi$. In particular, there exists a function $\pounds \in D(\delta)$ such that $(\delta f)(0) \neq 0$.

Next we describe an arbitrary continuous flow on an open interval.

Proposition 3.21. Let $-\infty \leqq \mathrm{a}<\mathrm{b} \leqq \infty$. A mapping\\
$\phi: \mathbb{R} \times(a, b) \rightarrow(a, b)$ defines a continuous flow if and only if there exists a finite or countable set of disjoint intervals\\
$\left(a_{n}, b_{n}\right) \subset(a, b) \quad(n \in J)$ and for every $n \in J$ there exists $a$ homeomorphism $r_{n}$ from $\left(a_{n}, b_{n}\right)$ onto $(-\infty, \infty)$ such that\\
$\phi(t, x)= \begin{cases}x & \text { if } x \notin \cup_{n \in J}\left(a_{n}, b_{n}\right) \\ r_{n}^{-1}\left(r_{n}(x)+t\right) & \text { if } x \in\left(a_{n}, b_{n}\right), n \in J\end{cases}$\\
for all $t \in \mathbb{R}$\\
Note: $J=\emptyset$ if and only if $\phi(t, x)=x$ for all $x \in(a, b)$ and $t \in \mathbb{R}$.

Proof. It is not difficult to see that the construction in the proposition defines a continuous flow on (a,b). Now let $\phi$ be a continuous flow. The set $K=\{x \in(a, b): \phi(t, x)=x$ for all $t \in \mathbb{R}\}$ is closed in $(a, b)$. Thus $(a, b) \backslash k$ is the union of a finite or countable set of disjoint intervals ( $a_{n}, b_{n}$ ), (n € J) . Pick $x_{n} \in\left(a_{n}, b_{n}\right),(n \in J)$. Then $\alpha_{n}(t):=\phi\left(t, x_{n}\right)$ defines an injective mapping from $\mathbb{R}$ into $\left(a_{n}, b_{n}\right)$. Thus $\alpha_{n}$ is strictly monotonous. It is easy to see that $\lim _{t \rightarrow \infty} \phi\left(t, x_{n}\right)$ is an element of K whenever the limit exists in $(a, b)$; similary for the limit as $t \rightarrow-\infty$. Consequently, $a_{n}(R)=\left(a_{n}, b_{n}\right)$; i.e., $a_{n}$ is a homeomorphism from $\mathbb{R}$ onto $\left(a_{n}, b_{n}\right)$. Define $r_{n}$ to be the inverse of $a_{n}$. Let $x \in\left(a_{n}, b_{n}\right)$. Then $\phi(t, x)=\phi\left(t, a_{n}\left(r_{n}(x)\right)\right)$ $=\phi\left(t, \phi\left(r_{n}(x), x_{n}\right)\right)=\phi\left(t+r_{n}(x), x_{n}\right)=\alpha_{n}\left(t+r_{n}(x)\right)=r_{n}^{-1}\left(r_{n}(x)+t\right)$ for all $t \in \mathbb{R}$. This proves that $\phi$ has the desired form.

If $m$ is an admissible function on $(a, b)$, then $D(\delta)$ contains many differentiable functions. This can be expressed by two facts:\\
a) $C_{c}^{1}(a, b):=\left\{f \in C^{1}(a, b): E\right.$ vanishes in a neighbourhood of $a$ and $b$ ) is contained in $D\left(\delta_{m}\right)$ (this follows from the definition of $\delta_{m}$, ; and\\
b) the set $D_{0}\left(\delta_{m}\right)$ of all differentiable functions in $D_{0}\left(\delta_{m}\right)$ is a core of $\delta_{\mathrm{m}}$ (this follows from Theorem 3.17).\\
We will show below that these two properties are characteristic for the operators $\delta_{\mathrm{m}}$. For other generators of automorphism groups they can be violated dramatically as the following example shows.

Example 3.22. There exists a generator $\delta$ of an automorphism group on $C_{0}(\mathbb{R})$ such that $D(\delta) \cap C^{1}(\mathbb{R})=\{0\}$. In fact, consider a strictly increasing continuous map $q$ from $\mathbb{R}$ onto $\mathbb{R}$ such that\\
$q^{\prime}(x)=0$ a.e. Then $v: c_{0}(\mathbb{R}) \rightarrow c_{0}(\mathbb{R})$ given by $v f=f \circ q$ is an algebra isomorphism. Let $A$ be the generator of the translation group on $C_{0}(\mathbb{R})$ and $\delta=V^{-1} A V$. Then $D(\delta)=\left\{f \in C_{0}(\mathbb{R}): V f \in D(A)\right\}=$ $\left\{f \in C_{0}(\mathbb{R}): f \circ q \in C^{1}(\mathbb{R}),(f \circ q), \in C_{0}(\mathbb{R})\right\}$.\\
Let $f \in C^{1}(\mathbb{R}) \cap D(\delta)$. If $f \neq 0$, then $\pounds$ is not constant. Hence there exists $x_{0} \in \mathbb{R}$ such that $f^{\prime}\left(x_{0}\right) \neq 0$. Then $f$ has a continuously differentiable inverse in some open neighbourhood of $x_{0}$. Since foq $\in C^{1}(\mathbb{R})$, it follows that $q$ is continuously differentiable in some neighborhood of $q^{-1}\left(x_{0}\right)$. This is a contradiction since $q^{\prime}(y)=0$ a.e.

Theorem 3.23. Let $\delta$ be the generator of an automorphism group on $C_{0}((a, b))$, where $-\infty \leqq a<b \leqq \infty$. The following assertions are equivalent.\\
(i) There exists a continuous admissible function $m:(a, b) \rightarrow \mathbb{R}$ such that $\delta=\delta_{\mathrm{m}}$.\\
(ii) $C_{C}^{1}(a, b) \subset D(\delta)$ and $D_{o}(\delta)=\{f \in D(\delta)$ : $f$ is differentiable \} is a core of $\delta$.

Proof. We have already pointed out that (i) implies (ii).\\
So assume that (ii) holds. Let $(T(t))_{t \in \mathbb{R}}$ be the group generated by $\delta$ and $\phi$ the continuous flow associated with the group. We can assume that $\phi$ is of the form given in Prop. 3.21.\\
Let $n \in J$. We show that $r_{n}^{-1}: \mathbb{R} \rightarrow\left(a_{n}, b_{n}\right)$ is continuously differentiable. Let $x_{0} \in\left(a_{n}, b_{n}\right)^{n}$. There exists $f \in c_{c}^{1}(a, b)$ such that $f(x)=x$ in a neighborhood of $x_{0}$. Then $r_{n}^{-1}\left(r_{n}\left(x_{0}\right)+t\right)$\\
$=f\left(\phi\left(t, x_{0}\right)\right)=(T(t) f)\left(x_{0}\right)$ for all $t$ in some neighborhood of 0 . since $f \in D(\delta)$ it follows that the function $t \rightarrow r_{n}^{-1}\left(r_{n}\left(x_{0}\right)+t\right)$ is continuously differentiable in some neighborhood of ${ }^{n} 0$ and so $r_{n}^{-1}$ is continuously differentiable in $r_{n}\left(x_{0}\right)$. since $r_{n}:\left(a_{n}, b_{n}\right) \rightarrow \mathbb{R}$ is surjective this proves the claim.\\
Next we show $\left(r_{n}^{-1}\right)^{\prime}(t) \neq 0$ for all $t \in \mathbb{R}$. In fact, let $x_{0} \in\left(a_{n}, b_{n}\right)$ and assume that $\left(x_{n}^{-1}\right) \cdot\left(r_{n}\left(x_{0}\right)\right)=0$. Then for all $f \in D_{o}(\delta)$ one has $(\delta f)\left(x_{0}\right)=\left.\frac{\partial}{\partial t}\right|_{t=0} f\left(r_{n}^{-1}\left(r_{n}\left(x_{0}\right)+t\right)\right)=f^{\prime}\left(x_{0}\right)\left(r_{n}^{-1}\right) \cdot\left(r_{n}\left(x_{0}\right)\right)=0$. since $D_{0}(\delta)$ is a core of $\delta$ this implies that $\phi\left(t, x_{0}\right)=x_{0}$ for all $t \in \mathbb{R}$. Hence $x_{0} \in \mathrm{~K}$, a contradiction. It follows that $r_{n}:\left(a_{n}, b_{n}\right) \rightarrow \mathbb{R}$ is a $c^{1}$-diffeomorphism for all $n \in J$.

Define $m:(a, b) \rightarrow \mathbb{R}$ by

$$
m(x)= \begin{cases}0 & \text { if } x \in k \\ l / r_{n}^{\prime}(x) & \text { if } x \in\left(a_{n}, b_{n}\right), n \in J\end{cases}
$$

Then $m$ is continuous and admissible. The given flow coincides with the one constructed from $m$ in Theorem 3.17 . Thus $\delta=\delta_{\mathrm{m}}$.

Remark. Let $m:(a, b) \rightarrow \mathbb{R}$ be continuous. Then $m$ is admissible if and only if the initial value problem

$$
\dot{y}(t)=m(y(t)) \quad(t \in \mathbb{R}) ; \quad y(0)=x
$$

has a unique solution $y \in C^{1}(\mathbb{R},(a, b))$ which depends continuously on the initial value $x$ (i.e., if $x_{n} \rightarrow x$ in $(a, b)$ then the solution $y_{n} \in C^{1}(\mathbb{R},(a, b))$ with initial value $y_{n}(0)=x_{n}$ satisfies\\
$y_{n}(t) \rightarrow Y(t) \quad(n \rightarrow \infty)$ for all $\left.t \in \mathbb{R}\right)$. This is not difficult to see.\\
As we have seen above the operators $\delta_{\mathrm{m}}$, where m is an admissible function, do not exhaust all generators of automorphism groups. But one can obtain every such generator by a similarity transformation (see A-I, 3.0) from some $\delta_{\mathrm{m}}$.

Theorem 3.24. Let $-\infty \leqq \mathrm{a}<\mathrm{b} \leqq \infty$. An operator $\delta$ on $\mathrm{C}_{0}(\mathrm{a}, \mathrm{b})$ is the generator of an automorphism group on $c_{0}(a, b)$ if and only if there exists an algebra isomorphism $v$ from $c_{0}(a, b)$ onto $c_{0}(a, b)$ and an admissible function $\mathrm{m}:(\mathrm{a}, \mathrm{b}) \rightarrow \mathbb{R}$ such that $\delta=\mathrm{V}^{-1} \delta_{\mathrm{m}} \mathrm{V}$.

Proof. In order to prove the non-trivial implication let (T) $(t))_{t \in \mathbb{R}}$ be an automorphism group on $C_{0}(a, b)$ with generator $\delta$, Let $\phi$ be the continuous flow on $(a, b)$ such that $T(t) \pounds=\pounds \circ \phi_{t}\left(\pounds \in C_{0}(a, b)\right.$, $t \in \mathbb{R})$. Then $\phi$ is of the form given in Prop. 3.21. For every $n \in J$ choose a $c^{1}$-diffeomorphism $q_{n}$ from $\left(a_{n}, b_{n}\right)$ onto $(-\infty, \infty)$ satisfying $q_{n}^{\prime}(x)>0$ for all $x \in\left(a_{n}, b_{n}\right)$ in the case when $r_{n}$ is increasing and $q_{n}^{\prime}(x)<0$ for all $x \in\left(a_{n}, b_{n}\right)$ in the case when $r_{n}$ is decreasing. Then $B_{n}:=x_{n}^{-1} \circ q_{n}$ is a homeomorphism from ( $a_{n}, b_{n}$ ) onto itself satisfying $\lim _{x \downarrow a_{n}} \beta_{n}(x)=a_{n}$ and $\lim _{x \uparrow b_{n}} \beta_{n}(x)=b_{n}$.\\
Let $\beta:(a, b) \rightarrow(a, b)$ be defined by\\
$\beta(x)= \begin{cases}x & \text { if } x \in k \\ \beta_{n}(x) & \text { if } x \in\left(a_{n}, b_{n}\right), n \in J .\end{cases}$\\
Then $\beta$ is a homeomorphism from $(a, b)$ onto $(a, b)$ and $\psi_{t}:=\beta^{-1} \circ \phi_{t} \circ \beta \quad(t \in \mathbb{R})$ defines a continuous flow on $(a, b)$.

Define $m:(a, b) \rightarrow \mathbb{R}$ by\\
$m(x)= \begin{cases}0 & \text { if } x \in K \\ 1 / q_{n}^{\prime}(x) & \text { if } x \in\left(a_{n}, b_{n}\right)\end{cases}$\\
Then $m$ is continuous and admissible and the flow $\psi$ coincides with the flow constructe from m in Theorem 3.17. Hence $\delta_{m}$ is the generator of the group $(S(t))_{t \in \mathbb{R}}$ given by $S(t) f=f \circ \psi_{t}=f \circ B^{-1} \circ \phi_{t} \circ B$ $=\operatorname{VT}(t) \mathrm{V}^{-1} f$, where $V$ is the isomorphism on $c_{0}(a, b)$ given by $\mathrm{Vf}=\mathrm{f} \circ \mathrm{B}$. Consequently, $\delta=\mathrm{v}^{-1} \delta_{\mathrm{m}} \mathrm{V}$.

Now we are able to describe arbitrary generators of positive groups on $c_{0}(a, b)$.

Theorem 3.25. Let $-\infty \leqq \mathrm{a}<\mathrm{b} \leqq \infty$. An operator A generates a positive group on $c_{0}(a, b)$ if and only if there exist

\begin{itemize}
  \item a lattice isomorphism $V$ on $C_{0}(a, b)$,
  \item an admissible function $m$ on (a,b),
  \item a bounded continuous function $h:(a, b) \rightarrow \mathbb{R}$ such that
\end{itemize}


\begin{equation*}
A=V^{-1} \delta_{m} V+h \tag{3.24}
\end{equation*}


Proof. Let $A$ be the generator of a positive group on $c_{0}(a, b)$. By Theorem 3.14 there exist a continuous bounded function $p:(a, b) \rightarrow \mathbb{R}$ such that inf $x \in(a, b) p(x)>0$ and $h \in C^{b}(a, b)$ and the generator $\delta$ of an automorphism group such that $A=M \delta M^{-1}+h$ where $M \in L\left(C_{O}(a, b)\right)$ is given by $M f=p \cdot f$. By Theorem 3.24 there exist an admissible continuous function $m:(a, b) \rightarrow \mathbb{R}$ and a lattice isomorphism $U \in L\left(C_{0}(a, b)\right)$ such that $\delta=U \delta_{m} U^{-1}$. setting $V=M U$ we obtain $A=V \delta_{m} V^{-1}+h$.

Finally we consider compact intervals. Let $-\infty \leqq \mathrm{a}<\mathrm{b} \leqq \infty$ and $\phi$ be a continuous flow on $[a, b]$. Then it is easy to see that $\phi(a, t)=a$ and $\phi(b, t)=b$ for $a l l$ t $t \mathbb{R}$. So the restriction $\phi_{0}$ of $\phi$ to $(a, b)$ is a continuous flow on $(a, b)$.\\
Conversely, if $\phi_{0}$ is a continuous flow on $(a, b)$ the extension $\phi_{0}$ to $\phi: \mathbb{R} x[a, b] \rightarrow[a, b]$ by setting $\phi(t, a)=a ; \phi(t, b)=b$ for all $t \in \mathbb{R}$ defines a continuous flow on [a,b]. This consideration allows us to extend easily the preceding results to the space c[a,b]. Let $m:(a, b) \rightarrow \mathbb{R}$ be a continuous function. We define the operator $\tilde{\delta}_{\mathrm{m}}$ on $\mathrm{C}[\mathrm{a}, \mathrm{b}]$ by $\tilde{\delta}_{\mathrm{m}} \mathrm{f}=\mathrm{g}$ such that\\
(3.25) $g(x)= \begin{cases}m(x) f^{\prime}(x) & \text { if } m(x) \neq 0 \\ 0 & \text { if } m(x)=0\end{cases}$\\
for all $x \in(a, b)$ and $D\left(\tilde{\delta}_{m}\right)=\{f \in C[a, b]: f$ is differentiable in $x \in(a, b)$ whenever $m(x) \neq 0$ and there exists a (necessarily unique) $g \in C[a, b]$ such that (3.25) holds $\}$.

Theorem 3.26. Let $m$ be a continuous function on ( $a, b$. The operator $\tilde{\delta}_{m}$ is generator of an automorphism group on $c[a, b]$ if and only if $m$ is admissible.

Proof. If $\tilde{\delta}_{m}$ generates an automorphism group (T) $\left.(t)\right)_{t \in \mathbb{R}}$ then by the remark above $T(t) C_{0}(a, b)=c_{0}(a, b)$ ( $\in \mathbb{R}$ ). The generator of the restricted group has the domain\\
$\left\{f \in C_{0}(a, b) \cap D\left(\tilde{\delta}_{m}\right): \tilde{\delta}_{m} f \in C_{0}(a, b)\right\}=D\left(\delta_{m}\right)$. Hence $\delta_{m}$ is a generator and so m is admissible by Theorem 3.17. Conversely, if m is admissible, then $\delta_{m}$ generates a group on $c_{0}(a, b)$ given by a flow $\phi_{0}$ on $(a, b)$. Extending $\phi_{0}$ to $[a, b]$ as above one obtains $a$ continuous flow $\phi$ on [a,b] which defines a group (T(t)) $t \in \mathbb{R}$. It is easy to verify, that the generator of this group is $\tilde{\delta}_{\mathrm{m}}$.

Theorem 3.27. Let $\delta$ be the generator of an automorphism group on $c[a, b]$. Then there exists an admissible function $m:(a, b) \rightarrow \mathbb{R}$ and an algebra isomorphism $v$ from $c[a, b]$ onto $c[a, b]$ such that $\delta=\mathrm{V}^{-1} \tilde{\delta}_{\mathrm{m}} \mathrm{V}$.

Proof. The restriction $\delta_{0}$ of $\delta$ to $c_{0}(a, b)$ is the generator of an automorphism group. Thus by Theorem 3.24 there exists a continuous admissible function $m:(a, b) \rightarrow \mathbb{R}$ and an algebra isomorphism $V_{0}$ from $c_{0}(a, b)$ onto $c_{0}(a, b)$ such that $\delta_{0}=V_{0}^{-1} \delta_{m} V_{0}$. Let $V$ be the unique algebra isomorphism on $C[a, b]$ which extends $V_{0}$. Then it is easy to see that $\delta=\mathrm{V}^{-1} \tilde{\delta}_{\mathrm{m}} \mathrm{V}$.

Theorem 3.28. An operator $A$ on $C[a, b]$ is generator of a positive group on C[a,b] if and only if there exist

\begin{itemize}
  \item a lattice isomorphism $V$ on $C[a, b]$
  \item an admissible function $m:(a, b) \rightarrow \mathbb{R}$
  \item and a function $h \in C[a, b]$ such that $A=V^{-1} \tilde{\delta}_{m} V+h$.
\end{itemize}

The proof follows from theorem 3.14 via Theorem 3.27 in the same way as Theorem 3.25 (via Theorem 3.24).

NOTES.\\
Section 1, Concerning bounded generators of positive semigroups and the positive minimum principle we refer to the corresponding notes in Chapter C-II.\\
Theorem 1.6 and 1.8 are due to Arendt-Chernoff-Kato (1982), but we give a more direct proof here. Theorem 1.13 and its corollary are from the same source.\\
In the case when $A$ is dissipative Theorem 1.20 is due to Dorroh (1966). We use precisely Dorroh's arguments to verify the range condition. Other extensions of Dorroh's result have been given by Lumer (1974) and Lumer (1975).

Section 2. A characterization of generators of lattice semigroups by Kato's equality is due to Nagel-Uhlig (1981) if the underlying space has order continuous norm (see C-II, Sec.5), for general Banach spaces and $C$ (X) in particular the problem has been considered in Arendt (1982). Theorem 2.10 is due to Uhlig (1979).

Section 3. The characterization of generators of lattice semigroups as perturbation of a derivation (Theorem 3.5 and 3.6 ) is due to Derndinger-Nagel (1979). The corresponding result for positive groups on $C(X)$ (Theorem 3.14 ) was obtained by Arendt-Greiner (1984). Lin-Montgomery-Sine ( 1977 ) consider multiplicative perturbations of a generator of an automorphism group on $C(K)$ ( K compact) by a function m which has a finite number of zeros. The function $m$ is assumed to satisfy the "generalized Osgood condition" which is similar to being admissable (in our sense) but in addition the given flow is involved in the definition.\\
Batty (1981) determined all densely defined derivations $\hat{\delta}$ on c[0,1] which are well-behaved (i.e., $\pm \delta$ is dispersive) by a representation similar to Theorem 3.24. In contrast to Batty, here we assume that $\delta$ is the generator of a group. This simplifies the matter considerably since all continuous flows on an interval are easy to determine (Prop, 3.21). Our approach is inspired by deLaubenfels (1984) to whon Theorem 3.24 is due.\\
For simplicity we confined ourselves to groups. Uhlig (1979) determined all semiflows on an interval.\\[0pt]
In the sequel of Batty's work (loc.cit.) a characterization of all densely defined closed derivations on c[0,1] has been obtained by Kurose in a series of papers (1981), (1982), (1983).

\section*{CHAPTER B-III}
SPECTRALTHEORYOF\\
\includegraphics[max width=\textwidth, center]{2024_12_23_c6487cc0859199a15bd9g-173}\\
by\\
Günther Greiner

It is known that for a single operator $T \in L\left(C_{O}(X)\right)$ the positivity of $T$ has influence on the spectrum of $T$, mainly on the peripheral spectrum, i.e. the part of the spectrum containing all spectral values of maximal absolute value. This part of the spectrum is of interest because it determines the asymptotic behavior of the iterates $\mathrm{T}^{\mathrm{n}}$ for large $n \in \mathbb{N}$. The spectral properties indicated above were first proved by Perron (1907) and Frobenius (1909) for positive square matrices, i.e. for positive operators on the Banach lattice $\mathbb{C}^{n}$. Later these results were extended to the infinite dimensional setting; important contributions are due to Jentzsch, Karlin, Krein, Krasnoselski'i, Lotz, Rota, Rutman, Schaefer and others (see Chapt.V of Schaefer (1974)).\\
In this chapter we investigate the spectrum o(A) of the generator A of a positive semigroup $T=(T(t))_{t \geqslant 0}$ on the Banach space $C_{0}(x)$, Throughout this chapter we assume that $C_{0}(X)$ is the space of all complex-valued functions on the locally compact space $X$. In case we restrict to compact spaces we write K instead of X .

\section*{1. THE SPECTRAL BOUND}
One of the basic results on the spectrum of a positive operator is the fact that its spectral radius is an element of the spectrum (see V.Prop.4.1 of Schaefer (1974)). We begin the investigation of the spectrum of positive semigroups with the analogous result. To that purpose we recall that the spectral bound $s(A)$ of a generator $A$ is defined as the least upper bound of the real parts of all spectral values (Cf. A-III,(1.2)).

Theorem 1.1. If $A$ is the generator of a positive semigroup on $E=c_{0}(X)$, then $s(A) \in \sigma(A)$ unless $s(A)=-\infty$. In case X is compact we always have $\mathrm{s}(\mathrm{A})>-\infty$.

Proof. We suppose $\sigma(A) \neq \emptyset$ (i.e., $s(A)>-\infty$ and assume $s(A)$ Then there exist $\varepsilon>0$ and $\alpha_{0}, \beta_{0} \in \mathbb{R}$ such that\\
(1.1) $[s(A)-\varepsilon, \infty) \subset \rho(A), \mu_{0}:=\alpha_{0}+i \beta_{0} \in \sigma(A)$ and $\alpha_{0}>s(A)-\varepsilon$. Now we choose $\lambda_{0} \in \mathbb{R}$ large enough such that\\
(1.2) $\left|\lambda_{0}-\mu_{0}\right|<\lambda_{0}-(s(A)-\varepsilon)$,\\
and, in addition, such that $\lambda_{0}>\omega(A)$. Then the resolvent $R\left(\lambda_{0}, A\right)$ is a positive bounded operator, hence its spectral radius $r\left(R\left(\lambda_{0}, A\right)\right)$ is a spectral value. From A-III, Prop. 2.5 it follows that\\
(1.3) $\quad \lambda_{0}-r\left(R\left(\lambda_{0}, A\right)\right)^{-1} \in \sigma(A) \quad$ and $\quad r\left(R\left(\lambda_{O}, A\right)\right) \geq\left|\lambda_{0}-\mu_{0}\right|^{-1}$.

This and (1.2) implies that $\lambda_{0}-r\left(R\left(\lambda_{0}, A\right)\right)^{-1}$ is a real spectral value which is greater than $s(A)-E$. We have derived a contradiction to (1.1) and thus have proved the first statement of the theorem. To establish the second statement we recall that $\lim _{\lambda \rightarrow \infty} \lambda R(\lambda, A) f=f$ for every $f \in E$. In particular, for $E=1_{X}$ we find a (large) $\lambda_{0} \in \mathbb{R}$ such that\\
(1.4) $\lambda_{0} R\left(\lambda_{0}, A\right) 1_{X} \geqq 1 / 2 \cdot 1_{X} \quad$ hence $\quad R\left(\lambda_{O}, A\right) 1_{X} \geqq\left(2 \lambda_{0}\right)^{-1} \cdot 1_{X}$. We may assume $\lambda_{0}>\omega(A)$ then $R\left(\lambda_{0}, A\right) \geqq 0$, and iterating (1.4) we obtain\\
(1.5) $R\left(\lambda_{0}, A\right)^{n} 1_{X} \geqq\left(2 \lambda_{0}\right)^{-n} \cdot 1_{X}>0$ for every $n \in \mathbb{N}$.

It follows that $\left\|R\left(\lambda_{0}, A\right)^{n}\right\| \geq\left(2 \lambda_{0}\right)^{-n}$ and therefore\\
(1.6) $r\left(R\left(\lambda_{0}, A\right)\right)=\lim _{n \rightarrow \infty}\left\|R\left(\lambda_{0}, A\right)^{n}\right\|^{1 / n} \geqq\left(2 \lambda_{0}\right)^{-1}>0$.

Thus $\sigma\left(\mathrm{R}\left(\lambda_{0}, A\right)\right)$ contains non-zero spectral values which in view of A-III, Prop. 2.5 is equivalent to $\sigma(A) \neq \emptyset$.

The following examples show that the spectrum may be empty in case x is not compact or if the semigroup is not positive.

Examples 1.2.(a) Consider $X=[0,1)$ and $(T(t))$ on $C_{0}(X)$ given by\\
(1.7) (T(t)f)(x):=\{ $\begin{array}{cll}f(x+t) & \text { if } & x+t<1 \\ 0 & \text { if } & x+t \geqq 1 .\end{array}$.

Then ( $T(t))_{t \geqq 0}$ is nilpotent (we have $T(t)=0$ for $t \geqq 1$ ). It follows that $\sigma(T(t))=\{0\}$ for all $t>0$ and by A-III,Thm.6.2 we have $\sigma(\mathrm{A})=\varnothing$.\\
(b) The operator A on $\mathrm{E}:=\mathrm{C}_{0}[0, \infty)$ given by\\
(1.8) (Af) $(x)=f^{\prime}(x)-x f(x), D(A)=\left\{f \in E: f \in C^{1}, A f \in E\right\}$\\
has empty spectrum. It is the generator of a positive non-nilpotent semigroup which is given by\\
(1.9) $(T(t) f)(x)=\exp \left(-\left(t^{2} / 2\right)-x t\right) \cdot f(x+t)$.\\
(c) Taking into account that $C_{0}([0,1))$ as well as $C_{0}([0, \infty))$ both are topologically (but not isometrically) isomorphic to $\mathrm{C}([0,1])$ (see Semadeni (1971), sec.21.5), one obtains from (a) and (b) (non-positive) semigroups on $C([0,1])$ whose generators have empty spectrum.

The proof of Thm.1.1 given above is based on the fact that the spectral radius of a bounded positive operator is an element of the spectrum. A direct proof not using this fact is given in C-III, Cor.1.4.

Corollary 1.3. Suppose $\lambda_{0} \in \rho(A)$. Then $R\left(\lambda_{0}, A\right)$ is a positive operator if and only if $\lambda_{0}>s(A)$.\\
For $\lambda>s(A)$ we have $r(R(\lambda, A))=(\lambda-s(A))^{-1}$.

Proof. The second statement is an immediate consequence of Thm.1.1 and A-III, Prop. 2.5 .\\
Given $\lambda_{0}>s(A)$ we choose $\lambda_{1}>\max \left\{\lambda_{0}, \omega(A)\right\}$. Since $\left|\lambda_{1}-\lambda_{0}\right|<\left|\lambda_{1}-s(A)\right|=r\left(R\left(\lambda_{1}, A\right)\right)^{-1}$ we have


\begin{equation*}
R\left(\lambda_{0}, A\right)=\sum_{n=0}^{\infty}\left(\lambda_{1}-\lambda_{0}\right)^{n} \cdot R\left(\lambda_{1}, A\right)^{n+1} \tag{1.10}
\end{equation*}


Since $R\left(\lambda_{1}, A\right)$ is positive, it follows that $R\left(\lambda_{0}, A\right)$ is positive as well.\\
On the other hand, assuming that $R\left(\lambda_{0}, A\right)$ is a positive operator, then $\lambda_{0}$ has to be a real number (note that for $g \geqq 0$ we have $f:=R\left(\lambda_{O}, A\right) g \geqq 0$ hence $\left.\lambda_{0} f-A f=g=\bar{g}=\overline{\left(\lambda_{O}-\bar{A}\right) f}=\bar{\lambda}_{0} f-A f\right)$. As we have shown above $R(\lambda, A)$ is positive for $\lambda>\max \left\{\lambda_{0}, s(A)\right\}$ hence an application of the resolvent equation yields:\\
(1.11) $R\left(\lambda_{0}, A\right)=R(\lambda, A)+\left(\lambda-\lambda_{0}\right) R(\lambda, A) R\left(\lambda_{0}, A\right) \geqq R(\lambda, A) \geqq 0$.

It follows that for all $\lambda>\max \left\{\lambda_{0}, s(A)\right\}$ we have\\
(1.12) $(\lambda-s(A))^{-1}=r(R(\lambda, A)) \leqq\|R(\lambda, A)\| \leqq\left\|R\left(\lambda_{0}, A\right)\right\|$,\\
which can be true only if $\lambda_{0}$ is greater than $s(A)$.

Corollary 1.4. Suppose $X$ is compact and $A$ has compact resolvent. Then there exists a real eigenvalue $\lambda_{0}$ admitting a positive eigenfunction such that $\operatorname{Re} \lambda \leqq \lambda_{0}$ for every $\lambda \in \sigma(A)$.

Proof. By Thm.1.1. $\lambda_{0}:=s(A)$ is a real number, contained in the spectrum and obviously Re $\lambda \leqq \lambda_{0}$ for every $\lambda \in \sigma(A)$. Since $A$ has compact resolvent it follows that $\lambda_{0}$ is a pole of the resolvent. Let $k$ be its order, then the highest coefficient in the Laurent series is given by


\begin{equation*}
Q:=\lim _{\lambda \rightarrow s}(A)(\lambda-s(A))^{k} R(\lambda, A) \tag{1.13}
\end{equation*}


It follows from cor.1.3 that $Q$ is a positive operator. Since $Q \neq 0$ there exists $g \geqq 0$ such that $h:=Q g>0$. Moreover, by $\mathrm{A}-\mathrm{III}, 3.6$. we have $\left(\lambda_{0}-A\right) h=\left(\lambda_{0}-A\right) Q g=0$.

The example of the rotation semigroup (A-III,Ex.5.6) shows that the assumptions in cor.1.4. do not imply that $s(A)$ is dominant. Additional hypotheses ensuring this stronger property will be given below (see cor.2.11, 2.12).

The following result is elementary. However, positivity is the crucial point in its proof. Note that it is not just a consequence of the spectral mapping theorem for the point spectrum.

Proposition 1.5. Suppose A is the generator of the positive semigroup $(T(t))_{t \geq 0}$. Take $r>0, r>0$ and let $\alpha:=T^{-1} \log (r)$. (a) If $r$ is an eigenvalue of $T(\tau)$ with positive eigenfunction $h_{0}$, then there is a positive $h \in D(A)$ such that $A h=\alpha h$ and $\left\{x \in X: h_{O}(x)>0\right\} \subset\{x \in X: h(x)>0\}$.

(b) If $r$ is an eigenvalue of $T(\tau)$ with positive eigenvector $\phi_{0}$, then there is a positive $\phi \in D\left(A^{*}\right)$ such that $A^{*} \phi=\alpha \phi$ and supp $\phi_{O} \subset \operatorname{supp} \$$.

Proof. Without loss of generality we may assume $r=1$, hence $\alpha=0$ and $T(\tau) h_{0}=h_{0}$.\\
(a) Defining\\
(1.14) $h:=\int_{0}^{t} \mathrm{~T}(\mathrm{~s}) \mathrm{h}_{\mathrm{o}} \mathrm{ds}$\\
then for $0 \leqq t \leqq \tau$ we have\\
$T(t) h=\int_{0}^{\tau} T(s+t) h_{0} d s=\int_{t}^{\tau} T(s) h_{0} d s+\int_{\tau}^{\tau+t} T(s-\tau) T(\tau) h_{0} d s=$ $=\int_{t}^{\mathrm{t}} \mathrm{T}(\mathrm{s}) \mathrm{h}_{0} \mathrm{ds}+\int_{0}^{\mathrm{t}} \mathrm{T}(\mathrm{s}) \mathrm{T}(\mathrm{t}) \mathrm{h}_{0} \mathrm{ds}=\mathrm{h}$.

It follows that $A h=\lim t^{-1}(T(t) h-h)=0$. So far, positivity was not used. The point is that in general, $h$ may be zero. But if (T(t)) is positive and $h_{0} \geqq 0$, then $s \rightarrow\left(T(s) h_{0}\right)(x)$ is a continuous positive function, hence $0<h_{0}\left(x_{0}\right)=\left(T(0) h_{0}\right)\left(x_{0}\right)$ implies $h\left(x_{0}\right)=$ $\int_{0}^{\mathrm{T}}\left(\mathrm{T}(\mathrm{s}) \mathrm{h}_{0}\right)\left(\mathrm{x}_{0}\right) \mathrm{ds}>0$.\\
(b) Defining $\phi:=\int_{0}^{\tau} \mathrm{T}(\mathrm{s})^{\prime} \phi_{0} \mathrm{ds}$, one can proceed as in (a) to obtain the desired result.

We use Prop.1.5 to prove an analogue of the famous Krein-Rutman result. For the sake of completeness we include the proof of this classical result, which states that the spectral radius of a positive operator $T$ on $C(K)$ (or more generally on an order unit space) is an eigenvalue of the adjoint $\mathrm{T}^{\prime}$ (see the Corollary of Thm.2.6 in the appendix of Schaefer (1966)).

Theorem 1.6. Suppose $K$ is compact and (T(t)) $t \geqslant 0$ is a positive semigroup with generator A . Then there exists a positive probability measure $\phi \in D\left(A^{\prime}\right)$ such that $A^{\prime} \phi=\omega(A) \phi$.

Proof. Consider $T:=T(1), r:=r(T)=e^{\omega(A)}$. In view of Prop. 1.5 it is enough to show that $r$ is an eigenvalue of $T^{\prime}$ with a positive eigenvector. Given $\lambda \in \mathbb{C},|\lambda|>r$ and $f \in C(K)$ we have $|R(\lambda, T) f|=\left|\sum_{n=0}^{\infty} \lambda^{-n-1} T^{n} f\right| \leqq \sum_{n=0}^{\infty}|\lambda|^{-n-1} T^{n}|f|=R(|\lambda|, T)|f|$. It follows that $\|R(\lambda, T)\| \leqq\|R(|\lambda|, T)\|$ and therefore


\begin{equation*}
\lim _{\lambda \downarrow r}\|R(\lambda, T)\|=\infty \tag{1.15}
\end{equation*}


By the uniform boundedness principle there exist a sequence ( $\lambda_{n}$ ), $\lambda_{n}+r$ and a positive $\Psi \in M(K)$ such that $\left\|R\left(\lambda_{n}, T\right)^{\prime} \Psi\right\| \rightarrow \infty$. Defining $\psi_{n}:=\left\|R\left(\lambda_{n}, T\right)^{\prime} \Psi\right\|^{-1} R\left(\lambda_{n}, T\right)^{\prime} \Psi$ we have\\
(1.16) $\left(r-T^{\prime}\right) \Psi_{n}=\left\|R\left(\lambda_{n}, T\right) \cdot \Psi\right\|^{-1} \cdot\left(\left(r-\lambda_{n}\right)+\left(\lambda_{n}-T '\right)\right) R\left(\lambda_{n}, T\right) \cdot \Psi=$ $=\left(r-\lambda_{n}\right) \psi_{n}+\left\|R\left(\lambda_{n}, T\right) \cdot \Psi\right\|^{-1}{ }^{-1} 0$.\\
since $r-T^{\prime}$ is $\sigma(M(K), C(K))$-continuous, (1.16) implies that every $\sigma(M(K), C(K)) \quad$ cluster point of $\left(\Psi_{n}\right)$ is a positive eigenvector, provided that it is non-zero. Because K is compact we have $\{\phi \in \mathrm{M}(\mathrm{K}): \phi \geqq 0,\|\phi\|=1\}=\{\phi \in \mathrm{M}(\mathrm{K}): \phi \geqq 0,\langle\phi, 1>=1\}$ which shows that the set of probability measures is $\sigma(M(K), C(K))$-compact. Therefore the sequence ( ${ }_{n}$ ) has non-zero cluster points.

This theorem implies that for positive semigroups on $C(K)$ the growth and spectral bounds coincide (cf. A-III,4.4). Actually, this is true for locally compact spaces as well and can be proved directly (see B-IV,Thm.1.4). Using this result one can prove Thm.1.6 by applying the classical Krein-Rutman theorem to any resolvent operator $R(\lambda, A)$ for $\lambda \in \mathbb{R}$ sufficiently large.\\
The theorem ensures that $A^{\prime}$ always has eigenvalues. The generator itself may have no eigenvalue at all. Multiplication operators have no eigenvalues unless the multiplier is constant on an open subset. Thm.1.6 fails to be true for locally compact spaces as the following example shows:

Examples 1.7. Consider $E=C_{0}\left(\mathbb{R}^{n}\right)$ and the semigroup $(T(t))_{t \geqslant 0}$ generated by the Laplacian (cf. A-I,2.8). From the explicit representation of $\mathrm{T}(\mathrm{t})$,\\
(1.17) $(T(t) f)(x)=(4 \pi t)^{-n / 2} \int_{\mathbb{R}^{n}} \exp \left(-(x-y)^{2} / 4 t\right) \cdot f(y) d y$,\\
i.t follows that $\lim _{t \rightarrow \infty} T(t) f=0$ for every $\pounds \in C_{0}\left(\mathbb{R}^{n}\right) \quad$ (Note that $\|T(t) f\| \leqq(4 \pi t)^{-n / 2} \int_{\mathbb{R}^{n}}^{t \rightarrow \infty}|f(y)| d y \rightarrow 0$ provided that $f$ has compact support and that $\|T(t)\|=1$ for all $t \geqq 0$ ).\\
If $\phi$ is an eigenvector of $A^{\prime}$ corresponding to $s(A)=\omega(A)=0$, we have $T(t){ }^{\prime} \phi=\phi$ for all $t \geqq 0$, hence $\langle\phi, f\rangle=\lim _{t \rightarrow \infty}\langle T(t) f, \phi\rangle=0$ for every f, i.e., $\phi=0$.

\section*{2. THE BOUNDARY SPECTRUM}
In this section we restrict our attention to the boundary spectrum $\sigma_{b}(A)$ of a generator $A$, which, by definition, is the intersection of $\sigma(A)$ with the line $\{\lambda \in \mathbb{C}: \operatorname{Re} \lambda=s(A)\}$. Thus $\sigma_{b}(A)$ contains all spectral values of $A$ which have maximal real part. Note that in general the boundary spectrum is a proper subset of the topological boundary of $\sigma(A)$. Our aim is to prove results ensuring that $\sigma_{b}(A)$ is a cyclic set (see Def.2.5).\\
While most of the results of the preceding section were obtained by transforming the problem to a resolvent operator $R(\lambda, A)(\lambda \in \mathbb{R}$ large enough), this procedure fails here. The reason is that there is no one-to-one correspondence between the boundary spectrum of $A$ und the peripheral spectrum of $R(\lambda, A)$. Actually, from Thm. 1.1 and A-III, Prop.2.5 it follows that the peripheral spectrum of $R(\lambda, A)$ (i.e., the set of spectral values having maximal absolute value) is trivial, since it only contains the spectral radius $r(R(\lambda, A))=(\lambda-s(A))^{-1}$. We begin our discussion with two lemmas.

Lemma 2.1. Suppose $K$, $L$ are compact and $T: C(K) \rightarrow C(L)$ is a linear operator satisfying $T 1_{K}=1_{L}$. Then we have $T \geqq 0$ if and only if $\|T\| \leqq 1$.

Proof. If T is positive, then\\
(2.1) $|T f| \leqq T|f| \leqq T\left(\|f\| \cdot I_{K}\right)=\|f\| \cdot T\left(I_{K}\right), f \in C(K)$,\\
hence $\|\mathrm{T}\|=\left\|T 1_{K}\right\|$, whenever $T$ is positive. This shows that $\mathrm{T} \geqq 0$ implies $\|\mathrm{T}\| \leqq 1$ whenever $\mathrm{TI}_{\mathrm{K}}=1_{\mathrm{L}}$.\\
To prove the reverse direction, we first observe that for complex numbers and hence for complex-valued functions the following equivalence holds:

$$
-1 \leqq f \leqq 1 \text { if and only if }
$$


\begin{equation*}
\|f-i \cdot r \cdot 1\| \leq \rho_{r}:=\left(1+r^{2}\right)^{1 / 2} \text { for every } r \in \mathbb{R} . \tag{2.2}
\end{equation*}


Now suppose $f \in C(K), 0 \leqq f \leqq 2 \cdot 1_{K}$. Then we have $-1_{K} \leqq \mathrm{f}-1_{\mathrm{K}} \leqq 1_{\mathrm{K}}$ hence by (2.2) $\left\|\mathrm{f}-1_{\mathrm{K}}-\mathrm{i} \cdot \mathrm{r} \cdot 1_{\mathrm{K}}\right\| \leqq \rho_{\mathrm{r}}$ for every $r \in \mathbb{R}$. From $\mathrm{T}_{\mathrm{K}}=1_{\mathrm{L}}$ and $\|\mathrm{T}\| \leqq 1$ it follows that\\
$\left\|T f-1_{L}-i \cdot r \cdot 1_{L}\right\| \leqq \rho_{r}$ for every $r \in \mathbb{R}$. Using (2.2) once again, we obtain $-1_{L} \leqq \mathrm{Tf}-1_{L} \leqq 1_{\mathrm{L}}$ or $0 \leqq \mathrm{Tf} \leqq 2 \cdot 1_{\mathrm{L}}$.

Before we can formulate the second lemma we have to fix some notation:

Definition 2.2.(a) Given $h \in C_{0}(x)$ such that $h(x) \neq 0$ for all $x \in X$ then the operator $S_{h}$ is defined to be the multiplication operator with sign h, i.e.,\\
(2.3) $\quad s_{h} \pounds=h|h|^{-1} f \quad\left(f \in C_{0}(x)\right)$.\\
(b) For $f \in C_{0}(x), n \in \mathbb{Z}$ we define $f^{[n]} \in c_{0}(x)$ by (2.4) $\quad f^{[n]}(x)=\left\{\begin{array}{ccc}(f(x) /|f(x)|)^{n-1} \cdot f(x) & \text { if } & f(x) \neq 0 \\ 0 & \text { if } & f(x)=0\end{array}\right.$

The following assertions are immediate consequences of the definition. They will be used frequently in the following.\\
(2.5) $S_{h}$ is a linear isometry satisfying $\left|s_{h} f\right|=|f|$, its inverse being $s_{\bar{h}}$ where $\bar{h}$ is the complex conjugate of $h$.\\
(2.6) $\mathfrak{f}^{[1]}=\mathbf{f}, \mathrm{f}^{[0]}=|\mathrm{f}|, \mathrm{f}^{[-1]}=\overline{\mathbf{f}}$, $\left|f^{[n]}\right|=|f|$ for every $n \in \mathbb{Z}$.\\
(2.7) If $h(x) \neq 0$ for all $x \in x$, then $h^{[n]}=s_{h}^{n}|h|=s_{h}^{n-1} h$.

Lemma 2.3. Let T and R be bounded linear operators on $\mathrm{C}_{\mathrm{O}}(\mathrm{X})$ and assume that $h \in C_{O}(x)$ has no zeros. Suppose we have\\
(2.8) $\mathrm{Rh}=\mathrm{h}, \mathrm{T}|\mathrm{h}|=|\mathrm{h}|$ and $|\mathrm{Rf}| \leqq \mathrm{T}|\mathrm{f}|$ for every $f \in \mathrm{C}_{0}(\mathrm{X})$. Then $R$ and $T$ are similar, more precisely, $T=S_{h}^{-1} R S_{h}$. In particular, the spectra (and point spectra resp.) of $T$ and $R$ coincide.

Proof. We first note that the assertion $|R f| \leqq T|f|$ ( $f \in E$ ) implies that $T$ is a positive operator. Therefore $T|h|=|\mathrm{h}|$ implies that the principal ideal $\mathrm{E}_{\mathrm{h}}=\left\{f \in \mathrm{C}_{0}(\mathrm{X}):|\mathrm{f}| \leqq \mathrm{n}|\mathrm{h}|\right.$ for some $\left.n \in \mathbb{N}\right\}$ is an invariant subspace for $T$ and for $R$ as well. $E_{h}$ is isomorphic to $C^{b}(X) \cong C(\beta X) \quad(\beta X$ denotes the stone-Cech compactification of $\mathrm{X})$, an isomorphism is given by $f \rightarrow f|h|$. Considering the restrictions ${ }^{T} \mid E_{h}$ and ${ }^{R} \mid E_{h}$ as operators on $C(B X)$ and denoting them $\tilde{T}$ and $\tilde{\mathbb{R}}$ respectively, we have\\
(2.9) $\tilde{\mathrm{R}} \tilde{h}=\tilde{\mathrm{h}}, \tilde{\mathrm{T}} 1=1, \tilde{\mathrm{~T}} \geqq 0,|\tilde{\mathrm{R}} \mathrm{f}| \leqq \tilde{\mathrm{T}}|\mathrm{f}|$ for all f .

Here $\tilde{h}$ denotes the continuous extension of $h /|h|$ to $B X$. Defining $\mathrm{T}_{\mathrm{I}}:=\mathrm{M}_{\tilde{\mathrm{h}}}{ }^{1} \tilde{\mathrm{R}} \mathrm{M}_{\tilde{\mathrm{h}}}$ we have by (2.9)\\
(2.10) $\mathrm{T}_{1} 1=M_{\tilde{h}}^{-1} \tilde{\mathrm{R}} \tilde{\mathrm{h}}=1$ and\\
(2.11) $\left|T_{1} \mathrm{f}\right|=\left|\mathrm{M}_{\tilde{h}}^{-1} \tilde{R}_{\tilde{h}} f\right|=\left|\tilde{R} M_{\tilde{h}} f\right| \leq \tilde{T}\left|M_{\tilde{h}} f\right|=\tilde{\mathrm{T}}|f|$ for all $f$.

Hence we have $\left\|\mathrm{T}_{1}\right\| \leqslant\|\tilde{\mathrm{T}}\|=1$ (by (2.11), (2.9), (2.1)). Then it follows from Lemma 2.1 that $\mathrm{T}_{1}$ is a positive operator. Thus (2.11) implies that $0 \leq \mathrm{T}_{1} \leq \tilde{\mathrm{T}}$ and therefore $\left\|\tilde{\mathrm{T}}-\mathrm{T}_{1}\right\|=\left\|\left(\tilde{\mathrm{T}}-\mathrm{T}_{1}\right) I\right\|=0$ (by (2.10), (2.9), (2.1)).

We are now able to prove a result which in some sense is the key to cyclicity results for the spectrum. These general results will be proved by reducing the problem in such a way that the following theorem can be applied.

Theorem 2.4.(a) Let $T$ be a positive linear operator on $C_{0}(X)$, let $h \in C_{0}(X)$ and $\lambda \in \mathbb{C},|\lambda|=1$. If $T h=\lambda h$ and $T|h|=|h|$, then we have $T h^{[n]}=\lambda^{n_{h}}{ }^{[n]}$ for every $n \in Z$ (cf. (2.4)). If $h$ does not have zeros in $X$, then $\lambda T=s_{h}^{-1} T S_{h}$.\\
(b) Suppose A is the generator of a positive semigroup, $h \in C_{0}(\mathrm{X})$, $\alpha, \beta \in \mathbb{R}$ such that $A h=(\alpha+i \beta) h$ and $A|h|=\alpha|h|$. Then we have $A h^{[n]}=(\alpha+i n \beta) h^{[n]}$ for every $n \in \mathbf{Z}$.\\
If $h$ does not have zeros then $S_{h} D(A)=D(A)$ and $i \beta+A=s_{h}^{-1} A_{h}$.\\
Proof. (a) The closed principal ideal $\overline{E_{h}}$ which is canonically isomorphic to $c_{0}\left(x_{1}\right)$ with $x_{1}=\{x \in x: h(x) \neq 0\}$, is T -invariant. We give an object a tilde when we consider it as an element of $\overline{\mathrm{E}_{\mathrm{h}}} \cong \mathrm{C}_{0}\left(\mathrm{X}_{1}\right)$. Defining $\tilde{\mathrm{R}}:=\bar{\lambda} \tilde{\mathbb{T}}$, then $\tilde{\mathrm{T}}, \tilde{\mathrm{R}}, \tilde{\mathrm{h}}$ satisfy (2.8), hence we have\\
(2.12) $\tilde{\mathrm{T}}=\mathrm{S}_{\tilde{\mathrm{h}}}^{-1} \circ \tilde{\mathrm{R}} \cdot \mathrm{S}_{\tilde{\mathrm{h}}}=\bar{\lambda} \cdot \mathrm{S}_{\tilde{\mathrm{h}}}^{-1} \circ \tilde{\mathrm{~T}} \circ \mathrm{~S}_{\tilde{\mathrm{h}}}$\\
which by iteration yields\\
(2.13) $\tilde{\mathrm{T}}=\bar{\lambda}^{-\mathrm{n}} \cdot \mathrm{S}_{\tilde{\mathrm{h}}}^{-\mathrm{n}} \circ \tilde{\mathbb{T}} \circ \mathcal{S}_{\tilde{\mathrm{h}}}^{\mathrm{n}}$ for all $n \in \mathbf{Z}$.

It follows that\\
$\tilde{\mathrm{T}} \tilde{h^{[n]}}=\tilde{\mathrm{T}} \cdot \mathrm{s}_{\tilde{\mathrm{h}}}^{\mathrm{n}}|\tilde{\mathrm{h}}|=\lambda^{\mathrm{n}} \cdot \mathrm{s}_{\tilde{\mathrm{h}}}^{\mathrm{n}} \cdot \tilde{\mathrm{T}}|\tilde{\mathrm{h}}|=\lambda^{\mathrm{n}} \cdot \mathrm{s}_{\tilde{\mathrm{h}}}^{\mathrm{n}} \tilde{\mathrm{h}}=\lambda^{\mathrm{n}} \cdot \tilde{\mathrm{h}}^{[\mathrm{n}]}$\\
(see (2.7) and (2.12)), which is precisely $T^{[n]}=\lambda^{n_{h}}{ }^{[n]}$ for all $\mathrm{n} \in \boldsymbol{Z}$. If $h$ does not have zeros, then $\overline{\mathrm{E}_{\mathrm{h}}}=\mathrm{E}$, hence $\mathrm{T}=\tilde{\mathbb{T}}$, $\mathrm{h}=\tilde{\mathrm{h}}$ and the remaining assertion follows from (2.12).\\
(b) This can be deduced easily from (a) as follows:

If $A h=(\alpha+i \beta) h, A|h|=\alpha|h|$, then we have by A-III, Cor. 6.4 :\\
$e^{-\alpha t_{T}}(t) h=e^{i \beta t_{h}}$ and $e^{-\alpha t_{T}(t)|h|=|h|}$ for every $t \geqq 0$.\\
Hence by (a) $e^{-\alpha t_{T}(t) h^{[n]}}=e^{i n \beta t_{h}[n]} \quad(t \geqq 0, n \in \mathbb{Z})$, which is equivalent to $A h^{[n]}=(\alpha+i n \beta) h^{[n]}$. If $h$ does not have zeros, then $e^{-\alpha t} T(t)=e^{-\alpha t} e^{i \beta t} S_{h}^{-1} T(t) S_{h}$ for every $t \geq 0$ which is equivalent to the final statement of (b).

Before we state a first cyclicity result we give the definition and illustrate it by some examples.

Definition 2.5. A subset $M \subset \mathbb{C}$ is called imaginary additively cyclic (or simply cyclic), if it satisfies the following condition: $\alpha+i \beta \in M, \alpha, \beta \in \mathbb{R}$ implies that $\alpha+i k \beta \in M$ for every $k \in \mathbb{Z}$.

Every subset of $\mathbb{R}$ is cyclic. On the other hand, if $M$ is cyclic and $M \notin \mathbb{R}$, then $M$ has to be unbounded.\\
For a subset $M$ of $i R$ we give the following equivalent conditions:\\
(i) M is imaginary additively cyclic;\\
(ii) M is the union of (additive) subgroups of iR ;\\
(iii) $M=U_{\alpha \in S}$ iaZ for some set $S \subset \mathbb{R}$.

Here are some concrete cyclic subsets of iR :\\
$M_{I}=\{0\}, M_{2}=i \mathbb{R}, M_{3}=i \alpha \mathbb{Z} \quad(\alpha>0)$,\\
$\mathrm{M}_{4}=i \alpha \mathbb{Z}+i \beta \mathbb{Z}=\{i n \alpha+i m \beta: n, m \in \mathbb{Z}\}(\alpha, \beta \in \mathbb{R})$,\\
$M_{5}=\{0\} U\{i \lambda: \lambda \in \mathbb{R},|\lambda| \geqq 1\}$,\\
$M_{6}=U_{n=0}^{\infty}\{i \lambda: \lambda \in \mathbb{R}, n \alpha \leqq|\lambda| \leqq n \beta\} \quad(0<\alpha \leqq \beta, \alpha, \beta \notin \mathbb{R})$.

In the following we consider the boundary spectrum of several semigroups. The letter $M_{i}$ refers to the sets just defined.

Examples 2.6.(a) For the Laplacian $\Delta$ on $\mathbb{R}^{n}$ or the second derivative on $[0,1]$ with Neumann boundary conditions the boundary spectrum is $M_{1}$.\\
(b) The first derivative on $\mathbb{R}$ or $\mathbb{R}_{+}$is an example where the boundary spectrum of the generator is $\mathrm{M}_{2}$.\\
(c) The rotation semigroup on $C(\Gamma)$ (see A-III,Ex.5.6) with period $2 \pi / \alpha$ has boundary spectrum $\mathrm{M}_{3}$.\\
(d) For the semigroup on $C(\Gamma \times \Gamma)$ given by\\
$(T(t) f)(z, w)=f\left(z \cdot e^{i \alpha t}, w \cdot e^{i \beta t}\right) \quad(f \in C(r \times \Gamma),(z, w) \in \Gamma \times \Gamma)$\\
we have $\operatorname{Po}(A)=M_{4}$. If $\alpha / \beta$ is irrational, then this is a dense subset of $i \mathbb{R}$ and $\sigma_{b}(A)=\sigma(A)=i \mathbb{R}$.\\
(e) Consider $D:=\{z \in \mathbb{C}:|z| \leqq 1\}=\left\{r \cdot e^{i \omega}: r \in[0,1], \omega \in \mathbb{R}\right\}$, and a strictly positive function $k \in C[0,1]$. The flow on $D$ governed by the differential equation $\dot{r}=0, \dot{\omega}=k(r)$ induces a strongly continuous semigroup on $C(D)$ (which is given by $\left.(T(t) f)(z)=f\left(z \cdot e^{i k(|z|) t}\right)\right)$. The boundary spectrum is $M_{6}$ with $\alpha:=\inf k(r), \beta:=\sup \kappa(r)$. In particular, for $k(r)=1+r$ we obtain as boundary spectrum the set $\mathrm{M}_{5}$.\\
(f) Suppose $M$ is a closed cyclic subset of $\mathrm{iR}, \mathrm{M}=U_{\alpha \in S}$ ial for a suitable $\mathrm{s}^{\prime} \subset \mathbb{R}$ (e.g. $\mathrm{S}=\mathrm{M}$ ).\\
The space $E_{1}:=\left\{\left(f_{\alpha}\right)_{\alpha \in S}: f_{\alpha} \in C(\Gamma)\right.$, sup $\left.\left\|f_{\alpha}\right\|<\infty\right\}$ is a Banach space under the norm $\left\|\left(f_{\alpha}\right)\right\|:=\sup \left\|f_{\alpha}\right\|$. The closure of the linear subspace $E_{0}:=\left\{\left(f_{\alpha}\right) \in E_{1}: f_{\alpha} \neq 0\right.$ only for finitely many $\left.\alpha \in S\right\}$ is isomorphic to $C_{O}(x)$ where $x$ is the topological sum of $|s|$ copies of $\Gamma$.\\
Let $\left(T_{\alpha}(t)\right)_{t \geqq 0}$ denote the rotation semigroup on $C(\Gamma)$ with period $2 \pi / a$, then we define a semigroup $(T(t))_{t \geqq 0}$ on $E:=C_{0}(X)$ as follows:\\
$\left(T(t)\left(E_{\alpha}\right)\right):=\left(T_{\alpha}(t) f_{\alpha}\right) \quad\left(\left(f_{\alpha}\right){ }_{\alpha \in S} \in E\right)$.\\
This is a positive semigroup on $E=C_{O}(X)$ whose boundary spectrum is precisely the given closed cyclic set $M$. We leave the verification as an excercise.

Our first result concerns cyclicity of the eigenvalues contained in the boundary spectrum, i.e., of the set\\
$P_{\sigma_{b}}(A):=\operatorname{Po}(A) \cap_{b}(A)=\left\{\lambda \in P_{\sigma}(A): \operatorname{Re} \lambda=s(A)\right\}$.\\
It is almost a straightforward consequence of Thm.2.4.

Proposition 2.7. Assume that for some $t_{0}>0$ there is a strictly positive measure $\phi$ such that $T\left(t_{0}\right)^{\prime} \phi=\exp \left(s(A) t_{0}\right)^{\prime} \phi \cdot$\\
Then $\mathrm{Po}_{\mathrm{b}}(\mathrm{A})$ is imaginary additively cyclic.

Proof. If $\mathrm{Po}_{b}(\mathrm{~A})$ is empty there is nothing to prove. Otherwise we have $s(A)>-\infty$. In view of the rescaling procedure we may assume $s(A)=0$. By Prop.1.5(b) there exists $\Psi>0$ such that $T(t){ }^{\prime} \Psi=\Psi$ for all $t \geqq 0$. Given $i \alpha \in \operatorname{Po}_{b}(A)$ then there is $h \in C_{0}(X), h \neq 0$ such that $A h=i_{\alpha h}$ or $T(t) h=e^{i \alpha t_{h}}$ for all $t$ (A-III, Cor. 6. 4).

Then we have\\
(2.14) $|h|=\left|e^{i \alpha t_{h}}\right|=|T(t) h| \leqq T(t)|h|$ or $T(t)|h|-|h| \geqq 0$,\\
(2.15) $\langle T(t)| h|-|h|, \psi\rangle=\langle | h\left|, T(t)^{\prime} \psi\right\rangle-\langle | h|, \psi\rangle=0$.

Since $\Psi$ is strictly positive, (2.14) and (2.15) imply that\\
$T(t)|h|=|h|$ for $t \geqq 0$ or equivalently $A|h|=0$.\\
Now Thm.2.4 implies that $A h^{[n]}=$ inoh ${ }^{[n]} \quad(n \in \mathbf{Z})$.

Concerning the hypothesis $T\left(t_{0}\right)^{\prime} \phi=\exp \left(s(A) t_{0}\right) \cdot \phi \gg 0_{0}$ we recall that in case X is compact there are always positive linear forms such that $T(t)^{\prime} \phi=e^{s(A) t}{ }_{\phi}$ (see Thm.1.6). If the semigroup is irreducible, then one also has $\phi \gg 0$ (see Sec. 3 below).

In a second result we consider semigroups having compact resolvent. An important step of the proof is isolated as a lemma. Before stating it we recall that given a closed ideal I $\subset C_{0}(X)$ then $I$ as well as $C_{0}(X) / I$ are spaces of continuous functions on a locally compact space vanishing at infinity. More precisely, if $I=\left\{f \in C_{0}(X): E_{M}=0\right\}$ for a suitable closed subset $M \subset X$, then $I \cong C_{0}(X \backslash M)$ and $C_{0}(X) / I$ $\cong C_{O}(M)$ (cf. B-I). It follows that given another closed ideal\\
$J=\left\{\pounds \in C_{O}(X): f_{\mid N}=0\right\}$ such that $I \subset J \quad$ i.e., $N \subset M$, then $J / I$ can be identified with $C_{0}(M \backslash N)$. We do not use this concrete representation of J/I . However, this shows that we stay within our setting of Banach spaces of continuous functions on locally compact spaces.

Lemma 2.8. Suppose $A$ is the generator of a positive semigroup $T$ such that the spectral bound $s(A)$ is a pole of the resolvent of order $k$. Then there is a sequence\\
(2.16) $\mathrm{I}_{-1}:=\{0\} \subset \mathrm{I}_{0} \varsubsetneqq \mathrm{I}_{1} \varsubsetneqq \ldots \varsubsetneqq \mathrm{I}_{\mathrm{k}}:=\mathrm{E}$\\
of T-invariant closed ideals with the following properties: Denoting by $A_{n}(n=0,1, \ldots, k)$ the generator of the semigroup on $I_{n} / I_{n-1}$ which is induced by $(T(t))$ we have :\\
(a) $s\left(A_{0}\right)<s(A)$;\\
(b) If $n \geqq 1$ then $s\left(A_{n}\right)=s(A)$ is a first order pole of the resolvent $\mathrm{R}\left(., \mathrm{A}_{n}\right.$ ' . The corresponding residue is a strictly positive operator.

Proof. We can assume that $s(A)=0$ and we will denote the negative coefficients of the Laurent series of $R(., A)$ at 0 by $Q_{n}$. Thus the following relations hold (see A-III, 3.6),

$$
Q_{n}=\frac{1}{2 \pi \dot{i}} \cdot \int_{\gamma} z^{n-1} R(z, A) d z \quad(n \in \mathbb{N}) ;
$$


\begin{align*}
& Q_{n} \neq 0 \text { if } n \leqq k \text { and } Q_{n}=0 \text { for } n>k ;  \tag{2.17}\\
& Q_{n}=A^{n-1} Q_{I} \quad(n \in \mathbb{N}) ; Q_{k}=1 i m_{z \rightarrow 0} z^{k} \cdot R(z, A) ;
\end{align*}


We define $I_{n}$ as follows $(n=0,1, \ldots, k-1)$ :

$$
I_{n}:=\left\{f \in E: Q_{n+1}|f|=Q_{n+2}|f|=\ldots=Q_{k}|f|=0\right\}
$$

At first we restrict our attention to $\mathrm{I}_{\mathrm{k}-1}$.\\
Since $R(\lambda, A)$ is positive if $\lambda>0$ (Cor.1.3), it follows from (2.17) that $Q_{k}$ is a positive bounded operator, hence\\
$I_{k-1}=\left\{f \in E: Q_{k}|f|=0\right\}$ is a closed ideal. Since $Q_{k}$ commutes with $\mathrm{R}(\lambda, \mathrm{A})$ (see (2.17) ), it follows that $I_{k-1}$ is a $T$-invariant ideal. By A-III, Cor. 4.3 the generators ${ }^{A} / I_{k-1}$ and $A_{k}$ induced by $A$ on $I_{k-1}$ and $E / I_{k-1}$ respectively have a pole at 0 . The coefficients of the Laurent series are the operators induced by ${ }^{Q}{ }_{n}$ on $E / I_{k-1}$ and $I_{k-1}$ respectively.\\
Suppose that the pole order of $R\left(., A_{k}\right)$ is greater than 1 , say $m$. Then $Q_{m /}=\lim _{z \rightarrow 0} z^{m} R\left(z, A_{k}\right)$ is a positive non-zero operator, hence we find for every $x \in E_{+}$an element $y \in I_{k-1}$ such that $\theta_{m} x+y \geqq 0$. Then we have\\
$0 \leqq Q_{k}\left|Q_{m} x+y\right|=Q_{k} Q_{m} x+Q_{k} y=Q_{k+m-1} x+Q_{k} y=0+Q_{k} y \leqq Q_{k}|y|=0$ hence $Q_{m} x=\left(Q_{m} x+y\right)-y \in I_{k-1}\left(x \in E_{+}\right)$. It follows that $Q_{\mathrm{m} /}=0$ which is a contradiction.\\
So far we know that the resolvent of $A_{k}$ has a pole of order $\leqq 1$. Moreover, since $Q_{k} \mid I_{k-1}=0$, the resolvent of ${ }^{A} \mid I_{k-1}$ has a pole of order $\leqq k-1$. From A-III, Cor. 4.3 it follows that the pole order of $A_{k}$ and ${ }^{A} \mid I_{k-1}$ is precisely 1 and $k-1$, respectively. The residue $Q_{1 / I_{k-1}}=\lim _{z \rightarrow 0} z R\left(z, A_{k}\right)$ is positive since $R\left(z, A_{k}\right) \geqq 0$ for $z>0$ (Cor.1.3). To prove that $i$ t is strictly positive we assume\\
$Q_{1 / I_{k-1}}\left(\left|x+I_{k-1}\right|\right)=0$ which means $Q_{1}|x| \in I_{k-1}$ hence $Q_{k}|x|=$ $A^{k-1} Q_{1}|x|=0$, that is, $x \in I_{k-1}$ or $x+I_{k-1}=0$. Applying what we have proved so far to $I_{k-1}$ and ${ }^{A} \mid I_{k-1}$ we obtain $I_{k-2}, A_{k-1}$, and so on. After $k$ steps $(n=1)$ we conclude that $I_{o}$ is T-invariant and that the order of the pole of $R\left(.{ }^{A} \mid I_{0}\right.$ ) is 0 ,\\
which means that $0 \in \rho\left({ }^{A} \mid I_{0}\right)$. Since ${ }^{A} \mid I_{0}$ generates a positive semigroup and $R\left(\lambda,{ }^{A} \mid I_{0}\right)=R(\lambda, A) \mid I_{0}$ is positive for $\lambda>0$ it follows from cor.1.3. that $s\left(A_{0}\right)=s\left({ }^{A_{0}} \mid I_{0}\right)<0$.

One can check the different steps of the proof by studying the following example. Consider the following matrix as generator on $\mathbb{C}^{4}$.

$$
\left(\begin{array}{rrrr}
-1 & a & b & c \\
0 & 0 & d & e \\
0 & 0 & 0 & f \\
0 & 0 & 0 & 0
\end{array}\right) \quad \text { where } a, b, c, d, e, f \geqq 0
$$

The result is sumarized in the following diagram ( $e_{j}:=\left(\delta_{j k}\right)$ ) :\\
\includegraphics[max width=\textwidth, center]{2024_12_23_c6487cc0859199a15bd9g-186}

This example also shows that the operators $Q_{k-1}, \ldots, Q_{1}$ are not necessarily positive (e.g. $a>0, b=c=0, d=e=f=2$ ). A more general (and more interesting) example is the following:\\
Suppose that $A_{i}(i=1, \ldots, n)$ are generators of positive semigroups on $C_{0}(X)$ such that $s\left(A_{i}\right)=0$ is a first order pole of the resolvent. And let $A_{i j}(1 \leqq i<j \leqq n)$ be positive bounded operators on $\mathrm{C}_{\mathrm{O}}(\mathrm{X})$.\\
Then $A:=\left(\begin{array}{cccc}A_{1} & A_{12} & \cdots & { }^{A}{ }_{1 n} \\ 0^{1} & A_{12} & \cdots & { }^{A_{2}} \\ \vdots & :^{2 n} & \cdot & { }^{A}\end{array}\right)$\\
is the generator of a positive semigroup on\\
$c_{0}\left(x, \mathbb{C}^{n}\right) \cong c_{0}(x) \times c_{0}(x) \times \ldots \times c_{0}(x)$, and $s(A)=0$ is a pole of the resolvent of order $k$ where $1 \leqq k \leqq n$.

Theorem 2.9. Suppose $A$ is the generator of a positive semigroup on $c_{0}(X)$ such that every point of $\sigma_{b}(A)$ is a pole of the resolvent. Then $P_{\sigma_{b}}(A)=\sigma_{b}(A)$ is cyclic.

Proof. If $\sigma(A)=\emptyset$ there is nothing to prove, thus we can assume that $s(A)=0$. In view of the lemma and A-III, Prop.4.3(i) we can assume that $s(A)$ is a first order pole with strictly positive residue, which we call $Q$. We have $A Q=Q A=s(A) A=0$ (see $A-I I I$, 3.6), hence\\
(2.18) $Q T(t)=T(t) Q=Q$ for all $t \geqq 0$.

If $A h=i \alpha h$ for some $\alpha \in \mathbb{R}, h \neq 0$, then $T(t) h=e^{i \alpha t_{h}}$ (by A-III, Cor.6.4). Hence $|h|=\left|e^{i \alpha t_{h}}\right|=|T(t) h| \leqq T(t)|h|$, or equivalently, $T(t)|h|-|h| \geq 0$. By $(2.18)$ we have $Q(T(t)|h|-|h|)=0$.\\
Since $Q$ is strictly positive, it follows that $T(t)|h|=|h|$ or $A|h|=0$. Now we can apply Thm. 2.4 and obtain $A h^{[n]}=$ inah $[n]$ for every $n \in \mathbf{Z}$. This shows that $\operatorname{po}_{b}(A)=\sigma(A) \cap i R$ is cyclic.

If A has compact resolvent then every point of $\sigma(\mathrm{A})$ is a pole of the resolvent (see $A-I I I, 3.6$ ) hence we have:

Corollary 2.10. If $A$ has compact resolvent, then $P \sigma_{b}(A)=\sigma_{b}(A)$ is cyclic.

If it is known that the boundary spectrum of a generator is cyclic and nonvoid, the following alternative holds:\\
(2.19) Either $\sigma_{b}(A)=\{s(A)\}$ or else $\sigma_{b}(A)$ is an infinite unbounded set.

If one can exclude the second alternative, then there is a unique spectral value having maximal real part. A real spectral value $\lambda_{0}$ of a generator $A$ is called a dominant provided that $\operatorname{Re} \lambda<\lambda_{0}$ for every $\lambda \in \sigma(A)$, it is called strictly dominant if for some $\delta>0$ one has $\operatorname{Re} \lambda \leqq \lambda_{0}-\delta$ for every $\lambda \in \sigma(A), \lambda \neq \lambda_{0}$. The assumptions of Cor.2.10 do not imply that s(A) is dominant, the rotation semigroup (A-III,Ex.5.6) is a counterexample.

Corollary 2.11. Assume that for some $t_{0}>0$ (hence for all $t>0$ ) one has $r_{\text {ess }}\left(T\left(t_{0}\right)\right)<r\left(T\left(t_{0}\right)\right), e . g .$, that $T\left(t_{0}\right)$ is compact and $r\left(T\left(t_{0}\right)\right)>0 \quad(\operatorname{see} A-I I I, 3.7)$.\\
Then s(A) is a strictly dominant eigenvalue.

Proof. If $s(A)$ is not strictly dominant, then we have by Thm. 2.9 and A-III, Cor. 6.5 that $\{\lambda \in \sigma(A): \operatorname{Re} \lambda>s(A)-r\}$ contains infinitely many eigenvalues for every $r>0$. From A-III, Cor.6.4 it follows that $\{\lambda \in \sigma(T(t)):|\lambda|>r\}$ contains infinitely many eigenvalues (counted according to their multiplicities) for every $\mathbf{r}<\exp (\mathrm{s}(\mathrm{A}) \mathrm{t})$ $=r(T(t))$. This contradicts the assumption $r_{\text {ess }}(T(t))<r(T(t))$ (see A-III, 3.7).

Corollary 2.12. Suppose $A$ has compact resolvent and non-empty spectrum. If the corresponding semigroup is eventually norm continuous (e.g., if it is holomorphic or differentiable), then there is a strictly dominant eigenvalue admitting a positive eigenfunction.

Proof. Since $(T(t))_{t \geq 0}$ is eventually norm continuous, $\{\lambda \in \sigma(A)$ : Fe $\lambda \geqq s(A)-r$ \} is compact for every $r>0$ (see $A-I I, T h m .1 .20$ ) and this set does not have accumulation points because A has compact resolvent. In other words, it is a finite set. The assertion now follows from Thm.2.9 and Cor.1.4.

We now consider some examples. The first one shows that there are positive semigroups with $\mathrm{P}_{b}(\mathrm{~A})$ being not cyclic. It is unknown if there are semigroups where $\sigma_{b}(A)$ is not cyclic.

Example 2.13. Consider $E=C(\Gamma) \times C_{0}(\mathbb{R})$ ( $\equiv C_{0}(\Gamma \dot{R})$. We fix a positive function $k \in C_{O}(\mathbb{R})$ with compact support. The operator $A$ given by


\begin{align*}
A(f, g) & :=\left(f^{\prime}, g^{\prime}+\frac{1}{2 \pi} \int_{0}^{2 \pi} f(\theta) d \theta \cdot k\right)  \tag{2.20}\\
D(A) & :=\left\{(f, g) \in E: f, g \in C^{\perp}, g^{\prime} \in C_{0}(\mathbb{R})\right\}
\end{align*}


generates a semigroup $(T(t))_{t \geqslant 0}$ which is given by


\begin{align*}
& T(t)(f, g)=\left(f_{t}, g_{t}\right) \text { with } f_{t}(\theta):=f(\theta+t)  \tag{2.21}\\
& g_{t}(x):=g(x+t)+\frac{1}{2 \pi} \int_{0}^{2 \pi} f(\theta) d \theta \cdot \int_{x}^{x+t} k(u) d u
\end{align*}


Then $(\mathrm{T}(t))_{t \geq 0}$ is a positive semigroup and $\|\mathrm{T}(t)\| \leqq\left(1+\|k\|_{1}\right)$. In particular, $s(A) \leqq w(A) \leqq 0$. It is easy to see that 0 is not an eigenvalue of $A$, while all $i k, k \in \mathbf{Z}, k \neq 0$ are eigenvalues, the corresponding eigenfunctions being $\left(e_{k}, 0\right)$ with $e_{k}(\theta)=e^{i k \theta}$.

Example 2.14.(a) (One-dimensional Schrödinger operator).\\
Let $X=\mathbb{R}, E=C_{O}(X)$ and $V: \mathbb{R} \rightarrow \mathbb{R}$ be a continuous function such that inf $V(x)>-\infty$.\\
If we define


\begin{align*}
(A f)(x) & :=f^{\prime \prime}(x)-V(x) f(x), \\
D(A) & :=\left\{f \in C_{0}(X): f \in C^{2}, A f \in C_{O}(X)\right\}, \tag{2.22}
\end{align*}


then $A$ is the generator of a positive semigroup.\\
In case $\lim |\mathrm{x}| \rightarrow \infty \mathrm{V}(\mathrm{x})=\infty, \mathrm{A}$ has compact resolvent. Then by cor.2.10 there exists a dominant real eigenvalue with corresponding positive eigenfunction. Actually, the eigenfunction is strictly positive. (In fact, if $f \in c^{2}, f \geqq 0$ and $f\left(x_{0}\right)=0$ for some $x_{0}$, then $f^{\prime}\left(x_{0}\right)=0$. Therefore the uniqueness theorem for ordinary differential equations implies that $f$ is identically zero).\\
(b) (A retarded linear differential equation).

Consider $\mathrm{E}=\mathrm{C}[-1,0]$ and define $A_{m}, A_{O}$ as follows:


\begin{equation*}
A_{m} f:=f^{\prime}, f \in D\left(A_{m}\right)=C^{I}[-1,0] \tag{2.23}
\end{equation*}



\begin{equation*}
A_{0} f:=f^{\prime}, f \in D\left(A_{0}\right)=\left\{\pounds \in C^{1}[-1,0]: f^{\prime}(0)=0\right\} . \tag{2,24}
\end{equation*}


$A_{0}$ generates a contraction semigroup $\left(T_{0}(t)\right)_{t \geqslant 0}$ which is given by

\[
\left(T_{0}(t) f\right)(x)=\left\{\begin{array}{cc}
f(x+t) & \text { if } x+t \leqq 0,  \tag{2.25}\\
f(0) & \text { if } x+t \geqq 0 .
\end{array}\right.
\]

This semigroup is positive, eventually norm continuous $\left(\mathrm{T}_{0}(t)=\delta_{0} 1\right.$ for $t \geqq 1$ ) and has compact resolvent. Given a linear functional $\Psi$ on $c[-1,0]$, we consider\\
(2.26) $\quad A_{\Psi}:={ }^{A} m \mid D\left(A_{\Psi}\right)$ with $D\left(A_{\Psi}\right):=\left\{f \in C^{1}[-1,0]: f^{\prime}(0)=\langle f, \Psi\rangle\right\}$. Denoting the exponential function $x \rightarrow e^{\lambda x}$ by $e_{\lambda}$, we have for real $\lambda$ and $\lambda>\|\Psi\|$ :\\
(2.27) Id - I/ $\lambda \cdot \Psi e_{\lambda}$ is a bijection of $D\left(A_{Y}\right)$ onto $D\left(A_{O}\right)$ and

$$
\lambda-A_{\Psi}=\left(\lambda-A_{0}\right)\left(I d-1 / \lambda \cdot \Psi 8 e_{\lambda}\right) .
$$

Using the Neumann series expansion of $\left(I d-1 / \lambda \cdot \Psi \otimes e_{\lambda}\right)^{-1}$ one obtains the following estimate:\\
(2.28) $\|\left(\text { Id }-1 / \lambda \cdot \Psi \otimes e_{\lambda}\right)^{-1} \| \lambda /(\lambda-\|\Psi\|) \quad$ if $\lambda>\|\Psi\|$.

It follows from (2.25) and (2.26) that for $\lambda>\|\psi\| \quad R\left(\lambda, A_{\psi}\right)$ exists and satisfies $\left\|\mathrm{R}\left(\lambda, \mathrm{A}_{\Psi}\right)\right\| \leq \lambda /(\lambda-\|\psi\|) \cdot 1 / \lambda=1 /(\lambda-\|\psi\|)$. Then the Hille-Yosida Theorem (A-II,Thm.1.7) implies that $A_{\psi}$ generates a semigroup (T(t)) satisfying $\|T(t)\| \leqq \exp (\|\Psi\| t)$, Moreover, this semigroup is eventually norm continuous (see B-IV, Cor. 3.3).\\
By B-II,Ex.1.22 we have the following equivalence:\\
(2.29) $A_{\Psi}$ generates a positive semigroup if and only if $\Psi+r \delta_{0} \geqq 0$ for some $r \in \mathbb{R}$.

Thus Cor. 2.12 is applicable if $\Psi+r_{0} \geqq 0$ for some $r \in \mathbb{R}$. Since every eigenvalue of $A_{\Psi}$ is an eigenvalue of $A_{m}$ and since ker $\left(\lambda-A_{m}\right)$ $=\left\{\alpha e_{\lambda}: \alpha \in \mathbb{C}\right\}$, the spectral bound $s\left(A_{\psi}\right)$ is determined by the (unique) real $\lambda \in \mathbb{R}$ such that $e_{\lambda} \in \mathrm{D}\left(\mathrm{A}_{\Psi}\right)$ or equivalently, $\lambda$ is a solution of the so-called characteristic equation


\begin{equation*}
\lambda=\Psi\left(e_{\lambda}\right), \lambda \in \mathbb{R} . \tag{2.30}
\end{equation*}


(The assumption $\Psi+r_{\delta_{0}} \geqq 0$ implies that the function $\lambda \rightarrow \Psi\left(e_{\lambda}\right)$ is strictly decreasing and $\lim _{\lambda \rightarrow \infty}\left\langle e_{\lambda}, \Psi\right\rangle>-\infty, \lim _{\lambda \rightarrow-\infty}\left\langle e_{\lambda}, \Psi\right\rangle=\infty$ unless $\psi=r_{0} \delta_{0}$ for some $r_{0} \in \mathbb{R}$.)

We conclude this section with some additional remarks related to Thm.2.9 and its corollaries.

Remarks 2.15.(a) If $s(A)$ is a pole of the resolvent, then for generators of positive semigroups one has the following equivalences:\\
(i) $s(A)$ is a first order pole.\\
(ii) For every $0<f \in \operatorname{ker}(\mathrm{~s}(\mathrm{~A})-\mathrm{A})$ there exists $0 \leqq \Psi \in \operatorname{ker}\left(s(A)-A^{\prime}\right) \quad$ such that $\left\langle E_{q} \psi\right\rangle>0$.\\
(iii) For every $0<\Psi \in \operatorname{ker}\left(s(A)-A^{\prime}\right)$ there exists $0 \leqq \mathrm{f} \in \operatorname{ker}(\mathrm{s}(\mathrm{A})-\mathrm{A}) \operatorname{such}$ that $\langle\mathrm{f}, \Psi\rangle\rangle 0$.

In particular, if ker(s(A) - A) contains a strictly positive function or if ker(s(A) - A') contains a strictly positive measure, then $s(A)$ is a first order pole.

We sketch the proof of (i) $\Leftrightarrow$ (ii) assuming that $s(A)=0$. If 0 is a first order pole, then the residue $P$ is a positive projection satisfying $P E=\operatorname{ker} A, P^{\prime} E^{\prime}=\operatorname{ker} A^{\prime}$ (see A-III, 3.6). Thus given $0<f \in$ ker $A$ and any $0 \leqq \phi \in E^{\prime}$ such that $<f, \phi>>0$, we have for $\left.\Psi:=\mathrm{P}^{\prime} \phi:\langle\mathrm{f}, \Psi\rangle=\left\langle\mathrm{f}, \mathrm{P}^{\prime} \phi\right\rangle=\langle\mathrm{Pf}, \phi\rangle=\langle\mathrm{f}, \phi\rangle\right\rangle 0$. To prove the reverse direction we first observe that the highest coefficient $Q_{k}$ of the Laurent expansion is a positive operator. Thus if 0 is a pole of order $k \geq 2$ we choose $0<h \in E$ such that $f:=Q_{k} h>0$. Then $A f=A Q_{k} h=0$ and for every $\Psi \in$ ker $A$ ' we have\\
$\langle\mathrm{f}, \Psi\rangle=\left\langle Q_{k} h, \Psi\right\rangle=\left\langle h_{,} Q_{k}{ }^{\prime} \Psi\right\rangle=\left\langle h_{,} Q_{k-1}{ }^{\prime}{ }^{\prime} \Psi\right\rangle=0$.\\
(b) If a linear operator $s$ on $c_{0}(X)$ is weakly compact, then $s^{2}$ is compact (see B-IV,Prop.2.4(b)). Therefore every non-zero spectral value of a weakly compact operator is a pole of the resolvent. This shows that Thm. 2.9 is applicable if either $T\left(t_{0}\right)$ is weakly compact for some $t_{0}$ or $R(\lambda, A)$ is weakly compact for some $\lambda \in \rho(A)$. We quote two criteria for weak compactness:\\
(2.31) If $\mathrm{T} \in \mathrm{L}(\mathrm{C}(\mathrm{K})), \mathrm{K}$ compact, is positive, then it is weakly compact if and only if its biadjoint $\mathrm{T}^{\text {" }}$ maps the bounded Borel functions into $C(K)$ (see B-IV,Prop.2.4).\\
(2.32) A positive operator T on $\mathrm{C}_{\mathrm{O}}(\mathrm{X})$ which is dominated by a finite rank operator, is weakly compact. (Actually, its adjoint $T$ is dominated by a finite rank operator as well, hence it maps the unit ball in an order interval. It follows that $\mathrm{T}^{\prime}$ is weakly compact hence so is T .)\\
(c) Stronger results than Thm. 2.9 will be proved in Chapter C-III. Actually, assuming only that $s(A)$ is a pole of finite algebraic multiplicity one can show that ${ }^{\sigma_{b}}(\mathrm{~A})$ contains only poles of finite multiplicity (C-III,Thm.3.13). In C-III, Cor.2.12 we will show that $\sigma_{b}(A)$ is cyclic whenever $s(A)$ is a pole of the resolvent.\\
(d) Example 2.14 (b) can be extended to systems of functional differential equations even the infinite dimensional case. For details we refer to Sec. 3 of Chapter B-IV.

\section*{3. IRREDUCIBLE SEMIGROUPS}
In the case of matrices it is well known that considerably stronger results are available if one considers positive matrices which are irreducible. The concept of irreducibility can be extended to our setting and in many cases one can check easily whether a given semigroup has this property (see Ex. 3.4). We will show that irreducible semigroups have many interesting properties. For example, the spectrum o(A) is always non-empty, positive eigenfunctions are strictly positive and if s(A) is a pole, it is algebraically (and geometrically) simple (see Prop.3.5) . Moreover, in certain cases irreducibility ensures that $\sigma_{b}(A)$ and $P_{\sigma_{b}}(A)$ are not only cyclic subsets but 'subgroups' (see Thm.3.5 and Thm.3.11 for details).\\
We start the discussion with several, mutually equivalent, definitions of irreducibility.

Definition 3.1. A positive semigroup $T=(T(t))$ on $E=C_{0}(X)$, $X$ locally compact, with generator A is called irreducible if one of the following mutually equivalent conditions is satisfied:\\
(i) There is no T-invariant closed ideal except $\{0\}$ and E.\\
(ii) Given $0<\pounds \in E, 0<\phi \in E^{\prime}$, then $\left\langle T\left(t_{0}\right) f, \phi\right\rangle>0$ for some $t_{0} \geqq 0$.\\
(iii) For every $f \in E_{+}$we have $U_{t \geqq 0}\{x \in X:(T(t) f)(x)>0\}=X$.\\
(iv) For some (every) $\lambda>s(A)$ there exists no closed ideal\\
which is invariant under $R(\lambda, A)$ except $\{0\}$ and $E$.\\
(v) For some (every) $\lambda>s(A)$ we have:\\
$R(\lambda, A) f$ is strictly positive whenever $\mathrm{f}>0$.\\
(vi) $U_{t \geqq 0} \operatorname{supp} T(t)^{\prime} \delta_{x}$ is dense in $x$ for every $x \in X$.

That these six conditions are actually equivalent can be seen as follows:\\
(i) $\rightarrow$ (ii): Suppose there are $0<E \in E, 0<\phi \in E^{\prime}$ such that $\langle T(t) f, \phi\rangle=0$ for every $t \geqq 0$. Then the ideal $I$ generated by\\
\includegraphics[max width=\textwidth]{2024_12_23_c6487cc0859199a15bd9g-192} Obviously I is T-invariant.\\
(ii) $\rightarrow$ (iii): Given $0<\pounds \in E, x \in X$. By (ii) there exists $t_{0}$ such that $\left(T\left(t_{0}\right) f\right)(x)=\left\langle T\left(t_{0}\right) f, \delta_{x}\right\rangle>0$.\\
(iii) $\rightarrow$ (vi): Suppose that $U_{t \geq 0} \operatorname{supp} T(t)^{\prime} \delta y$ is not dense for some $y \in X$. Then there exists $f_{O} \in E, f_{O}>0$ such that\\
supp $f_{0} \cap \operatorname{supp} T(t)^{\prime} \delta=\emptyset$ for every $t \geqq 0$. Hence $\left(T(t) f_{0}\right)(y)=$ $<f_{0}, T(t) ' \delta_{y}>>$, that is, $y \notin U_{t \geq 0}\left\{x \in X:\left(T(t) f_{0}\right)(x)>0\right\}$. (vi) $\rightarrow$ (v): Given $0<\mathrm{f} \in \mathrm{E}, \lambda>\omega(\mathrm{A}), \mathrm{y} \in \mathrm{X}$, there exists $t_{0} \geqq 0$ such that $\{x: f(x)>0\} \cap \operatorname{supp} T\left(t_{0}\right)^{\prime} \delta_{y} \neq \emptyset$. Hence, $\left(T\left(t_{0}\right)(f)(y)=\left\langle f, T\left(t_{0}\right)^{\prime} \delta_{y}\right\rangle>0\right.$ and therefore $(R(\lambda, A) f)(y)=\int_{0}^{\infty} e^{-\lambda t}(T(t) f)(y) d t>0$. since $\lambda \rightarrow R(\lambda, A) f$ is decreasing in the interval $(s(A), \infty)$ (use the resolvent equation and the fact that $R(\lambda, A)$ is positive) we have $R(\lambda, A) f \gg 0$ for all $\lambda>s(A)$.\\
(v) $\rightarrow$ (vi): If I is a $R(\lambda, A)$-invariant ideal and $0<f \in I$, then $g:=R(\lambda, A) f \in I$. By (v) $g$ is strictly positive thus $I$ has to be dense (it contains all functions of compact support).\\
(iv) $\rightarrow$ (i): At first we recall that a closed linear subspace which is invariant for $R\left(\lambda_{0}, A\right) \quad\left(\lambda_{0} \epsilon \rho(A)\right)$, is invariant for $R(\lambda, A)$ whenever $\lambda$ and $\lambda_{0}$ belong to the same component of $\rho(A)$. Hence by $A-I, 3.2$ every $R\left(\lambda_{O}, A\right)$-invariant subspace where $\lambda_{0} \in \rho_{+}(A)$ is T-invariant and vice versa.

Remark 3.2. Obviously, irreducibility of a semigroup (T)(t)) $t \geq 0$ is implied by the following condition:\\
(vii) $T(t) f \gg 0$ whenever $f>0$ and $t>0$.

The rotation semigroup (see $A-1,2.5$ ) is irreducible but it does not satisfy condition (vii). However, assuming that the semigroup (T(t)) is holomorphic, then (vii) of Def. 3.1 is equivalent to irreducibility. We will give a proof of this result in the more general situation of Banach lattices (see C-III, Thm.3.2(b)).

A semigroup $(T(t))_{t \geqslant 0}$ is irreducible if and only if $\left(e^{-\alpha t_{T}(t)}\right)_{t \geqslant 0}$, $\alpha \in \mathbb{R}$ is. More generally, irreducibility is invariant under perturbations by multiplication operators. In fact, we have the following result:

Proposition 3.3. Suppose A generates a positive semigroup $T$ on $C_{0}(X)$ and let $h$ be a continuous, bounded real-valued function on X . Then the semigroup $S$ generated by $B:=A+M_{h}$ is irreducible if and only if $T$ has this property.

Proof. Since every closed ideal is of the form $\left\{\mathrm{f} \in \mathrm{E}: \mathrm{f}_{\mid M}=0\right\}$ where $M \in X$ is a closed subset (cf. Sec. 1 of $B-I$ ) it is clear that all closed ideals are invariant under the multiplication operator $M_{h}$ and $M_{-h}$ respectively. Thus the assertion follows from the expansions which are true for $\lambda$ sufficiently large.

$$
\begin{aligned}
& R(\lambda, B)=\left(1-R(\lambda, A) M_{h}\right)^{-1} R(\lambda, A)=\sum_{n=0}^{\infty} \quad\left(R(\lambda, A) M_{h}\right)_{R} n_{R}(\lambda, A) \\
& R(\lambda, A)=\left(1-R(\lambda, B) M_{-h}\right)^{-1} R(\lambda, B)=\sum_{n=0}^{\infty} \quad\left(R(\lambda, B) M_{-h}\right)^{n_{R}(\lambda, B)}
\end{aligned}
$$

Before discussing further properties of irreducible semigroups we consider several examples.

Examples 3.4.(a) (cf. B-II, Sec.3). Suppose $(T(t))_{t \geqslant 0}$ is governed by a continuous semiflow $\phi: \mathbb{R}_{+} \times X \rightarrow X$, i.e., $T(t) f=f_{0} \phi_{t} \quad\left(f \in C_{0}(X)\right)$. Then the following assertions are equivalent:\\
(i) $\quad(T(t))_{t \geq 0}$ is irreducible.\\
(ii) There is no closed subset of x which is $\phi$-invariant except $\emptyset$ and $X$.\\
(iii) Every orbit $\left\{\phi(t, x): t \in \mathbb{R}_{+}\right\}$is dense in $\mathbb{X}$.

More generally, these equivalences still hold if the semigroup (T(t)) is given by $T(t) f=h_{t} \cdot\left(f \circ \phi_{t}\right.$ ) where $h_{t}$ are suitable continuous, strictly positive, bounded functions on X .\\
(b) Suppose that the semigroup (T( $t)_{t \geq 0}$ has the following form: There exist a positive measure $\mu$ on $X$ and a positive continuous function $k=(0, \infty) \times X \times X \rightarrow \mathbb{R}$ such that\\
(3.1) $(T(t) f)(x)=\int_{X} k(t, x, y) f(y) d \mu(y) \quad\left(t>0, f \in C_{0}(x), x \in X\right)$.

$$
\begin{aligned}
& \text { Then }(T(t)){ }_{t \geqq 0} \text { is irreducible if and only if } \\
& U_{t>0} \operatorname{supp}\{k(t, x, .)\} \text { is dense in } x \text { for every } x \in x \text {. }
\end{aligned}
$$

(c) We consider the first derivative $A f=f^{\prime}$ (cf. A-I,2.4). If $E=C_{0}(\mathbb{R})$, then the corresponding semigroup $(T(t))_{t \geqq 0}$ is not irreducible. Note however, that there is no closed invariant ideal I\\
\includegraphics[max width=\textwidth]{2024_12_23_c6487cc0859199a15bd9g-194} generated by A .\\
For $E=C_{0}[0, \infty)$ and $E=C_{0}(-\infty, 0)$ the corresponding semigroups are reducible (i.e. not irreducible) as well. If $E=C_{2 \pi}$ (R) (i.e. the $2 \pi$-periodic functions), then $A f=f^{\prime}$ generates an irreducible semigroup on E . It is (isomorphic to) the semigroup of rotations on the unit circle.\\
(d) (cf. Ex.2.14(b)) Consider $A f=f^{\prime}$ on $E=c[-1,0]$ with $D\left(A_{\psi}\right)=\left\{f \in C^{1}: f^{\prime}(0)=\Psi(f)\right\}$ where the linear functional $\Psi$ satisfies $\Psi+\alpha \delta_{0} \geqq 0$ for some $\alpha \in R$ (see B-II,Ex.1.22). The corresponding semigroup is irreducible if and only if $-1 € \operatorname{supp} \Psi$.\\
(e) The second derivative $A f=f^{\prime \prime}$ generates an irreducible semigroup on $C_{0}(\mathbb{R})$ and on $C_{0}(0,1)$ (cf. $A-I, 2.7$ ) . With Neumann boundary conditions (or more generally: $\mathrm{f}^{\prime}(0)=\alpha_{0} f(0), f^{\prime}(1)=\alpha_{1} f(1)$ where $\alpha_{0}, \alpha_{1} \in \mathbb{R}$ ) the second derivative generates an irreducible semigroup on c[0,1] (cf. A-I, 2.7).\\
The operator $A f=f^{\prime \prime}-V f$ on $C_{0}(\mathbb{R})$, where $V$ is continuous, real-valued with inf $V(x)>-\infty$ (see Example $2.14(a))$ also generates an irreducible semigroup. This can be derived from the maximum principle as follows: For $\lambda>\inf V(x), f \in C_{0}(\mathbb{R}), g:=\mathbb{R}(\lambda, A) f$ we have $g \in C^{2}$ and $g^{\prime \prime}-(\lambda+V) g=-f$. If $f>0$, then $g>0$, hence [Protter-Weinberger (1967), Chap.I, Thm.3] implies that $g$ is strictly positive.\\
(f) The Laplacian $\Delta$ generates an irreducible semigroup on $C_{0}\left(\mathbb{R}^{n}\right)$ as can be seen easily from A-I, 2.8. More general elliptic operators will be discussed below (see Ex. $3.10(\mathrm{~b})$ ).

We now return to the general situation and show that irreducible semigroups possess several interesting properties.

Proposition 3.5. Suppose A is the generator of a strongly continuous semigroup on $C_{0}(X)$ which is irreducible. Then the following assertions are true:\\
(a) $\sigma(A) \neq \varnothing$;\\
(b) every positive eigenfunction of A is strictly positive;\\
(c) every positive eigenvector of $A^{\prime}$ is strictly positive;\\
(d) if $\operatorname{ker}\left(\mathrm{s}(\mathrm{A})-\mathrm{A}^{\mathrm{Y}}\right.$ ) contains a positive element (e.g., if X is compact (cf. Thm.1.6)), then dim(ker(s(A) - A)) 1 ;\\
(e) if s(A) is a pole of the resolvent, then it is algebraically simple. The residue has the form $\mathrm{P}=\phi \otimes \mathrm{u}$ where $\phi \in \mathrm{E}^{\prime}$ and $u \in E$ are strictly positive eigenvectors of $A$ and $A$, respectively, satisfying $\langle u, \phi\rangle=1$.

Proof. (a) Take any $f_{0} \in C_{0}(X)$ which is positive and has compact support. If $\lambda>s(A)$, then $R(\lambda, A) f_{0}$ is strictly positive (by Def.3.1(v)), hence there exists $\varepsilon>0$ such that $R(\lambda, A) f_{0} \geqq \varepsilon f_{0}$. It follows that $R(\lambda, A){ }^{n_{O}} \geq \varepsilon^{n_{I_{O}}} \geqq 0$ for all $n \in \mathbb{N}$ and therefore\\
$r(R(\lambda, A))=\lim \left\|R(\lambda, A)^{n}\right\|^{1 / n} \geqq \varepsilon>0$.\\
The assertion now follows from A-III, Prop.2.5 .\\
(b) Suppose $A h=r h$ where $h \neq 0$ is positive. Then $r$ has to be real and we have $T(t) h=e^{r t_{h}}$ (A-III,Cor.6.4). For $|f| \leqq n \cdot h$ ( $\mathrm{n} \in \mathbb{N}$ ) we have\\
(3.2) $|T(t) f| \leqq T(t)|f| \leqq n \cdot T(t) h=n \cdot e^{r t h}$.

This shows that the ideal generated by $h$ is invariant, hence dense by irreducibility. This is true if and only if $h$ is strictly positive.\\
(c) Suppose $A^{\prime} \phi=r \phi$ for some $0<\phi \in E^{\prime}$. Again $r$ has to be real and $T(t)^{\prime} \phi=e^{r t_{\phi}} \quad(t \geqq 0)$. From\\
(3.3) $\langle | T(t) f|, \phi\rangle \leqq\langle T(t)| f|, \phi\rangle=\langle | f\left|, e^{r t} \phi\right\rangle$, $f \in E$\\
it follows that $\mathrm{I}:=\{f \in E: \phi(|f|)=0\}$ is an invariant ideal. We have I $\neq \mathrm{E}$ (because $\phi \neq 0$ ), hence the irreducibility implies I = $\{0\}$, i.e., $\phi$ is strictly positive.\\
(d) By (a) we know that $s(A)>-\infty$ hence we can assume without loss of generality that $s(A)=0$. By (c) there exists a strictly positive $\Psi \in E^{\prime}$ such that $A^{\prime} \Psi=0$. It follows from (2.14) and (2.15) that\\
(3.4) h $\in$ ker A implies $|h| \in$ ker A.

Assuming $\operatorname{dim}(k e r A) \geqq 2$, then there is an eigenfunction $h \in$ ker $A$, $\mathrm{h} \neq 0$ which has at least one zero in $\mathrm{x} \quad\left(\mathrm{h}:=\mathrm{h}_{1}\left(\mathrm{x}_{0}\right) \cdot \mathrm{h}_{2}-\mathrm{h}_{2}\left(\mathrm{x}_{\mathrm{o}}\right) \cdot \mathrm{h}_{1}\right.$, where $h_{1}, h_{2}$ are linearly independent, $x_{0} \in \mathrm{X}$ ). By (3.4) |h| is a positive eigenfunction but not strictly positive. This is a contradiction with (b).\\
(e) If $s(A)$ is a pole, then there exists a corresponding positive eigenfunction (see the proof of Cor.1.4). By (b) it is even strictly positive, thus $\mathrm{s}(\mathrm{A})$ is a first order pole by Rem.2.15(a) . The residue $P$ is a positive operator satisfying $\mathrm{PE}=\operatorname{ker}(\mathrm{s}(\mathrm{A})-\mathrm{A})$ and P'E' $^{\prime}=k e r\left(s(A)-A^{\prime}\right)$, therefore the remaining assertion follows from (e), (b) and (d).

In the remainder of this section we focus our interest on the boundary spectrum of irreducible semigroups, more precisely, on the eigenvalues and the corresponding eigenfunctions of the boundary spectrum. In view of assertion (a) of Prop.3.5 the assumption "s $(\mathrm{A})=0 "$ is not crucial in the following theorem. However, it allows a simpler formulation.

Theorem 3.6. Suppose $T=(T(t))$ is an irreducible semigroup with generator $A$ and spectral bound $s(A)=0$. Assume that there exists a positive linear form $\Psi \neq 0$ such that $A^{\prime} \Psi=0$. (This is automatically satisfied whenever $X$ is compact (see Thm.1.6).)\\
If $P \sigma(A) \cap i \mathbb{R}$ is non-empty, then the following assertions are true:\\
(a) $\operatorname{Po}(A) \cap \mathcal{R}$ is a (additive) subgroup of $i R$.\\
(b) The eigenspaces corresponding to $\lambda \in \operatorname{Po}(\mathrm{A}) \cap \mathrm{R}$ are one-dimensional.\\
(c) If $A h=i \alpha h \quad(h \neq 0, \alpha \in \mathbb{R})$, then $h$ has no zeros in $x$. In case $\alpha=0$ then $h(x) /|h(x)|$ is constant; otherwise, $\{h(x) /|h(x)|: x \in x\}$ is a dense subset of $\Gamma$.\\
(d) If Ah $=i \alpha h \quad(h \neq 0, \alpha \in \mathbb{R})$, then\\
(3.5) $S_{h}(D(A))=D(A)$ and $S_{h}^{-1} \circ A \circ S_{h}=(A+i \alpha)$.

In particular, spectrum and point spectrum of A are invariant under translations by ia.\\
(e) 0 is the only eigenvalue admitting a positive eigenfunction.

Proof. By Prop.3.5(c) the invariant linear form $\psi$ is strictly positive and it satisfies $T(t)^{\prime \Psi}=\Psi(t \geqq 0) \cdot$\\
(d) Supposing $A h=i \alpha h \quad(h \neq 0, \alpha \in \mathbb{R})$ then $A|h|=0$ by (2.14) and (2.15). By Prop.3.5(b) $|h|$ is strictly positive, thus Thm.2.4(b) implies (3.5).\\
(b) Assertion (d) implies that $S_{h}$ maps ker(ia +A ) onto ker A whenever $i \alpha \in P \sigma(A) \cap_{i} \mathbb{R}$. Moreover, we have seen in the proof of (d) that ker $A \neq\{0\}$ hence it is one-dimensional by Prop.3.5(d).\\
(a) Assume that $\mathrm{Ah}=i \alpha h, \mathrm{Ag}=i \beta g(\alpha, \beta \in \mathbb{R}, \mathrm{~h} \neq 0, g \neq 0)$. By (3.5) we have $S_{\bar{g}} A S_{g}=A+i B$ and $S_{h} A S_{\bar{h}}=A-i \alpha$, therefore


\begin{equation*}
A+i(B-\alpha)=S_{h}(A+i B) S_{\bar{h}}=S_{h} S_{g} A S_{g} S_{h} . \tag{3.6}
\end{equation*}


It follows that $\operatorname{ker}[A+i(B-\alpha)]=S_{h} S_{g}(\operatorname{ker} A) \neq\{0\}$, hence $i(B-\alpha) \in P o(A)$.\\
(e) If $A f=\lambda f$ where $f>0$, then


\begin{equation*}
\lambda \cdot\langle f, \Psi\rangle=\langle A f, \Psi\rangle=\left\langle f, A^{\prime} \Psi\right\rangle=0 . \tag{3.7}
\end{equation*}


Since $\Psi$ is strictly positive we have $\langle f, \Psi\rangle>0$ hence $\lambda=0$.\\
(c) We already know that $\mathrm{Ah}=i \alpha h$ implies that $\mathrm{A}|\mathrm{h}|=0$. It follows from Prop.3.5(b) that $h$ is strictly positive; i.e., $h$ has no zeros in X . By Prop.3.5(d) ker A is one-dimensional hence every\\
eigenfunction corresponding to 0 is the scalar multiple of a strictly positive function. If $A h=i \alpha h, h \neq 0, \alpha \neq 0$ we consider $\tilde{h}(\mathrm{x}):=\mathrm{h}(\mathrm{x}) /|\mathrm{h}(\mathrm{x})|$. Assuming that $\tilde{\mathrm{h}}(\mathrm{x})$ is not not dense in $\Gamma$, there exists a sequence of polynomials ( $p_{n}$ ) neN such that


\begin{equation*}
p_{n}(z) \rightarrow 1 / z \text { uniformly in } z \in(\mathrm{f}(\mathrm{x}) \text {. } \tag{3,8}
\end{equation*}


It follows that $h(x) \cdot p_{n}(\tilde{h}(x)) \rightarrow|h|(x)$ uniformly in $x \in X$. Obviously, $h \cdot p_{n}(h)$ is a linear combination of $h^{[1]}, h^{[2]}, h^{[3]}, \ldots$, that is, it is an element of $\operatorname{span}\left\{U_{k=1} \operatorname{ker}(i k a-A)\right\} \quad$ (cf. Thm.2.4). By (3.7) the linear form $\Psi$ vanishes on $\operatorname{ker}(\lambda-A)$ whener $\lambda \neq 0$. Therefore $\left\langle h * p_{n}(h), \psi\right\rangle=0$ and we have $0<\langle | h|, \Psi\rangle=\lim _{n \rightarrow \infty}\left\langle h \cdot p_{n}(h), \Psi\right\rangle=0$ which is a contradiction.

The group Po(A)niR need not be discrete. For example, the semigroup described in Ex.2.6(d) satisfies the assumptions of Thm. 3.6 if $\alpha / \beta$ is irrational. In this case $P_{\sigma}(A)=i \alpha \mathbb{Z}+i \beta \mathbb{Z}$ is a dense subgroup of $i \mathbb{R}$. Actually one can show that for every subgroup $H$ of $i \mathbb{R}$ there is an irreducible semigroup on $C(G), G:=\left(H_{Q}\right)^{\wedge}$, such that $P_{\sigma}(A)=$ $H$. Here $\left(H_{d}\right)^{\wedge}$ denotes the dual of the abelian group $H$ equipped with the discrete topology. For details see [Greiner (1982), p.62].

An immediate consequence of assertion (d) of Theorem 3.6 is the following corollary.

Corollary 3.7. Suppose $T$ satisfies the hypotheses of Thm.3.6 and let $A$ be its generator. If $k$ is a bounded continuous real-valued function, $M_{k}$ the corresponding multiplication operator, then for $B:=A+M_{k}$ we have $\sigma(B)+\operatorname{Po}(A) \cap i \mathbb{R}=\sigma(B)$.\\
In particular, $s(B)+P \sigma(A) \cap i \mathbb{R} \subset \sigma(B)$.

The next two corollaries are essentially consequences of assertion (c) of Theorem 3.6. The statement of the first one can be summarized as follows: In case there are non-real eigenvalues in the boundary spectrum then the semigroup 'contains' the semigroup of rotations on $\Gamma$.

Corollary 3.8. Suppose that the hypotheses of Thm.3.6 are satisfied and that there is an eigenvalue ia of A with $\alpha>0$.\\
Let $\tau:=2 \pi / a$.\\
Then there exists a continuous injective lattice homomorphism\\
$j: C(\Gamma)+C_{0}(X)$ such that the diagram\\
\includegraphics[max width=\textwidth, center]{2024_12_23_c6487cc0859199a15bd9g-199}\\
commutes. ( $R_{\tau}(t)$ ) denotes the rotation semigroup of period $\tau$ (see A-I, 2.5).

If $X$ is compact, then $j$ is a topological embedding.

Proof. Assume that $A h=i \alpha h, \alpha>0$, and let $\tilde{h}(x):=h(x) /|h(x)|$. Then we define $j$ by\\
(3.9) $j(f):=|h| \cdot f \circ \tilde{h} \quad(i . e .,(j(f))(x)=|h(x)| \cdot f(\tilde{h}(x))$ ).

Obviously, $j$ is a lattice homomorphism and because $h$ has no zeros and $\tilde{h}$ has a dense image in $P$ (Thm.3.6(c)), it follows that $j$ is injective. For the functions $e_{n} \in C(\Gamma)$ given by $e_{n}(z)=z^{n} \quad(n \in \mathbb{Z})$ one has $j\left(e_{n}\right)=h^{[n]} \quad(n \in \mathbf{Z})$ and therefore\\
$T(t) \circ j\left(e_{n}\right)=T(t) h^{[n]}=e^{i n \alpha t} \cdot h^{[n]}$ (cf. Thm.2.4) and $j \circ R_{\tau}(t)\left(e_{n}\right)=j\left(e^{i n \alpha t} e_{n}\right)=e^{i n \alpha t} \cdot h^{[n]}$.\\
since $\left\{e_{n}: n \in \mathbb{Z}\right\}$ is a total subset of $C(I)$ we have $T(t) \circ j=j \circ R_{\tau}(t)$ for every $t>0$.\\
If X is compact, then $\tilde{\mathrm{h}}(\mathrm{X})$ is closed, hence $\tilde{\mathrm{h}}$ is onto, moreover, $|h| \geqq \varepsilon$ for some $\varepsilon>0$ thus the definition of $j$ implies that $\|j(f)\|>E\|f\|$ for every $f \in C(\Gamma)$.

A consequence of cor. 3.8 is the following: If $\{s(A)\} \varsubsetneqq \mathrm{Po}(\mathrm{A}) \cap \mathrm{R}$, then for every $\varepsilon>0$ there exists $g>0$ such that $T(t) g$ and $T(s) g$ have disjoint support whenever $|s-t|=\varepsilon$. Another consequence is that there exist positive functions $\mathrm{f}_{1}$ and $\mathrm{f}_{2}$ such that $T(t) f_{1}$ and $T(t) f_{2}$ have disjoint support for every $t \geqq 0$ (consider the images under $j$ of two disjoint functions on $C(r))$. This observation proves the following corollary.

Corollary 3.9. Suppose that the hypotheses of Thm.3.6 are satisfied and that for some $t_{0}>0$ we have $T\left(t_{0}\right) f>0$ whenever $f>0$. Then $\operatorname{P\sigma }(\mathrm{A}) \cap i \mathbb{R}=\{0\}$.

Cor. 3.9 can be applied if $\mathrm{T}_{\left(t_{0}\right)}$ is a kernel operator with strictly positive kernel. We give some examples:

Examples 3.10. (a) We assume that the semigroup (T(t)) satisfies the hypotheses of Thm. 3.6 and that it is given by

$$
(T(t) f)(x)=\int_{X} k(t, x, y) f(y) d \mu(y),
$$

where $\mu$ is a positive measure and $k$ is a positive continuous function (see Ex. 3.4(b)). We will show that $\operatorname{Po}(A) \cap(s(A)+i \mathbb{R})=\{s(A)\}$. Assuming the contrary, by Thm. $3.6(d)$ there exist $\alpha \neq 0$, $h \in C_{0}(x)$ such that


\begin{equation*}
S_{\bar{h}}^{\circ} T(t) \circ S_{h}=e^{i \alpha t} \cdot T(t) \text { for all } t \geq 0 \tag{3.10}
\end{equation*}


This implies that $k$ satisfies


\begin{equation*}
\frac{\overline{h(x)}}{T h(x)} \cdot \frac{h(y)}{\mid h(y)} \cdot k(t, x, y)=e^{i \alpha t} k(t, x, y) \quad(t>0, x, y \in x) \tag{3.11}
\end{equation*}


It follows that for $0<|s-t|<2 \pi / a k(t, \ldots)$ and $k(s, \ldots)$ have disjoint support. This is impossible if $k$ is continuous.\\
(b) Let $\Omega$ be a domain in $\mathbb{R}^{n}$ and define $L_{0}$ as follows:

$$
\begin{aligned}
& L_{o} f:=\sum_{i, j=1}^{n} a_{i j} f_{i j}^{\prime}+\sum_{i=1}^{n} b_{i} f_{i}^{\prime}+c f, \\
& \text { with domain } D\left(L_{0}\right):=\left\{f \in C_{0}(\Omega): f \text { is } c^{\infty}, L_{0} f \in C_{0}(\Omega)\right\} .
\end{aligned}
$$

$$
\text { (Here } f_{i}^{\prime} \text { stands for } \partial f / \partial x_{i} \text {, thus } f_{i j}^{\prime \prime}=\partial^{2} f / \partial x_{i} \partial x_{j} \text {.) }
$$

Suppose that $L_{c}$ is elliptic, $a_{i j}, b_{i}, c$ are real-valued $c^{\infty}$-functions with $\lambda_{0}:=\sup c<\infty$, assume further that the closure $L$ of $L_{0}$ is the generator of a positive semigroup on $C_{0}(\Omega)$ which has compact resolvent. For example this is true if $\partial \Omega$ is $C^{\infty}$ and $a_{i j} \in C^{\infty}(\bar{\Omega}) \quad$ (cf Thm.4.8.3 of Fattorini (1983)). We will show that $\operatorname{Po}(A) \cap(s(A)+i R)=\{s(A)\}$. In order to apply Thm. 3.6 we have to show that the corresponding semigroup ( $T(t)$ ) is irreducible:\\
Given $0<f \in E$ then there is $g \in D\left(L_{0}\right)$ such that $0<g \leqq f$. $h:=R(\lambda, L) g$ is $C^{\infty}$ (Weyl's Lemma) and satisfies $L_{0} h-\lambda h=-g<0$. Assuming that $\lambda>\lambda$ o then $h$ is positive, even strictly positive by the maximum principle [Protter-Weinberger (1967), Chap.2, Thm.6]. It follows from $R(\lambda, L) f \geqq R(\lambda, L) g=h \gg 0$ that $(T(t))$ is irreducible. Next we apply Thm.3.6(d) in order to show that the spectral bound is a dominant eigenvalue. We can assume that $s(\mathrm{~L})=0$. If $\mathrm{s}(\mathrm{L})$ is not dominant, then by Thm.3.6(d) we have\\
(3.12) $L_{0} h=i \alpha h, L_{0}|h|=0, L_{0} \bar{h}=-i \alpha \bar{h}$ for some $h \neq 0, \alpha>0$

If we define $u:=|h|$ and $w:=h /|h|$, then (3.12) reads


\begin{equation*}
L_{0}(u w)=i \alpha u w, L_{O}(u)=0, I_{O}(u / w)=-i \alpha \cdot u / w \tag{3.13}
\end{equation*}


Explicit calculation of $L_{0}(u w)$ and $L_{0}(u / w)$ using the product formula for differentation yields (as above we write $f_{i}^{\prime}$ instead of $\left.\partial f / \partial x_{i}\right):$


\begin{align*}
& L_{o}(u w)=w L_{o}(u)+u \sum_{i, j} a_{i j} w_{i j}^{\prime}+\sum_{i}\left(u b_{i}+\sum_{j} a_{i j} u_{j}^{\prime}\right) w_{i}^{\prime}  \tag{3.14}\\
& L_{o}(u / w)=1 / w \cdot L_{o}(u)+u \sum_{i, j} a_{i j}(1 / w)_{i j}^{\prime}+\sum_{i}\left(u b_{i}+\sum_{j} a_{i j} u_{j}^{\prime}\right)(1 / w)
\end{align*}


Observing that $(1 / w) w_{i}=-w^{-2} \cdot w_{i}^{\prime}$ and $(1 / w)_{i j}^{\prime}=w^{-3} \cdot\left(2 w_{i}^{\prime} w_{j}^{\prime}-w_{i j}^{\prime}\right)$ we obtain:


\begin{equation*}
L_{0}(u w)+w^{2} L_{0}(u / w)=2 w L_{0}(u)+2 u / w \cdot \sum_{i j} a_{i j} w_{i}^{\prime} w_{j}^{\prime} . \tag{3.15}
\end{equation*}


This identity and (3.13) implies that $2 u / w \cdot \sum_{i j} a_{i j} w_{i}^{\prime} w_{j}^{\prime}=0$. Since u has no zeros and (a ${ }_{i j}$ ) is positive definite, it follows that grad $w=\left(w_{i}^{\prime}\right)=0$ in $\Omega$, hence $w=$ const. . Then by (3.13) we have iauw $=L_{O}(u w)=w_{O}(u)=0$, a contradiction.

The assumption that $L_{0}$ is elliptic, i.e., that ( ${ }_{i j}$ ) is positive definite, is essential in order to show that there is only one eigenvalue in the boundary spectrum. In the following example (a ij) is positive semi-definite and $\mathrm{P} \sigma_{b}(\mathrm{~A})=\mathrm{s}(\mathrm{A})+i \alpha Z$.\\
(c) We consider $\Omega=\left\{(x, y) \in \mathbb{R}^{2}: 1<\left(x^{2}+y^{2}\right)^{1 / 2}<2\right\}$, and the second order differential operator $L_{0}$ given by $\left(L_{0} f\right)(x, y)=1 /\left(x^{2}+y^{2}\right) \cdot\left(x^{2} f_{x x}+2 x y f_{x y}+y^{2} f_{y y}\right)+\left(x f_{y}-y f_{x}\right) \cdot$ The assertion concerning the boundary spectrum can be verified easily by using polar coordinates: $x=r \cdot \cos \omega, y=r \cdot \sin \omega$. Then $L_{0}$ becomes $L_{0} f=f_{x r}+f_{w}$ on the space $C_{0}(1,2) \otimes C_{2 \pi}(\mathbb{R})$.

In this section we have seen that the eigenvalues in the boundary spectrum of an irreducible semigroup form a subgroup of $i \mathbb{R}$ (provided that $s(A)=0$ ). We conclude this section mentioning an analogous statement for the whole boundary spectrum of Markov semigroups on $C(K)$, K compact. It seems to be unknown if this is true for irreducible semigroups in general. To prove this result one uses the proof of the analogous result for a single operator (cf. Schaefer (1968), Thm.7) as a guideline.

Theorem 3.11. Suppose that $T$ is an irreducible semigroup of Markov operators on $C(K)$, $K$ compact. Then ${ }^{\sigma}{ }_{b}(A)$ is a subgroup of $i R$. Hence either $\sigma_{b}(A)=\{0\}$ or $=i \mathbb{R}$ or $=i a \mathbb{Z}$ for some $\alpha>0$.

\section*{4. SEMIGROUPS OF LATIICE HOMOMORPHISMS}
As we have seen in Section 2 the boundary spectrum of a many positive semigroups is a cyclic set. However, there are hardly any restrictions on the set $\{\lambda \in \sigma(A): \operatorname{Re} \lambda<s(A)\}$, except that it is symmetric with respect to the real axis. For semigroups of lattice homomorphisms the situation is quite different. We will show that the whole spectrum is an imaginary additively cyclic subset of $\mathbb{C}$ (see Def.2.5). A complete proof of this results requires some facts of the theory of Banach lattices, therefore, we postpone it to Part C (see C-III, Thm. 4.2).

Theorem 4.1. If $A$ is the generator of a semigroup of lattice homomorphisms, then $\sigma(A), A \sigma(A)$ and $P \sigma(A)$ are cyclic subsets of $\mathbf{C}$.\\
$1^{\text {st }}$ part of the proof. We prove the assertion concerning $A \sigma(A)$ and $\mathrm{P} \sigma(\mathrm{A})$. Assume that $\mathrm{Ah}=(\alpha+i \beta) \mathrm{h}, \alpha, \beta \in \mathbb{R}, \mathrm{h} \neq 0$, then $T(t) h=e^{\alpha t} e^{i \beta t} h$ for all $t \geqq 0$ (A-III, Cor. 6.4 ). Since $T(t)$ is a lattice homomorphism we have $T(t)|h|=|T(t) h|=e^{\alpha t}|h| \quad(t \geqq 0)$ or $A|h|=a|h|$, hence $A h^{[n]}=(\alpha+\operatorname{in} B) h^{[n]}$ for all $n \in Z$ by Thm.2.4(b) . We have shown that Po(A) is cyclic.\\
To prove that $A \sigma(A)$ is cyclic as well, one considers a semigroup $F$-product $E_{F}^{T}$ of $E$ (see $A-I I I, 4.5$ ). It is easy to see that $E_{F}^{T}$ is a Banach lattice and $\left(T_{F}(t)\right.$ is a semigroup of lattice homomorphisms. The proposition in $A-I I I, 3.5$ implies $A \sigma(A)=\operatorname{Po}\left(A_{F}\right)$. Thus the assertion follows from the cyclicity of point spectrum.

Performing a similar construction as in Ex.2.6(f) one can show that every closed cyclic subset of $\mathbb{C}$ which is contained in a left halfplane is the spectrum of a suitable semigroup of lattice homomorphisms. For details see Derndinger-Nagel (1979).\\
In the following we restrict ourselves to the case of compact spaces. Then a semigroup of lattice homomorphisms can be described explicitly by a semi-flow $\phi$ and real-valued functions $h$ and $p$ (see B-II, Thms.3.5 \& 3.6). The function $p$ has no influence on spectral properties (cf. B-II,(3.7)). Therefore we will assume that (T)(t)) has the following form (cf. B-II, Thm.3.5):\\
(4.1) $T(t) f=h_{t} \cdot f \circ \phi_{t}$ ( $t \geqq 0, f \in C(K)$ ) where $\phi=\left(\phi_{t}\right): \mathbb{R}_{+} \times \mathrm{K} \rightarrow \mathrm{K}$ is a continuous semiflow and $h_{t}(x)=\exp \int_{0}^{t} h(\phi(s, x)) d s \quad(t \geqq 0, x \in K)$ for some continuous function $h: K \rightarrow \mathbb{R}$.

In the following we will describe the spectrum of the semigroup given by (4.1) in terms of $\phi$ and $h$. At first we have to fix some notation. Let $\mathrm{K}, \phi, \mathrm{h}$ be as in (4.1).\\
(4.2) $\quad K_{t}:=\phi_{t}(K) \quad(t<\infty), K_{\infty}:=n_{t<\infty} K_{t}$.

Some properties of the sets $K_{t}$ are listed in the following lemma. The proof is not very difficult and is left as an exercise.

Lemma 4.2. Every $K_{t}(0 \leqq t \leqq \infty)$ is a non-empty closed subset of $k$ which is invariant under the semiflow $\phi$. Moreover, the following assertions are true:\\
(a) For $s>t$ we have $k_{s} \subset k_{t}$. In case that $k_{s}=k_{t}$ then

$$
\mathrm{K}_{t}=\mathrm{K}_{\infty} \text {. }
$$

(b) $\phi_{t}\left(K_{\infty}\right)=K_{\infty}$ for all $t \geqq 0$.\\
(c) If one partial mapping $\phi_{t}, t>0$, is injective (surjective), then all mappings $\phi_{s}$ are injective (surjective).

We call a semiflow $\phi$ injective (surjective) if one and hence all of the partial mappings $\phi_{t}$ are injective (surjective). Studying the spectrum of the semigroup given by (4.1) we divide the complex plane into three sets:

\begin{verbatim}
(4.3) {\lambda \in\mathbb{C}:\operatorname{Re}\lambda<\underline{C}(h,\phi)}
    {\lambda\in\mathbb{C}:\underline{C}(h,\phi)\leqq\operatorname{Re \lambda}\leqq\overline{C}(h,\phi)}
    {\lambda\in\mathbb{C}:\overline{C}(h,\phi)<\operatorname{Re \lambda}}.
\end{verbatim}

The quantities $\underline{c}(h, \phi)$ and $\bar{c}(h, \phi)$ are defined as follows:\\
(4.4) $\bar{c}(h, \phi):=\lim _{t \rightarrow \infty} \bar{c}_{t}(h, \phi)=\inf _{t>0} \bar{c}_{t}(h, \phi)$ where

$$
\bar{c}_{t}(h, \phi):=\sup _{x \in K}^{\tau \rightarrow \infty}\left\{1 / t \cdot \int_{0}^{t} h(\phi(s, x)) d s\right\} \quad(t>0)
$$

$$
\begin{aligned}
\underline{c}(h, \phi) & :=\lim _{t \rightarrow \infty} \frac{c}{t}(h, \phi)=\sup _{t>0} \underline{c}_{t}(h, \phi) \text { where } \\
\underline{c}_{t}(h, \phi) & :=\inf _{x \in K}\left\{1 / t \cdot \int_{0}^{t} h(\phi(s, x)) d_{s}\right\}(t>0) .
\end{aligned}
$$

It is easy to see that $\bar{c}_{t}(h, \phi)=1 / t \cdot \log \|T(t)\|$, hence in the definition of $\bar{c}(h, \phi)$, both the limit and the infimum exist and coincide with the growth bound (see A-I,(1.1)). Furthermore, $\underline{c}_{t}(h, \phi)=-\bar{c}_{t}(-h, \phi)$. Therefore, $\underline{c}(h, \phi)$ is well defined too.

First we will describe the part of $\sigma(\mathrm{A})$ which is contained in the left half-plane determined by $\underline{(h, \phi)}$. It turns out that either the whole half-plane is contained in $\sigma(\mathrm{A})$ or it has empty intersection with $\sigma(A)$. This depends only on properties of $\phi$. Essentially there\\
are three different cases. Before we state the general result (Thm.4.4) we give some typical examples.

Examples 4.3.(a) Consider on $K=[0, \infty]$ the semiflow defined by $\phi(t, x):=x+t \quad(\infty+t=\infty)$. Then we have $k_{t}=[t, \infty]$ and $K_{\infty}=\{\infty\}$. The spectrum of the corresponding semigroup $T(t) f=f \circ \phi_{t}$ is given by $\sigma(A)=\operatorname{A} \sigma(A)=\{\lambda \in \mathbb{C}: \operatorname{Re} \lambda \leqq 0\}$.\\
(b) Consider again $\mathrm{K}=[0, \infty]$ and define a semiflow by\\
$\phi(t, x):=\left\{\begin{array}{cl}x-t & \text { if } x \geqq t \\ 0 & \text { if } x<t\end{array} \quad(\infty-t=\infty)\right.$.\\
Then we have $K_{t}=K$ for all $t$, hence $K_{\infty}=K$ and $\sigma(A)=\{\lambda \in \mathbb{C}: \operatorname{Re} \lambda \leqq 0\}, \operatorname{Ro}(A)=\{\lambda \in \mathbb{C}: \operatorname{Re} \lambda<0\} U\{0\}$.\\
(c) Consider on $\mathrm{K}_{1}:=[-1, \infty)$ the equivalence relation defined by " $x \sim y$ if ond only if $x, y \geqq 0$ and $x-y \in \mathbf{z}$ " The semiflow $\phi_{1}$ on $K_{1}$ given by $\phi_{1}(t, x)=x+t$ induces a semiflow $\phi$ on the quotient space $\mathrm{K}:=\mathrm{K}_{1 / \sim}$. We have for $0<t<1$ : $\mathrm{K}_{\boldsymbol{I}} \mathrm{K}_{\mathrm{t}} \varsubsetneqq \mathrm{K}_{\infty}$ $\left.\left(K_{\infty}=[0,1]\right)_{\sim} \cong \Gamma\right)$. The spectrum of the corresponding semigroup on K is given by $\sigma(\mathrm{A})=2 \pi i \mathbb{Z}$.\\
(d) Consider on $\mathrm{K}=[-1,1]$ the flow $\phi$ given by $\phi(t, x):= \begin{cases}-1 & \text { if } x<0 \text { and } t>-\frac{x+1}{x} \\ \frac{x}{1+t x} & \text { otherwise. }\end{cases}$\\
Then we have $K_{t}=\left[-1, \frac{1}{1+t}\right], K_{\infty}=[-1,0]$ and\\
$\sigma(A)=\{\lambda \in \mathbb{C}: \operatorname{Re} \lambda \leqq 0\},\{\lambda \in \mathbb{C}: \operatorname{Re} \lambda<0\} \not \mp A \sigma(A) \cap \operatorname{R\sigma }(A)$.

Further examples related to ordinary differential equations on $\mathbb{R}^{n}$ will be given after we have stated and proved the general result:

Theorem 4.4. Suppose $T$ is a semigroup of lattice homomorphisms given by (4.1) with generator A . Considering $H:=\{\lambda \in \mathbb{C}$ : $\operatorname{Re} \lambda<\underline{\subseteq}(h, \phi)\}$, where $\subseteq(h, \phi)$ is given by $(4.4)$, we have:\\
(a) If $K_{t} \neq K_{\infty}$ for every $t<\infty$, then $H \subseteq A \sigma(A)$.\\
(b) If $\phi \mid K_{\infty}$ is not injective, then $\mathrm{H} \subseteq \operatorname{Ro}(\mathrm{A})$.\\
(c) If $K_{S}=K_{\infty}$ for some $s<\infty$ and ${ }^{\phi} \mid K_{\infty}$ is injective, then $H \cap \sigma(A)=\varnothing$.

Proof. For $\varepsilon>0$ we define $H_{\varepsilon}=\{\lambda \in \mathbb{C}: \operatorname{Re} \lambda<\underline{c}(h, \phi)-\varepsilon\}$. Obviously it is enough to prove assertion (a), (b) and (c) respectively for ${ }^{\mathrm{H}}{ }_{2 \varepsilon}$, E arbitrary, instead of H .\\
(a) By the definition given in (4.4) there exists a $\tau>0$ such that $\underline{c}_{t}(h, \phi) \geq c(h, \phi)-\varepsilon$ for all $t \geq \tau$. It follows that (4.5) $\quad h_{t}(x) \geqq e^{(\alpha+\varepsilon) t}$ whenever $t \geqq \tau, x \in K, \alpha<\subseteq(h, \phi)-2 \varepsilon$.

Now we fix $\lambda=\alpha+i \beta \in H_{2 E}(\alpha, \beta \in \mathbb{R})$ and construct an approximate eigenvector $\left(g_{n}\right)$ of $A$ corresponding to $\lambda$. For $n \leq \tau+1$ we choose an arbitrary function $g_{n} \neq 0$. Now suppose $n>\tau+1$. We choose $x_{n} \in K_{n+1 / 2} \backslash K_{n+1}$ (cf. Lemma 4.2(a)), then there exists $y_{n} \in K$ such that $\phi\left(n+1 / 2, y_{n}\right)=x_{n}$. We have\\
$\phi\left([0, n+1 / 2], y_{n}\right) \cap K_{n+1}=\emptyset$ and the mapping $t \rightarrow \phi\left(t, y_{n}\right)$ is a continuous injection, hence a homeomorphism from $[0, n+1 / 2]$ into $k$ (this is true because $\left.\phi\left(n+1 / 2, y_{n}\right) \notin K_{n+1}\right)$. By Tietze's Theorem there is $f_{n} \in C(K)$ such that\\
(4.6) $\left\|f_{n}\right\| \leq 1, f_{n \mid k_{n+1}}=0$,

$$
\begin{aligned}
& f_{n}\left(\phi\left(t, y_{n}\right)\right)=0 \text { for } 0 \leqq t \leqq n-(1+\delta) \text { and } n+\delta \leqq t \leqq n+1 \\
& f_{n}\left(\phi\left(t, y_{n}\right)\right)=e^{i \beta t} \text { for } n-1 \leqq t \leqq n .
\end{aligned}
$$

The constant $\delta \in(0,1 / 2)$ will be determined later.\\
Considering $g_{n}:=\int_{0}^{n+1} e^{-\lambda t} T(t) f_{n} d t$, then $g_{n} \in D(A)$ and


\begin{align*}
(\lambda-A) g_{n} & =\left(1-e^{-\lambda(n+1)} T(n+1)\right) f_{n}=  \tag{4.7}\\
& =f_{n}-e^{-\lambda(n+1)} \cdot h_{n+1} \cdot f_{n}^{\circ} \phi_{n+1}=f_{n}
\end{align*}


Moreover,\\
$\left\|g_{n}\right\| \geqq\left|g_{n}\left(y_{n}\right)\right|=\left|\int_{0}^{n+1} e^{-\lambda t} h_{t}\left(y_{n}\right) f_{n}\left(\phi\left(t, y_{n}\right)\right) d t\right| \geqq$\\
$\left|\int_{n-1}^{n} e^{-\lambda t} h_{t}\left(y_{n}\right) e^{i \beta t} d t\right|-\left[\int_{n-(1+\delta)}^{n-1}+\int_{n}^{n+\delta} \mid e^{-\lambda t_{h}}\left(y_{n}\right) f_{n}\left(\phi\left(t, y_{n}\right) \mid d t\right]\right.$\\
$\geqq \int_{n-1}^{n} e^{-\alpha t} e^{(\alpha+\varepsilon) t} d t-\left[\int_{n-(1+\delta)}^{n-1}+\int_{n}^{n+\delta} e^{-\alpha t}\left|h_{t}\left(y_{n}\right)\right| d t\right]$\\
$=1 / \varepsilon \cdot\left(e^{\varepsilon n}-e^{\varepsilon(n-1)}\right)-\left[\int_{n-1}^{n-1}(1+\delta)+\int_{n}^{n+\delta} e^{-a t}\left|h_{t}\left(y_{n}\right)\right| d t\right]$.\\
The constant $\delta$ can be chosen such that


\begin{equation*}
\left\|g_{n}\right\| \geq 1 / 2 \varepsilon \cdot\left(e^{\varepsilon n}-e^{\varepsilon(n-1)}\right) \rightarrow \infty \text { for } n \rightarrow \infty \tag{4.8}
\end{equation*}


It follows from (4.8) and (4.7) that $g_{n} /\left\|g_{n}\right\|$ is an approximate eigenvector of A corresponding to $\lambda$. Thus (a) is proved. The proofs of (b) and (c) will be handled simultaneously. First we show that we can restrict attention to the case where $\mathrm{K}=\mathrm{K}_{\infty}$. Indeed, $K_{\infty}$ is a $\phi$-invariant subset, hence $I_{\infty}:=\left\{f \in C(K):{ }^{f} / K_{\infty}=0\right\}$ is a T-invariant ideal. Identifying $\mathrm{C}(\mathrm{K}) / \mathrm{I}_{\infty}$ with $\mathrm{C}\left(\mathrm{K}_{\infty}\right)$ (cf. $\mathrm{B}-\mathrm{I}$, sec.1), then ( $\mathrm{T}(\mathrm{t}) / I_{\infty}$ ) is the semigroup governed by $\phi / \mathrm{K}_{\infty}$ and ${ }^{h} \mid K_{\infty}$. Since one always has $R_{\sigma}\left(A_{\rho}\right) \subseteq R_{\sigma}(A)$, assertion $(b)$ is proved\\
when we can show that $\mathrm{H}_{2 E} \subseteq \operatorname{Ro}\left(\mathrm{~A}\right.$, ). In case (c) one has $\mathrm{K}_{\mathrm{s}}=\mathrm{K}_{\infty}$ for some $s<\infty$, which implies $T(s) \mid I_{\infty}=0$. Hence we have $\sigma\left({ }^{A} \mid I_{\infty}\right)=\varnothing$ and therefore $\sigma(A)=\sigma\left(A / I_{\infty}\right)$ by $A-I I I, \operatorname{Prop} .4 .2$. Henceforth we will assume that $K=K_{\infty}$, that is, is surjective (cf. Lemma 4.2(b)).\\
We choose $\tau>0$ such that (4.5) is true. Since $\phi$ is surjective, for every $f \in C(K)$ there is a $x_{f} \in K$ such that $\|f\|=\left\|f\left(\phi\left(\tau, x_{f}\right)\right)\right\|$ and we obtain for $\lambda \in \mathrm{H}_{2 \varepsilon}, \lambda=\alpha+i \beta, \alpha, \beta \in \mathbb{R}$ :\\
(4.9) $\left\|\left(e^{\lambda \tau}-T(\tau)\right) f\right\| \geqq\left|h_{\tau}\left(x_{f}\right) f\left(\phi\left(\tau, x_{f}\right)\right)-e^{\lambda \tau} f\left(x_{f}\right)\right|$

$$
\geq h_{\tau}\left(x_{f}\right)\|f\|-e^{\alpha \tau}\left|f\left(x_{f}\right)\right|
$$

$$
\begin{aligned}
& \geqslant e^{(\alpha+\epsilon) \tau}\|f\|-e^{\alpha \tau}\|f\| \\
& =e^{\alpha \tau}\left(e^{\varepsilon \tau}-1\right)\|f\| .
\end{aligned}
$$

It follows that the disc $D:=\{\lambda \in \mathbb{C}:|\lambda|<\exp (\mathrm{c}(\mathrm{h}, \phi)-2 \varepsilon)\}$ has an empty intersection with $A \sigma(T(\tau))$ and therefore $H_{2 \varepsilon} \cap A \sigma(A)=\varnothing$ by A-III, 6.2 . Since every boundary point of the spectrum is an approximate eigenvalue (A-III, Prop.2.2(i)) we have the following alternative:\\
(4.10) Either $D \subseteq \rho(T(\tau))$ and $\mathrm{H}_{2 \varepsilon} \subseteq \rho(A)$ or else $\mathrm{D} \subseteq \operatorname{Ro}(\mathrm{T}(\tau))$ and $\mathrm{H}_{2 \varepsilon} \subseteq \operatorname{Ro}(\mathrm{~A})$.

It is not difficult to see that $0 \in \rho(T(\tau))$ whenever $\phi_{t}$ is bijective and that 0 is an eigenvalue of $T(\tau)$ if $\phi_{\tau}$ is not injective. Since we assumed that that $\phi$ is surjective, assertions (b) and (c) of the theorem are immediate consequences of (4.10).

The examples $4.3(a)$, (b) and (c) respectively are prototypes of the three different cases considered in Thm.4.4 . Ex.4.3(c) also shows that there are semigroups whose spectrum is contained in a right half-plane, although they cannot be embedded in a group (compare Cor.4.5 below!). Ex.4.3(d) shows that (a) and (b) do not exclude each other.

Corollary 4.5. If $\phi$ is injective or surjective, then the following assertions are equivalent:\\
(i) A is the generator of a strongly continuous group.\\
(ii) $\sigma(\mathrm{A})$ is contained in a right half-plane.

Proof. (i) $\rightarrow$ (ii) holds true because -A is a generator of a semigroup. (ii) $\rightarrow$ (i): We have to show that one (hence each) operator $T(t)$, $t \geqq 0$ is invertible. Obviously this is true if $\phi$ is bijective. At first we assume that $\phi$ is surjective, that is, $\mathrm{K}=\mathrm{K}_{\infty}$. By Thm. 4.4 we have that ${ }^{\phi} \mid K_{\infty}$ is injective if (ii) is true. Thus $\phi$ is bijective. Now we assume that $\phi$ is injective. We have to show that $\mathrm{K}=\mathrm{K}_{\infty}$. By Thm.4.4 we have $\mathrm{K}_{\infty}=\mathrm{K}_{\mathrm{s}}$ for some s , whenever (ii) is true. Given $\mathrm{x} \in \mathrm{K}$ then by Lemma $4.2(b)$ there exists $\mathrm{y} \in \mathrm{K}_{\infty}$ such that $\phi(s, x)=\phi(s, y)$. If $\phi$ is injective we have $x=y \in K_{\infty}$.

In the following example we consider semiflows related to ordinary differential equations on $\mathbb{R}^{n}$. In case there exists a corresponding global flow, it induces a group on $C_{0}\left(\mathbb{R}^{n}\right)$ in a canonical way . Even if there is no global flow, one can construct semigroups governed by a semiflow, and apply Thm.4.4(a) in order to describe the spectrum. These examples can be easily extended to differential equations on manifolds (see Sec.18.2 of Dieudonné (1971)).

Example 4.6. Suppose $F: \mathbb{R}^{n} \rightarrow \mathbb{R}^{n}$ is continuously differentiable. We denote the maximal flow corresponding to the differential equation $y^{\prime}=F(y)$ by $\phi_{0}$. In general, $\phi_{0}$ is only defined on an open subset of $\mathbb{R} \times \mathbb{R}^{n}$ which contains $\{0\} \times \mathbb{R}^{\mathbf{n}}$. For $\mathbf{x} \in \mathbb{R}^{\mathbf{n}}$ there exist $t_{x}$ and $\bar{t}_{x}$ such that\\
(4.11) $-\infty \leqq \underline{t}_{\mathrm{x}}<0<\bar{t}_{\mathrm{x}} \leqq \infty$;

$$
\begin{aligned}
& \phi_{0}(t, x) \text { is defined if } \underline{t}_{x}<t<\bar{t}_{x} ; \\
& \text { if } \bar{t}_{x}<\infty \quad\left(\underline{t}_{x}>-\infty\right) \text { then }\left|\phi_{0}(t, x)\right| \rightarrow \infty \text { as } t \uparrow \bar{t}_{x}\left(t+\underline{t}_{x}\right) .
\end{aligned}
$$

For details see Sect.18.2 of Dieudonné (1971)\\
(a) If $\phi_{0}$ is a global flow, i.e., if $\phi_{0}$ is defined on $\mathbb{R} \times \mathbb{R}^{n}$, then one has a corresponding (semi-)group on $C_{0}\left(\mathbb{R}^{n}\right)$. If $F$ is differentiable, its generator is the closure of $\mathrm{A}_{1}$ which is defined as follows (cf. B-II,Ex.3.15):\\
(4.12) $\quad \mathrm{A}_{1} \mathrm{f}=(\mathrm{F} \mid$ grad f$):=\sum \mathrm{F}_{i} \cdot \partial_{i} \mathrm{f}$ with domain\\
$D\left(A_{1}\right):=\left\{f \in C^{1}\right.$ : supp $f$ is compact $\}$.\\
$\phi_{0}$ can be uniquely extended to a flow $\tilde{\phi}_{O}$ on $\mathbb{R}^{\mathrm{n}} U\{\infty\}$ by defining $\tilde{\phi}_{O}(t, \infty):=\infty$ for all $t \in \mathbb{R} \cdot \phi_{O}$ and $\tilde{\phi}_{O}$ satisfy condition (c) of Thm. 4. 4 .

A global flow exists for example if $F$ is globally Lipschitz continuous or if $F$ is uniformly bounded. In case $\left\{x \in R^{n}:(x \mid F(x))>0\right\}$ is bounded in $\mathbb{R}^{\text {n }}$ a global semiflow always exists (see [Deimling (1977), Sec.5.21).\\
(b) We do not assume that $\phi$ is globally defined. Instead we consider a bounded domain $\Omega \in \mathbb{R}^{\mathrm{n}^{\circ}}$ with smooth boundary $\partial \Omega$ such that $(F(x) \mid v(x))>0$ for every $x \in \partial \Omega$. Here $v(x)$ denotes the outward normal vector.

Then for $x \in \vec{\Omega}$ we have $t_{x}=-\infty$. Moreover, either $\phi_{0}(t, x) \in \Omega$ for all $t \geqq 0$ or else there exists a unique $s_{x}$ with $0 \leqq s_{x}<E_{x}$ such that $\phi_{0}\left(s_{x}, x\right) \in \partial \Omega$. In the first case we write $s_{x}:=\infty$. Then we define $\phi: \mathbb{R}_{+} \times \bar{\Omega} \rightarrow \bar{\Omega}$ as follows:\\
$\phi(t, x):=\left\{\begin{array}{llr}\phi_{0}(t, x) & \text { if } 0 \leqq t<s_{x} \\ \phi_{0}\left(s_{x}, x\right) & \text { if } & t \geqq s_{x}\end{array}\right.$\\
$\phi$ is a continuous semiflow on the compact set $K:=\bar{\Omega}$. We have $\mathrm{K}_{\infty}=\mathrm{K}$ and $\phi / \mathrm{K}_{\infty}$ is not injective.\\
In case $F$ is differentiable, the generator of the corresponding semigroup is the closure of the operator $A_{2}$ defined by\\
$A_{2} f:=(F \mid g r a d f), D\left(A_{2}\right):=\left\{f \in C^{1}(\bar{\Omega}):(F \mid g r a d f)=0\right.$ on $\left.\partial \Omega\right\}$.\\
(c) We consider $\Omega$ as in (b) and assume that $(F(x) \mid v(x)) \leqq 0$ for every $x \in \partial \Omega$. Then for every $x \in \bar{\Omega}$ we have $\bar{E}_{x}=\infty$. Thus $\phi:={ }^{\phi} \mid \mathbb{R}_{+} \times \bar{\Omega}$ is a continuous semiflow on $K:=\bar{\Omega}$.\\
If $(F(x) \mid \nu(x))<0$ for some $x \in \partial \Omega$ we have $K_{t} \varsubsetneqq K_{s}$ whenever $t>s$ and ${ }^{\phi} \mid \mathrm{K}_{\infty}$ is injective. For a differentiable vector field $F$ the generator of the corresponding semigroup is the closure of $\mathrm{A}_{3}$ defined as follows: $A_{3} f:=(F \mid g r a d f), D\left(A_{3}\right):=C^{l}(\bar{\Omega})$.

We conclude the discussion of semi-flows associated with ordinary differential equations by remarking that the ideas of (b) and (c) can be combined to obtain semigroups for more general subsets $\Omega$.

We continue the discussion of the spectrum of semigroups of lattice homomorphisms on $C(K)$ given by (4.1). Thm. 4.4 gives a good description of the part which is contained in $\{\lambda \in \mathbb{C}: \operatorname{Re} \lambda<\underline{c}(h, \phi)\}$. It is easy to see that the half-plane $\{\lambda \in \mathbb{C}: \operatorname{Re} \lambda>\bar{c}(h, \phi)\}$ is always a subset of the resolvent set (see prop.4.8(a) below). The description of the remainig part $\{\lambda \in \sigma(A): c(h, \phi) \leqq \operatorname{Re} \lambda \leqq \bar{c}(h, \phi)\}$ is more difficult. First we discuss some examples and then give a partial answer to this problem (see Prop.4.8(b)-(e)).

Example 4.7.(a) Consider the flow on $\left[-\frac{\pi}{2}, \frac{\pi}{2}\right]$ defined by $\phi(t, x):=\arctan (\tan x-t), x \in\left[-\frac{\pi}{2}, \frac{\pi}{2}\right], t \in \mathbb{R}$\\
(it belongs to the differential equation $y^{\prime}=-\cos ^{2} y$ ), and a continuous function $h:\left[-\frac{\pi}{2}, \frac{\pi}{2}\right] \rightarrow \mathbb{R}$ with $h\left(-\frac{\pi}{2}\right) \leqq h\left(\frac{\pi}{2}\right)$. Then we have $\underline{c}(h, \phi)=h\left(-\frac{\pi}{2}\right)$ and $\bar{c}(h, \phi)=h\left(\frac{\pi}{2}\right)$. The spectrum of the corresponding semigroup is given by $\sigma(A)=\left\{\lambda \in \mathbb{C}: h\left(-\frac{\pi}{2}\right) \leqq \operatorname{Re} \lambda \leqq \mathrm{h}\left(\frac{\pi}{2}\right)\right\}$.\\
(b) Consider $K=\{z \in \mathbb{C}: 1 \leqq|z| \leqq 2\}=\left\{r \cdot e^{i \omega}: \omega \in \mathbb{R}, 1 \leqq r \leqq 2\right\}$ and a continuous function $k:[1,2] \rightarrow \mathbb{R}_{+}$.\\
Let $\bar{\phi}$ be the flow on K governed by the differential equation $\dot{\omega}=k(r), \dot{r}=0 \quad$ (hence $\bar{\phi}\left(t, r \cdot e^{i \omega}\right)=r \cdot e^{i(\omega+k(r) t)}$ ).\\
For a continuous function $h: K \rightarrow \mathbb{R}$ let $h^{\wedge}(r):=\frac{1}{2 \pi} \cdot \int_{0}^{2 \pi} h\left(r^{\prime} e^{i t}\right) d t$ $(1 \leqq r \leqq 2)$. The spectrum of the semigroup corresponding to $\phi$ and h (cf. (4.1)) is given by\\
$\sigma(A)=\left\{h^{\wedge}(r)+i k_{k}(r): k \in \mathbf{Z}, 1 \leqq r \leqq 2\right\}^{-} U\{h(z): k(|z|)=0\}$.

Proposition 4.8. Suppose the semigroup $(T(t))_{t \geq 0}$ on $C(K)$ is given by $(4.1)$ and let $\underline{c}(h, \phi), \bar{c}(h, \phi)$ be defined as in (4.4). Then the following assertions hold:\\
(a) $\{\lambda \in \mathbb{C}: \operatorname{Re} \lambda>\bar{c}(h, \phi)\} \subset \rho(A) ;$\\
(b) $\bar{c}(h, \phi)$ and $c(h, \phi)$ are spectral values;\\
(c) If $\phi\left(t, x_{0}\right)=x_{0}$ for every $t \geqq 0$, then $h\left(x_{0}\right) \in \operatorname{Ro}(A)$;\\
(d) Assume $x_{0}$ has a finite orbit (i.e., $\phi\left(\mathbb{R}_{+}, x_{0}\right)=\phi\left([0, T], x_{0}\right)$ for some $T<\infty)$ and $\tau:=\inf \left\{t>0: \phi\left(T+t, x_{0}\right)=\phi\left(T, x_{0}\right)\right\}>0$, then $h^{\wedge}\left(x_{0}\right)+\frac{2 \pi}{\tau} i \mathbb{Z} \subset \operatorname{Ro}(A)$ where $h^{\wedge}\left(x_{0}\right):=1 / \tau \int_{T}^{T+\tau} h\left(\phi\left(s, x_{0}\right)\right) d s$.\\
(e) If $x_{0}$ has an infinite orbit and $h^{\wedge}:=\lim _{t \rightarrow \infty} h\left(\phi\left(t, x_{0}\right)\right.$ ) exists, then $h^{\wedge}+i \mathbb{R} \sigma(\mathrm{~A})$.

Proof. (a) and (b): A look at (4.4) shows that $\bar{c}_{t}(h, \phi)=$ $1 / t \cdot \log \|\mathrm{~T}(t)\|$ hence $\bar{c}(h, \phi)=\omega(A) \quad(c f . A-I,(1,1))$. Consequently, we have $\{\lambda \in \mathbb{C}: \operatorname{Re} \lambda>\bar{c}(h, \phi)\} \subset \rho(A)$ and $\bar{c}(h, \phi) \in \sigma(A)$ by Thm.1.6. To prove $\underline{c}(h, \phi) \epsilon \sigma(A)$, we can assume by Thm.4.4 that $K_{\infty}=K_{s}$ for some $s$ and that $\phi \mid K_{\infty}$ is injective. It is easy to see that $\subseteq(h, \phi)=c\left({ }^{h}\left|K_{\infty}, \phi\right| K_{\infty}\right)$, moreover, we have $\sigma\left({ }^{A} \mid I_{\infty}\right)=\emptyset$ hence $\sigma(A)=$ $\sigma\left(A / I_{\infty}\right)$ by A-III, Prop.4.2. This shows that we also can assume that $K=K_{\infty}$, i.e., $\phi$ is bijective or $A$ is the generator of a group. Now the assertion follows from\\
$c(h, \phi)=c\left(h, \phi^{-1}\right)=-\bar{c}\left(-h, \phi^{-1}\right)=-s(-A)$.\\
(c) and (d): One can check easily that in case of (c) the Dirac functional ${ }^{\delta} x_{0}$ is an eigenvector of $A^{\prime}$ corresponding to $h\left(x_{0}\right)$.

A little bit more calculation is necessary to check that in case of\\
(d) the functional $\Psi_{k}$ defined by\\
$\Psi_{k}(f):=\int_{T}^{T+T} \exp \left(-i \cdot \frac{2 \pi k}{T} \cdot t\right) \cdot h_{t}\left(x_{o}\right) \cdot f\left(\phi\left(t, x_{o}\right)\right) d t \quad(k \in \mathbf{Z}, f \in C(K))$ is an eigenvector of $A^{\prime}$ corresponding to $h^{\wedge}\left(x_{0}\right)+i \cdot \frac{2 \pi k}{\tau}$.\\
(e) Given $\beta \in \mathbb{R}$ we will show that $h^{\wedge}+i \beta \in \mathbb{A \sigma}\left(A^{\prime}\right) \subseteq \sigma(A)$. For $\mathrm{n}, \mathrm{m} \in \mathbb{N}$ we define a linear functional $\psi_{\mathrm{nm}}$ as follows:\\
${ }^{\Psi} n m(f):=\frac{1}{n} \cdot \int_{0}^{n} \exp \left(-\left(h^{\wedge}+i \beta\right) t\right) \cdot h_{t}\left(\phi\left(m, x_{0}\right)\right) \cdot f\left(\phi\left(m+t, x_{0}\right)\right) d t, f \in C(K)$. For $f \in D(A)$ we have\\
$<\left(h^{\wedge}+i \beta-A\right) f, \psi{ }_{n m}>=$\\
$=\frac{1}{n} \cdot\left(f\left(\phi\left(m, x_{0}\right)\right)-\exp (-j \beta n) \exp \left(\int_{m}^{m+n}\left(h\left(\phi\left(s, x_{0}\right)\right)-h^{\wedge}\right) d s\right) f\left(\phi\left(m+n, x_{0}\right)\right)\right)$.\\
It follows that $\phi_{n m} \in D\left(A^{\prime}\right)$ and, since $\left.\lim _{t \rightarrow \infty} h\left(\phi, x_{0}\right)\right)=h^{\wedge}$,\\
(4.13) $\quad \lim \sup _{m \rightarrow \infty}\left\|\left(h^{\wedge}+i \beta-A^{\prime}\right) \Psi_{n m}\right\| \leqq 1 / n$ for every $n \in \mathbb{N}$.

Because the orbit is infinite we have

$$
\begin{aligned}
\| \Psi n d & =\frac{1}{n} \cdot \int_{0}^{n}\left|e^{-(h+i \beta) t} h_{t}\left(\phi\left(m, x_{0}\right)\right)\right| d t= \\
& =\frac{1}{n} \cdot \int_{0}^{n} \exp \left(\int_{m}^{m+t}\left(h\left(\phi\left(s, x_{0}\right)\right)-h^{n}\right) d s\right.
\end{aligned}
$$

which shows that\\
(4.14) $\quad \lim _{m \rightarrow \infty}\left\|_{n m}\right\|=1$ for every $n \in \mathbb{N}$.

In view of (4.13) and (4.14) it is not difficult to find a subsequence $k(n)$ of $\mathbb{N}$ such that $\left(\Psi_{n, k(n)}\right)$ is an approximate eigenvector of $A^{\text {: }}$ corresponding to $h^{\wedge}+i \beta$.

We are now going to apply the results obtained so far to the special case where $h=0$, i.e., we consider semigroups of lattice homomorphisms which are Markov operators.

Theorem 4.9. Suppose $T$ is a semigroup of Markov lattice homomorphisms on $C(K)$ governed by the semiflow $\phi$.\\
(a) If $\phi \mid k_{\infty}$ is not injective or if $k_{t} \neq k_{\infty}$ for every $t<\infty$, then $\sigma(A)=\{\lambda \in \mathbb{C}: \operatorname{Re} \lambda \leqq 0\}$.\\
(b) If $K_{\infty}=K_{S}$ for some $s$ and $\phi \mid K_{\infty}$ is injective, then $\sigma(A)$ is a cyclic closed subset of iR . Moreover, we have $\sigma(A) \neq i R$ if and only if there is a $T<\infty$ such that every orbit of $\phi$ has length less than $T$ (i.e., $\phi\left(\mathbb{R}_{+}, x\right)=\phi([0, T], x)$ for every $\left.x \in K\right)$.

Proof.(a) This is an immediate consequence of Thm.4.4 and Prop.4.8. (b) The first assertion follows form Thm. 4.4 and Thm. 4.1 . Moreover, as in the proof of Thm. $4.4(b)$ and (c) we can assume without loss of generality that $K=K_{\infty}$, hence $\phi$ is bijective. If there is no upper bound for the length of the orbits, then $\sigma(A)=i \mathbb{R}$ by assertions (d) and (e) of Prop.4.8\\
Now we assume that the lengths of the orbits are bounded by $T$. Because $\phi$ is bijective, for every $x \in K$ there exists a $r=r_{x}$ with $T / 2 \leqq r \leqq T$ such that\\
$\phi(t, x)=\phi(t+r, x)=\phi(t+2 r, x)=\ldots=\phi(t+k r, x) \quad\left(t \in \mathbb{R}_{+}, k \in \mathbb{N}\right)$. Therefore we have for $\lambda \in \mathbb{C}$, $\operatorname{Re} \lambda>0, f \in C(K), x \in K$ :


\begin{align*}
& (R(\lambda, A) f)(x)=\int_{0}^{\infty} e^{-\lambda t} f(\phi(t, x)) d t=  \tag{4.15}\\
& =\sum_{k=0}^{\infty} \exp (-\lambda k r) \cdot \int_{k r}^{(k+1) r} \exp (-\lambda(t-k r) \cdot f(\phi(t-k r, x)) d t \\
& =\left(1-e^{-\lambda r}\right)^{-1} \cdot \int_{0}^{r} \exp (-\lambda t) f(\phi(t, x)) d t .
\end{align*}


If $0<\beta<2 \pi / T$, then the assumption $T / 2 \leqq r \leqq T$ implies that there exists a neighborhood $U$ of $\lambda_{0}:=i \beta$ such that the functions $\lambda \rightarrow\left(1-\exp \left(-\lambda r_{x}\right)\right)^{-1}$ are uniformly bounded on $U$, by $M$ say. Then (4.15) implies that $\|R(\lambda, A) f\| \leq M\left(\int_{0}^{r}\left|e^{-\lambda t}\right| d t\right)\|f\|$ for $\lambda \in U$, $\operatorname{Re} \lambda>0$, therefore $\lambda_{0}=i \beta \in \rho(A)$.\\
\includegraphics[max width=\textwidth]{2024_12_23_c6487cc0859199a15bd9g-211} only finite orbits. Therefore every point $x \in K_{\infty}$ has a well-defined period ${ }^{\tau} \mathrm{x}:=\inf \{\tau>0: \phi(\tau, x)=x\}$. A more detailed analysis yields the following description of $\sigma(A)$ :


\begin{equation*}
\sigma(A)=\left\{i \cdot 2 \pi k / \tau_{x}: k \in \mathbf{Z}, x \in K_{\infty}, \tau_{x}>0\right\} \cup\{0\} \tag{4.16}
\end{equation*}


The inclusion "c" can be derived from Thm.4.11 which is stated below. The reverse inclusion follows from prop.4.8(d) .

In our detailed discussion of the spectrum of lattice homomorphisms we restricted ourselves to the case where the space K is compact. The main reason is that there is no description as given in (4.1) of the semigroups for locally compact spaces X . In general, it is difficult to define a semiflow on x because points may tend to infinity in a finite time. But even if one can find a flow on a suitable compactification of $C$, it may be impossible to find a multiplicator. This can be seen by studying the following example:\\
Suppose $\phi_{1}$ is a semiflow on a compact space $K_{1}$ and $K_{0}$ is a closed $\phi_{1}$-invariant subset, $h$ a continuous function on $K_{1}$. The\\
semigroup $\left(\mathrm{T}_{1}(t)\right)$ on $C\left(K_{1}\right)$ corresponding to $\phi_{1}$ and $h$ leaves the ideal $I:=\left\{f \in C\left(\mathrm{~K}_{1}\right): \mathrm{f} \mid \mathrm{K}_{0}=0\right\}$ invariant and induces via restriction a semigroup $(T(t))$ on $I=C_{0}(X)$, where $X=K_{1} \_{0}$. In this case one can construct semi-flows associated with (T) (t)) on $\mathrm{XU}\{\infty)$ or on $\overline{\mathrm{X}}$ (closure in $\mathrm{K}_{1}$ ), but in general one cannot find a corresponding multiplicator which is defined on one of these compactifications.\\
The situation is much nicer when groups of lattice homomorphisms instead of semigroups are considered. In this case there is an analogue of (4.1) (cf. B-II,Thm.3.14) and the spectrum can be described completely. For more details and the proof of the following result we refer to Arendt-Greiner (1984).

Theorem 4.11. Suppose $X$ is a locally compact space and ( $T(t)$ ) $t \in \mathbb{R}$ is a group of lattice homomorphisms governed by the flow $\phi$ and the multiplicator $h$. Then we have:\\
(a)

$$
\begin{aligned}
& \sigma(A)=\sigma_{1} \cup \sigma_{2} \cup \sigma_{3} \quad \text { where the sets } \sigma_{i} \text { are defined as follows } \\
& \sigma_{1}:=\left\{h^{\wedge}(x)+i \cdot 2 \pi k / \tau_{x}: x \in x, 0<{ }^{\tau}{ }^{\tau} x<\infty\right\}, \\
& \sigma_{2}:=\left\{h(x): x \in x,{ }_{x}{ }^{\tau}=0\right\}, \sigma_{3}:=\{\lambda \in \mathbb{C}: \lambda+i \mathbb{R} \subseteq \sigma(A)\}
\end{aligned}
$$

(b) $\sigma(T(t))=\overline{\exp (t \sigma(A))}$ for every $t \geqq 0$.\\
(c) Every isolated point of $\sigma(A)$ is a first order pole of the resolvent.

NOTES.\\
Spectral theory for a single positive operator is an essential cornerstone for spectral theory of positive one-paratmeter semigroups. Many of the results we have presented in this chapter have analogues in the discrete case (i.e. for a single operator). This relation may serve as a guide. However, only in few cases can the continuous version be deduced directly from its discrete analogue. Therefore we will not try to trace back the roots of every result to the discrete version. Instead we refer to Schaefer (1974) and the notes and references given there.\\
Many of the results we have presented in this chapter can be extended (more or less easily) to the more general setting of Banach lattices, which include the very important examples of $\mathrm{L}^{\mathrm{p}}$-spaces. Others are typical for $\mathrm{C}(\mathrm{X})$ and allow no extension. We will discuss this fact in more detail in Chapter ${ }^{\circ} \mathrm{C}-\mathrm{III}$. The more general setting considered there also allows us to prove results for $C_{( }(X)$ which we could not obtain staying within the framework of spaces of continuous functions.

Section 1. Theorem 1.1 was stated by Karlin (1959), but a complete proof is given in Derndinger (1980). Propositon 1.5 is taken from Greiner (1982) and Theorem 1.6 is (implicitely) contained in Derndinger-Nagel (1979). A generalization to (nonlattice) ordered Banach spaces can be found in Sec. 2.4 of Batty-Robinson (1984).

Section 2. Lemma 2.3 dates back to Rota (see Schaefer (1974)). Our approach follows Greiner (1981). The notion '(imaginary) additively cyclic' was introduced by Derndinger (1980) (and Schaefer (1980) respectively). Derndinger proves some fyclicity results for the boundary spectrum. A result similar to Proposition 2.7 is given in Sec.7.4 of Davies (1980). Lemma 2.8 in combination with C-III,Lemma 3.13 can be used to characterize semigroups whose spectral bound is a pole of finite algebraic multiplicity (see C-III, (3.19)). The hypothesis of Theorem 2.9 can be weakened, one only needs that $s(A)$ is a pole of the resolvent (see C-III, Cor.2.12). Further results on the cyclicity of the boundary spectrum will be given in Chapter C-III. In particular we refer to C-III, Thms.2.10, 3.11 and 3.13. The dichotomy stated in (2.19) is probably the most interesting consequence of cyclicity results. It has far reaching consequences on the asymptotic behavior of positive semigroups. Example 2.13 is due to Davies (unpublished note). Example 2.14(b) will be discussed in more detail and more generality in Section 3 of Chapter B-IV. We return to Remark 2.15(b) in Section 2 of B-IV .

Section 3. The concept of irreducibility as defined in 3.1 is closely related to various other notions: In topological dynamics flows inducing irreducible semigroups are called 'minimal flows' (cf. Example $3.4(a)$ ). Moreover, 'ergodicity' and 'unique ergodicity' are closely related to irreducibility (see Cornfeld-Fomin-Sinai (1982) or Krengel (1985)). Irreducible semigroups are discussed to some extent in Davies (1980). E.g. he proves a special case of Theorem 3.6 . Proposition 3.3 will be generalized in C-III,Prop.3.3. Assertion (a) of Proposition 3.5 was proven by Schaefer (1983) while Theorem 3.6 is taken from Greiner (1982). Elliptic operators (more general than Example $3.10(b)$ ) as generators on spaces of continuous functions, were investigated by many people, e.g. Bony-Courrège-Priouret (1968), Kuhn (1985), Roth (1976)\&(1978) and Stewart (1974).

Section 4. Theorem 4.1 is due to Derndinger (1984). The spectrum of semigroups of Markov lattice homomorphisms is investigated by Derndinger-Nagel (1979). In particular they prove Theorem 4.4 for Markov semigroups. Earlier results are due to Scarpellini (1974). We indicated briefly in Example 4.6 that there is a relationship between spectral properties of lattice semigroups and differentiable dynamics. For more details we refer to Chicone-Swanson (1981) and Sacker-Sell (1978). E.g., the 'annular hull theorem' is a special case of Theorem 4.11 (b). The general result 4.11 was proven by Arendt-Greiner (1984).

\begin{abstract}
CHAPTER B-IV\\
ASYMPTOTICSSOF\\
POSITIVE SEMIGROUPS ON $\quad \mathrm{C}_{\mathrm{O}}(\mathrm{X})$
\end{abstract}

In the following chapter we will examine the asymptotic behavior of positive semigroups on spaces of continuous functions.\\
The first section is devoted to the various "growth constants" defined in Chapter A-IV and to their coincidence for positive semigroups.\\
In the second section we treat the asymptotic behavior of positive semigroups which do not differ "too much" from compact semigroups. Properties such as eventual compactness or quasimcompactness allow to describe the long term behavior of the semigroup by using the results from $A-I I I$ and $B$-III on the spectrum of the generator.\\
In the last section we investigate differential delay equations by semigroup methods. In particular, we characterize the spectral bound of the solution semigroups thereby finding simple conditions for stability. Numerous examples conclude the chapter.

\begin{enumerate}
  \item STABILITY OF POSITIVE SEMIGROUPS ON $\mathrm{C}_{\mathrm{O}}(\mathrm{X})$\\
by Frank Neubrander
\end{enumerate}

In Chapter A-IV we have seen that the long term behavior of a semigroup $(T(t))_{t \geqq 0}$ is strongly connected with the existence (and growth) of the resolvent of the generator $A$ in a right halfplane. In particular, the exponential growth of certain semigroups is determined solely by the location of the spectrum (see A-IV, (1.7) and (1.8)). In these cases spectral bound $s(A)$ and growth bound $\omega(A)$ coincide and the equality


\begin{equation*}
s(A)=w_{1}(A)=w(A) \tag{1.1}
\end{equation*}


holds.

Unfortunately, (1.1) does not hold for positive semigroups in general. In A-IV, Example 1.2(2), we have seen that for the generator A of the (positive) translation semigroup on the Banach lattice $C_{0}\left(\mathbb{R}_{+}\right) \cap I^{I}\left(\mathbb{R}_{+}, e^{x} d x\right)$ the strict inequalitiy $\omega_{1}(A)<\omega(A)$ is valid. For positive semigroups on certain nice Banach Iattices ( 1.1 ) is true. One of these nice Banach lattices is $C_{0}(X)$. This will be proved in Theorem 1.4.

For compact $X$, (1.1) was already proved in B-II, Cor.1.14 and B-III, Thm.1.6 respectively. Actually much more is true and for positive semigroups on $C(K), k$ compact, all stability concepts mentioned in chapter A-IV are mutually equivalent.

Theorem 1.1. Let A be the generator of a positive semigroup $(T(t))_{t \geq 0}$ on $C(K)$, $K$ compact. Then\\
(1.2) $s(A)=\inf \{\lambda \in \mathbb{R}: A f \leqq \lambda f$ for some $0 \ll f \in D(A)\}$

Moreover, $s(A)=w(A) \in R \sigma(A)=P \sigma\left(A^{\prime}\right)$ and the following statements are mutually equivalent:\\
(i) $s(A)<0$,\\
(ii) (T(t)) $t \geq 0$ is uniformly exponentially stable,\\
(iii) (T(t)) $t \geq 0$ is weakly stable; i.e. $\langle T(t) f, \mu>\rightarrow 0$ as $t \rightarrow \infty$ for every $f \in D(A)$ and every $\mu \in C(X)^{\prime}$.

Proof. (1.2) follows directly from A-III. 4.4 and the results from $B-I I$ and $B-I I I$ mentioned above. It remains to show the implication (iii) $\rightarrow$ (i).

If $\langle T(t) f, \mu\rangle \rightarrow 0$ for every $\mu \in C(K)$, then, by the uniform Boundedness Principle, $\|T(t) f\| \leq M_{f}$ for every $f \in D(A)$. Using $s(A) \leqq \sup \{\omega(f): f \in D(A)\}=\omega_{1}(A)$ (A-IV,Thm.1.4) we obtain that $s(A) \leqq 0$. Suppose $0=s(A)$. From B-III, Thm. 1.6 it follows that $s(A) \in P \circ\left(A^{\prime}\right)$, hence there is $0<\mu \in C(K)$ ' such that $T(t) ' \mu$ $=\mu$ for $t \geqq 0$. Since $D(A)$ is dense, there exists $f \in D(A)$ such that $\langle\mathrm{f}, \mu\rangle \neq 0$. Then $|\langle T(t) \mathrm{f}, \mu\rangle|=|\langle\mathrm{f}, \mu\rangle|>0$ which contradicts the weak stability. Therefore $s(A)<0$.

For spaces $C_{0}(X)$, $X$ locally compact, the different stability concepts are no longer equivalent.

Examples 1.2. (a) The left-translation semigroup on $C_{0}\left(\mathbb{R}_{+}\right)$or the semigroup generated by the Laplacian on $C_{0}\left(\mathbb{R}^{n}\right)$, see B-III,Ex.I.7, are uniformly stable but not exponentially stable.\\
(b) The left translations $T(t) f(x)=f(x+t)$ on $C_{0}(\mathbb{R})$ form a group of isometries. Hence $(T(t))_{t \geqslant 0}$ is not stable. However, ( $(t)$ ) $t \geqslant 0$ is weakly stable. Indeed, identifying $C_{0}(\mathbb{R})$ ' with the space of all bounded Borel measures on $\mathbb{R}$, for $\pounds \in C_{0}(\mathbb{R}), \mu \in C_{0}(\mathbb{R})^{\prime}$ we have

$$
\langle T(t) f, \mu\rangle=\int(\mathbb{T}(t) f)(x) d \mu(x)
$$

Obviously, $T(t) f$ tends pointwise to 0 as $t \rightarrow \infty$ and is dominated by the $\mu$-integrable function $\|f\|_{\infty} \cdot 1$. Thus Lebesque's Dominated Convergence Theorem implies $\lim \langle\mathbb{T}(t) \mathrm{f}, \mu\rangle=0$.\\
(c) Finally we give an example of a positive semigroup on $c_{0}(X)$ which is not weakly stable but satisfies $\operatorname{Re}(\operatorname{Po}(A) \cup \operatorname{Ro}(A))<0$. (Compare with A-IV, Cor.1.14).\\
Consider in the space $\mathbb{C} \backslash\{0\}$ a flow $\phi$ having the following properties:

\begin{itemize}
  \item The orbits starting at $z$ with $|z| \neq 1$ spiral towards the unit circle $\Gamma$;
  \item 1 is a fixed point and $\Gamma \backslash\{1\}$ is a homoclinic orbit\\
(i.e. $\lim _{t \rightarrow+\infty} \phi(t, z)=\lim _{t \rightarrow-\infty} \phi(t, z)=I$ for every $z \in \Gamma$ ).
\end{itemize}

A concrete example of this type is the flow governed by the following differential equations for the polar coordinates (i.e. $z=r \cdot e^{i \omega}$ )

$$
\begin{aligned}
& \dot{r}=1-r \\
& \dot{\omega}=1+\left(r^{2}-2 r \cdot \cos \omega\right)
\end{aligned}
$$

The locally compact set $\mathrm{X}:=\{\mathrm{z} \in \mathbb{C}: 0<|\mathrm{z}|<2, \mathrm{z} \neq 1\}$ is invariant under the flow $\phi$ and we consider on the space $C_{0}(x)$ the semigroup $(T(t))_{t \geqq 0}$ associated with $\phi$ (i.e. $T(t) f=f_{t},{ }_{t}$, $f \in C_{O}(X)$ ). We claim that\\
(i) $(T(t))_{t \geq 0}$ is not weakly uniformly stable ;\\
(ii) $\quad \operatorname{Po}(\mathrm{A}) \cap \mathrm{R}=\varnothing$;\\
(iii) $\operatorname{Ro}(A) \cap i R=\varnothing$.

Proof of (i): Given $z \in X,|z| \neq 1$, there exist sequences ( $t_{n}$ ), $\left(s_{n}\right)$ both tending to $\infty$ such that $\phi\left(t_{n}, z\right) \rightarrow 1$ and $\phi\left(s_{n}, z\right) \rightarrow-1$. Hence for $f \in C_{0}(x)$ we have\\
$\left\langle\mathrm{T}\left(t_{n}\right) f_{r} \delta_{z}\right\rangle=f\left(\phi\left(t_{n}, z\right)\right) \rightarrow 0$,\\
$\left\langle T\left(s_{n}\right) f, \delta_{z}>=f\left(\phi\left(s_{n}, z\right)\right) \rightarrow f(-1)\right.$.\\
Thus $\lim _{t \rightarrow \infty}\left\langle T(t) f, \delta_{z}>\right.$ does not exist for every $f \in C_{0}(x)$.\\
Proof of (ii): Assume that $T(t) f=e^{i \alpha t}{ }_{f}$ for every $t \geqq 0$ and some $\alpha \in \mathbb{R}$ (cf. A-III, Cor.6.4). Given $z \in X$, there exists a sequence $\left(t_{n}\right)$ such that $\phi\left(t_{n}, z\right) \rightarrow I$, hence\\
\includegraphics[max width=\textwidth, center]{2024_12_23_c6487cc0859199a15bd9g-217}\\
Proof of (iii): At first we point out that for $f \in C_{0}(x)$ such that $f$ vanishes on the unit circle $\Gamma$, we have $\lim _{t \rightarrow \infty}\|\mathrm{~T}(\mathrm{t}) \mathrm{f}\|=0$. Assume that $\mu$ is a bounded Borel measure such that $T(t)^{\prime} \mu=e^{i a t_{\mu}}$ for every $t \geqq 0$ and some $\alpha \in \mathbb{R}$. Then $\left\langle f, \mu>=e^{i \alpha t}<\mathcal{f}, T(t){ }^{\prime} \mu>=\right.$ $=e^{i \alpha t}\langle T(t) f, \mu\rangle \rightarrow 0$ for every $f \in C_{O}(x)$ with $f_{\mid \Gamma}=0$. It follows that the support of $\mu$ is contained in $\Gamma$. Since $\lim _{t \rightarrow \infty} \phi(t, z)=1$ for every $z \in \Gamma$, we obtain for arbitrary $f \in c_{0}(X)$ : $(T(t) f)(z) \rightarrow 0 \quad \mu-a \cdot e \ldots$ Lebesgue's Dominated Convergence Theorem implies $\langle f, \mu\rangle=e^{-i \alpha t}\langle T(t) f, \mu\rangle \rightarrow 0$ as $t \rightarrow \infty$ for every $f \in C_{O}(X)$. Thus $\mu=0$.

Now we are going to prove the main result of this section. At first we note that the positive part of the domain of the adjoint operator is sufficiently large. In fact, we know that $\lambda R(\lambda, A) \rightarrow I d$ strongly as $\lambda \rightarrow \infty$. It follows that $\lambda^{2} \mathrm{R}(\lambda, \mathrm{A})^{2}+\mathrm{Id}$ strongly, hence $\lambda^{2} R(\lambda, A),{ }^{2} \rightarrow$ Id with respect to o( $\left.{ }^{\prime}, E\right)$-topology. If A generates a positive semigroup then $\lambda^{2} R(\lambda, A)^{2} \mu \in D(A *):=D\left(A^{*}\right) \cap E_{+}^{;}$for $\mu \in E_{+, j}^{\prime}$ (Note that $R(\lambda, A)^{\prime} E^{\prime} \subset D^{\prime}\left(A^{\prime}\right) \subset E^{*}$, thus $R(\lambda, A){ }^{2} E^{\prime} \subset R(\lambda, A) ' E^{*}=D(A *)$.)\\
We summarize these observations in the following lemma.

Lemma 1.3. Let $A$ be the generator of a positive semigroup on a Banach lattice E. Then every $\mu \in E^{\prime}$ is the $\sigma\left(E^{\prime}, E\right)-l i m i t$ of elements in $D\left(A^{*}\right)_{+}$; i.e., $\overline{D_{\left(A^{*}\right)_{+}}^{+}}\left(E^{\prime}, E^{+}\right)=E_{+}^{\prime}$.

Theorem 1.4 Let $A$ be the generator of a positive semigroup on $C_{0}(\mathrm{X})$. Then

$$
s(A)=\omega_{1}(A)=\omega(A) \in \sigma(A)
$$

Proof. Rescaling the semigroup we may assume $\omega(\mathrm{A})=0$. (In case $\omega(A)=-\infty$, then $\sigma(A)=\emptyset$ hence $s(A)=-\infty$ )\\
Suppose $0 \notin \sigma(A)=\sigma\left(A^{*}\right)$. Then, by the holomorphy of the resolvent and by A-II, Prop.1.11\\
$R(0, A *) \Phi=\sum_{n=0}^{\infty} R\left(1, A^{*}\right)^{n+1} \Phi=\sum_{n=0}^{\infty} \int_{0}^{\infty} \frac{1}{n!} t^{n} e^{-t} T(t) * \Phi d t$ for every $\Phi \in C_{0}(X)^{*}$. If $0 \leqq \epsilon C_{0}(X)^{*}$ and $0 \leqq \rho \epsilon C_{0}(X)^{\prime \prime}$ we can interchange integration and summation by the Monotone Convergence Theorem; i.e.


\begin{align*}
\left\langle R\left(0, A^{*}\right) \Phi, \rho\right\rangle & =\sum_{n=0}^{\infty} \int_{0}^{\infty} \frac{1}{n!} t^{n} e^{-t}\langle T(t) * \Phi, \rho\rangle d t  \tag{1.3}\\
& =\int_{0}^{\infty}\langle T(t) * \Phi, \rho\rangle d t .
\end{align*}


since every element of $C_{0}(X)^{*}$ and $C_{0}(X)^{\prime \prime}$ is the difference of posi-\\
tive elements the equation (1.3) holds for every $\Phi \quad \in C_{0}(X)$ *, $\rho \in C_{0}(X)^{\prime \prime}$. This means that the net $\left.\int_{0}^{r} T(t) * \Phi d t\right)_{r}>0$ converges weakly to $R\left(0, A^{*}\right) \Phi$. But for positive $\Phi$ the net is monotone and therefore strongly convergent by Dini's Theorem (see Schaefer (1974), II.Thm. 5.9). Hence $R(0, A *) \Phi=\int_{0}^{\infty} T(t) * \Phi$ at for every $\Phi \in C_{0}(X) *$.

Now we make use of the special character of the space $C_{O}(X)$. For positive functions $f_{1}, f_{2} \in C_{0}(x)$ we have $\sup \left(\left\|f_{1}\right\|,\left\|f_{2}\right\|\right)=$ $\left\|\sup \left(f_{1}, f_{2}\right)\right\|$. Let $\mu_{1}, \mu_{2} \in C_{0}(X)_{+}^{\prime}$ and $\varepsilon>0$. Then there are positive elements $f, g$ in the unit ball of $C_{0}(X)$ such that $\langle f, \mu\rangle \geqq\|\mu\|-\varepsilon$ and $\left\langle g, \mu_{2}>\geqq\left\|\mu_{2}\right\|-\varepsilon\right.$. For $h:=\sup (f, g)$ we obtain $\|\mathrm{h}\| \leqq 1$ and\\
$\left\|\mu_{1}+\mu_{2}\right\| \geqq\left\langle h_{1} \mu_{1}+\mu_{2}\right\rangle \geqq\left\langle f, \mu_{1}\right\rangle+\left\langle g, \mu_{2}\right\rangle \geqq\left\|\mu_{1}\right\|+\left\|\mu_{2}\right\|-2 \varepsilon$.\\
Hence $\left\|\mu_{1}^{2}+\mu_{2}\right\|=\left\|\mu_{1}\right\|+\left\|\mu_{2}\right\|^{1}$ for $\mu_{1}, \mu_{2} \in C_{0}(X){ }_{+}^{2}$ (see also C-I). Approximating the integral by Riemann sums one obtains\\
\includegraphics[max width=\textwidth]{2024_12_23_c6487cc0859199a15bd9g-218} therefore, for $r \rightarrow \infty,\|R(0, A *) \mu\|=\left\|\int_{0}^{\infty} T(t) * \mu d t\right\|=\int_{0}^{\infty}\|T(t) * \mu\| d t$ $\left(\mu \in C_{0}(X)_{+}^{*}\right)$. Given $\mu \in C_{O}(X)^{\prime}$ there is a sequence $\mu_{n} \in C_{0}(X) *$ converging $\sigma\left(E^{\prime}, E\right)$ to $|\mu|$ (Lemma 1.3). From $\left|\left\langle f, T(t){ }^{\prime} \mu\right\rangle\right| \leqq$ $\langle T(t)| f\left|,|\mu|>=\lim _{n \rightarrow \infty}\langle T(t)| f\right|, \mu_{n}>\quad$ we conclude $\left|<f, T(t)^{\prime} \mu>\right| \leqq$ $\liminf _{n \rightarrow \infty}\|f\|\left\|T(t) * \mu_{n}\right\|$ and therefore $\|T(t) \cdot \mu\| \leqq \liminf _{n \rightarrow \infty}\left\|T(t) \mu_{n}\right\|$ $(t \geqq 0)$. Applying Fatou's Lemma we obtain\\
$\int_{0}^{\infty}\left\|\mathrm{T}(t)^{\prime} \mu\right\| \mathrm{d} t \leq \int_{0}^{\infty}\left(\liminf \left\|\mathrm{T}(t)^{\prime} \mu\right\|\right) \mathrm{d} t \leq$\\
$\liminf \int_{0}^{\infty}\|T(t) ' \mu\| d t=\liminf \left\|R\left(0, A^{*}\right) \mu_{n}\right\| \leqq\left\|R\left(0, A^{*}\right)\right\| \cdot \liminf \left\|\mu_{n}\right\|<\infty$. cobserve that $t \rightarrow\left\|T(t)^{\prime} \mu\right\|=\sup \{\langle T(t) f, \mu\rangle:\|f\| \leqq$ l\} is lower semi-continuous and hence measurable). Using A-IV,Thm.1.10 we obtain $\omega(A *)<0$. But $\omega(A)=\omega\left(A^{*}\right)$ by $A-I I I, 4.4$ (iii), which contradicts $\omega(A)=0$.

Remark 1.5. If (T(t)) is a positive semigroup on an a-directed ordered Banach space E (see Asimow-Ellis (1980), p.39), then the dual of E admits a reversion of the triangle inequality;\\
i.e. $\quad\left[\left\|\mu_{i}\right\| \leqq \alpha\left\|\mu_{i}\right\|\right.$ for $\mu_{i} \in E_{+}^{\prime}$, and Theorem 1.4 remains valid (see Batty-Davies (1983)). The proof given above may be used with almost no modification.

At this point we close the discussion of the stability of positive semigroups on $C_{0}(X)$ and refer to section 1 of $C-I V$ and D-IV respectively, where the stability of positive semigroups on arbitrary Banach lattices and on C*-algebras will be treated.

\section*{2. COMPACT AND QUASI-COMPACT SEMIGROUPS}
by\\
Günther Greiner

Using the Riesz-Schauder Theory for compact operators (see e.g. Chapter VII. 4 of Dunford-Schwartz (1958) or Section 26 of Pietsch (1978)) and the results of Chapter A-III, one can easily describe the asymptotic behavior of eventually compact semigroups. Since no positivity is involved we state the fundamental result for arbitrary Banach spaces.

Theorem 2.1. Let $(T(t))_{t \geqq 0}$ be a strongly continuous semigroup on a Banach space $G$ which is eventually compact (i.e., there is $t_{0}>0$ such that $T\left(t_{0}\right)$ is a compact operator). Then the spectrum of the generator A is a countable set (possibly finite or empty) and contains only poles of finite algebraic multiplicity. Furthermore, the set $\{\mu \in \sigma(A): \operatorname{Re} \mu \geqq r\}$ is finite for every $r \in \mathbb{R}$. Thus $\sigma(A)=\left\{\lambda_{1}, \lambda_{2}, \lambda_{3}, \ldots\right\}$ with $\operatorname{Re} \lambda_{n+1} \leqq \operatorname{Re} \lambda_{n}$ for all $n \in \mathbb{N}$ and $\lim _{n \rightarrow \infty} \operatorname{Re} \lambda_{n}=-\infty$ if $\sigma(A)$ is infinite. Denoting the pole order at $\lambda_{n}$ by $k(n)$ and the corresponding residue by $P_{n}$, we have for every $m \in \mathbb{N}$\\
(2.1) $T_{n}(t)=\exp \left(\lambda_{n} t\right) \cdot \sum_{j=0}^{k(n)-1} \frac{1}{j} \cdot t^{j}\left(A-\lambda_{n}\right)^{j} \circ P_{n} \quad(t \geq 0)$,\\
$\left\|R_{m}(t)\right\| \leqq C \cdot \exp \left(\left(\varepsilon+\operatorname{Re} \lambda_{m}\right) t\right)$ for $t \geqq 0, \varepsilon>0$ and a suitable constant $\mathrm{C}=\mathrm{C}(\varepsilon, \mathrm{m})$.

Proof. Fix $r \in \mathbb{R}$. By the Riesz-Schauder Theory we know that $\left\{z \in \sigma\left(\mathbb{T}\left(t_{0}\right)\right):|z| \geq \exp \left(r t_{0}\right)\right\}$ is a finite set and contains only poles of finite algebraic multiplicity. Thus the first assertion follows from A-III, Cor.6.5 .\\
To prove the remaining assertion we fix $\mathrm{m} \in \mathbb{N}$ and apply the spectral decomposition as described in Section 3 of Chapter A-III. For simplicity we assume $\operatorname{Re} \lambda_{m+1}<\operatorname{Re} \lambda_{m}$. Let $P$ be the spectral projection of $T\left(t_{0}\right)$ corresponding to the spectral set $\left\{z \in \sigma\left(T\left(t_{0}\right)\right)\right.$ : $\left.|z| \geq \exp \left(\operatorname{Re} \lambda_{m} \cdot t_{0}\right)\right\}$. Then $P$ reduces the semigroup and we have $\sigma\left(T\left(t_{0}\right) \mid \operatorname{ker} P\right) \subset\left\{z \in \mathbb{C}:|z|<\exp \left(\operatorname{Re} \lambda_{m} \cdot t_{0}\right)\right\}$. Hence the type of ( $T\left(t_{0}\right.$ ) |ker $\mathrm{P}^{\prime}$ ) is less than Re $\lambda_{m}$. Then there exists a constant $\mathrm{C}_{\mathrm{O}}$ such that\\
$\|T(t)(I d-P)\| \leqq \| \mathbb{T}(t) \mid$ ker $p\|\cdot\| I d-P\|\leqq\| I d-P \| \cdot C_{0} \cdot \exp \left(\operatorname{Re} \lambda_{m} \cdot t\right) \cdot$\\
We define $R_{m}(t):=T(t)(I d-P), T_{n}(t):=T(t) P_{n}(n \in \mathbb{N})$.\\
Then $R_{m}(t)$ satisfies the estimate stated in (2.1) and we have $T(t)=$ $\sum_{n=1}^{m} T_{n}(t)+R_{m}(t)$ because $p=\sum_{n=1}^{m} p_{n}$ by A-III, Cor. $6.5(i i)$. The family of projections Id-P, $P_{1}, P_{2}, \cdots, P_{m}$ reduces the semigroup. Thus in order to prove the representation of $T_{n}(t)$ stated in (2.1) we only have to consider elements $f \in P_{n} E=\operatorname{ker}\left(\lambda_{n}-A\right)$. That is we can assume $E=P_{n} E, \sigma(A)=\left\{\lambda_{n}\right\}, P_{n}=I d$ and for simplification we drop the index $n$, i.e., $\lambda=\lambda_{n}, k=k(n)$. Then $A$ is a bounded operator satisfying $(\lambda-A)^{k}=0$ and its resolvent is given by $R(\mu, A)=(\mu-\lambda)^{-1} \sum_{j=0}^{k-1}(\mu-\lambda)^{-j}(A-\lambda)^{j}$ for $\mu \neq \lambda$. It follows that $R(\mu, A)^{i}=(\mu-\lambda)^{-i} \sum_{j=0}^{k-1}\binom{j+i-1}{i-1}(\mu-\lambda)^{-j}(A-\lambda)^{j}$. Hence we have $\left(\frac{i}{t} R\left(\frac{i}{t}, A\right)\right)^{i}=\left(1-\lambda \frac{t}{i}\right)^{-i} \sum_{j=0}^{k-1}(\underset{i-1}{j+i-1})(i-\lambda t)^{-j} t^{j}(A-\lambda)^{j}$ for every $i \in \mathbb{N}$. Since $\lim _{i \rightarrow \infty}\left(1-\lambda \frac{t}{i}\right)^{-i}=e^{\lambda t}$ and $\lim _{i \rightarrow \infty}\binom{j+i-1}{i-1}(i-\lambda t)^{-j}=\frac{1}{j!}$ for every $j \in \mathbb{N}$ the assertion follows from formula $A-I I,(1.3)$.

Combining Thm.2.1 with the results of Chapter B-III one can describe the behavior of $T(t)$ as $t \rightarrow \infty$ provided that $\left(T(t){ }_{t \geqq 0}\right.$ is a positive semigroup. We give a typical example.

Corollary 2.2. Let $(T(t))_{t \geq 0}$ be a positive semigroup on a space $C_{0}(x)$ which is irreducible and eventually compact. Then there exist a unique real number $r \in \mathbb{R}$, a strictly positive function $h$ and a strictly positive bounded Borel measure $v$ such that for suitable constants $\delta>0, M \geqq 1$ one has\\
(2.2) $\|\exp (-r t) \cdot T(t)-v \otimes h\| \leq M \cdot e^{-\delta t}$ for all $t \geqq 0$.

In particular, for every $f \in C_{0}(x)$ and $t \geqq 0$ one has (2.3) $\quad e^{r t}\left(\left|\int f d v\right|-M \cdot e^{-\delta t}\|f\|\right) \leq\|T(t) f\| \leqq e^{r t}\left(\left|\int f d v\right|+M \cdot e^{-\delta t}\|f\|\right)$.

Proof. We take $r:=s(A)$. By B-III, Prop. $3.5(a)$ we have $r>-\infty$. Moreover, by assertion (e) of the same proposition we know that $r$ is an algebraically simple pole and the corresponding residue $P$ has the form $P=v \otimes \mathrm{~h}$ for strictly positive eigenvectors $v$ and $h$ of $A$ and $A^{\prime}$, respectively. Without loss of generality we may assume $\|h\|=1$. Corollary 2.11 of Chapter B-III implies that $r$ is strictly dominant, i.e., enumerating the eigenvalues as described in Thm.2.1 we have $\operatorname{Re} \lambda_{2}<\lambda_{1}=r$. Now (2.2) follows from (2.1) for $\mathrm{m}=1$.

Applying the triangle inequality to $T(t) f=e^{r t}\left(P f+\left(e^{-r t} T(t) f-P f\right)\right)$ and using (2.2) one easily deduces (2.3).

Let us point out the following consequence of Corollary 2.2 :\\
For every positive, non-zero initial value f the solution $\mathrm{T}()$. of the abstract Cauchy problem $\dot{u}=A u$ decreases or increases exponentially in norm according to the sign of $\mathbf{r}=\mathbf{s}(\mathrm{A})$.\\
If $s(A)=0$ then $T()$.$f tends to an equilibrium state which is$ unique up to a constant and non-zero whenever the initial value is positive and non-zero.

In order to apply Thm.2.1 and its corollary to concrete problems one needs conditions which ensure that the semigroup is eventually compact. We discuss this problem for the spaces $\mathrm{C}(\mathrm{K}), \mathrm{K}$ compact, in more detail. The crucial tool is the following characterization of weakly compact subsets in the dual space $M(K)=C(K)$ ' due to Grothendieck (1953) .

Proposition 2.3. Let $K$ be a compact space. For a subset $M \subset M(K)=$ $C(K)$ ' the following assertions are equivalent:\\
(i) $M$ is relatively compact for the weak topology $\sigma\left(\mathrm{M}(\mathrm{K}), \mathrm{M}(\mathrm{K})^{\prime}\right.$ ) ;\\
(ii) for each weak null sequence ( $f_{n}$ ) in $C(K), \lim _{n \rightarrow \infty}\left\langle f_{n}, v\right\rangle=0$ uniformly for $\nu \in M$;\\
(ii.i) for each sequence $\left(U_{n}\right)$ of disjoint open subsets of K , $\lim _{n \rightarrow \infty} v\left(U_{n}\right)=0$ uniformly for $v$ in $M$.

For a proof of this result see e.9. II.9.8 in Schaefer (1974). We use this proposition in order to describe weakly compact operators on spaces $C(K)$. As usual we identify in the natural way the bounded Borel functions on $K$ with a subspace $B(K)$ of $M(K)^{\prime}=C(K)^{\prime \prime}$; in general, $\mathrm{C}(\mathrm{K}) \varsubsetneqq \mathrm{B}(\mathrm{K}) \varsubsetneqq \mathrm{C}(\mathrm{K})^{\prime \prime}$.

Proposition 2.4. Let $K$ be a compact space, $G$ be a Banach space and let $R: C(K) \rightarrow G$ be a bounded linear operator.\\
(a) The following assertions are equivalent :\\
(i) $R$ is weakly compact;\\
(ij) for every bounded Borel function $g$ on $K$ we have $R^{\prime \prime} g \in G$;\\
(ii.i) for every Borel set $C \subset K$ we have $R^{\prime \prime}\left(\chi_{C}\right) \notin G$.

In case $G=C(K)$ these conditions are equivalent to the following:\\
(iv) if $\left(f_{n}\right) \in C(K)$ is a bounded sequence then (Rf $f_{n}$ ) has a subsequence which converges pointwise to a continuous function.\\
(b) If $R$ is weakly compact then it maps weakly convergent sequences into norm convergent sequences. In particular, the square of a weakly compact operator $\mathrm{T}: \mathrm{C}(\mathrm{K}) \rightarrow \mathrm{C}(\mathrm{K})$ is a compact operator.

Proof.(a) (i) $\rightarrow$ (ii) follows from the following characterization of weakly compact operators (see e.g.,II.Prop.9.4 of Schaefer (1974)): An operator is weakly compact if and only if its second adjoint maps the bidual into the original space.\\
(ii) $\rightarrow$ (iii) is trivial and it remains to show that (iii) implies (i): On the Borel field $B$ we define $m$ by $m(C):=R^{\prime \prime}\left(x_{C}\right)$. Then $m$ is a G-valued additive set function. For $y^{\prime} \epsilon G^{\prime}$ we have $y^{\prime} \circ m=R^{\prime} y^{\prime} \in M(K)$. Hence for every $y^{\prime} \in G^{\prime} y^{\prime}$ om is a countable additive set function, i.e., m is weakly countably additive. By Pettis" Theorem (see IV.Thm.10.1 in Dunford-Schwartz (1958)) we have that $m$ is countably additive with respect to the norm. In particular, for a sequence $U_{n}$ of mutually disjoint Borel sets we have $\lim _{n \rightarrow \infty}\left\|m\left(U_{n}\right)\right\|=0$. It follows that $\lim _{n \rightarrow \infty} y^{\prime} \circ m\left(U_{n}\right)=0$ uniformly for $y^{\prime} \in G^{\prime},\left\|y^{\prime}\right\| \leqq 1$. Now condition (iii) of Prop.2.3 shows that \{R'y' $\left.: Y^{\prime} \in G^{\prime},\left\|y^{\prime}\right\| \leqq 1\right\}$ is relatively weakly compact, i.e., $R^{\prime}$ is weakly compact. Thus $R$ is weakly compact as well.\\
In case $G=C(K)$ the equivalence of (i) and (iv) is a consequence of two results: First, Eberlein's Theorem states that for the weak topology in any Banach space compactness and sequential compactness are equivalent. Second, Lebesgue's Dominated Convergence Theorem assures that a sequence ( $f_{n}$ ) $\subset C(K)$ converges weakly to $f \in C(K)$ if and only if it is bounded and $f_{n}(x) \rightarrow f(x)$ for every $x \in K$.\\
(b) Suppose ( $\mathrm{f}_{n}$ ) is a sequence in $C(K)$ which converges to 0 for the weak topology. Since $R$ is weakly compact the same is true for the adjoint $R^{\prime}$, i.e., $\left\{R^{\prime} y^{\prime}: y^{\prime} \in G^{\prime},\left\|y^{\prime}\right\| \leqq 1\right\}$ is relatively weakly compact in M(K) . From Prop.2.3 (i) + (ii) we obtain that $\left\langle R f_{n} y^{\prime}\right\rangle=\left\langle f_{n}, R^{\prime} y^{\prime}\right\rangle \rightarrow 0$ as $n \rightarrow \infty$ uniformly for $y^{\prime} \in G^{\prime},\left\|y^{\prime}\right\| \leq 1$. That is $\lim _{n \rightarrow \infty}\left\|R f_{n}\right\|=0$.\\
The final assertion follows from the first and the characterization of weakly compact operators stated in (iv) of (a) .

The next result which is an immediate consequence of Thm.2.1 and Prop. 2.4 is motivated by the theory of Markov processes. For a Markov operator (see B-I, Sec.3) condition (ii) of Prop.2.4(a) is called the strong Feller property .

Theorem 2.5. Let (T(t)) $t \geq 0$ be semigroup of Markov operators on $C(K)$, K compact, such that one operator $T\left(t_{0}\right)$ has the strong Feller property. Then there exists a positive projection P of finite rank such that $\|T(t)-p\| \leqq M \cdot e^{-\delta t}$ for suitable constants $\delta>0, M \geqq 1$.

Proof. By Prop.2.4(a) T(t. is weakly compact. Thus by Prop.2.4(b) $\mathrm{T}\left(2 \mathrm{t}_{\mathrm{O}}\right)$ is compact, i.e., $(\mathrm{T}(t)){ }_{t \geqslant 0}$ is eventually compact. Moreover, by B-III, Cor.2.11 $\mathrm{s}(\mathrm{A})=0$ is strictly dominant and a first order pole of the resolvent by B-II,Rem.2.15(a). The assertion now follows easily from Thm. 2.1.

We close the discussion of eventually compact semigroups by describing a situation where Thm.2.5 can be applied. A more detailed description of the relation between Markov processes and positive semigroups on C(K) is given in Chap. 2 of van Casteren (1985).

Example 2.6. Let K be a compact space and $\left\{P_{t}: t>0\right\}$ be a Markov transition function on $K$ which satisfies the strong Feller property and which is stochastically continuous. That is, every $P_{t}$ is a real-valued function defined on the product $K \times B$ where $B$ denotes the Borel field on K, such that\\
(a) for $x \in K$ and $t>0$ fixed, $P_{t}(x,$.$) is probability measure;$ (b) for $C \in B$ and $t>0$ fixed, $P_{t}(., c)$ is a continuous function; (c) $P_{t+s}(x, c)=\int_{K} P_{s}(y, c) P_{t}(x, d y)$ for all $s, t>0, x \in K, c \in B$; (d) $\quad \lim _{t \downarrow 0} P_{t}(x, U)=1$ for every open set $U$ containing $x$.

Condition (b) is the strong Feller property, (c) is the ChapmanKolmogorov equation and (d) expresses stochastic continuity. Given $\left\{P_{t}\right\}$ as above one defines for $f \in C(K), x \in K$ and $t>0$ (2.4) (T(t)f)(x):= $\int_{K} f(y) P_{t}(x, d y)$.

Then it is not difficult to verify that $T(t) f \in C(K)$, that $T(t)$ is a Markov operator on $C(K)$ and that $(T(t))_{t \geq 0}-$ with $T(0)=I d-$ is a one-parameter semigroup. In fact, the first assertion is a consequence of (a) and (b), the second follows from (a) and the semigroup property is implied by the Chapman-Kolmogorov equation.

Moreover, the semigroup $(T(t))_{t \geqq 0}$ is strongly continuous. This can be seen as follows: In view of Prop.1.23 in Davies (1980) we only have to show that $\lim _{t \downarrow 0}\langle T(t) f-f, v\rangle=0$ for every $f \in C(K), v \in M(K)$. Due to Lebesgue's Dominated Convergence Theorem this is true whenever $\lim _{t+0}(T(t) f)(x)=f(x)$ for every $f \in C(K), x \in K$. Given $f, x$ and $\varepsilon>0$ there exists an open neighborhood $U$ of $x$ such that $|f(x)-f(y)|<\varepsilon$ for every $y \in U$. Then we have $(T(t) f)(x)-f(x)=\int_{K} f(y) P_{t}(x, d y)-\int_{K} f(x) P_{t}(x, d y)=$ $\int_{U}(f(y)-f(x)) P_{t}(x, d y)+\int_{K \backslash U}(f(y)-f(x)) P_{t}(x, d y) \leqq$ $\varepsilon \cdot P_{t}(x, U)+2\|f\|_{\infty} \cdot P_{t}(x, K \backslash U)$. since $P_{t}(x, U) \leqq 1$ and $\lim _{t \downarrow 0} P_{t}(x, U)=1=P_{t}(x, K)$ this estimate implies $\left.\limsup _{t+0}(\mathrm{~T}(\mathrm{t}) \mathrm{f})(\mathrm{x})-\mathrm{f}(\mathrm{x})\right) \leqq \varepsilon$. Since $\varepsilon>0$ was arbitrary we have pointwise convergence hence strong continuity of the semigroup.\\
Finally we observe that every operator $T(t)$ defined by (2.4) has the strong Feller property since $T(t){ } \chi_{C}=P_{t}(., C)$ for every Borel set $\mathrm{C} \subset \mathrm{K}$ (see Prop.2.4(a)).\\
Thus Thm.2.5 can be applied in this situation.

We now turn our interest from eventually compact semigroups to quasicompact semigroups. While "eventually compact" means that the operators $T(t)$ with $t \geqq t_{0}$ have to be compact, "quasi-compactness" only means that $\mathrm{T}(\mathrm{t})$ approaches the compact operators as $t \rightarrow \infty$. To make this precise we introduce the following notations.\\
For a Banach space $G$ the ideal of all compact linear operators on $G$ is denoted by $K(G)$. For $T \in L(G)$ we define\\
$\operatorname{dist}(T, K(G)):=\inf \{\|T-K\|: K \in K(G)\}$.

Definition 2.7. A strongly continuous semigroup (T(t)) $t \geqq 0$ on a Banach space $G$ is called quasi-compact if $\lim _{t \rightarrow \infty} \operatorname{dist}(T(t), K(G))=0$.

Quasi-compactness can be characterized in different ways. Two of them are stated in the following proposition. The first one uses the notion of the essential growth bound ${ }^{w}$ ess $(T)$ of a semigroup $T$ which was introduced in A-III,3.7.

Proposition 2.8. For a strongly continuous semigroup $T=(T(t))_{t \geq 0}$ on a Banach space G the following conditions are equivalent:\\
(i) $T$ is quasi-compact;\\
(ii) wess ${ }^{(T)}$ < 0 ;\\
(iii) There exist $t_{0}>0, K \in K(G)$ such that $\left\|T\left(t_{0}\right)-K\right\|<1$.

Proof.(i) $\rightarrow$ (iii) is obvious by the definition of quasi-compactness. (iii) $\rightarrow$ (ii): Recalling the definition of the essential spectral radius from A-III, (3.6) , assertion (iii) implies $r_{\text {ess }}\left(\mathrm{T}\left(t_{0}\right) \leq\left\|\mathrm{T}\left(t_{0}\right)\right\|_{\text {ess }}<1\right.$. Then $\omega_{\text {ess }}(T)<0$ by $\mathrm{A}-\mathrm{III},(3.10)$. (ii) $\rightarrow$ (i): By A-III, (3.10) we have $r_{\text {ess }}(T(1))<1$. Then A-III, (3.6)\\
\includegraphics[max width=\textwidth]{2024_12_23_c6487cc0859199a15bd9g-225} for suitable $n_{0} \in N, a<1$ we have $\|T(n)\|_{e s s}<a^{n}$ for $n \geq n_{0}$. choosing a sequence $k_{n} \in K(G)$ such that $\left\|T(n)-K_{n}\right\|<a^{n}$ for $n \geq n_{0}$ and defining $M:=\sup _{0 \leq s \leq 1}\|T(s)\|$ we obtain for $t \in[n, n+1]$ $\left(n \geqq n_{0}\right)\left\|T(t)-T(t-n) K_{n}\right\| \leqq\|T(t-n)\| T(n)-K_{n} \| \leqq M \cdot a^{n}$. This implies that $\lim _{t \rightarrow \infty}$ dist(T(t), $\left.K(G)\right)=0$.

A typical situation where quasi-compact semigroups occur is the following. If $T=(T(t))_{t \geqq 0}$ is a strongly continuous semigroup with $\omega_{\text {ess }}(T)<\omega(T)$ then the rescaled semigroup $(\exp (-\omega(T)) T(t)) t \geq 0$ is quasi-compact. Obviously every semigroup with growth bound less than zero is quasi-compact. A more interesting situation is the following: If $\left(\mathrm{T}_{\mathrm{o}}(t)\right)_{t \geq 0}$ is a semigroup with growth bound less than zero and $A_{0}$ is its generator, then for every compact operator $K$ the perturbed operator $A:=A_{0}+K$ generates a quasi-compact semigroup. More generally we have the following result:

Proposition 2.9. If $(T(t))_{t \gtrless 0}$ is a quasi-compact semigroup on a Banach space $G$ with generator $A$ and $K$ is a compact operator then $A+K$ generates a quasi-compact semigroup.

Proof. If (T(t)) ${ }_{t \geq 0}$ and $(S(t))_{t \geq 0}$ are the semigroups generated by $A$ and $A+K$ respectively we have $S(t)=T(t)+\int_{0}^{t} T(t-s) K s(s) d s$. In view of prop.2.8(iii) it is enough to show that $\int_{0}^{t} \mathrm{~T}(\mathrm{t}-\mathrm{s}) \mathrm{KS}$ (s) ds is a compact operator.\\
Since the mapping $(t, x) \rightarrow T(t) x$ is jointly continuous on $\mathbb{R}_{+} x$ G and since $K$ is compact the set $M:=\{T(s) K x: 0 \leqq s \leqq t,\|x\| \leqq 1\}$ is relatively compact in $G$. Having in mind that $\int_{0}^{t} \mathrm{~T}(t-s) \mathrm{Ks}(\mathrm{s}) \mathrm{x}$ ds ( $\mathrm{x} \in \mathrm{G}$ ) is the norm limit of Riemann sums, one observes that $(c t)^{-1} \int_{0}^{t} T(t-s) K s(s) x$ ds is an element of the closed convex hull $\overline{\mathrm{CO}} \mathrm{M}$ of M , provided that $\mathrm{C}:=\sup \{\|\mathrm{S}(\mathrm{s})\|: 0 \leq \mathrm{s} \leq \mathrm{t}\}$ and $\|x\| \leqq 1$. Since $\overline{C O} \mathrm{M}$ is compact (see II.4.3 in Schaefer (1966)) the assertion follows.

We will now show that for quasi-compact semigroups one can give a description of the asymptotic behavior similar to the one stated for eventually compact semigroups in Thm.2.1 . One obtains a representation as in (2.1) with a remainder of exponential decay but the rate of the decay cannot be chosen arbitrarily large.

Theorem 2.10. Let $T=(T(t))_{t \geqq 0}$ be a quasi-compact semigroup on a Banach space $G$ with generator $A$. Then $\{\lambda \in \sigma(A): \operatorname{Re} \lambda \geqq 0\}$ is a finite set (possibly empty) and contains only poles of finite algebraic multiplicity. Denoting the eigenvalues with nonnegative real part $\lambda_{1}, \lambda_{2}, \ldots, \lambda_{m}$, the corresponding residues $\mathrm{P}_{1}, \mathrm{P}_{2}, \ldots, \mathrm{P}_{\mathrm{m}}$ and the orders of the poles $k(1), k(2), \ldots, k(m)$ we have\\
(2.5) $\quad T_{n}(t)=\exp \left(\lambda_{n} t\right) \cdot \sum_{j=0}^{k(n)-1} \frac{1}{j}!t^{j}\left(A-\lambda_{n}{ }^{j} \circ P_{n} \quad(t \geqq 0) \quad\right.$ and $\|R(t)\| \leqq c \cdot e^{-\varepsilon t} \quad$ for suitable constants $\varepsilon>0, C \geqq 1$.

Proof. We have wess $(T)<0$ hence $r_{\text {ess }}(T(1))<1$ (see A-III, (3.10)). Therefore $\{z \in \sigma(T(1)):|z| \geqq l\}$ is a finite set and contains only poles of finite algebraic multiplicity (cf. A-III, (3.8)). Let $P$ denote the spectral projection of $T(1)$ corresponding to $\{z \in o(T(1)\}$ : $|z| \geqq 1\}$. Then A-III, Cor. 6.5 implies that $\{\lambda \in \sigma(A): \operatorname{Re} \lambda \geqq 0\}$ is a finite set, it contains only poles of R(.,A) of finite algebraic multiplicity and $\mathrm{P}=\mathrm{P}_{1}+\mathrm{P}_{2}+\ldots+\mathrm{P}_{\mathrm{m}}$. One can now prove the representation of $\mathrm{T}(\mathrm{t})$ stated in (2.5) in the same way as statement (2.1).

In case we consider positive quasi-compact semigroups on $C_{O}(X)$ one can combine Thm.2.10 with the results of B-III. For example, B-III, Cor.2.11 assures that, in case there is at least one eigenvalue with nonnegative real part, the generator has a strictly dominant eigenvalue $r \in \mathbb{R}$. Thus in (2.5) the operators $T_{j}(t)$ belonging to $\lambda_{j}=r$ will determine the long term behavior of (T(t)) . More precisely one has the following.

Corollary 2.11. Let $T=(T(t))_{t \geqq 0}$ be a positive semigroup on $C_{0}(X)$ which is quasi-compact and let $A$ be its generator.\\
(a) Let $r$ be an eigenvalue of $A$ admitting a stricly positive eigenfunction and satisfying $\operatorname{Re} r \geqq 0$. Then $r=\omega(T)=s(A)$ and there is a positive projection P of finite rank such that for\\
suitable constants $\delta>0, \mathrm{M} \geqq 1$ we have\\
(2.6) $\left\|e^{-r t} \cdot T(t)-P\right\| \leqq M \cdot e^{-\delta t}$ for all $t \geqq 0$.\\
(b) In case $(T(t))_{t \geq 0}$ is irreducible and $w(T) \geqq 0$ there exist a strictly positive function $h \in C_{0}(X)$ and a strictly positive bounded measure $v \in M(X)$ such that for suitable constants $\delta>0, M \geqq 1$ one has\\
(2.7) $\|\exp (-\omega(T) t) \cdot T(t)-v \otimes h\| \leqq M \cdot e^{-\delta t}$ for all $t \geqq 0$.

In both cases (a) and (b) the estimates (2.3) for $\|T(t) f\|$ hold true (in case (a) one has to replace $\left|\int \mathrm{fdu}\right|$ by $\|P f\|$ ).

Proof. (a) By B-III, Cor. 2.11 we know that $s(A)$ is a strictly dominant eigenvalue of $A$. By Thm. 2.10 both $s:=s(A)$ and $r$ are poles of the resolvent. Moreover, there exists a positive measure $v$ such that $A^{\prime} V=s V$. Denoting the strictly positive eigenfunction corresponding to $r$ by $h$ we have $\langle h, v\rangle\rangle 0$. Hence $s\langle h, v\rangle=\left\langle h, A^{\prime} v\right\rangle=\langle A h, v\rangle=$ $=r\langle h, v\rangle$ implies $r=s$. By B-III,Rem. 2.15 we know that $s$ is a first order pole of the resolvent. Since $s$ is strictly dominant (2.6) follows from (2.5).

Assertion (b) can be proved in the same way as cor. 2.2 . We omit the details.

Cor.2.11 can be used to describe the asymptotic behavior as $t \rightarrow \infty$ of certain semigroups if only the generators are known. We explain this by discussing a concrete example.

Example 2.12. Let $\mathrm{X}:=[0, \infty)$ and define on $E:=\mathrm{C}_{\mathrm{O}}(\mathrm{x})$ the operator A as follows

$$
\text { Af }:=-f^{\prime}+m f \text { with domain } D(A) \text { given by }
$$

(2.8) $D(A):=\left\{f \in C_{O}(X): f\right.$ is differentiable, $f^{\prime} \in C_{O}(X)$ and $\left.f^{\prime}(0)=\operatorname{af}(0)-\int_{0}^{\infty} f(x) d v(x)\right\}$.

Here $\alpha$ is a real number, $v$ is a bounded positive Borel measure with $v(\{0\})=0$ and $m$ is a continuous function on $x$ such that $m(\infty):=\lim _{x \rightarrow \infty} m(x)$ exists. It is not difficult to see that A generates a positive semigroup. Moreover, one can show that it is quasi-compact if (and only if) $m(\infty)<0$. In order to find eigen-\\
values and eigenfunctions one has to solve the ordinary differential equation $\mathrm{f}^{\prime}=\mathrm{mf}-\lambda \mathrm{f}$. Any solution has (up to a constant) the following form\\
(2.9)

$$
g_{\lambda}(x)=\exp \left\{\int_{0}^{x}(m(y)-\lambda) d y\right\}=e^{-\lambda x} \cdot \exp \left\{\int_{0}^{x} m(y) d y\right\}
$$

We assume that $m(\infty)<0$ and $r \geqq 0$. Then $g_{r}$ is differentiable with $g_{r}, g_{r}^{\prime} \in C_{0}(X)$. Thus $g_{r} \in D(A)$ if and only if\\
$g_{r}^{\prime}(0)=\alpha g(0)-\int_{0}^{\infty} g_{r}(y) d v(y)$. Inserting (2.9) this condition becomes $m(0)-r=\alpha-\int_{0}^{\infty} e^{-r y} \cdot \exp \left\{\int_{0}^{Y} m(z) d z\right\} d v(y)$.\\
By monotonicity this equation has a unique solution $x \geq 0$ if and only if\\
(2.10) $m(0)+\int_{0}^{\infty} \exp \left\{\int_{0}^{y} m(z) d z\right\} d v(y) \geqq \alpha$.

In case $\alpha, v$ and $m$ satisfy $(2.10)$ and $m(\infty)<0$ then $g_{r}$ is a strictly positive eigenfunction of A corresponding to $r \geq 0$, Thus all assumptions of Cor.2.11(a) are satisfied. In addition, the semigroup is irreducible if (and only if) the support of $v$ is an unbounded subset of $[0, \infty)$.

Similar examples will be discussed in B-IV,Sec. 3 and C-IV,Sec. 3 .\\
We finally give a criterion for quasi-compactness of positive semigroups on spaces $C(K)$. It is based on a criterion given by Doeblin for operators on spaces of bounded measurable functions and can be easily deduced from [Lotz (1981), Prop.3].

Proposition 2.13. Let $T=(T(t))_{t \geq 0}$ be a semigroup of Markov operators on $C(K)$, $K$ compact satisfying the following condition.\\
(2.12) There exist $t_{0}>0,0<\mu \in M(K)$ and $\gamma \in \mathbb{R}, 0<\gamma<1$ such that $T\left(t_{0}\right) f-\mu(f) 1_{K} \leqq \gamma \cdot 1_{K} \quad$ for $\quad a 11 \quad 0 \leqq \mathrm{f} \leqq 1_{K}$.\\
Then $T$ is quasi-compact.\\
3. A SEMIGROUP APPROACH TO RETARDED DIFFERENTIAL EQUATIONS\\
by\\
Annette Grabosch and Ulrich Moustakas

The aim of this section is to put into evidence the connection between retarded differential equations and one-parameter semigroups. Special emphasis will lie, as the general theme of this chapter suggests, on positive solutions of such equations and on their asymptotic behavior. Scalar examples were already considered in B-III,Ex.2.14, B-II,Ex. 1.21 , B-II,Ex.1.23, B-II,Ex.2.11 and B-IV,Ex.2.12. In this section we will treat retarded differential equations, also called "delay differential equations", with values in arbitrary Banach spaces. A slight modification of the methods used in the scalar case will work in this setting, too. The main question is whether or how a time delay affects the qualitative behavior of the solution of an abstract Cauchy problem. In particular we will show in Thm.3.7, resp. Cor.3.8 that under certain positivity assumptions the delay has no influence on the stability.

Let $F$ be a Banach space, let $E=C([-1,0], F)$ be the Banach space of all continuous functions on $[-1,0]$ with values in $F$ normed by the supremum norm, and let $\Phi$ be a bounded linear operator from $E$ into F. For $u \in C([-1, \infty), F)$ and $t \geqq 0$ we define the function $u_{t} \in E$ by $u_{t}(s):=u(t+s)$ for all $s \in[-1,0]$. This is the "history" segment of $u$ with length 1 starting at $t-1$. Furthermore, let $B$ be the generator of a strongly continuous semigroup on $F$ such that $B-w$ generates a contraction semigroup for some $w \in \mathbb{R}_{+}$. This additional condition can always be satisfied by renorming the Banach space F (see e.g. [Goldstein (1985a), Thm.2.13]).

Using this framework throughout this section it should be mentioned that in general $E=C([-1,0], F)$ is not a space of type $C(K)$ or even $C_{0}(X)$. Nevertheless, the formal appearence justifies a treatment in this chapter. Moreover, if $F=C(M)$ (M compact) it is well-known that $E$ is isomorphic to $C([-1,0] \times M)$ and thus is a space of type $\mathrm{C}(\mathrm{K})$ (K compact) as well.

With the above notations we consider\\
(RCP)

$$
\dot{u}(t)=B u(t)+\Phi\left(u_{t}\right), t \geqq 0,
$$

$$
u_{0}=g \in E .
$$

We call (RCP) an abstract retarded Cauchy problem.\\
A function $u \in C([-1, \infty), F)$ is a solution of (RCP), if\\
(a) u is right-sided differentiable at 0 and continuously differentiable for $t>0$,\\
(b) $u(t) \in D(B)$ for $t \geqq 0$,\\
(c) (RCP) is satisfied for $t \geqq 0$.

To (RCP) we associate the following operator $A$ on the Banach space E. Let $A$ be the differential operator


\begin{equation*}
\mathrm{Af}:=\mathrm{f}^{\prime} \tag{3.1}
\end{equation*}


$D(A):=\left\{f \in C^{1}([-1,0], F): f(0) \in D(B), f^{\prime}(0)=B f(0)+\Phi f\right\}$.

First we show that $A$ is a generator on $E$.

Theorem 3.1. The operator A defined in (3.1) is the generator of a strongly continuous semigroup $(T(t))_{t \geqq 0}$ on $E$ satisfying the "translation property"

\[
T(t) f(s)=\left\{\begin{array}{ll}
f(t+s) & \text { if } t+s \leqq 0  \tag{T}\\
T(t+s) f(0) & \text { if } t+s>0
\end{array} \quad, f \in E\right.
\]

Proof. We argue as in B-III, Example 2.14.(b) and consider the operator $A_{0} f:=f$, on the domain\\
$D\left(A_{0}\right):=\left\{f \in C^{1}([-1,0], F): \pounds(0) \in D(B), f^{\prime}(0)=B f(0)\right\}$.\\
If $(S(t))_{t \geqq 0}$ is the semigroup on $F$ generated by $B$, then $A_{0}$ generates the semigroup $\left(T_{0}(t)\right)_{t \geqq 0}$ given by

$$
T_{0}(t) f(s)=\left\{\begin{array}{ll}
f(t+s) & \text { if } t+s \leqq 0 \\
s(t+s) f(0) & \text { if } t+s>0
\end{array}, f \in E\right.
$$

For $\lambda>w$ define the map $s_{\lambda} \in L(E)$ by $s_{\lambda} f:=f-\varepsilon_{\lambda} \otimes R(\lambda, B) \Phi f$ where $\varepsilon_{\lambda}(s)=e^{\lambda s}$ and $(h \otimes x)(s):=h(s) \cdot x$ for $h \in c[-1,0], x \in F$ and $s \in[-1,0]$. since $\|R(\lambda, B)\| \leqq(\lambda-w)^{-1}$ it follows that $S_{\lambda}$ is invertible for $\lambda>\|\Phi\|+w$ and that $\left\|s_{\lambda}^{-1}\right\| \leqq(\lambda-w) \cdot(\lambda-\|\Phi\|-w)^{-1}$. Moreover, $s_{\lambda}$ induces a bijection from $D(A)$ onto $D\left(A_{0}\right)$ such that


\begin{gather*}
\lambda-A=\left(\lambda-A_{0}\right) S_{\lambda} \\
R(\lambda, A)=S_{\lambda}^{-1} R\left(\lambda, A_{0}\right) \tag{3.2}
\end{gather*}


Proceeding as in the example mentioned above we obtain

$$
\begin{aligned}
\|R(\lambda, A)\| & \leqq(\lambda-w) \cdot(\lambda-\|\Phi\|-w)^{-1} \cdot(\lambda-w)^{-1} \\
& \leqq(\lambda-\|\Phi\|-w)^{-1} .
\end{aligned}
$$

Thus A is a generator by A-II, Thm,1.7.

It suffices to show the translation property ( $T$ ) for $\mathrm{f} \in \mathrm{D}(\mathrm{A})$ only, To that purpose we treat two cases separately.

\begin{enumerate}
  \item Let $t \geqq 0, s \in[-1,0]$ and $t+s>0$. It suffices to prove $T(-s) g(s)=g(0)$ for $g:=T(t+s) f$. For arbitrary $g \in D(A)$ we define the map
\end{enumerate}

$$
h:[-t, 0] \rightarrow F \quad \text { by } \quad h(r)=\delta_{r} T(-r) g,
$$

where $\delta_{r}$ denotes the point evaluation $f \rightarrow f(r)$ on $E$. For $\theta \neq 0$ we have\\
$1 / \theta \cdot(h(r+\theta)-h(r))=1 / \theta \cdot(T(-r-\theta) g(r+\theta)-T(-r) g(r))$


\begin{equation*}
=1 / \theta \cdot(T(-r-\theta) g(r)-T(-r) g(r)) \tag{1}
\end{equation*}



\begin{equation*}
+1 / \theta \cdot\left(\delta_{r+\theta}-\delta_{r}\right)(T(-x-\theta) g-T(-r) g) \tag{2}
\end{equation*}



\begin{equation*}
+1 / \theta \cdot(T(-r) g(r+\theta)-T(-r) g(r)) \tag{3}
\end{equation*}


As $0 \rightarrow 0$, (1) converges to $-A[T(-r) g](x),(2)$ converges to zero and (3) converges to $A[T(-r) g](r)$. Thus $h$ is continuously differentiable with derivative zero, whence $h(r)=h(0)$ for all $r \in[-t, 0]$. Taking $r=s$ yields $T(-s) g(s)=g(0)$.\\
2. Let $t \geqq 0, s \in[-1,0]$ and $t+s \leqq 0$. As in the first case we show that the map $k:[0, t] \rightarrow F: r \rightarrow[T(r) f](t+s-r)$ is continuously differentiable with derivative zero. Thus $f(t+s)=k(0)=k(t)=T(t) f(s)$.

The translation property (T) enables us to specify the correspondence between the semigroup $(T(t))_{t \geq 0}$ generated by the operator in (3.1) and the solution of the retarded cauchy problem (RCP).

Corollary 3.2. For $g \in D(A)$ define $u:[-1, \infty)+F$ by

$$
u(t):=\quad \begin{cases}g(t) & \text { if }-1 \leqq t \leqq 0 \\ T(t) g(0) & \text { if } 0<t .\end{cases}
$$

Then $u$ is the unique solution of (RCP).

Proof. Evidently $u \in C([-1, \infty), F)$ for $g \in D(A)$. From A-I,Prop.1.6.(iii) and the definition of D(A) we obtain $T(t) g(0)-g(0)=\left[A\left(\int_{0}^{t} T(s) g d s\right)\right](0)$ $=\mathrm{B}\left[\left(\int_{0}^{t} \mathrm{~T}(\mathrm{~s}) \mathrm{g} \mathrm{ds}\right)(0)\right]+\Phi\left(\int_{0}^{t} \mathrm{~T}(\mathrm{~s}) \mathrm{g} \mathrm{ds}\right)$ $=B\left(\int_{0}^{t} T(s) g(0) d s\right)+\int_{0}^{t} \Phi T(s) g d s$ $=B\left(\int_{0}^{t} u(s) d s\right)+\int_{0}^{t} \Phi T(s) g d s$,\\
since $\int_{0}^{t} T(s) g d s \in D(A)$.

Since $u(t)=(T(t) g)(0) \in D(B)$ for $t \geq 0$ the above calculation shows that $u$ is right-sided differentiable at 0 and differentiable for $t>0$; hence

$$
\dot{u}(t)=B u(t)+\Phi(T(t) g) .
$$

By the translation property (T) we have $T(t) g=u_{t}$ indeed\\
$u_{t}(s)=u(t+s)=\left\{\begin{array}{ll}g(t+s) & \text { if } t+s \leq 0 \\ T(t+s) g(0) & \text { if } t+s>0\end{array} \quad=T(t) g(s)\right.$.\\
Therefore $\dot{u}(t)=B u(t)+\Phi\left(u_{t}\right)$, i.e. $u$ solves (RCP).

In order to show uniqueness of the solution we take $w$ to be a solution of (RCP) satisfying $w_{0}=0$. Let $x(t):=w_{t}, t \geq 0$. It is easy to see that $x(t) \in C^{1}([-1,0], F)$; moreover, since $\dot{w}_{t}(0)=\dot{w}(t)$ $=B w(t)+\Phi\left(w_{t}\right)$ we obtain $x(t) \in D(A)$. By the definition of $A$ we have $A x(t)=\dot{w}_{t}$. On the other hand, $x(\cdot) \in C^{1}([0, \infty), E)$ and

$$
\begin{aligned}
(\dot{x}(t))(s) & =1 \lim _{h \rightarrow 0} 1 / h \cdot\left(w_{t+h}(s)-w_{t}(s)\right) \\
& =\lim _{h \rightarrow 0} 1 / h \cdot\left(w_{t}(h+s)-w_{t}(s)\right)=\dot{w}_{t}(s), \text { whence } \dot{x}(t)=\dot{w}_{t}
\end{aligned}
$$

Therefore we obtain $\dot{x}(t)=A x(t)$. As $x(0)=w_{0}=0$ it follows by the well-posedness of the abstract Cauchy problem corresponding to A that $x(t)=0$ for each $t \geqq 0$. This proves $w \equiv 0$.

Remarks 1. By similar arguments the following can be proved. If $u$ is a solution of (RCP) such that $u_{0} \in D(A)$, then $x$ given by $x(t):=$ $u_{t}$ is a solution of the abstract Cauchy problem associated with the operator A defined in (3.1). In this sense, (RCP) and the semigroup generated by A correspond to each other.\\
2. If additionally to the assumptions of Cor.3.2 $B \in L(F)$ then $u$ is a solution of (RCP) for every $g \in E$. [Indeed, a careful inspection shows that the proof of Cor. 3.2 can be generalized to this situation, since $u(t)=(T(t) g)(0) \epsilon F=D(B)$ for all $g \in E$ and $t \geq 0$.\\
3. For general $g \in E$ the retarded Cauchy problem (RCP) may not have a solution. Indeed, if $u$ is a solution of (RCP) then the following is valid for $0 \leqq s \leqq t$ :\\
$\frac{\mathrm{d}}{\mathrm{ds}} \mathrm{S}(t-s) u(s)=-B S(t-s) u(s)+s(t-s) \dot{u}(s)$

$$
=-B S(t-s) u(s)+S(t-s) B u(s)+S(t-s) \Phi\left(u_{s}\right)=S(t-s) \Phi\left(u_{s}\right)
$$

Hence

$$
u(t)-s(t) u(0)=\int_{0}^{t} s(t-s) \Phi\left(u_{s}\right) d s .
$$

Let $(S(t))_{t \geq 0}$ be a stroncly continuous semigroup which is not differentiable (for examples see A-II,1.28). Define $g \in E$ by $g(s):=\tilde{g}$ for all $s \in[-1,0]$ where $\tilde{g} \in F$ is chosen such that\\
$t \rightarrow S(t) \tilde{g}$ is not differentiable in $t^{\prime} \in \mathbb{R}_{+} \cdot$\\
Assume that there exists a solution of (RCP). By the preceding considerations\\
$u(t)=s(t) g(0)+\int_{0}^{t} s(t-s) \Phi\left(u_{s}\right) d s=s(t) \tilde{g}+\int_{0}^{t} s(t-s) \Phi\left(u_{s}\right) d s$.\\
Thus $u$ is not differentiable in $t^{\prime}$ and we have a contradiction.

Corollary 3.3. Keep the above notation and let F be finite dimensional. Then the solution semigroup $(T(t))_{t \geq 0}$ in $E$ corresponding to (RCP) is compact for each $t>1$ and therefore is eventually norm continuous.

Proof. Let $t>1$. By the translation property (T) we have $T(t) f(s)=T(t+s) f(0)$. Whenever $t+s>0$ then Rem. 2 following Cor.3.2 shows that (T(t)f)(s)=(T(t+s)f)(0)=u(t+s) is differentiable with respect to $s \in[-1,0]$ for each $f \in E$.\\
Since $t>1$ we thus have $T(t) f \in C^{1}([-1,0], F)$ for all $f \in E$. The closed graph theorem yields the continuity of $T(t)$ from $E$ into $C^{1}$. Hence $T(t)$ maps the unit ball of $E$ into a bounded set of $C^{1}([-1,0], F)$. Again we use the assertion that $\operatorname{dim} F<\infty$ and obtain by the theorem of Arzela-Ascoli that every bounded set of $C^{1}([-1,0], F)$ is relatively compact in E. Thus $T(t)$ is compact for each $t>1$.

The assertion of cor.3.3 remains true if $(S(t))_{t \geqq 0}$ is a compact semigroup on a (not necessarily finite dimensional) Banach space F (see [Travis-Webb (1974)]).

In order to describe the asymptotic behavior of the solutions of (RCP) it is enough to examine the corresponding semigroup $(T(t))_{t \geqq 0}$ on $E$. Indeed, Cor.3.2 shows that the solutions $u$ are given by $u(t)=$ $T(t) g(0)$ for all $t>0$ and thus the long term behavior of $u$ can be deduced from that one of $(T(t))_{t \geq 0}$. Our approach is based on the characterization of the stability of the semigroup $(T(t))_{t \geq 0}$ by the location of the spectrum $\sigma(A)$ of the generator $A$ as developed in A-IV, Sec.1, B-IV,Sec. 1 and C-IV,Sec. 1 .

We define, for $\lambda \in \mathbb{C}$, operators $\Phi_{\lambda} \in L(F)$ by


\begin{equation*}
\Phi_{\lambda} \mathrm{x}:=\Phi\left(\varepsilon_{\lambda} \otimes \mathrm{x}\right), \mathrm{x} \in \mathrm{~F} . \tag{3.3}
\end{equation*}


since $\Phi_{\lambda}$ is bounded the operator $B+\Phi_{\lambda}$ is a generator on $F$. The spectrum of $A$ can now be characterized in terms of the spectrum of the operators $B+\Phi_{\lambda}$.

Proposition 3.4. Take the operators A, B and as above. For every $\lambda \in \mathbb{C}$ the following equivalence holds:


\begin{equation*}
\lambda \in \sigma(A) \quad \text { if and only if } \quad \lambda \in \sigma\left(B+\Phi_{\lambda}\right) . \tag{3.4}
\end{equation*}


Proof. By definition, $\lambda \in \rho(A)$ if and only if for every $g \in \mathrm{E}$ there exists a unique $f \in D^{\prime}(A)$ such that $\lambda f-f^{\prime}=g$. This equality is satisfied if and only if there exists $x \in F$ such that

$$
f(t)=\int_{t}^{0} e^{\lambda(t-s)} g(s) d s+e^{\lambda t} \cdot x \text { for }-1 \leqq t \leqq 0
$$

On the other hand $f \in D(A)$ if and only if $x \in D(B)$ and $\lambda x-g(0)$ $=B \mathrm{x}+\Phi \mathrm{H}_{\lambda} \mathrm{g}+\Phi_{\lambda} \mathrm{x}$ where $\mathrm{H}_{\lambda} g(t):=\int_{t}^{0} \mathrm{e}^{\lambda(\mathrm{t}-\mathrm{s})} \mathrm{g}(\mathrm{s}) \mathrm{ds}$.\\
Thus $\lambda \in \rho(A)$ if and only if for every $g \in E$ there exists a unique $x \in D(B)$ such that $\left(\lambda-B-\Phi_{\lambda}\right) x=g(0)+\Phi H_{\lambda} g$. Notice that the map $\mathrm{x}+\mathrm{x}+\Phi \mathrm{H}_{\lambda}\left(\varepsilon_{\mu} \otimes \mathrm{x}\right) \quad(\mathrm{x} \in \mathrm{F})$ is surjective on F if $\mu$ is chosen so large that $\left\|\Phi H_{\lambda}\left(\varepsilon_{\mu} \otimes x\right)\right\| \leq 1 / 2 \cdot\|x\|$ for all $x \in F$. Hence the map $g \rightarrow$ $g(0)+\Phi H_{\lambda} g$ is surjective from $E$ onto $F$ and this shows that $\lambda \in$ $\rho(A)$ if and only if $\lambda-B-\Phi_{\lambda}$ is invertible.

An immediate consequence of the proof is the following corollary.

Corollary. With the notations of the above proposition and $A_{0}$ as in the proof of Thm.3.1 we have:\\
(a) $R(\lambda, A) g=\varepsilon_{\lambda} \otimes R\left(\lambda, B+\Phi_{\lambda}\right)\left(g(0)+\Phi H_{\lambda} g\right)+H_{\lambda} g$ for $\lambda \in \rho(A), g \in E$.\\
(b) $R\left(\lambda, A_{0}\right) g=\varepsilon_{\lambda} 8 R(\lambda, B) g(0)+H_{\lambda} g$ for $\lambda \in \rho\left(A_{0}\right), g \in E$.

We now turn to the aspect of positivity in (RCP) and its impact on the asymptotic behavior of the solution semigroup $(T(t))_{t \geq 0}$. To this end we let $F$ be a Banach lattice which makes $E=C([-1,0], F)$ into a Banach lattice as well. Furthermore, let $(S(t))_{t \geqq 0}$ be a positive semigroup with generator $B$ and let $\Phi \in L(E, F)$ be a positive operator. As before we restrict our attention to the case that $B$ - w generates a positive contraction semigroup for some $w \in \mathbb{R}$. Indeed, if $B$ generates a bounded positive semigroup $(S(t))_{t \geq 0}$ on $F$, then $\|x\|:=\sup _{t \geq 0}\|s(t)|x|\|$ for $x \in F$ defines an equivalent lattice norm on $F$, for which $(S(t))_{t \geq 0}$ is contractive.

Proposition 3.5. If $\Phi \in L(E, F)$ is a positive operator and if B generates a positive semigroup on $F$, then the semigroup $(T(t))_{t \geq 0}$ on $E$ generated by $A f:=f^{\prime}$ with domain $D(A):=\left\{f \in C^{l}: f(0) \in\right.$ $\left.D(B), f^{\prime}(0)=B f(0)+\Phi \mathrm{f}\right\}$ is positive.

Proof. By Formula (3.2) we have $R(\lambda, A)=s_{\lambda}^{-1} \mathrm{R}\left(\lambda, A_{0}\right)$ for $\lambda>\|\Phi\|+w$, (where $S_{\lambda} f=f-\varepsilon_{\lambda} \otimes R(\lambda, B) \Phi f$ for $f \in E$ ). Thus the fact that $R\left(\lambda, A_{0}\right)$ is positive (C-II,Prop.4.1) reduces the problem to showing that $S_{\lambda}^{-1}$ is a positive operator for $\lambda>\|\Phi\|+w$.\\
Since $S_{\lambda}=I d-\varepsilon_{\lambda} \otimes R(\lambda, B) \Phi$ and $\left\|\varepsilon_{\lambda} \otimes R(\lambda, B) \Phi\right\| \leqq(\lambda-w)^{-1} \cdot\|\Phi\|<1$ we see that $s_{\lambda}^{-1}=\sum_{n=0}^{\infty}\left(\varepsilon_{\lambda} \oslash R(\lambda, B) \Phi^{n} \quad\right.$ is positive. Hence $(T(t))_{t \geq 0}$ is a positive semigroup again by C-II,Prop.4.1.

Remark. Suppose that $\Phi$ has no mass in zero (i.e., for every $\varepsilon>0$ there exists $\delta>0$ such that $\|\Phi \mathrm{f}\| \varepsilon\|f\|$ for all $\mathrm{f} \in \mathrm{E}, \operatorname{supp(f)} \subset$ $[-\delta, 0])$. Then the positivity hypotheses in the above proposition are necessary in order to obtain positivity of (T(t)) $t \geqq 0$ (cf. B-II,1.22 for the case $\operatorname{dim} F<\infty$ and [Kerscher (1986)] for the general case).

Proposition 3.6. Let $\Phi \in L(E, F)$ be positive and assume that $B$ generates a positive semigroup on $F$. The "spectral bound function" $\lambda \rightarrow s\left(B+\phi_{\lambda}\right)$ is decreasing and continuous from the left on $\mathbb{R}$. If, additionally, $B$ has compact resolvent and there exists $\lambda^{\prime} \in \mathbb{R}$ with $\sigma\left(B+\Phi_{\lambda^{\prime}}\right) \neq \varnothing$, then $\lambda \rightarrow s\left(B+\Phi_{\lambda}\right)$ is continuous and the spectral bound $s(A)$ is the unique solution of the equation


\begin{equation*}
\lambda=s\left(B+\Phi_{\lambda}\right) \tag{3.5}
\end{equation*}


Proof (cf. also C-IV,Lemma 3.4). For $\lambda \leqq \mu$ we have $0 \leqq \Phi_{\mu} \leqq \Phi_{\lambda}$ and hence $0 \leqq R_{\mu}(t) \leqq R_{\lambda}(t), t \geqq 0$, for the respective semigroups generated by $B+\Phi_{\mu}$ and $B+\Phi_{\lambda}$ (see $A-I I, \sec .1$ ). This implies $s\left(B+\Phi_{\mu}\right) \leqq s\left(B+\Phi_{\lambda}\right)$. The left-continuity follows by the semicontinuity of the spectrum (see [Kato (1976) Chap.IV, Thm.3.1]).\\
If $B$ has compact resolvent then $B+\Phi_{\lambda}$ has compact resolvent as well. Now C-III,Thm.1.1.(a) shows that $s\left(B+\Phi_{\lambda}\right)$ belongs to $\sigma\left(B+\Phi_{\lambda}\right)$ and, by $A-I I I, 3.6$ is a pole with residue of finite rank. This completes the proof since spectral points of compact operators depend continuously on smooth perturbations (see [Dunford-Schwartz (1958), VII, 6.Thm.9]).

If $\sigma(B) \neq \varnothing$, then $-\infty<s(B) \leqq s\left(B+\Phi_{\lambda}\right)$ for all $\lambda \in \mathbb{R}$ which implies $\sigma\left(B+\Phi_{\lambda}\right) \neq \varnothing$. On the other hand, if $\sigma\left(B+\Phi_{\lambda}\right)=\varnothing$ for all $\lambda \in \mathbb{R}$ then $\sigma(A)=\emptyset$ by Prop.3.4.

We are now able to characterize the spectral bound of the generator A in E through spectral bounds of generators in F .

Theorem 3.7. Let $\Phi \in L(E, F)$ be positive and let $B$ be the generator of a positive semigroup on F . The following implications are valid :\\
(a) If $s\left(B+\Phi_{\lambda}\right)<\lambda$, then $s(A)<\lambda$.\\
(b) If $s\left(B+\Phi_{\lambda}\right)=\lambda$, then $s(A)=\lambda$.\\
(c) Suppose that $B$ has compact resolvent and there exist $\lambda^{\prime} \in \mathbb{R}$ with $\sigma\left(B+\Phi_{\lambda^{\prime}}\right) \neq \varnothing$. Then\\
(3.6) $\quad \mathrm{s}\left(\mathrm{B}+\Phi_{\lambda}\right) \leqq \lambda$ if and only if $s(A) \leqq \lambda$.

Proof. (a) If $\lambda>s\left(B+\Phi_{\lambda}\right)$, then $\mu>s\left(B+\Phi_{\mu}\right)$ for all $\mu \geq \lambda$ by Prop. 3.6 . Therefore, $\mu \in \rho\left(B+\Phi_{\mu}\right)$ for all $\mu \geqq \lambda$. By Prop. 3.4 this implies $\mu \in \rho(A)$ for all $\mu \geqq \lambda$. Since $s(A) \epsilon \sigma(A)$ by [C-III, Thm.1.1.(a)] we obtain $\lambda>s(A)$.\\
(b) If $\lambda=s\left(B+\Phi_{\lambda}\right)$, then again $\lambda \in \sigma\left(B+\Phi_{\lambda}\right)$ whence we obtain from Prop.3.4 $\lambda \in \sigma(A)$ and therefore $\lambda \leq s(A)$. In the same way as in (a) we conclude that $\mu \in \rho(A)$ if $\mu>\lambda$; hence $\lambda=s(A)$.\\
(c) It suffices to prove that $s(A)>\lambda$ whenever $s\left(B+\Phi_{\lambda}\right)>\lambda$.

Assume the latter inequality. According to Prop.3.6 there exists a unique $\mu$ satisfying $\mu=\mathrm{s}\left(B+\Phi_{\mu}\right)$. Still by Prop. 3.6 it follows that $\lambda<\mu$. Assertion (b) now completes the proof.

Remark. We call (3.5) the generalized characteristic equation corresponding to (RCP). A justification for this terminology will be given in a remark following cor.3.8 of chapter C-IV.

The characterization (3.6) of $s(A)$ uses the continuity of $\lambda \rightarrow s(B$ $+\Phi_{\lambda}$ ). In the general case we apply the following lemma which is due to W. Arendt.

Lemma, Let $\Phi \in L(E, F)$ be positive and assume that $B$ generates a positive semigroup on $F$. If we define\\
$\mu:= \begin{cases}\sup \{\lambda \in \mathbb{R}: \operatorname{s}(B+\Phi \\ -\infty & \text { if } \sigma\left(B+\Phi_{\lambda}\right) \neq \varnothing \text { for some } \lambda \in \mathbb{R}, \\ \text { otherwise, }\end{cases}$ then $s(A)=\mu$.

Proof. If $\sigma\left(B+\Phi_{\lambda}\right)=\emptyset$ for all $\lambda \in \mathbb{R}$ then $\sigma(A)=\emptyset$ by Prop.3.4 and there is nothing to prove.\\
Take now $\lambda \in \mathbb{R}$ with $\sigma\left(B+\Phi_{\lambda}\right) \neq \varnothing$ and show $\mu \in \sigma\left(B+\Phi_{\mu}\right)$.\\
Case 1: If $\mu=s\left(B+\Phi_{\mu}\right)$ then $\mu \in \sigma\left(B+\Phi_{\mu}\right)$ by C-III,Thm.1.1.\\
Case 2: If $\mu<s\left(B+\Phi_{\mu}\right)$ we show $r \in \sigma\left(B+\Phi_{\mu}\right)$ for every $r \in$ $\left(\mu, s\left(B+\Phi{ }_{\mu}\right)\right]$.

Let $r \in\left(\mu, s\left(B+\Phi_{\mu}\right)\right]$ and assume $r \in \rho\left(B+\Phi_{\mu}\right)$. BY the definition of $\mu$ we have $r \in \rho\left(B+\Phi{ }_{\mu+\varepsilon}\right)$ for all $\varepsilon>0$. By C-III,Thm.1.1 $R\left(r, B+\Phi_{\mu+\varepsilon}\right) \geqq 0$ and by the assumption $R\left(r, B+\Phi{ }_{\mu}\right) \geqq 0$ as well. Now C-III, Thm. 1.1 implies $r>s\left(B+\Phi_{\mu}\right)$ which yields a contradiction to the choice of $r$. Thus $r \in \sigma\left(B+\Phi_{\mu}\right)$ for every $r \in\left(\mu, s\left(B+\Phi_{\mu}\right)\right]$ and hence $\mu \in \sigma\left(B+\Phi_{\mu}\right)$. Consequently $s(A) \geq \mu$. Finally we assume $s(A)>\mu$. The definition of $\mu$ yields $s(A)>$ $s\left(B+\Phi_{s}(A)^{\prime}\right.$. Hence $s(A) \in \rho\left(B+\Phi_{s}(A){ }^{\prime}\right.$ and thus $s(A) \in \rho(A)$ by Prop. 3.4 . This yields a contradiction, since A generates a positive semigroup, hence $s(A)=\mu$.

An immediate consequence of the preceding lemma is the following stability criterion.

Corollary 3.8. Let $\Phi \in L(E, F)$ be positive and let $B$ be the generator of a positive semigroup. The following assertions are equivalent:\\
(i) The semigroup generated by $A$ is exponentially stable in $E$.\\
(ii) The semigroup generated by $\mathrm{B}+\Phi_{0}$ is exponentially stable in F .

Proof. We can assume that there exists $\lambda \in \mathbb{R}$ with $\sigma\left(B+\Phi{ }_{\lambda}\right) \neq \varnothing$. The implication "(i) + (ii)" follows immediately from Thm. 3.7.(a). It remains to show "(ii) $\rightarrow$ (i)". Let $s\left(B+\sigma_{0}\right)<0$. By the lemma and since $\lambda \rightarrow s\left(B+\Phi_{\lambda}\right)$ is non-increasing we have $s(A)=\mu=\sup \{\lambda$ : $\left.\mathbf{s}\left(\mathrm{B}+\Phi_{\lambda}\right)>\lambda\right\}<0$. Thus the semigroup generated by A is exponentially stable.

Remark. In the situation of Thm.3.7(c) we have the stronger result that $s(A)$ and $s\left(B+\Phi_{0}\right)$ have the same sign.

Example 3.9 (see also $C-I I$, Ex. 4. 14). Take $E=C([-1,0], \mathbb{C}), \alpha \in C$ and $\mu \in M[-1,0]_{+}$such that $\mu(\{0\})=0$. Then the operator $A$ given by $A f=f^{\prime}$ on $D(A)=\left\{f \in C^{1}([-1,0], C): f^{\prime}(0)=\alpha f(0)+<f, \mu>\right\}$ generates a strongly continuous semigroup $(T(t))_{t \geqq 0}$. In fact, this follows from Thm.3.1 if we put $F=C, \phi=\mu$ and $B$ the multiplication by $\alpha$. Moreover $\Phi_{0}$ is the multiplication by $\left\langle\varepsilon_{0}, \Phi\right\rangle=\|\Phi\|$ (notice $\Phi \geq 0$ ) and $s\left(B+\Phi_{0}\right)=\alpha+\|\Phi\|$. Since $\omega(A)=s(A)$ by $B-I V,(1.1)$ we obtain from Cor.3.8 that A generates a uniformly exponentially stable semigroup if and only if $\alpha+\|\Phi\|<0$.

The preceding considerations remain true if we consider an (arbitrary) finite time delay $\tau$ where $0<\tau<\infty$. Clearly, (RCP) can be treated as an differential equation with corresponding generator A (see (3.1) for the definition) in $C([-\tau, 0], F)$ (instead of $C([-1,0], F)$ ).

Example 3.10. In order to illustrate the consequences of Cor. 3.8 we consider the Cauchy problem


\begin{align*}
& \dot{u}(t)=\mathrm{Bu}(t)+\mathrm{Su}(t-\tau), t \geqq 0,  \tag{3.7}\\
& u(t)=\psi(t),-\tau \leqq t \leqq 0(0<\tau<\infty), \psi \in E,
\end{align*}


where $B$ is the generator of a positive semigroup on $F, \sigma(B) \neq \varnothing$ and $S \in L(F)$ is positive.\\
Using the above terminology, we have $\Phi \mathrm{f}=\mathrm{S}(\mathrm{f}(-\mathrm{r}))$ for all $\mathrm{f} \in \mathrm{E}$, hence $\Phi_{0}=\mathrm{S}$. By Cor. 3.8 the solution semigroup corresponding to the retarded differential equation (3.7) is exponentially stable if and only if the semigroup generated by $\mathrm{B}+\mathrm{S}$ is exponentially stable. But the semigroup generated by $B+S$ is the solution semigroup of the "undelayed" Cauchy problem

\[
\begin{array}{ll}
\dot{u}(t)=B u(t)+S u(t), & t \geqq 0,  \tag{3.8}\\
u(0)=x, & x \in F .
\end{array}
\]

More precisely, we obtain the following corollary.\\
Corollary. The solution of (3.7) is exponentially stable for every $\tau>0$ if and only if the solution of (3.8) is exponentially stable.

In other words, the corollary states that for this "positive-type" delay equations $\left((S(t))_{t \geqq 0}\right.$ and $\Phi$ positive) exponential stability is independent of the delay (see [Kerscher (1986)] for a detailed analysis of this phenomenon).\\
This is a rather untypical behavior since even a scalar valued delay differential equation may be stable for "small" delays but unstable for "large" delays.\\
We give an example and show how a stable Cauchy problem with nonpositive solutions (see the remark following Prop.3.5) can be destabilized by an increase of the time lag $\tau$.\\
Let $0<\tau<\infty$ and $p, q \in \mathbb{R}$ and consider the following ( $R C P$ ) :\\
$\dot{u}(t)=\mathrm{pu}(\mathrm{t})+\mathrm{qu}(\mathrm{t}-\tau), \quad t \geqq 0$,\\
(3.9) ${ }_{\tau}$\\
$u(t)=\Psi(t), \quad-\tau \leqq t \leqq 0, \Psi \in C[-\tau, 0]$.

The characteristic equation (in the classical sense) is:


\begin{equation*}
\lambda=p+e^{-\lambda \tau} q \tag{3.10}
\end{equation*}


We consider the case where the Cauchy problem without delay

$$
\dot{u}(t)=(p+q) u(t)
$$

is asymptotically stable, i.e. we choose $0<p<1$ and $q+p<0$.\\
Claim. For every $0<\lambda^{\prime}<\mathrm{p}$ there exists $\tau>0$ such that $e^{\lambda^{\prime} t}$ is a solution of (3.9) .\\
Consider the map $g: \mathbb{R} \times\left(\mathbb{R}_{+} \backslash\{0\}\right) \rightarrow \mathbb{R}$ defined by $g(\lambda, \tau)=p+e^{-\lambda \tau} q$. This function is continuous in $\lambda$ and $\tau$ and increasing in $\lambda$. Furthermore $g(0, \tau)=p+q<0$ for every $\tau>0$ and $g(\lambda, \tau) \rightarrow p$ as $\tau \rightarrow \infty$ for every $\lambda \in \mathbb{R}_{+}$. For $0<\lambda^{\prime}<\mathrm{p}$ fixed we can find $\tau>0$ such that $g\left(\lambda^{\prime}, \tau\right)=\lambda^{\prime}$.\\
Let $\psi(t)=e^{\lambda^{\prime} t}$ for $-\tau \leqq t \leqq 0$. If we define $u(t):=e^{\lambda^{\prime} t}$ for $t \geq 0$ then the following is valid:\\
$p u(t)+q u(t-\tau)=p e^{\lambda^{\prime}} t+q e^{\lambda^{\prime}} t e^{-\lambda^{\prime} \tau}=\left(p+q e^{-\lambda^{\prime} \tau}\right) e^{\lambda^{\prime} t}=\lambda^{\prime} e^{\lambda^{\prime} t}=\dot{u}(t)$. Thus $u$ is a solution of $(3.9)_{\tau}$ which is exponentially increasing as $t \rightarrow \infty$. In particular (3.9) is not stable.\\[0pt]
The precise region of stability in the scalar valued case is given, for example in [Hadeler (2978)] and [Hale (1977),107ff].

Remark. Consider the case $F=C(M)$ (M compact). Then $E=C([-1,0] \times M)$ and $(T(t))_{t \geq 0}$ is a positive semigroup on $C(K)$ where $K=[-1,0] \times M$ is compact. Thus spectral bound and growth bound of the semigroup generator coincide (B-IV,(1.1)). This yields a statement analogous to Cor.3.8 for uniform exponential stability.

We conclude this section with two examples fitting into the above framework.

Example 3.11. Consider the equation

$$
\frac{\partial}{\partial t} u(t, x)=\frac{\partial^{2}}{\partial^{2} x} u(t, x)-d(x) u(t, x), b(x) u(t-1, x) \quad(t \geqq 0, x \in[0,1])
$$

with boundary condition\\
(3.11) $\left.\frac{\partial}{\partial x} u(t, x)\right|_{x=0}=0=\left.\frac{\partial}{\partial x} u(t, x)\right|_{x=1}$\\
and initial condition

$$
u(s, x)=\psi(s, x) \quad(s \in[-1,0], x \in[0,1])
$$

Let $F=C[0,1], E=C([-1,0] \times[0,1])$ and let $\tilde{B}$ be defined by $\tilde{B} h=h^{\prime \prime}$ with domain $D(\tilde{B}):=\left\{h \in C^{2}[0,1]: h^{\prime}(0)=h^{\prime}(1)=0\right\}$. Denote by $M_{b}$ and $M_{d}$ the respective multiplication operators for $0 \leqq \mathrm{~b}, \mathrm{~d} \in \mathrm{~F}$. Then (3.11) takes the abstract form

$$
\begin{aligned}
\dot{u}(t) & =\tilde{B} u(t)-M_{d} u(t)+M_{b} u(t-1) \\
u_{0} & =\psi \in E .
\end{aligned}
$$

It is well-known that $\tilde{B}$ generates a positive contraction semigroup and has compact resolvent (see $A-1,2,7$ ). The same is true for the operator $B:=\tilde{B}-M_{d}$ (see $A-I I, T h m .1 .29$ and Thm.1.30). Thus by the above results the solution semigroup of (3.11) is positive and its asymptotic behavior can be investigated by the "undelayed" equation

$$
\dot{u}(t)=\left(\tilde{B}+M_{h}\right) u(t) \text {, where } h:=b-d .
$$

Let $h(x)<0$ for all $x \in[0,1]$.\\
Then $s\left(\tilde{B}+M_{h}\right) \leqq \max \{h(x): x \in[0,1])<0$. Hence the solutions of (3.11) are uniformly exponentially stable.

Interpretation. The solution $u$ of (3.11) can be interpreted as the density of a population, distributed over an "area" [0,1].\\
The operator $\frac{\partial^{2}}{\partial^{2} x}$ describes the internal migration of the population and the functions b and $d$ are the "place specific" birth- resp. death rate of the population members. The time delay 1 stands for the gestation period. The stability condition $h(x)<0$ for all $x \in$ $[0,1]$ means that the death rate has to majorize the birth rate in each spatial point to lead to extinction of the population, no matter whether the equation with or without delay is considered.

Example 3.12. An interesting example from cell biology is given by Gyllenberg-Heijmans (1985). They investigate a balance equation for the size distribution of a cell population which is structured by size. To point out the main ideas we will restrict the complex situation to the simple case of linear cell growth and refer to the original paper for details and the more general case.\\
Let $0<r<1$ and let $a=r$ be the minimal cell size. Furthermore let $F=L^{1}([\alpha, 1])$ and $E=C([-r, 0], F)$. The retarded differential equation of interest is the following.


\begin{align*}
\frac{d}{d t} u(t) & =B u(t)+L u(t-r)  \tag{3.12}\\
u & =\Psi \in E
\end{align*}


Here $B f:=-f$, on $D(B)=\left\{f \in L^{1}[\alpha, 1]: f \in A C[\alpha, 1], f(\alpha)=0\right\}$ and $\mathrm{L}: \mathbf{F} \rightarrow \mathrm{F}$ is defined by

$$
L f(x)= \begin{cases}k(x) f(2 x-r) & \text { if } x \in[\alpha, 1 / 2(r+1)] \\ 0 & \text { if } x \in(1 / 2(r+1), 1]\end{cases}
$$

where $k \in C[\alpha, 1]$.\\
It is easy to verify that L is positive and bounded, and that B is the generator of the positive semigroup $(S(t))_{t \geq 0}$ defined by

$$
[S(t) f](x)=\left\{\begin{array}{ll}
f(x-t) & \text { if } x-t \geqq \alpha \\
0 & \text { if } x-t<\alpha
\end{array} \quad(x \in[\alpha, 1]) .\right.
$$

Furthermore $B$ has compact resolvent. Define $\Phi \mathrm{f}:=\mathrm{L}(\mathrm{f}(-r))$ for $\mathrm{f} \in \mathrm{E}$ such that (3.12) can be written as retarded Cauchy problem (RCP).

As before (see Formula (3.3)) $\Phi_{\lambda}$ is defined by $\Phi_{\lambda} x:=\Phi\left(\varepsilon_{\lambda}{ }_{x}\right)$ for $x \in F$. Gyllenberg and Heijmans have shown that $s\left(B+\Phi_{\lambda}\right)>-\infty$. Thus we can apply Thm.3.7 and obtain that $s(A)=\lambda$ if and only if $\lambda=$ $\mathrm{s}\left(\mathrm{B}+\Phi_{\lambda}\right)$.

NOTES.\\
Section 1. The coincidence of spectral bound and growth bound for positive semigroups on $C(K)$ was first observed by Derendinger (1980) and then generalized to C (X) and non-commutative C*-algebras by Batty-Davies (1982) and Groh-Neubrander (1981). The stability theorem 1.1 is a continuous version of a result of ChoquetFoias (see Schaefer (1974), V.8.8).

Section 2. For the Riesz-Schauder Theory of compact operators we refer to DunfordSchwartz (1958), Sec.VII. 4 and Pietsch (1978), Sec. 26 . Theorem 1.1 seems to be folklore. Prop 2.3 is due to Grothendieck (1953) and can be found in Sec. II. 9 of Schaefer (1974). Proposition 2.4 is due to Dieudonné (see \$3 of Grothendieck (1953) and Schaefer (1974), II, Exc. 27). The notion 'strong Fieller property' used in Theorem 2.5 is due to Girsanov (see Dynkin (1965)) while the theorem itself was proven by Davies (1982). It is well known that there is a close relationship between Markov processes and Markov semigroups. A description of this relation more detailed than Example 2.6 can be found e.g. in Dynkin (1965), in Chap. 2 of van Casteren (1985) or in Chap. 7 of Lamperti (1977). The notion "quasi-compact" for a single operator dates back to Eberlein (1949) (see also Yosida-Kakutani (1941) and Sec.26.4 of Pietsch (1978)). Quasi-compactness for strongly continuous semigroups and its relationship to uniform ergodicity is investigated in Lin (1975). Proposition 2.9 is due to Voigt (1980), a special case was proven by Vidav (1970). Corollaries 2.2 and 2.11 can be found in Greiner (1984). The criterion stated in (2.12) is known as 'Doeblin's condition' (see e.g. Yosida-Kakutani (1941)). It is sufficient and\\
necessary for quasi-compactness of the semigroup. A new proof of this result is given in Lotz (1981).

Section 3. The standard reference to retarded differential equations is Hale (1977), where it is shown that the solutions of (RCP), with values in a finite dimensional space $F$, yield an operator semigroup. The extension to arbitrary Banach spaces F was first made by Travis-Webb (1974). Plant (1977) showed the translation property ( T ) for the solution semigroup. Among the many papers pursuing this functional analytic investigation of partial differential equations with delay we quote DiBlasio-Kunisch-Sinestrari (1984) and Kunisch-Schappacher (1983).\\
Our approach is essentially due to W. Kerscher. We show that the first derivative with an appropriate domain is the generator of a one-parameter semigroup on an abstract function space. Due to the translation property this semigroup yields the solutions of (RCP).\\
The aspect of positivity in (RCP) and its influence on the stability of the solutions was first studied in Section 4 of Kerscher-Nagel (1984). In Kerscher (1986) this is pursued by showing how Theorem 3.7 in combination with the domination of semigroups (see C-II, Section 4) can be used to derive many of the known "stability independent of the delay" - results (e.g., Cooke-Ferreira (1983)).

\section*{POSITIVE SEMIGROUPS ON BANACH LATTICES }
\section*{CHAPTER C-I}
\section*{BASTC RESULTS ON B A NAC H L A T T I C E S }
\section*{AND POSITIVEOPERATORS}
by\\
Rainer Nagel and Ulf Schlotterbeck

This introductory chapter is intended to give a brief exposition of those results on Banach lattices and ordered Banach spaces which are indispensable for an understanding of the subsequent chapters. We do not want to give proofs of the results we are going to present, since these can easily be found in the literature (e.g., in schaefer 1974). We rather want to give the reader who is unfamiliar with these results or with the terminology used in this book the necessary information for an intelligent reading of the main discussions. Since relatively few general results on ordered Banach spaces are needed, we will primarily talk about Banach lattices. The scalar field will be $\mathbb{R}$ except for the last three sections, where complex Banach lattices will be discussed.

The notion of a Banach lattice was devised to get a common abstract setting within which one could talk about phenomena related to positivity that had previously been studied in various types of spaces of real-valued functions, such as the spaces $C(K)$ of continuous functions on a compact topological space $K$, the Lebesgue spaces $L^{1}(\mu)$ or more generally the spaces $I^{P}(\mu)$ constructed over a measure space $(\mathrm{X}, \Sigma, \mu)$ for $1 \leqq \mathrm{p} \leqq \infty$. Thus it is a good idea to think of such spaces first in order to get a feeling for the concrete meaning of the abstract notions we are going to introduce. Later we will see that the connections between abstract Banach lattices and the "concrete" function lattices $C(K)$ and $L^{1}(\mu)$ are closer than one might think at first. We will use without further explanation the terms order relation (ordering), ordered set, majorant, minorant, supremum, infimum.

By definition, a Banach lattice is a Banach space (E,| |) which is endowed with an order relation, usually written $\leqq$, such that ( $\mathrm{E}, \leqq$ ) is a lattice and the ordering is compatible with the Banach space structure of E. We are going to elaborate this in more detail now.

The axioms of compatibility between the linear structure of $E$ and the order are as follows:\\
$\left(\mathrm{LO}_{1}\right) \quad \mathrm{f} \leqq \mathrm{g}$ implies $\mathrm{f}+\mathrm{h} \leqq \mathrm{g}+\mathrm{h}$ for all $\mathrm{f}, \mathrm{g}, \mathrm{h}$ in E.\\
$\left(\mathrm{LO}_{2}\right) \quad \mathrm{f} \geqq 0$ implies $\lambda \mathrm{f} \geqq 0$ for all f in E and $\lambda \geqq 0$.

Any (real) vector space with an ordering satisfying ( $\mathrm{LO}_{1}$ ) and ( $\mathrm{LO}_{2}$ ) is called an ordered vector space. The property expressed in ( $\mathrm{LO}_{1}$ ) is sometimes called translation invariance and implies that the ordering of an ordered vector space E is completely determined by the positive part $E_{+}=\{f \in E: f \geqq 0\}$ of $E$. In fact, one has $f \leq g$ if and only if $g-\mathrm{f} \in \mathrm{E}_{+} \cdot\left(\mathrm{LO}_{1}\right)$ together with $\left(\mathrm{LO}_{2}\right)$ furthermore imply that the positive part of E is a convex set and a cone with vertex 0 (often called the positive cone of $E$ ). It is easily verified that conversely any proper convex cone $C$ with vertex 0 in $E$ is the positive part of $E$ with respect to a uniquely determined compatible ordering.

An ordered vector space $E$ is called a vector lattice if any two elements $f, g$ in $E$ have a supremum, which is denoted by $\sup (f, g)$ or by $\mathrm{f}_{\vee} \mathrm{g}$, and an infimum, denoted by inf(f,g) or by fag. It is obvious that the existence of, e.g., the supremum of any two elements in an ordered vector space implies the existence of the supremum of any finite number of elements in $E$ and, since $f \leqslant g$ is equivalent to -g $\leqq-\mathrm{f}$ this automatically implies the existence of finite infima. However, suprema (infima) of infinite majorized subsets need not exist in a vector lattice. If they do, then the vector lattice is called order complete (countably order complete or o-order complete if suprema of countable majorized subsets exist). At any rate, the binary relations sup and inf in a vector lattice automatically satisfy the (infinite) distributive laws

$$
\begin{aligned}
& \inf \left(\sup _{\alpha} f_{\alpha}, h\right)=\sup _{\alpha}\left(\inf \left(f_{\alpha}, h\right)\right) \\
& \sup \left(\inf f_{\alpha} f_{\alpha}, h\right)=\inf f_{\alpha}\left(\sup \left(f_{\alpha}, h\right)\right)
\end{aligned}
$$

whenever one side exists and give rise to the following definitions:

$$
\begin{aligned}
& \sup (f,-f)=|f| \text { is called the absolute value of } f \\
& \sup (f, 0)=f^{+} \text {is called the positive part of } f \\
& \sup (-f, 0)=f^{-} \text {is called the negative part of } f
\end{aligned}
$$

Note that the negative part of $f$ is positive.

We call two elements $f, g$ of a vector lattice orthogonal or lattice disjoint and write $\mathrm{f}^{\perp} \mathrm{g}$, if $\inf (|\mathrm{f}|,|\mathrm{g}|)=0$. Apart from this, the above definitions allow us to formulate the axiom of compatibility between norm and order requested in a Banach lattice in the following short way:\\
(LN) $\quad|f| \leqq|g|$ implies $\|f\| \leq\|g\|$.

A norm on a vector lattice is called a lattice norm, if it satisfies (LN), and with these notations we can now give the definition of a Banach lattice as follows: A Banach lattice is a Banach space E endowed with an ordering $\leq$ such that ( $\mathrm{E}, \leq$ ) is a vector lattice and the norm on E is a lattice norm. By a normed vector lattice we understand a vector lattice endowed with a lattice norm.

There is a number of elementary, but very important formulas valid in any vector lattice, such as

$$
\begin{array}{rlr}
\mathbf{f} & =\mathbf{f}^{+}-\mathbf{f}^{-} & |f+g| \leqq|f|+|g| \\
|f| & =\mathbf{f}^{+}+\mathbf{f}^{-} & f+g=\sup (f, g)+\inf (f, g)
\end{array}
$$

etc.

Let us note in passing the following consequences:\\
(i) The lattice operations (f,g) $\rightarrow \sup (f, g)$ and (f,g) $\rightarrow$ inf(f,g) and the mappings $f \rightarrow f^{+}, f \rightarrow f^{-}, f \rightarrow|f|$ are uniformly continuous.\\
(ii) The positive cone is closed.\\
(iii) Order intervals, i.e. sets of the form $[f, g]=\{h \in E: f \leqq h \leqq g\}$ are closed and bounded.

Instead of dwelling upon a detailed discussion of the above equalities and inequalities let us rather formulate the following principle,\\
which allows us to verify any of them and to invent, prove or disprove new ones whenever necessary:

Any general formula relating a finite number of "variables" to each other by means of lattice operations and/or linear operations is valid in any Banach lattice as soon as it is valid in the real number system.

In fact, we are going to see below that any Banach lattice $E$ is, as a vector lattice, "locally" of type $C(X)$, more exactly: Given any finite number $x_{1}, \ldots, x_{n}$ of elements in $E$ there is a compact topological space $X$ and a vector sublattice $J$ of $E$ which is isomorphic to $C(X)$ and contains $x_{1}, \ldots, x_{n}$ (see section 4). The above principle is an easy consequence of this: In a space $C(X)$ it is clear that a formula of the type considered need only be verified pointwise, i.e. in $\mathbb{R}$.

The reader may now be prepared to follow a concise presentation of the most basic facts on Banach lattices.

\section*{1. SUBLATTICES, IDEALS, BANDS}
The notion of a vector sublattice of a vector lattice E is selfexplanatory, but it should be pointed out that a vector subspace $F$ of E which is a vector lattice for the ordering induced by $E$ need not be a vector sublattice of $E$ (formation of suprema may differ in $E$ and in F ), and that a vector sublattice need not contain (or may lead to different) infinite suprema and infima. The following are necessary and sufficient conditions on a vector subspace $G$ of $E$ to be a vector sublattice:

\begin{center}
\begin{tabular}{ll}
(i) $|h| \in G$ & for all \\
(ii) & $h^{+} \in G$ \\
(iii) & $h^{-} \in G$ \\
for all & $h \in G$ \\
for all & $h \in G$ \\
\end{tabular}
\end{center}

A subset $B$ of a vector lattice is called solid if $f \in B,|g| \leqq|f|$ implies $g \in B$. Thus a norm on a vector lattice is a lattice norm if and only if its unit ball is solid. A solid linear subpace is called an ideal. Ideals are automatically vector sublattices since |sup(f,g)| $\leq|f|+|g|$. On the other hand, a vector sublattice $F$ is an ideal in $E$ if $g \in F$ and $0 \leqq f \leqq g$ imply $f \in F$. A band in a vector lattice $E$ is an ideal which contains arbitrary suprema, more ex-\\
actly: $B$ is a band in $E$ if $B$ is an ideal in $E$ and sup $M$ is contained in $B$ whenever $M$ is contained in $B$ and has a supremum in $E$. Since the notions of sublattice, ideal, band are invariant under the formation of arbitrary intersections there exists, for any subset $B$ of $E$, a uniquely determined smallest sublattice (ideal, band) of $E$ containing $B$ : the sublattice (ideal, band) generated by B •\\
If we denote by $B^{d}$ the set $\{h \in E ; \inf (|h|,|f|)=0$ for all $f \in B\}$, then $B^{d}$ is a band for any subset $B$ of $E$, and $\left(B^{d}\right)^{d}=B^{d d}$ is a band containing $B$. If $E$ is a normed vector lattice (more generally, if E is archimedean ordered, see e.g. Schaefer (1974)), then $B^{d d}$ is the band generated by $B$.

If two ideals I, J of a vector lattice $E$ have trivial intersection $\{0\}$, then $I$ and $J$ are lattice disjoint, i.e. I $\subset J^{d}$. Thus if $E$ is the direct sum of two ideals $I$, J then one has automatically $I=J^{\mathrm{d}}$ and $J=I^{\mathrm{d}}$, hence $I=I^{d d}$ and $J=J^{\text {dd }}$ must be bands in this situation. In general, an ideal I need not have a complementary ideal $J$, even if $I=I^{d d}$ is a band. This amounts to the same as saying that even if $I=I^{d d}$ (which is always true if I is a band in a normed vector lattice) one need not necessarily have $E=I+I^{d}$. An ideal I is called a projection band if it does have a complementary ideal, and in this case the projection of $E$ onto I with kernel $I^{d}$ is called the band projection belonging to I . An example of a band which is not a projection band is furnished by the subspace of C([0,1]) consisting of the functions vanishing on $[0,1 / 2]$. The Riesz Decomposition Theorem asserts that in an order complete vector lattice every band is a projection band. As a consequence, if $E$ is order complete and $B$ is an arbitrary subset of $E$, then $E$ is the direct sum of the complementary bands $B^{d}$ and $\mathrm{B}^{\mathrm{dd}}$. This Theorem, which is quite easy to prove, is widely used in Analysis and gives an abstract background to, e.g., the decomposition of a measure into atomic and diffuse parts (the atomic measures being those contained in the band generated by the point measures, the diffuse measures those disjoint to the latter) or, more specifically, to the well-known decomposition of a measure on [a,b] into an atomic part and a diffuse part, which latter can in turn be decomposed into the sum of a measure which is absolutely continuous (which means, contained in the band generated by Lebesgue measure) and a singular measure (which means, a diffuse measure disjoint to Lebesgue measure).

A band in a normed vector lattice is necessarily closed. By contrast, an ideal need not be closed, but the closure of an ideal is again an ideal. The situation where every closed ideal is a band will be briefly discussed in Section 5.

\section*{2. ORDER UNITS, WEAK ORDER UNITS, QUASI-INTERIOR POINTS}
An element $u$ in the positive cone of a vector lattice $E$ is called an order unit, if the ideal generated by u is all of E. If the band generated by $u$ is all of $E$ (which is equivalent to $\{u\}^{d}=0$ whenever $E$ is archimedean, hence in particular if $E$ is a normed vector latticel then $u$ is called a weak order unit of $E$. If $E$ is a Banach lattice, then any order unit in $E$ is an interior point of the positive cone $E_{+}$, and conversely any interior point of $\mathrm{E}_{+}$must be an order unit of $E$. Every space $C(K)$ has order units (namely, the strictly positive functions), and conversely by the Kakutani-Krein Representation Theorem (see Section 4) every Banach lattice with an order unit is isomorphic to a space $c(K)$. If an element $u$ in the positive cone of a Banach lattice $E$ has the property that the closed ideal generated by $u$ is all of $E$, then $u$ is called a quasi-interior point of $E_{+}$. Quasi-interior points of the positive cone exist, e.g., in any separable Banach lattice. If $E=C(K)$, then the quasi-interior points and the interior points of $E_{+}$coincide, while the weak order units of $E$ are the (positive) functions vanishing on a nowhere dense subset of K . If E is a space $\mathrm{L}^{\mathrm{P}}(\mu)$ with $\sigma-\mathrm{fi}^{-}$ nite $\mu$ and $1 \leqq p<\infty$, then the weak order units and the quasi-interior points of $E_{+}$coincide with the functions strictly positive $\mu-a . e .$, while $E_{+}$does not contain any interior point.

\section*{3. LINEAR FORMS AND DUALITY}
A linear functional $\phi$ on a vector lattice $E$ is called

\begin{verbatim}
order-bounded, if $ is bounded on order intervals of E,
positive, if }\phi(f)\geqq0\mathrm{ for all f \ 0 ,
strictly positive, if }\phi(f)>0\mathrm{ for all f> 0.
\end{verbatim}

Any positive linear functional is order bounded, and the positive functionals form a proper convex cone with vertex 0 in the linear space $E^{\#}$ of all order bounded functionals, thus defining a natural\\
ordering (given by $\phi \leqq \psi$ if and only if $\phi(f) \leqq \psi(f)$ for all $\mathrm{f} \in \mathrm{E}_{+}$, under which $\mathrm{E}^{\#}$ is an order complete vector lattice. In particular, positive part, negative part and absolute value exist for any order bounded functional on $E$, the absolute value of $\phi \in E^{\#}$ being given by

$$
|\phi|(f)=\sup \{\phi(h):|h| \leqq f\} \quad\left(f \in E_{+}\right) .
$$

As a consequence, one has $|\phi(f)| \leqq|\phi|(|f|)$ for all $f$ in E whenever $\phi$ is order bounded, and $|\phi(f)| \leqq \phi(|f|)$ if and only if $\phi$ is positive. An order bounded linear functional $\phi$ is called ordercontinuous (o-order-continuous) if both positive and negative part of $\phi$ have the property that they transform any decreasing net lany decreasing sequence) with infimum 0 into a net (sequence) converging to 0 in $\mathbb{R}$. The order-continuous ( $\sigma$-order-continuous) functionals form a band in $E^{\#}$. In general, a vector lattice $E$ need not admit any non-zero order-bounded linear functional. However, if $E$ is a normed lattice, then any continuous functional is order-bounded, and if $E$ is a Banach lattice then one has coincidence between $E^{\# \#}$ and E'. Still, order-continuous functionals $\neq 0$ need not exist on a Banach lattice. Situations where every order-bounded functional is order-continuous will be briefly discussed in Section 5 .

If $E$ is a Banach lattice, then the dual norm on $E^{\prime}=E^{\#}$ is a lattice norm, hence $E^{\prime}$ is an order-complete Banach lattice under the natural norm and order. The evaluation map from E into the second dual E" is a lattice homomorphism (for the definition see Section 6) into the band of order-continuous functionals on E'. In particular, every dual Banach lattice $E$ admits sufficiently many order-continuous functionals to separate the points of $E$.

\section*{4. AM- AND AL-SPACES}
If the norm on a Banach lattice E satisfies


\begin{equation*}
\|\sup (f, g)\|=\sup (\|f\|,\|g\|) \text { for } f, g \in E_{+} \tag{M}
\end{equation*}


then E is called an abstract M-space or an AM-space. If in addition the unit ball of $E$ contains a largest element $u$, then $u$ must be an order unit of $E$ and $E$ is then called an (AM)-space with unit. Condition (M) in $E$ implies that in the dual of $E$ one has


\begin{equation*}
\|f+g\|=\|f\|+\|g\| \text { for } f, g \geqq 0 . \tag{L}
\end{equation*}


Any Banach lattice satisfying (L) is called an abstract L-space or an AL-space. Thus the dual of an AM-space is an AL-space. It is quite easy to verify that on the other hand the dual of an AL-space is an AM-space with unit, the unit being the uniquely determined linear functional that coincides with the norm on the positive cone. Putting this together, one gets that the second dual of an AM-space $E$ is an AM-space with unit. If $E$ already has a unit $u$, then $u$ is also the unit of $\mathrm{E}^{\prime \prime}$, so that the ideal of $\mathrm{E}^{\prime \prime}$ generated by E is all of E" . By contrast, if $E$ is an $A L-s p a c e$, then $E$ is an ideal (even a band) in $\mathrm{E}^{\prime \prime}$. Infinite-dimensional AL- or AM-spaces are never reflexive.

The importance of AL- and AM-spaces in the general theory of Banach lattices is due to the fact that these spaces have very special concrete representations as function lattices and that, on the other hand, any general Banach lattice $E$ is in a very intimate way connected to certain families of $\mathrm{AL-}$ and AM-spaces canonically associated with E. Let us first discuss the natural representations of AM- and AL-spaces.

If $E$ is an AM-space with unit $u$, then the set $K$ of lattice homomorphisms (cf. Section 6) from $E$ into $\mathbb{R}$ taking the value 1 on $u$ is a non-empty, compact subset of the weak dual of $E$ and the natural evaluation map from $E$ into $\mathbb{R}^{\mathrm{K}}$ maps E isometrically onto the continuous real-valued functions on $K$. This is the KakutaniKrein Representation Theorem, which is an order-theoretic counterpart to the Gelfand Representation Theorem in the theory of commutative $C^{*}$-algebras. If $E$ is an $A M-$ space without unit, then the second dual of E has a unit and thus gives a representation of E as a closed sublattice of a space $C(K)$. If $E$ is an AL-space, then the representation of the dual of $E$ as a space $C(K)$ leads to an interpretation of the elements of $\mathrm{E}^{\prime}$ as Radon measures on K . If $\mathrm{E}_{+}$has a quasi-interior point $h$, then in this interpretation E consists exactly of the measures absolutely continuous with respect to (the measure corresponding to) $h$, thus by the Radon-Nikodym Theorem $\mathrm{E}=\mathrm{L}^{1}(\mathrm{~K}, \mathrm{~h})$. In general, a similar argument leads to a representation of E as a space $L^{1}(X, \mu)$ constructed over a locally compact space X .

If $E$ is an arbitrary Banach lattice, $f \in E_{+}$, then the ideal I generated by f in E (which is the union of the positive multiples of the interval [-f,f]) can be made into an AM-space with unit f\\
by endowing it with the gauge function $p_{f}$ of $[-f, f]$. We denote ( $I, \mathrm{P}_{f}$ ) by $\mathrm{E}_{f}$. On the other hand, if $\mathrm{f}^{\prime}$ is a positive linear functional on $E$, then the mapping $g_{f}, f \rightarrow\langle | f\left|, f^{\prime}\right\rangle$ is a semi-norm on $\underset{E}{E}$. The kernel $J$ of $q_{f}$ is an ideal in $E$, and the completion of $E_{J}$ with respect to the norm canonically derived from $q_{f}$, becomes an AL-space which we denote by (E, $\mathrm{x}^{\prime}$ ) . A good illustration for these constructions is the following: If $E=C(K)$ and if $\mu$ is a positive linear form (Radon measure) on $E$, then $(E, \mu)$ is just $L^{1}(K, \mu)$; if $E=L^{p}(\mu)(1 \leqq \mathrm{p}<\infty, \mu$ finite $)$ then $E_{I_{X}}=\mu^{\infty}(\mu)$.\\
5. SPECIAL CONNECTIONS BETWEEN NORM AND ORDER

If an increasing net $\left(x_{\alpha}\right)_{\alpha \in A}$ in a normed vector lattice is convergent, then its limit must be the supremum: this is a consequence of the closedness of the positive cone. on the other hand, if $\left\{x_{\alpha}: \alpha \in A\right\}$ has a supremum, the net $\left(x_{\alpha}\right)_{\alpha \in A}$ need not converge. A Banach lattice is said to have order-continuous norm (o-order-continuous norm) if any increasing net (sequence) which has a supremum is automatically convergent. This is of course equivalent to saying that any decreasing net (sequence) with an infimum is convergent, and since infimum and limit must coincide, the order continuity (o-order continuity) of the norm in a Banach lattice is also equivalent to the following:

Any decreasing net (sequence) with infimum 0 converges to 0 .

A Banach lattice with order-continuous norm must be order complete, but o-order-continuity of the norm need not imply order completeness. At any rate, one has the following characterization:

A Banach lattice $E$ has order-continuous norm if and only if any one of the following equivalent assertions holds:\\
(a) E is o-order complete and has o-order-continuous norm.\\
(b) Every order interval in $E$ is weakly compact.\\
(c) E is (under evaluation) an ideal in E" .\\
(d) Every continuous linear form on E is order continuous.\\
(e) Every closed ideal in E is a projection band.

An even more stringent condition than order-continuity of the norm is\\
that any increasing norm-bounded net be convergent. This condition is satisfied if and only if any one of the following equivalent assertions holds:\\
(a) E is (under evaluation) a band in E".\\
(b) E is weakly sequentially complete.\\
(c) Every order-continuous linear form on $\mathrm{E}^{\text {' }}$ belongs to E .\\
(d) No closed sublattice of E is isomorphic to $c_{0}$.

The most important examples of non-reflexive Banach lattices with this property are the AL-spaces.

\section*{6. POSITIVE OPERATORS, LATTICE HOMOMORPHISMS}
A Iinear mapping $T$ from an ordered Banach space $E$ into an ordered Banach space $F$ is called positive (notation: $T \geqq 0$ ) if $T f \in F_{+}$ for all $f \in E_{+} ; T$ is called strictly positive if $T \geqq 0$ and $\{f \in E: T|f|=0\}=\{0\}$. The set of all positive linear mappings is a convex cone in the space $\mathrm{L}(\mathrm{E}, \mathrm{F})$ of all linear mappings from E into $F$ defining the natural ordering of $L(E, F)$. The linear subspace of L(E,F) generated by the positive maps (i.e. the space of linear maps that can be written as differences of positive maps) is denoted by $L^{r}(E, F)$ and its elements are called regular mappings. If $E$ and $F$ are Banach lattices, then any regular mapping from $E$ into $F$ is continuous, but $L^{r}(E, F)$ is in general a proper subspace of the space $L(E, F)$ of all continuous linear mappings. One has coincidence of $L^{r}(E, F)$ and $L(E, F)$ e.g. when $E=F$ is an order complete AM-space with unit or an AL-space. At any rate, if $F$ is order complete, then $L^{r}(E, F)$ under the natural ordering is an order-complete vector lattice, and a Banach lattice under the norm

$$
T \rightarrow\|T\|_{r}=\||T|\|
$$

the right hand side denoting the operator norm of the absolute value of $T$. The absolute value of $T \in L^{\Upsilon}(E, F)$, if it exists, is given by

$$
|T|(f)=\sup \{T h:|h| \leqq \mathrm{f}\} \quad\left(f \in E_{+}\right\}
$$

Thus $T$ is positive if and only if $|T f| \leq T|f|$ holds for any $f$ in E .\\
$\mathrm{T} \in \mathrm{L}(\mathrm{E}, \mathrm{F})$ is called a lattice homomorphism if $|\mathrm{Tf}|=\mathrm{T}|\mathrm{f}|$ holds for all $f \in \mathrm{E}$. Lattice homomorphisms are alternatively characterized by any one of the following conditions:\\
(a) $(\mathrm{Tf})^{+}=\mathrm{T}\left(\mathrm{f}^{+}\right) \quad(\mathrm{f} \in \mathrm{E})$\\
(a') $(T f)^{-}=T\left(f^{-}\right) \quad(f \in E)$\\
(b) T(Evg) $T \mathrm{Tf} \vee \mathrm{Tg} \quad(f \in \mathrm{E})$\\
(b') T(fag) $=T f_{\wedge} T g \quad(f \in E)$\\
(c) $T\left(f^{+}\right) \wedge T\left(f^{-}\right)=0 \quad(f \in E)$

The kernel of a lattice homomorphism is an ideal. If T is bijective, then $T$ is a lattice homomorphism if and only if $T$ and $\mathrm{T}^{-1}$ are positive.

\section*{7. COMPLEX BANACH LATTICES}
Although the notion of a Banach lattice is intrinsically related to the real number system, it is possible and often desirable to extend discussions to complexifications of Banach lattices in such a way that the order-related terms introduced in the real situation essentially retain their meaning. Thus we define a complex Banach lattice $E$ to be the complexification of a real Banach lattice $E_{\mathbb{R}}$ in the sense that

$$
\mathrm{E}=\mathrm{E}_{\mathbb{R}} \oplus i \mathrm{E}_{\mathbb{R}}
$$

which means more exactly $E=E_{\mathbb{R}} \times E_{\mathbb{R}}$ with scalar multiplication $(\alpha+i \beta)(x, y)=(\alpha x-\beta y, \beta x+\alpha y)$. $E_{R}$ will sometimes be called the underlying real Banach lattice or the real part of E . The classical complex Banach spaces $C(X), L^{P}(\mu)$ are complex Banach lattices in this sense, the underlying real Banach lattices being the corresponding (real) subspaces of real-valued functions. We want to extend the formation of absolute values, which is a priori defined only in the real part of $E$, in such a way that in the classical situation $E=C(X)$ or $E=L^{\mathrm{P}}(\mu)$ the usual absolute value of a function is obtained. This is in fact possible by putting, for $h=f+i g$ in E

$$
|h|=\sup \left\{\operatorname{Re}\left(e^{i \theta} h\right): 0 \leqq \theta \leqq 2 \pi\right\}
$$

the only problem with this definition being the existence of the right hand side without the assumption of order-completeness on $E_{\mathbb{R}}$\\
(the set in brackets is an order bounded subset of $\mathbb{E}_{\mathbb{R}}$ ). But for this we just have to observe that the set $M=\left\{\operatorname{Re}\left(e^{i \theta} h\right): 0 \leqq \theta \leqq 2 \pi\right\}$ is contained and order bounded in the ideal generated in $E_{\mathbb{R}}$ by $|f|+|g|$, which in turn is by the Kakutani-Krein Representation Theorem isomorphic to a space $\mathbb{C}_{\mathbb{R}}(\mathrm{X})$ under the pointwise ordering. Now the pointwise supremum of $M$ in $\mathbb{R}^{X}$ is readily seen to be a continuous function (namely, the modulus of the complex valued continuous function corresponding to $f+i g$ ) so that $M$ has a supremum in $C_{R}(X)=\left(E_{R}\right)|f|+|g|$.

Since the mapping $f \rightarrow|f|$ now has a meaning in $E$, the definition of an ideal can be extended formally unchanged to the complex situation. We observe that $|f+i g|=|f-i g| \leqq|f|+|g|$ implies that any ideal J in a complex Banach lattice is conjugation invariant and itself the complexification of the ideal $J \cap E_{R}$ of the real part of $E$. Suffice it now to say that the meaning of most of the terms introduced for real Banach lattices above can be extended to the complex situation under retention (mutatis mutandis) of the corresponding results valid in the real case by either using the complex modulus or else, if the formation of suprema or infima is involved, by relating them to real parts. For example $f \in E$ is called positive if $f=|f|$ which means that $f$ is a positive element of $E_{\mathbb{R}}, \mathrm{E}$ is called order complete if $\mathrm{E}_{\mathbb{R}}$ is order complete, and an ideal J is called a band if the real part of $J$ is a band. We refer to Chapter II, Section 11 of Schaefer (1974) for a detailed discussion of this and restrict ourselves to a short discussion of linear mappings.

Let $E$ and $F$ be complex Banach lattices with real parts $E_{\mathbb{R}}$ and $F_{\mathbb{R}}$. Then a linear mapping $T$ from $E$ into $F$ is determined by its restriction $T_{0}$ to $E_{\mathbb{R}}$, and $T_{0}$ can be written in the form $T_{0}=T_{1}+i T_{2}$ with real-lineax mappings $T_{j}$ from $E_{\mathbb{R}}$ into $F_{\mathbb{R}}$. Thus $L(E, F)$ is the complexification of the real linear space $L\left(E_{\mathbb{R}}, F_{\mathbb{R}}\right)$. With the above notation, T is called real if $\mathrm{T}_{2}$ is $=0$, positive if $T$ is real and $T_{1}$ is positive, and a lattice homomorphism if $T$ is real and $T_{1}$ is a lattice homomorphism. Lattice homomorphisms are characterized by the equality $|\mathrm{Th}|=\mathrm{T}|\mathrm{h}|$ as in the real case.

\section*{8. THE SIGNUM OPERATOR}
We discuss in some detail how a mapping of the form

$$
g \rightarrow(\operatorname{sign} f) g
$$

which has an obvious meaning, depending on f , in spaces $\mathrm{C}(\mathrm{K})$, can be defined in an abstract complex Banach lattice. We prove the following:

Let $E$ be a complex Banach lattice and let $f \in E$. If either $E$ is order-complete or $|f|$ is a quasi-interior point in $E_{+}$, then there exists a unique linear mapping $s_{f}$, called the signum operator with respect to $f$, with the following properties:\\
(i) $\quad S_{f} \bar{f}=|f|$, where $\bar{f}=\operatorname{Re} f-i$-Imf\\
(ii) $\left|s_{f} g\right| \leqq|g|$ for every $g$ in $E$\\
(iii) $\mathrm{s}_{\mathrm{f}}^{\mathrm{g}}=0$ for every $g$ in E orthogonal to f .

In fact, if $E=C(K)$ and if $|f|$ is a quasi-interior point in $E$, then $|\mathrm{f}|$ is a strictly positive function and multiplication with the function sign $f=f \cdot|f|^{-1}$ has the desired properties. Uniqueness follows from Zaanen (1983) Chap. 20. We reduce the general situation to the case just considered in the following way:

\begin{enumerate}
  \item If $|f|$ is quasi-interior to $E_{+}$, then $E_{|f|}$ is a dense subspace of $E$ isomorphic to a space $C(K)$, and with the above arguments one gets a uniquely determined operator $S_{0}$ on $E_{|f|}$ with the desired properties. Since (ii) implies the continuity of $s_{0}$ for the norm induced by $E, S_{O}$ can be extended to $E$.
  \item If $f$ is arbitrary, then as above one gets an operator $s_{0}$ on $E_{|f|}$ with (i) - (ii). If $E$ is order complete, an extension $S_{f}$ of $S_{0}$ to $E$ satisfying (i) - (iii) is possible as soon as $s_{0}$ can be extended to the band $\{x\}^{d d}$ of $E$ : on the complementary band $\{x\}^{d}$ one has necessarily the values $\equiv 0$ for $S_{f}$. The extension to $\{x\} d a$ is obtained as follows:\\
If $s_{0}$ is positive (which means $f \geqq 0$ ) then
\end{enumerate}

$$
s_{f} h=\sup \left\{s_{f}^{g}: g \in[0, h] \cap E|f|\right\} \quad(h \geqq 0)
$$

will do. In general, the problem can be reduced to this situation by decomposing $s_{0}$ into a sum of the form $S_{0}=\left(S_{1}-S_{2}\right)+i\left(S_{3}-S_{4}\right)$\\
with positive operators $s_{j}$. Such a decomposition of $S_{0}$ exists since the order completeness of E implies the order completeness of $E_{|f|}=C(K)$ and since every continuous linear operator on a space $C(K)$ is necessarily order-bounded.\\
9. THE CENTER OF L(E)

We give a short description of a special, but important class of operators.\\
Let $E$ be a (complex) Banach lattice. For $T \in L(E)$ the following conditions are equivalent:\\
(a) $f \perp g$ implies $T f \perp g \quad(f, g \in E)$\\
(b) $\pm T \leqq\|T\| I d$\\
(c) $T J \subset J$ for every ideal $J$ in E.

If E is countably order complete, then this is equivalent to:\\
(a) $\mathrm{TJ} \subset \mathrm{J}$ for ervery projection band $J$ in $E$.

The last assertion also means that $T$ commutes with every band projection.\\
The set of all $T \in L(E)$ satisfying the above conditions is called the center of $L(E)$ and denoted $Z(E)$. $Z(E)$ is under the natural ordering, the operator norm and the composition product isomorphic to a Banach lattice algebra $C(K)$ ( K compact). The following examples may illustrate the situation and explain why the term "multilication operator" is often used for operators in $Z(E)$.\\
(i) If $\mathrm{E}=\mathrm{L}^{\mathrm{P}}(X, \Sigma, \mu) \quad(1 \leqq \mathrm{p} \leqq \infty)$ with o-finite $\mu$, then $Z(E)$ is isomorphic to $L^{\infty}(\mu)$ via the natural identification of a function $\mathbf{f} \in L^{\infty}(\mu)$ with the multiplication operator $g \rightarrow f \cdot g$ on $E$.\\
(ii) If $X$ is locally compact, $E=C_{0}(X)$ then similarly $Z(E) \cong C^{b}(x)$ via the identification of $f \in C^{b}(x)$ with the mapping

$$
g \rightarrow f \cdot g \quad\left(g \in C_{0}(X)\right)
$$

\section*{CHAPTER C-II}
\section*{CHARACTERIA A T I O N \\
 OF POSITIVE SEMIGROUPS }
\section*{ON BANACH LATTICES}
by\\
Wolfgang Arendt

In this chapter our first goal is to find conditions on a generator A of a semigroup $(T(t))_{t \geqq 0}$ which are equivalent to the positivity of the semigroup. After the preparations in A-II, Sec. 2 this is easy if in addition we ask that the semigroup be contractive: $\mathrm{T}(\mathrm{t})$ is a positive contraction for all $t \geq 0$ if and only if $A$ is dispersive (Section 1). For arbitrary (not necessarily contractive) semigroups a condition on the generator had been found in the case when $E=C(K)$ (K compact), namely the positive minimum principle (P) (see B-II). One may easily reformulate this condition in arbitrary Banach lattices and show its necessity. However, only in special cases for example if A is bounded (see section 1)) the positive minimum principle is sufficient for the positivity of the semigroup. In fact, on $I^{2}(\mathbb{R})$ there exists a non-positive semigroup whose generator satisfies (P) (Section 3).

Looking for another condition we consider the Laplacian $\Delta$ as a prototype. Defined on a suitable domain, $\Delta$ generates a positive semigroup on $L^{P}\left(\mathbb{R}^{n}\right)$. Kato proved the following distributional inequality for the Laplacian:

$$
(\operatorname{sign} \bar{E}) \Delta f \leqq \Delta|f|
$$

for all f $\in \mathrm{L}_{100}^{1}$ such that $\Delta f \in \mathrm{~L}_{100^{\circ}}^{1}$. In section 3 we will show that an abstract version of Kato's inequality for a generator A together with an additional condition is equivalent to the positivity of the semigroup generated by $A$.

Domination of one semigroup by another can be characterized by an analoguous condition for the generators (Section 4). The results will be applied to Schrödinger operators on $L^{\mathrm{P}}\left(\mathbb{R}^{n}\right)$.

Finally, in section 5 we show that $(T(t))_{t \geqq 0}$ is a lattice semigroup (i.e., $|T(t) f|=T(t)|f|$ for all $t \geqq 0, f \in E$ I if and only if $A$ satisfies Kato's equality. This parallels the case when $E=C_{O}(X)$, but if E has order continuous norm the strong form of Kato's equality can be considered (in particular, $f \in D(A)$ implies $|f| \in D(A)$ if $A$ is the generator of such a semigroup).

\section*{1. POSITIVE CONTRACTION SEMIGROUPS AND BOUNDED GENERATORS}
In this section we first characterize generators of positive contraction semigroups on a real Banach lattice E .\\
For that we use the results developed in A-II, section 2 for the canonical half-norm $\mathrm{N}^{+}: \mathrm{E} \rightarrow \mathbb{R}$ given by


\begin{equation*}
N^{+}(E)=\left\|E^{+}\right\| \quad(E \in E) . \tag{1.1}
\end{equation*}


Remark. It is, easy to see that $\mathrm{N}^{+}(\mathrm{f})=\inf \left\{\|\mathrm{f}+\mathrm{g}\|: g \in \mathrm{E}_{+}\right\}=$ dist (-f, $\mathrm{E}_{+}$) (cf. A-II,Rem.2.8).

It is obvious that $\mathrm{N}^{+}$is a strict half-norm (see A-II, (2.12)). The subdifferential of $\mathrm{N}^{+}$is given by\\
(1.2) $\mathrm{dN}^{+}$(f) $=\left\{\phi \in \mathrm{E}_{+}^{\mathbf{1}}:\|\phi\| \leqq 1,\langle f, \phi\rangle=\left\|\mathrm{f}^{+}\right\|\right\}$\\
(this follows from the definition, see A-II, (2.5)).

Examples 1.1. a) Let $E=C_{0}(X)$ ( $X$ locally compact). Let $f \in E$. There exists $x \in X$ such that $f(x)=\left\|\mathrm{f}^{+}\right\|_{\infty}$. Then $\delta_{x} \in d N^{+}(f)$.\\
b) Let $E=I^{P}(X, \Sigma, \mu)$, where $(X, \Sigma, \mu)$ is a $\sigma$-finite measure space and $1<p<\infty$. Let $f \in E$ satisfy $\mathrm{f}^{+} \neq 0$. Let

$$
\phi(x)= \begin{cases}c \cdot f(x)^{p-1} & \text { if } f(x)>0 \\ 0 & \text { if } f(x) \leqq 0\end{cases}
$$

where $c>0$ is such that $\int f(x) \phi(x) d x=\left\|\mathrm{I}^{+}\right\|$.\\
Then $d \mathrm{~N}^{+}(f)=\{\phi\}$.\\
c) Let $\mathrm{E}=\mathrm{L}^{1}(\mathrm{X}, \Sigma, \mu)$, where $(\mathrm{X}, \Sigma, \mu)$ is a $\sigma$-£inite measure space, and $\pounds \in E$. Let $\phi \in \mathrm{L}^{\infty}(\mathrm{X}, \Sigma, \mu)_{+}$. Then $\phi \in \mathrm{dN}^{+}(f)$ if and only if $\phi(x)=1$ if $f(x)>0$,\\
$0 \leqq \phi(x) \leqq 1$ if $f(x)=0$ and $\phi(x)=0 \quad$ if $f(x)<0$.

An operator $A$ on $E$ is called [strictly] dispersive if $A$ is [strictly] $\mathrm{N}^{+}$-dissipative; that is, for every $f \in D(A)$ one has <Af, $\phi$ > $\leqq 0$ for some [resp., all] $\phi \in \mathrm{dN}^{+}$(f) (see A-II, Sec. 2). Generators of positive contraction semigroups are characterized by the following theorem which is due to Phillips (1962).

Theorem 1.2. Let $A$ be a densely defined operator on a real Banach lattice E. The following assertions are equivalent.\\
(i) A is the generator of a positive contraction semigroup.\\
(ii) $A$ is dispersive and $(\lambda-A)$ is surjective for some $\lambda>0$.

Frequently an operator is known explicitly only on a core. In that case one can use the following result.

Corollary 1.3. Let A be a densely defined dispersive operator on a real Banach lattice $E$. If $(\lambda-A) D(A)$ is dense in $E$ for some $\lambda>0$, then $A$ is closable and the closure $\bar{A}$ of $A$ is the generator of a positive contraction semigroup.

Theorem 1.2 and Corollary 1.3 immediately follow from A-II, Thm. 2.11 and A-II, Cor.2.12 if one observes the following.

Lemma 1.4. A bounded linear operator $T$ on a Banach lattice $E$ is a positive contraction if and only if $\left\|(\mathrm{Tf})^{+}\right\| \leqq \mathrm{f}^{+} \|$for all $\mathrm{f} \in \mathrm{E}$ (i.e., if T is $\mathrm{N}^{+}$-contractive).

Proof of the lemma. If $T$ is a positive contraction, then $0 \leqq(T f)^{+}$ $\leqq \mathrm{Tf}^{+}$and so $\mathrm{N}^{+}$(Tf) $\leqq\left\|T \mathrm{f}^{+}\right\| \leqq\left\|\mathrm{f}^{+}\right\|=\mathrm{N}^{+}$(£) for all £ $\epsilon$ E. Conversely, assume that $T$ is an $\mathrm{N}^{+}$-contraction. Let $f \geq 0$. Then $\left\|(T f)^{-}\right\|=N^{+}(T(-f)) \leqslant N^{+}(-f)=\left\|f^{-}\right\|=0$. Hence $(T f)^{-}=0$; i.e., Tf $\geqq 0$. We have proved that $T$ is positive. In particular, $|\mathrm{T} f| \leqq \mathrm{T}|\mathrm{f}|$ for all $\mathrm{f} \in \mathrm{E}$. Hence $\|\mathrm{Tf}\|=\||\mathrm{Tf}|\| \leqq \| \mathrm{T}|\mathrm{f}| \mid=\mathrm{N}^{+}(\mathrm{T}|\mathrm{f}|)$ $\leqq \mathbb{N}^{+}(|f|)=\|f\|$ for all $E \in E$. So $T$ is a contraction.

Examples 1.5. a) Consider the second derivative with Dirichlet boundary condition on $E=C_{0}(0,1)$; i.e., we let $A f=\pounds^{\prime \prime}$ with domain $D(A)=\left\{f \in C^{2}[0,1]: f(0)=f(1)=f^{\prime \prime}(0)=f^{\prime \prime}(1)=0\right\}$.\\
$A$ is dispersive. In fact, let $f \in D(A)$. Then there exists $x \in(0,1)$ such that $f(x)=\sup _{y \in[0,1]} f(y)=\left\|f^{+}\right\|_{\infty}$. Thus $\delta_{\mathrm{x}} \in \mathrm{dN}^{+}(\mathrm{f})$. But $<\mathrm{Af}, \delta_{\mathrm{x}}=\mathrm{f}^{\prime \prime}(\mathrm{x}) \leq 0$ since f has a maximum in x . $\stackrel{x}{\text { Let }} g \in E$. Define $f_{o}(x)=1 / 2\left[e^{x} \int_{x}^{1} e^{-y} g(y) d y-e^{-x} \int_{x}^{1} e^{y} g(y) d y\right]$.

Then $f_{0} \in c^{2}[0,1]$ and $f_{0}-f_{O_{x}}^{\prime \prime}=g$. There exist $a, b \in \mathbb{R}$ such that $f(x)=E_{0}(x)+a e^{x^{0}}+b e^{-x}$ defines a function $f \in c^{2}[0,1]$ satisfying $f(0)=f(1)=0$. Since $f-f^{\prime \prime}=f_{0}-f_{0}^{\prime \prime}=g \quad$ this implies that $f \in D(A)$ and $f-A f=g$. We have shown that (Id - A) is surjective. It follows from Thm.1.2 that $A$ is the generator of a positive contraction semigroup.\\
b) Let $E=L^{P}[0,1] \quad(1 \leqq p<\infty)$ and $A$ be given by $A f=f^{\prime \prime}$ on $D(A)=\left\{f \in E: f \in C^{1}[0,1], f^{\prime} \in \operatorname{AC}[0,1], f^{\prime \prime} \in L^{p}[0,1], f(0)=\right.$ $f(1)=0\}$. Then $A$ is the generator of a positive contraction semigroup.

Proof. A is dispersive. In fact, let $f \in D(A)$. Since the set $M=\{x \in(0,1): f(x)>0\}$ is open, there exists a countable set of disjoint intervals $\left(a_{n}, b_{n}\right)$ such that $M=U_{n \in N}\left(a_{n}, b_{n}\right)$.\\
First case: p > 1.\\
Let $\phi \in \mathrm{dN}^{+}(f)$. Then there exists $c \geq 0$ such that $\phi(x)=c f(x)^{p-1}$ for all $x \in M$ and and $\phi(x)=0$ if $f(x) \leqq 0$ (see Ex. 1.1b). Thus integration by parts yields

$$
\begin{aligned}
\langle A f, \phi\rangle & =\sum_{n} \int_{a_{n}}^{b_{n}} f^{\prime \prime}(x) \phi(x) d x \\
& =-c \sum_{n} \int_{a_{n}}^{b_{n}} f^{\prime}(x) f^{\prime}(x)(p-1) f(x)^{p-2} d x \\
& \leqq 0 .
\end{aligned}
$$

Second case: $p=1$.\\
Let $\phi(x)=1$ for $x \in M$ and $\phi(x)=0$ for $x \notin M$. Then $\phi \in \mathrm{dN}^{+}$(f) and\\
$\langle A f, \phi\rangle=\sum_{n} \int_{a_{n}}^{b_{n}} f^{\prime \prime}(x) d x=\sum_{n}\left(f^{\prime}\left(b_{n}\right)-f^{\prime}\left(a_{n}\right)\right) \leqq 0$\\
since $f^{\prime}\left(b_{n}\right) \leq 0$ and $f^{\prime}\left(a_{n}\right) \geq 0$ for all $n$.\\
We have shown that $A$ is dispersive. As in a) one shows that\\
(Id - A) is surjective. Now the claim follows from Thm.1.2.\\
c) Consider $E=C_{0}\left(\mathbb{R}^{n}\right)$. Let $D(A)=S\left(\mathbb{R}^{n}\right)$ (the schwartz space of all infinitely differentiable rapidly decreasing functions) and Af $=\Delta f$ (f $\epsilon D(A)$ ). Then $A$ is closable and the closure of $A$ generates a positive contraction semigroup on $E$.

Remark. In addition one can show that the closure $\bar{A}$ of $A$ is given by $\bar{A} f=\Delta f$ with domain $D(\bar{A})=\{f \in E: \Delta f \in E\}$ where for\\
$f \in C_{0}\left(\mathbb{R}^{n}\right)$ the expression $\Delta f$ is understood in the sense of distributions. Moreover, the space $C_{C}^{\infty}\left(\mathbb{R}^{n}\right)$ (of all infinitely differentiable functions with compact support) is a core of $\overline{\mathbb{A}}$ (cf. d).

Proof. A is dispersive. In fact, let $f \in D(A)$. If $\mathrm{f}^{+}=0$, then $\phi:=0 \in \mathrm{dN}^{+}$(f) . So assume that $\mathrm{f}^{+} \neq 0$. Then there exists $\mathrm{x} \in \mathbb{R}^{\mathbf{n}}$ such that $f(x)=\|f\|_{\infty}=\sup \left\{f(y): y \in \mathbb{R}^{n}\right\}$. Thus $\delta_{x} \in \mathrm{dN}^{+}(f)$. Since $f$ has a maximum in $x$ it follows that $\left\langle A f, \delta_{x}=(\Delta f)(x)=\operatorname{tr}\left(\partial^{2} f / \partial x_{i} \partial x_{j}\right)(x) \leq 0\right.$. Moreover, (1.3) (Id - $\Delta$ ) is an isomorphism from $S\left(\mathbb{R}^{n}\right)$ onto $S\left(\mathbb{R}^{n}\right)$. In fact, the Fourier transform $f \rightarrow \hat{f}$ is a bijection from $S\left(\mathbb{R}^{n}\right)$ onto $S\left(\mathbb{R}^{\mathrm{n}}\right)$.\\
But $[(I d-\Delta) f]^{n}=\mathrm{mf}$ where $(\mathrm{Mg})(y)=\left(1+\sum_{i=1}^{n} y_{i}^{2}\right) g(y)$ ( $g \in S\left(\mathbb{R}^{n}\right)$ ). It follows from (1.3) that (Id - A)D(A) is dense in E . So the claim follows from cor.1.3.\\
d) Let $E=L^{P}\left(\mathbb{R}^{n}\right) \quad(1 \leqq \mathrm{p}<\infty)$ and $A$ be given by $A f=\Delta f$ with domain $D(A)=\left\{f \in L^{\mathrm{P}}\left(\mathbb{R}^{\mathrm{n}}\right): \Delta f \in \mathrm{~L}^{\mathrm{P}}\left(\mathbb{R}^{\mathrm{n}}\right)\right\}$ where for $\mathrm{f} \in \mathrm{L}^{\mathrm{P}}\left(\mathbb{R}^{\mathrm{n}}\right)$ the expression $\Delta f$ is understood in the sense of distributions. Then A is the generator of a positive contraction semigroup. Moreover, the space $C_{C}^{\infty}\left(\mathbb{R}^{n}\right)$ is a core of $A$.\\
proof. It is easy to see that $A$ is closed. Let $A_{0}$ denote the restriction of $A$ to $S:=S\left(\mathbb{R}^{n}\right)$. Then $A_{0} f=\Delta f$ in the classical sense for all $f \in S$. One can show in an analogous way as in b) that $A_{0}$ is dispersive. Moreover, it follows from (1.3) that (Id - $A_{0}$ )D( $A_{0}$ ) is dense. Hence by Cor. 1.3 the closure $\bar{A}_{0}$ of $A_{0}$ is the generator of a positive contraction semigroup.\\
$B y$ construction one has $\bar{A}_{0} \subset A$. We prove that $\bar{A}_{0}=A$. For that it is enough to show that

In fact, since the restriction (Id - $\bar{A}_{0}$ ) of (Id - A) is bijective from $D\left(\bar{A}_{0}\right)$ onto $E$ it follows from (1.4) that $D\left(\bar{A}_{0}\right)=D(A)$. So let us show (1.4). Assume that there is $f \in E$ such that $f-A f=0$. Let $\phi \in C_{C}^{\infty}\left(\mathbb{R}^{n}\right)$. Then


\begin{equation*}
\langle\phi-\Delta \phi, f\rangle=0 . \tag{1.5}
\end{equation*}


Since $C_{c}^{\infty}\left(\mathbb{R}^{n}\right)$ is dense in $S$ for the topology of $S$, it follows from (1.3) that (Id - $\Delta) C_{c}^{\infty}\left(\mathbb{R}^{n}\right)$ is dense in $S$. Hence (1.5) implies\\
that $\langle\phi, f>=0$ for all $\phi \in S$. Consequently, $f=0$.

Remark. Using the Fourier transform one can show that the semigroups in example c) and d) are given by\\
$(T(t) f)(x)=(4 \pi t)^{-n / 2} \int_{\mathbb{R}^{n} n} \exp \left(-(x-y)^{2} / 4 t\right) f(y) d y$ (f $\in E$ ), where $z^{2}:=\sum_{i=1}^{n} z_{i}^{2} \quad\left(z \in \mathbb{R}^{n}\right)$.\\
e) The following example is the analog of a) for higher dimension. Let $\Omega \subset \mathbb{R}^{n}$ be a bounded open and connected set and $E=C_{0}(\Omega)$. We assume that the Dirichlet problem

\[
\begin{array}{ll}
u(x)-\Delta u(x)=0 & (x \in \Omega) \\
u(x)=b(x) & (x \in \partial \Omega) \tag{1.7}
\end{array}
\]

has a solution $u \in C^{2}(\Omega) \cap C(\bar{\Omega})$ for every $b \in C(a \Omega)$. For example, this is the case if the boundary $\partial \Omega$ is $c^{2}$ (see [Gilbarg-Trudinger (1977), Thm. 6.13]).

Let $A$ be given by $A f=\Delta f$ on\\
$D(A)=\left\{f \in C^{2}(\Omega) \cap C_{0}(\Omega): \Delta f \in C_{0}(\Omega)\right\}$.\\
Then A is closable and the closure of $A$ is the generator of a positive contraction semigroup.

Proof. D(A) is clearly dense in E . Moreover, one can show as in c) that $A$ is dispersive. It remains to prove that (Id - A)D(A) is dense in $E$. The space $C_{C}^{\infty}(\Omega)$ of all infinitely differentiabel functions on $\Omega$ with compact support contained in $\Omega$ is dense in E . Let $g \in C_{C}^{\infty}(\Omega)$. We show that there exists $f \in D(A)$ satisfying $(I d-A) f=g$. Let $\bar{g}: \mathbb{R}^{n} \rightarrow \mathbb{R}$ be given by $\bar{g}(x)=g(x)$ if $x \in \Omega$ and 0 if $x \notin \Omega$. Then $\bar{g} \in S\left(\mathbb{R}^{n}\right)$. By (1.3) there exists $\overline{\mathbf{f}} \in S\left(\mathbb{R}^{\mathrm{n}}\right)$ such that $\overline{\mathbf{E}}-\Delta \overline{\mathrm{f}}=\overline{\mathrm{g}}$. Consider the function $\mathrm{b} \in \mathrm{C}(2 \Omega)$ given by $b(x)=\overline{\mathrm{f}}(\mathrm{x})$ for all $\mathrm{x} \in \partial \Omega$. Then by our hypothesis there exists $u \in C(\bar{\Omega}) \cap C^{2}(\Omega)$ satisfying (1.7). Let $\mathbf{f}(\mathrm{x})=\overline{\mathrm{f}}(\mathrm{x})-\mathrm{u}(\mathrm{x})$ $(x \in \bar{\Omega})$. Then $f \in C^{2}(\Omega) \cap C_{0}(\Omega)$ and $(f-\Delta f)(x)=g(x) \quad(x \in \bar{\Omega})$. Thus $\Delta f=f-g$ vanishes on $\partial \Omega$. Hence $f \in D(A)$ and $f-A f=g$. f) Let $\Omega \subset \mathbb{R}^{n}$ be as in e) and $E=L^{p}(\Omega)$. Define $A f=\Delta f$ on $D(A)=\left\{f \in C^{2}(\Omega) \cap C_{0}(\Omega): \Delta f \in C_{0}(\Omega)\right\}$. Then $A$ is closable and the closure of $A$ is the generator of a positive contraction semigroup on E.

Proof. D(A) is dense and one can show in an analoguous manner as in b) that $A$ is dispersive. We know from d) that $C_{C}^{\infty}(\Omega) \subset(I d-A) D(A)$. Thus (Id - A)D(A) is dense in $E$ and the claim follows from Cor.1.3.

We now turn to the problem to characterize generators of arbitrary (not necessarily contractive) positive semigroups. of course, as in $B-I I$, Sec. 1 one sees that a semigroup (T(t)) $t \geqq 0$ is positive if and only if $R(\lambda, A) \geq 0$ for all $\lambda>\omega(A)$ where $A$ denotes the generator of $(\mathrm{T}(t))_{t \geqq 0}$. We are looking for an intrinsic condition on A .

The positive minimum principle which is characteristic for generators of strongly continuous semigroups on C(K) (see B-II, Thm.1.6) can be reformulated on an arbitrary Banach lattice E .

Definition 1.6. An operator $A$ on $E$ satisfies the positive minimum principle if for all $\mathrm{f} \in \mathrm{D}(\mathrm{A})_{+}, \phi \in \mathrm{E}_{+}^{\prime}$, (P) $\langle f, \phi\rangle=0$ implies $\langle A f, \phi\rangle \geqq 0$.

Remark. It is easy to see that this definition coincides with that given in B-II, Sec. 1 in the case when $E=C(K)$ ( $K$ compact). [In fact, suppose that for all $\mathrm{f} \in \mathrm{D}(\mathrm{A})+$ and $\mathrm{x} \in \mathrm{K}, \mathrm{f}(\mathrm{x})=0$ implies $(A f)(x) \geq 0$. Let $g \in D(A)_{+}, \mu \in M(K)_{+}$such that $\langle g, \mu\rangle=0$. Then $g(x)=0$ for all $x \in$ supp $\mu$. Thus by hypothesis, (Ag) (x) $\geqq 0$ for all $x \in$ supp $\mu$. Consequently $\langle A g, \mu\rangle \geqq 0$. This proves one direction. The other is obvious by considering point measures.]

Proposition 1.7. The generator of a strongly continuous positive semigroup satisfies the positive minimum principle (P).

Proof. Let $(T(t))_{t \geqq 0}$ be a strongly continuous positive semigroup with generator $A$ and $0 \leqq f \in D(A), \phi \in E_{+}^{\prime}$ such that $\langle\mathbb{E}, \phi\rangle=0$. Then $\langle\mathrm{Af}, \phi\rangle=\lim _{t \rightarrow 0} 1 / t\langle T(t) \mathrm{f}-\mathrm{f}, \phi\rangle=\lim _{t \rightarrow 0} 1 / t\langle T(t) f, \phi\rangle \geq 0$.

We will see that the positive minimum principle is not sufficient for the positivity of the semigroup, in general (Remark 3.16). However, the following special case is of interest.

Theorem 1.8. Let $A$ be the generator of a strongly continuous semigroup $(T(t))_{t \geq 0}$ on a Banach lattice $E$. Assume that\\
a) there exists $w \in \mathbb{R}$ such that $\|\mathrm{T}(t)\| \leqq e^{w t}$ for all $t \geqq 0$;\\
b) there exists a core $D_{0}$ of $A$ such that $f \in D_{0}$ implies $|f| \in D_{0}$.\\
If the restriction of $A$ to $D_{0}$ satisfies the positive minimum principle, then the semigroup is positive.

Remark. Elementary examples show that neither a) nor b) hold for generators of positive semigroups, in general.

The proof of Theorem 1.8 is based on the following proposition.

Proposition 1.9. Let A be a densely defined dissipative operator which possesses a core $D_{0}$ such that $f \in D_{0}$ implies $|f| \in D_{0}$. If the restriction of $A$ to $D_{0}$ satisfies the positive minimum principle (P), then A is dispersive.

Proof. By $A-I I$, Prop. 2.9 , it is enough to show that $A_{0}:=A \mid D_{0}$ is dispersive.\\
Let $f \in D_{O}$ and $\phi \in \mathrm{dN}^{+}(f)$. Then $\phi \in E_{+}^{\prime},\|\phi\| \leqq 1$ and $\langle f, \phi>=$ $\left\|\mathbf{f}^{+}\right\|$. Hence, $\left\langle\mathrm{f}^{-}, \phi\right\rangle=\left\langle\mathrm{f}^{-}, \phi\right\rangle+\langle\mathrm{E}, \phi\rangle-\left\|\mathrm{f}^{+}\right\|=\left\langle\mathrm{f}^{+}, \phi\right\rangle-\left\|\mathrm{f}^{+}\right\| \leqq 0$. Thus $\left\langle\mathrm{f}^{-}, \phi\right\rangle=0$. Consequently, $\left\langle\mathrm{f}^{+}, \phi\right\rangle=\langle\mathrm{f}, \phi\rangle=\left\|\mathrm{f}^{+}\right\|$; and so $\phi \in \mathrm{dN}\left(\mathrm{f}^{+}\right)$. Since A is dissipative it follows that $\left\langle A \mathrm{f}^{+}, \phi>\leq 0\right.$. Moreover, since $A$ satisfies ( P ) we have $<\mathrm{Af}^{-}, \phi>\geqq 0$. So we finally obtain, $\langle A f, \phi\rangle=\left\langle A f^{+} \phi\right\rangle-\left\langle A f^{-}, \phi\right\rangle \leqq 0$.

Proof of Theorem 1.8. The operator A - W satisfies (P) as well. So it follows from Proposition 1.9 that $A$ - w is dispersive. Consequently, the semigroup $\left(e^{-w t_{T}} T(t)\right)_{t \geqq 0}$, which is generated by $A-w$, is positive. Thus $(T(t))_{t \geqq 0}$ is positive as well.

Next we give a reformulation of the positive minimum principle. For $0<u \in E_{+}$we denote by $E_{u}$ the principal ideal generated by $u$. If $g \in E_{+}$, then $g \in \bar{E}_{u}$ if and only if $\lim _{n \rightarrow \infty}\|u-n u a g\|=0$.

Lemma 1.10. An operator $A$ on $E$ satisfies (P) if and only if (1.8) $(A u)^{-} \epsilon \overline{E_{u}}$ for all $u \in D(A)+:=D(A) \cap E_{+}$.

Proof. Let $u \in D(A), g=A u$. Assume that $(A u)^{-} \in \overline{E_{u}}$.\\
Then, if $0 \leqq \phi \in E_{+}^{\prime}$ such that $\langle u, \phi\rangle=0$ one has $\langle f, \phi\rangle=0$ for\\
\includegraphics[max width=\textwidth]{2024_12_23_c6487cc0859199a15bd9g-265} This proves one direction. To prove the other assume that $g^{-} \bar{E}_{u}^{-}$. Then there exists $\phi \in\left(E_{u}\right)^{\circ}$ such that $\left\langle g^{-}, \phi\right\rangle \neq 0$. since $\left(E_{u}\right)^{\circ}$ has a generating cone (by [Schaefer (1974),II,4.7]), we can assume that $\phi>0$.\\
Define $\psi_{O}(f)=\sup \phi\left([0, f] \cap E_{\left(g^{-}\right)}\right)$for $f \in E_{+}$. Then $\psi_{O}$ is positive homogeneous on $\mathrm{E}_{+} . \quad$ Thus the linear extension of $\psi_{0}$ defines a positive linear form $\psi$ on E . We have $\left\langle\mathrm{g}^{-}, \psi\right\rangle=\left\langle g^{-}\right.$, $\left.\phi\right\rangle$ $\rangle 0$ and $\left\langle g^{+}, \psi\right\rangle=0$. Thus $\langle A u, \psi\rangle=-\left\langle g^{-}, \psi\right\rangle\langle 0$. But $\langle u, \psi\rangle \leqq$ $\langle u, \phi\rangle=0$. Thus (P) does not hold.

Bounded generators of positive semigroups can now be characterized as follows.

Theorem 1.11. Let A be a bounded operator on a Banach lattice E. The following assertions are equivalent:\\
(i) $e^{t A} \geq 0 \quad(t \geqq 0)$.\\
(ii) $f \in E_{+}, \phi \in E_{+}^{\prime},\langle f, \phi\rangle=0$ implies $\langle A f, \phi\rangle \geq 0$.\\
(iii) (Af) ${ }^{-} \bar{E}_{f}$ for all $f \in D(A)+$.\\
(iv) $A+\|A\| \cdot I d \geqslant 0$.

Proof. It follows by Proposition 1.7 that (i) implies (ii). Since $\| e^{t A} \leqq e^{t\|A\|} \quad(t \geqq 0)$, (ii) implies (i) by Theorem 1.8. The equivalence of (ii) and (iii) is established by Lemma 1.10. If (iv) holds, then $e^{t(A+\|A\|)} \geqq 0 \quad(t \geqq 0)$. Thus $e^{t A}=e^{-t\|A\|} e^{t(A+\|A\|)} \geqq 0(t \geqq 0)$. We have shown that (i), (ii) and (iii) are equivalent and (iv) implies (i).\\
It remains to show that (i) implies (iv). Since assertions (i) and (iv) are satisfied for A if and only if they are satisfied for $\mathrm{A}^{\prime}$, we can assume that $E$ is order complete (considering $A^{\prime}$ instead of A if necessary). Assume that (i) holds. Then by what we have proved above (iii) holds as well. In particular\\
(1.9) $(\mathrm{Au})^{-} \in\{u\}^{\mathrm{dd}}$ for all $u \in E_{+}$.

Let $\lambda \geq 0$ and $f \in E_{+}$such that $g=(A+\lambda) f \geq 0$. We have to show that $\lambda \leq|A| \mid$. Denote by $P$ the band projection onto the band generated by $\mathrm{g}^{-}$. Then PAf $+\lambda \mathrm{Pf}=\mathrm{Pg}=\mathrm{g}^{-}<0$. Since by (1.9), $[A(I d-P) f]^{-} \epsilon$ (Id-P)E, it follows $0>\lambda P f+P A f=\lambda P f+$ PAPf + $P A(I d-P) f=\lambda P f+P A P f+P(A(I d-P) f)^{+} \geq \lambda P f+P A P f$.

Hence $0 \leqq \lambda P f<-P A P f$. This implies that $\operatorname{Pf} \neq 0$ and $\lambda\|P\|_{\|} \leqq\|P A P f\|$ $\leqq\|A\| \cdot\|P f\|$. Consequently $\lambda \leqq\|A\|$.

Remark 1.12. It follows from the proof of Theorem 1.11 that on a o -order complete Banach lattice condition (1.9) is equivalent to the positivity of the semigroup $\left(e^{t A}\right)_{t \geqslant 0}$.

Examples 1.13. Let $\mathrm{E}=\ell^{\mathrm{P}} \quad(1 \leqq \mathrm{p} \leqq \infty)$ or $\mathrm{E}=\mathrm{c}_{0}$.\\
a) An operator $A \in L(E)$ can be canonically represented by a matrix $\left(a_{i j}\right.$ ) . It follows from Thm. 1.11 that $e^{t A} \geqq 0$ for all $t \geqq 0$ if and only if $a_{i j} \geq 0$ whenever $i \neq j$.\\
b) Let $A$ be the generator of a strongly continuous contraction semigroup $(T(t))_{t \geq 0}$ on $E$. Suppose that the space $c_{o 0}$ of all sequences which vanish off a finite set is a core of A . Let $\left(a_{n m}\right)_{m \in \mathbb{N}}=\left(A e_{n}\right)$ where $e_{n}=\left(\delta_{n m}\right)_{m \in N}$ denotes the $n^{t h}$ unit vector. Then it follows from Thm.1.8 that the semigroup is positive if and only if $a_{n m} \geq 0$ whenever $n \neq m$.

\section*{2. KATO's INEQUALITY}
A strongly continuous semigroup on $C(K)$ ( K compact) or a norm continuous semigroup on an arbitrary Banach lattice is positive if and only if its generator A satisfies the positive minimum principle (P). However, we will see that in general (P) is not sufficient for the positivity of the semigroup. One reason seems to be that (P) involves merely positive elements in $D(A)$ but $D(A)+$ can be small if the semigroup is not positive (cf. Remark 3.16). Our aim in this section is to find a different condition on the generator which is necessary for the positivity of the semigroup.

We recall from Chapter $\mathrm{C}-\mathrm{I}, \mathrm{Sec} .8$ definition and properties of the signum operator.

Proposition 2.1. Let $E$ be a o-order complete (real or complex) Banach lattice. For every $\mathbf{f} \in \mathrm{E}$ there exists a unique linear operator (sign f) on E which satisfies

\begin{center}
\begin{tabular}{lll}
$(2.1)$ & $|(\operatorname{sign} f) g| \leqq|g|$ & $(g \in E)$ \\
$(2.2)$ & $(\operatorname{sign} f) g=0$ & if inf $\{|f|,|g|\}=0$ \\
$(2.3)$ & $(\operatorname{sign} \bar{f}) f=|f|$ & (where $\overline{\mathrm{E}}:=\operatorname{Ref}-\mathrm{iImf})$. \\
\end{tabular}
\end{center}

The operator (siggn f) (which is non-linear in general) is defined by (2.4) $\quad(\operatorname{sign} \mathrm{f}) \mathrm{g}=(\operatorname{sign} \mathrm{f}) \mathrm{g}+(I d-\mathrm{P}|\mathrm{f}|)|\mathrm{g}|$\\
where for $h \in E_{+}$we denote by $P_{h}$ the band projection onto the band $\{h\}^{d d}$ generated by $h$.

If $\mathbf{E}$ is a real o-order complete Banach lattice, then\\
(2.5) $\operatorname{sign} f=P_{\left(f^{+}\right)}-P_{\left(f^{-}\right)}$.

Example 2.2. Let $f \in E:=L^{p}(X, \Sigma, \mu)$ (real or complex) where $(x, \Sigma, \mu)$ is a $\sigma$-finite measure space and $1 \leqq p \leqq \infty$. Define

$$
m(x)= \begin{cases}f(x) /|f(x)| & \text { if } f(x) \neq 0 \\ 0 & \text { if } f(x)=0 .\end{cases}
$$

Then sign f is the multiplication operator defined by m ; i.e.,

Moreover,

$$
\begin{array}{ll}
(\operatorname{sign} f) g=m \cdot g & (g \in E), \\
(\operatorname{sign} f) g=m \cdot g+1_{[f(x)=0]}|g| & (g \in E) .
\end{array}
$$

The operator siĝn f is related to the Gateaux-derivative (B-II,Definition 3.2) of the modulus. We explain this by an example.

Example 2.3. Let $E$ be the real or complex space $\mathrm{L}^{\mathrm{P}}(\mathrm{X}, \Sigma, \mu)$ where $(X, \Sigma, \mu)$ is a o-finite measure space and $1 \leqq p<\infty$. Let $f, g \in E$ and $\mathrm{x} \in \mathrm{X}$. Then by B-II,Lemma 2.4\\
$\lim _{t \nmid 0} 1 / t(|f(x)+\operatorname{tg}(x)|-|f(x)|)= \begin{cases}\operatorname{Re}(\operatorname{sign} \overline{f(x)}) g(x) & \text { if } f(x) \neq 0 \\ |g(x)| & \text { if } f(x)=0 .\end{cases}$\\
If $\theta: E \rightarrow E_{+}$denotes the modulus function given by $\theta(h)=|h|$, then it follows from the dominated convergence theorem that $\theta$ is right-sided Gateaux-differentiable and


\begin{equation*}
D_{g} \in(f)=\operatorname{Re}(\operatorname{sign} \bar{f}) g . \tag{2.6}
\end{equation*}


Later we will see that (2.6) holds in every Banach lattice with order continuous norm.

Now let $A$ be the generator of a strongly continuous positive semigroup $(T(t))_{t \geqq 0}$. The positivity of the semigroup is equivalent to\\
(2.7) $|T(t) f| \leqq T(t)|f| \quad(t \geqq 0, f \in E)$.

In order to deduce from (2.7) a property for the generator A it is natural trying to differentiate $(2.7)$ in $t=0$. Let us assume for a moment that $E=\mathrm{L}^{\mathrm{P}}(\mathrm{X}, \Sigma, \mu)$ (as in Ex. 2.3). Let $f \in D(A)$ and $0 \leqq \phi \in D\left(A^{\prime}\right)$. Then by (2.7),\\
(2.8) $\langle | T(t) f|, \phi\rangle \leqq\langle T(t)| f|, \phi\rangle \quad(t \geqq 0)$\\
where the equality holds for $t=0$. Hence the inequality remains valid if we differentiate in 0 on both sides of (2.8).\\
Since $\phi \in D\left(A^{\prime}\right)$ we obtain $d /\left.d t\right|_{t=0}\langle T(t)| f|, \phi\rangle=\langle | f\left|, A^{\prime} \phi\right\rangle$ on the right side. By (2.6) and the chain rule B-Ix,Prop.2.3 one obtains $d / d t|t=0| T(t) f \mid=\operatorname{Re}((\operatorname{sign} \bar{f}) A f)$ on the left side.\\
\includegraphics[max width=\textwidth, center]{2024_12_23_c6487cc0859199a15bd9g-268}\\
(K) $\operatorname{Re}\langle(\operatorname{sign} \overline{\mathrm{f}}) \mathrm{Af}, \phi\rangle \leqq\langle | f\left|, A^{\prime} \phi\right\rangle \quad\left(f \in D(A), 0 \leqq \phi \in \mathrm{D}^{\prime}\left(A^{\prime}\right)\right.$.

We refer to this as Kato's inequality, since it represents an abstract version of the classical inequality proved by Kato for the Laplacian (see Example 2.5).\\
We will see in the next section that, together with an additional condition, this inequality is characteristic for the positivity of the semigroup.

By a different proof, we now show that Kato's inequality holds for generators of positive semigroups in general.

Theorem 2.4. The generator A of a strongly continuous positive semigroup (T(t)) ${ }_{t \geq 0}$ on a o-order complete (real or complex) Banach lattice E satisfies Kato's inequality; i.e.,\\
(K) $\quad \operatorname{Re}<(\operatorname{sign} \bar{f}) A f, \phi\rangle \leqq<|f|, A^{\prime} \phi>\quad$ (f $\left.\in D(A), 0 \leqq \phi \in D\left(A^{\prime}\right)\right)$.

Proof. Let $f \varepsilon D(A), 0 \leqq \phi \varepsilon D\left(A^{\prime}\right)$. Then

$$
\begin{aligned}
<(\operatorname{sign} \overline{\mathrm{f}}) \mathrm{Af}, \phi> & =\lim _{t \rightarrow 0} 1 / t<(\operatorname{sign} \overline{\mathrm{f}})(\mathrm{T}(t) \mathrm{f}-\mathrm{f}), \phi> \\
& =\lim _{t \rightarrow 0} 1 / t<(\operatorname{sign} \overline{\mathrm{f}}) \mathrm{T}(t) \mathrm{f}-|\mathrm{f}|, \phi> \\
& \leq \lim _{t \rightarrow 0} 1 / t<|T(t) \mathrm{f}|-|f|, \phi>
\end{aligned}
$$

$$
\begin{aligned}
& \leqq \lim _{t \rightarrow 0} 1 / t\langle T(t)| f|-|f|, \phi\rangle \\
& =\lim _{t \rightarrow 0}\langle | f\left|, 1 / t\left(T(t)^{\prime} \phi-\phi\right)\right\rangle \\
& =\langle | f\left|, A^{\prime} \phi\right\rangle .
\end{aligned}
$$

Let $D\left(A^{\prime}\right)_{+}=E_{+}^{\prime} \cap D\left(A^{\prime}\right)$. Consider the condition\\
(2.9) $\quad{\overline{D\left(A^{\top}\right)_{+}}}^{\sigma\left(E^{\prime}, E\right)}=E_{+}^{\prime}$\\
(which is satisfied if the semigroup is positive). If (K) and (2.9) hold, then Kato's inequality holds in the strong form as well, wherever it makes sense; i.e.,\\
(2.10) $\operatorname{Re}((\operatorname{sign} \bar{f}) A f) \leqq A|f| \quad$ (whenever $f,|f| \in D(A)$ ).

Example 2.5. Kato's inequality in its classical form says the following (see Kato (1973) or [Reed-Simon (1975); X.27]).\\
Let $f \in \operatorname{L}_{10 c}^{1}\left(\mathbb{R}^{\mathrm{n}}\right)$ be such that the distributional Laplacian satisfies $\Delta f \in \mathrm{I}_{l o c}^{I}\left(\mathbb{R}^{\mathrm{n}}\right)$. Then the inequality

$$
\operatorname{Re}((\operatorname{sign} \bar{f}) \Delta f) \leqq \Delta|f|
$$

holds in the sense of distributions; i.e.,\\
$\langle\phi, \operatorname{Re}((\operatorname{sign} \overline{\mathrm{f}}) \Delta f)\rangle \leq\langle\phi, \Delta| f| \rangle(=\langle\Delta \phi| f \mid,>)$ holds for all\\
$0 \leq \phi \in C_{C}^{\infty}\left(\mathbb{R}^{n}\right)$. Note that the closure of $\Delta$ defined on $C_{c}^{\infty}\left(\mathbb{R}^{n}\right)$ generates a strongly continuous positive semigroup on $L^{\mathrm{P}}\left(\mathbb{R}^{\mathrm{n}}\right)$\\
(1 $\leqq \mathrm{p}<\infty$ ) (see Example 1.5.d and Example 4.7)).

We want to discuss the relation between the classical (distributional) inequality and our version given in Theorem 2.4.

Let

$$
A=\sum|\alpha| \leq m a_{\alpha} D^{\alpha}
$$

be a differential operator, where $a_{\alpha} \in C_{c}^{\infty}\left(\mathbb{R}^{n}\right)$. Here we let\\
$\mathrm{D}^{\alpha}=\left(\partial / \partial \mathrm{x}_{1}\right)^{\alpha}{ }_{1} \ldots\left(\partial / \partial \mathrm{x}_{\mathrm{n}}\right)^{\alpha} \mathrm{n}$ for all multi-indices $\alpha=\left(\alpha_{1}, \ldots, \alpha_{n}\right)$ $\in \mathbb{N}_{0}^{n} \quad\left(\mathbb{N}_{0}:=\mathbb{N U}\{0\}\right)$ of order $|\alpha|:=\alpha_{1}+\ldots+\alpha_{n}$.\\
We say that A satisfies Kato's inequality in the sense of distributions if\\
$\left(K_{d}\right) \quad \operatorname{Re}\left\langle((\operatorname{sign} \bar{f}) A f, \phi\rangle \leqq\langle | f \mid, A^{*} \phi\right\rangle$ for all $f \in C_{C}^{\infty}\left(\mathbb{R}^{n}\right), 0 \leqq \phi \in C_{C}^{\infty}\left(\mathbb{R}^{n}\right)$, where $A^{*}$ denotes the formal adjoint of A .\\
Let now $A$ be the generator of a positive semigroup (T) $(t))_{t \geq 0}$ on $E:=L^{P}\left(\mathbb{R}^{n}\right) \quad(1 \leq p<\infty)$ or $c_{0}\left(\mathbb{R}^{n}\right)$. Assume that there exists a core\\
$D_{0}$ of $A$ such that $C_{c}^{\infty} C_{0}$ and $A f=A f$ in the sense of distributions for all $f \in D_{0}$. Then $A$ satisfies Kato's inequality in the sense of distributions.\\
In fact, let $0 \leq \phi \in C_{C}^{\infty}\left(\mathbb{R}^{n}\right)$. Then $\langle A f, \phi\rangle=\langle A f, \phi\rangle=\left\langle f, A^{*} \phi\right\rangle$ for all $\pounds \in D_{0}$. Since $D_{0}$ is a core of $A$, this implies that\\
$\phi \in D^{\prime}\left(A^{\prime}\right)$ and $A^{\prime} \phi=A^{*} \phi$. Thus $(K)$ gives $\operatorname{Re}<((\operatorname{sign} \bar{f}) A f, \phi>=$ $\operatorname{Re}\langle((\operatorname{sign} \bar{f}) A f), \phi\rangle \leq\langle | f\left|, A^{\prime} \phi\right\rangle=\langle | f\left|, A^{*} \phi\right\rangle=\langle A| f|, \phi\rangle$ for all\\
$\pounds \in C_{C}^{\infty}\left(\mathbb{R}^{n}\right), 0 \leqq \phi \in C_{C}^{\infty}\left(\mathbb{R}^{n}\right)$. This is Kato's inequality in the distributional sense.

Remark. It has been proved by Miyajima and Okasawa (1984) that ( $\mathrm{K}_{\mathrm{d}}$ ) implies that $m \leqslant 2$ and that the principal part $A_{0}=\left.\sum_{\alpha}\right|_{=2} a_{\alpha} D^{\alpha}$ of A is elliptic; i.e., if we write the operator $A_{0}$ in the form $A_{0}=\sum_{i, j=1} b_{i j} \partial^{2} / \partial x_{i} \partial x_{i j}$, then the matrix $\left(b_{i j}(x)\right)$ is positive semidefinite for all $x \in \mathbb{R}^{n}$.

Finally we formulate Theorem 2.4 for the space $\mathrm{E}:=\mathrm{C}_{0}(\mathrm{X})$ ( X locally compact) (which is not o-order complete unless $x$ is $\sigma-5$ tonian). Recall, for $f \in C_{0}(X)$, sign $f$ is defined as a Borel function and for any bounded Borel function $g$ on $X$ and any $\phi \in M(X)=C_{0}(X)^{\prime}$ we let $\langle g, \phi\rangle=\int g(x) d \phi(x)$ (see B-II, Sec.2).

Theorem 2.6. Let $X$ be a locally compact space and $A$ be the generator of a strongly continuous positive semigroup on $C_{0}(X)$. Then\\
(K) $\quad \operatorname{Re}\langle(\operatorname{sign} \overline{\mathrm{f}}) \mathrm{Af}, \phi\rangle \leq\langle | f\left|, A^{\prime} \phi\right\rangle \quad\left(f \in D(A), \phi \in D^{\prime} A^{\prime}\right)$. .

The proof of Theorem 2.4 can be taken over literally. Also the analogue of the proof given for $\mathrm{L}^{\mathrm{P}}$-spaces (preceding Theorem 2.4) is valid if one uses B-II, Lemma 2.6.

\section*{3. A CHARACTERIZATION OF GENERATORS OF POSITIVE SEMIGROUPS}
In this section we confine ourselves to real Banach lattices. This does not mean a restriction since every positive semigroup on a complex Banach lattice leaves the real part of the space invariant.

Remark 3.1. Let $(S(t))_{t \geqq 0}$ be a semigroup on a complex Banach lattice $E$ with generator A. Then $S(t) E_{\mathbb{R}} \subset E_{\mathbb{R}}$ for all $t \geqq 0$ if and only if\\
(3.1) $f \in D(A)$ implies $\bar{f} \in D(A)$ and $A \bar{f}=(A f)^{-}$.

In that case the generator $A_{\mathbb{R}}$ of the restriction semigroup on $E_{\mathbb{R}}$ is given by $A_{\mathbb{R}} f=A f$ on $D\left(A_{\mathbb{R}}\right)=D(A) \cap E_{R}$.

We will see below that for generators of a strongly continuous semigroup Kato's inequality alone is not sufficient to ensure the positivity of the semigroup. So we introduce another condition.

Definition 3.2. A subset $M^{\prime}$ of $E^{\prime}$ is called strictly positive if for every $f \in \mathrm{E}_{+},\langle f, \phi\rangle=0$ for all $\phi \in \mathrm{M}^{\prime}$ implies $\mathrm{f}=0$. Accordingly, an element $\phi$ of $E_{+}^{\prime}$ is called strictly positive if the set $\{\phi\}$ is strictly positive.

Example 3.3. Let $E=L^{\mathrm{P}}(\mathrm{X}, \mu) \quad(1 \leqq \mathrm{P}<\infty)$, where $(\mathrm{X}, \mu)$ is a $\sigma$-finite measure space. Then $\phi \in E^{\prime}=L^{q}(\mathrm{X}, \mu) \quad$ (where $1 / p+1 / q=1$ ) is strictly positive if and only if $\phi(x)>0$-a.e. Note that strictly positive elements of E' always exist in this case.

Definition 3.4. Let $B$ be an operator on a Banach lattice $F$ and let $u \in F$. Then $u$ is called a positive subeigenvector of $B$ if\\
a) $0<u \in D(B)$ and\\
b) $\mathrm{Bu} \leqq \lambda u$ for some $\lambda \in \mathbb{R}$.

Proposition 3.5. Let $(T(t))_{t \geq 0}$ be a positive semigroup on a real Banach lattice with generator A . Then there exists a strictly positive set M' of subeigenvectors of $A^{\prime}$ (the adjoint of the generator A ). Moreover, if there exist strictly positive linear forms on E , then there exists a strictly positive subeigenvector of $A^{\prime}$.

Proof. Let $\lambda>0$ be such that $R(\lambda, A)=(\lambda-A)^{-1}$ exists and such that $R(\lambda, A) \geqq 0$. Let $N^{\prime} \mathbb{E}_{+}^{\prime}$ be strictly positive. Then $M^{\prime}:=\left\{R(\lambda, A)^{\prime} \psi: \psi \in N^{\prime}\right\} \subset D\left(A^{\prime}\right)_{+}$. We show that $M^{\prime}$ is strictly positive. Indeed, let $f \in \mathrm{E}_{+}$such that $\left.<\mathrm{f}, \phi\right\rangle=0$ for all $\phi \in \mathrm{M}^{\prime}$. Then $\langle R(\lambda, A) f, \psi\rangle=0$ for all $\psi \in N^{\prime}$. Hence $R(\lambda, A) f=0$ since $N^{\prime}$ is strictly positive. Consequently, $f=0$. The set M' consists of\\
subeigenvectors of $A^{\prime}$. In fact, let $\psi \in N^{\prime}, \phi=R(\lambda, A)^{\prime} \psi$. Then $A^{\prime} \phi=\lambda \phi-\psi \leqq \lambda \phi$.

The fact that $\phi \in D\left(A^{\prime}\right)$, is a subeigenvector can be expressed by the semigroup (if it is positive).

Proposition 3.6. Assume that $A$ is the generator of a positive semigroup $(T(t))_{t \geq 0}$ on a real Banach lattice $E$. Let $\phi \in D\left(A^{\prime}\right)_{+}$and $\lambda \in \mathbb{R}$. Then

$$
A^{\prime} \phi \leqq \lambda \phi \quad \text { if and only if } T(t)^{\prime} \phi \leqq e^{\lambda t_{\phi}} \quad(t \geqq 0)
$$

Proof. If $T(t) \phi \leqq e^{\lambda t} \phi$ for all $t \geqq 0$, then\\
$A^{\prime} \phi=\sigma\left(E^{\prime}, E\right)-\lim _{t \rightarrow 0} 1 / t\left(T(t)^{\prime} \phi-\phi\right) \leqq \lim _{t \rightarrow 0} 1 / t\left(e^{\lambda t} \phi-\phi\right)=\lambda \phi$.\\
For the converse let $f \in E_{+}$. Then

$$
\begin{aligned}
\left\langle f, T(t)^{\prime} \phi\right\rangle & =\langle f, \phi\rangle+\int_{0}^{t}\left\langle f, T(s)^{\prime} A^{\prime} \phi\right\rangle d s \\
& \leqq\langle f, \phi\rangle+\lambda \int_{0}^{t}\left\langle f, T(s)^{\prime} \phi\right\rangle d s .
\end{aligned}
$$

It follows from Gronwall's lemma that $\left\langle f, T(t)^{\prime} \phi\right\rangle \leqq e^{\lambda t}\langle f, \phi\rangle$.

Remark 3.7. a) Using Prop. 3.6 it is immediately clear that $(\mathrm{T}(\mathrm{t}))_{t \geq 0}$ is irreducible if and only if every positive subeigenvector of A, is strictly positive (cf. C-III, Def.3.1).\\
b) In the proof of the "only if" - part of Prop. 3.6 we needed the positivity of the semigroup in order to be able to apply Gronwall's lemma. However, if instead of assuming that the semigroup is positive we suppose that $A$ satisfies Kato's inequality and $A^{\prime} \phi \leqq \lambda \phi$ for some strictly positive $\phi \in D\left(A^{\prime}\right)$ then we will show that $T(t){ }^{\prime} \phi \leqq$ $e^{\lambda t} \phi$ and that the semigroup is positive (see cor.3.9).

The following is our characterization.

Theorem 3.8. Let $(T(t))_{t \geqq 0}$ be a semigroup on a o-order complete real Banach lattice $E$. The semigroup is positive if and only if its generator A satisfies the following cordition.\\
There exists a core $D_{0}$ of $A$ and a strictly positive set $M^{\prime}$ of subeigenvectors of $A^{\prime}$ such that\\
(K)

$$
\left\langle(s i g n \text { f) } A f, \phi\rangle \leqq\langle | f \mid, A^{\prime} \phi\right\rangle
$$

for all $f \in D_{O}, \phi \in M^{\prime}$.

Corollary 3.9. Assume in addition that $\mathrm{E}^{\prime}$ contains a strictly positive functional. Then the semigroup is positive if and only if there exists a core $D_{0}$ of $A$ and a strictly positive subeigenvector $\phi$ of $A^{\prime}$ such that\\
(K)

$$
<(\operatorname{sign} \text { f) } A f, \phi\rangle \leqq\langle | f\left|, A^{\prime} \phi\right\rangle \text { for all } f \in D_{0} \text {. }
$$

From the proof of Theorem 3.8 we isolate the following

Proposition 3.10. Let $B$ be a densely defined operator on $E$ and $D_{0}$ be a core of $B$. Suppose that $\phi \in D\left(B^{\prime}\right)_{+}$is such that $B^{\prime} \phi \leq 0$. Denote by $p$ the sublinear functional given by $p(f)=\left\langle f^{+}, \phi\right\rangle$. If (K) <(sign f) $B f, \phi>\leqq|f|, B^{\prime} \phi>\quad\left(f \in D_{O}\right.$ ), then $B$ is p-dissipative.

Proof. Let $f \in D_{O}$. Set $P_{+}:=P_{f}^{+}, P_{-}:=P_{f}^{-}$and let $P:=I d-P_{+}-P_{-}, Q=P_{+}+1 / 2 P$ and $\psi=Q^{\prime} \phi$. We show that (3.2) $\psi \in \operatorname{dp}(f)$.

Let $g \varepsilon E$. Since $0 \leqq Q \leqq \mathrm{Id}$ we have $\langle g, \psi\rangle=\langle Q g, \phi\rangle \leqq\left\langle Q g^{+}, \phi\right\rangle \leqq$ $\left\langle\mathrm{g}^{+}, \phi\right\rangle=\mathrm{p}(\mathrm{g})$. Moreover, $\langle\mathrm{f}, \psi\rangle=\langle Q f, \phi\rangle=\left\langle\mathrm{P}_{+} \mathrm{f}+1 / 2 \mathrm{Pf}, \phi\right\rangle=\left\langle\mathrm{f}^{+}, \phi\right\rangle$ $=p\left(f^{+}\right)$. So (3.2) follows by the definition of dp(f). We show that (3.3) $\langle B f, \psi\rangle \leqq 0$.

One has trivially\\
(3.4) $\left\langle\left(\mathrm{P}_{+}+\mathrm{P}_{-}+\mathrm{P}\right) \mathrm{Bf}, \phi\right\rangle=\left\langle\mathrm{f}, \mathrm{B}^{\prime} \phi\right\rangle$.

Addition of (3.4) and (K) gives\\
$\left\langle\left(2 \mathrm{P}_{+}+\mathrm{P}\right) \mathrm{Bf}, \phi\right\rangle \leqq\left\langle 2 \mathrm{f}^{+}, \mathrm{B}^{\prime} \phi\right\rangle \leqq 0$. Hence $\langle\mathrm{Bf}, \psi\rangle=\langle Q B f, \phi\rangle \leqq 0$.\\
Thus we have proved that $B_{1}{ }_{0}$ is p-dissipative. Hence $B$ is p-dissipative as well (by A-II, Cor.2.5) .

Proof of Theorem 3.8. Proposition 3.5 and Theorem 2.4 yield one implication. In order to show the other assume that the condition in Theorem 3.8 is satisfied. We have to show that $T(t) \geqq 0$ for all $t \geqq 0$.\\
Let $\phi \in M^{\prime}$. Consider the half-norm $p(f)=\left\langle f^{+}, \phi\right\rangle$ and the operator $B=A-\lambda$, where $\lambda \in \mathbb{R}$ is such that $A^{\prime} \phi \leqq \lambda \phi$. Then $B$ satisfies $B^{\prime} \phi \leqq 0$ and $(K)$ as well. So it follows from Proposition 3.10 that B is p-dissipative.\\
Since B generates the semigroup $\left(e^{-\lambda t} T(t)\right)_{t \geq 0}$ we obtain from

A-II, Thm.2.6 that $p\left(e^{-\lambda t_{T}}(t) f\right) \leqq p(f)$ (f $\left.\in E, t \geqq 0\right)$. Hence, (3.5) <(T(t)f) ${ }^{+}, \phi>\leqq e^{\lambda t}\left\langle\mathrm{f}^{+}, \phi>\quad(f \in E, t \geqq 0)\right.$.

Now let $t>0$ and $f \leq 0$; then $\mathrm{f}^{+}=0$. It follows from (3.5) that $\left\langle(T(t) f)^{+}, \phi\right\rangle \leqq 0$. Since $\phi \in M^{\prime}$ is arbitrary and $M^{\prime}$ is strictly positive, it follows that $(T(t) f)^{+}=0$; i.e., $T(t) f \leqq 0$. This implies that $\mathrm{T}(\mathrm{t}) \geq 0$.

Remark 3.11. a) The proof of Theorem 3.8 shows the following. If A is the generator of a positive semigroup and $E^{\prime}$ contains strictly positive linear forms, then there exist a continuous half-norm $p$ on $E$ and $w \in \mathbb{R}$ such that $A-w$ is p-dissipative. We stress that $p$ cannot be replaced by the norm (or by $\mathrm{N}^{+}$), since in general none of the semigroups $\left(e^{-w t} \mathrm{~T}(t)\right)_{t \geq 0} \quad(w \in \mathbb{R})$ is contractive for the norm (cf. Derndinger (1984) and Batty-Davies (1982)).\\
b) Using Proposition 3.10 one can show with the help of the proof of A-II, Prop. 2.9 that a densely defined operator is closable whenever there exists a strictly positive set $M^{\prime}$ of subeigenvectors of $A^{\prime}$ such that $(K)$ holds for all $f \in D(A)$ and $\phi \in M^{\prime}$.

Remark 3.12 In Theorem 3.8 and Corollary 3.9 one can replace inequality (K) by the inequality


\begin{equation*}
\left\langle\mathrm{P}\left(\mathrm{f}^{+}, \mathrm{Af}, \phi\right\rangle \leqq\left\langle\mathrm{f}^{+}, \mathrm{A}^{\prime} \phi\right\rangle,\right. \tag{3.6}
\end{equation*}


(with the notation of Prop.3.10).

Indeed, (3.6) for -f yields $\left\langle-P\left(f^{-}\right)\right.$Af, $\phi>\leq\left\langle f^{-}, A^{\prime} \phi\right\rangle$. Adding up both inequalities one obtains <(sign f)Af, $\phi\rangle \leq\langle | f\left|, A^{\prime} \phi\right\rangle$.\\
On the other hand, if A generates a positive semigroup, one sees by the obvious alterations in the proof of Theorem 2.4 that (3.6) holds for all $f \in D(A)$ and $\phi \in D\left(A^{\prime}\right)_{+}$.\\
Next we formulate the result for the space $C_{0}(x)$, where $x$ is a locally compact space (concerning the notation cf. Thm. 2.6 and Sec. 2 of $B-I I)$.

Theorem 3.13. Let $A$ be the generator of a semigroup on $C_{0}(X)$. The semigroup is positive if and only if there exists a core $D_{0}$ of A and a strictly positive set $M^{\prime}$ of subeigenvectors of $A^{\prime}$ such that\\
(K) <(sign f)Af, $\phi\rangle \leq\langle | f\left|, A^{\prime} \phi\right\rangle$ for all $f \in D_{0}, \phi \in M^{\prime}$.

This theorem can be proved in the same way as Theorem 3.8.

Remark. If x is separable, then there exist strictly positive measures on $\mathrm{C}_{\mathrm{O}}(\mathrm{X})$. In that case the analogue of Corollary 3.9 holds as well.

Now we want to discuss the results obtained so far.

As a first example we consider the first derivative with boundary conditions on $E=L^{\mathrm{P}}[0,1](1 \leqq \mathrm{P}<\infty)$. $\mathrm{By} \quad \mathrm{AC}[0,1]$ we denote the space of all absolutely continuous functions on $[0,1]$. Let $A_{\text {max }}$ be given by

$$
\begin{aligned}
D\left(A_{\max }\right) & =\left\{f \in A C[0,1]: f^{\prime} \in \mathrm{L}^{P}[0,1]\right\} \\
A_{\max } f & = \pm, \quad\left(f \in D\left(A_{\max }\right)\right) .
\end{aligned}
$$

The following lemma is easy to prove.

Lemma 3.14. Let $f \in \operatorname{AC}[0,1]$. Then $|f| \in \operatorname{AC}[0,1]$ and $|f|^{\prime}=\left(\right.$ sign f) $\cdot f^{\prime}$ (a.e.).

As a consequence of the lemma, $D\left(A_{\max }\right)$ is a sublattice of $E$ and (3.7) $(\operatorname{sign} f) A_{\max } f=A_{\max }|f| \quad\left(f \in D\left(A_{\max }\right)\right)$. For $\lambda>0$ one has\\
(3.8) $\operatorname{ker}\left(\lambda-A_{\max }\right)=\mathbb{R} \cdot e_{\lambda} \quad$ where $e_{\lambda}(x)=e^{\lambda x}$.

Hence $A_{\text {max }}$ is not a generator. We impose the following boundary conditions.

Let $d \in \mathbb{R}$. Consider the restriction $A_{d}$ of $A_{\max }$ to the domain

$$
D\left(A_{d}\right)=\left\{f \in D\left(A_{\text {max }}\right): f(1)=d f(0)\right\}
$$

Then $A_{d}$ is the generator of the semigroup $\left(T_{d}(t){ }_{t \geqslant 0}\right.$ given by (3.9) $T_{d}(t) \pounds(x)=d^{n} \cdot f(x+t-n)$ if $x+t \in[n, n+1) \quad(n \in \mathbb{N})$. This is not difficult to prove. Actually (3.9) defines a group if $d \neq 0$ and if we let $t \in \mathbb{R}, \mathrm{n} \in \mathbb{Z}$. For $\mathrm{d}=0$ one obtains the nilpotent shift semigroup on $E$. It follows from (3.9) that the semigroup $\left(T_{d}(t)\right)_{t \geqq 0}$ is positive if and only if $d \geqq 0$.

Let us fix $d<0$. Let $A=A_{d}$ and $T(t)=T_{d}(t)$ for $t \geq 0$. Then\\
(T(t)) $t \geqslant 0$ is a semigroup which is not positive . Nevertheless its generator A satisfies Kato's inequality. Even the equality is valid; i.e.,\\
$(3.10)<(\operatorname{sign} f) \mathrm{Af}, \phi\rangle=\langle | f\left|, \mathrm{~A}^{\prime} \phi\right\rangle$ for all $f \in D(A), 0 \leq \phi \in D\left(A^{\prime}\right)$.

Proof. It is not difficult to see that


\begin{align*}
D\left(A^{\prime}\right) & =\left\{\phi \in \mathrm{AC}[0,1]: \phi^{\prime} \in \mathrm{I}^{q}[0,1], \phi(0)=d \phi(1)\right\}  \tag{3.11}\\
A^{\prime} \phi & \left.=-\phi^{\prime} \text { for all } \phi \in D^{\prime}\right) .
\end{align*}


where $1 / p+1 / q=1$. Let $\phi \in D\left(A^{\prime}\right)+$. Since $d<0$, it follows that $\phi(0)=\phi(1)=0$. Hence for $f \in D(A)$, $\left\langle(\right.$ sign f) $\left.A f, \phi\rangle=\left\langle(\operatorname{sign} f) f^{\prime}, \phi\right\rangle=\langle | f \mid ', \phi\right\rangle$

$$
\begin{aligned}
& =\int_{0}^{1}|f| '(x) \phi(x) d x \\
& =|f(1)| \phi(1)-|f(0)| \phi(0)-\int_{0}^{1}|f(x)| \phi^{\prime}(x) d x \\
& =|f(1)| \phi(1)-|f(0)| \phi(0)+\langle | f\left|, A^{\prime} \phi\right\rangle \\
& =\langle | f\left|, A^{\prime} \phi\right\rangle
\end{aligned}
$$

Remark 3.15. The equality (3.10) does not hold for all $\phi \in D\left(A^{\prime}\right)$. In fact, this would imply that $|f| \in D(A)$ and (sign f)Af $=A|f|$ for all $f \in D(A)$. Thus by Cor. 5.8 below the semigroup would be positive. The reason why in this example the equality holds will be explained from a more general point of view in Section 5 (see Rem.5.12).

Relation (3.10) shows that A also satisfies Kato's inequality formally in the strong sense. In order to formulate this more precisely, observe that it follows from (3.8) that $D\left(A_{\max }\right)=$ $D(A)+\mathbb{R} \cdot e_{\lambda}$ (for any fixed $0<\lambda \varepsilon \rho(A)$ ). Thus the extension $A_{\max }$ of A satisfies the following.\\
(3.12) A $A_{\max }$ is closed.\\
(3.13) $D\left(A_{\max }\right)$ is a sublattice of $E$.\\
(3.14) $D(A)$ has codimension one in $D\left(A_{\max }\right)$.\\
(3.15) $\quad(\operatorname{sign} f) A f=A_{\max }|f|$ for all $f \in D(A)$.

It is also remarkable that there exists a dense sublattice $D_{O}:=\{f \in D(A): f(0)=f(1)=0\}$ of $E$ which is included in $D(A)$. But $D_{0}$ is not a core of A (this would imply the positivity of the semigroup by Thm.1.8 if $|d| \leqq 1$ ).

Since (T(t)) $t \geq 0$ is not positive but Kato's inequality holds, it follows from Theorem 3.8 that there does not exist a strictly positive subeigenvector of $A^{\prime}$. In fact, even the following is true.\\
(3.16) $0 \leqq \phi \in D\left(A^{\prime}\right), A^{\prime} \phi \leqq \mu \phi$ for some $\mu \in \mathbb{R}$ implies $\phi=0$.

Proof. Suppose that $0 \leqq \phi \in D\left(A^{\prime}\right)$ such that $-\phi^{\prime}=A^{\prime} \phi \leq \mu \phi$. We can assume that $\mu \geqq 0$. Let $\psi(x)=\phi(1-x)$. Then $\psi^{\prime}(x)=-\phi^{\prime}(1-x) \leqq$ $\mu_{\phi}(1-x)=\mu_{\psi}(x)$. Since $\psi(0)=0$, we obtain $\psi(x)=\int_{0}^{x} \psi^{\prime}(y) d y \leqq \mu \int_{0}^{x} \psi(y) d y \quad(x \in[0,1])$. It follows from Gronwall's lemma that $\psi \leqslant 0$. Hence $\phi=\psi=0$.

Remark 3.16. Let $B$ be the generator of a strongly continuous semigroup on a real Banach lattice with order continuous norm. Assume that the following two conditions hold.


\begin{equation*}
\left\langle(\operatorname{sign} f) B f_{,} \phi\right\rangle \leqq\langle | f\left|, B^{\prime} \phi\right\rangle \quad\left(f \in D(B), \phi \in D\left(B^{\prime}\right)_{+}\right) . \tag{K}
\end{equation*}



\begin{equation*}
\left(D\left(B^{\prime}\right)_{+}\right)^{-\sigma\left(E^{\prime}, E\right)}=E_{+}^{\prime} . \tag{3,17}
\end{equation*}


Because of (3.17) condition (K) implies that $\mathrm{P}_{\mathrm{f}} \mathrm{Bf} \leqq$ (sign f) Bf $\leqq \mathrm{Bf}$ whenever $\mathrm{f} \in \mathrm{D}(\mathrm{B})_{+} \cdot$\\
This is Kato's inequality in the strong form for positive $f \in D(B)$ and is equivalent to $(B f)^{-} \in\{f\}^{d d}=\bar{E}_{f}^{-} \quad$ ( $f \in D(A)+$ (recall that $E$ has order continuous norm) . By Lemma 2.5 this again is equivalent to (P) $\quad 0 \leqq f \in D(B), \phi \in E_{+}^{\prime},\langle f, \phi\rangle=0$ implies $\langle B f, \phi\rangle \geqq 0$.

It is easy to see that the operator $A$ in the example satisfies conditions ( $K$ ) and (3.17). Thus the positive minimum principle (P) is not sufficient for the positivity of the semigroup.

In view of the preceding example and remarks one might presume that the existence of a strictly positive set of subeigenvectors of the adjoint of the generator actually implies the positivity of the semigroup. This is not the case.

To give an example consider $E=L^{2}(\mathbb{R})$ and the operator $B$ given by

$$
\begin{aligned}
& B f=f^{(3)} \text { with domain } \\
& D(B)=\left\{f \in L^{2}(\mathbb{R}): f \in C^{2}(\mathbb{R}) ; f^{\prime \prime} \in A C(\mathbb{R}) ; f, f^{\prime}, f^{\prime \prime}, f^{(3)} \in L^{2}(\mathbb{R})\right\}
\end{aligned}
$$

Then $B$ is the generator of a unitary group $(U(t))_{t \in \mathbb{R}}$. In particular, $B$ is skew-adjoint, i.e. $B^{\prime}=-B$.

Moreover, we claim that $B^{\prime}$ has a strictly positive subeigenvector $\phi$.

Proof. Let $\lambda>0$ and $\phi \in C^{3}(\mathbb{R})$ such that $\phi(x)=e^{-|x|}$ for $|x| \geq 1, \phi(x)>0$ for all $x \in \mathbb{R}, \phi(0)=1$ and $\phi^{\prime}(0)=\phi^{\prime \prime}(0)=0$. Then $\phi \in D\left(B^{\prime}\right)$. Moreover, $-\phi^{(3)}(x) \leqq \phi(x)$ for $|x| \geqq 1$. Hence there exists $\mu \geqq 1$ such that $B^{\prime} \phi=-\phi^{(3)} \leqq \mu \phi$.

But the semigroup $(U(t))_{t \geqq 0}$ is not positive. In fact, we show that there exists $f \in D(B)$ such that


\begin{equation*}
\langle(\operatorname{sign} f) B f, \phi\rangle>\langle | f\left|, B^{\prime} \phi\right\rangle . \tag{3.19}
\end{equation*}


Proof. Let $f \in D(B)$ be such that $f(x)=e^{-x}$ sin $x$ in a neighborhood of 0 , while $f(x)>0$ for $x>0$ and $f(x)<0$ for $\mathrm{x}<0$. Then\\
$<\left(\operatorname{sign}\right.$ f) Bf, $\phi>=-\int_{-\infty}^{0} \mathrm{f}^{(3)}(x) \phi(x) d x+\int_{0}^{\infty} f^{(3)}(x) \phi(x) d x$.\\
Hence, $\langle | f\left|, B^{\prime} \phi\right\rangle=\int_{-\infty}^{0}(-f(x))\left(-\phi^{(3)}(x)\right) d x+\int_{0}^{\infty} f(x)\left(-\phi^{(3)}(x)\right) d x$ $=-\int_{-\infty}^{0} \mathrm{f}^{(3)}(\mathrm{x}) \phi(\mathrm{x}) \mathrm{dx}+\int_{0}^{\infty} \mathrm{f}^{(3)}(\mathrm{x}) \phi(\mathrm{x}) \mathrm{dx}$ $+\left.[f " \phi]\right|_{-\infty} ^{0}-\left.\left[f^{\prime \prime} \phi\right]\right|_{0} ^{\infty} \quad\left(\right.$ since $\left.\phi^{\prime \prime}(0)=\phi^{\prime}(0)=0\right)$ $=\langle(\operatorname{sign} \mathrm{f}) \mathrm{Bf}, \phi\rangle+2 \mathrm{f}^{\prime \prime}(0) \phi(0)$\\
$<<\left(\right.$ sign f) Bf, $\phi>\quad\left(\right.$ since $\left.\mathrm{f}^{\prime \prime}(0) \phi(0)=\mathrm{f}^{\prime \prime}(0)=-2\right)$.

We now show that $B$ satisfies Kato's inequality for positive elements, though; i.e.,


\begin{equation*}
P_{f} B f \leqq B f \quad \text { for all } \pounds \in D(B)_{+} \tag{3.20}
\end{equation*}


In fact, more is true. $B$ is local, i.e.\\
(3.21) $f \perp g$ implies $B f \perp g$ for all $f \in D(B), g \in L^{2}(\mathbb{R})$.

Proof. Let $A$ be the generator of the translation group which, in particular, is a lattice semigroup (see section 5). We obtain from Proposition 5.4 below that $A$ is local. Hence $B=A^{3}$ is local as well.

This example shows that even if there exists a strictly positive subeigenvector of the adjoint of the generator, Kato's inequality for positive elements alone does not suffice for the positivity of the semigroup. Note also that (because of the order continuous norm) Kato's inequality holds for positive elements if and only if the positive minimum principle is satisfied (see Remark 3.14).

\section*{4. DOMINATION OF SEMIGROUPS}
Frequently it is useful to be able to compare two semigroups on a Banach lattice with respect to the ordering (for example, in order to decide whether a semigroup is stable (see Chapter A-IV and Example 4.14).

In this section we assume that $E$ is a o-order complete complex Banach lattice. Let (T(t)) $t \geq 0$ be a positive semigroup with generator $A$ and $(S(t))_{t \geq 0}$ a semigroup with generator $B$. We say, (T( $)_{t \geq 0}$ dominates $(S(t))_{t \geq 0}$ if


\begin{equation*}
|S(t) f| \leqq T(t)|f| \quad \text { for all } \pounds \in E, t>0 \text {. } \tag{4.1}
\end{equation*}


We first observe that domination of the semigroup is equivalent to domination of the resolvents.\\
proposition 4.1. The semigroup $(T(t))_{t \geqq 0}$ dominates $(S(t))_{t \geqq 0}$ if and only if\\
(4.2) $|R(\lambda, B) f| \leqq R(\lambda, A)|f| \quad(f \in E)$ for large real $\lambda$.

Proof. (4.2) follows from (4.1) since the resolvent is given by the Laplace transform of the semigroup. Conversely, if (4.2) holds, then

$$
\begin{aligned}
|S(t) f| & =\lim _{n \rightarrow \infty}\left|((n / t) R(n / t, B))^{n} f\right| \\
& \leqq \lim _{n \rightarrow \infty}((n / t) R(n / t, A))^{n}|f| \\
& =T(t)|f| \quad(t \geq 0, f \in E) .
\end{aligned}
$$

One can describe domination by an inequality for the generators in a manner analoguous to the characterization of positive semigroups in Section 1; however, no positive subeigenvectors are needed here.

Theorem 4.2. Let (T(t)) $t \geq 0$ be a positive semigroup with generator $A$ and $(S(t))_{t \geq 0}$ a semigroup with generator $B$. The following assertions are equivalent.\\
(i) $|S(t) \pounds| \leqq T(t)|f| \quad$ for all $\pounds \in E, t \geqq 0$.\\
(ii) $\operatorname{Re}\langle(\operatorname{sign} \bar{f}) B f, \phi\rangle \leqq\langle | f\left|, A^{\prime} \phi\right\rangle$ for all $f \in D(B), \phi \in D\left(A^{\prime}\right)_{+}$.

Proof. (i) implies (ii). Let $E \in D(B), \phi \in D\left(A^{\prime}\right)+$. Then\\
$\operatorname{Re}<(\operatorname{sign} \bar{f}) B f, \phi\rangle=\operatorname{Re}\left\langle(\operatorname{sign} \bar{f}) \lim _{t \downarrow 0} 1 / t(S(t) f-f), \phi>\right.$

$$
\begin{aligned}
& =\left\langle\lim _{t \downarrow 0} 1 / t(\operatorname{Re}((\operatorname{sign} \bar{f}) S(t) f)-|f|), \phi\right\rangle \\
& \left.\leqq \lim _{t+0}<1 / t(|S(t) f|-|f|), \phi\right\rangle \\
& \leqq \lim _{t \downarrow 0}\langle 1 / t(T(t)|f|-|f|), \phi\rangle=\langle | f\left|, A^{\prime} \phi\right\rangle .
\end{aligned}
$$

(ii) implies (i). Let $\lambda>\max \{\omega(A), \omega(B)\}$ and $g \in E$. We show that


\begin{equation*}
|R(\lambda, B) g| \leqq R(\lambda, A)|g| \tag{4.3}
\end{equation*}


Let $\psi \in E_{+}^{\prime}$. Then $\phi:=R(\lambda, A)^{\prime} \psi \in D\left(A^{\prime}\right)_{+} \cdot$\\
Setting $f:=R(\lambda, B) g \in D(B)$ we obtain by (ii)\\
$\langle | R(\lambda, B) g|, \psi\rangle=\langle | f\left|,\left(\lambda-A^{\prime}\right) \phi\right\rangle \leqq\langle\lambda| f|, \phi\rangle-\operatorname{Re}\langle(\operatorname{sign} \bar{f}) B E, \phi\rangle=$\\
$\operatorname{Re}\langle(\operatorname{sign} \bar{f})(\lambda f-B f), \phi\rangle=\operatorname{Re}\langle(\operatorname{sign} \bar{f}) g, \phi\rangle \leqq\langle | g|, \phi\rangle=\langle R(\lambda, A)| g|, \psi\rangle$.\\
since $\psi \in E_{+}^{\prime}$ is arbitrary (4.3) follows.

In order to deduce that (ii) implies (i) in Theorem 4.2, it is not necessary to assume that B is a generator. Merely a range condition is sufficient. The precise formulation is the following.

Theorem 4.3. Let $(T(t))_{t \geq 0}$ be a positive semigroup with generator $A$. Let $B$ be a densely defined operator such that


\begin{equation*}
\operatorname{Re}\langle(\operatorname{sign} \bar{f}) \mathrm{Bf}, \phi\rangle \leqq\langle | f\left|, A^{\prime} \phi\right\rangle \tag{4.4}
\end{equation*}


$$
\text { for all } f \in D(B), \phi \in D\left(A^{\prime}\right)+.
$$

Then B is closable. Moreover, if $(\lambda-B) D(B)$ is dense in $E$ for some $\lambda>\max \{0, s(A)\}$, then $\bar{B}$ (the closure of $B$ ) generates a semigroup which is dominated by $(T(t))_{t \geqq 0}$.

Proof. 1. We show that B is closable.\\
Let $u_{n} \in D(B)$ satisfy $u_{n} \rightarrow 0$ and $B u_{n} \rightarrow v \quad(n \rightarrow \infty)$. We have to\\
show that $v=0$. Considering $A-\mu$ and $B-\mu$ for some $\mu>\mathrm{s}(\mathrm{A})$ instead of $A$ and $B$ we may assume that $s(A)<0$. Then there exists a strictly positive set $M^{\prime} \subset E^{\prime}$ such that\\
$\phi \in D\left(A^{\prime}\right)$ and $A^{\prime} \phi \leqq 0$ for all $\phi \in M^{\prime}$\\
(see the proof of Proposition 3.5).\\
Let $\phi \in M^{\prime}$ and $p$ be the seminorm given by $p(f)=\langle | f|, \phi\rangle$. We show that $B$ is p-dissipative (see end of $A-I I, S e c .2)$.\\
Let $f \in D(B), \psi=(\operatorname{sign} \bar{f})^{\prime} \phi$. Then it is easy to see that\\
$\psi \in \operatorname{dp}(f)$. Moreover, by (4.4) and (4.5) one obtains that Re<Bf, $\psi>=$ Re<(sign $\mathbf{\text { I }}) \mathrm{Bf}, \phi\rangle \leqq\langle | f\left|, A^{\prime} \phi\right\rangle \leqq 0$. Thus $B$ is p-dissipative. By the proof of A-II,Prop.2.9 one sees that $\mathrm{p}(\mathrm{v})=0$; i.e., $\langle | v|, \phi\rangle=0$. Since $\phi \in M^{\prime}$ was arbitrary we conclude that $v=0$.\\
2. Let $\lambda>\lambda_{0}:=\max \{s(A), 0\}$. We show that for $f \in D(B)$,\\
(4.6) $\quad g=(\lambda-B) f$ implies $|f| \leqq R(\lambda, A)|g|$.

Let $\psi \in E_{+}^{\prime}$. We have to show that $\langle | f|, \psi\rangle \leqq\langle R(\lambda, A)| g|, \psi\rangle$.\\
Let $\phi=R(\lambda, A)^{\prime} \psi \varepsilon D\left(A^{\prime}\right)+$. Then by (4.4)\\
$\langle | f|, \psi\rangle=\langle | f\left|,\left(\lambda-A^{\prime}\right) \phi\right\rangle=\operatorname{Re}\langle(\operatorname{sign} \bar{f})(\lambda f), \phi\rangle-\langle | f\left|, A^{\prime} \phi\right\rangle$\\
$\leqq \operatorname{Re}\langle(\operatorname{sign} \overline{\mathrm{f}})(\lambda-\mathrm{B}) \mathrm{f}, \phi\rangle=\operatorname{Re}\langle(\operatorname{sign} \overline{\mathrm{f}}) g, \phi\rangle$\\
$\leqq\langle | g|, \phi\rangle=\langle R(\lambda, A)| g|, \psi\rangle$.\\
It follows from $(4.6)$ that for $\lambda>\lambda_{0}$ and $f \in D(\bar{B})$


\begin{equation*}
g=(\lambda-\vec{B}) \pounds \text { implies }|E| \leqq R(\lambda, A)|g| . \tag{4.7}
\end{equation*}


In particular, $(\lambda-\bar{B})$ is injective for $\lambda>\lambda_{0}$. Moreover,


\begin{align*}
& |R(\lambda, \bar{B}) g| \leqq R(\lambda, A)|g| \text { for all } g \in E  \tag{4.8}\\
& \text { whenever } \lambda_{O}<\lambda \in \rho(\bar{B}) \text {. }
\end{align*}


Assume now that $\mu>\lambda_{0}$ such that $(\mu-B) D(B)$ is dense in $E$. Then $(\mu-\bar{B}) D(\bar{B})=E$. (Indeed, let $h \in E$. There exists $f_{n} \in D(B)$ such that $g_{n}:=(\mu-B) f_{n} \rightarrow h(n \rightarrow \infty)$. By (4.6) it follows that $\left|f_{n}-f_{m}\right| \leq R(\lambda, A)\left|g_{n}-g_{m}\right|$. Thus $\left(f_{n}\right)$ is a Cauchy sequence. Let $f=\lim _{n \rightarrow \infty} f_{n}$. Then $f \in D(\bar{B})$ and $(\mu-\bar{B}) f=h$. $f$ Thus $\mu \in \rho(\bar{B})$. It follows from the hypothesis that there esists $\lambda_{1} \in \rho(\bar{B})$ such that $\lambda_{0}<\lambda_{1}$. Since $\mathrm{R}\left(\lambda_{1}, \mathrm{~A}\right) \leqq \mathrm{R}\left(\lambda_{1}, \mathrm{~A}\right)$ (by B-II,Lemma 1.9), it follows from (4.8) that $\|R(\lambda, \bar{B})\| \leqq\|R(\lambda, A)\| \leqq\left\|R\left(\lambda_{1}, A\right)\right\|:=c$; hence aist $(\lambda, \sigma(\bar{B}))=r(R(\lambda, \bar{B}))^{-1} \geqq\|R(\lambda, \bar{B})\|^{-1} \geqq 1 / \mathrm{C}$ for all\\
$\lambda \in \rho(\bar{B}) \cap\left[\lambda_{1}, \infty\right]$. This implies that $\left[\lambda_{1}, \infty\right) \subset \rho(\bar{B})$. Moreover, it follows from (4.8) that


\begin{equation*}
\left|R(\lambda, \bar{B})^{n} f\right| \leqq R(\lambda, A)^{n}|f| \quad\left(f \in E, n \in \mathbb{N}, \lambda_{I}<\lambda\right) . \tag{4.9}
\end{equation*}


Let $w>\omega(A), \lambda_{1}$. Then it follows from (4.9) that\\
$\left\|(\lambda-w)^{n} R(\lambda, \bar{B})^{n}\right\| \leqq\left\|(\lambda-w)^{n} R(\lambda, A)^{n}\right\|$ for all $\lambda>w, n \in \mathbb{N}$. so by the Hille-Yosida theorem, $\bar{B}$ is the generator of a semigroup\\
$(S(t))_{t \geqq 0}$. Finally, the domination of $(S(t))_{t \geq 0}$ by $(T(t))_{t \geqq 0}$ follows from (4.8) and prop.4.1.

Example 4.4. a) Let $E$ be a o-order complete complex Banach lattice and $(T(t))_{t \geqq 0}$ be a positive semigroup with generator A. Let $M \varepsilon Z(E)$ (the center of $E$ (see C-I, Sec.9). For example, if $E=$ $I^{P}(X, \mu)$ (where $(X, \mu)$ is a o-finite measure space and $1 \leqq p \leqq \infty$ ) then M is the multiplication operator defined by a function in $L^{\infty}(X, \mu)$.\\
Let $B=A+M$. Then $B$ generates a semigroup $(S(t))_{t \geq 0}$. Assume that $R e M \leqq 0$. Let $f \in D(B)$ and $\phi \in D\left(A^{\prime}\right)+$. Then $\operatorname{Re}<(\operatorname{sign} \overline{\mathrm{f}}) \mathrm{Bf}, \phi>=\operatorname{Re}\langle(\operatorname{sign} \overline{\mathrm{f}}) \mathrm{Af}, \phi>+\operatorname{Re}<(\operatorname{sign} \overline{\mathrm{f}}) \mathrm{Mf}, \phi>$

$$
\begin{aligned}
& =\operatorname{Re}\langle(\operatorname{sign} \overline{\mathrm{f}}) \mathrm{Af}, \phi\rangle+\operatorname{Re}\langle M| f|, \phi\rangle \\
& \leqq\langle | f\left|, A^{\prime} \phi\right\rangle .
\end{aligned}
$$

Thus, by Theorem 4.2, $(S(t))_{t \geq 0}$ is dominated by $(T(t))_{t \geqq 0}$.\\
b) Let $E$ be an order complete complex Banach lattice and $B$ be a regular bounded operator on $E$. Then $B$ can be written as $B=B_{0}+$ $M$ where $M \in Z(E)$ and $B_{0} \in L^{r}(E)$ such that inf $\left\{\left|B_{0}\right|, I d\right\}=0$. Let $A=\left|B_{0}\right|+\operatorname{Re} M$. Then the semigroup $\left(e^{t B}\right)_{t \geqslant 0}$ is dominated by $\left(e^{t A}\right) t \geq 0$.\\
In fact, let $f \in E$. Then $\operatorname{Re}[(\operatorname{sign} \overline{\mathrm{f}}) \mathrm{Bf}]=\operatorname{Re}\left[(\operatorname{sign} \bar{f}) \mathrm{B}_{0} f\right]+\operatorname{ReM} \cdot|f|$ $\leqq\left|B_{O}\right||f|+\operatorname{ReM} \cdot|f|=A|f|$. This implies condition (ii) in Thm. 4.2.

Domination and positivity are characterized simultaneously as follows.

Proposition 4.5. Let $E$ be a o-order complete real Banach lattice. Let $(T(t))_{t \geqq 0}$ be a positive semigroup with generator $A$ and let $(S(t))_{t \geqslant 0}$ be a semigroup with generator $B$. The following are equivalent.\\
(i) $0 \leqq S(t) \leqq T(t)$ for all $t \geqq 0$.\\
\includegraphics[max width=\textwidth, center]{2024_12_23_c6487cc0859199a15bd9g-282}

Remark 4.6. Condition (ii) implies (4.4) (cf. Remark 3.12).

Proof. One proves as in Theorem 4.2 that (i) implies (ii).\\
It is trivial that (ii) implies (iii). Assume that (iii) holds. Let $\lambda>\lambda_{0}=\max \{s(A), s(B), 0\}$. In a similar way as $(4.6)$ one shows that for all $f \in D_{0}$\\
(4.10) $\quad \lambda f-B f=g$ implies $f^{+} \leqq R(\lambda, A) g^{+}$.\\
since $D_{0}$ is a core of $B$ it follows that (4.10) also holds for all $f \in D(B)$. This implies that $(R(\lambda, B) g)^{+} \leqq R(\lambda, A) g^{+}$for all $g \in E$, $\lambda>\lambda_{0}$. Consequently, $0 \leqq \mathrm{R}(\lambda, \mathrm{B}) \leqq \mathrm{R}(\lambda, \mathrm{A})$ for all $\lambda>\lambda_{0}$. Hence (i) holds.

In the following example we apply Theorem 4.3 to Schrodinger operators. Here the range condition is proved by an elegant argument due to Kato (1986) with the help of Kato's classical inequality.

Example 4.7 (Schrödinger operators on $L^{\mathrm{P}}$ ).\\
Let $E=L^{\mathrm{P}}\left(\mathbb{R}^{n}\right), 1 \leqq \mathrm{p}<\infty$, and $V \in \mathrm{~L}_{\mathrm{Loc}}^{\mathrm{P}}\left(\mathbb{R}^{\mathrm{n}}\right)$ such that $\operatorname{ReV} \geqq 0$. Define $B$ on $E$ by $B f=\Delta f$ - Vf with domain $D(B)=C_{C}^{\infty}\left(\mathbb{R}^{n}\right)$. Then $B$ is closable and $\bar{B}$ is the generator of a semigroup ( $S(t){ }_{t} \geq 0$ which is dominated by the diffusion semigroup (Example 1.5 d and $\mathrm{A}-\mathrm{I}$, $2.8)$. If $V \geqq 0$, then $(S(t))_{t \geqq 0}$ is positive.

Proof. Denote by $A$ the generator of the diffusion semigroup. Then $\overline{C_{c}^{\infty}:}=C_{C}^{\infty}\left(\mathbb{R}^{n}\right)$ is a core of $A$ and $A f=\Delta f$ for $f \in C_{c}^{\infty}$ (see Example 1.5d). Let $0 \leqq \phi \in D\left(A^{\prime}\right)$. Then\\
\includegraphics[max width=\textwidth]{2024_12_23_c6487cc0859199a15bd9g-283} $\langle | f\left|, A^{\prime} \phi\right\rangle$ for all $f \in C_{C}^{\infty}$ by Theorem 2.4. Thus (4.4) holds.\\
We show that $(\lambda-B)$ has dense range for $\lambda>0$. If not, then there exists $0 \neq \phi \in E^{\prime}=L^{q}\left(\mathbb{R}^{n}\right)$ such that $\langle(\lambda-\Delta+V) f, \phi>=0$ for all $f \in C_{c}^{\infty}$; i.e., $(\lambda-\Delta+V) \phi=0$ in the sense of distributions. By Kato's classical inequality (see Example 2.5) this implies that\\
$(\lambda-\Delta+\operatorname{ReV})|\phi| \leqq \lambda|\phi|-\operatorname{Re}[(\operatorname{sign} \bar{\phi})(\lambda \phi-\Delta \phi+V \phi)]=0$ (here we use that $\Delta \phi=\lambda \phi+V \phi \in \mathrm{~L}_{10 c}^{1}$ ). Hence $(\lambda-\Delta)|\phi| \leqq-(\operatorname{ReV})|\phi| \leqq 0$. Since $(\lambda-\Delta)^{-1}$ is a positive linear mapping from $S\left(\mathbb{R}^{n}\right)^{\prime}$ onto $S\left(\mathbb{R}^{n}\right)^{\prime}$, this implies that $\phi=0$. It follows from Thm. 4.3 that $\overline{\mathrm{B}}$ is the generator of a semigroup $(S(t))_{t \geq 0}$ which is dominated by the semigroup generated by A.\\
If $V=R e V \geqq 0$, we may consider the real space $L^{p}\left(\mathbb{R}^{n}\right)$. Then for every $f \in C_{C}^{\infty}, 0 \leq \phi \in D\left(A^{\prime}\right)$ we have

$$
\begin{aligned}
\left\langle\mathrm{P}\left(\mathrm{f}^{+}\right) \mathrm{Bf}, \phi\right\rangle & =\left\langle\mathrm{P}\left(\mathrm{f}^{+}\right) \mathrm{Af}, \phi\right\rangle-\left\langle V \mathrm{f}^{+}, \phi\right\rangle \\
& \left.\leqq\left\langle\mathrm{P} \mathbf{f}^{+}\right) \mathrm{Af}, \phi\right\rangle \\
& \leqq\left\langle\mathrm{f}^{+}, \mathrm{A}^{\prime} \phi\right\rangle
\end{aligned}
$$

by (3.6). It follows from prop. 4.5 that $(S(t))_{t \geqq 0}$ is positive.

Finally, if it is known that the semigroup $(S(t))_{t \geq 0}$ is positive, domination can be characterized as follows.\\
proposition 4.8. Let $E$ be a real Banach lattice, ( $T(t))_{t \geqq 0}$ a positive semigroup with generator $A$ and (S(t)) tき0 a positive semigroup with generator $B$. Consider the following conditions.


\begin{equation*}
S(t) \leqq T(t) \quad(t \geqq 0) \tag{i}
\end{equation*}


(ii) $\langle B f, \phi\rangle \leqq\left\langle f, A^{\prime} \phi\right\rangle$ for all $f \in D(B)_{+}, \phi \in D\left(A^{\prime}\right)_{+} \cdot$\\
(iii) $B f \leqq A f$ for $0 \leqq \pounds \in D(A) \cap D(B)$.

Then (i) and (ii) are equivalent and imply (iii).\\
Moreover, if $D(A) \subset D(B)$ or $D(B) \subset D(A)$, then (iii) implies (i).

Proof. Assume that (i) holds. Then for $f \in D(B)_{+}, \phi \in D\left(A^{\prime}\right)_{+}$,\\
$\left.\left.\langle B f, \phi\rangle=\lim _{t \rightarrow 0} 1 / t<S(t) f-f, \phi\right\rangle \leqq \lim _{t \rightarrow 0} 1 / t<T(t) f-f, \phi\right\rangle$

$$
=\left\langle f, A^{\prime} \phi\right\rangle .
$$

So (ii) holds. (iii) is proved similarly.\\
Now assume (ii). Let $\lambda>\max \{s(A), s(B)\}$. Let $g \in E_{+}, \psi \in E_{+}^{\prime}$. Then $\langle R(\lambda, B) g-R(\lambda, A) g, \psi>$\\
$=\left\langle R(\lambda, A) g, \lambda R(\lambda, B)^{\prime} \psi-\psi\right\rangle-\left\langle\lambda R(\lambda, A) g-g, R(\lambda, B)^{\prime} \psi\right\rangle$\\
$=\left\langle f, B^{\prime} \phi\right\rangle-\langle A f, \phi\rangle \leqq 0$,\\
where $f=R(\lambda, A) g \in D(A)+$ and $\phi=R(\lambda, B)^{\prime} \psi \in D\left(B^{\prime}\right)_{+}$. Hence $R(\lambda, B)$ $\leqq R(\lambda, A)$ and (i) follows.\\
Finally, we prove that (iii) implies (i) if $D(B) \subset D(A)$, say.\\
Let $\lambda>\max \{s(A), s(B)\}$. Then $(A-B) R(\lambda, B)$ is a positive operator.\\
Hence $R(\lambda, A)-R(\lambda, B)=R(\lambda, A)(A-B) R(\lambda, B) \geqq 0$. This implies (i).

The preceding results can be applied to the perturbation by multiplication operators. Let $(\mathrm{X}, \mu)$ be a $\sigma$-finite measure space and $\mathrm{E}=$ $L^{P}(X, \mu) \quad(1 \leqq P<\infty)$. Consider a positive semigroup $(T(t)), \geq 0$ with generator $A$. Let $m: X \rightarrow \mathbb{R}$ be a measurable function such that $m(x) \leqq 0$ for all $x \in X$. Let $D(m)=\{f \in E: f \cdot m \in E\}$. Define the operator $B$ with domain $D(B)=D(A) \cap D(m)$ by $B f=A f+m \cdot f$ ( $\pounds \in \mathrm{D}(\mathrm{B})$ ).

Theorem 4.9. If there exists a quasi-interior subeigenvector u of A such that $u \in D(m)$, then $B$ is closable and the closure $\vec{B}$ of $B$ is the generator of a positive semigroup $(S(t))_{t \geq 0}$ which is dominated by $(T(t))_{t \geq 0}$.

For the proof of the theorem we need the following lemma.

Lemma 4.10. Let $A$ and $B$ be generators of positive semigroups $(T(t))_{t \geq 0}$ and $S(t)_{t \geq 0}$, respectively. If $(T(t))_{t \geq 0}$ dominates $(S(t))_{t \geqq 0}$, then $s(B) \leqq s(A)$.

Proof of Lemma 4.10. Let $\lambda>s(A)$. Then for all $\mu>\max \{\lambda, s(B)\}$ one has $0 \leqq R(\mu, A) \leqq R(\lambda, A)$ (by B-II, Lemma 1.9), and so $\|R(\mu, B)\| \leqq$ $\|R(\mu, A)\| \leqq\|R(\lambda, A)\|$. Thus $\|$ dist $(\mu, \sigma(B)) \geqq\|R(\mu, B)\|^{-1} \geqq\|R(\lambda, A)\|^{-1}$. This implies that $[\lambda, \infty) \subset \rho(B)$. Hence $s(B) \leqq \lambda$.

Proof of Theorem 4.9. There exists $\mu>0$ such that $A u \leqq \mu u$. Let $\lambda>\max \{s(A), \mu\}$. Then $\lambda R(\lambda, A) u=A R(\lambda, A) u+u \leqq \mu R(\lambda, A) u+u$. Hence $R(\lambda, A) u \leq c \cdot u$ where $c>0$. It follows that $R(\lambda, A) E_{u} c$ $E_{u} \cap D(A) \subset D(B)$. Hence $D(B)$ is dense. Let $f \in D(B), \phi \in D\left(A^{\prime}\right)_{+}$and set $P_{+}:=P_{f^{+}}, P_{-}:=P_{f^{-}}$. Then (4.11) $\left\langle F_{+} B F, \phi\right\rangle \leqq\left\langle f^{+}, A^{\prime} \phi\right\rangle$.

In fact, $\left\langle P_{+} B f, \phi\right\rangle=\left\langle P_{+} \mathrm{Af}, \phi\right\rangle+\left\langle P_{+} m \cdot f_{1}, \phi\right\rangle$

$$
\begin{aligned}
& =\left\langle\mathrm{P}_{+}^{\top} \mathrm{Af}, \phi\right\rangle+\left\langle\mathrm{m}^{\top} \cdot \mathrm{f}^{+}, \phi\right\rangle \\
& \leqq\left\langle\mathrm{P}_{+} \mathrm{Af}, \phi\right\rangle \\
& \leqq\left\langle\mathrm{f}^{+}, \mathrm{A}^{\prime} \phi\right\rangle \quad(\text { by }(3,6)) .
\end{aligned}
$$

But (4.11) implies (4.4). So it follows from Theorem 4.3 that B is closable. Moreover, if we can show that $(\lambda-\bar{B}) D(\bar{B})$ is dense in $E$, it follows that $\bar{B}$ is the generator of a semigroup $(S(t))_{t \geq 0}$. In that case (4.11) implies that $(S(t))_{t \geqq 0}$ is dominated by $(T(t)){ }_{t \geqslant 0}$ (by Proposition 4.5).\\
Now we show that $(\lambda-\bar{B}) D(\bar{B})$ is dense in $E$.\\
Let $m_{n}=\sup \left\{m_{,}-n_{1}\right\} \quad(n \in \mathbb{N})$ and $B_{n}=A+m_{n}$. Then $B_{n}$ is the generator of a positive semigroup and it follows from proposition 4.8 that $0 \leqq R\left(\lambda, B_{n+1}\right) \leqq R\left(\lambda, B_{n}\right) \leqq R(\lambda, A)$ for all $n \in \mathbb{N}, \lambda>s(A)$. (Note that $s\left(B_{n}\right) \leqq s(A)$ by Lemma 4.10). Let $0 \leqq f \in E_{u}$ and $g_{n}=$ $R\left(\lambda, B_{n}\right) f$. Then $g=\inf _{n \in \mathbb{N}} g_{n}=\lim _{n \rightarrow \infty} g_{n}$ exists. Moreover $g_{n} \in D(B)$ and $\quad \lim _{n \rightarrow \infty}(\lambda-B) g_{n}=f+\lim _{n \rightarrow \infty}\left(B_{n}-B\right) g_{n}=f$, since $\left|\left(B_{n}-B\right) g_{n}\right| \leqq\left(m_{n}-m\right)\left|g_{n}\right|=\left(m_{n}-m\right)\left|R\left(\lambda, B_{n}\right) f\right| \leq\left(m_{n}-m\right) R(\lambda, A)|f| \leq$ $c^{\prime}\left(m_{n}-m\right) u$ for some positive constant $c^{\prime}$.

But $\lim _{n \rightarrow \infty}\left(m_{n}-m\right) u=0$ since $u \in D(m)$. Thus $g \in D(\bar{B})$ and $(\lambda-\bar{B}) g=E$. We have shown that $E_{u} \subset(\lambda-\bar{B}) D(\bar{B})$. Hence $(\lambda-\bar{B}) D(\bar{B})$ is dense in E.

Example 4.11. If in the situation explained before Theorem 4.9 $\mathrm{D}(\mathrm{A}) \subset \mathrm{L}^{\infty}(\mathrm{X}, \mu)$ and $\mathrm{m} \in \mathrm{L}^{\mathrm{P}}(\mathrm{X}, \mu)$, then the hypotheses of Theorem 4.9 are satisfied.

Now we want to indicate how the results of this section look like for $c_{0}(\mathrm{X})$. In fact, most of them carry over with a different interpretation of "sign" but the same proofs. We want to state the analogs of Theorem 4.2 and Theorem 4.3 explicitly. Here we use the notation of B-II, Sec. 2 .

Theorem 4.12. Let $E=C_{0}(X)$ where $X$ is locally compact. Let $(T(t))_{t \geqq 0}$ be a strongly continuous positive semigroup with generator $A$ and $(S(t))_{t \geqq 0}$ a semigroup with generator $B$. The following assertions are equivalent.


\begin{equation*}
|S(t) f| \leqq T(t)|f| \quad \text { for all } f \in E, t>0 \text {. } \tag{i}
\end{equation*}


(ii)

\begin{verbatim}
Re<(sign \overline{f) Bf, \phi> \ <|f|,A'\phi>}\\
\end{verbatim}

for all $f \in D(B), \phi \in D\left(A^{\prime}\right)_{+} \cdot$

Recall that by definition\\
$\operatorname{Re}\langle(\operatorname{sign} \overline{\mathrm{f}}) \mathrm{Bf}, \phi\rangle=\int[(\operatorname{sign} \overline{\mathrm{f}(\mathrm{x})}) \cdot(B f)(x)] \mathrm{d} \phi(x)$ where $\operatorname{sign} \mathrm{f}(\mathrm{x})=$ $f(x) /|f(x)|$ if $f(x) \neq 0$ and $\operatorname{sign} 0=0$.

Theorem 4.13. Let $E=C_{0}(X)$ (X locally compact) and let $(T(t))_{t \geqslant 0}$ be a positive semigroup on E with generator A. Let $B$ be a densely defined operator such that

\begin{verbatim}
Re <(sign \overline{f})Bf,\phi>\leqq <|f|,A'\phi>
for all f \in D(B), , { E D(A')}+\mathrm{ .
\end{verbatim}

Then $B$ is closable. Moreover, if $(\lambda-B) D(B)$ is dense in $E$ for some $\lambda>\max \{0, s(A)\}$, then $\bar{B}$ (the closure of $B$ ) generates a semigroup which is dominated by $(T(t))_{t \geq 0}$.

Example 4.14. Let $E:=C([-1,0], \mathbb{C}), \alpha \in \mathbb{R}, B \in \mathbb{C}, \mu \in M[-1,0]_{+}$ and $v \in M[-1,0]$ such that $\mu(\{0\})=v([0\})=0$. Then the operator $A$ given by $A f=f^{\prime}$ on $D(A)=\left\{f \in C^{1}([-1,0], \mathbb{C}\}: f^{\prime}(0)=\alpha f(0)+\right.$ $\langle f, \mu>\}$ generates a strongly continuous positive semigroup (T(t)) $t \geqslant 0$ (see B-II,Example 1.22).

Consider the operator $B$ given by $B f=f^{\prime}$ with domain $D(B)=\{f \in$ $\left.C^{1}([-1,0], \mathbb{C}): f^{\prime}(0)=\beta f(0)+\langle f, V\rangle\right\}$. We claim that\\
$B$ is the generator of a strongly continuous semigroup\\
(4.13) (S(t)) $t_{t \geqq 0}$. Moreover, $(S(t))_{t \geqq 0}$ is dominated by (T( $t$ ) ${ }_{t \geqq 0}$ if and only if $\operatorname{Re} \beta \leqq \alpha$ and $|\nu| \leqq \mu$.

Remark. It is of interest to find a condition on $B$ which implies that the semigroup $(S(t))_{t \geqq 0}$ is stable (see A-IV, Sec.1).\\
Using the positivity of $(T(t))_{t \geqslant 0}$ it is shown in $B-I V, E x .3 .9$, that $(T(t))_{t \geq 0}$ is stable if and only if $\|\mu\|+\alpha<0$. Since a semigroup which is dominated by a stable semigroup is itself stable we obtain from (4.13) that $(S(t))_{t \geqq 0}$ is stable if $\|v\|+\operatorname{Re} \beta<0$.

Proof of (4.13). We first assume that $\alpha:=\operatorname{Re} \beta$ and $\mu=|\nu|$. We show that (4.12) is satisfied. Consider the operator $A_{\text {max }}$ on $C[-1,0]$ given by $A_{\max } f=f$ with domain $D\left(A_{\max }\right)=C^{1}[-1,0]$. We know by $B$-II, Example 2.12 that $\operatorname{Re}^{<}(\operatorname{sign} \overline{\mathrm{f}}) \mathrm{Af}, \phi>\leqq \operatorname{Re}<(\operatorname{sign} \overline{\mathrm{f}})(\mathrm{Af}), \phi>=$ $\left.<|f|,\left(A_{\max }\right)^{\prime \phi}\right\rangle$ for all $\mathrm{f} \in \mathrm{D}\left(\mathrm{A}_{\max }\right)^{\prime,} 0 \leqq \phi \in D\left(\left(A_{\max }\right)^{\prime}\right)$.\\
In particular\\
(4.14) $\operatorname{Re}\langle(\operatorname{sign} \bar{f}) \mathrm{Bf}, \phi\rangle \leqq\left\langle\mid \mathrm{f}, \mathrm{A}^{\prime} \phi\right\rangle$\\
holds for all $f \in D(B), 0 \leqq \phi \in D\left(\left(_{\text {max }}\right)^{\prime}\right)$. It is not difficult to see that $\left.D\left(A^{\prime}\right)=D\left(A_{\max }\right)^{\prime}\right)+C_{0}$, and since $D\left(\left(A_{\max }\right)^{\prime}\right)=B V[-1,0]$ (see B-II, Example 2.12 ) this is an order direct sum.\\
Thus, in view of (4.14), it remains to show that\\
(4.15) $\left.\quad \operatorname{Re}<(\operatorname{sign} \mathrm{f}) \mathrm{Bf}, \delta_{0}\right\rangle \leq\langle | f\left|, \mathrm{~A}^{\prime} \delta_{0}\right\rangle$\\
for all $f \in D(B)$. By the definition of $A, \delta_{0} \in D\left(A^{\prime}\right)$ and $A^{\prime} \delta_{0}=$ $a \delta_{0}+\mu$. Hence for $f \in D(B)$, $\operatorname{Re}<(s i g n \bar{f}) B f, \delta_{0}>=\operatorname{Re}\left((\operatorname{sign} \bar{f}) f^{\prime}\right)(0)=\operatorname{Re}((\operatorname{sign} \overline{f(0)}) \cdot(B f(0)+$ $\langle f, v\rangle)|\leq \operatorname{Re} B| f(0)|+|\langle f, v\rangle| \leqq \alpha| f(0) \mid+\langle | f|, \mu\rangle=\langle | f\left|, A^{\prime} \delta_{0}\right\rangle$. Thus (4.15) and so also (4.12) are proved.\\
As in the proof in Example 2.14 one shows that $\lambda$ - $B$ is surjective for large real $\lambda$. Hence by Theorem 4.14, B is the generator of a strongly continuous semigroup $(S(t))_{t \geq 0}$ which is dominated by $(T(t))_{t \geq 0}$. This proves the first assertion of (4.13) and the sufficiency of the second.\\
Now we assume that the semigroup $(S(t))_{t \geq 0}$ is dominated by $(T(t))_{t \geq 0}$. We have to show that $\operatorname{Re} \beta \leqq \alpha$ and $|\nu| \leqq \mu$. Since $\delta_{0} \in D\left(A^{\prime}\right) \cap D\left(B^{\prime}\right)$ we have for all $f \in C[-1,0]_{+}$satisfying $\left.f(0)=0,|\langle f, v\rangle|=\left|\left\langle f, B \cdot \delta_{0}\right\rangle=\lim _{t+0+} 1 / t\right|<S(t) f-f, \delta_{0}\right\rangle \mid$

$$
=\lim _{t \rightarrow 0+} 1 / t|(S(t) f)(0)|
$$

$$
\begin{aligned}
& \leqq \lim _{t \rightarrow 0+} 1 / \mathrm{t}((\mathrm{~T}(t)|\mathrm{f}|)(0) \\
& =\lim _{t \rightarrow 0+}\langle | \mathrm{f}\left|, 1 / \mathrm{t}\left(\mathrm{~T}(z)^{\prime} \delta_{0}-\delta_{0}\right)\right\rangle \\
& =\langle | \mathrm{f}\left|, \mathrm{~A}^{\prime} \delta_{0}\right\rangle=\langle | \mathrm{f}|, \mu\rangle .
\end{aligned}
$$

since $\mu(\{0\})=v\{0\})=0$, this implies that $|v| \leqq \mu$.\\
Moreover, for arbitrary $f \in C[-1,0]_{+}$we have $\left\langle f, \operatorname{Re}_{B} \delta_{0}+\operatorname{Rev}\right\rangle=1 i m_{t \rightarrow 0+} 1 / t \operatorname{Re}<(S(t) f-f), \delta_{0}>$\\
$\left.\leqq \lim _{t \rightarrow 0+} 1 / \mathrm{t} \operatorname{Re}<(T(t) \mathrm{f}-\mathrm{f}), \delta_{0}\right\rangle=\left\langle\mathrm{f}, \mathrm{A}^{\prime} \delta_{0}\right\rangle=\left\langle f, \alpha \delta_{0}+\mu\right\rangle$.\\
Consequently, $\left(\operatorname{Re}_{\beta}\right) \delta_{0}+\operatorname{Re} \nu \leq \alpha \delta_{0}+\mu$. This implies $\operatorname{Re} \beta \leqq a$ since $\mu(\{0\})=v\{0\})=0$.

We conclude this section discussing the following question. Let $(S(t))_{t \geqq 0}$ be a semigroup which is dominated by some positive semigroup. Does there exist a smallest semigroup (T( $t$ ) ${ }_{t \geq 0}$ which dominates $(S(t))_{t \geq 0}$ ? More precisely, we look for a positive semigroup $(T(t))_{t \geqq 0}$ dominating $(S(t))_{t \geq 0}$ such that $(T(t))_{t \geq 0}$ is dominated by any other positive semigroup which dominates $(S(t))_{t \geqq 0}$. If such a minimal dominating semigroup exists, it is unique and we call it the modulus semigroup of $(\mathrm{S}(\mathrm{t}))_{t \geqq 0}$.

Example 4.15 (the modulus semigroup associated with $\Delta$ - V). Let $E$ be the complex space $L^{p}\left(\mathbb{R}^{n}\right) \quad(1 \leqq \mathrm{p}<\infty)$ and $V \in L_{l o c}^{p}\left(\mathbb{R}^{n}\right)$ satisfying ReV $\geqq 0$. Denote by $B$ the closure of $\Delta-V$ on $C_{C}^{\infty}$ (cf. Example 4.7). The modulus semigroup of the semigroup ( $\mathrm{S}(\mathrm{t}))_{t}{ }^{0}$ generated by $B$ exists and its generator $A$ is given by\\
$A f=\Delta f-(R e V) f$ for all $f \in C_{C}^{\infty}$ (and $C_{C}^{\infty}$ is a core of $A$, see Example 4.7).

Proof. The operator A defined above generates a positive semigroup (see Example 4.7). For $f \in C_{c}^{\infty}, \phi \in D\left(A^{1}\right)_{+}$one has $\operatorname{Re}\langle(\operatorname{sign} \overline{\mathrm{f}}) \mathrm{Bf}, \phi\rangle=\operatorname{Re}\langle(\operatorname{sign} \overline{\mathrm{f}})(\Delta \mathrm{E}-\mathrm{Vf}), \phi\rangle=$ $\operatorname{Re}\langle(\operatorname{sign} \overline{\mathrm{f}}) \Delta f, \phi\rangle-\langle(\operatorname{ReV})| f|, \phi\rangle=\operatorname{Re}\langle(\operatorname{sign} \overline{\mathrm{E}}) \mathrm{Af}, \phi\rangle \leqq\langle | f\left|, \mathrm{~A}^{\prime} \phi\right\rangle$ by Thm.2.4. Since $C_{c}^{\infty}$ is a core of $B$, it follows from Thm. 4.3 that the semigroup generated by $A$ dominates $(S(t))_{t \geq 0}$. Let $C$ be the generator of a semigroup $(U(t))_{t \geqq 0}$ dominating $(S(t))_{t \geqq 0}$. Then $\operatorname{Re}\langle(\operatorname{sign} \bar{f}) A f, \phi\rangle=\operatorname{Re}\langle(\operatorname{sign} \bar{f}) \Delta f, \phi\rangle-\langle(\operatorname{ReV})| E|, \phi\rangle=\operatorname{Re}\langle(\operatorname{sign} \bar{f}) B f, \phi\rangle$ $\leq\langle | f\left|, C^{\prime} \phi\right\rangle$ for all $\ddagger \in C_{C}^{\infty}, \phi \in D\left(C^{\prime}\right)_{+}$by Thm. 4.2 . It follows from Thm.4.3 that $(U(t))_{t \geqq 0}$ dominates the semigroup generated by $A$.

Example 4.16. Let $A_{0}$ be the generator of a positive semigroup on an order complete Banach lattice $E$ and $M \in Z(E)$. The semigroup generated by $A_{0}+M$ possesses a modulus semigroup. Its generator is $A_{0}+$ ReM . (This can be proved as the assertion in Example 4.15.)

If a semigroup has a bounded regular generator, then it possesses a modulus semigroup. Its generator is bounded too (see C-I, Sec. 6 for the notion of regular operators).

Theorem 4.17. Let $B$ be a regular, bounded operator on an order complete complex Banach lattice $E$. The semigroup ( $\mathrm{e}^{t B}$ ) t 30 possesses a modulus semigroup. Its generator is $A=\mid B_{0}!+R e M$, where $B=B_{0}+M$ is the unique decomposition of $B$ in $L^{r}$ (E) satisfying $M \in Z(E), B_{0} \in Z(E)^{d}$.

For the proof we need the following result which is of independent interest.

Lemma 4.18. Let $A$ be the generator of a positive semigroup on a Banach lattice $E$. If $A f \geqq 0$ for all $f \in D(A)+$, then $A$ is bounded.

Proof. There exists $M \geqq 1$ such that $\|R(\lambda, A)\| \leqq M / \lambda$ for all $\lambda \geqq$ $\omega(A)+1$. Fix $\mu \geq \omega(A)+1$. Then $\operatorname{AR}(\mu, A) A f=\mu R(\mu, A) A f-A f=$ $\mu^{2} R(\mu, A) f-\mu f-A f$; hence $0 \leqq A f \leqq \mu^{2} R(\mu, A) f$ whenever $f \in D(A)+$ Thus $\|A f\| \leqq c\|f\|$ for all $\pounds \in D(A)+$ (where $c:=\mu^{2}\|R(\mu, A)\|$ ). Consequently,\\
$\|(\lambda R(\lambda, A)-I d) f\|=\|A R(\lambda, A) f\| \leqq c\|R(\lambda, A) f\| \leqq(M C / \lambda)\|f\|$ for all\\
$E \in E_{+}$and all $\lambda \geqq \omega(A)+1$. Hence\\
$\|(\lambda R(\lambda, A)-I d) g\| \leqq M c / \lambda\left(\left\|g^{+}\right\|+\left\|g^{-}\right\|\right) \leqq(2 M c / \lambda)\|g\|$ for all $g \in E$.\\
Thus $\lambda R(\lambda, A)$ is invertible if $\lambda$ is large enough and\\
$D(A)=\operatorname{im}(\lambda R(\lambda, A))=E$.

Proof of Thm. 4.17. Let $A=\left|B_{0}\right|+\operatorname{ReM}$. It has been shown in Example $4.4 b$ that $\left(e^{t A}\right) d_{t \geqslant 0}$ dominates $\left(e^{t B}\right)_{t \geqslant 0}$. Let $(U(t))_{t \geqslant 0}$ be a positive semigroup dominating $\left(e^{t B}\right)_{t \geq 0}$ and $c$ its generator. We first show that $C$ is bounded.\\
Let $f \in D(C)+$. Then $\operatorname{Re}(B f)=\lim _{t \downarrow 0} 1 / t\left(\operatorname{Re}\left(e^{t B} f\right)-f\right) \leqq$\\
$\lim _{t \downarrow 0} 1 / t(U(t) f-f)=C f$. Hence $(C+|B|) f \geqq(C-R e B) f \geqq 0$ for all $\mathrm{f} \in \mathrm{D}(\mathrm{C})_{+}$. By Lemma 4.18 this implies that $\mathrm{C}+|\mathrm{B}|$ is bounded. Hence $C$ is bounded as well.\\
since $c+\|c\| \cdot I d \geqq 0$ by Thm.1.11, $c$ is regular. Let $c=c_{0}+N$ where $c_{0} \in Z(E)^{d}$ and $N \in Z(E)$. Since $C \geq$ ReB by what we just proved, it follows that $\mathrm{N} \geqq \operatorname{ReM}$.\\
Let $f \in E_{+}, \phi \in E_{+}^{\prime}$ satisfy $\langle\mathcal{E}, \phi\rangle=0$. Then for all $\alpha \in \mathbb{R}$,\\
\includegraphics[max width=\textwidth]{2024_12_23_c6487cc0859199a15bd9g-290(1)} Thus $C-R e\left(e^{i \alpha} B\right)$ satisfies the positive minimum principle ( $P$ ) for all $\alpha \in \mathbb{R}$. It follows from Thm. 1.11 that $C-\operatorname{Re}\left(e^{i \alpha} B\right)+$\\
$(\|C\|+\|B\|) I d \geqq 0$ for all $\alpha \in \mathbb{R}$. Applying the band projection onto $Z(E)^{\mathrm{d}}$ on both sides of this inequality one obtains that\\
\includegraphics[max width=\textwidth, center]{2024_12_23_c6487cc0859199a15bd9g-290}\\
$T \in L^{r}(E)$, see $\left.C-I, \sec .7\right)$.\\
We have proved that $\operatorname{ReM} \leq \mathrm{N}$ and $\left|\mathrm{B}_{\mathrm{O}}\right| \leq \mathrm{C}_{0}$. This implies that\\
$\operatorname{Re}((\operatorname{sign} \bar{E}) B f)=\operatorname{Re}\left((\operatorname{sign} \bar{f}) B_{0} f\right)+(\operatorname{ReM})|f| \leqq C_{O}|f|+N|f|=C|f|$ for all f $\epsilon E$. It follows from Thm. 4.2 that $\left(e^{t B}\right){ }_{t \geq 0}$ is dominated by $\left(e^{t C}\right)_{t \geqq 0}$.

Remark. The proof of Thm. 4.17 shows that any semigroup dominating a semigroup whose generator is bounded and regular has a bounded generator as well.

Example 4.19. Let $\mathrm{E}=\ell^{\mathrm{P}}(1 \leq \mathrm{P}<\infty)$ or $\mathrm{c}_{0}$ and $\mathrm{B} \in \mathcal{L}^{r}(\mathrm{E})$ be given by the matrix $\left(b_{i j}\right)$. The generator $A$ of the modulus semigroup of $\left(e^{t B}\right), 0$ is given by the matrix $\left(a_{i j}\right)$ where $a_{i j}=\left|b_{i j}\right|$ when $i \neq j$ and $a_{i j}=\operatorname{Re} b_{i j}$.

A related question is under which condition a semigroup (S(t)) $t \geqq 0$ is dominated by some positive semigroup. Of course, a necessary condition is that every $S(t)$ is a regular operator. On an AL-space this condition is automatically satisfied. But Kipnis (1974) gives an example of a strongly continuous semigroup on $\ell^{1}$ which is not dominated. On the other hand, it has been independently shown by Kipnis (1974) and Kubokawe (1975) that every contraction semigroup on an $L^{1}$-space possesses a modulus semigroup (which is contractive as well).

\section*{5. SEMIGROUPS OF DISJOINTNESS PRESERVING OPERATORS}
In this section we consider a special case of domination. Recall from C-I,Sec. 6 that a linear operator 5 on $E$ is called lattice homomorphism if\\
(5.1) $|S f|=S|E|$ for all $f \in E$.

An operator $S \in L(E)$ is called disjointness preserving if\\
(5.2) $f \perp g$ implies $\mathrm{Sf} \perp \mathrm{Sg}$ for all $\mathrm{f}, \mathrm{g} \in \mathrm{E}$.

Note that an operator $S$ is a lattice homomorphism if and only if $S$ is positive and disjointness preserving.

In the following we will consider disjointness preserving semigroups (by this we mean semigroups of disjointness preserving operators) and lattice semigroups (i.e., semigroups of lattice homomorphisms). For example, the semigroup $\left(\mathrm{T}_{\mathrm{d}}(t)\right)_{t \geqq 0}$ defined in Section 3 is disjointness preserving for all $d \in \mathbb{R}$ and a lattice semigroup if $\mathrm{d} \geqq 0$.

Proposition 5.1. A bounded operator $S$ on a complex Banach lattice E is disjointness preserving if and only if there exists a linear operator $|S|$ on $E$ such that\\
(5.3) $\quad|S \pounds|=|S||E| \quad(\pounds \in E)$.

In that case the operator $|\mathrm{s}|$ is uniquely determined by (5.3). $|s|$ is a lattice homomorphism and the modulus of $S$ (i.e., one has $|S| \leqq T$ for all $T \in L(E)$ such that $|S f| \leqq T|f|$ ( $\mathrm{T} \in \mathrm{E})$ ).

For the proof of the proposition we refer to Arendt (1983) and de Pagter (1985).

Proposition 5.2. Let $(S(t))_{t \geqq 0}$ be a disjointness preserving semigroup. Let $T(t)=|S(t)|(t \geqq 0)$. Then $\left(T(t){ }_{t \geq 0}\right.$ is a strongly continuous semigroup.

Proof. Let $0 \leqq s, t$ and $f \in E_{+}$. Then by (5.1), T(s)T(t) $f=T(s)|s(t) f|$ $=|S(s) S(t) f|=|S(s+t) f|=T(s+t) f$. since span $E_{+}=E$, it follows that $(T(t))_{t \geq 0}$ is a semigroup. Moreover, for $f \in E_{+}, \lim _{t \rightarrow 0} T(t) f$ $=\lim _{t \rightarrow 0}|S(t) f|=|f|=f$. This implies that $(T(t))_{t \geqq 0}$ is strongly continuous.

Example 5.3. Let $d \in \mathbb{C}$ and $S(t)=T_{d}(t)$ be given by (3.9). Then $\left|T_{d}(t)\right|=T|d|(t) \quad(t \geqq 0)$.

Proposition 5.4. Let B be the generator of a disjointness preserving semigroup $(S(t))_{t \geq 0}$ on a Banach lattice E. Then $B$ is local; i.e. (5.4) $f \perp g$ implies $B f \perp g$ for all $f \in D(B), g \in E$.

Proof. Let $f \in D(B)$ and $g \in E$ such that inf $(|f|,|g|\}=0$. Then $|1 / t(S(t) f-f)| a|g| \leqq|1 / t S(t) f|$ a $|g|+1 / t|f|$ a $|g|$

$$
=1 / t|S(t) f| \wedge|g|
$$

$$
\begin{aligned}
& \leqq 1 / t|S(t) f| \wedge|S(t) g-g|+(1 / t|S(t) f|) \wedge|S(t) g| \\
& =1 / t|S(t) f| \wedge|S(t) g-g| \\
& \leqq|S(t) g-g| .
\end{aligned}
$$

Letting $t+\infty$ one obtains $|B f| \wedge|g|=0$.

We now describe the relation between the generator of a disjointness preserving semigroup and the generator of the modulus semigroup.

Theorem 5.5. Assume that $E$ is a complex Banach lattice with order continuous norm. Let $(S(t))_{t \geqq 0}$ be a semigroup with generator $B$. The following assertions are equivalent.\\
(i) (S(t)) $t \geqq 0$ is disjointness preserving.\\
(ii) There exists a semigroup $(T(t))_{t \geqq 0}$ with generator $A$ such that\\
(5.5) $f \in D(B)$ implies $|f| \in D(A)$ and $\operatorname{Re}((\operatorname{siggn} \bar{f}) B f)=A|f|$.

Moreover, if these equivalent conditions are satisfied, then $T(t)=|s(t)|(t \geqq 0)$.

Remark. By B-II,Lemma 2.9 the relation (5.5) is equivalent to $\langle\operatorname{Re}((s i g n \mathrm{E}) \mathrm{Bf}), \phi\rangle=\langle | \mathrm{f}\left|, \mathrm{A}^{\prime} \phi\right\rangle \quad\left(\mathrm{f} \in \mathrm{D}(\mathrm{B}), \phi \in \mathrm{D}\left(\mathrm{A}^{\prime}\right)\right)$.\\
b) It is remarkable that, in contrast with the situation considered in Theorem 3.8, here condition (ii) implies the positivity of $\left(T(t)_{t \geq 0}\right.$ without further assumptions.\\
The basic idea of the proof of Theorem 5.5 is to differentiate the equation $|S(t) f|=T(t)|f|$ (where $T(t)=|S(t)|$, cf. (5.3)). For\\
that we need that the modulus function is differentiable. If $\mathrm{E}=\mathrm{L}^{\mathrm{p}}(\mathrm{X}, \Sigma, \mu)(1 \leqq \mathrm{p}<\infty)$ this had been proved in section 2 (Ex.2.3). We extend this result to Banach lattices with order continuous norm.

Proposition 5.6. Let $E$ be a real or complex Banach lattice with order continuous norm. Then the modulus function $\theta: E \rightarrow E$ (given by $\theta(h)=|h|)$ is right-sided Gateaux differerentiable and


\begin{equation*}
D_{g} \Theta( \pm)=\operatorname{Re}((\operatorname{sign} \mathbf{f}) g) \quad(f, g \in E) \text {. } \tag{5.6}
\end{equation*}


Proof. Let $f, g \in E$. Define $k: \mathbb{R} \rightarrow E$ by $k(t)=|f+t g|-|f|$. Then $k(0)=0$ and $k$ is convex (i.e., $k(\lambda s+(1-\lambda) t) \leqslant \lambda k(s)+$ $(1-\lambda) k(t)$ for all $s, t \in \mathbb{R}, \lambda \in[0,1]$ ).\\
We show that


\begin{equation*}
k(s) / s \leqq k(t) / t \tag{5.7}
\end{equation*}


whenever $s<t, s, t \neq 0$.\\
First case: s < t < 0.\\
Choose $\lambda=t / s \in(0,1)$. Then $t=(1-\lambda) 0+\lambda s$. Consequently, $k(t) \leqq(1-\lambda) k(0)+\lambda k(s)=t / s k(s)$.\\
Second case: $s<0<t$.\\
Let $0<\lambda:=t /(t-s)<1$. Then $0=\lambda s+(1-\lambda) t$. Hence $0=k(0) \leqq$ $\lambda k(s)+(1-\lambda) k(t)=t /(t-s) k(s)-s /(t-s) k(t)$, which implies (5.7). Third case: $0<s<t$.\\
Let $\lambda=s / t \in(0,1)$. Then $s=(1-\lambda) 0+\lambda t$. Consequently, $k(s) \leqq$ $(1-\lambda) k(0)+\lambda k(t)=s / t k(t)$, which implies (5.7).

It follows from (5.7) that the net $(k(t) / t)$ is decreasing and bounded below (by $-k(-1)$, for instance). Since $E$ has order continuous norm, it follows that $D_{g} \theta(f)=i m_{t \rightarrow 0+} k(t) / t$ exists.\\
It remains to show that $D_{g} \theta^{(f)}=\operatorname{Re}(\operatorname{sign} \bar{f}) g$.\\
First of all denote by $P^{9}$ the band projection onto \{f\}d. Then it follows from the definition of $D_{g} \theta(f)$ that $D_{g} \theta(f)=P_{g}{ }^{\theta(f)}+$ $(I d-P) D_{g} \theta(f)=D_{P g} \theta(f)+|(I d-P) g|$. Thus it remains to show that (5.8) $D_{h} \theta(f)=\operatorname{Re}((\operatorname{sign} \bar{f}) h) \quad$ whenever $h \in\{f\}^{d d}$.

According to the Kakutani-Krein theorem there exists a compact space K such that $\mathrm{E}_{|f|}$ can be identified with $C(K)$. Then by B-II, Lemma 2.4\\
(5.9) $\quad \lim _{t \rightarrow 0+} 1 / t(|f+t h|-|f|)(x)=\operatorname{Re}(\operatorname{sign}(\overline{f(x)}) h(x)) \quad(x \in K)$. Let $\phi \in E_{+}^{\prime}$. Then $\phi$ restricted to $E_{|f|}$ can be identified with a regular Borel measure $\mu$ on $\mathrm{C}(\mathrm{K}$ ).

So it follows from (5.9) and the dominated convergence theorem that $\left\langle D_{h} \theta(f), \phi\right\rangle=\lim _{t \rightarrow 0+} I / t\langle(|f+t h|-|f||, \phi\rangle$

$$
=\int_{\mathrm{K}} \operatorname{Re}(\operatorname{sign}(\overline{\mathrm{f}}(\mathrm{x})) \mathrm{h}(\mathrm{x})) \mathrm{d} \mu(\mathrm{x})=\langle\operatorname{Re}((\operatorname{sign} \overline{\mathrm{f}}) \mathrm{h}), \phi\rangle
$$

(the last identity holds since by the definition of sign $\overline{\mathrm{f}} \in L(E)$, we have (sign $\overline{\text { f) }} h \in E_{|f|}=C(K)$ whenever $h \in C(K)$ and $((\operatorname{sign} \overrightarrow{\mathrm{f}}) \mathrm{h})(\mathrm{x})=(\operatorname{sign} \overrightarrow{\mathrm{f}}(\overline{\mathrm{x}}) \mathrm{h}(\mathrm{x}) \quad($ see $\mathrm{C}-\mathrm{I}, \mathrm{sec} .8))$. Consequently, $D_{h} \ominus(f)=\operatorname{Re}(\operatorname{sign} \bar{f}) h$ whenever $h \in E_{|f|}$. Since $D_{h} \ominus(f)$ is continuous in $h$ (in fact, $\left|D_{h} \theta(f)-D_{k} \theta(f)\right| \leqq|h-k|$ for all $h, k \in E\}$ and $E|f|$ is dense in \{f\}d, it follows that (5.8) holds for all $h \in\{f\}^{d d}$.

Remark 5.7. a) By the same argument as given in the proof one sees that 0 is left-sided Gateaux differentiable and

$$
\mathrm{D}_{\mathrm{g}}^{-} \theta(f)=\operatorname{Re}((\operatorname{sign} \bar{f}) g)-\mathrm{p}_{\mathrm{f}}^{d}|g|
$$

for all $f, g \in E$, where $D_{g}^{-} \theta(f)=\lim _{t_{f} 0} 1 / t(\theta(f+t g)-\theta(f))$ and $P_{f}^{d}$ denotes the band projection onto $\{f\}$. In particular,


\begin{equation*}
D_{g}^{+} \theta(f)=D_{g}^{-} \theta(f) \quad \text { whenever } g \in\{f\}^{d d} \tag{5.10}
\end{equation*}


b) The proof of Prop.5.6 shows that every convex function $\theta: E+E_{\mathbb{R}}$ (where $E$ is a Banach lattice with order continuous norm) is right(and left-) sided Gateaux differentiable (cf. Arendt (1982)).

Proof of Theorem 5.5. Assume that (i) holds. Let $f \in D(B)$. Then $S(t) f$ is differentiable in $t$. By the chain rule B-II, Prop.2.3, $T(t)|E|=|S(t) f| \quad$ is also differentiable and $d / d t|t=0 T(t)| f \mid=$ $\bar{d} /\left.\mathrm{dt}\right|_{t=0}|\mathrm{~S}(t) f|=\operatorname{Re}(\operatorname{sig} \mathrm{n} \mathrm{f}) \mathrm{Bf} \quad$ (by Prop.5.6). Hence $|f| \in D(A)$ and $|\mathrm{A}| \mathbf{f} \mid=\operatorname{Re}(\operatorname{sig} n \mathrm{f}) \mathrm{Bf}$.\\
Conversely, assume that (ii) holds. Let $s>0, f \in E$. We show that $|S(s) f|=T(s)|f|$. This implies that $S(s)$ is disjointness preserving and $|S(s)|=T(s)$ (by Proposition 5.1)). Since $D(B)$ is dense we can assume that $f \in D(B)$. Let $\xi(t)=T(s-t)|S(t) f|(t \in[0, s])$, since by assumption $|S(t) f| \in D(A)$ one obtains $d / d t \xi(t)=-A T(s-t)|s(t) f|+T(s-t) d / d r|r=t| s(r) f \mid$

$$
=-A T(s-t)|S(t) f|+T(s-t)(\operatorname{Re}(\operatorname{sign} \overline{S(t) f}) B S(t) f)
$$

(by Prop.5.6 and the chain rule B-II, Prop.2.3)\\
$=0$ by the assumption (ii).\\
Hence $\xi(0)=\xi(s)$; i.e., $|S(s) f|=T(s)|f|$.

The case when $S(t)=T(t) \quad(t \geqq 0)$ is of special interest: it yields a characterization of generators of lattice semigroups.\\
Recall that if a semigroup $(T(t))_{t \geq 0}$ is positive, i.e., if\\
(5.13) $|T(t) f| \leqq T(t)|f| \quad$ (E $\in E)$,\\
then its generator A satisfies Kato's inequality. We now obtain from Theorem 5.5: the semigroup consists of lattice homomorphisms (i.e., the equality holds in (5.13)) if and only if A satisfies Kato's equality. The precise statement is the following.

Corollary 5.8. Iset $A$ be the generator of a strongly continuous semigroup $(T(t))_{t \geqq 0}$ on a Banach lattice $E$ with order continuous norm. The following assertions are equivalent.\\
(i) (T(t) ${ }_{t \geqq 0}$ is a lattice semigroup.\\
(ii) $f \in D(A)$ impiies $|f| \in D(A)$ and $\operatorname{Re}((\operatorname{sign} \bar{f}) A f)=A|f|$.\\
(iii) $f \in D(A)$ implies $|f|, \bar{f} \in D(A)$ and $\operatorname{Re}((\operatorname{sign} \bar{f}) A f)=A|f|$ (Kato's equality).

Proof. The equivalence of (i) and (ii) follows directly from Thm. 5.5. If (i) holds, then $A$ is local by Prop. 5.4. Thus (sign $\bar{f}$ )Af $=$ (sign $\bar{f}$ )Af for all $f \in D(A)$ and so (iii) holds since (ii) is valid.\\
Assume now that (iii) holds. Then Kato's equality implies that $A f \in\{f\}^{d d}$ whenever $f \in D(A)+$. Since $D(A)$ is a sublattice of $E$ by hypothesis, this implies that $A$ is local. Thus (ii) follows from (iii).

In the case when $E$ is real this result can be reformulated.

Corollary 5.9. Let A be the generator of a strongly continuous semigroup $(T(t))_{t \geqq 0}$ on a real Banach lattice $E$ with order continuous norm. The following assertions are equivalent.\\
(i) $(\mathrm{T}(\mathrm{t}))_{t \geq 0}$ is a lattice semigroup.\\
(ii) D(A) is a sublattice and $A$ is local.

Proof. Assume that (ii) holds. Let $E \in D(A)$, and set $P_{+}:=P_{f}+$ and $P_{-}:=P_{f^{-}}$.\\
Then $\left.\overline{(P}_{+}\right) A f^{-}=\left(P_{-}\right) A f^{+}=0$ since $A$ is local. Hence\\
(sign f) $\mathrm{Af}_{-}=\left(\mathrm{P}_{+}-\mathrm{P}_{-}\right) \mathrm{Af}=\left(\mathrm{P}_{+}-\mathrm{P}_{-}\right)\left(\mathrm{Af}^{+}-\mathrm{Af}^{-}\right)=\left(\mathrm{P}_{+}\right) \mathrm{Af}^{+}+\left(\mathrm{P}_{-}\right) \mathrm{Af}^{-}=$ $\mathrm{Af}^{+}+\mathrm{Af}^{-}=\mathrm{A}|\mathrm{f}|$.

Thus Kato's equality holds and it follows from corollary 5.8 that $(\mathrm{T}(t))_{t \geq 0}$ is a lattice semigroup. The other implication follows directly from Corollary 5.8.

Example 5.10. Let $E=L^{P}(X, \mu)$ (where $(X, \mu)$ is a o-finite measure space and $1 \leq p<\infty$ ) and let $A_{0}$ be the generator of a semigroup of lattice homomorphisms. Let $h \in L^{\infty}$ and $B=A_{0}+h$ (i.e., $B$ is given by $B f=A_{0} f+h \cdot f$ for $f \in D(B)=D\left(A_{0}\right)$. Let $A=A_{0}+\operatorname{Re} h$. Since $A_{0}$ generates a semigroup of lattice homomorphisms, we have $|f| \in D\left(A_{0}\right)$ whenever $f \in D\left(A_{0}\right)$ and $\operatorname{Re}\left((s i g n \operatorname{f}) A_{0} f\right)=A_{0}|f|$. Hence $\left.\operatorname{Re}((\operatorname{sign} \bar{f}) B f)=\operatorname{Re}\left((s i g n n \bar{f}) A_{0} f\right)+(\operatorname{Re} h) \cdot|f|\right)=$ $A_{0}|f|+(\operatorname{Re} h) \cdot|f|=A|f|$ for all $f \in D(B)$. Thus it follows from Theorem 5.5 that $B$ generates a disjointness preserving semigroup whose modulus semigroup is generated by A .

Next we describe when a disjointness preserving semigroup is positive.

Proposition 5.11. Let $E$ be a complex Banach lattice with order continuous norm and $B$ be the generator of a disjointness preserving semigroup $(S(t))_{t \geqslant 0}$. The semigroup is positive if and only if $B$ is real and $\operatorname{span} D(B)_{+}=D(B)$.

Proof. The conditions are clearly necessary. In order to prove sufficiency, we can assume that $E$ is real. Denote by $A$ the generator of $(T(t))_{t \geq 0}$, where $T(t)=|S(t)|$. Let $\pounds \in D(B)_{+}$. since $B$ is local we have $\mathrm{Bf}=\mathrm{P}_{\mathrm{f}} \mathrm{Bf}=(\operatorname{sign} \mathrm{f}) \mathrm{Bf}=\mathrm{A}|\mathrm{f}|=\mathrm{Af}$. By assumption, $\operatorname{span} D(B)_{+}=D(B)$. Thus it follows that $B \in A$. This implies that $B=A$ since $\rho(B) \cap \rho(A) \neq \varnothing$.

Remark 5.12. If $B$ is the generator of a disjointness preserving semigroup $(S(t))_{t \geq 0}$ on a real Banach lattice $E$ with order continuous norm then Kato's inequality holds in the reverse sense; i.e.,

\begin{verbatim}
<(sign f) Bf,\phi\rangle \geq <|f|,B'\phi> for all f \in D(B), \phi \in D(B'),
\end{verbatim}

(cf. (3.9) for a concrete example). In fact, let $T(t)=|S(t)|$ and denote by $A$ the generator of $(T(t))_{t \geqslant 0}$. Let $f \in D(B), \phi \in D\left(B^{\prime}\right)_{+}$. Then $\langle(\operatorname{sign} f) \mathrm{Bf}, \phi\rangle=\langle\mathrm{A}| \mathrm{f}|, \phi\rangle=\lim _{t \rightarrow 0}(1 / t)\langle T(t)| f|-|\mathrm{f}|, \phi\rangle \geqq$ $\left.\lim _{t \rightarrow 0} 1 / t<S(t)|f|-|f|, \phi\right\rangle=\langle | f\left|, B^{\prime} \phi\right\rangle$.

Finally, we come back to Corollary 5.9. If in condition (ii) we demand that $D(A)$ is not only a sublattice but an ideal of $E$ we obtain a characterization of multiplication semigroups.

Here we call a semigroup $\left(T(t){ }_{t \geq 0}\right.$ multiplication semigroup if $\mathrm{T}(\mathrm{t})$ is a multiplication operator (i.e., an element of the center) for every $t>0$.

Theorem 5.13. Let $A$ be the generator of a strongly continuous semigroup $(T(t))_{t \geqslant 0}$ on a $\sigma$-order complete real or complex Banach lattice E. The following assertions are equivalent.\\
(i) $(T(t))_{t \geq 0}$ is a multiplication semigroup.\\
(ii) There exists $\lambda \in \rho(\mathrm{A})$ such that $\mathrm{R}(\lambda, \mathrm{A})$ is a multiplication operator.\\
(iii) $R(\lambda, A)$ is a multiplication operator for all $\lambda \in \rho(A)$.\\
(iv) $A$ is local and $D(A)$ is an ideal in E.\\
(v) If $f \in D(A)$ then $P f \in D(A)$ for every band projection $P$ on E and APf = pAf.

Proof. Assume that (i) holds and let $\lambda>\omega(A)$. Since $R(\lambda, A)$ is the Laplace transform of the semigroup, it follows that $R(\lambda, A)$ is local since $T(t)$ is local for all $t \geqq 0$. This implies $R(\lambda, A) \in Z(E) \quad$ (see $C-I, \operatorname{Sec} .9$ ).\\
We show that (ii) implies (v). Assume that $\lambda \in \rho(A)$ such that $R(\lambda, A)$ is a multiplication operator. Let $P$ be a band projection. Then $\operatorname{PR}(\lambda, A)=R(\lambda, A) P$. Let $f \in D(A), g:=(\lambda-A) f$. Then $P f=$ $\operatorname{PR}(\lambda, A) g=R(\lambda, A) P g$. Hence $\operatorname{Pf} \in D(A)$ and $(\lambda-A) P f=\operatorname{Pg}$. Thus $\mathrm{APf}=\lambda \mathrm{Pf}-\mathrm{Pg}=\mathrm{P}(\lambda f-g)=P A f$.\\
We show that ( $v$ ) implies (iii). Let $\lambda \in \rho(A)$ and $P$ be a band projection. We have to show that $\operatorname{PR}(\lambda, A)=R(\lambda, A) P$. Let $g \in E$, $f:=R(\lambda, A) g$. Then $\operatorname{Pf} \in D(A)$ and $A P f=\operatorname{PAf}$. Hence $\operatorname{PR}(\lambda, A) g=\operatorname{Pf}$ $=R(\lambda, A)(\lambda-A) P f=R(\lambda, A) P(\lambda-A) f=R(\lambda, A) P g$. It follows from $C-I$, Sec. 9 that $R(\lambda, A) \in Z(E)$.\\
(iii) implies (i) since $T(t)=\lim _{n \rightarrow \infty}[n / t R(n / t, A)]^{n}$ strongly for all $t>0$.\\
It remains to show the equivalence of (iv) and (v). Assume that (iv) holds, let $f \in D(A)$ and $P$ be a band projection. Then $P f \in D(A)$ and $(I d-P) f \in D(A)$ by the assumption. Since $A$ is local we have $A P f=$ PAPf $+(I d-P) A P f=$ PAPf $=$ PAPf + PA(Id-P)f $=$ PAf. Conversely, assume (v). Let $f \in D(A)$ and $|g| \leqq|f|$. Then there exists a band projection $P$ such that $p f=g$. Hence $g \in D(A)$. We have shown\\
that $D(A)$ is an ideal. Assume that inf\{ $|h|,|f|\}=0$. Denote by $P$ the band projection onto $(|h|\}^{\text {dd }}$. Then $\mathrm{PAf}=\mathrm{APf}=\mathrm{A} 0=0$. Thus Af $\in\{|h|\}^{d}$. We have proved that $A$ is local.

Corollary 5.14. A multiplication semigroup (T(t)) $t \geq 0$ on a complex Banach lattice $E$ with order continuous norm is positive if and only if its generator $A$ is real; i.e., $f \in D(A)$ implies $\bar{f} \in D(A)$ and $\overline{A f}=\overline{(\mathrm{Af})}$.

Proof. The condition is equivalent to $T(t) E_{\mathbb{R}} \subset E_{\mathbb{R}} \quad(t \geqq 0)$ (cf. Rem. 3.1), so it is clearly necessary. Conversely, if A is real, then denote by $\left(\mathrm{T}_{\mathbb{R}}(t)\right)_{t \geqq 0}$ the restriction semigroup on $\mathrm{E}_{\mathbb{R}}$ and by ${ }^{A_{R}}$ its generator. Then $A_{\mathbb{R}}$ is local (since $A$ is local) and $D\left(A_{R}\right)$ is a sublattice of $E_{\mathbb{R}}$. Thus $\left(T_{\mathbb{R}}(t)\right)_{t \geqq 0}$ is a lattice semigroup (and so positive) by Cor. 5.9.

The class of bounded operators which generate a lattice semigroup is very restricted.

Proposition 5.15. Let $E$ be a real or complex Banach lattice and $A \in L(E)$. The following assertions are equivalent.\\
(i) $A \in Z(E)$.\\
(ii) $e^{t A}$ is disjointness preserving for all $t \geqq 0$.\\
(iii) $e^{t A} \in Z(E)$ for all $t \in \mathbb{R}$.

Moreover, if $A \in Z$ (E) is real, then $e^{t A} \geqq 0$ for all $t \in \mathbb{R}$.

Proof. Since $Z(E)$ is a closed subalgebra of $L(E)$ (see C-I, Sec.9), it is clear that (i) implies (iii). Assertion (ii) follows trivially from (iii). If (ii) holds, then $A$ is local by Prop.5.4. Hence $A \in Z(E)$.\\
The last assertion follows from the fact that $Z(E)$ is isomorphic to a space $C(K)$ as a Banach lattice and a Banach algebra.

Proposition. 5.16. Let $E$ be a complex Banach lattice. Every strongly continuous group $(T(t))_{t \geqslant 0}$ of real operators contained in $Z(E)$ has a bounded generator.

Proof. Let $(T(t))_{t \geq 0}$ be a strongly continuous multiplication semigroup. There exist $w \in \mathbb{R}, M \geqq 1$ such that $\|T(t)\| \leqq M e^{w|t|}$\\
$(t \geqq 0)$. Then $\|f\|_{1}:=\sup _{t \geqq 0}\left\|e^{-w t}(t)|f|\right\|$ defines an equivalent lattice norm on $E$ for which $\|T(t)\|_{1} \leqq e^{\text {wt }}(t \geq 0)$. Since $Z(E)$ is isometrically isomorphic to a space $\mathrm{C}(\mathrm{K})$ (as a Banach lattice), for an operator $s \in Z(E)$ one has $\|s\|=\inf \{c>0:|s| \leqq c \cdot I d\}$. Hence the operator norm of $s$ is independent of which lattice norm equivalent to the given one is considered on E . Consequently, $\|\mathrm{T}(\mathrm{t})\|=\|\mathrm{H}(\mathrm{t})\|_{1} \leqq e^{\mathrm{wt}} \quad(\mathrm{t} \geqq 0)$.\\
If $(T(t))_{t \geqslant 0}$ is a strongly continuous group contained in $Z(E)$, then it follows that $\|T(t)\| \leqq e^{w|t|}$ ( $t \varepsilon \mathbb{R}$ ) for some $w \geqq 0$. If in addition the operators $T(t)$ are real one obtains from the above expression for the operator norm that

$$
e^{-w t} \cdot I d \leqq \mathbb{T}(t) \leqq e^{w t} \cdot I d \quad(t \geqq 0)
$$

Consequently, $\lim _{t+0}\|\mathrm{~T}(t)-I d\|=0$.

The assumption that the group consists of real operators is essential in Proposition 5.16. In fact, many differential operators on $\mathrm{L}^{2}\left(\mathbb{R}^{\mathrm{n}}\right)$ generate a strongly continuous group which (via Fourier transformation) is similar to a multiplication group. A concrete example is the Laplacian (A-I, Example 2.8).\\
on the other hand, if $E=C(K)$ ( K compact), then every strongly continous multiplication semigroup $(T(t)){ }_{t \geqq 0}$ has a bounded generator.\\[0pt]
[In fact, let $m_{t}=T(t) 1(t \geqq 0)$. Then $\lim _{t \neq 0}\|T(t)-I d\|=$ $\lim _{t+0}\left\|\mathrm{~m}_{t}-1\right\|_{\infty}=0 . I$

Lemma 5.17. Let E be a real Banach lattice with order continuous norm. Let $A \in L(E)$. Assume that there exists a dense sublattice $D$ of E such that for all $\mathrm{f} \in \mathrm{D}, \mathrm{g} \in \mathrm{E}, \mathrm{f} \perp \mathrm{g}$ implies $\mathrm{Af} \perp \mathrm{g}$. Then $A \in Z(E)$.

Proof. Let $0 \leqq f \in D, \phi \in E_{+}^{\prime}$ such that $\langle f, \phi\rangle=0$. Since Af $\epsilon$ \{f $^{\text {da }}$ by assumption, it follows that $\langle A f, \phi\rangle=0$. Thus ${ }^{A} / D$ and $-A_{D}$ satisfy (P). It follows from Thm.1.8 that $\left(e^{t A}\right)_{t \in \mathbb{R}}$ is a positive group. Thus $A \in Z(E)$ by Prop.5.15.

Let $A$ be the generator of a positive semigroup and $B \in L(E)$. The semigroup generated by $A+B$ is positive whenever $\left(e^{t B}\right) t \geq 0$ is positive (this follows from (1.8)). However this condition is not\\
necessary. [For example, let $A \in L(E)$ such that ( $\left.e^{t A}\right)_{t \geq 0}$ is positive and let $B=-A$. Then $A+B$ generates a positive semigroup, but $\left(e^{t B}\right)_{t \geq 0}$ is positive only if $A \in Z(E)$. The situation is different when A generates a lattice semigroup.

Theorem 5.18. Let E be a real Banach lattice with order continuous norm and $A$ be the generator of a lattice semigroup. Let $B \in L(E)$. The semigroup generated by $A+B$ is positive if and only if $\left(e^{t B}\right)_{t \geq 0}$ is positive. The semigroup generated by $A+B$ is a lattice semigroup if and only if $B \in Z(E)$.

Proof. Assume that $A+B$ generates a positive semigroup. Let $f \in D(A)_{+}, \phi \in E_{+}^{\prime}$ such that $\left\langle E_{,} \phi\right\rangle=0$. since $A$ is local, it follows that $\langle A f, \phi\rangle=0$. But $\langle(A+B) f, \phi\rangle \geqq 0$ by Prop.1.7. Hence $\langle B f, \phi\rangle \geqslant 0$. We have shown that ${ }^{B} \mid D_{D}(\mathrm{~A})$ satisfies the positive minimum principle (Def.1.6). Since $D(A)$ is a sublattice of E (by Cor.5.9), it follows from Thm.1.8 that $\left(e^{t B}\right)_{t \geqq 0}$ is positive.\\
By Cor.5.9 the operator A + B generates a lattice semigroup if and only if $A+B$ is local. Since $A$ is local, this is equivalent to ${ }^{B} \mid D(A)$ being local. By Lemma 5.17 this is true if and only if $B \in Z(E)$.

NOTES.\\
Section 1. The notion of dispersiveness is due to Phillips (1962) who uses a semiscalar product instead of the subdifferential of the canonical half-norm. Our approach follows Arendt-Chernoff-Kato (1982). Bounded generators of positive semigroups on a special class of ordered Banach spaces (which includes Banach lattices and C*-algebras) were characterized by the positive minimum principle by Evans and Hanche-01sen (1979). The equivalence of (i) and (iv) in Theorem 1.10 is due to NagelUhlig (1981). Theorem 1.8 has been obtained independently by Arendt (1984a) and van Casteren (1984).

Section 2. The classical distributional Kato's inequality for the Laplacian is due to Kato (1973). It is a most elegant tool to prove essential selfadjointness of Schrödinger operators with domain $C_{c}^{\infty}\left(\mathbb{R}^{n}\right)$ (cf. Example 4.7).\\
The relation between Kato's inequality and positivity of $e^{t \Delta}$ was firgt pointed out by Simon (1977). A criterion for a formegative operator on a space $\mathrm{L}^{2}$ to generate a positive semigroup is given by Beurling-Deny (1958), see also Reed-Simon (1978), Vol. IV, Sec.XIII.12. It was a conjecture of Nagel that some abstract version of

Kato's inequality characterizes the positivity of the semigroup (cf. Nagel-Uhlig (1981)). The necessity of Kato's inequality in the form given in Thm. 2.4 was first proved in [Arendt (1982), Remark 3.10] with a different proof. The proof we give here appeared in Arendt (1984). Miyajima-Okazawa (1984) use this inequality to show that a differential operator on $\mathrm{L}^{\mathrm{P}}\left(\mathbb{R}^{\mathrm{n}}\right)$ which generates a positive semigroup is necessarily of order $\leq 2$ and has an elliptic principal part. This result is generalized to the spaces $\mathrm{L}^{\mathrm{P}}(\Omega), \Omega \subset \mathbb{R}^{\mathrm{n}}$ suitable, by Miyajima (1986).

Section 3. In this section we closely follow Arendt (1984). Theorem 3.8, in a similar form but with different proof, has been obtained independently by Schep (1985).

Section 4. The characterization of domination by Kato's inequality on a Hilbert space is due to Simon (1977). Further contributions are due to Hess-Schrader-Uhlenbrock (1977) and Kishimoto-Robinson (1980). Theorem 4.3 is due to Arendt (1984b). The result on Schrödinger operators on $L^{\mathrm{p}}\left(\mathbb{R}^{\mathrm{n}}\right)$ stated in Example 4.7 is due to Kato (1986). The case $\mathrm{p}=2$ was proved in Kato (1973), where the classical Kato's inequality was established. Extensive information on Schrödinger semigroups on $\mathrm{L}^{\mathrm{p}}\left(\mathbb{R}^{\mathrm{n}}\right)$ is given in Simon (1982). Other recent results on the $\mathrm{L}^{\mathrm{P}}$-theory of Schrödinger operators are obtained by Davies (1986), Okazawa (1984) and Voigt (1984).\\
The existence of the modulus semigroup of semigroups with bounded, regular generator (Theorem 4.17) is due to Derndinger (1984) (in the real case).\\
Proposition 5.15 had been proved in Schaefer-Wolff-Arendt (1978) by a completely different method.

Section 5. The characterization of generators of lattice semigroups on a Banach lattice with order continuous norm (Cor. 5.8) is due to Nagel-Uhlig (1981). An extension of this result to arbitrary Banach lattices is given by Arendt (1982) from which the proof of Prop. 5.6 is taken as we 11.\\
Local closed operators having an ideal as domain (i.e., operators satisfying condition (iv) of Thm. 5.13) are investigated in detail by Nakano (1950) who calls them dilatators. Peetre (1959) characterizes differential operators by locality (see also Luxemburg (1979)). In the context of C*-algebras local operators are investigated by Batty (1985) and Batty-Robinson (1985).

\section*{SPECTRALTHEORY OF POSITIVE SEMIGROUPS ON BANACH LA T T I C E S }
by\\
Günther Greiner

In Chapter B-III we have shown that positive semigroups on spaces $C_{0}(x)$ possess several interesting spectral properties. Now we are going to extend many of the results obtained there to the more general setting of Banach lattices. We will improve some of the results of B-III considerably and give the complete proof of B-III, Thm.4.1.

Throughout this chapter we will assume that $E \neq\{0\}$ is a complex Banach lattice.

\section*{1. THE SPECTRAL BOUND}
The fact that the spectral bound of a positive semigroup is always contained in the spectrum (provided that the spectrum is non-empty) is also true in the general setting of Banach lattices. The proof given in B-III,Thm.1.I for spaces $C_{0}(X)$ works in the general case too. Another proof is given below (cf. Cor.1.4). Furthermore, Cor.1.3 and Prop.1.5 of B-III are true in the setting of Banach lattices and their proofs can be carried over to the general case. For the sake of completeness we summarize these results in the following theorem.

Theorem 1.1. Let $A$ be the generator of a positive semigroup $(T(t))_{t \geqq 0}$ on a Banach lattice E.\\
(a) $s(A) \in \sigma(A)$ unless $\sigma(A)=\varnothing$.\\
(b) For $\lambda_{0} \in \rho(A)$ we have:\\
$R\left(\lambda_{o_{0}}, A\right)$ is positive if and only if $\lambda_{O_{1}}>s(A)$. In this case $r\left(R\left(\lambda_{O}, A\right)\right)=\left(\lambda_{0}-s(A)\right)^{-1}$.\\
(c) If $T(1)$ has a positive fixed vector $h_{0}$, then ker $A$ contains a positive element $h$ such that $h_{0} \in \overline{E_{|h|}}$.\\
(d) If $T(1)^{\prime} \phi_{O}=\phi_{O}$ for some $\phi_{O} \in E_{+}^{\prime}$ then there exists $\phi \in D\left(A^{*}\right)_{+}$with $\{f \in \mathrm{E}:\langle | \mathrm{f}|, \phi\rangle=0\} \subseteq\left\{f \in \mathrm{E}:\langle | f\left|, \phi_{0}\right\rangle=0\right\}$ such that $A^{*} \phi=0$.

The fact that $s(A)$ is always an eigenvalue of the adjoint (cf. B-III Thm.1.6) is characteristic for spaces $C(K), K$ compact, as can be seen considering the Laplacian on $\mathrm{L}^{\mathrm{P}}\left(\mathbb{R}^{\mathrm{n}}\right)$ where $1<\mathrm{p}<\infty$ or on $C_{o}\left(\mathbb{R}^{n}\right)$ (see B-III,Ex.1.7). Another result which cannot be extended to arbitrary Banach lattices is that spectral bound and growth bound coincide (cf. B-IV,Thm.1.4); an example is given in A-III,Ex.1.3. Despite of this the resolvent $R(\lambda, A)$ of a positive semigroup is given as the Laplace transform of the semigroup in the half-plane $\{\mathbf{z} \in \mathbb{C}: \operatorname{Re} z>s(A)\}$ (even in case that $\omega(A)>s(A)$ ). Note however that the integral exists only as an improper Riemann integral. By Datko's Theorem (A-IV,Thm.1.11) the function $t \rightarrow e^{-\lambda t}$.T(t) f cannot be Bochner integrable for all $\mathrm{f} \in \mathrm{E}$ in case $\operatorname{Re} \lambda \leqq \omega(\mathrm{A})$.

Theorem 1.2. Suppose $A$ is the generator of a positive semigroup $(T(t))_{t \geq 0}$. For $\operatorname{Re} \lambda>s(A)$ we have:\\
(1.1) $R(\lambda, A) f=\lim _{t \rightarrow \infty} \int_{0}^{t} e^{-\lambda s} T(s) f d s$ for all $f \in E$.

Moreover, the operators $\int_{0}^{t} e^{-\lambda s} T(s) d s$ tend to $R(\lambda, A)$ with respect to the operator norm as $t \rightarrow \infty$.

Proof. We fix $\lambda_{0}>\omega(A)$. Then by A-I,Prop.1.11\\
(1.2) $R\left(\lambda_{0}, A\right)^{n+1} f=\frac{1}{n!} \int_{0}^{\infty} s^{n} \exp \left(-\lambda_{0} s\right) T(s) f d s \quad\left(n \in N_{O}, f \in E\right)$

Given $\mu$ such that $s(A)<\mu<\lambda_{0}, E \in E_{+}, \phi \in E_{+}^{*}$ then\\
(1.3) $\langle R(\mu, A) f, \phi\rangle=\sum_{n=0}^{\infty}\left(\lambda_{0}-\mu\right)^{n}\left\langle R\left(\lambda_{o}, A\right)^{n+1} f, \phi\right\rangle=$

$$
\begin{aligned}
& =\sum_{n=0}^{\infty} \int_{0}^{\infty} \frac{1}{n!}\left(\left(\lambda_{o}-\mu\right) s\right)^{n_{e}} \exp \left(-\lambda_{0} s\right)<T(s) f, \phi>d s= \\
& =\int_{0}^{\infty} \sum_{n=0}^{\infty} \frac{1}{n!}\left(\left(\lambda_{o}-\mu\right) s\right)^{n} \exp \left(-\lambda_{0} s\right)<T(s) f, \phi>d s= \\
& =\int_{0}^{\infty} \exp \left(\left(\lambda_{0}-\mu\right) s\right) \exp \left(-\lambda_{o} s\right)<T(s) f, \phi>d s= \\
& =\int_{0}^{\infty} \exp (-\mu s)<T(s) f, \phi>d s=1 i m_{t \rightarrow \infty}<\int_{0}^{t} \exp (-\mu s) T(s) f d s, \phi>
\end{aligned}
$$

Note that one can interchange sumation and integration because all the integrands are positive functions.\\
It follows from (1.3) that the net $\left(\int_{0}^{r} \exp (-\mu s) \mathrm{T}(\mathrm{s}) \mathrm{f} \mathrm{ds}\right)_{r \geqslant 0}$ converges weakly to $\mathrm{R}(\mu, \mathrm{A}) \mathrm{f}$ for $\mathrm{r} \rightarrow \infty$. Because it is monotone increasing (f $\geqq 0$ ), we have strong convergence (see the corollary to II.Thm. 5.9 in Schaefer (1974)).

If $\lambda=\mu+$ iv with $\mu, \nu$ real and $\mu>s(A)$ we have for arbitrary $\mathbf{f} \in \mathrm{E}, \phi \in \mathrm{E}^{\prime}$ :\\
$\left|<\int_{r}^{t} e^{-\lambda s_{T}}(s) f d s, \phi>\left|\leq \int_{r}^{t} e^{-\mu s}<T(s)\right| f\right|,|\phi|>d s \quad$ hence\\
$\left\|<\int_{r}^{t} e^{-\lambda s_{T}(s) f d s \| \leqq}\right\| \int_{r}^{t} e^{-\mu s_{T}}(s)|f| d s \| \quad$ which shows that\\
$\lim _{t \rightarrow \infty} \int_{0}^{t} e^{-\lambda s} \mathrm{~T}(s) f \mathrm{ds}$ exists.\\
Thus $R(\lambda, A) f=\int_{0}^{\infty} e^{-\lambda s} T(s) f d s$ by $A-I, P r o p .1 .11$.\\
It remains to prove that the net $\left(\int_{0}^{r} \exp (-\mu s) \mathrm{T}(s) \mathrm{ds}\right)_{r \geqq 0}$ converges with respect to the operator norm. We fix $\mu$ such that\\
$s(A)<\mu<\operatorname{Re} \lambda$. As we have seen above the map $s \rightarrow e^{-\mu S}\langle T(s) f, \phi\rangle$ is Lebesgue integrable for every $(f, \phi) \in E \times E^{\prime}$, thus defining a bilinear map $b: E \times E^{\prime} \rightarrow L^{1}\left(\mathbb{R}_{+}\right)$. Using the closed graph theorem it is easy to see that $b$ is separately continuous, hence jointly continuous by [Schaefer (1966). III.Thm.5.1]. Thus there is a constant $M$ such that\\
(1.4) $\quad \int_{0}^{\infty} e^{-\mu s}|<T(s) f, \phi\rangle \mid d s=\|b(f, \phi)\| \leqq M\|f\|\|\phi\| \quad\left(f \in E, \phi \in E^{\prime}\right)$

Given $0 \leqq t<r$ and setting $\varepsilon:=\operatorname{Re} \lambda-\mu$ we have:\\
$\left|\int_{t}^{r} e^{-\lambda s}\langle T(s) f, \phi\rangle d s\right| \leq \int_{t}^{r} \exp (-(\operatorname{Re} \lambda-\mu) s) e^{-\lambda s}|\langle T(s) f, \phi\rangle| d s$\\
$\leqq e^{-\varepsilon t} \int_{t}^{x} e^{-\lambda s} \mid\langle T(s) f, \phi>| d s \leqq e^{-\varepsilon t} M\|f\|\|\phi\|$.\\
It follows that $\| \int_{t}^{r} e^{-\lambda s_{T}(s)}$ ds $\| \mathrm{Me}^{-\varepsilon t}$, hence\\
$\left(\int_{0}^{t} e^{-\lambda s} T(s) d s\right) t \geqslant 0$ is a Cauchy net with respect to the operator norm.

Theorem 1.2 has many consequences. In particular, we can conclude that $s(A) \epsilon \sigma(A)$ whenever $s(A)>-\infty$ (without using the analogous result for bounded operators, cf, Cor. 1.4 below). In each of the following corollaries we assume that $A$ is the generator of a positive semigroup $(T(t))_{t \geqq 0}$ on a Banach lattice E.

Corollary 1.3. If $\operatorname{Re} \lambda>\mathrm{s}(\mathrm{A})$ then we have


\begin{equation*}
|R(\lambda, A) f| \quad \leqq \quad R(\operatorname{Re} \lambda, A)|f| \quad(f \in E) . \tag{1.5}
\end{equation*}


The proof is an immediate consequence of Thm.1.2.

Corollary 1.4. We have $s(A) \in \sigma(A)$ unless $s(A)=-\infty$.

Proof. Assume that $s(A)>-\infty$ and $s(A) \notin \sigma(A)$, then it follows from (1.5) that $\{R(\lambda, A): \operatorname{Re}>s(A)\}$ is uniformly bounded in $L(E)$,\\
by $M$ say. Then $\{z \in \mathbb{C}: \operatorname{Re} z=s(A)\} \subseteq \rho(A)$ and $\|R(z, A)\| \leqq M$ for $z$ with $\operatorname{Re} z=s(A)$. It follows that $\left\{z \in \mathbb{C}:|\operatorname{Re} z-s(A)|<M^{-1}\right\}$ $\subseteq \rho(A)$, which is absurd by the definition of $s(A)$.

Corollary 1.5. Suppose that $s(A)$ is a pole of order $m$ of the resolvent $R(\lambda, A)$. Then $m$ is a majorant for the order of any other pole on the line $s(A)+i \mathbb{R}$.

Proof. Without loss of generality we may assume that $\mathrm{s}(\mathrm{A})=0$. By (1.5) we have $\|R(\varepsilon+i B, A)\| \leq\|R(\varepsilon, A)\|$ for every $B \in \mathbb{R}, \varepsilon>0$. Therefore $\lim _{\varepsilon \rightarrow 0}\left\|\varepsilon_{R}(\varepsilon+i B, A)\right\| \leqq \lim _{\varepsilon \rightarrow 0}\left\|\varepsilon k_{R}(\varepsilon, A)\right\|=0$ for $k>m$.

The spectrum of a positive semigroup may be empty (see B-III, Ex.1.2(a)) and the spectrum of a general group may be empty as well (see [Hille-Phillips (1957), Sec.23.16]). However, for positive groups this cannot occur. More precisely, we have the following result:

Corollary 1.6. If A is the generator of a positive group then $\sigma(\mathrm{A}) \cap \mathbb{R} \neq \varnothing$.

Proof. Both $A$ and $-A$ are generators of positive semigroups, hence if $\sigma(A)=\emptyset$, then $s(A)=s(-A)=-\infty$ and (1.5) implies that $\{R(\lambda, A): \operatorname{Re} \lambda \geq 0\} U\{R(\lambda,-A): \operatorname{Re} \lambda \geqq 0\}$ is bounded in $L(E)$, i.e.., $\{R(\lambda, A): \lambda \in C\}$ is bounded. By Liouville's Theorem the function $\lambda \rightarrow \mathrm{R}(\lambda, \mathrm{A})$ is constant, hence identically zero because $\lim _{\lambda \rightarrow \infty} R(\lambda, A)=0$. Thus we arrive at a contradiction.

We conclude this section by indicating possible extensions and further consequences of the results stated above.

Remarks 1.7.(a) Many of the results of this section remain true for positive semigroups on ordered Banach spaces more general than Banach lattices. The interested reader is referred to Greiner-Voigt-Wolff (1981) .\\
(b) From Thm.1.2 one can easily deduce that for positive semigroups on $L^{1}$-spaces, spectral bound and growth bound coincide. To prove the analoguous result for $\mathrm{L}^{2}$-spaces one makes use of Cor.1.3. For details we refer to C-IV,Thm.I.I.

\section*{2. THE BOUNDARY SPECTRUM}
In Chapter B-III we have seen that under suitable assumptions the boundary spectrum ${ }^{\sigma_{b}}(\mathrm{~A})$, which consists of all spectral values with maximal real part, is a cyclic set (cf. B-III, Def.2.5). In the main theorem of this section we prove a result which is more general and which is true for arbitrary Banach lattices.

We first want to extend some of the notions used in B-III to the more general setting considered here. If $E$ is a Banach lattice and $f, g \in E$ such that $g \in E_{|f|}$, then (sign f)g is well-defined (cf. Sec. 8 of C-I). Thus the following definition makes sense:

Definition 2.1. If $E$ is a Banach lattice then for $f \in E, n \in \mathbb{Z}$ we define $f^{[n]}$ recursively as follows:


\begin{align*}
& f^{[0]}:=|f| \\
& f^{[n]}:=(\operatorname{sign} f) f^{[n-1]}  \tag{2.1}\\
& f^{[n]}:=(\operatorname{sign} \bar{f}) f^{[n+1]} \\
& \text { if } \quad \\
& \text { if } \quad n<0
\end{align*}


Obviously, for $E=C_{0}(X)$ this amounts to the same as B-III, Def.2.2. Moreover, in case $E$ is an $L^{P}$-space, then $f^{[n]}$ is the function given by\\
(2.2) $\quad f^{[n]}(x)=\left\{\begin{array}{cc}(f(x) /|f(x)|)^{n-1} f(x) & \text { if } f(x) \neq 0 \\ 0 & \text { otherwise }\end{array}\right.$

The following properties are immediate consequences of Def.2.1 :\\
(2.3) $\mathrm{f}^{[0]}=|f|, f^{[1]}=\mathbf{f}, \mathrm{f}^{[-1]}=\overline{\mathbf{E}},\left|\mathrm{f}^{[n]}\right|=|f| \quad(n \in \mathbb{Z})$\\
(2.4) $f^{[n]}=(\operatorname{sign} f) f^{[n-1]}=(\operatorname{sign} \bar{f}) f^{[n+1]}$ for all $n \in \mathbb{Z}$;\\
(2.5) $(\alpha f)^{[n]}=\alpha(\alpha /|\alpha|)^{n-1}{ }_{f}^{[n]}$ for $n \in \mathbb{Z}, \alpha \in \mathbb{C}, \alpha \neq 0$.

Next we show that B-III, Thm. 2.4 is true for arbitrary Banach lattices. For defintion and simple properties of the signum operator $S_{h}$ see C-I, Sec. 8 .

Theorem 2.2. Let $\left(T(t){ }_{t \geqq 0}\right.$ be a positive semigroup on a Banach lattice $E$ with generator $A$ and suppose that for $h \in E$, $\alpha, \beta \in \mathbb{R}$ we have\\
(2.6) $\quad \mathrm{Ah}=(\alpha+\mathrm{i} \beta) \mathrm{h}, \mathrm{A}|\mathrm{h}|=\alpha|\mathrm{h}|$.

Then the following holds true:\\
(2.7) $\mathrm{Ah}^{[\mathrm{n}]}=(a+i n B) \mathrm{h}^{[n]}$ for all $\mathrm{n} \in \mathbb{Z}$.

In case $|\mathrm{h}|$ is a quasi-interior point of $\mathrm{E}_{+}$, then $S_{h} D(A)=D(A)$ and $A+i_{B}=S_{h}^{-1} A S_{h}$.

Proof. Without loss of generality we may assume that $\alpha=0$.\\
Then the assumption (2.6) implies that $\mathrm{T}(\mathrm{t})|\mathrm{h}|=|\mathrm{h}|$ and $T(t) h=e^{i \beta t_{h}}$ for $t \geqq 0$ (see A-III, Cor.6.4). In particular, the principal ideal $\mathrm{E}_{|\mathrm{h}|}$ is invariant under every operator $\mathrm{T}(\mathrm{t})$. By the Kakutani-Krein Theorem (C-I,Sec.4) we can identify E $|\mathrm{h}|$ with a space $C(K), K$ compact. Then the restrictions $\tilde{T}(t):=T(t) \mid E_{|h|}$ are positive operators on $C(K)$ satisfying $\tilde{T}(t)|\tilde{h}|=|\tilde{h}|$ and $\tilde{T}(t) \tilde{h}=e^{i \beta t} \tilde{h}$. From B-III,Thm.2.4(a) we conclude $\tilde{T}(t) \tilde{h}^{[n]}=e^{i \beta t_{h}^{[n]}}$ for all $t \geqq 0$, $n \in \mathbb{Z}$. Translating this back to $T(t)$ and $E$ this means precisely $T(t) h^{[n]}=e^{i n B_{h}[n]}(n \in \mathbb{F})$, hence $h^{[n]} \in D(A)$ and $A h^{[n]}=i n B h^{[n]}$. Moreover, by B-III,Thm.2.4(a) we have $e^{i \beta t} \boldsymbol{t}_{\tilde{T}}(t)=s_{\tilde{h}}^{-1} \tilde{T}(t) S_{\tilde{h}}$. If $|h|$ is a quasi-interior point this relation extends by continuity from the dense subspace $E_{|h|}$ to the whole space $E$, i.e., we have $e^{i \beta t} T(t)=s_{h}^{-1} T(t) s_{h}$ for $a l l$ t $t 0$.

In the proof above we could not apply assertion (b) of B-III, Thm.2.4 because the semigroup $(\tilde{T}(t))$ on $E_{|h|} \cong C(K)$ need not be strongly continuous with respect to the sup-norm.\\
As a first application of Thm. 2.2 we prove a cyclicity result for the point spectrum of contraction semigroups on a class of Banach lattices which includes the $\mathrm{L}^{\mathrm{P}}$-spaces.

Corollary 2.3. Suppose $E$ is a Banach lattice such that the norm is strictly monotone on $E_{+}$(i.e., $0 \leqq f<g \Rightarrow\|f\|<\|g\|$ ).\\
If (T(t)) is a positive contraction semigroup with $s(A)=0$, then $P_{b}(A)=\operatorname{P\sigma }(A) \quad n$ iR is imaginary additively cyclic.

Proof. Suppose that $A h=i \beta h(\beta \in \mathbb{R}, h \in E)$. Then we have $T(t) h=$ $e^{\overline{i B t} h}(t \geqq 0)$ and $|h|=|T(t) h| \leqq T(t)|h|$ since $T(t)$ is positive. Moreover, $\|h\| \leq\|T(t)|h|\| \leq\|h\|$ since $\|\mathrm{T}(t)\| \leq 1$. The assumption on the norm of $E$ implies that $T(t)|h|=|h|$ for all $t \geqq 0$, equivalently $\mathrm{A}|\mathrm{h}|=0$. Now we can apply Thm. 2.2 in order to obtain the desired result.

A more general result on cyclicity of the eigenvalues in the boundary spectrum will be proved in Sect. 4 (see Cor.4.3). In the remaining part of this section we focus our interest on the entire boundary spectrum. We will prove that it is cyclic provided that the resolvent $R(\lambda, A)$ does not grow too fast as $\lambda \downarrow s(A)$. We start with some preparations. An important tool in the proof are pseudo-resolvents.

Definition 2.4. Let D be an open (non-empty) subset of $\mathbb{C}$ and let G be a Banach space. A mapping $R=D \rightarrow L(G)$ which satisfies (2.8) $\quad R(\lambda)-R(\mu)=-(\lambda-\mu) R(\lambda) R(\mu) \quad(\lambda, \mu \in D)$ is called a pseudo-resolvent on G.

An equivalent (often quite useful) version of (2.8) is the following: (2.9) $\quad(1-(\lambda-\mu) R(\lambda))(1-(\mu-\lambda) R(\mu))=1 \quad(\lambda, \mu \in D)$

Obviously, the resolvent of a closed linear operator A on G is a pseudo-resolvent on $D=\rho(A)$. In general a pseudo-resolvent need not be the resolvent of an operator. Further information can be found in Hille-Phillips (1957), Pazy (1983) or Yosida (1974). For our purposes the following examples are of particular interest:

Example 2.5.(a) Suppose $A$ is a densely defined linear operator on $G$ with $\rho(A) \neq \varnothing$ and let $G_{F}$ be an F-product of $G$ (cf. A-I,3.6). Then the canonical extensions $R(\lambda, A)_{F}$ of $R(\lambda, A)$ form a pseudoresolvent $R_{F}$ on $G_{F}$ with $\rho(A)$ as domain of definition. If $A$ is unbounded, then $0 \in A \sigma(R(\lambda, A))$ hence $0 \in P \sigma\left(R_{F}(\lambda, A)\right)$ (cf. A-III, 4.5). It follows that $R_{F}$ is not the resolvent of an operator on $G$. (b) If $\{R(\lambda)\}_{\lambda \in D}$ is a pseudo-resolvent on $G$, then $\left\{R(\lambda)^{\prime}\right\}_{\lambda \in D}$ is a pseudorresolvent on $G^{\prime}$. Moreover, if $H$ is a closed linear subspace of $G$ which is $\{R(\lambda)\}_{\lambda \in D^{-i n v a r i a n t}}(R(\lambda) H C H$ for all $\lambda \in D)$, then the operators on $H$ and $G / H$ induced by $R(\lambda)$ in the canonical way form a pseudo-resolvent on H and $\mathrm{G} / \mathrm{H}$ respectively.

In the following we list several simple properties. We assume that $\mathrm{R}: \mathrm{D} \rightarrow$ (G) is a pseudo-resolvent on a Banach space G .\\
(2.10) Given $\lambda_{0} \in D, \mu \in \mathbb{C}$ there exists at most one operator $s \in L(G)$ such that\\
$R\left(\lambda_{0}\right)-S=-\left(\lambda_{0}-\mu\right) R\left(\lambda_{0}\right) S=-\left(\lambda_{0}-\mu\right) S R\left(\lambda_{0}\right)$.\\
In case such an operator exists we have $R(\lambda)-S=-(\lambda-\mu) R(\lambda) S=-(\lambda-\mu) S R(\lambda)$ for all $\lambda \in D$\\
(2.11) Given $\lambda_{0} \in D$ then for $\mu \in D$ with $\left|\mu-\lambda_{0}\right|<\left\|R\left(\lambda_{0}\right)\right\|^{-1}$ we have $R(\mu)=\sum_{n=0}^{\infty}\left(\lambda_{0}-\mu\right)^{n_{R}}\left(\lambda_{0}\right)^{n+1}$\\
(2.12) $\lambda \rightarrow R(\lambda)$ is a locally holomorphic function defined on $D \subseteq \mathbb{C}$ with values in L(G) .

We only sketch the proof of these assertions: (2.12) follows from (2.11) and the latter is a consequence of (2.10). The identity stated in (2.10) can be rewritten as follows:\\
$\left(1-\left(\lambda_{0}-\mu\right) R\left(\lambda_{0}\right)\right)\left(1-\left(\mu-\lambda_{0}\right) S\right)=1=\left(1-\left(\mu-\lambda_{0}\right) S\right)\left(1-\left(\lambda_{0}-\mu\right) R\left(\lambda_{0}\right)\right)$\\
Thus $S=\left(\mu-\lambda_{0}\right)^{-1}\left(1-\left(1-\left(\lambda_{0}-\mu\right) R\left(\lambda_{0}\right)\right)^{-1}\right)$ has to be unique.\\
It follows from (2.11) and (2.12) that every pseudo-resolvent has a unique maximal extension.\\
Further properties of pseudo-resovents are given in the following two propositions.

Proposition 2.6. Suppose $G$ is a Banach space, $D \subseteq \mathbb{C}$ and $\mathrm{R}: \mathrm{D} \rightarrow \mathrm{L}(\mathrm{G})$ is a pseudo-resolvent.\\
(a) Given $\alpha \in \mathbb{C}, x \in G$ one has $(\lambda-\alpha) R(\lambda) x=x$ either for all $\lambda \in D$ or for none.\\
(b) Suppose $\mu \in \bar{D} \backslash D$. Then $R$ can be extended to an open set containing $\mu$ if and only if there exists a sequence $\left(\lambda_{n}\right) \subset D$ converging to $\mu$ such that $\left\|R\left(\lambda_{n}\right)\right\|$ is bounded.

Proof. (a) Suppose that $(\lambda-\alpha) R(\lambda) x=x$ for some fixed $\lambda \in D, x \in G$. Then using (2.8) we have for $\mu \in D:(\mu-\lambda) R(\mu) x=(\lambda-\alpha)(\mu-\lambda) R(\mu) R(\lambda) x$ $=(\lambda-\alpha)(R(\lambda) x-R(\mu) x)=x-(\lambda-\alpha) R(\mu) x$. It follows that $(\mu-\alpha) R(\mu) x=x$ for all $\mu \in D$.\\
(b) If there exists an extension, then $\left\|R\left(\lambda_{n}\right)\right\|$ is bounded for every sequence $\left(\lambda_{n}\right)$ converging to $\mu$ by (2.12). On the other hand assuming that $\left\|R\left(\lambda_{n}\right)\right\|$ is bounded by $M$ for a fixed sequence $\left(\lambda_{n}\right) \subset D$ with $\lambda_{n} \rightarrow \mu(M \geqq 0)$, we have\\
$\left\|R\left(\lambda_{n}\right)-R\left(\lambda_{m}\right)\right\|=\left|\lambda_{n}-\lambda_{m}\right|\left\|R\left(\lambda_{n}\right) R\left(\lambda_{m}\right)\right\| \leq M^{2}\left|\lambda_{n}-\lambda_{m}\right|$\\
which shows that $\left(R\left(\lambda_{n}\right)\right)$ is a Cauchy sequence in $L(G)$, hence $s:=\lim _{n \rightarrow \infty} R\left(\lambda_{n}\right)$ exists. The assertion now follows from (2.10) and (2.11) .

In the next proposition we consider a positive pseudo-resolvent $R$ on a Banach lattice E ; i.e., we assume that the domain $D$ of $R$ contains the positive real axis and that $R(\mu)$ is a positive operator for every $\mu>0$. Applying Pringsheim's Theorem (see Thm.2.1 in the appen-\\
dix of Schaefer (1966) to the expansion given in (2.11) one can conclude that $R$ has an extension to the halfplane $\{z \in \mathbb{C}: \operatorname{Re} \mathbf{z}>0\}$. This shows that without loss of generality one can assume that the domain of a positive pseudo-resolvent contains the halfplane $\{z \in \mathcal{C}$ : $\operatorname{Re} z>0\}$.

Proposition 2.7. Suppose $R: \Delta \rightarrow L(E)$ is a positive pseudo-resolvent on a Banach lattice $E$ and $\Delta:=\{z \in \mathbb{C}: \operatorname{Re} z>0\}$.\\
If for some $\beta \in \mathbb{R}, h \in E$ we have\\
$\lambda \mathrm{R}(\lambda+i \beta) \mathrm{h}=\mathrm{h}$ and $\lambda_{\mathrm{R}}(\lambda)|\mathrm{h}|=|\mathrm{h}| \quad(\lambda \in \Delta)$, then\\
$\lambda_{R}(\lambda+i n B) h^{[n]}=h^{[n]}$ for all $n \in \mathbb{Z}, \lambda \in \Delta$.

Proof. At first we prove the following domination property which is the extension of (1.5) to pseudo-resolvents.\\
(2.13) $|R(\lambda) f| \leqq R(\operatorname{Re} \lambda)|f|$ for every $\lambda \in \Delta, \pounds \in E$.

To do this we fix $\lambda \in \Delta$. Then there exists $r_{0}>0$ such that $|r-\lambda|<r$ whenever $r>r_{0}$. Therefore $R(\lambda)=\sum_{n=0}^{\infty}(r-\lambda)^{n} R(r)^{n+1}$ for $r>r_{0}$, which implies for $f \in E$\\
$|R(\lambda) f| \leqq \sum_{n=0}^{\infty}|r-\lambda|^{n} R(r)^{n+1}|f|=\sum_{n=0}^{\infty}(r-(r-|r-\lambda|))^{n} R(r)^{n+1}|f|$\\
$=R(r-|\lambda-r|)|f|$. Since $\lim _{r \rightarrow \infty}(r-|\lambda-r|)=\operatorname{Re} \lambda$, we obtain (2.13). As a consequence of (2.13) and the assumption $r R(r)|h|=|h|$ we have that the principal ideal $\mathrm{E}_{|h|}$ is $\{\mathrm{R}(\lambda)\}{ }_{\lambda \in \Delta}$-invariant. Identifying, according to the Kakutani-Krein Theorem $E_{|h|}$ with a space $C(K), K$ compact, and by restricting the operators $R(\lambda)$ to $E_{|h|} \cong C(K)$ we obtain a positive pseudo-resolvent $\tilde{R}: \Delta \rightarrow L(C(K))$. Then we have for every $\alpha>0$ and $f \in E$ :\\
$\alpha \tilde{R}(\alpha+i \beta) h=h, \quad \alpha \tilde{R}(\alpha)|h|=|h|=I_{K}, \alpha|\tilde{R}(\alpha+i \beta) f| \leqq \alpha \tilde{R}(\alpha)|f|$. Applying B-III,Lemma 2.3 we obtain $\tilde{\mathrm{R}}(\alpha)=S_{\tilde{h}}{ }^{-1} \tilde{\mathrm{R}}(\alpha+i \beta) S_{\tilde{h}}$ for every $\alpha>0$ and using the uniqueness theorem for holomorphic functions we get $\tilde{R}(z)=S_{\tilde{h}}^{-1} \tilde{\mathrm{R}}(z+i \beta) S_{\tilde{h}}$ for every $z \in \Delta$. Iterating this identity we obtain:\\
(2.14) $\tilde{R}(z)=S_{\tilde{h}}^{-n} \tilde{R}(z+i n \beta) S_{\tilde{h}}^{n}$ for all $z \in \Delta, n \in \mathbb{Z}$

In particular, $s_{\tilde{h}}^{n}|h|=s_{\tilde{h}}^{-n} z \tilde{R}(z)|h|=z \tilde{R}(z+i n \beta) s_{\tilde{h}}^{n}|h|$. In terms of the initial space this means precisely $h^{[n]}=z R(z+i n \beta) h^{[n]}$, and the propositon is proved.

We will prove cyclicity of the boundary spectrum under a growth condition which is stated in the following definition.

Definition 2.8. Iet $A$ be the generator of a positive semigroup $(T(t))_{t \geq 0}$ with spectral bound $s(A)>-\infty$. The resolvent is said to grow slowly if one of the following (equivalent) conditions is satisfied:\\
(2.15a) $\{(\lambda-s(A)) R(\lambda, A): \lambda>s(A)\}$ is bounded in $L(E)$.\\
(2.15b) \{ $\left.\frac{1}{t} \int_{0}^{t} \exp (-\tau s(A)) T(\tau) d \tau: t>0\right\}$ is bounded in $L(E)$.

Before proving the equivalence of the two assertions we make some remarks.\\
(a) Since one always has $\lambda R(\lambda, A) \rightarrow$ Id for $\lambda \rightarrow \infty$ $\{(\lambda-s(A)) R(\lambda, A): \lambda>s(A)+\varepsilon\}$ is bounded for every $\varepsilon>0$. Thus in (2.15a) the essential fact is boundedness near s(A). On the other hand, $\left\{\frac{1}{t} \int_{0}^{t} \exp (-\tau s(A)) T(\tau) d \tau: 0 \leqq t \leqq T\right\}$ is bounded for every $\mathrm{I} \geqq 0$, hence in (2.15b) the boundedness for $t \rightarrow \infty$ is important.\\
(b) By Cor.1.4 we have $\|R(\lambda, A)\| \geqq r(R(\lambda, A))=(\lambda-s(A))^{-1}$. Hence $\|R(\lambda, A)\|$ grows at least as fast as $(\lambda-s(A))^{-1}$. Thus if (2.15a) is satisfied the resolvent grows as slowly as it possibly can for $\lambda \downarrow s(A)$.\\
(c) We do not assume in Def.2.8 that spectral bound and growth bound coincide. A slight modification of A-III, Example 1.3 shows that there are semigroups with slowly growing resolvent and $s(A)<\omega(A)$.

To prove equivalence of (2.15a) and (2.15b) we assume $s(A)=0$ and write $C(t):=\frac{1}{t} \int_{0}^{t} \mathrm{~T}(\tau) \mathrm{d} \tau$.\\
$(2.15 a) \rightarrow(2.15 b)$ : Consider $\lambda>0, t>0$ such that $\lambda t=1$. Then we have

$$
\lambda \cdot \exp (-\lambda s) \geqq\left\{\begin{array}{cl}
(e t)^{-1} & \text { if } s \leqq t \\
0 & \text { if } s>t
\end{array}\right.
$$

Now (1.1) implies $\lambda R(\lambda, A)=\int_{0}^{\infty} \lambda \exp (-\lambda s) \mathrm{T}(s) \mathrm{ds} \geq \mathrm{e}^{-1} \mathrm{C}(\mathrm{t})=\mathrm{e}^{-1} \cdot \mathrm{C}\left(\frac{1}{\lambda}\right)$ $\geq 0$. Thus $C(t)$ is bounded for $t \rightarrow \infty$ whenever $\lambda R(\lambda, A)$ is bounded for $\lambda \downarrow 0$.\\
$(2.15 b) \rightarrow(2.15 a):$ Let $M:=\sup \{\|C(t)\|: t>0\}$. Given $\mathbf{f} \in \mathrm{E}$, $\lambda>0, r>0$ then integration by parts yields :\\
$\lambda \int_{0}^{r} e^{-\lambda s} T(s) f d s=\lambda e^{-\lambda r} \int_{0}^{r} T(\sigma) f d \sigma+\lambda^{2} \int_{0}^{r} s e^{-\lambda s}\left(\frac{1}{s} \int_{0}^{s} T(\sigma) f\left(\bar{d}_{\sigma}\right) d s\right.$ It follows that $\left\|\lambda \int_{0}^{r} e^{-\lambda s} \mathrm{~T}(s) f \mathrm{ds}\right\| \leqq\left(r \lambda e^{-r}+\lambda^{2} \int_{0}^{r} s e^{-\lambda s} d s\right) M\|f\|$\\
$=\left(1-e^{-\lambda r}\right) m\|f\|$. Letting $r \rightarrow \infty$ we obtain by (1.1)\\
$\|\lambda R(\lambda, A) f\| \leqq M\|f\| \quad(f \in E, \lambda>0)$ hence $\|\lambda R(\lambda, A)\| \leqq M$

Two sufficient conditions for a resolvent to grow slowly are stated in the following proposition. Its simple proof is omitted.

Proposition 2.9. Suppose $\left(T(t){ }_{t \geq 0}\right.$ is a positive semigroup with generator A . Each of the following conditions guarantees that the resolvent grows slowly.\\
(a) (T(t)) $t \geqq 0$ is bounded and $s(A)=0$;\\
(b) s(A) is a first order pole of the resolvent.

In case $s(A)$ is a pole of order greater than 1 , the resolvent does not grow slowly. We will treat this case in Cor.2.12.

Theorem 2.10. The boundary spectrum of a positive semigroup with slowly growing resolvent is cyclic.

Proof. Without loss of generality we can assume that $s(A)=0$. Given $i \beta \in \sigma(A)(B \in i R)$, then $i B \in A \sigma(A) \quad(A-I I I, \operatorname{Prop} .2 .2)$ and $(\lambda-i B)^{-1} \in A \sigma(R(\lambda, A))(A-I I I, P r o p .2 .5)$. We consider an F-product $E_{F}$ of $E$ and for convenience write $E_{1}$ instead of $E_{F}$. The canonical extensions of $R(\lambda, A)$ to $E_{1}$ form a positive pseudo-resolvent $\left\{\left(R_{1}(\lambda)\right\}_{R e \lambda>0}\right.$ on $E_{1}$. By Prop.2.6(a) and A-III, 4.5 there exists $h_{1} \in \mathrm{E}_{1}, \mathrm{~h}_{1} \neq 0$ such that\\
(2.16) $\lambda \mathrm{R}_{1}(\lambda+i \beta) h_{1}=\mathrm{h}_{1}$ for $\operatorname{Re} \lambda>0$.

By (2.13) we have\\
(2.17) $\quad\left|h_{1}\right|=\left|r R_{1}(r+i \beta) h_{1}\right| \leqq r R_{1}(x)\left|h_{1}\right| \quad(x>0)$.

Next we choose any $\phi \in E_{1}^{\prime}$ such that $\left\langle h_{1}, \phi\right\rangle \neq 0$. since $\left\|R_{1}(\lambda)^{\prime}\right\|=$ $\left\|R_{1}(\lambda)\right\|=\|R(\lambda, A)\|$, the assumption of slow growth implies that $\left\{\lambda \mathrm{R}_{1}(\lambda)^{\prime}|\phi|: \lambda>0\right\}$ is bounded in $\mathrm{E}_{1}$ ', hence $\sigma\left(E_{1},{ }^{\prime} E_{1}\right)$-relatively compact by Alaoglu's Theorem. Thus there exist\\
$\phi_{1} \in \cap_{\varepsilon>0}\left\{r R_{1}(r)^{\prime}|\phi|: 0<r<\varepsilon\right\}-\sigma$.\\
Using the resolvent equation (2.8) we get for $r>0, \varepsilon>0$ :\\
$\left(1-r R_{1}(r)^{\prime}\right) \varepsilon R_{1}(\varepsilon)^{\prime}|\phi|=\varepsilon(r-\varepsilon)^{-1}\left(r R_{1}(r)^{\prime}|\phi|-\varepsilon R_{1}(\varepsilon)^{\prime}|\phi|\right)$.\\
Since the right hand side tends to 0 as $\varepsilon \rightarrow 0$, we have\\
$\left(1-r R_{1}(r){ }^{T}\right) \phi_{1}=0$ or\\
(2.18) $\quad \lambda \mathrm{R}_{1}(\lambda) \phi_{1}=\phi_{1} \quad(\operatorname{Re} \lambda>0)$.

Moreover, from $0<\left|\left\langle h_{1}, \phi\right\rangle\right| \leq\langle | h_{1}|,|\phi|\rangle \leq\left\langle r R_{1}(r)\right| h_{1}|,|\phi|\rangle=$ $<\left|h_{1}\right|, r R_{1}(r) \cdot|\phi|>$ it follows that


\begin{equation*}
\langle | h_{1}\left|, \phi_{1}\right\rangle>0 . \tag{2.19}
\end{equation*}


For arbitrary $f_{1} \in E_{1}$, Re $\lambda>0$ we have $<\left|R_{1}(\lambda) f_{1}\right|, \phi_{1}>\leqq$ $\left\langle R_{1}(\operatorname{Re} \lambda)\right| f_{1}\left|, \phi_{1}\right\rangle=\langle | E_{1}\left|, R_{1}(\operatorname{Re} \lambda)^{\prime} \phi_{1}\right\rangle=(\operatorname{Re} \lambda)^{-1}\langle | f_{1}\left|, \phi_{1}\right\rangle \cdot$\\
Therefore the ideal $I:=\left\{f_{1} \in E_{1}:\langle | f_{1}\left|, \phi_{1}\right\rangle=0\right\}$ is invariant under $\left\{\left(\mathrm{R}_{1}(\lambda)\right\}_{\text {Re } \lambda>0}\right.$. Furthermore we have (see (2.17), (2.18)), $\langle | r R_{1}(r)\left|h_{1}\right|-\left|h_{1}\right|\left|, \phi_{1}\right\rangle=\left\langle r R_{1}(r)\right| h_{1}\left|-\left|h_{1}\right|, \phi_{1}\right\rangle$

$$
\left.=\langle | h_{1}\left|, r R_{1}(r)^{\prime} \phi_{1}-\phi_{1}\right\rangle=0 \text { for } r\right\rangle 0
$$

which implies


\begin{equation*}
r R_{1}(r)\left|h_{1}\right|-\left|h_{1}\right| \in I \quad(r>0) . \tag{2.20}
\end{equation*}


Denoting by $E_{2}$ the quotient space $E_{I} / I$ and by $\left\{\left(R_{2}(\lambda)\right\}_{\text {Re }}>0\right.$ the pseudo-resolvent on $E_{2}$ induced by $\left\{\left(R_{1}(\lambda)\right\}_{R e \lambda>0}\right.$ in the canonical way, then $h_{2}:=h_{1}+I \neq 0$ (by (2.19)). Moreover, $\lambda R_{2}(\lambda+i \beta) h_{2}=h_{2}$ (by (2.16)) and $\lambda \mathrm{R}_{2}(\lambda)\left|\mathrm{h}_{2}\right|=\left|\mathrm{h}_{2}\right|$ (by (2.20) and Prop.2.6(a)). Now we apply Prop.2.7(b) and obtain


\begin{equation*}
\lambda R_{2}(\lambda+i n \beta) h_{2}[n]=h_{2}[n] \text { for } \operatorname{Re} \lambda>0, n \in \mathbb{Z} \text {. } \tag{2.21}
\end{equation*}


In particular, we have $\left\|R_{2}(r+i n \beta)\right\| \geqq \frac{1}{r}$, thus $\|R(r+i n \beta, A)\|=\left\|R_{1}(r+i n \beta)\right\| \geqq\left\|R_{2}(r+i n \beta)\right\| \geqq \frac{1}{r}$ for $r>0$.\\
This finally implies that in $\beta \in \sigma(\mathrm{A})$ for $\mathrm{n} \in \mathbb{Z}$.

To prove cyclicity of the boundary spectrum in case s(A) is a pole (of arbitrary order) one applies B-III, Lema 2.8 to reduce the problem to the case of first order poles. Actually, B-III,Lemma 2.8 is true for arbitrary Banach lattices and the proof given in chapter B-III works in the general case as well. For completeness we recall this result.

Proposition 2.11. Let $A$ be the generator of a positive semigroup $T$ on a Banach lattice $E$ and suppose that the spectral bound $s(A)$ is a pole of the resolvent of order $k$. Then there is a sequence


\begin{equation*}
I_{-1}:=\{0\} \subset I_{0} \varsubsetneqq I_{1} \varsubsetneqq \ldots I_{k}:=E \tag{2.22}
\end{equation*}


of T-invariant closed ideals with the following properties:\\
If $A_{n}$ is the generator of the semigroup induced by $T$ on the quotient $I_{n} / I_{n-1}$, then we have\\
(a) $s\left(A_{0}\right)<s(A)$;\\
(b) If $n \geq l$ then $s\left(A_{n}\right)=s(A)$ is a first order pole of the resolvent $\mathrm{R}\left(., \mathrm{A}_{n}\right.$ ) . The corresponding residue is a strictly positive operator.

Combining Thm.2.10 and Prop.2.11 one obtains the following generalization of B-III,Thm.2.9. In contrast with the former result we do not assume that every point of $\sigma_{b}(A)$ is a pole.

Corollary 2.12. If $A$ is the generator of a positive semigroup such that $s(A)$ is a pole of the resolvent then $\sigma_{b}(A)$ is cyclic.

Proof. Considering the sequence of ideals as described in Prop.2.11 and the corresponding generators $A_{n}(0 \leqq n \leq k)$, then we have by A-III, Prop.4.2 $\sigma_{b}(A)=U_{n=1}^{k} \sigma_{b}\left(A_{n}\right)$. By Thm. 2.10 each set $\sigma_{b}\left(A_{n}\right)$ is cyclic hence so is $\sigma_{b}(A)$.

The proof of the preceding corollary indicates a possible generalization of Thm.2.10. Instead of assuming that the resolvent grows slowly one merely needs that there exist a finite chain of closed T-invariant ideals as described in (2.22) such that the semigroups induced on the corresponding quotient spaces have slowly growing resolvents.

Knowing that $\sigma_{b}(A)$ is cyclic one has the alternative (cf. B-III, (2.19)):

Either $\sigma_{b}(A)=\{s(A)\}$ or else $\sigma_{b}(A)$ is an unbounded set.

Obviously one can use this fact to prove the existence of a dominant spectral value (cf. the explanation preceding B-III, Cor.2.11). We give a typical example.

Corollary 2.13. Let $A$ be the generator of a positive, eventually norm-continuous semigroup. Suppose either that the resolvent grows slowly or that $s(A)$ is a pole of the resolvent. Then $s(A)$ is a dominant spectral value.

Proof. The boundary spectrum $\sigma_{b}(A)$ is cyclic (Thm.2.10 and Cor.2.12 resp.) and compact $(A-I I, T h m .1 .20)$. Hence $\sigma_{b}(A)=\{s(A)\}$.

A situation in which Cor. 2.13 can be applied is described in the following example. For more details and proofs we refer to Amann (1983)

Example 2.14. Let $\Omega$ be a bounded domain in $\mathbb{R}^{n}$ of class $\mathrm{C}^{2}$. Assume that $\partial \Omega=\Gamma_{0} U \Gamma_{1}$ where $\Gamma_{0}$ and $\Gamma_{1}$ are disjoint open and\\
closed subsets of $\partial \Omega$. on $E=L^{p}(\Omega)(1 \leqq p<\infty)$ we consider a differential operator $L_{p, 0}$ which is defined as follows:\\
(2.23) $L_{p, 0}{ }^{f}:=\sum_{i, j=1}^{n} a_{i j} f_{i j}^{\prime}+\sum_{i=1}^{n} b_{i} f_{i}^{\prime}+c f$, with domain

$$
D\left(L_{p, 0}\right):=\left\{f \in C^{2}(\bar{\Omega}): f(x)=0 \text { for } x \in \Gamma_{0}\right. \text { and }
$$

$$
\left.\partial f / \partial v(x)+\gamma(x) f(x)=0 \text { for } x \in \Gamma_{1}\right\}
$$

Here $f_{i}^{\prime}$ stands for $\partial f / \partial x_{i}$, thus $f_{i j}^{\prime}=\partial^{2} f / \partial x_{i} \partial x_{j}$. We assume that $L_{p, 0}$ is elliptic and that all coefficients are real-valued satisfying $a_{i j} \in c^{2}(\bar{\Omega}), b_{i} \in c^{1}(\bar{\Omega}), \gamma \in c^{1}(\bar{\Omega}), c \in c^{1}(\bar{\Omega})$. Then $L_{p, 0}$ is closable and its closure $L_{p}$ is the generator of a holomorphic semigroup of positive operators. Moreover, the resolvent is compact. Thus Cor. 2.13 is applicable and one obtains that s(A) is strictly dominant provided that $\sigma(A) \neq \varnothing$. Using the results of section 3 one can show that $\sigma(A) \neq \varnothing$ and that $s(A)$ is an algebraically simple eigenvalue (see Thm.3.7 and prop.3.5).

We conclude with some remarks.

Remarks 2.15.(a) In the proof of Thm.2.10 we did not use the assumption that $R$ is the resolvent of a semigroup. In fact one can state this theorem for closed operators having positive resolvent. In this case one has to assume that $\{(\lambda-s(A)) R(\lambda, A): s(A)<\lambda<s(A)+1\}$ is bounded in $L(E)$.\\
One can go even further and consider positive pseudo-resolvents $\{R(\lambda)\}_{\lambda \in D}$. This is also possible with respect to Cor. 2.12 .\\
(b) If s(A) is a pole, then the criteria stated in B-III, Rem. 2.15 (a) for $s(A)$ to be a first order pole are valid in the setting of arbitrary Banach lattices as well. In particular, one has a first order pole provided that ker(s(A) - A) contains a quasi-interior point or in case that ker(s(A) - A') contains a strictly positive linear form.\\
(c) It is not difficult to give examples of semigroups whose resolvent does not grow slowly or cannot be reduced by a finite chain of invariant ideals as described after cor.2.12 . E.g., one can take a bounded positive operator $A$ which is not nilpotent and satifies $\sigma(\mathrm{A})=\{0\}$. However, the following question is still unanswered:\\
(2.23) Does every positive semigroup have cyclic boundary spectrum?

\section*{3. IRREDUCIBLE SEMIGROUPS}
The concept of irreducibility is very natural in the general setting of Banach lattices. However, some of the (equivalent) assertions stated in B-III, Def.3.1 do not ma e sense here, others need a slightly different formulation.

Definition 3.1. A positive semigroup $(T(t))_{t \geqslant 0}$ on a Banach lattice E with generator A is called irreducible if one of the following (mutually equivalent) conditions is satified:\\
(i) There is no (T(t))-invariant closed ideal\\
except $\{0\}$ and $E$.\\
(ii) Given $\pounds \in E, \phi \in E^{\prime}$ such that $\pounds>0, \phi>0$ then $\left\langle\mathrm{T}\left(t_{0}\right) f, \phi\right\rangle>0$ for some $t_{0} \geqq 0$.\\
(iii) For arbitrary $f, g \in E_{+}, f>0, g>0$ there exists $t_{0}$ such that $\inf \left\{T\left(t_{0}\right) f, g\right\}>0$.\\
(iv) For some (every) $\lambda>\mathrm{s}(\mathrm{A})$ there is no closed ideal other than $\{0\}$ or $E$ which is invariant under $\mathrm{R}(\lambda, \mathrm{A})$.\\
(v) For some (every) $\lambda>s(A)$ we have:\\
$\mathrm{R}(\lambda, \mathrm{A}) \mathrm{f}$ is a quasi-interior point of $\mathrm{E}_{+}$whenever $\mathrm{f}>0$.

Equivalence of the five conditions above is obtained by a slight modification of the arguments given in B-III, Def.3.1 . Since there are no difficulties we omit a detailed proof. Obviously, a semigroup is irreducible if one single operator $T\left(t_{0}\right)$ is irreducible. In general the converse does not hold (see p. 65 in Greiner (1982)). The situation is different when holomorphic semigroups are considered. Then an even stronger assertion holds: In fact irreducibility of a holomorhic semigroup implies that every single operator maps the positive elements onto quasi-interior points. This is the second statement of the following theorem (see also B-III, Rem.3.2).

Theorem 3.2.(a) If $(T(t))_{t \geq 0}$ is an irreducible semigroup then every operator $T(t)$ is strictly positive.\\
I.e., given $f>0, t \geqq 0$, then $T(t) f>0$.\\
(b) Suppose $\left(T(t){ }_{t \geq 0}\right.$ is a holomorphic positive semigroup.

If (T(t)) is irreducible then $T(t) f$ is a quasi-interior point of $\mathbf{E}_{+}$whenever $\pounds>0$ and $t>0$. Equivalently, given $\mathbf{f} \in \mathrm{E}, \Phi \in \mathrm{E}^{\prime}$ such that $f>0, \phi>0$, then $\langle T(t) f, \phi\rangle>0$ for all $t>0$.

Proof. (a) Given $t>0$ and $\mathrm{f}>0$ such that $T(t) f=0$ and $\lambda>s(A)$ then we have $T(t)(R(\lambda, A) f)=R(\lambda, A) T(t) f=0$. Since $R(\lambda, A) f$ is a quasi-interior point it follows that $T(t)=0$. Thus for fixed $t \in \mathbb{R}_{+}$we have either $T(t)$ is strictly positive or else $\mathbf{T}(t)=0$. Then strong continuity and $T(0)=I d \neq 0$ implies that there exists $\tau>0$ such that $T(t)$ is strictly positive for $0 \leqq t \leqq \tau$. For arbitrary $t \in \mathbb{R}_{+}$we find $n \in \mathbb{N}$ such that $\frac{t}{n} \leqq \tau$. Then $T(t)=T\left(\frac{t}{n}\right)^{n}$ is also strictly positive.\\
(b) We will prove that for an arbitrary holomorphic positive semigroup $(T(t))_{t \geqq 0}$ the following holds:

Given $f>0, \phi>0$ then either $\langle T(t) f, \phi\rangle=0$ for all $t \geqq 0$ or $\langle\mathrm{T}(t) f, \phi\rangle>0$ for all $t>0$.\\
Then it follows from Def.3.1(ii) that for irreducible semigroups always the second case occurs.\\
Assume that $\left\langle T\left(t_{0}\right) f, \phi\right\rangle=0$ for some $t_{0}>0$. We consider a null sequence $\left(t_{n}\right), 0<t_{n}<t_{0}$ such that $\left\|T\left(t_{n}\right) f-f\right\| \leq 2^{-n}$ and define $f_{n}:=T\left(t_{n}\right) f, g_{n}:=f-\sum_{k=n}^{\infty}\left(f-f_{k}\right)^{+}$. Then we have $g_{n} \leqq f, f=\lim _{n \rightarrow \infty} g_{n}$ and for $m \geqq n$ : $g_{n} \leqq f-\left(f-f_{m}\right)^{+}=\inf \left\{f, f_{m}\right\} \leqq f_{m}$.\\
For $n \in \mathbb{N}$ fixed and $m \geq n$ we obtain\\
$0 \leqq\left\langle T\left(t_{0}-t_{m}\right) g_{n}^{+}, \phi\right\rangle \leqq\left\langle T\left(t_{0}-t_{m}\right) f_{m}, \phi\right\rangle=\left\langle T\left(t_{0}\right) f, \phi\right\rangle=0$.\\
Thus the function $t \rightarrow\left\langle T(t) g_{n}{ }^{+}, \phi\right\rangle$ is identically zero by the uniqueness theorem for analytic functions. Since $f=\lim _{n \rightarrow \infty} g_{n}{ }^{+}$we have $\langle T(t) h, \phi\rangle=0$ for all $t \in \mathbb{R}_{+}$.

The next result can be used to check irreducibility for a semigroup whose generator is a bounded perturbation of a known semigroup. It is a generalization (and an extension to Banach lattices) of B-III, Prop. 3.3 .

Proposition 3.3. Suppose that $A$ is the generator of a positive semigroup $T$, further assume that $K$ is a bounded positive operator and $M$ is a bounded real multiplier (cf. C-I,Sec.8). Let $S$ be the semigroup generated by $B:=A+K+M$.\\
For a closed ideal I $\subset \mathrm{E}$ the following assertions are equivalent:\\
(i) I is S-invariant.\\
(ii) I is invariant both under $T$ and $K$.

Proof. We recall that a closed subspace $I \subset E$ is invariant for a semigroup generated by $C$ if and only if $C(D(C) \cap I) \subset I$ and the restriction $C_{\mid I}$ of $C$ with domain $D_{\mid}:=D(C) \cap I$ generates a semi-\\
group on I (see A-I, 3.2). Now let I be a closed ideal of E. (ii) $\rightarrow$ (i). If $I$ is T-invariant then $\mathrm{A} / \mathrm{I}$ generates a semigroup on $I$. The restrictions $K / I$ and $M / I$ of $K$ and $M$ respectively are bounded linear operators on I . Note that each closed ideal is invariant for $M$, cf. C-I,Sec.8.) . Thus $\left.{ }^{B}\right|_{I}=\mathrm{A}_{\mid I}+\mathrm{M}_{\mid I}+\mathrm{K}_{\mid I}$ with domain $D(A \mid I)=D(A) \cap I=D(B) \cap I$ is the generator of a semigroup on I . It follows that I is invariant for the semigroup generated by B .\\
(i) $\rightarrow$ (ii). Without loss of generality we assume $M \geqq 0$. Then we have $0 \leqq \mathrm{~T}(t) \leqq \mathrm{S}(\mathrm{t})$ for $a 11 \mathrm{t} \geqq 0$. It follows that I is T-invariant. Thus for $x \in D(A) \cap I=D(B) \cap I$ we have $K x=B x-A x-M x$. This shows that $\mathrm{K}(\mathrm{D}(\mathrm{B}) \cap \mathrm{I}) \subset \mathrm{I}$. Since $\mathrm{B}_{I}$ is a generator $\mathrm{D}(\mathrm{B}) \mathrm{OI}$ is dense in I . Then by continuity we have KI ᄃ I ; i.e., I is K-invariant.

Next we consider some concrete examples.

Examples 3.4. (a) Suppose that on $E=\mathrm{L}^{\mathrm{p}}(\mu)(1 \leqq \mathrm{p}<\infty)$ the semigroup ( $T(t)$ is given by\\
(3.1) $(T(t) f)(x)=\int_{X} k(t, x, y) f(y) d \mu(y) \quad(x \in X, t>0)$\\
for some measurable function $\mathrm{k}: \mathbb{R}_{+} \times \mathrm{X} \times \mathrm{X} \rightarrow \mathbb{R}_{+} \cdot$\\
Then (T(t)) is irreducible if and only if for any two measurable sets $M$ and $N$ with $0<\mu(M)<\infty, 0<\mu(N)<\infty, \mu(\mathrm{M} \cap \mathrm{N})=0$ there exist $t_{0}>0$ such that $\int_{M} f_{N} k\left(t_{0}, x, y\right) d \mu(x) d \mu(y)>0$\\
(b) Consider the first derivative on $\mathbb{R}, \mathbb{R}_{+}$or $\mathbb{R}_{2 \pi} \cong \Gamma$ as operator on the corresponding $I^{\text {P }}$-space (with respect to the Lebesgue measure.) Then the statements made in B-III, Ex.3.4(c) are true. The same is true for B-III,Ex.3.5(e) and (f) (second order differential operator) when the corresponding $\mathrm{L}^{\mathrm{P}}$-spaces are considered.\\
(c) Let $E=L^{1}[-1,0]$ and for $g \in L^{\infty}$ consider the operator $A_{g}$ given by\\
(3.2) $A_{g} f:=f^{\prime}, D\left(A_{g}\right):=\left\{f \in W^{1}[-1,0]: f(0)=\int_{-1}^{0} f(x) g(x) d x\right\}$

If $g \geqq 0$ then $A_{g}$ generates a positive semigroup. In case there exist $\varepsilon>0$ such that $g$ vanishes a.e. on $[-1,-1+\varepsilon]$, then $I:=\left\{f \in \mathrm{~L}^{1}: \mathrm{f}_{\{[-1+\varepsilon, 0]}=0\right\}$ is a non-trivial closed ideal which is invariant under the semigroup. It is not difficult to see that the condition on $g$ stated above is also necessary for ( $T(t)$ ) to be reducible (i.e., not irreducible.)\\
(d) Let $E=I^{1}([0,1] \times[-1,1])$ and consider the semigroup $(T(t))_{t \geqq 0}$ defined as follows:\\
(3.3) $(T(t) f)(x, v):=\left\{\begin{array}{cl}f(x-v t, v) & \text { for } 0 \leqq x-v t \leqq 1 \\ 0 & \text { otherwise }\end{array}\right.$\\
(T(t) ${ }_{t \geqq 0}$ is a positive semigroup on $E$ and\\
$D_{0}:=\left\{f \in C^{1}([0,1] \times[-1,1]): f(0, v)=E_{X}(0, v)=0\right.$ if $v \geqq 0$,

$$
\left.f(1, v)=f_{x}(1, v)=0 \quad \text { if } \quad v \leqq 0\right\}
$$

is a core for its generator A (cf. A-I, Cor.1.34). We have


\begin{equation*}
(A f)(x, v)=-v \cdot \frac{\partial f}{\partial x}(x, v) \quad\left(E \in D_{O}\right) . \tag{3.4}
\end{equation*}


The Laplace transform of (T(t)) is the resolvent of A . An explicit calculation yields:\\
(3.5) $\quad(R(\lambda, A) f)(x, v)=\int_{0}^{1} r_{\lambda}\left(x, x^{\prime}, v\right) f\left(x^{\prime}, v\right) d x^{\prime} \quad(\lambda>0)$ where $r_{\lambda}:[0,1] \times[0,1] \times[-1,1] \rightarrow \mathbb{R}$ is given by $r_{\lambda}\left(x, x^{\prime}, v\right)=\left\{\begin{array}{cl}|v|^{-1} \exp \left(-\lambda\left(x-x^{\prime}\right) v^{-1},\right. & \text { if either } v>0 \text { and } x^{\prime} \leqq x \\ 0 & \text { or } v<0 \text { and } x^{\prime} \geqq x ;\end{array}\right.$

Let $\sigma:[0,1] \times[-1,1] \rightarrow \mathbb{R}_{+}$and $k:[0,1] \times[-1,1] \times[-1,1] \rightarrow \mathbb{R}_{+}$be measurable functions and consider the operators $M$ and $K$ given by (3.6) Mf $:=\sigma f, \quad K f:=\int_{-1}^{1} k\left(., v^{\prime}\right) f\left(., v^{\prime}\right) d v^{\prime}$.

Then $B:=A-M+K$ with domain $D(B):=D(A)$ is the generator of a positive semigroup.\\
Using Prop. 3.3 we can prove the following irreducibility criterion for the semigroup $(S(t))_{t \geqslant 0}$ generated by $B$ :\\
(3.7) If $k$ is strictly positive then $(S(t))_{t \geqslant 0}$ is irreducible.

Actually, in view of Prop.3.3 we have to show that a closed ideal which is invariant under $\mathrm{R}(\lambda, \mathrm{A})$ and K has to be $\{0\}$ or E . We recall that closed ideals of $E$ are uniquely determined (up to sets of measure zero) by measurable subsets y of [0,1]×[-1,1]; i.e., every closed ideal has the form $I_{Y}=\{\mathbf{f} \in \mathrm{E}: \mathbf{f}$ vanishes (a.e.) on $[0,1] \times[-1,1] \backslash Y\}$.\\
Since we assumed that $k$ is strictly positive, $I_{Y}$ is K-invariant if and only if $Y=X \times[-1,1]$ for some measurable set $X \subset[0,1]$. If we assume that $x$ has positive measure and define $a:=\sup \{x \in[0,1]$ : $\left.\int_{0}^{x} 1_{X}(s) d s=0\right\}$ and $B:=\inf \left\{x \in[0,1]: \int_{X}^{1} 1_{X}(s) d s=0\right\}$ then we have $\alpha<\beta$ and the support of the function $h:=R(\lambda, A) I_{Y}$\\
(Y := X $\times[-1,1]$ is given by supp $h=[\alpha, 1] \times[0,1] \cup[0, \beta] \times[-1,0]$. Since we assumed that $I_{Y}$ is $R(\lambda, A)$-invariant we have $h \in I_{Y}$, i.e., supp $h \subset \mathrm{Y}=\mathrm{X} \times[-1,1]$. Obviously, this is true only if $\mathrm{Y}=[0,1] \times[-1,1]$ or $\mathrm{I}_{\mathrm{Y}}=\mathrm{E}$. A weaker condition than (3.7) entailing irreducibility is the following.\\
(3.8) There exists $\delta>0$ such that $k$ is strictly positive on the sets $[0, \delta] \times[-1,1]$ and $[1-\delta, 1] \times[-1,1]$.

For details we refer to Greiner (1984d).

In the following proposition we list some properties which are consequences of irreducibility. This extends B-III, Prop. 3.5 to the setting of Banach lattices. The first assertion of the latter proposition is no longer true in the general setting (see Ex.3.6 and Thm.3.7).

Proposition 3.5. Suppose A is the generator of an irreducible, positive semigroup on a Banach lattice $E$. Then the following assertions are true:\\
(a) Every positive eigenvector of A is a quasi-interior point.\\
(b) Every positive eigenvector of A' is strictly positive.\\
(c) If ker(s(A) - A') contains a positive element, then $\operatorname{dim}(\operatorname{ker}(s(A)-A)) \leqq 1$.\\
(d) If $s(A)$ is a pole of the resolvent, then it has algebraic (and geometric) multiplicity 1 . The corresponding residue has the form $\mathrm{P}=\phi \otimes \mathrm{u}$, where $\phi \in E^{\prime}$ is a positive eigenvector of $A^{\prime}, u \in E$ is a positive eigenvector of $A$ and $\langle u, \phi\rangle=1$.

Proof. To prove (a), (b) and (d) one can proceed as in the case $C_{0}(X)$ (see B-III, Prop.3.5). We only prove $(c)$ and assume $s(A)=0$. By assumption and by assertion (a) there exists $\phi \gg 0$ such that $T(t)^{\prime} \phi=\phi(t \geq 0)$. Given $\pounds \in \operatorname{ker} A$ then $T(t) \pounds=\pounds$ hence $|f|=$ $|T(t) f| \leqq T(t)|f|$. Since $\phi$ is strictly positive and $\langle | f|, \phi\rangle \leqq\langle T(t)| f|, \phi\rangle=\langle | f|, \phi\rangle$ it follows that $|f|=T(t)|f|$. We have shown that ker $A$ is a sublattice. Then for $\mathrm{f} \in$ ker A , f real, i,e., $\mathbf{f}=\overline{\mathbf{I}}$, we have that $\mathrm{I}^{+}$and $\mathrm{f}^{-}$are elements of ker $A$. Hence the principal ideals generated by $\mathrm{f}^{+}$and $\mathrm{f}^{-}$are $T_{\text {-inva- }}$ riant. Since these ideals are orthogonal the irreducibility of $T$ implies that either $\mathrm{f}^{+}$or $\mathrm{f}^{-}$is zero.

We have shown that ker $A \cap E_{\mathbb{R}}$ is totally ordered, hence at most onedimensional (see Prop.3.4 of Schaefer (1974)).

In arbitrary Banach lattices it is no longer true that an irreducible semigroup has necessarily nonvoid spectrum. We indicate how an irreducible semigroup having empty spectrum can be constructed.

Example 3.6. Consider the Banach lattice $E=\mathrm{L}^{\mathrm{P}}(\Gamma \times \Gamma)$. For (fixed) positive numbers $\alpha, \beta$ such that $\frac{\alpha}{\beta}$ is irrational we define a positive semigroup $\left(T_{0}(t)\right)_{t \geqslant 0}$ as follows:\\
(3.9) $\left(T_{0}(t) f\right)(z, w):=f\left(z \cdot e^{i \alpha t}, w \cdot e^{i \beta t}\right) \quad(z, w \in \Gamma=\{\xi \in \mathbb{C}:|\xi|=1\})$

Next we define for a positive function $m: \Gamma \times \Gamma \rightarrow \mathbb{R}$ which is continuous on $\Gamma \times \Gamma \backslash(1,1)$ functions $m_{t}, t \geqq 0$, as follows:\\
(3.10) $m_{t}(z, w):=\exp \left(-\int_{0}^{t} m\left(z \cdot e^{i \alpha s}, w \cdot e^{i \beta s}\right) d s\right)$

Then $(T(t))_{t \geq 0}$ defined by\\
(3.11) $T(t) f:=m_{t} \cdot\left(T_{o}(t) f\right)$\\
is a strongly continuous semigroup of positive contractions on E . Since $\frac{\alpha}{\beta}$ is irrational the semigroup $\left(T_{0}(t)\right)$ is irreducible. Moreover, each $m_{t}$ is strictly positive (i.e., $m_{t}>0$ a.e.) thus (T(t)) is irreducible as well. If one chooses $m$ such that $m(z, w)$ tends to $+\infty$ sufficiently fast as $(z, w) \rightarrow(1,1)$, one can get $\|\mathrm{T}(\mathrm{t})\|=\left\|\mathrm{m}_{t}\right\|_{\infty} \leqq \exp \left(-\mathrm{t}^{2}\right)$ for all $t \geqq 0$. Obviously such an estimate of $\|T(t)\|$ implies $\omega(A)=-\infty$, hence $\sigma(A)=\varnothing$.

In the following theorem we collect some hypotheses which in combination with irreducibility guarantee that $\sigma(A) \neq \varnothing$. For the sake of completeness we include B-III, Prop.3.5(a).

Theorem 3.7. Suppose that $(T(t))_{t \geq 0}$ is an irreducible, positive semigroup on the Banach lattice E. Each of the following conditions on $E$ and $(T(t))$, respectively, implies that $\sigma(A) \neq \varnothing$.\\
(a) $E=C_{O}(X)$ where $X$ is locally compact;\\
(b) $\quad \mathrm{E}=\ell^{\mathrm{p}}(1 \leq \mathrm{p}<\infty)$ (more generally, E contains atoms);\\
(c) either $T\left(t_{0}\right)$ is compact for some $t_{0}$ or $R\left(\lambda_{0}, A\right)$ is compact for some $\lambda_{0} \in \rho(A)$;\\
(d) E has order continuous norm and either $T\left(t_{0}\right)$ or $R\left(\lambda_{0}, A\right)$ is a kernel operator for some $t_{0} \geqq 0\left(\lambda_{0} \in \rho(A)\right)$. (For a precise definition of a kernel operator we refer to Sec.IV. 9 of Schaefer (1974) or Chap. 13 of Zaanen (1983)).\\
(e) E is reflexive and there exist $t_{0}>0, h \in E_{+}$such that $T\left(t_{0}\right) E \subset E_{h}$;

Proof. (a) is proved in B-III, Prop.3.5(a).\\
Assertion (b)-(f) will be proved utilizing the analoguous results for a single operator. In view of A-III, Prop. 2.5 we have to show that $r(R(\lambda, A))>0$ for some $\lambda \in \rho(A)$. Moreover, from $A-I,(3.1)$ we deduce\\
$T(t) R(\lambda, A)=e^{\lambda t} R(\lambda, A)-e^{\lambda t} \int_{0}^{t} e^{-\lambda s_{T}}(s) d s \leqq e^{\lambda t_{R}(\lambda, A)} \quad(t \geqq 0, \lambda>s(A))$. Since the spectral radius is an isotone function on the cone of positive operators, it is enough to show that\\
(3.12) $r(T(t) R(\lambda, A))>0$ for some $t \geqq 0, \lambda>s(A)$.

Using Thm.3.2(a) it is easy to see that $T(t) R(\lambda, A)=R(\lambda, A) T(t)$ is irreducible.

The assertions (b),(d) and (e) now follow from the corresponding results for a single operator as presented in sect.v. 6 of Schaefer (1974) (see Prop.6.1, Thm.6.5 Cor. and Thm.6.5 1.c.). (c) follows from the recent result of de Pagter (1986) which ensures that every positive operator on a Banach lattice which is compact and irreducible has positive spectral radius.

The theorem can be used to prove that elliptic operators as described in Ex.2.14 have non-empty spectrum. It is shown in Amann (1983) that these operators have compact resolvent and generate irreducible semigroups. Thus the assumption of (c) is satisfied.

Concerning the eigenvalues of an irreducible semigroup which are contained in $\sigma_{b}(A)$ all statements established for spaces $C_{0}(X)$ in B-III,Thm.3.6 are true in the setting of Banach lattices. The proof can be translated without difficulties and will be omitted (see also [Greiner (1982), Thm.2.6]).

Theorem 3.8. Suppose $T$ is an irreducible semigroup on the Banach lattice $E$ and let $A$ be its generator. Assume that $s(A)=0$ and that there exists a positive linear form $\psi \in D\left(A^{\prime}\right)$ with $A^{\prime} \psi \leq 0$.

If $\operatorname{Po}(\mathrm{A}) \cap \mathrm{I}$ is non-empty, then the following assertions are true:\\
(a) $\quad P o(A) \cap i R$ is a (additive) subgroup of iR .\\
(b) The eigenspaces corresponding to $\lambda \in P_{\sigma}(A) \cap i \mathbb{R}$ are onedimensional.\\
(c) If Ah $=$ iah $(h \neq 0, \alpha \in \mathbb{R})$ then $|h|$ is a quasi-interior point and the following holds:\\
(3.13) $\quad s_{h}(D(A))=D(A)$ and $s_{h}^{-1} \circ A \circ S_{h}=(A+i \alpha)$.\\
(d) 0 is the only ejgenvalue of $A$ admitting a positive eigenvector.

One can apply the theorem in order to prove that the rotation semigroup on $\Gamma$ (cf. $A-I, 2.5$ ) is the only positive periodic semigroup which is irreducible.

Corollary 3.9. Let $(T(t))_{t \geqq 0}$ be a positive irreducible semigroup on a Banach lattice $E$ which is periodic of period $\tau$. Assume that dim E > 1 . Then thexe exist continuous lattice homomorphisms $i: C(\Gamma) \rightarrow E$ and $j: E \rightarrow L^{1}(r)$, both injective with dense range, such that the diagramm commutes for all $t \geqq 0$. Moreover, joi is the canonical inclusion of $C(\Gamma)$ in $\mathrm{L}^{1}(\Gamma)$.\\
\includegraphics[max width=\textwidth, center]{2024_12_23_c6487cc0859199a15bd9g-323}

Proof. By Thm. 3.8 and A-III, Thm. 5.4 we have $\operatorname{Ro}(A)=\operatorname{Po}(\mathrm{A})=\sigma(\mathrm{A})=$ iaZ with $\alpha:=\frac{2 \pi}{n \tau}$ for suitable $n \in \mathbb{N}$. We fix h $\in$ ker(ia - A), $\mathrm{h} \neq 0$. Then $|\mathrm{h}| \in$ ker $A$ and there exists $\phi \in$ ker $A^{\prime}$ such that $\langle | h|, \phi\rangle=1$. According to the Kakutani-Krein Theorem we identify $E_{|h|}$ with $C(K)$. Then $h$ is a unimodular function onto $r$ (use the argument given in the proof of B-III,Thm.3.6(c)).\\
We define $i: C(\Gamma) \rightarrow E$ by $i(f):=\pounds \circ h \in C(K) \cong E_{h} \subset E$, then $i$ is injective. For the monomials $e_{n}(z):=z^{n}(n \in Z$ ) we have $i\left(e_{n}\right)=h^{[n]}$ thus $i$ has dense image in $E$ (by A-III, Thm.5.4). Moreover, $2 \pi \cdot \delta_{n 0}=\left\langle h^{[n]}, \phi\right\rangle=\left\langle i\left(e_{n}\right), \phi\right\rangle=\int_{0}^{2 \pi} e_{n}\left(e^{i t}\right) d t$ for all $n \in T$, hence $\int_{0}^{2 \pi} f\left(e^{i t}\right) d t=\langle i(f), \phi\rangle$ for all $f \in C(\Gamma)$. It follows that $(E, \phi) \cong L^{1}(\Gamma)$ and we define $j$ to be the canonical mapping from $E$ into $(E, \phi) \cong L^{1}(\Gamma)$ (see $C-I$, Sec.4). Then $j$ has dense image and is injective since $\phi$ is strictly positive (cf. Prop.3.5(b)). One easily verifjes that the diagramm commutes.

Now we are going to prove the main result of this section. As in the proof of Thm.2.10 we will utilize pseudo-resolvents on a suitable F-product of the Banach lattice. To simplify the proof we isolate two lemmas.

Lemma 3.10. Let $F$ be a filter on $\mathbb{N}$ which is finer than the Frechet filter and let $E_{F}$ be the F-product of the Banach lattice E. Given $R \in L(E)$ and denoting its canonical extension to $E_{F}$ by $R_{F}$ the following is true:\\
If $\alpha \in A \sigma(R) \backslash P \sigma(R)$ then ker( $\left.\alpha-R_{F}\right)$ is infinite dimensional.

Proof: Let ( $\left.f_{n}\right)_{n \geqq 1}$ be a normalized approximate eigenvector of $R$ corresponding to $\alpha$. Since every accumulation point of ( $f_{n}$ ) is an eigenvector of $R$, the assumption $\alpha \notin P \sigma(A)$ implies that ( $f_{n}$ ) does not have any accumulation points. Then there exist an $\varepsilon>0$ and a subsequence $\left(g_{n}\right)$ of ( $f_{n}$ ) such that\\
(3.14) $\quad\left\|g_{n}-g_{m}\right\| \geqq$ whenever $n \neq m$.\\
obviously, ( $g_{n}$ ) is a normalized approximate eigenvector of $R$ and so is every subsequence of $\left(g_{n}\right)$. In particular for $k \in \mathbb{N}$ the sequence $\left(g_{n+k}\right)_{n \geqq 1}$ is a normalized approximate eigenvector of $R$. Then the elements $\hat{g}^{k} \in E_{F}$ given by $\hat{g}^{k}:=\left(\left(g_{n+k}\right)_{n \geqq 1}+c_{F}(E)\right)$ are normalized eigenvectors of $R_{F}$ corresponding to $\alpha$. As a consequence of (3.14) we obtain\\
$\left\|\hat{g}^{k}-\hat{g}^{m}\right\|=F-l i m \sup \left\|g_{n+k}-g_{n+m}\right\| \geqq \varepsilon$ provided that $k \neq m$.\\
This shows that the unit ball in ker $\left(\alpha-R_{F}\right)$ is not relatively compact, hence ker $\left(\alpha-R_{F}\right)$ has to be infinite dimensional.

Lemma 3.11. Let $E$ be a Banach lattice and let $M$, L be two linear subspaces of $E$.\\
Assume that $\pounds \in M$ implies $|f| \in L$, then $\operatorname{dim} L \geqq \operatorname{dim} M$.

Proof. Let $\left\{g_{1}, g_{2}, \ldots, g_{m}\right\}(m \geqq 1)$ be any (finite) subset of $M$ which is linearly independent. For $u:=\sum_{n=1}^{m}\left|g_{n}\right|$ all vectors $g_{n}$ are contained in the principal ideal $\mathrm{E}_{\mathrm{u}}$ which (by the Kakutani-Krein Theorem) is isomorphic to a space $C(K)$. Considering $g_{1}, g_{2}$, . , $g_{m}$ as continuous functions on K , there exist points $\mathrm{x}_{1}, \mathrm{x}_{2}$, , , $x_{m} \in \mathrm{~K}$ and functions $h_{1}, h_{2}, \cdots, h_{m} \in \operatorname{span}\left\{g_{1}, g_{2}, \ldots, g_{m}\right\}$ such that $h_{i}\left(x_{j}\right)=\delta_{i j}$. Then $\left|h_{i}\right|\left(x_{j}\right)=\delta_{i j}$ hence $\left\{\left|h_{j}\right|: 1 \leqq j \leqq \mathrm{~m}\right\}$\\
is a subset of m linearly independent vectors which (by assumption) is contained in L .

The surprising fact in the following theorem is the conclusion that every point in the boundary spectrum is a simple algebraic pole if only $s(A)$ is supposed to be a pole.

Theorem 3.12. Let $T$ be an irreducible semigroup on a Banach lattice and let $A$ be its generator.\\
If s(A) is a pole of the resolvent then there exists $\alpha \geqq 0$ such that $\sigma_{b}(A)=s(A)+i a \mathbb{Z}$. Moreover, $\sigma_{b}(A)$ contains only algebraically simple poles.

Proof. We will assume that $s(A)=0$. Assuming first that every element of $\sigma_{b}(A)$ is an eigenvalue of $A$ one can conclude as follows: By Thm.3.8(a) we know that $\sigma_{b}(A)$ is an additive subgroup of iR . Since it is a closed subset and 0 is an isolated point it follows that $\sigma_{b}(A)=i \alpha Z$ for some $\alpha \geqq 0$. Moreover as a consequence of (3.13), for every $k \in \mathbf{Z}$ we obtain


\begin{equation*}
R(\lambda+i k \alpha, A)=s_{h}^{-k} \circ R(\lambda, A) \circ S_{h}^{k} \quad(\lambda \in \rho(A), k \in \mathbb{Z}) \tag{3.15}
\end{equation*}


By Prop.3.5(d) 0 is an algebraically simple pole. Then (3.15) implies that every point ika has the same property.\\
We now show that every element iB is an eigenvalue of A . By Prop.3.5(d) the residue of $R(., A)$ in $\lambda=0$ has the form $P=\phi \otimes u$ with $\phi(u)=1$. Given an ultrafilter $u$ on $N$ which is finer than the Frechet filter, then $\lim _{u} \phi\left(f_{n}\right)$ exists for every bounded sequence $\left(f_{n}\right.$ ) $\subset E$. Using this fact it is easy to see that the canonical extension $P_{U}$ of $P$ to the $U$-product $E_{U}$ of $E$ has the following form:\\
(3.16) $\quad P_{u}=\hat{\phi} \hat{u} \quad$ where $\hat{u}:=(u, u, u, \ldots)+c_{u}(E) \in E_{u}$ and $\hat{\phi} \in\left(E_{u}\right)^{\prime}$ is given by $\hat{\phi}\left(\left(f_{n}\right)+c_{u}(E)\right):=\lim _{u}{ }^{\left(f_{n}\right)} \quad\left(\left(f_{n}\right)+c_{u}(E) \in E_{u}\right)$.

Given i $\beta \in \sigma_{b}(A)$ then $i \beta \in A \sigma(A)$ hence $1 \in A \sigma(\lambda R(\lambda+i \beta, A)$. Assuming i $k$ Po $(A)$, then $1 \notin P_{\sigma}(\lambda R(\lambda+i \beta, A))$. Then Lemma 3.10 implies that $\mathrm{M}:=\operatorname{ker}\left(1-\lambda \mathrm{R}(\lambda+i \beta, A){ }_{U}\right.$ ) is infinite dimensional (and independent of $\lambda$ by Prop.2.6(a).) For $\hat{f} \in M$ we have $|\hat{f}|=\left|\gamma R(\gamma+i B, A) u^{\hat{f}}\right| \leqq \gamma R(\gamma, A) u^{\mid \hat{f}} \mid$ for every $\gamma>0$. It follows that $\phi(|\hat{f}|)=p_{U}|\hat{f}|=\lim _{\gamma \rightarrow 0} \gamma{ }^{\gamma}(\gamma, A){ }_{u}|\hat{f}| \geq|\hat{f}|$. Thus considering the closed ideal $I:=\left\{\hat{f} \in E_{U}: \hat{\phi}(|\hat{f}|)=0\right\}$ we have


\begin{equation*}
\hat{\phi}(|\hat{E}|)-|\hat{E}| \in I \text { for every } \hat{\tilde{E}} \in M . \tag{3.17}
\end{equation*}


This implies that $\mathrm{M} \cap \mathrm{I}=\{0\}$. Hence the canonical image M / of M in the quotient space $E_{U} / I$ is infinite dimensional as well. By (3.16) and (3.17) the absolute value of an element $\mathfrak{f} \in \mathrm{M}$ / is a scalar multiple of $\tilde{u}:=\hat{u}+I$. This is a contradiction by Lemma 3.11.

In view of A-III,Prop.4.2 the result above has consequences for semigroups which can be reduced (by considering restrictions to invariant ideals or quotients) to semigroups which satisfy the hypothesis of Thm.3.12. Semigroups having this property are precisely those for which seA) is a pole of the resolvent of finite algebraic multiplecity. The latter claim is a consequence of Prop.2.11 and the following lemma.

Lemma 3.13. Suppose that $T=(T(t))_{t \geq 0}$ is a positive semigroup such that $s(A)$ is a first order pole of the resolvent. Moreover assume that the corresponding residue is a strictly positive operator of finite rank.

Then there are closed $(T(t))$-invariant ideals $J_{1}, J_{2}, \ldots J_{m}$ which are mutually orthogonal such that the following is true: (We denote the restriction of $T$ to $J_{k}$ by $T_{k}$ and set $\mathrm{J}:=J_{1}{ }^{\oplus J_{2}}{ }^{\oplus} \ldots \oplus \mathrm{J}_{\mathrm{m}}{ }^{\prime}$\\
(a) $T_{k}$ is irreducible and has spectral bound $s(A)$;\\
(b) $s(A / J)<s(A)$.

Proof. We assume $s(A)=0$. Since $P$ is a strictly positive projectron PE = er A is a sublattice of E . Actually, if $\mathrm{x} \in \mathrm{PE}$ i.e., $P x=x$, then $P|x| \geqq|P x|=|x|$. Hence $p(|p| x|-|x||)=$ $P^{2}|x|-P|x|=0$ which implies that $P|x|-|x|=0$ or $|x| \in P E$. Thus we know that kex $A$ is a finite dimensional sublattice of $E$ hence it is isomorphic to a space $\mathbb{C}^{\mathbb{m}}$ for some $m \in \mathbb{N}$ (see Sec. II. 4 of Schaefer (1974)). Then there exist vectors $e_{j} \epsilon E_{+}(1 \leq j \leq m)$ such that the following holds:\\
(3.18) $\operatorname{ker} A=\operatorname{span}\left\{e_{1}, e_{2}, \ldots, e_{m}\right\} \quad$ and $\inf \left\{e_{i}, e_{j}\right\}=0$ for $i \neq j$.

We have $T(t) e_{k}=e_{k}$ hence the closed ideal generated by $e_{k}$ is T-invariant. We denote this ideal $J_{k}$ and define $J:=J_{1} \oplus \mathcal{J}_{2} \oplus \ldots \oplus \mathcal{J}_{\mathrm{m}} \cdot$ J is closed (see [Schaefer (1974), III.Thm.1.2]), T-invariant and we have $P E=$ ger $A \subset J$. Then $P_{/ J}=0$ hence the spectral bound $s(A / J)$\\
is strictly less than zero (by Thm.1.1(a)). Moreover, the residue corresponding to the resolvent of $T_{k}$, we denote it $P_{k}$, is the restriction of P to $J_{k}$.\\
$\mathrm{P}_{k}$ is strictly positive and $\left.\mathrm{P}\left(J_{k}\right)=\operatorname{span} \mathrm{e}_{\mathrm{k}}\right\}$. To show that $T_{k}$ is irreducible we consider an invariant ideal I. Then we have\\
$R\left(\lambda, A_{k}\right) I C I$ for $\lambda>0$ hence $P_{k}=\lim _{\lambda \rightarrow 0} \lambda R\left(\lambda, A_{k}\right)$ leaves $I$ invariant. If I $\neq\{0\}$ then $P_{k} I \neq\{0\}$ since $P_{k}$ is strictly positive. Then $e_{k} \in P_{k}{ }^{J} \subset \mathrm{I}$ which implies that $J_{k} \in \mathrm{I}$.

Combining the lemma with Prop.2.11 one obtains the following: If s(A) is a pole of finite algebraic multiplicity then there exists a finite chain of $T$-invariant ideals $I_{-1}:=\{0\} \subset I_{0} \subset \ldots \subset I_{N}:=E$ ( $N \in \mathbb{N}$ ) such that the following is true:\\
(3.19) For the semigroup $T_{n}$ on $I_{n} / I_{n-1}$ ( $0 \leq n \leq N$ ) which is induced by $T$ we have either $s\left(A_{n}\right)=s(A)$ and $T_{n}$ is irreducible or $s\left(A_{n}\right)<s(A)$.

The following theorem is an immediate consequence of (3.19), Thm.3.12 and A-III, Prop.4.2.

Theorem 3.14. Let $T$ be a positive semigroup on a Banach lattice with generator A . If $s(A)$ is a pole of finite algebraic multiplicity then $\sigma_{b}(A)$ is a finite union of discrete subgroups (i.e., $\sigma_{b}(A)=s(A)+U_{k=1}^{N} i \alpha_{k} \mathbb{Z}$ with $\left.\alpha_{k} \in \mathbb{R}\right)$. Moreover, $\sigma_{b}$ contains only poles of finite algebraic multiplicity.

Here the assumption that the multiplicity of s(A) is finite is essential as can be seen from the following example.

Example 3.15. Consider $\mathrm{X}:=[0,1] \times \mathrm{V}, \mathrm{V}:=\left\{\mathrm{v} \in \mathbb{R}: \mathrm{v}_{1}<|\mathrm{v}|<\mathrm{v}_{2}\right\}$ $\left(0<v_{1}<v_{2}<\infty\right)$. The flow in the phase space $x$ which describes the free motion in the interval $[0,1]$ with velocities in $V$ assuming that the particles are reflected at the endpoints generates a positive semigroup on $L^{p}(X, \mu)$ ( $\mu$ the Lebesgue measure). For the spectrum of the generator A one obtains $\sigma(A)=\left\{i \gamma: n_{\gamma} \leqq|\gamma| \leqq n \gamma_{2}\right.$ for some $\left.n \in \mathbb{N}_{0}\right\}$ with $\gamma_{1}:=\pi v_{1}^{-1}, \gamma_{2}:=\pi v_{2}^{-1}$. Moreover, 0 is a first order pole of the resolvent, obviously the only pole in $\sigma_{b}(A)=\sigma(A)$. These statements can be verified by calculating the resolvent explicitely. This can be done using the integral representation. The semi-\\
group is given as follows :\\
(3.20) $(T(t) f)(x, v)=\left\{\begin{array}{cc}f(x-v t+k, v) & \text { if } k-1 \leqq v t-x \leqq k \text { and } k \text { even; } \\ f(1-(x-v t+k),-v) & \text { if } k-1 \leqq v t-x \leqq k \text { and } k \text { odd. }\end{array}\right.$

Obviously one can apply Thm. 3.12 and Thm.3.14 respectively, in order to prove existence of strictly dominant eigenvalues. We consider two typical cases in the following corollaries. The meaning of $r_{e s s}(T(t))$ and $\omega_{\text {ess }}(T)$ is explained in $\mathrm{A}-I I I, 3.7$.

Corollary 3.16. Suppose that $T$ is a positive semigroup such that $\omega_{\text {ess }}(T)<w(T)$. Then $s(A)=\omega(T)$ is a strictly dominant eigenvalue. If in addition there exists an eigenfunction which is a quasi-interior point of $E_{+}$(e.g., if $T$ is irreducible) then $s(A)$ is a first order pole of R(.,A).

Proof. There exist $\varepsilon>0$ such that for every $t>0$ the set $\{\lambda \in \sigma(\mathbb{T}(t)):|\lambda| \geqq \exp ((s(A)-\varepsilon) t)\}$ contains only (finitely many) poles of R(.,T(t)) each being of finite algebraic multiplicity. In view of $A-I I I, \operatorname{Cor} .6 .5$ the set $\{\lambda \in \sigma(A): \operatorname{Re} \lambda>s(A)-\varepsilon\}$ is finite and contains only poles of R(.,A) . Thus we can apply Thm.3.14. It follows that $s(A)$ is strictly dominant. For the final assertion we refer to Rem.2.15(b).

Corollary 3.17. Suppose that $T$ is an irreducible, eventually norm continuous semigroup having compact resolvent.\\
Then $s(A)=\omega(T)$ is an algebraically simple pole and a strictly dominant eigenvalue.

Proof. By Thm.3.7(c) we know that $s(A)>-\infty$. Thm.3.12 is applicable since we assumed that $T$ is irreducible and has compact resolvent. Thus $s(A)$ is an algebraically simple pole and $\sigma_{b}(A)=s(A)+i \alpha Z$ for some $\alpha \geq 0$. In addition $\{\lambda \in \sigma(A): \operatorname{Re} \lambda \geq-1\}$ is compact since $T$ is eventually norm-continuous (see A-II, Thm.1.20). It follows that $s(A)$ is strictly dominant. By A-III, Thm. 6.6 we have $s(A)=\omega(T)$.

In the following proposition we give a condition which ensures that for certain perturbations Thm. 3.14 can be applied. Moreover, we state a criterion ensuring existence of a dominant eigenvalue.

Proposition 3.18. Suppose that A is the generator of a positive semigroup and that $K \in L(E)$ is a positive linear operator.\\
If $K$ is $A$-compact (i.e., if $K R\left(\lambda_{0}, A\right)$ is compact for some $\lambda_{0} \in \rho(A)$ ) and if $s(A+K)>s(A)$ then $B:=A+K$ satisfies the assumptions of Thm.3.14.

If, in addition, $K$ is irreducible then $s(B)$ is a dominant eigenvalue and the semigroup generated by $B$ is irreducible.

Proof. The resolvent equation $R(\lambda, A)=R\left(\lambda_{0}, A\right)\left(1-\left(\lambda-\lambda_{0}\right) R(\lambda, A)\right)$ implies that $K R(\lambda, A)$ is a compact operator for every $\lambda \in \rho(A)$. For $\lambda>s(A)$ we have $\lambda-B=(1-\operatorname{KR}(\lambda, A))(\lambda-A)$ and $(1-\operatorname{KR}(\lambda, A))^{-1}$ exists for $\lambda>s(B)$. Therefore Thm.XIII. 13 of Reed-simon (1979) implies that $\mathrm{R}(\lambda, B)=\mathrm{R}(\lambda, \mathrm{A})(1-\operatorname{KR}(\lambda, A))^{-1}$ has only poles of finite algebraic multiplicity in $\{\lambda \in \mathbb{C}: \operatorname{Re} \lambda>\mathrm{s}(\mathrm{A})\}$. This proves the first claim. In order to prove the second, we denote the semigroup corresponding to $A$ and $B$ by ( $T(t)$ ) and (S(t)) respectively. It follows from Prop.3.3 that $(S(t))$ is irreducible and we have $s(t)=T(t)+\int_{0}^{t} T(t-s) K S(s) d s \quad$ (see A-II, (1.9)). Iterating this identity we obtain for every $m \in \mathbb{N}, t \geqq 0$ :


\begin{align*}
s(t) & =\sum_{n=0}^{m-1} T_{n}(t)+R_{m}(t) \text { where }  \tag{3.21}\\
T_{0}(t) & :=T(t), T_{n}(t):=\int_{0}^{t} T(t-s) K T_{n-1}(s) d s \quad(n \in \mathbb{N}) \\
R_{m}(t) & :=\int_{0}^{t} \int_{0}^{t_{1}} \cdots \int_{0}^{t_{m-1}} T\left(t-t_{1}\right) K T\left(t_{1}-t_{2}\right) K \ldots K s\left(t_{m-1}-t_{m}\right) d t_{m} \ldots d t_{1}
\end{align*}


We fix $0<f \in E, 0<\phi \in E^{\prime}, t>0$. By Thm. $3.2(a), S(t) f>0$. Since $K$ is irreducible there exists $m \in \mathbb{N}$ such that $\left\langle K^{\mathrm{m}} \mathrm{S}(t) f, \phi\right\rangle>0$. Thus the integrand appearing in the the representation (3.21) of 〈R $\mathrm{R}_{m}(t) \mathrm{f}, \phi$ > is non-zero at $t_{1}=t_{2}=\ldots=t_{m-1}=t, t_{m}=t$. Since the integrand is positive and continuous we conclude


\begin{equation*}
\langle S(t) f, \phi\rangle \geqq\left\langle R_{m}(t) f, \phi\right\rangle>0 \text { for } 0<f, 0<\phi, t>0 \tag{3.22}
\end{equation*}


It follows that $\left(e^{-t s(B)} S(t)\right)_{t \geqq 0}$ cannot contain the rotation semigroup on $\Gamma$. On the other hand, assuming that $s(B)$ is not dominant, then $\operatorname{dim}(\operatorname{ker}(\exp (\tau \cdot s(B))-S(\tau)))>1$ for some $\tau>0$. Hence the restriction $\left(e^{-t s(B)} S(t) \mid F\right)$ t where $F:=\operatorname{ker}(\exp (\tau \cdot s(B))-S(\tau))$, contains the rotation semigroup by cor.3.9.

We conclude this section considering once again Example 3.4(d). The generator considered there is $B=(A-M)+K$, where $K$ is\\
positive linear. From (3.5) and (3.6) one deduces that\\
$(\operatorname{KR}(\lambda, A) f)(x, v)=\int_{0}^{t} \int_{0}^{t} k\left(x, v, x^{\prime}, v^{\prime}\right) f\left(x^{\prime}, v^{\prime}\right) d x^{\prime} d v^{\prime}$\\
where the kernel $k$ is given by $k\left(x, v^{\prime} x^{\prime}, v^{\prime}\right):=k\left(x, v^{\prime} v^{\prime}\right) r_{\lambda}\left(x, x^{\prime}, v^{\prime}\right)$ (cf. (3.5), (3.7)). Using this representation of KR( $\lambda, \mathrm{A})$ it follows that $K$ is A-compact. Moreover for $\lambda$ sufficiently large one has $\mathrm{R}(\lambda, \mathrm{A}-\mathrm{M})=\mathrm{R}(\lambda, \mathrm{A})(1-\mathrm{MR}(\lambda, A))^{-1}$ which shows that K is also (A-M)-compact. In order to apply Thm. 3.14 one needs $s(B)>s(A-M)$ (see Prop.3.17) which is difficult to verify. In case the function $\sigma$ is continuous one can state a sufficient condition as follows: There exist $r \in \mathbb{R}$ and $g \in L^{1}([0,1] \times[-1,1]), g>0$ such that $r<\inf \{\sigma(x, 0): x \in[0,1]\}$ and $B g \geqq-r g$.\\
The additional assumption made in the second part of Prop.3.17 is not satisfied in this example. Nevertheless one can show that s(B) is strictly dominant in this situation (provided that s(B) > s (A)). For details we refer to Greiner (1984d) or Voigt (1985) where the linear transport equation in higher dimensional spaces is discussed.

\section*{4. SEMIGROUPS OF LATTICE HOMOMORPHISMS}
In Section 2 we proved that the boundary spectrum of certain positive semigroups is a cyclic set. For semigroups of lattice homomorphisms much more can be said: The whole spectrum is an imaginary additively cyclic subset of $\mathbb{C}$ (cf. Thm. 4.2). This result can be used to derive cyclicity results for the eigenvalues in the boundary spectrum of positive semigroups (cf. Cor.4.3). In the last part of this section we discuss a spectral decomposition of positive groups (cf. Thm.4.10).

Lemma 4.1. Suppose that $(T(t))_{t \geq 0}$ is a semigroup of lattice homorphisms on a Banach lattice E with generator A.\\
In case ia $\in \operatorname{Ro}(A), \alpha \in \mathbb{R}$, then one of the following assertions is true:\\
(a) $i \alpha \mathbb{Z} \subset \operatorname{Ro}(A) ;$\\
(b) $\{\lambda \in \mathbb{C}: \operatorname{Re} \lambda<0\} \subset \operatorname{Ro}(A)$.

Proof. There exists $\phi \in E^{\prime}, \phi \neq 0$ such that $T(t)^{\prime} \phi=e^{i \alpha t_{\phi}}$ ( $t \geqq 0$ ). Then we have $|\phi|=\left|T(t)^{\prime} \phi\right| \leqq T(t)^{\prime}|\phi|(t \geqq 0)$. If we fix $r>\omega(T)$ and define $\psi:=r R(r, A)^{\prime}|\phi|$, we have\\
(4.1) $T(t)^{\prime} \psi \leqq e^{x t} \psi, T(t)^{\prime} \psi \geqq \psi(t \geq 0)$ and $|\phi| \leqq \psi$.

In fact, $A-I,(3.1)$ implies $\left(e^{r t}-T(x)\right) R(x, A) \geq 0$ hence $T(t)^{\prime} \psi=r R(r, A)^{\prime} T(t)^{\prime}|\phi| \leqq r \cdot e^{r t} R(r, A)^{\prime}|\phi|=e^{r t} \psi$. Moreover, the inequality $T(t) '|\phi| \geqq|\phi|(t \geqq 0)$ implies $T(t)^{\prime} \phi=r R(r, A)^{\prime} T(t)^{\prime}|\phi| \geqq r R(r, A)^{\prime}|\phi|=\psi \quad$ and $\psi=r R(r, A) \cdot|\phi|=r \int_{0}^{\infty} e^{-r t} T(t) '|\phi| d t \geqq r \int_{0}^{\infty} e^{-r t}|\phi| d t=|\phi|$.

Considering the AI-space (E,廿) (see C-I,Sec.4) the first inequality of (4.1) implies that $(T(t))$ induces a strongly continuous semigroup $\left(\mathrm{T}_{1}(t)\right)_{t \geqq 0}$ on $(E, \psi)$.\\
That is we have\\
(4.2) $\quad \mathrm{T}_{1}(t) \circ \mathrm{q}_{\psi}=\mathrm{q}_{\psi} \circ \mathrm{T}(t) \quad(t \geqq 0)$

Denoting by $A_{1}$ the generator of $\left(\mathbb{T}_{1}(t)\right)$ we have $\operatorname{R\sigma }\left(A_{1}\right) \subset \operatorname{R\sigma }(A)$.\\
\includegraphics[max width=\textwidth, center]{2024_12_23_c6487cc0859199a15bd9g-331}

Indeed, ${ }^{A_{1} *} x=\lambda x$ implies $T_{1}(t)^{\prime} x=e^{\lambda t} x$ hence by (4.2) $T(t)^{\prime} q_{\psi}^{\prime}(x)^{\prime}=e^{\lambda t} q_{\psi}^{\prime}(x)$ or equivalently $A^{*}\left(q_{\psi}^{\prime}(x)\right)=q_{\psi}^{\prime}(x)$. Thus it remains to show that either $i a \mathbb{Z}$ or $\{\lambda \in \mathbb{C}: \operatorname{Re} \lambda<0\}$ is contained in $\mathrm{Ro}_{0}\left(\mathrm{~A}_{1}\right)$. Obviously, $\left(\mathrm{T}_{1}(t)\right)$ is a semigroup of lattice homomorphisms as well. The second inequality of (4.1) implies\\
(4.3) $\quad\left\|\mathrm{T}_{1}(t) f\right\|_{\psi}=\langle | \mathrm{T}_{1}(t) \mathrm{f}|, \psi\rangle=\langle | \mathrm{f}\left|, \mathrm{T}_{1}(t){ }^{\prime} \psi\right\rangle \geqq\langle | \mathrm{f}|, \psi\rangle=\|f\|_{\psi} \cdot$

Then for $\lambda \in \mathbb{C}$ with $\operatorname{Re} \lambda<0$ we have\\
$\left\|\left(e^{\lambda t}-T_{1}(t)\right) f\right\|_{\psi} \geqq\left\|T_{1}(t) f\right\|_{\psi}-\left\|e^{\lambda t} f\right\|_{\psi} \gtrsim\left(1-\left|e^{\lambda t}\right|\right)\|f\|_{\psi} \quad(f \in(E, \psi))$\\
and we obtain for the corresponding generator\\
(4.4) $\left\|\left(\lambda-A_{1}\right) f\right\|_{\psi}=1 i m_{t \rightarrow 0}\left\|\frac{1}{t}\left(e^{-\lambda t_{T_{1}}}(t) f-f\right)\right\|_{\psi} \geqq 1 i m_{t \rightarrow 0} \frac{1}{t}\left(e^{-t \operatorname{Re} \lambda}-1\right)\|f\|_{\psi}$ $=-\operatorname{Re} \lambda \cdot\|f\|_{\psi} \quad$ for $\operatorname{Re} \lambda<0$ and $f \in(E, \psi)$.

It follows from (4.3) and (4.4) that $A \sigma\left(T_{1}(t)\right) \cap\{z \in \mathbb{C}:|z|<1\}=\emptyset$ and $\operatorname{Ao}\left(A_{1}\right) \cap\{\lambda \in \mathbb{C}: \operatorname{Re} \lambda<0\}=\varnothing$. Since the toplogical boundary of the spectrum is always contained in the approximate point spectrum (see $A-I I I, P r o p .2 .2)$ and $R \sigma(T(t)) \backslash\}=\exp (t R \sigma(A))$ (see A-III, Thm.6.3), precisely one of the following two cases occurs :\\
(A) $\{\lambda \in \mathbb{C}: \operatorname{Re} \lambda<0\} \in \rho\left(A_{1}\right)$ and $\{z \in \mathbb{C}:|z|<1\} \subset \rho\left(T_{1}(t)\right)$;\\
(B) $\{\lambda \in \mathbb{C}: \operatorname{Re} \lambda<0\} \subset \operatorname{Ro}\left(A_{1}\right)$ and $\{z \in \mathbb{C}:|z|<1\} \subset \operatorname{Ro}\left(T_{1}(t)\right)$.

We mentioned above that $R_{\sigma}\left(A_{1}\right) \subset R_{\sigma}(A)$. Thus we only have to analyze case (A). In this case each operator $T_{1}(t)$ is an invertible lattice homomorphism hence a lattice isomorphism. It follows that $T_{1}$ ( $t$ )' is a lattice isomorphism as well. The third inequality in (4.1) implies that $\phi$ can be considered as an element of $(E, \psi)^{\prime}$ and $T(t)^{\prime} \phi=$ $e^{i_{\alpha} t_{\phi}}(t \geqq 0)$ implies $T_{1}(t){ }_{\phi}=e^{i_{\alpha} t}{ }_{\phi}$. Furthermore, we have\\
$T_{1}(t)^{\prime}|\phi|=\left|I_{1}(t)^{\prime} \phi\right|=\left|e^{i \alpha t} \phi\right|$ or equivalently $A_{1} *|\phi|=0$. Now we can apply Thm. 2.2 and obtain $i \alpha Z \subset P \sigma\left(A_{1} *\right)=R \sigma\left(A_{1}\right)$.

Theorem 4.2. Let $A$ be the generator of a semigroup (T( $t)_{t \geq 0}$ of lattice homomorphisms on a Banach lattice E. Then $\sigma(A), A \sigma(A)$ and Po(A) are imaginary additively cyclic subsets of $\mathbf{C}$.

Proof. We first consider the point spectrum. If $\lambda \in \operatorname{PG}(\mathrm{A})$, $\lambda=\alpha+i \beta(\alpha, \beta \in \mathbb{R})$, then there exists $f \in E, f \neq 0$ such that Af $=\lambda f$. It follows that $T(t) f=e^{\lambda t} f(t \geqq 0)$ hence $T(t)|f|=$ $|T(t) f|=e^{\alpha t}|f|(t \geq 0)$, or equivalently, $A|f|=\alpha|f|$. Now Thm. 2.2 is applicable and we obtain $A\left(f^{[n]}\right)=(\alpha+i n B) f^{[n]}$ for all $n \in \mathbb{Z}$. To prove the assertion for $A \sigma(A)$ we consider an F-product semigroup in order to reduce the problem to the point spectrum. We use the notation of $\mathrm{A}-\mathrm{I}, 3.6$. Obviously the space m(E) is a Banach lattice and every operator $\hat{T}(t)$ is a lattice homomorphism. We have $|T(t)| f|-|E||=||T(t) f|-|f|| \leqq|T(t) f-f| \quad(f \in E$ ), hence $\left(\left|f_{n}\right|\right) \in m^{\top}(E)$ whenever $\left(f_{n}\right) \in m^{\top}(E)$. This proves that $m^{\top}(E)$ is a sublattice, hence a Banach lattice as well. Obviously, $c_{F}(E) \cap m^{\top}$ (E) is an order ideal. Thus $E_{F}^{T}$ is a Banach lattice and $\left(T_{F}(t)\right)$ is a semigroup of lattice homomorphisms. It follows that Po( $A_{F}$ ) is cyclic hence $A \sigma(A)$ is cyclic by $A-I I I, 4.5$.

Cyclicity of the entire spectrum now follows from the cyclicity of Ao (A) and Lemma 4.1.

One can use Thm.4.2 in order to prove cyclicity for the eigenvalues in the boundary spectrum of positive semigroups. We list some typical cases in the following corollary.

Corollary 4.3. Let $T=(T(t))_{t \geq 0}$ be a positive semigroup on a Banach lattice $E$ which is bounded. Each of the following conditions implies that $P o(A) \cap i R$ is imaginary additively cyclic.\\
(a) E is weakly sequentially complete (e.g. $\mathrm{E}=\mathrm{L}^{\mathrm{P}}(\mu), 1 \leqq \mathrm{p}<\infty$ );\\
(b) Every operator $T(t)$ is mean ergodic (i.e. the Césaro means $\frac{1}{n} \sum_{k=0}^{n-1} T(t)^{k}$ converge strongly as $\left.n \rightarrow \infty\right)$;\\
(c) There is a strictly positive Iinear form which is T-invariant.

Proof. We will show that each of the conditions (a), (b) , (c) implies that ker(1 - $\mathbb{T}(\mathrm{s})$ ) is a Banach lattice (not necessarily a sublattice of $E$ f for every $s \geqq 0$. Then one argues as follows: Given $i \alpha \in \operatorname{Po}(A), \alpha \in \mathbb{R}$ then $T(t) g=e^{i \alpha t} g$ for suitable $g \neq 0$. For $\tau:=2 \pi|\alpha|^{-1}$ we have $g \in F:=\operatorname{ker}(1-T(\tau))$. Then the restriction ( $\mathrm{T}(\mathrm{t}) \mathrm{F}^{\prime} t \geqslant 0$ is a $\tau$-periodic positive semigroup on $F$. Since $\left.T(t)\right|^{-1}=T(n t-t) \mid \geqq 0$ it follows that $(T(t) \mid$ is a semigroup of\\
lattice isomorphisms. Since $g \in F$ we have ia $\in P_{\sigma}(A)$ hence $i_{\alpha} \mathbb{Z} \in P_{\sigma}(\mathrm{A},) \subset \operatorname{Po}(\mathrm{A})$ by Thm. 4.2 .\\
Now we show that ker(1 - T(s)) is a vector lattice for the induced order and a Banach lattice for an equivalent norm.\\
In case (c), ker(1 - T(s)) is even a sublattice of E . Indeed, assume $T(t)^{\prime} \phi=\phi$ and $\phi>0(t \geqq 0)$ then $T(s) f=f$ implies\\
$T(s)|f| \geqq|f|$. Thus from $\langle T(s)| f|-|f|, \phi\rangle=\langle | f\left|, T(s)^{\prime} \phi-\phi\right\rangle=0$ it follows that $T(s)|f|=|f|$.\\
Now we assume that $E$ is weakly sequentially complete, which is equivalent to (cf. Sec. 5 of $\mathrm{C}-\mathrm{I}$ ):\\
(4.5) Every increasing norm-bounded net of $\mathrm{E}_{+}$converges.

We fix $s>0$ and define $F:=\operatorname{ker}(1-\mathrm{T}(\mathrm{s})$ ), $\mathrm{T}:=\mathrm{T}(\mathrm{s})$. Obviously $\mathrm{f} \in \mathrm{F}$ implies $\overline{\mathrm{f}} \in \mathrm{F}$ hence $F=F \cap \mathrm{E}_{\mathbb{R}}+i \mathrm{~F}_{\cap} \mathrm{E}_{\mathbb{R}}$. Thus we have to show that $F_{\mathbb{R}}=F \cap E_{\mathbb{R}}$ is a sublattice. Given $f \in F_{\mathbb{R}}$ then $\mathrm{Tf}=\mathrm{f}$ hence $|\mathrm{f}| \leqq \mathrm{T} \mid \mathbf{f |}$. Iterating this inequality we obtain $|f| \leqq T|f| \leqq T^{2}|f| \leqq T^{3}|f| \leqq \ldots$. By (4.5) $|f|_{0}:=\lim _{n \rightarrow \infty} T^{n}|f|$ exists and we have $T|f|_{0}=\lim _{n \rightarrow \infty} \mathrm{~T}^{n+1}|f|=|f|_{0}$, i.e. $|f|_{0} \in F_{\mathbb{R}}$. For $g \in F_{\mathbb{R}}$ satisfying $\pm f \leqq g$ we have $|f|_{0} \leqq g$ thus $|f|_{0}=\sup _{F}\{f,-f\}$. Moreover, $\|f\|_{O}:=\left\||f|_{O}\right\|$ ( $f \in F$ ) is an equivalent norm on $F$ such that ( $F,\|\cdot\|_{0}$ ) is a Banach lattice.\\
(b) If $T(s)$ is mean-ergodic then we have ker(1 - T(s)) $=P E$ where $P$ is the mean-ergodic projection, i.e. Pf $=\lim _{n \rightarrow \infty} \sum_{k=0}^{\mathrm{n}-1} \mathrm{~T}(\mathrm{~s}) \mathrm{k}_{\mathrm{f}}$. Obviously $P$ is positive, hence II.11.5 of Schaefer (1974) implies that PE is a Banach lattice (for the induced order and an equivalent norm).

The assumptions made in Cor. 4.3 can be weakened slightly (cf. Greiner (1982)). However, one cannot prove cyclicity of $\mathrm{P}_{\sigma_{b}}(\mathrm{~A})$ for arbitrary positive semigroups.

Example 4.4. At first we recall Ex. 2.13 of Chapter B-III. There we constructed a bounded semigroup on the space $C(\Gamma) \times C_{0}(\mathbb{R})$ such that $P \sigma_{b}(A)=\{i k: k \in \mathbf{Z}, k \neq 0\}$.

Let us perform the same construction on the Hilbert space\\
$\mathrm{H}:=\mathrm{L}^{2}(\mathrm{~T}) \times \mathrm{L}^{2}(\mathbb{R})$. For a fixed positive, non-zero function $k \in C_{C}(\mathbb{R})$ we define $\mathrm{T}(\mathrm{t})$ on H as follows:\\
(4.6) $f_{t}(z):=f\left(z \cdot e^{i t}\right) \quad(z \in \Gamma)$ and

$$
g_{t}(x):=g(x+t)+\frac{1}{2 \pi} \cdot \int_{0}^{2 \pi} f\left(z \cdot e^{i s}\right) d s \cdot \int_{x}^{x+t} k(u) d u .
$$

Then $(T(t))_{t \geqq 0}$ is a positve semigroup on $H$ and for the spectrum of the generator we obtain $\sigma(A)=i R, \operatorname{Po}(A)=i \mathbb{Z} \backslash\{0\}$. In view of Cor.4.3(a) the semigroup cannot be bounded. (The explicit representation $(4.6)$ only allows the estimate $\left.\|\mathrm{T}(\mathrm{t})\| \leqq \sqrt{2}+t \cdot\|k\|_{2}(t \geqq 0).\right)$

In the next proposition we show that for semigroups of lattice homomorphisms on $\mathrm{L}^{1}$-spaces there is a spectral mapping theorem for the real part of the spectrum.

Proposition 4.5. Let $(T(t))_{t \geqq 0}$ be a strongly continuous semigroup of lattice homomorphisms on an $\mathrm{L}^{1}$-space and denote by $A$ its generator. Then we have\\
(4.7) $\exp (t \sigma(A) \cap \mathbb{R})=\sigma(T(t)) \cap(0, \infty)$ for every $t \geqq 0$.

Proof. In view of A-III, 6.2 it is enough to prove that the left hand side contains the set on the right.\\
Fix $t>0$ and assume $r \in \sigma(T(t)), r>0$ and let $\alpha:=\frac{1}{t} \log r$. At first we assume $r \in R(\sigma(T)(t))$. Then by A-III,Thm. 6.3 there exists $\beta \in \mathbb{R}$ such that $\alpha+i \beta \in R_{\sigma}(A)$. By Lemma 4.1 either $\alpha+i \beta \mathbb{Z} \subset R_{\sigma}(A)$ or $\{\lambda \in \mathbb{C}: \operatorname{Re} \lambda<\alpha\} \in \operatorname{Ro}(A)$. In both cases we have $\alpha \in \sigma(A)$.\\
Now we assume $r \in A_{o}(T(t))$. Then there exists a normalized sequence $\left(f_{n}\right) \subset E$ such that $\lim _{n \rightarrow \infty}\left\|T(t) f_{n}-r f_{n}\right\|=0$. Since we have $|T(t)| f|-r| f||=||T(t) f|-r| f|| \leqq|T(t) f-r f|(f \in E$ ) we may assume that $\left(f_{n}\right)$ is a sequence in $E_{+}$.\\
Defining $g_{n}:=\int_{0}^{t} e^{-\alpha S} T(s) f_{n}$ ds we have $g_{n} \in D(A)$ and\\
$(A-\alpha) g_{n}=(A-\alpha) \int_{0}^{t} e^{-\alpha s_{T}}(s) f_{n} d s=e^{-\alpha t_{T}} T(t) f_{n}-f_{n}=\frac{1}{x}\left(T(t) f_{n}-r f_{n}\right)$. It follows that $\lim _{n \rightarrow \infty}\left\|(A-\alpha) g_{n}\right\|=0$ and it remains to show that liminf ${ }_{n \rightarrow \infty}\left\|g_{n}\right\|>0$. The latter assertion is a consequence of the following facts:

\begin{itemize}
  \item since $\mathrm{f}_{\mathrm{n}}$ is positive and the norm is additive on $\mathrm{E}_{+}$, we have $\left\|g_{n}\right\|=\int_{0}^{t} e^{-\alpha s}\left\|T(s) f_{n}\right\| d s$.
  \item The inequality $\|T(t) f\| s\|T(t-s)\|\|T(s) f\|$ implies $\|T(s) f\| \geq M^{-1}\|T(t) f\|$ for $0 \leqq s \leq t, f \in E$ and $M:=\sup _{0 \leq s \leq t}\|T(s)\|$.
  \item Since $\lim _{n \rightarrow \infty}\left\|T(t) f_{n}-r f_{n}\right\|=0$ and $\left\|f_{n}\right\|=1$ we have $\lim _{n \rightarrow \infty}\left\|T(t) f_{n}\right\|=r>0$.
\end{itemize}

For semigroups satisfying the assumption of Prop.4.5 $\sigma$ (A) is additively cyclic (by Thm.4.2) and o(T(t)) is multiplicatively cyclic (by Schaefer (1974), V.Thm.4.4). Then the relation (4.7) implies that decompositons of the spectrum by vertical lines allow a spectral decomposition of the semigroup (cf. A-III,Def.3.1). (One simply performs a spectral decomposition of a single operator $T(t)$ ). In the following we will show that for positive groups (on arbitrary Banach lattices) spectral decompositions of this type always exist. Moreover, it will turn out that the decomposition is compatible with the lattice structure. The proof of this result uses Kato's equality (see sec. 5 of C-II). As a consequence of C-II, Cor.5.8 we have the following:

Let $E$ be a Banach lattice with order continuous norm and (T(t)) $t \in \mathbb{R}$ be a group of positive operators on $E$ with generator A. Then the domain $\mathrm{D}(\mathrm{A}$ ) is a sublattice of E and (4.8) $A|f|=\operatorname{Re}[(\operatorname{sign} \bar{f}) A f]$ for every $f \in D(A)$. For real $\mu$ one has $\mu|f|=\operatorname{Re}\left[(\operatorname{sign} \bar{f})_{\mu} f\right]$, hence $(\mu-A)|f|=\operatorname{Re}[(\operatorname{sign} \bar{f})(\mu-A) f]$ for $\mu \in \mathbb{R}, f \in D(A)$. The relations $f^{+}=\frac{1}{2}(|f|+f), f^{-}=\frac{1}{2}(|f|-f)$ Yield $(\mu-A) f^{+}=\frac{1}{2}((\operatorname{sign} f)(\mu-A) f+(\mu-A) f)$ and $(\mu-A) f^{-}=\frac{1}{2}((\operatorname{sign} f)(\mu-A) f-(\mu-A) f)$, in case $f$ is contained in the underlying real Banach lattice $\mathbb{E}_{\mathbb{R}}$. For $\mu \in \rho(A) \cap \mathbb{R}$, we can apply $R(\mu, A)$ on both sides and the substitution $f=R(\mu, A) g$ finally leads to


\begin{align*}
& (R(\mu, A) g)^{+}=\frac{1}{2} R(\mu, A)((\operatorname{sign} R(\mu, A) g) g+g)  \tag{4.9}\\
& (R(\mu, A) g)^{-}=\frac{1}{2} R(\mu, A)((\operatorname{sign} R(\mu, A) g) g-g)
\end{align*} \text { for all } g \in E_{\mathbb{R}} .


If we set $g_{1}:=\frac{1}{2} \cdot(g+(\operatorname{sign} R(\mu, A) g) g)$ and

$$
g_{2}:=\frac{1}{2} \cdot(g-(\operatorname{sign} R(\mu, A) g) g)
$$

then obviously $g=g_{1}+g_{2}$. Moreover, $g$ is positive if and only if both, $g_{1}$ and $g_{2}$ are positive. We summarize these considerations in the following lemma.

Lemma 4.6. Let $A$ be the generator of a positive group on a Banach lattice $E$ which has order continuous norm. Given $\mu \in \rho(A) \cap \mathbb{R}$ then every $g \in E_{\mathbb{R}}$ is representable as sum of two elements $g_{1}$ and $g_{2}$ such that\\
(a) $g \geq 0$ if and only if both $g_{1}$ and $g_{2}$ are positive;\\
(b) $\quad \mathrm{R}(\mu, \mathrm{A}) \mathrm{g}_{1}=(\mathrm{R}(\mu, \mathrm{A}) \mathrm{g})^{+}$;\\
(c) $\quad R(\mu, A) g_{2}=-(R(\mu, A) g)^{-}$.

We need another lemma. It can be formulated for arbitrary positive semigroups on Banach lattices.

Lemma 4.7. Let $(T(t))_{t \geq 0}$ be a positive semigroup on a Banach lattice $E$ with generator $A$. Given $\mu \in \rho(A) \cap \mathbb{R}$ and $h \in E_{+}$then the following assertions are equivalent:\\
(i) $\quad \mathrm{R}(\mu, \mathrm{A}) \mathrm{h} \geqq 0$;\\
(ii) $\left\{\int_{0}^{t} e^{-\mu s} T(s) h d s: t \in \mathbb{R}_{+}\right\}$is bounded in $E$.

Proof. (i) $\rightarrow$ (ii): We have\\
$\int_{0}^{t} e^{-\mu s} \mathrm{~T}(s) h d s=\left(I d-e^{-\mu t} T(t)\right) R(\mu, A) h \quad(\operatorname{see} A-I,(3.2))$.\\
Since $R(\mu, A) h \geqq 0$ and $T(t)$ is a positive operator we obtain\\
$\int_{0}^{t} e^{-\mu s} T(s) h d s=R(\mu, A) h-e^{-\mu t} T(t) R(\mu, A) h \leqq R(\mu, A) h \quad$ which implies assertion (ii) .\\
(ii) -(i): The assumption implies that $\int_{0}^{\infty} e^{-V S T}(s) h d s:=$ $\lim _{t \rightarrow \infty} \int_{0}^{t} e^{-v s_{T}}(s) h$ ds exists for $v>\mu$. Using that $A$ is a closed operator it follows that $(v-A)\left(\int_{0}^{\infty} e^{-v s} T(s) h \mathrm{ds}\right)=h$. For $v$ sufficiently close to $\mu$ such that $v \in \rho(A) \cap \mathbb{R}$ we have $R(v, A) h=$ $\int_{0}^{\infty} e^{-v s} \mathrm{~T}(\mathrm{~s}) \mathrm{h}$ ds $\geqq 0$. By continuity we conclude $\mathrm{R}(\mu, \mathrm{A}) \mathrm{h} \geqq 0$.

By now we are prepared to prove the spectral decomposition for positive groups. Before we formulate the theorem we recall the following consequence of Thm.4.2 : For any $\mu \in \rho(A) \cap \mathbb{R}$ the line $\mu+i \mathbb{R}$ is a subset of the resolvent set and divides o(A) into disjoint sets. Both sets will be unbounded in general.

Theorem 4.8. Let $(T(t))_{t \in \mathbb{R}}$ be strongly continuous group of positive operators on a Banach lattice $E$ with order continuous norm.\\
If $A$ is the generator and $\mu \in \rho(A) \cap \mathbb{R}$ then\\
$I_{\mu}:=\{f \in E: R(\mu, A)|f| \geqq 0\}$ and $J_{\mu}:=\{f \in E: R(\mu, A)|f| \leqq 0\}$\\
are $(T(t))_{t \in \mathbb{R}^{-i n v a r i a n t}}$ projection bands, the direct sum of them\\
is E, and the spectra of the restrictions satisfy\\
$\sigma\left(A \mid I_{\mu}\right)=\sigma(A) \cap\{\lambda \in \mathbb{C}: \operatorname{Re} \lambda<\mu\}$,\\
$\sigma\left(A \mid J_{\mu}\right)=\sigma(A) \cap\{\lambda \in \mathbb{C}: \operatorname{Re} \lambda>\mu\}$.

Proof. At first we consider $I_{\mu}$. Obviously it is a closed subset. From Lemma 4.7 we deduce that it is a lattice ideal. Moreover, I ${ }_{\mu}$ is $R(\mu, A)$-invariant and $(T(t))_{t \in \mathbb{R}^{-1 n v a r i a n t}}$ as well (use Lemma 4.7 again).\\
Since $-A$ is the generator of the positive group $(T(-t))_{t \in \mathbb{R}}$ and $J_{\mu}=\{f \in E: R(-\mu,-A)|f| \geqq 0\}, J_{\mu}$ has the same properties. If $f \in I_{\mu} \cap J_{\mu}$ then $R(\mu, \mathrm{~A})|f|=0$ hence $f=0$ which shows that $I_{\mu} \cap J_{\mu}=\{0\}$. On the other hand, decomposing $0 \leq \mathrm{h}=\mathrm{h}_{1}+\mathrm{h}_{2} \in \mathrm{E}_{+}$ according to Lemma 4.6 , then assertion (b) of this lemma implies that $h_{1} \in I_{\mu}$, while assertion $(c)$ ensures that $h_{2} \in J_{\mu}$. Since the positive cone $E_{+}$is generating we have $E=I_{\mu} \oplus J_{\mu}$ and the first part of the theorem is proved.\\
Since $I_{\mu}$ is $R(\mu, A)$-invariant we have $\mu \in \rho\left(A \mid I_{\mu}\right)$ and $R(\mu, A \mid I)_{\mu}=R(\mu, A) \mid I_{\mu} \geqq 0 . C-I I I$, Thm.1.1(b) then implies $\sigma\left(A \mid I_{\mu}\right) \subset\{\lambda \in \mathbb{C}: \operatorname{Re} \lambda<\mu\}$. The same argument applied to $-A$ and $-\mu$ yields $\sigma\left(A_{\mu} J_{\mu}\right) \subset\{\lambda \in \mathbb{C}:$ Re $\lambda>\mu\}$. Now the assertion follows from A-III, Prop.4.2.

The spectral projections corresponding to the spectral decomposition described above have the expected representation as an integral 'around' the spectral sets, see Corollary 3 in Greiner (1984c).

Corollary 4.9. Assume that the assumptions of the theorem are satisfied, $\mu \in \rho(A) \cap \mathbb{R}, B>s(A), \alpha<-s(-A)$. If we denote the projections corresponding to the decomposition $\mathrm{E}=\mathrm{I}_{\mu} \oplus \mathrm{J}_{\mu}$ by $\mathrm{P}_{\mu}$ and $Q_{\mu}$ (i.e., $P_{\mu} E=\operatorname{ker} Q_{\mu}=I_{\mu}, Q_{\mu} E=\operatorname{ker} P_{\mu}=J_{\mu}$ ), then for $f \in D\left(A^{2}\right.$ ) we have


\begin{align*}
& P_{\mu} f=\frac{1}{2 \pi} \cdot \int_{-\infty}^{\infty} R(\mu+i \tau, A) f d \tau-\frac{1}{2 \pi} \cdot \int_{-\infty}^{\infty} R(\alpha+i \tau, A) f d \tau, \\
& Q_{\mu} f=\frac{1}{2 \pi} \cdot \int_{-\infty}^{\infty} R(B+i \tau, A) f d \tau-\frac{1}{2 \pi} \cdot \int_{-\infty}^{\infty} R(\mu+i \tau, A) f d \tau . \tag{4,10}
\end{align*}


(The integrals appearing in (4.10) are improper Riemann integrals.)

We mention another consequence of Thm.4.8. Like Prop.4.5 it is a spectral mapping theorem for the real part of the spectrum.

Corollary 4.10. If $(T(t))_{t \in \mathbb{R}}$ is a positive group on a space $L^{2}$ or $C_{0}(\mathrm{X})$ with generator A , then\\
(4.11) $\sigma(T(t)) \cap R_{+}=\exp (t \sigma(A) \cap \mathbb{R})$ for every $t \geqq 0$.

Proof. We borrow from the next chapter that for positive semigroups on spaces $L^{1}, L^{2}$ or $C_{O}(X)$ spectral bound and growth bound coincide (see C-IV,Thm.1.1).\\
We only have to show that $\exp (t \rho(A) \cap \mathbb{R}) \subset \rho(T(t)) \cap \mathbb{R}{ }_{+} \cdot$\\
If we consider a positive semigroup on an $\mathrm{L}^{2}$-space, Thm. 4.8 can be applied directly: Given $\mu \in \rho(A) \cap \mathbb{R}$, then $E=I_{\mu} \oplus J_{\mu}$ according to Thm.4.8 . The result mentioned above implies $r\left(T(t) / I_{\mu}\right)<e^{\mu t}$ and $r\left(T(-t) \mid J_{\mu}\right)<e^{\mu t}$. Hence $\sigma\left(T(t) \mid I_{\mu}\right\} \subset\left\{\lambda \in \mathbb{C}:|\lambda|<e^{\mu t}\right\}$ and $\sigma\left(T(t) \mid J_{\mu}\right)=\left(\sigma\left(\mathbb{T}(-t) \mid J_{\mu}\right)\right)^{-1} \subset\left\{\lambda \in \mathbb{C}:|\lambda|>e^{\mu t}\right\}$.\\
Thus $\sigma(T(t))=\sigma\left(T(t) \mid I_{\mu}\right) \cup \sigma\left(T(t)!J_{\mu}\right)$ does not contain $e^{\mu t}$. In case $(T(t))$ is a positive group on $C_{0}(X)$ then the adjoint\\
\includegraphics[max width=\textwidth]{2024_12_23_c6487cc0859199a15bd9g-338} follows that $E^{*}$ is a sublattice of $C_{0}(X)^{\prime} \cong M_{b}(X)$ hence a $\mathrm{L}^{1}$-space. The argument given for the $\mathrm{L}^{2}$-space yields $\sigma(T(t) *) \cap \mathbb{R}_{+}=\exp \left(t \sigma\left(A^{*}\right) \cap R\right)$ for every $t \geqq 0$. Thus the assertion follows from A-III, 4. 4 .

We conclude by describing a general situation where lattice semigroups occur. In section 4 of B-III we constructed semigroups of lattice homomorphisms on $C_{C}(X)$ starting with a continuous (semi-)flow on the locally compact space $X$ and a multiplication operator. One can perform similar constructions on spaces $\mathrm{L}^{\mathrm{P}}(\mu)$ for $1 \leqq \mathrm{p}<\infty$ under certain conditions on the flow. We consider an example which shows where the problems are.\\
Define the semiflow $\phi$ on $\mathbb{R}_{+}$as follows: $\phi(t, x):=x-t$ for $x \geqq t$ and $\phi(t, x):=0$ for $x \leqq t$. For $f \in L^{P}(\mu)$ one has difficulties to define fo申 t properly since the preimage of the zero-set $\{0\}$ does not have measure zero. This problem does not arise in case every transformation $\phi_{t}$ is measure preserving, i.e. $\left.\mu \phi_{t}^{-1}(C)\right)=\mu(C)$ for every Borel set $C$. A more general criterion is stated in the following proposition.

Proposition 4.11. Let $x$ be a locally compact space and let $\mu$ be a regular, positive Borel measure on $X$. Assume that the continuous semiflow $\phi: \mathbb{R}_{+} \times X \rightarrow X$ satisfies the following condition:\\
(4.12) $\phi_{t}^{-1}(\mathrm{~K})$ is compact for every compact set $\mathrm{K} \subset \mathrm{X}, \mathrm{t} \geq 0$.\\
(a) For every $p, 1 \leqq \mathrm{p}<\infty$ the following assertions are equivalent.\\
(i) The operators $T(t)$ defined by $T(t) f:=f \circ \phi_{t}$ for $f \in L^{p}(\mu)$, $t \geqq 0$, are well-defined as bounded linear operator on $\mathrm{L}^{\mathrm{P}}(\mu)$ and $(T(t))_{t \geq 0}$ is a strongly continuous semigroup.\\
(ii) There exist constants $t_{0}>0, M>0$ such that $\mu\left(\phi_{t}^{-1}(C)\right) \leqq M \cdot \mu(C)$ for every open (compact) set $C \subset X$ and every $t \leqq t_{0}$.\\
(b) In case the conditions (i) and (ii) are fulfilled then (T( $t$ ) $t \geqq 0$ is a semigroup of lattice homomorphisms on $L^{P}(\mu)$ and $C_{C}(X) \cap D(A)$ is a core of the generator.

Proof. (a) since $\mu$ is assumed to be regular, the inequality stated in (ii) holds true for all Borel sets provided it is true for all open sets (compact sets, respectively).\\
(i) $\rightarrow$ (ii): Assume that $(T(t))$ is a strongly continuous semigroup on $\mathrm{L}^{\mathrm{p}}(\mu), 1 \leqq \mathrm{p}<\infty$. For $t_{0}>0$ we define $M:=\left(\sup \left(\|\mathrm{T}(t)\|: 0 \leqq t \leqq t_{0}\right\}\right)^{1 / p}$. Given a Borel set $c \in x$ we write $c(t):=\phi_{t}{ }^{-1}(c)$. Then we have $T(t) x_{C}=x_{C}(t)$, hence\\
$\mu\left(\phi_{t}^{-1}(C)\right)=\left\|x_{C}(t)\right\|_{P}^{P}=\left\|T(t) x_{C}\right\|_{P}^{P} \leqq M \cdot\left\|x_{C}\right\|_{P}^{P}=M \cdot \mu(C)$.\\
(ii) $\rightarrow$ (i): since the inequality in (ii) holds for all Borel sets, $\phi_{t}^{-1}(c)$ is a p-null set whenever $C$ is a u-null set. Thus given Borel functions $f, g$ such that $f=g \mu-a . e$ then $f \circ \phi_{t}=g \circ \phi_{t}$ $\mu-a . e$. Moreover, for $0 \leqq \pounds \in \mathrm{~L}^{\mathrm{P}}(\mu)$, there exists an increasing sequence ( $h_{n}$ ) of simple functions converging pointwise to $f$. Then ( $h_{n}{ }^{\circ} \phi_{t}$ ) is a monotone sequence converging pointwise to fo ${ }_{t}$. Using the fact that $\chi_{C}{ }^{\circ}{ }_{t}=x_{C(t)}, C(t)$ as above, and the assumption $\mu(C(t)) \leqq M \cdot \mu(C) \quad$ it is easy to see that $\left\|h_{n}{ }^{\circ} \phi_{t}\right\|_{p}^{p} \leqq M \cdot\left\|h_{n}\right\|_{p}^{p} \leqq M \cdot\|f\|_{p}^{p} \cdot$ Thus by the Monotone Convergence Theorem we have fo ${ }_{t} \in \mathrm{~L}^{\mathrm{P}}(\mu)$ and $\|$ o $_{t}\left\|_{p} \leftrightarrows M^{1 / p}\right\| f \|_{p}$. It follows that $T(t)$ is a bounded linear operator on $L^{P}(\mu)$ and $\|T(t)\| \leqq M^{1 / p}$ for $0 \leqq t \leqq t_{0}$. Since $\phi$ is semiflow we have $T(0)=I d$ and $T(t+s)=T(s) T(t)(0 \leqq s, t<\infty)$. It remains to prove strong continuity. Since $\phi$ is continuous and (4.12) holds, we know that $T(t)\left(C_{C}(X)\right) C_{C}(X)$ and that $T(t) \pounds$ tends to $f$ uniformly as $t \rightarrow 0$ provided that $f \in C_{C}(x)$. It follows that $\lim _{t \rightarrow 0}\|T(t) f-f\|_{p}=0$ for $f \in C_{C}(X)$. since $C_{C}(X)$ is dense in $L^{P}(\mu)$ and $\|T(t)\| \leqq M^{1 / P}$ for $0 \leqq t \leqq t_{0}$, the semigroup is strongly continuous.\\
(b) Obviously every operator $T(t)$ defined in assertion (i) of (a) is a lattice homomorphism. Above we pointed out that $C_{C}(X)$ is\\
invariant under $(T(t))$, then $D(A) \cap C_{c}(X)$ is invariant as well. It is dense because the elements of the form $\int_{0}^{r} T(s) f d s, f \in C_{c}(X)$, $r>0$, belong to $C_{C}(X)$ and to $D(A)$. Hence $D(A) \cap C_{C}(X)$ is a core (by A-I, cor.1.34).

Prop.4.11 can be used to prove that flows corresponding to certain ordinary differential equations on $\mathbb{R}^{\text {n }}$ generate strongly continuous semigroups on $\mathrm{L}^{\mathrm{p}}\left(\mathbb{R}^{\mathrm{n}}\right)$ (where $\mathbb{R}^{\mathrm{n}}$ is equipped with the Lebesgue measure). One has to impose conditions on the corresponding vector field. Note that for continuous flows condition (4.12) is automatical$1 y$ satisfied because for a compact $K \subset X$ the set $\phi_{t}{ }^{-1}(K)=\phi_{-t}(K)$ is compact as the continuous image of a compact set.

Example 4.12. Let $F=\mathbb{R}^{n} \rightarrow \mathbb{R}^{n}$ be a $C^{1}$-vector field and assume that the derivative DF is uniformly bounded on $\mathbb{R}^{n}$. Then the ordinary differential equation $y^{\prime}=F(y)$ possesses a global flow $\phi: \mathbb{R} \times \mathbb{R}^{n} \rightarrow \mathbb{R}^{n}$ which is $C^{1}$. Moreover, we have\\
(4.13) $\left\|D \phi_{t}(x)\right\| \leqq e^{M|t|}$ for all $x \in \mathbb{R}^{n}, t \in \mathbb{R}$, where $M:=\sup \left\{\|D F(x)\|: x \in \mathbb{R}^{n}\right\}$.

All these properties were proven in Ex. 3.15 of B-II.\\
We will show that $\phi$ satisfies condition (ii) of Prop. 4.11 (a). Hence it induces a strongly continuous (semi-)group of lattice homomorphisms on $L\left(\mathbb{R}^{n}\right)(1 \leqq p<\infty)$ via $T(t) \pm=f \circ \phi_{t}$.\\
This is done using the change of variables formula as follows:\\
Let $U$ be an open subset of $\mathbb{R}^{n}$, then $\phi_{t}^{-1}(U)=\phi_{-t}(U)=$ : $U(-t)$. If $\lambda$ denotes the Lebesgue measure then


\begin{align*}
& \lambda\left(\phi_{t}^{-1}(U)\right)=\int_{U(-t)} 1 d x=\int_{U} 10 \phi_{-t}(x) \cdot\left|\operatorname{det} D \phi_{-t}(x)\right| d x= \\
& \int_{U}\left|\operatorname{det} D \phi_{-t}(x)\right| d x \leqq \int_{U} e^{n M|t|} d x=e^{n M|t|} \cdot \lambda(U) \cdot \tag{4.14}
\end{align*}


Here we used (4.13) and the fact that the determinant of an n n-matrix $B$ satisfies $|\operatorname{det} B| \leqq\|B\|^{n}$.

In general, existence of a global flow does not ensure that one can associate a semigroup of bounded linear operators on ${ }_{4}{ }^{\mathrm{P}}\left(\mathbb{R}^{\mathrm{n}}\right)$, even if the vector field is $C^{\infty}$. For example the differential equation $y^{\prime}=\sin \left(y^{2}\right)$ does not induce a semigroup on $L^{p}(\mathbb{R})$.\\
There is another important class of differential equations which do induce semigroups of lattice homomorphisms on $\mathrm{L}^{\mathrm{P}}$-spaces: Hamiltonian differential equations. In fact, Liouville's Theorem states that the\\
flow corresponding to a Hamiltonian vector field preserves the volume (see Abraham-Marsden (1978, Sec.3.3). Thus assertion (ii) of Proposition $4.11(a)$ is trivially satisfied.

Further examples of flows which are measure preserving and therefore induce semigroups of lattice homomorphisms on LP-spaces are billiard flows on compact convex subsets of $\mathbb{R}^{n}$ and geodesic flows on Riemannian manifolds (see Cornfeld-Fomin-Sinai (1982)).

\section*{NOTES.}
Spectral theory for a single positive operator as developed in Chapter V of Schaefer (1974) is an essential tool for this chapter. Various results on the spectral theory of positive one-parameter semigroups can be found in Chapter 7 of Davies (1980) and in the second part of Batty-Robinson (1984).

Section 1. That the spectral bound is always an element of the spectrum was stated by Karlin (1959), but a valid proof was given by Derndinger (1980). This assertion as well as assertion (b) of Theorem 1.1 allow generalizations in various directions: They are valid for ordered Banach spaces (see Greiner-Voigt-Wolff (1981) and Klein (1984) ) and one only needs that $A$ has positive resolvent (see Kato (1982) or Nussbaum (1984)). Theorem 1.2 as well as its corollaries are also valid in ordered Banach spaces. For the analogue in the theory of the Laplace transform we refer to Sec.10.5 in Widder (1971) and Voigt (1982).

Section 2. Theorem 2.2 is the basis for the subsequent cyclicity results. Pseudoresolvents are discussed e.g. in Hille-Phillips (1957) or Yosida (1965). For nonpositive semigroups the two assertions stated in Def. 2.8 are no longer equivalent. A special case of Theorem 2.10 was proven by Derndinger (1980) while the general result is due to Greiner (1981). Instead of pseudo-resolvents on the whole F-product Derndinger works with the semigroup on the semigroup F-product. Therefore he can only consider eigenvalues. Elliptic differential operators as generators of positive semigroups are discussed by many authors, e.g., Amann (1983), Fattorini (1983), Friedmann (1972) or Pazy (1983).

Section 3. There exist various notions which are (more or less closely) related to irreducibility, e.g. 'positivity improving' in Reed-Simon (1979), u -positivity in Krasnosel'skii (1964) or 'quasi-strictly positive' in Karlin (1959)). Sawashima (1964) uses 'non-support' instead of irreducible. She also discusses several modifications (semi-non-support, strictly non-support, strongly positive) and the interrelationship between these notions. The notion of irreducibility can be extended to the (non-lattice) ordered setting (see Batty-Robinson (1984)). Assertion (b) of Theorem 3.2 is due to Majewski-Robinson (1983) while special cases can be found in Sec. XIII. 12 of Reed-Simon (1979) and in Kishimoto-Robinson (1981). Proposition 3.3 is due to Voigt (1984). Retarded equations as dicussed in Example 3.4(c) will be discussed in more detail in Section 3 of C-IV. Example 3.4(d) is a one-dimensional version of the linear transport equation. The higher dimensional equation is more delicate but can be treated similarly (see e.g. Greiner (1984), Kaper-LekkerkerkerHejtmanek (1983), or Voigt (1984b)). A special case of Proposition 3.5 can be found\\
in Davies (1980). Theorem 3.7 and Example 3.6 are taken from Schaefer (1985). The most interesting criterion of Thm. 3.7 seems to be condition (c), since it gives a sufficient condition for the existence of eigenvalues for a sufficiently large class of semigroups. For semigroups induced by measure-preserving flows Theorem 3.8 and Corollary 3.9 are proven in Cornfeld-Fomin-Sinai (1982). Corollary 3.9 is a special case of the Halmos-von Neumann Theorem which classifies irreducible semigroups having discrete spectrum (see Cornfeld-Fomin-Sinai (1982), Greiner (1982) and Schaefer (1974) for the general result). Lemna 3.10 is taken from Groh (1984b) and Theorems 3.12 and 3.14 can be found (with slightly different proofs) in Greiner (1981).

Section 4. It was Derndinger (1980) who proved Theorem 4.2. In Cor.4.3 one can replace boundedness of the semigroup by the assumption that the resolvent grows slowly (see Greiner (1982)). Example 4.4 is due to Davies and Proposition 4.5 to Kellermann (both unpublished). The spectral decomposition for positive groups as described in Theorem 4.8 is valid in arbitrary Banach lattices (see Arendt (1982) and Greiner (1984c)). This also holds for Corollaries 4.9 and 4.10 . Proposition 4.11 and Example 4.12 indicate the relationship of positive groups to dynamical systems. For example, the 'Annular Hull Theorem' (see Chicone-Swanson (1981)) is closely related to the results of this section.

\section*{A S Y M P T O T ICS \\
 OF POSITIVE SEMIGROUPS \\
 ON BANACH LATTICES }
In this chapter we describe the long term behavior of positive semigroups and discuss some concrete examples in more detail.\\
The first section is devoted to the stability of positive semigroups, and we give sufficient and necessary conditions which ensure that the semigroup (and the solution of the abstract Cauchy problem, respectively) converges to zero as $t \rightarrow \infty$. It is shown that for positive semigroups stability is determined fairly well by spectral properties of the generator.\\
In the second section we describe conditions ensuring convergence of the semigroup (as $t \rightarrow \infty$ ) to an equilibrium point or to a periodic solution. Again we are interested in spectral conditions ensuring such a behavior.\\
In the final section a series of examples is discussed in more detail. In particular we consider semigroups related to retarded equations and discuss existence of solutions, spectral properties and asymptotic behavior. Most of the examples are motivated by biological models.

\section*{1. STABILITY OF POSITIVE SEMIGROUPS ON BANACH LATTICES }
In section 1 of $B-I V$ we have seen that the growth bound of a positive semigroup on spaces $C_{0}(X)$ coincides with the spectral bound of the generator A, which is - for positive semigroups - an element of the spectrum of $A$. Now, using the results of A-III, A-IV, B-IV, Sec. 1 and C-III, it can be shown that this is valid for positive semigroups on AM- , AL- and Hilbert spaces.

Theorem 1.1. Let A be the generator of a positive semigroup $(T(t))_{t \geq 0}$ on a Banach lattice $E$ such that $s(A)>-\infty$. Each of the subsequent conditions implies

$$
s(A)=\omega_{1}(A)=\omega(A) \in \sigma(A)
$$

(a) Either $E$ is an $A M$-space or an $L^{2}$-space or an $L^{1}$-space.\\
(b) There exist $\tau>0, h \in E_{+}$such that $T(\tau) E \subset E_{h}$.\\
(c) There exist $\tau>0, \phi \in E_{+}^{\prime}$ such that $\|\mathrm{T}(\tau) f\| \leqq\langle f, \phi\rangle$ for all $f \in E_{+}$.

Proof. We know that $s(A) \leqq w_{1}(A) \leq \omega(A)$ (see $A-I V$, Cor.1.5) and $s(A) \in \sigma(A)$ (see $C-I I I$, cor.1.4). Thus we have to show $s(A)=\omega(A)$.\\
(a) For AM-spaces the proof given in section 1 of B-IV works (cf. B-IV,Rem.1.5.).\\
Since for positive semigroups we always have $\|R(\lambda, A)\| \leqq\|R(\operatorname{Re} \lambda, A)\|$ ( $\operatorname{Re} \lambda>s(A)$ (see C-III, Cor.1.3) the assertion for $\mathrm{L}^{2}$-spaces follows from A-III, Cor.7.10.\\
If E is an $L^{1}$-space the assumptions of (c) are satisfied.\\
(b) We identify $\mathrm{E}_{\mathrm{h}}$ according to the Kakutani-Krein Theorem with a space $C(K), K$ compact. Considering $T(\tau)$ as operatox from $E$ into $C(K)$, we denote it by $T_{0}$. Then $T_{0}$ is positive hence continuous (see Schaefer (1974), II.Thm.5.3). Let $j: C(K) \cong E_{h} \rightarrow E$ be the canonical inclusion. The spectral radii of $\mathrm{T}(\mathrm{T})=j \circ \mathrm{~T}_{0}$ and $\mathrm{T}_{0}{ }^{\circ}$ coincide and are given by $\rho:=\exp (\tau \cdot \omega(A))$, By the Krein-Rutman Theorem (cf. the Corollary to Thm. 2.6 in the Appendix of Schaefer (1966) ) there exists $0<\mu \in C(K)^{\prime}$ such that $\left(T_{0}{ }^{\circ}\right)^{\prime} \mu=\rho \cdot \mu$. Then $\phi:=T_{0}^{\prime} \mu$ is an eigenvector of (j०T$)_{0}$ ' with eigenvalue $\rho$. Thus $\rho \in \operatorname{Ro}(T(\tau))$ and hence $s(A) \geqq \omega(A)$ by $A-I I I$, Thm. 6.2.\\
(c) For $a>s(A), r>\tau, f \in E_{+}$we have\\
$\int_{0}^{r} e^{-\alpha S}\|T(s) f\| d s=\int_{0}^{\tau} e^{-\alpha s}\|T(s) f\| d s+e^{-\alpha \tau} \int_{0}^{r-\tau} e^{-\alpha s}\|T(\tau) T(s) f\| d s \leq$\\
$\leqq \int_{0}^{\tau} e^{-\alpha s}\|\mathrm{~T}(s) f\| d s+e^{-\alpha \tau} \int_{0}^{\mathrm{r}-\tau} e^{-\alpha s}\langle\mathrm{~T}(s) \mathrm{f}, \phi\rangle d s \leqq$\\
$\leq \int_{0}^{\mathrm{T}} e^{-\alpha \mathrm{S}}\|\mathrm{T}(\mathrm{s}) \mathrm{f}\| \mathrm{ds}+\|\mathrm{R}(\alpha, \mathrm{A}) \mathrm{f}\|$\\
(the last inequality follows from C-III, Thm.1.2).\\
Now Datko's Theorem (A-IV,Thm.1.11) implies $w(A)<\alpha$

For $L^{p}$-spaces, $p \neq I, 2, \infty$, it is not known whether spectral- and growth bound of an arbitrary positive semigroup coincide. Using interpolation techniques and Thm.1.1 one can treat some special cases. Before doing this we have to recall some facts on interpolation. For details we refer to [Dunford-Schwartz (1958), VI.10] or [Reed-Simon (1975), IX.4.].

Let $(X, \Sigma, \mu)$ be a $\sigma$-finite measure space, $1 \leqq \mathrm{p}<\mathrm{q}<\infty$ and suppose that $\mathrm{T}_{0}: \mathrm{L}^{\mathrm{P}}(\mu) \cap \mathrm{L}^{\mathrm{q}}(\mu) \rightarrow \mathrm{L}^{\mathrm{P}}(\mu) \cap \mathrm{L}^{\mathrm{q}}(\mu)$ is a linear operator which satisfies $\left\|T_{o} f\right\|_{p} \leqq c_{p}\|f\|_{p}$ and $\left\|T_{o} f\right\|_{q} \leqq c_{q}\|f\|_{q}$. Then for every $r \in[p, q], T_{0}$ has a (unique) continuous extension $T_{r}: L^{r}(\mu) \rightarrow L^{r}(\mu)$. Moreover ,\\
(1.1) $\mathrm{u} \rightarrow \log \left\|_{1 / \mathrm{u}}\right\|$ is a convex function on the interval $\left[\frac{1}{\mathrm{q}}, \frac{1}{\mathrm{p}}\right]$.

Applying this result to the powers $\mathbb{T}_{r}^{n}$ ( $\mathrm{n} \in \mathbb{N}$ ) and using the fact that the pointwise limit of convex functions is convex, we obtain that the logarithm of the spectral radius is convex, i.e.,\\
(1.2) $\left.u \rightarrow \log \left(r_{1 / u}\right)\right)=\lim _{n \rightarrow \infty} \frac{1}{n} \log \left\|T_{1 / u}^{n}\right\|$ is convex on $\left[\frac{1}{q}, \frac{1}{p}\right]$.

In the following we suppose that for every $r \in[p, q]$ we have a strongly continuous semigroup $\left(T_{r}(t)\right)_{t \geqslant 0}$ on $L^{r}(\mu)$ such that\\
(1.3) $T_{r}(t) / L^{r} \cap L^{s}=T_{s}(t) / L^{r} \cap L^{s}$ for all $r, s \in[p, q], t \geqq 0$.

Let $A_{r}$ be the generator of $\left(T_{r}(t)\right), w(r)$ its type and $s(r)$ the spectral bound of $A_{r}$. In this situation we have the following corollary of Thm.1.1.

Corollary 1.2. Suppose that the semigroups $\left(T_{r}(t)\right){ }_{t \geqq 0}$ are positive.\\
(a) In case $p<2<q$ and $w(r)$ independent of $r \in[p, q]$, one has $s(r)=\omega(r)$ for all $r \in[p, q]$.\\
(b) If $p=1, q \geqq 2$ and $s(r)$ independent of $r \in[p, q]$ then $s(r)=w(r)$ for $r \in[1,2]$.

Proof. Once it is shown that both functions $u \rightarrow s(1 / u)$ and $u \rightarrow \omega(1 / u)$ are convex on $\left[\frac{1}{q}, \frac{1}{p}\right]$, the assertion follows from Thm.1.1 and the relation $s(r) \leqq \omega(r)$ for every $r$. Since $\omega(u)=$ $\log r\left(T_{u}(1)\right)($ see $A-I I I,(1.4)),(I .2)$ implies that $\mathrm{u} \rightarrow \omega(I / \mathrm{u})$ is a convex function. By C-III, Thm. 1.1 we have $r\left(R\left(k, A_{u}\right)\right)=(k-s(u))^{-1}$ for $k \in \mathbb{N}$ sufficiently large. The assumption (1.3) implies that $R\left(\lambda, A_{r}{ }^{\prime} \mid L^{r} \cap L^{S}=R\left(\lambda, A_{S}{ }^{\prime} \mid L^{r} \cap L^{s}\right.\right.$ for $r, s \in[p, q]$ and $\lambda \in \mathbb{C}$ with $\operatorname{Re} \lambda$ large enough. Hence by (1.2) $u \rightarrow \log \left[r\left(R\left(k, A_{1 / u}\right)\right]\right.$ is a convex function for large $k \in \mathbb{N}$. We have\\
$\log \left[\left(1-\frac{1}{k} s(1 / u)\right)^{-k}\right]=k \cdot \log k+k \cdot \log [k-s(1 / u)]^{-1}=$\\
$=k \cdot \log k+k \cdot \log \left[r\left(R\left(k, A_{1} / u_{k}\right)\right]^{-1}\right.$,\\
hence all the functions $u \rightarrow \log \left[\left(1-\frac{1}{k} s(1 / u)\right)^{-k}\right], k \in \mathbb{N}$, are convex. It follows that $u \rightarrow s(1 / u)=\lim _{k \rightarrow \infty}\left(\log \left[\left(1-\frac{1}{k} s(1 / u)\right)^{-k}\right]\right)$ is convex as well.

One can apply the corollary to Schrödinger operators on the spaces $L^{\mathrm{P}}\left(\mathbb{R}^{\mathrm{n}}\right)$, i.e., operators $A=\Delta+V$ where $\Delta$ is the Laplacian and $V$ is a multiplication operator, see Simon (1982) for details. In Thm. B. 5.1 (1.c.) it is shown that for certain potentials $V$ the type is independent of $p \in[1, \infty)$. Thus the assumptions of (a) are satisfied. Part (b) can be applied if $q>2$ and if $A_{1}$ has compact resolvent. Then all operators $A_{r}, I \leqq r<q$ have compact resolvent and therefore their spectra coincide. In particular, $s\left(A_{r}\right)$ is independent of $r \in[1, q)$.

As shown in A-IV,Ex.1.2(2), the equality $s(A)=\omega(A)$ may not hold for positive semigroups on arbitrary Banach lattices. However, the knowledge of $s(A)$ is still sufficient to determine the growth bound $\omega_{1}$ (A) of the strong solutions of the abstract cauchy problem. In fact, combining Theorems 1.1 and 1.2 of C-III with Theorem 1.4 of A-IV we obtain the following fundamental result for the stability of positive semigroups.

Theorem 1.3. Let $A$ be the generator of a positive semigroup $(T(t))_{t \geqslant 0}$ on a Banach lattice. Then $s(A)=\omega_{1}(A) \in \sigma(A)$.

Recalling the definition of $\omega_{1}(A)$ (see A-IV,Def.1.1) and the fact that $s(A)$ is always an element of $\sigma(A)$, we can reformulate the statement of Thm.1.3 as follows.

Corollary 1.4. Let $\left(T(t) t_{t \geqslant 0}\right.$ be a positive semigroup on a (real or complex) Banach lattice with generator A . Each of the following conditions implies that the solutions of the abstract Cauchy problem are exponentially stable, i,e., there is $\delta>0$ such that $\lim _{t \rightarrow \infty} e^{\delta t_{T}(t)} f=0$ for every $f \in D(A)$.\\
(a) $\lambda-A$ is invertible for every $\lambda \geqq 0$;\\
(b) $A$ is invertible and $A^{-1} \leqq 0$.

Proof. In case of a real Banach lattice we consider the complexification (see Sec. 7 of $C-I$ ). Note that both, the hypotheses and the satement remain preserved.\\
Since $s(A) \in \sigma(A)$ assertion (a) implies $s(A)<0$. If (b) is satisfied then $R(0, A) \geqq 0$, hence $S(A)<0$ by C-III,Thm.I.I(b). It follows from Thm.1.3 that $\sup \{\omega(f): f \in D(A)\}=\omega_{1}(A)<0$.

In the following we give a spectral characterization of stability for eventually norm-continuous positive semigroups. An important tool in the proof is the following result on power bounded operators due to Katznelson-Tzafriri (1984):

Let $R$ be a Iinear operator on a Banach space\\
(1.4) such that $\sup _{n \in \mathbb{N}}\left\|R^{n}\right\|<\infty$. Then one has $\sigma(R) \cap \Gamma \subseteq\{1\} \quad$ if and only if $\lim _{n \rightarrow \infty}\left\|R^{n}-R^{n+1}\right\|=0$.

Theorem 1.5 . Let $(\mathrm{T}(\mathrm{t}))_{t \geq 0}$ be a positive semigroup on a Banach lattice $E$ which is bounded and eventually norm-continuous. The following two assertions are equivalent:\\
(i) $(T(t))_{t \geq 0}$ is uniformly stable;\\
(ii) $0 \notin R_{\sigma}(A) \quad\left(i . e ., \operatorname{ker} A^{\prime}=\{0\}\right)$.

In case E is reflexive (i) and (ii) are equivalent to\\
(iii) $0 \& P_{\sigma}(A)$ (i.e., ker $\left.A=\{0\}\right)$.

Proof. (i) $\rightarrow$ (ii) was proven in A-IV,Thm.1.12 in a more general setting.\\
(ii) $\rightarrow$ (i) In case $\omega(A)<0$ one trivially has (i). Therefore we can assume $\omega(A)=0$. By Cor.2.13 and Prop.2.9 of C-III we have $\sigma(A) \cap \mathbb{R}=\{0\}$. Since the spectral mapping theorem holds (cf. Thm. 6.6\\
and Thm. 6.3 of A -III) we have\\
(1.5) $\sigma(\mathrm{T}(1)) \cap \mathrm{O}=\{1\}$ and 1$\} \operatorname{Ro}(\mathrm{T}(1))$.

From (1.4) it follows that $\lim _{n \rightarrow \infty}\|T(n)-T(n+1)\|=0$ and therefore $\lim _{t \rightarrow \infty}\|\mathrm{~T}(t)-\mathrm{T}(t+1)\|=0$. Thus given $g \in \operatorname{im}(\mathrm{Id}-\mathrm{T}(1))$ then $g=f-T(1) f$ for some $f \in E$ hence $\|T(t) g\|=\|(T(t)-T(t+1)) f\| \leqq$ $\|(T(t)-T(t+1))\| \cdot\|f\| \rightarrow 0$. The second assertion of (1.5) ensures that im(Id - T(1)) is dense in E. Since the semigroup is bounded we have $\lim _{t \rightarrow \infty}\|T(t) f\|=0$ for every $f \in \overline{\operatorname{im(Id}-T(1))}=E$, i.e., (T(t)) is uniformly stable.\\
(i) $\rightarrow$ (iii) is always true and follows from A-IV,Thm.1.13.\\
(iii) + (ii): The adjoint semigroup $\left(T(t)^{\prime}\right){ }_{t \geqq 0}$ is eventually norm-continuous and bounded and we have $\operatorname{R\sigma }\left(A^{\prime}\right)=P \sigma\left(A^{\prime \prime}\right)=P \sigma(A)$. Thus the implication "(ii) $\rightarrow(i) "$ can be applied and we obtain that (T(t)') $t \geqq 0$ is stable. Then $A-I V, T h m .1 .13$ yields $0 \notin P \sigma\left(A^{\prime}\right)=R \sigma(A)$.

As an application of Thm.1.5 we consider the Laplacian as generator on $\mathrm{L}^{\mathrm{P}}\left(\mathbb{R}^{\mathrm{n}}\right), 1 \leqq \mathrm{p}<\infty$, (see $\mathrm{A}-\mathrm{I}, 2.8$ ). For $\mathrm{p}=1$ the constant functions are eigenvectors of the adjoint operator, hence $0 \in \operatorname{Ro}(\Delta)$. Thus the semigroup is not stable on $L^{1}\left(\mathbb{R}^{n}\right)$. On the other hand, for $1 \leq p<\infty$ there does not exist a non-zero function $h \in L^{p}\left(\mathbb{R}^{n}\right)$ with $\Delta h=0$. Hence $\Delta$ generates a stable semigroup on $\mathrm{L}^{\mathrm{P}}\left(\mathbb{R}^{\mathrm{n}}\right)$ for $1<\mathrm{p}<\infty$. (That ker $\Delta=\{0\}$ can be deduced from the following two facts:

\begin{itemize}
  \item since the semigroup consists of contractions and since the norm is strictly monotone on $E_{+}$it follows that ker $\Delta$ is a sublattice. Thus irreducibility of the semigroup (see $A-I, 2.8$ and C-III,Ex.3.4(a)) implies that dim ker $\Delta \leqq 1$;
  \item The semigroup commutes with the translations on $\mathbb{R}^{\mathbf{n}}$, hence ker $\Delta$ is invariant under translations.)
\end{itemize}

In the next results we give conditions on the range of the generator which ensure stability. We begin with a generalization of Cor.1.4(b) .

Propositon 1.6. Let $A$ be the generator of a positive semigroup on a (real or complex) Banach lattice, $D(A))_{-}=-\left(D(A) \cap_{f}\right.$ ).\\
Then $\omega_{1}(A)<0$ if and only if $E_{+} \subset \operatorname{im} A(D(A))_{-}$.

Proof. If $\omega_{1}(A)<0$ then $s(A)<0$ (A-IV, Cor. I. 5), hence $A^{-1}=-R(0, A) \leqq 0$ by C-III,Thm.1.1.\\
If $\mathrm{E}_{+} \subset$ im $\mathrm{A}\left(\mathrm{D}(\mathrm{A})_{-}\right.$), then, for every $\mathrm{f} \in \mathrm{E}_{+}$, there exists $g \in \mathrm{D}^{+}(\mathrm{A})+$ such that $\mathrm{Ag}=-\mathrm{f}$. We have $0 \leqq \mathrm{~T}(\mathrm{t}) \mathrm{g}={ }^{+} \mathrm{g}+\int_{0}^{t} \mathrm{~T}(\mathrm{~s}) \mathrm{Ag} \mathrm{ds}$\\
$=g-\int_{0}^{t} T(s) f d s$, hence $0 \leqq \int_{0}^{t} T(s) f d s \leqq g$ for every $t \geqq 0$. For $\alpha>0$ we have $\int_{0}^{t} e^{-\alpha S_{T}} \mathrm{~T}(\mathrm{~s}) \mathrm{f} \mathrm{ds} \leqq \int_{0}^{t} \mathrm{~T}(\mathrm{~s}) \mathrm{f} \mathrm{ds} \leqq \mathrm{g}$. Using C-III, Thm.1.2 we conclude that $R(\alpha, A) f \leqq g$ for $\alpha>\max \{0, s(A)\}$. By the Uniform Boundedness Principle we know that $\{R(\alpha, A)$ : $\alpha>\max \{0, s(A)\}\}$ is uniformly bounded. Since $\omega_{1}(A)=s(A) \in \sigma(A)$ (see Thm.1.3) it follows that $\omega_{1}(A)<0$.

Next we show that weak uniform stability implies uniform stability provided that $E$ is weakly sequentially complete (see C-I, Sec.5) and $(i m A)_{+}:=A(D(A)) \cap E_{+}$is a total subset of $E$. The left translations on $\mathrm{L}^{2}\left(\mathbb{R}_{+}\right)$are stable. Hence, by $A-I V$, Rem.1.17(a), $\operatorname{im} A=\left\{f \in L^{2}\left(\mathbb{R}_{+}\right): \int_{0}^{\infty} f(x) d x\right.$ exists $\}$ and we see that (im A) is a total subset of $I^{2}\left(\mathbb{R}_{+}\right)$. On the other hand, (im $\left.A\right)_{+}=\{0\}$ for the generator of the non stable, but weakly stable semigroup of left translations on $\mathrm{L}^{2}(\mathbb{R})$.

Proposition I.7. Let A be the generator of a positive semigroup $(T(t))_{t \geq 0}$ on a weakly sequentially complete Banach lattice $E$, such that $(i m A)_{+}$is total in E. Then $(T(t))_{t \geqq 0}$ is uniformly stable if and only if it is weakly uniformly stable.

Proof. If (T(t)) $t \geqq 0$ is weakly uniformly stable, then (T) $(t)$ is bounded by the Uniform Boundedness Principle. Using the weak version of $A-I V, T h m .1 .14, \int_{0}^{\infty}\langle T(t) f, \phi\rangle d t$ exists for every $f \in(i m A)+$ and $\phi \in E_{+}^{\prime}$. It follows that the net $\left(\int_{0}^{r} T(t) f d t\right)_{r \geqq 0}$ is weakly Cauchy. Hence $\sigma\left(E^{\prime}, E\right)-\lim _{r \rightarrow \infty} \int_{0}^{r} T(t) f d t$ exists for every $f \in(i m A)_{+}$. Since the net is monotone one obtains convergence in norm by Dini's Theorem [Schaefer (1974), II.Thm.5.9]. Now uniform stability follows from A-IV,Thm.1.16.

In A-IV, Thm. 1.13 we have seen that a generator A of a stable semigroup satisfies necessarily $s(A) \leqq 0, \operatorname{Re} \lambda<0$ for all\\
$\lambda \in P_{\sigma}(A) \cup R_{\sigma}(A)$ and, by $\lambda R(\lambda, A) f=R(\lambda, A) A f+f$, that $\lim _{\lambda \rightarrow 0+} \mathrm{R}(\lambda, \mathrm{A}) \mathrm{g}$ exists for all $\mathrm{g} \in \operatorname{Im} \mathrm{A}$. For positive semigroups similar properties are even sufficient for stability.

Lemma 1.8. Let A be the generator of a positive semigroup $(T(t))_{t \gtrless 0}$ on a Banach lattice $E$ with $s(A) \leqq 0$. Given $f \in E_{+}$then $\lim _{\lambda \rightarrow 0^{+}} R(\lambda, A) \pounds$ exists if and only if $\lim _{t \rightarrow \infty} \int_{0}^{t} T(s) f \mathrm{ds}$ exists.

Proof. In view of C-III, Thm. 1. 2 we have for $\phi \in \mathrm{E}_{+}^{\prime}$ : $\lim _{t \rightarrow \infty}<\int_{0}^{t} \mathrm{~T}(s) \mathrm{f} \mathrm{ds}, \phi>=\sup _{t>0} \int_{0}^{t}\langle\mathrm{~T}(\mathrm{~s}) \mathrm{f}, \phi\rangle \mathrm{ds}=$ $=\sup _{t>0} \sup _{\lambda>0} \int_{0}^{t} e^{-\lambda s}<T(s) f, \phi>d s=\sup _{\lambda>0} \sup _{t>0} \int_{0}^{t} e^{-\lambda s}<\mathrm{T}(s) f, \phi>d s=$ $=\sup _{\lambda>0}\langle\mathrm{R}(\lambda, \mathrm{A}) \mathrm{f}, \phi\rangle=\lim _{\lambda+0}\left\langle\mathrm{R}(\lambda, \mathrm{A}) \mathrm{E}_{,} \phi\right\rangle$.\\
Thus either both limits exist with respect to o(E,E')-topology or none. Since both nets are monotonically increasing, the assertion follows from Dini's Theorem (see Schaefer (1974), II.Thm.5.9).

Proposition 1.9. Let $A$ be the generator of a positive, bounded semigroup $(T(t))_{t \geqslant 0}$ on a Banach lattice $E$. If there is a subset $D \subset E_{+}$which is total in $E$ such that $\lim _{\lambda \rightarrow 0+} R(\lambda, A) f$ exists for every $f \in D$, then $(T(t))_{t \times 0}$ is uniformly stable.

Proof. By Lenma $1.8 \quad \int_{0}^{\infty} \mathrm{T}(t) \mathrm{f}$ dt exists for every $f$ in the linear hull of $D$. But $D$ is total, $(T(t))_{t \geqq 0}$ is bounded and hence, by A-IV, Thm.1.16, uniformly stable.

Remark 1.10. If $A$ is the generator of a positive seraigroup, then for every $n \in \mathbb{N}, D\left(A^{n}\right)_{+}$and $D_{+}^{\infty}=\left(n_{n=0}^{\infty} D\left(A^{n}\right)\right)_{+}$are total subsets of $E$. This follows from $f \in D\left(A^{n}\right), f=R(\lambda, A)^{n} g=R(\lambda, A)^{n}\left(g_{1}-g_{2}\right)$ $=f_{1}-f_{2}$ where $f_{1}, f_{2} \in D\left(A^{n}\right)$ and Thm. 1.43 in Davies (1980).

In the rest of this section we discuss the long term behavior of the solutions of the inhomogeneous equation


\begin{equation*}
\dot{u}(t)=A u(t)+F(t), u(0)=u_{0} \in D(A) \tag{1,6}
\end{equation*}


where the forcing term $F(t)$ converges to some $\mathrm{f}_{0} \in \mathrm{E}$ as $t \rightarrow \infty$. In case that A generates a positive semigroup the assumption $' \omega(A)<0 '$, which is needed to prove the next proposition for arbitrary generators (see [Pazy (1983), Thm.4.4.4]), can be replaced by the 'stability' of the semigroup. We recall that some important generators as, for example, the Laplacian on $\mathrm{L}^{\mathrm{P}}\left(\mathbb{R}^{\mathrm{n}}\right), 1<\mathrm{p}<\infty$, generate positive, stable semigroups which are not uniformly exponentially stable. Therefore, the weakening of the assumptions on A mentioned above - i.e., replacing ' $\omega(A)<0^{\prime}$ by 'positive and stable' widens the class of equations (1.6) for which the following stability result is applicable. For additional results of this kind see Neubrander (1985b).

Proposition 1.11, Let $A$ be the generator of a positive, stable semigroup $(\mathrm{T}(t))_{t \geq 0}$ on a Banach lattice E . Let $F(\cdot)$ be a locally integrable function from $\mathbb{R}_{+}$into E . If there are $G(\cdot) \in \mathbb{C}_{0}\left(\mathbb{R}_{+}, \mathbb{R}_{+}\right)$, $f_{0} \in i m A$ and $g_{0} \in i m A_{+}$such that $\left|F(s)-f_{0}\right| s G(s) g_{0}$ for every $s \geqq 0$, then every mild solution $u(\cdot)$ of $(1.6)$ converges as $t \rightarrow \infty$ and $\lim _{t \rightarrow \infty} u(t)=-h$ where $h \in D(A)$ with $A h=-f_{0}$.

Proof, Recall that every solution of (1.6) satisfies


\begin{equation*}
u(t)=T(t) f+\int_{0}^{t} T(t-s) f_{0} d s+\int_{0}^{t} T(t-s)\left(F(s)-f_{0}\right) d s . \tag{1.7}
\end{equation*}


By the stability of the semigroup and $f \in D(A)$, the first term converges to zero as $t+\infty$. Since $f_{0} \epsilon$ im $A$, the second term converges to $h:=\int_{0}^{\infty} \mathrm{T}(\mathrm{s}) \mathrm{f}_{0} \mathrm{ds} \in \operatorname{imA}(\mathrm{A}-\mathrm{IV}, \mathrm{Thm} .1 .16)$ and $\mathrm{Ah}=-\mathrm{f}_{0}$. Define $H(s):=F(s)-f_{0}=H_{+}(s)-H_{-}(s)$. We have to show that $\int_{0}^{t} \mathrm{~T}(t-s) \mathrm{H}_{ \pm}(s) \mathrm{ds} \rightarrow 0$ as $t \rightarrow \infty$. Again, the assumption $g_{0} \in$ im A is equivalent to the existence of $\int_{0}^{\infty} \mathrm{T}(t) g_{0} \mathrm{dt}$. Choose\\
(i) a constant $M$ such that

$$
0 \leqq \mathrm{H}_{ \pm}(\mathrm{s}) \leqq \mathrm{H}_{+}(\mathrm{s})+\mathrm{H}_{-}(\mathrm{s})=|\mathrm{H}(\mathrm{~s})| \leqq \mathrm{G}(\mathrm{~s}) \mathrm{g}_{\circ} \leqq \mathrm{Mg}_{\circ}
$$

(ii) a constant $t^{\prime}$ such that $\left\|\int_{t}^{\infty}, T(s) g_{O} d s\right\| \leqq \varepsilon /(2 M)$ and $G(s) \leqq \varepsilon / 2\left\|\int_{0}^{\infty} T(s) g_{0} d s\right\|$ for every $s \geqq t^{\prime}$.\\
Then, for $t>2 t^{\prime}$,

$$
\begin{aligned}
& 0 \leqq \int_{0}^{t} T(t) H_{ \pm}(s) d s \leqq \int_{0}^{t} T(t) G(s) g_{O} d s \\
&=\int_{0}^{t^{\prime}} T(t) G(s) g_{0} d s+\int_{t^{\prime}}^{t} T(t) G(s) g_{O} d s \\
& \leqq M \int_{t-t^{\prime}}^{t} T(t) g_{0} d s+\varepsilon / 2\left\|\int_{0}^{\infty} T(t) g_{O} d s\right\|^{-1} \int_{0}^{t-t^{\prime}} T(t) g_{0} d s \\
& \leqq M \int_{t^{\prime}}^{\infty} T(t) g_{0} d s+\varepsilon / 2\left\|\int_{0}^{\infty} T(t) g_{0} d s\right\|^{-1} \int_{0}^{\infty} T(t) g_{0} d s . \\
& \text { Hence }\left\|\int_{0}^{t} T(t) H_{ \pm}(s) d s\right\| \leqq \varepsilon \text { for every } t>2 t^{\prime} .
\end{aligned}
$$

We conclude with a result similar to the previous proposition. Instead of uniform stability we now require $s(A)<0$ while the assumption on the forcing term is weaker than in Prop.1.11.

Proposition 1.12 . Let $(T(t))_{\ell \geq 0}$ be a positive semigroup with $\mathrm{s}(\mathrm{A})<0$. Assume that the forcing term F has values in $\mathrm{D}(\mathrm{A})$, that it is continuous with respect to the graph norm and that $f_{0}:=\|\cdot\|_{A}-1 i m_{t \rightarrow \infty} F(t)$ exists. Then for every solution $u(\cdot)$ of (1.6) we have $\lim _{t \rightarrow \infty} u(t)=-A^{-1} f_{0} \cdot$\\[0pt]
(Note, that the assumptions imply that (1.6) has a unique strong solution, see [Pazy (1983), Thm.4.2.4].)

Proof. The solution of (1.6) is given by\\
(1.8) $u(t)=T(t) u_{0}+\int_{0}^{t} T(s) f_{0} d s+\int_{0}^{t} T(s)\left(F(t-s)-f_{0}\right) d s$

The first term tends to zero by cor.1.4. The second term tends to $R(0, A) f_{0}=-A^{-1} f_{0}$ by $C-I I I, T h m .1 .2$. By assumption we have $\lim _{s \rightarrow \infty}\left\|A\left(F(s)-f_{0}\right)\right\|=0$ and from Thm.1.3 and $A-I V,(1.3)$ we deduce that $\|\mathrm{T}(\mathrm{s}) \mathrm{R}(0, \mathrm{~A})\| \leqq \mathrm{Me}^{-\varepsilon S}$ for $\mathrm{s} \geqq 0$ and suitable constants $\mathrm{M} \geqq 1$, $\varepsilon>0$. Thus for the third term we have

$$
\begin{aligned}
\left\|\int_{0}^{t} T(s)\left(F(t-s)-f_{0}\right) d s\right\| & \leq \int_{0}^{t}\|T(s) R(0, A)\|\left\|A\left(F(t-s)-f_{0}\right)\right\| d s= \\
& =\int_{0}^{t / 2} \ldots d s+\int_{t / 2}^{t} \ldots d s .
\end{aligned}
$$

The first integral can be estimated by\\
$\sup \left\{\left\|A\left(F(s)-f_{0}\right)\right\|: s \in\left[\frac{t}{2}, t\right]\right\} \cdot \int_{0}^{\infty} \mathrm{M} \cdot \mathrm{e}^{-E s}$ ds while the second integral can be estimated by $\left.\sup \left\|A\left(F(s)-f_{0}\right)\right\|: s \geq 0\right\} \cdot \int_{t / 2} \mathrm{~m} \cdot e^{-\varepsilon s} \mathrm{ds}$.\\
It follows that the third term in (1.8) tends to zero.

\section*{2. CONVERGENCE OF POSITIVE SEMIGROUPS \\
 by \\
 Günther Greiner and Rainer Nagel }
The considerations in this section are motivated by the following guidieline:

The asymptotic behavior of a strongly continuous semigroup $(T(t))_{t \geqslant 0}$ is determined by the (structure, location of the) spectrum $\sigma(A)$ of the generator $A$.

Unfortunately this principle does not hold in general, e.g., there are semigroups with spectral bound less than zero and growth bound greater than zero (see A-III,Ex.1.3\& 1.4). In order to prove results in the above direction we have to assume additional hypotheses on the semigroup. Positivity may serve to this purpose. For example, the norm convergence to zero, i.e. $\lim _{t \rightarrow \infty}\|\mathrm{~T}(t)\|=0$, for a positive semigroup on certain Banach lattices is characterized by the condition $s(A)<0$ (see Thm.1.1). Thus in this case the location of the spectrum determines the norm convergence of the semigroup.\\
Here we concentrate on the case $s(A)=0$. At first we observe that\\
$\lim _{t \rightarrow \infty} T(t)$ - if it exists in some operator topology - is always a projection $P$ onto the fixed space of $(T(t))_{t \geqslant 0}$ which coincides with the kernel of $A$. In case $P=0$ we have stability which was discussed in Sec.l . In this section we mainly consider the case $s(A)=0 \in \operatorname{P\sigma }(\mathrm{~A})$ and show that the symmetric structure of the boundary spectrum of the generator of a positive semigroup yields interesting results.

We begin our discussion by considering quasi-compact semigroups. Using the general results presented in Sec. 2 of $B-I V$ and the spectral theoretical result of chapter C-III we obtain the following.

Theorem 2.1. Let $(\mathrm{P}(t))_{t \geq 0}$ be a positive semigroup on a Banach lattice E which is bounded, quasi-compact and has spectral bound zero. Then there exists a positive projection $P$ of finite rank and suitable constants $\delta>0, \mathrm{M} \geq 1$ such that\\
(2.1) $\|T(t)-P\| \leq M \cdot e^{-\delta t} \quad$ for all $t \geqq 0$.

Proof. By Thm. 2.9 of B-IV the set $\{\lambda \in \sigma(A): \operatorname{Re} \lambda=0\}$ is finite and by Thm.2.10 of C-III imaginary additively cyclic. Thus it contains only the value $s(A)=0$. Then by $B-I V,(2.5)$ we have

$$
T(t)=\sum_{j=0}^{k-1} \frac{1}{j!} \cdot t^{j}{ }_{A}^{j} \circ P+R(t) \quad(t \geqq 0)
$$

where $P$ is the residue of $R(., A)$ at $0, k$ is the pole order and $\|R(t)\| \leqq M \cdot e^{-\delta t}$ for suitable constants $\delta>0, M \geqq 1$. Since we assumed that $(T(t))_{t \geq 0}$ is bounded, the pole order $k$ has to be 1 .

Before discussing a concrete example we formulate some remarks related to Theorem 2.1.

Remarks 2.2. (a) If one has a positive semigroup $T=(T(t))_{t \geqq 0}$ satisfying $\omega_{\text {ess }}(T)$ < $\omega(T)$ then the rescaled semigroup with $\tilde{\mathrm{T}}(\mathrm{t}):=\exp (-\omega(T)) \mathrm{T}(t)$ is quasi-compact and has spectral bound zero. In order to apply Theorem 2.1 we still need the boundedness of ( $\tilde{T}(t))_{t \geq 0}$ (see the following remarks).\\
(b) Without assuming boundedness of the semigroup one can conclude that $T(t)-\sum_{j=0}^{k-1} \frac{1}{j}!\cdot t{ }_{A}^{j}{ }_{\circ P}$ tends to zero exponentially.\\
(c) In the proof of Theorem 2.1 we saw that a quasi-compact semigroup of positive operators having spectral bound zero is bounded if and only if the pole order at zero is one. This is automatically true\\
whenever there exists a fjxed vector which is a quasi-interior point of $E_{+}$. Indeed, if $k$ is the order of the pole at $s(A)=0$ then we have $0 \neq A^{k-I_{P}}=\lim _{\lambda \rightarrow 0} \lambda^{k}(\lambda, A)$. Thus $A^{k-1} P$ is a positive operator. Assuming $k>1$ and denoting the quasi-interior fixed vector by $u$ we have $A u=0$ hence $A^{k-1} P u=P A^{k-1} u=0$. since $A^{k-1} P$ is positive it vanishes on the principal ideal generated by u . since this ideal is dense we obtain $A^{k-1} P=0$ which is a contradiction.\\
(d) If $T=(T(t))_{t \geqq 0}$ is an irreducible semigroup with $s(A)=0$, then quasi-compactness implies boundedness of $T$ (This follows from (c) and C-III, Prop.3.5). Moreover, in this case the projection $P$ has the form $\mathrm{P}=\phi 0 \mathrm{~h}$ where u is a quasi-interior point of $\mathrm{E}_{+}$and $\phi$ is a strictly positive linear form on E. This also is a consequence of C-III, Prop.3.5.\\
(e) If $T=(T(t))_{t \geq 0}$ is irreducible and eventually compact then the rescaled semigroup (exp(-w(T)t)T(t)) satisfies the assumptions of Thm.2.1. Indeed, by C-III, Thm.3.7 we know that $w(T)>-\infty$, while ${ }^{\omega}$ ess $(T)=-\infty$. It follows that the rescaled semigroup is quasi-compact hence (d) is applicable.

The following example has a biological background, and the semigroup considered describes the time evolution of an age-structured population. For more details we refer to Greiner (1984a) or Webb (1984).

Example 2.3. On the Banach lattice $E=L^{1}([0, \infty)$ ) we consider the operator A defined by\\
$\begin{aligned} \text { Af }: & =-f^{\prime}-\mu f \text { with domain } \\ \text { (2.2) } D(A) & :=\left\{f \in E: E \text { absolutely continuous, } f^{\prime} \in E,\right.\end{aligned}$ $\left.f(0)=\int_{0}^{\infty} \beta(a) f(a) d a\right\}$.

Here we assume that $\mu$ and $\beta$ are positive, measurable, bounded functions on $[0, \infty)$. One can show that $A$ generates a strongly continuous semigroup $T$ of positive operators. Assuming that $\mu(\infty):=\lim _{a \rightarrow \infty} \mu(a)$ exists we obtain $\omega_{\text {ess }}(T)=-\mu(\infty)$. We suppose in addition that $\beta$ and $\mu$ satisfy\\
(2.3) $\int_{0}^{\infty} \beta(a)\left(\exp \left(-\int_{0}^{a} \mu(x) d x\right)\right) d a=1$ and $\mu(\infty)>0$.

The function $h$ with $h(a):=\exp \left(-\int_{0}^{a} \mu(s) d s\right)$ is differentiable, $h \in E$ and $h^{\prime}=-\mu h$. Moreover, (2.3) implies $\int_{0}^{\infty} \beta(a) h(a) d a=1=$ $h(0)$. Thus $h \in D(A)$ and $A h=0$. It follows that $s(A)=0$. Indeed, since s(A) is a pole of the resolvent there exists a positive eigenvector $w$ of $A$, corresponding to $s(A)$. Since $h$ is\\
strictly positive we have $\langle\mathrm{h}, \mathrm{w}\rangle>0$ hence $s(A)<h, w\rangle=\left\langle h, A^{\prime} w\right\rangle=$ $<A h, w>=0$ which implies $s(A)=0$.\\
Consequently the semigroup generated by $A$ satisfies all the assumptions of Thm.2.1 provided that $\mu$ and $B$ satisfy (2.3). (The boundedness of the semigroup follows from Rem.2.2(c)). It is not difficult to see that (up to a constant) $h$ is the unique eigenfunction of $A$ corresponding to 0 . Thus the projection $P$ has the form $P=v \otimes n$ for a suitable positive $v \in L^{\infty}([0, \infty))$.\\
For more general generators of the type (2.2) we refer to C-IV, Section 3.

Clearly, quasi-compactness was essential in the above example as well as in Thm.2.1. For spaces $C_{0}(X)$ we proved in B-IV,Thm.2.12 that Doeblin's condition is sufficient for quasi-compactness. Actually this is true in $\mathrm{L}^{\mathrm{P}}$-spaces with $1<\mathrm{p}<\infty$ as well. We quote the result from Lotz (1986).

Proposition 2.4. Let $(T(t))_{t \geqq 0}$ be a bounded positive semigroup on $E=\mathrm{L}^{\mathrm{P}}(\mu), 1<\mathrm{p}<\infty$.\\
Assume that there exist $t_{0} \geqq 0, \phi \in E_{+}^{\prime}, b<1$ such that\\
(2.4) $\left\|T\left(t_{0}\right) f\right\| \leqq<f, \phi>+b\|f\|$ for all $\mathrm{f} \geqq 0$.

Then (T(t) ${ }_{t \geqq 0}$ is quasi-compact.

In the following result we replace quasi-compactness by eventual norm-continuity of the semigroup.

Theorem 2.5. Let $T=\left(T(t){ }_{t \geq 0}\right.$ be a bounded, eventually normcontinuous positive semigroup with gererator A on a reflexive Banach lattice E , Then $\mathrm{Pf}:=\lim _{t \rightarrow \infty} T(t) \mathrm{f}$ exists for every $f \in \mathrm{E}$. P is a positive projection onto the fixed space $F i x(T)=$ ker $A$.

Proof. In view of Thm.1.5 it suffices to consider the case $s(A)=0 \in P \sigma(A)$. We define $F:=\left\{f \in E: \lim _{t \rightarrow \infty} T(t) f\right.$ exists\}. $F$ is closed since $(T(t))_{t \geq 0}$ is bounded and obviously ker $A \subset F$. Since $\sigma(A) n i \mathbb{R}$ is cyclic and bounded (see C-III,Thm.2.10 and A-II, Thm. 1.20 resp.) we have $\sigma(A) \cap i \mathbb{R}=\{0\}$. since the spectral mapping theorem holds (cf. A-III, Thm. 6. 6) we conclude $\sigma(T(t)) \cap T=\{1\}$ for all $t \geqq 0$. Then (1.4) implies $\lim _{n \rightarrow \infty}\|T(n)-T(n+1)\|=0$ hence $\lim _{t \rightarrow \infty}\|T(t)-T(t+1)\|=0$. Take $f=g-T(1) g$. Then $\|T(t) f\|=$ $\|T(t) g-T(t+1) g\| \leq\|T(t)-T(t+1)\|\|g\|$ implies $\lim _{t \rightarrow \infty} T(t) f=0$. Thus\\
(2.5) $\lim _{t \rightarrow \infty} T(t) f=0$ for every $f \in \operatorname{im}(I d-T(1))$.

That is, im(Id - T(1)) CF. Since ker $A=n_{t \geqq 0} \operatorname{ker}(I d-T(t))=$ ker(Id - T(1)) (cf. A-III, Cor.6.4) we have\\
$\operatorname{im}(I d-T(1))+\operatorname{ker}(I d-T(1)) \subset F$.\\
Since power bounded operators on a reflexive Banach space are mean ergodic (e.g., see Krengel (1985), Chap.2,Thm.1.2) we obtain that im(Id - T(1)) + ker(Id - T(1)) is dense in $E$, hence $F=E$.

Strong convergence of the semigroup $T=(T(t))_{t \geqslant 0}$ implies strong convergence of the Césaro means $C(t) f:=\frac{1}{t} \cdot \int_{0}^{t} T(s) f d s, f \in E$ which (by definition) is mean ergodicity of the semigroup $T$ (see Davies (1980), Chap.5.1). On the other hand an inspection of the proof of Thm. 2.5 shows that reflexivity of the underlying space can be replaced by the assumption that $T$ is a mean ergodic semigroup.\\
This remark also shows where to look for examples of semigroups not converging as $t \rightarrow \infty$ : Consider the positive contraction $R$ defined by $(R f)(x):=f(x+1)$ on $\left.E=L^{1}(R)\right)$. Then $T(t):=e^{t(R-I d)}$ defines a positive norm-continuous semigroup on E . Since $\operatorname{ker}(R-I d)=F i x R=\{0\}$ but $\|T(t) f\|=e^{-t} \sum_{n=0}^{\infty}\left\|R^{n} f\right\| t^{n} / n!=\|f\|>0$ for every $0<f \in E$ we see that $\lim _{t \rightarrow \infty} T(t)$ does not exist for the strong operator topology.\\
Finally we remark that in Thm.2.5 'eventual norm-continuity' is crucial as well. This can be seen by considering the translation (semi-) groups on $\mathrm{L}^{\mathrm{P}}(\mathbb{R})$.

In the next few results we study semigroups which are not necessarily eventually norm-continuous, but restrict our attention to positive semigroups on $\mathrm{L}^{\mathrm{p}}$-spaces $(1 \leqq \mathrm{p}<\infty)$. The essential tool will be the following '0-2 law' which we quote from Greiner (1982), Thm.3.7.

If $(X, \Sigma, \mu)$ is a measure space and $(T(t))_{t \geqq 0}$ is a positive semigroup on $L^{P}(\mu)$ then we call a subset $C \in \Sigma \quad(T(t))$-invariant if the principal ideal generated by the characteristic function $1_{C}$ is (T(t))-invariant in the usual sense.

Theorem 2.6. Let $(T(t))_{t \geqq 0}$ be a positive contraction semigroup on $\mathrm{L}^{\mathrm{P}}(\mu), 1 \leqq \mathrm{p}<\infty$, and assume that there exists a strictly positive fixed function $e \in$ ker $A$. Then the following holds:\\
(a) For every $\tau>0$ there exists a disjoint decomposition $\mathrm{X}=\mathrm{X}_{0} \cup \mathrm{x}_{2}$ into ( $\mathrm{T}(\mathrm{t})$ )-invariant measurable subsets such that\\
(0) $|T(t)-T(t+\tau)| e_{0}+0 \quad$ for $e_{0}=e \cdot 1_{X_{0}}$ as $t \rightarrow \infty$,\\
(2) $|T(t)-T(t+\tau)| e_{2}=2 e_{2}$ for $e_{2}=e \cdot 1_{X_{2}}$ and all $t \geq 0$.\\
(b) In case the semigroup is irreducible then for every $\tau>0$ one has the alternative\\
(0) $|T(t)-T(t+\tau)| e+0$ as $t \rightarrow \infty$ or\\
(2) $|T(t)-T(t+\tau)| e=2 e \quad$ for all $t \geqq 0$.

This '0-2 law' can be used in order to obtain results on convergence of positive semigroups.

Corollary 2.7. Assume that (in addition to the assumptions made in Thm.2.6) Po(A) $\cap i R=\{0\}$. If we decompose $\mathrm{X}=\mathrm{x}_{\mathrm{o}} \mathrm{U} \mathrm{X}_{2}$ for some $\tau>0$ according to assertion (a), then $\lim _{t \rightarrow \infty} T(t) f$ exists for every $f \in L^{\mathrm{P}}(\mu)$ vanishing $\mu-\mathrm{a} . \mathrm{e} . \mathrm{on} \mathrm{x}_{2}$.

Proof. From $T(t) e_{j} \leqq T(t) e=e$ we obtain $T(t) e_{j} \leqq e_{j}$ since $x_{0}$ and $x_{2}$ are $\left(T(t)\right.$-invariant. Then $T(t) e_{0}+T(t) e_{2}=T(t) e=e$ implies $T(t) e_{j}=e_{j}(j=0,2)$. Thus we can assume $X=X_{0}, e=e_{0}$. Given $g \in I^{P}(\mu)$ such that $|g| \leqq e$ we have\\
$|T(t)(I d-T(T)) g| \leqq|T(t)-T(t+\tau)| e+0$ for $t \rightarrow \infty$. Since $\left\{g \in L^{p}(\mu):|g| \leqq e\right\}$ is a total subset of $E$ (e is strictly positive) and $(T(t))_{t \geqq 0}$ is bounded we conclude\\
(2.6) $\lim _{t \rightarrow \infty} T(t) f=0$ for every $f \in \overline{\operatorname{im}(I d-T(T))}$.

The assumption $\operatorname{P\sigma }(A) \cap i R=\{0\}$ implies ker(Id - T( $\tau)$ ) ker $A \quad(c f$. A-III, Cor. 6.4), hence we have convergence on ker(Id -T( $\tau$ ) . Since $T(\tau)$ is a contraction on a reflexive Banach space we have $\mathrm{L}^{\mathrm{P}}(\mu)=\operatorname{ker}(\mathrm{Id}-\mathrm{T}(\tau)$ ) $\oplus \overline{\operatorname{im(Id-T(T))}}$ (see Krengel(1985), p.74) which finally proves the convergence on the whole space.

Typical examples for which Thm. 2.6 and Cor. 2.7 can be applied occur in the theory of stochastic processes (see also B-IV,Ex.2.6). We briefly describe this situation and remind that in this context the sets $x_{0}$ and $\mathrm{X}_{2}$ have a probabilistic meaning (see Greiner-Nagel (1982) or the Supplement in Krengel (1985)).

Example 2.8. Let $X$ be a set and $\Sigma$ be a a-algebra of subsets of $X$. We consider a Markov transition function $\left(P_{t}\right)_{t \geqslant 0}$ on $(x, \Sigma)$, i.e. each $P_{t}$ is a real-valued function on $X \times \Sigma$ such that\\
(2.7a) $P_{t}(x,$.$) is a probability measure for x \in X, t>0$;\\
(2.7b) $P_{t}(., C)$ is a measurable function for $C \in \Sigma, t>0$;\\
(2.7c) $P_{t+s}(x, c)=\int_{K} P_{S}(y, C) P_{t}(x, d y)$ for $a l l s, t>0, x \in K, C \in \varepsilon$. We assume that ( $\mathrm{P}_{t}$ ) possesses an invariant probability measure $\mu$, i.e. we assume\\
(2.7a) $\mu(C)=\int P_{t}(x, C) d \mu(x)$ for every $C \in \Sigma$.

Finally we assume that the following continuity condition holds true.\\
(2.7e) For every $C \in \Sigma$ one has $\lim _{t \rightarrow 0} P_{t}(x, C)=1_{C}(x) \mu-a . e$. .

Given $h \in L^{1}(\mu)$ we define a measure $p_{t} h$ on $\Sigma$ by\\
$P_{t} h(C):=\int P_{t}(x, C) h(x) d \mu(x)$. In case $\mu(C)=0$ then by (2.7d) $P_{t}(x, c)=0$ u-a.e. on $x$ hence $P_{t} h(C)=0$. That is, $P_{t} h$ is absolutely continuous with respect to $\mu$. By the Radon-Nikodym theorem $\mathrm{P}_{t} \mathrm{~h}$ has an integrable density with respect to $\mu$. We define $T(t) h$ to be this density (which is unique as an element of $\mathrm{I}^{{ }^{I}}(\mu)$ ). Thus for $h \in L^{1}(\mu), C \in \Sigma$ we have\\
(2.8) $\int_{C}(T(t) h)(x) d \mu(x)=\int P_{t}(x, C) h(x) d_{\mu}(x)$ for all $C \in \Sigma$.

It is not difficult to see that $\mathrm{T}(\mathrm{t})$ is a positive linear contraction on $L^{1}(\mu)$. We have $T(t)^{1}{ }_{X}=1_{X}$ and $T(t) 1_{X}=1_{X}$ for all $t \geqq 0$ and $T(t) T(s)=T(t+s)$ for $t, s \geqq 0$. This follows from (2.7a), (2.7d) and (2.7c) respectively. Moreover (2.7e) implies strong continuity of the semigroup $(T(t))_{t \geqslant 0}$. In fact by Prop.1.23 of Davies (1980) we only have to show weak continuity at $t=0$. Since the characteristic functions are total in $\mathrm{L}^{\infty}(\mu)$ this is true provided that $\lim _{t \rightarrow 0}\left\langle T(t) h, 1_{C}\right\rangle=\left\langle h, 1_{C}\right\rangle$ for $h \in L^{1}(\mu), C \in \varepsilon$. Given $h \in L^{1}(\mu)$, then by $(2.7 e) \lim _{t \rightarrow 0} P_{t}(x, c) h(x)=I_{C}(x) h(x)$ $\mu-a . e$. By Lebesgue's Theorem $\left\langle T(t) h, 1_{C}\right\rangle=\int P_{t}(x, c) h(x) d \mu(x)$ tends to $\int 1_{C}(x) h(x) a_{\mu}(x)=\left\langle h, I_{C}\right\rangle$ as $t+0$ and we have weak hence strong continuity.\\
Therefore a Markov transition function satisfying all the assumptions of (2.7) induces a strongly continuous semigroup on $\mathrm{L}^{1}(\mu)$, and by interpolation on $\mathrm{L}^{\mathrm{P}}(\mu)$, which satisfies the hypotheses of Thm.2.6.

In the following corollaries of Thm. 2.6 we give criteria which ensure convergence on the whole space. In view of Cor. 2.7 it is enough to show $\mathrm{X}_{2}=\varnothing$.

Corollary 2.9. Let $(T(t))_{t \geq 0}$ be a positive semigroup of contractions on the Banach lattice $\mathrm{L}^{1}(\mu)$ and assume that there exists a strictly positive eigenfunction $e \in \operatorname{ker} A$.

If $(T(t))_{t \geqq 0}$ is eventually norm-continuous then $\lim _{t+\infty} T(t) f$ exists for every $f \in I^{1}(\mu)$.

Proof. Since the semigroup is positive and eventually norm-continuous its boundary spectrum is cyclic and bounded, i.e. we have $\operatorname{Po}(A) \cap i R=\{0\}$. Moreover there exist $t_{0}>0$ and $\tau>0$ such that $\left\|T\left(t_{0}\right)-T\left(t_{0}+\tau\right)\right\|<1$.\\
For bounded linear operators $s \in L\left(L^{1}\right)$ one has $\|s\|=\||| |=$ (see IV,Thm.1.5 of Schaefer (1974)) hence $\left\|\left|T\left(t_{0}\right)-T\left(t_{0}+\tau\right)\right| f\right\|<\|f\|$ for every $f \in L^{I}(\mu), f \neq 0$. This shows that condition (2) of Thm.2.6(a) can be true only when $e_{2}=0$, i.e., $\mathrm{x}_{2}=\emptyset$.

Corollary 2.10. Let $(\mathrm{T}(t))_{t \geqq 0}$ be an irreducible semigroup on $\mathrm{L}^{\mathrm{P}}(\mu)$ satisfying the assumptions of Thm.2.6.\\
If $\operatorname{Po}(A) \cap i R=\{0\}$ and if there exist $0 \leqq r<s$, such that $\inf \{T(r), T(s)\}>0$ then there exists a strictly positive function $h \in L^{q}(\mu) \quad\left(P^{-1}+q^{-1}=1\right)$ such that $\lim _{t \rightarrow \infty} T(t) f=<f, h>e$ for every $f \in L^{P}(\mu)$.

Proof. Since inf\{T(r),T(s)\}>0 we have (inf\{T(r),T(s)\})e > 0 for the strictly positive fixed vector e . Since the regular operators on $L^{P}(\mu)$ form a vector lattice it follows by [schaefer (1974), II. 1.4, Formula (5) \& (5') ] that $|T(r)-T(s)| e=T(r) e+T(s) e-$ $2(i n f\{T(r), T(s)\}) e<2 e$. Consequently the first alternative of Thm.2.6(b) holds true with $\tau:=s-r$. Equivalently, we have $\mathrm{x}_{2}=\phi$ and by Cor.2.7 Pf := $\lim _{t \rightarrow \infty} T(t) f$ exists for every $f \in L^{P}(\mu)$. The limit $P$ is a positive projection, satisfying $P T(t)=T(t) P=P$ for all $t \geqq 0$. It follows that im $P \subset$ ker $A$ and im $P^{\prime} \subset$ ker $A^{\prime}$. Since $P \neq 0 \quad$ ( $\mathrm{Pe}=\mathrm{e})$ we conclude that ker $\mathrm{A}^{\prime}$ contains positive elements. Now C-III, Prop.3.5(a)-(c) implies that $p$ has the form $P=h$ for a strictly positive function $h \in L^{q}(\mu)=\left(L^{p}(\mu)\right)^{\prime}$.

In a last corollary we consider the case where one operator $T\left(t_{0}\right)$ is a kernel operator, i.e., $T\left(t_{0}\right)$ is induced by a $\mu \otimes \mu$-measurable kernel on $X \times X$. The corollary is of particular interest for semigroups on\\
\includegraphics[max width=\textwidth]{2024_12_23_c6487cc0859199a15bd9g-359} operator. For a precise definition and fundamental properties of kernel operators we refer to Sec.IV. 9 of Schaefer (1974) or Chap. 13 of Zaanen (1983). In particular we recall that the restriction of a kernel operator to a sublattice is again a kernel operator and that\\
the identity on $L^{P}(\mu), I \leqq P<\infty$, is a kernel operator if and only if the measure space $(X, \Sigma, \mu)$ is purely atomic, i.e. $L^{P}(\mu) \cong \ell_{I}^{p}$ for some index set I . Moreover, from Axmann (1980) we quote the following result (see satz 3.5 1.c.):\\
(2.9) If $T$ is an irreducible kernel operator then inf\{ $\left.\mathrm{T}^{\mathrm{n}}, \mathrm{T}^{\mathrm{m}}\right\}>0$ for some $n, m \in \mathbb{N}, \mathrm{n} \neq \mathrm{m}$.

Corollary 2.11. Let $T=(T(t))_{t \geq 0}$ be a semigroup on $L^{P}(\mu)$ satisfying the assumptions of Thm. 2.6 and assume that one operator $T\left(t_{0}\right)$ is a kernel operator. Then $\lim _{t \rightarrow \infty} T(t) f$ exists for every $f \in L^{p}(\mu)$.

Proof. First we note that ker $A=F i x(T)$ is a sublatice of $\mathrm{L}^{\mathrm{P}}(\mu)$, hence is itself an $\mathrm{L}^{\mathrm{P}}$-space. Since $T\left(t_{0}^{\prime} /\right.$ ker $A=I d$ we conclude that $\operatorname{ker} A \cong \ell_{I}$. Thus $L^{P}(\mu)$ contains an orthogonal system $\left\{e_{j} \in\right.$ ker $A$ : $j \in I\}$ of atoms such that $\sup _{j \in I_{j}}=e$. The closed principal ideal $E_{j}$ generated by $e_{j}$ in $L^{P}(\mu)$ is $(T(t))$-invariant and the restriction of $(\mathrm{T}(t))_{t \geq 0}$ to this ideal yields an irreducible semigroup $\left(T_{j}(t)\right)_{t \geq 0}$ having generator $A_{j}$. From C-III, Cor. 3.9 we conclude that $\operatorname{Po}\left(A_{j}\right) \cap i R=\{0\}$. It follows that $T_{j}\left(t_{0}\right)$ is an irreducible kernel operator hence by (2.9) all the assumptions of Cor. 2.10 are satisfied. Thus we have convergence on the the principal ideal $\mathrm{E}_{j}$. Since the semigroup is bounded and the union of these ideals is total in $L^{\mathrm{P}}(\mu)$ we have convergence on the whole space.

In all the results obtained so far we had to show or to assume that Po(A)niR $=\{0\}$. This is not surprising since for an eigenvector $g \in E$ corresponding to $i \alpha \neq 0, \alpha \in \mathbb{R}$, we have $T(t) g=e^{i \alpha t} g$ and $\lim _{t \rightarrow \infty} \mathrm{~T}(t) \mathrm{g}$ does not exist. Nevertheless in some cases with Po(A) niR $\neq\{0\}$ one can describe the asymptotic behavior of $(\mathrm{T}(t)\}$ t ${ }_{t \geq 0}$ for large $t$. Instead of convergence to an equilibrium point one obtains that $T(t) f$ 'converges to a periodic function' .

To that purpose we consider a bounded, irreducible semigroup $T=$ ( $T(t))_{t \geqq 0}$ of positive operators on some Banach lattice $E$ having order continuous norm. Under the assumption that the spectral bound $s(A)=0$ is a pole of the resolvent we can apply Theorem 3.12 of Chapter C-III. In particular, if 0 is not the only point in the boundary spectrum $\sigma(A) n i R$ we obtain that

$$
\sigma(A) \cap i R=\operatorname{Po}(A) \cap i R=i \alpha R \text { for some } 0<\alpha \in \mathbb{R} \text {. }
$$

Therefore the assumptions of C-III, Thm.3.8 are satisfied and formula C-III,(3.13) implies\\
(2.10) $\rho(A)=\rho(A)+i \alpha \mathbb{Z}$ and $\|R(\lambda, A)\|=\|R(\lambda+i \alpha k, A)\|$\\
for $\lambda \in \rho(A), k \in \mathbb{Z}$.\\
Since 0 was supposed to be a pole of the resolvent we can decompose

$$
\sigma(A)=\sigma_{1} \cup \sigma_{2}
$$

where $\sigma_{1}=i \alpha Z, 0<\alpha \in \mathbb{R}$, and sup\{Re $\left.: \lambda \in \sigma_{2}\right\}<0$. Moreover, for small $\varepsilon>0,\|R(-\varepsilon+i \lambda, A)\|$ is uniformly bounded for $\lambda \in \mathbb{R}$. Next, we construct a spectral decomposition of $E$ and $T$ corresponding to $\sigma_{1}$ and $\sigma_{2}$ (compare A-III, Sec. 3 ). Since 0 is an eigenvalue of $A$ it follows that $T$ has a quasiinterior fixed point $h \in E_{+}$(use C-III,Prop.3.5(a)). Hence, $\{T(t) f: t \geq 0\}$ is contained in the weakly compact (see $C-I, S e c .5$ ) order interval [-h,h] whenever $|f| \leqq h$. Since $h$ is a quasiinterior point and $T$ is bounded it follows that $T$ is relatively compact for the weak operator topology on $L(E)$. Therefore the Jacobs-DeLeeuw-Glicksberg Splitting Theorem (see Krengel (1985), Chap.2,Thm.4.4 and 4.5) can be applied to (the weak closure of) T and we obtain a projection $Q \in L(E)$ onto the closed subspace $E_{1}$ generated by the eigenvectors $h_{k}$ of $A$ corresponding to the eigenvalues $i \alpha k, k \in \mathbb{Z}$. Clearly, $Q$ splits the semigroup $T$ into the restricted semigroups $T_{1}$ on $E_{1}:=Q E$ and $T_{2}$ on $E_{2}:=$ ker $Q$. We first describe $T_{1}$ in more detail.\\
The projection $Q$ is positive as an element of the weak closure of $T$ and even strictly positive by the irreducibilitiy of $T$. Its range $E_{1}$ is a closed sublattice of $E$ (use Schaefer (1974), Prop.III.11.5) on which the semigroup $T_{1}$ is periodic, irreducible and positive. In fact, $T(2 \pi / \alpha) f=f$ for every $f=h_{k}, k \in \mathbf{Z}$, and hence for every $f \in E_{1}$, while irreducibility and positivity are inherited from $T$. It now follows from A-III, Lemma 5.2 that the generator $\mathrm{A}_{1}={ }^{\mathrm{A}} \mid \mathrm{E}_{1}$ of $T_{1}$ has spectrum $\sigma\left(A_{1}\right)=i \alpha Z$. Moreover in view of $A-I I$, Prop. 5.2 and Cor.5.3(ii) we have $\sigma\left(A_{2}\right)=\sigma(A)$ I $i \alpha Z$. Therefore the decomposition $\mathrm{E}=\mathrm{E}_{1} \oplus \mathrm{E}_{2}$ is a spectral decomposition corresponding to $\sigma_{1}$ and $\sigma_{2}$. This proves the first part of the following lemma.

Lemma 2.12. Under the above assumptions there exists a positive projection $Q$ with range $E_{1}:=Q E$ and kernel $E_{2}:=Q^{-1}(0)$ such that the following holds:\\
(a) $\mathrm{E}=\mathrm{E}_{1} \oplus \mathrm{E}_{2}, \mathrm{~T}=\mathrm{T}_{1} \oplus \mathrm{~T}_{2}$ and $\mathrm{A}=\mathrm{A}_{1} \oplus \mathrm{~A}_{2}$ is a spectral decomposition corresponding to the decomposition $\sigma(\mathrm{A})=\sigma_{1} \cup \sigma_{2}$ where $\sigma_{1}=\sigma\left(A_{1}\right)=i \alpha Z$ and $\sigma_{2}=\sigma\left(A_{2}\right)=\sigma(A) \backslash i \alpha \mathbb{Z}$.\\
(b) $s\left(A_{2}\right)<0$ and $\left\|R\left(\lambda, A_{2}\right)\right\|$ is uniformly bounded in each semiplane $\left\{\lambda \in \mathbb{C}, \operatorname{Re} \lambda>s\left(A_{2}\right)+\varepsilon\right\}$ with $\varepsilon>0$.\\
(c) $\mathrm{E}_{1}$ is a closed sublattice of E and $T_{1}$ is a periodic, irreducible, positive semigroup on $E_{1}$. In particular, ( $E_{1}, T_{1}$ ) is isomorphic to $\left(L, R_{\tau}(t)\right)$ where $L$ is a function lattice between $C(\Gamma)$ and $I^{l}(\Gamma)$ and $R_{\tau}(t)$ is the rotation group with period $\tau=2 \pi / \alpha$.

Proof. (a) has been derived above while (b) follows immediately from (2.10). The properties of $T_{1}$ mentioned in (c) have been stated above. Hence the representation of $T_{1}$ as a rotation group follows from C-III, Cor.3.9.

For Hilbert spaces $L^{2}(\mu)$ property (b) of the above lemma and A-III, Cor.7.11 imply that the growth bound $\omega\left(A_{2}\right)$ is less than zero. Therefore we obtain the following result on the asymptotic behavior of $T$.

Proposition 2.13. Let $T=(T(t))_{t \geqq 0}$ be a bounded, irreducible, positive semigroup on a Hilbert lattice $E=\mathrm{L}^{2}(\mu)$. Assume that $s(A)=0$ is a pole of the resolvent of the generator $A$ and that i $\alpha \in \sigma(A)$ for some $0 \neq \alpha \in \mathbb{R}$. Then $T$ behaves asymptotically as the rotation group $\left(R_{\tau}(t)\right)_{t \geqq 0}$ with period $\tau=2 \pi n / \alpha$ for some $n \in N$ on $L^{2}(\Gamma)$.\\
More precisely, we can identify $\mathrm{L}^{2}(\Gamma)$ with a sublattice of E, which is the range of a strictly positive projection $Q$ and we find constants $\varepsilon>0$ and $M \geqq 1$ such that for every $f \in E$ we have (2.11) $\left\|T(t) f-R_{\tau}(t) g\right\| \leqq M e^{-\varepsilon t}\|f\|$ for every $t \geqq 0$ where $g:=Q f$. For $\mathrm{L}^{\mathrm{p}}$-spaces the analogous statement can be shown only for a weaker type of convergence. The proof of this result uses interpolation for operators, mainly the Riesz Convexity Theorem (see the remarks preceding Cor.1.2).

Theorem 2.14. Let $T=(T(t))_{t \geqq 0}$ be a bounded, irreducible positive semigroup on a Banach lattice $\mathrm{E}=\mathrm{L}^{\mathrm{P}}(\mu), 1 \leqq \mathrm{p}<\infty$. Assume that $s(A)=0$ is a pole of the resolvent of the generator $A$ and that $i \alpha \in \sigma(A)$ for some $0 \neq \alpha \in \mathbb{R}$. Then $T$ behaves asymptotically as the rotation group $\left(R_{\tau}(t)\right)_{t \geqq 0}$ with period $\tau>0$ on $L^{\mathrm{P}}(\Gamma)$, i.e.., we can identify $\mathrm{L}^{\mathrm{P}}(\Gamma)$ with a sublattice of E such that for every $f \in E$ there exists $g \in L^{P}(\Gamma)$ satisfying


\begin{equation*}
\lim _{t \rightarrow \infty}\left\|T(t) f-R_{\tau}(t) g\right\|=0 \tag{2.12}
\end{equation*}


Proof. We only consider $1 \leqq p<2$. The assertion for $p>2$ then follows by duality while $\mathrm{p}=2$ was treated in Prop.2.13.\\
At first we observe that without loss of generality we may assume that $\mu$ is a probability measure and that $T(t) 1=1$ for every $t \geqq 0$. In fact, the assumptions imply that $\mathrm{T}(\mathrm{t}) \mathrm{h}=\mathrm{h}$ for some $\mathrm{h} \gg \mathrm{p}$, $\|h\|_{p}=1$. We consider the measure $v$ which has the density $h^{p}$ with respect to $\mu$. Then $v$ is a probability measure and $\mathrm{M}: \mathrm{L}^{\mathrm{P}}(\nu) \rightarrow \mathrm{L}^{\mathrm{P}}(\mu)$; defined by $\mathrm{Mh}:=\mathrm{hf}$, is an isometric lattice isomorphism of $\mathrm{L}^{\mathrm{P}}(\nu)$ onto $\mathrm{L}^{\mathrm{p}}(\mu)$. The semigroup defined by $\tilde{T}(t):=M^{-1} \mathrm{~T}(t) \mathrm{M}$ possesses the same properties as (T(t)) and satisfies $\tilde{T}(t) 1=1$ for $t \geq 0$.\\
Now the properties $T(t) I=1, T(t) \geqq 0$ imply that $L^{\infty}(\mu)$ is an invariant subspace for every operator $T(t)$ which is contractive with respect to the $\mathrm{L}^{\infty}$-norm. The Riesz Convexity Theorem [Dunford-Schwartz (1958), VI.10.11] then implies that by restricting the semigroup (T(t)) to $\mathrm{L}^{\mathrm{q}}(\mu) \quad(\mathrm{p}<\mathrm{q}<\infty)$ we obtain a strongly continuous semi-\\
\includegraphics[max width=\textwidth]{2024_12_23_c6487cc0859199a15bd9g-363} $t \geqq 0, q \geqq p$.\\
Let $A_{q}$ be the generator of $\left(T_{q}(t)\right)$. In order to apply Prop.2.13 we have to show that 0 is a pole of the resolvent of $A_{2}$. Denoting the residue of $\mathrm{R}(\ldots, \mathrm{A})$ at 0 by P then $\mathrm{P}=\mathrm{h}$ fl for a suitable $h \in\left(L^{P}(\mu)\right)^{\prime}$. Since $\left(L^{P}(\mu)\right)^{\prime} C\left(L^{2}(\mu)\right)^{\prime}, P$ can also be considered as bounded operator on $\mathrm{L}^{2}(\mu)$. We denote it by $\mathrm{P}_{2^{*}}$. From $A P=P A=0$ it follows that

$$
\begin{aligned}
& (R(1, A)(I d-P))^{n}=R(1, A)^{n}-P \quad(n \in \mathbb{N}) \quad \text { and } \\
& \left(R\left(1, A_{2}\right)\left(I d-P_{2}\right)\right)^{n}=R\left(1, A_{2}\right)^{n}-P_{2} \quad(n \in \mathbb{N})
\end{aligned}
$$

The Riesz Convexity Theorem yields the following estimate for the operator norm:

$$
\begin{aligned}
\left\|R\left(1, A_{2}\right)^{n}-P_{2}\right\| & \leqq\left\|R(1, A)^{n}-P\right\|^{2 / P}\left\|R(1, A)_{\infty}^{n}-P_{\infty}\right\|^{1-2 / p} \\
& \leqq\left\|R(1, A)^{n}-P\right\|^{2 / p}\left(1+\left\|P_{\infty}\right\|\right)^{1-2 / p}
\end{aligned}
$$

Since 0 is a pole with residue $P$, the spectral radius of the operator $R(1, A)(1-P)$ is less than 1 . Thus for the right hand side of the inequality tends to 0 as $n \rightarrow \infty$. It follows that $r_{\text {ess }}\left(R\left(1, A_{2}\right)\right)<1$, hence 1 is a pole of the resolvent of $R\left(1, A_{2}\right)$, or equivalently, 0 is a pole of $\mathrm{R}\left(., \mathrm{A}_{2}\right.$ ) (see A-III, Prop.2.5).\\
Now we can apply Prop. 2.13 and obtain a projection $Q$ such that $\lim _{t \rightarrow \infty}\left\|T(t) f-R_{\tau}(t) \circ Q f\right\|_{2}=0$ for every $f \in L^{2}(\mu)$. On order intervals of $L^{\infty}(\mu)$ both, $L^{P}$ and $L^{2}$-norm induce the same topology (see

Schaefer (1974), V.8.3), hence $\lim _{t+\infty}\left\|T(t) f-R_{T}(t) \circ Q\right\|_{p}=0$ for every $f \in L^{\infty}(\mu)$. Since $(T(t))$ is bounded we finally obtain convergence in the $L^{\mathrm{p}}$-norm for every $f \in \mathrm{~L}^{\mathrm{P}}(\mu)$.

We give an example for the situation described in Thm.2.14. The equation we consider describes the division of a cell population. For details we refer to Diekmann-Heijmans-Thieme (1984).

Example 2.15. Let $E=L^{1}\left(\left[\frac{1}{4}, 1\right], w d x\right)$, where the density $w$ is a continuous positive function on $\left[\frac{1}{4}, 1\right]$, vanishes at $x=1$ and is strictly positive in $\left[\frac{1}{4}, 1\right)$.\\
We consider the operator $C=A+B$ where $A$ is defined by (Af) $(x):=-x f^{\prime}(x)$ on the domain $D(A):=\left\{f \in A C: f\left(\frac{1}{4}\right\}=0\right\}$ and $B$ is defined by

$$
B f(x):=\left\{\begin{array}{cc}
k(x) f(2 x) & \text { if } x \leq \frac{1}{2} \\
0 & \text { if } x>\frac{1}{2}
\end{array}\right.
$$

Here $k$ is a positive continuous function on $\left[\frac{1}{4}, 1\right]$ satisfying (2.13) $k(x)>0$ for $\frac{1}{4}<x<\frac{1}{2}$ and $\int_{1 / 4}^{1 / 2} \frac{k(y)}{y} d y=1$.

In the following we show that under these hypotheses and for suitable w the semigroup generated by C fulfills the assertions of Thm.2.14. The operator A generates the nilpotent semigroup ( $T(t)$ ) defined by

$$
(T(t) f)(x)=\left\{\begin{array}{cl}
f\left(e^{-t} x\right) & \text { if } e^{-t} x \geqq \frac{1}{4} \\
0 & \text { otherwise }
\end{array}\right.
$$

We have $(R(\lambda, A) f)(x)=x^{-\lambda} \int_{1 / 4}^{x} y^{\lambda-1} f(y)$ dy (f $\in E, x \in\left[\frac{1}{4}, 1\right]$ ). It follows that $A$ has compact resolvent. Since $B$ is bounded and positive, $C$ is the generator of a positive semigroup ( $S(t)$ ) having compact resolvent as well. Using C-III,Prop.3.3 one can show that (S(t)) is irreducible. Indeed, the non-trivial (T(t))-ideals are of the form $I_{s}=\left\{f \in E: f\right.$ vanishes on $\left.\left[\frac{1}{4}, s\right]\right\}$ with $s$ satisfying\\
$\frac{1}{4}<s<1$. Since none of these ideals is invariant under $B$, the semigroup ( $\mathrm{S}(\mathrm{t})$ ) is irreducible.\\
A suitable choice of the weight function $w$ ensures that ( $\mathrm{S}(\mathrm{t}$ ) ) is bounded. Take\\
(2.14) $\quad w(x):=\left\{\begin{array}{ll}\frac{1}{x} & \text { for } x \leq \frac{1}{2}, \\ \frac{1}{x} \cdot\left\{1-\int_{1 / 4}^{x / 2} \frac{k(y)}{y} d y\right\} & \text { for } x \geqq \frac{1}{2}\end{array}\right.$.

Then integration by parts yields for $f \in D(A)=D(C)$\\
$\langle C f, 1\rangle=\int_{1 / 4}^{1 / 2}\left(-x f^{\prime}(x)+k(x) f(2 x)\right) w(x) d x-\int_{1 / 2}^{1} x f^{\prime}(x) w(x) d x=0$.\\
Thus $1 \in D\left(C^{\prime}\right)$ and $C^{\prime} 1=0$, equivalently $S(t)^{\prime} 1=1$ for all $t$. This shows that $(S(t))$ is a semigroup of contractions on $E$. It remains to show that there is $\alpha>0$ such that $i \alpha \in \sigma(C)$. In fact, considering $\alpha:=2 \pi(\log 2)^{-1}$ then $i \alpha$ is an eigenvalue of C . A corresponding eigenfunction is given by $h_{1}(x):=x^{-i \alpha_{0}}(x)$, where $h_{0}$ is the eigenfunction corresponding to 0 defined as

\[
h_{0}(x):=\left\{\begin{array}{cc}
\int_{1 / 4}^{x} \frac{k(y)}{y} d y & \text { for } \frac{1}{4} \leqq x \leqq \frac{1}{2},  \tag{2.15}\\
1 & \text { for } \frac{1}{2} \leqq x \leqq 1
\end{array}\right.
\]

The verification of these statements is left as an excercise.

In several of the above results we had to assume that the positive semigroup $(T(t))_{t \geq 0}$ is bounded and has spectral bound zero. In general, these conditions are difficult to verify, in particular, when only the generator is known. In the final example we described a method how to cope with this problem: If s(A) is an eigenvalue of the adjoint $A^{\prime}$ with a strictly positive eigenvector $\phi$, then $(T(t))_{t \geqq 0}$ induces in a canonical way a positive semigroup $\left.{ }^{\left(T_{\phi}\right.}{ }^{(t)}\right)_{t \geq 0}$ on the AL-space (E, $\phi$ ). This semigroup satisfies $\left\|\mathrm{T}_{\phi}(t)\right\| \leqq \exp (t \cdot s(A))$ and has spectral bound $s(A)$. Hence one may apply the results of this section to the rescaled semigroup $\left(\exp (-t \cdot s(A)) T_{\phi}(t)\right)_{t \geq 0}$ thus obtaining convergence of $(T(t))_{t \geqq 0}$ for the weaker topology on $E$ which is induced by ( $E, \phi$ ) .

\section*{3. A SEMIGROUP APPROACH TO RETARDED EQUATIONS}
by\\
Annette Grabosch and Ulrich Moustakas

As indicated by the above title of this section there is a close relationship to B-IV, Section 3. First, the considered Cauchy problems are "similar" to (RCP). Second, there again is a correspondence to a class of semigroups generated by the first derivative.\\
Instead of the differential equation in (RCP) we will study equations of the form\\
(RE)

$$
u(t)=\Phi\left(u_{t}\right), t \geqq 0
$$

$$
u_{0}=g .
$$

We use the following setting: Let $F$ be a Banach space, consider $E:=L^{1}([-1,0], F)$ and take $\Phi \in L(E, F)$. For $u \in \operatorname{L}_{10 C}^{1}([-1, \infty), F)$ we denote by $u_{t} \in E$ the function given by $u_{t}(s):=u(t+s), t \geqq 0$, $s \in[-1,0]$.

By a solution of (RE) with initial function $g \in E$ we understand a function $u \in L^{1}{ }_{l o c}([-1, \infty), F)$ which satisfies equation (RE). (RE) is called well-posed if for each $g \in E$ there exists exactly one solution.\\
Remarks. 1. The equation

$$
\begin{gathered}
u(t)=B u(t)+\Phi\left(u_{t}\right), t \geq 0, \\
u_{0}=g,
\end{gathered}
$$

(where $B$ is the generator of a bounded semigroup on $F$ ) is in better analogy to the retarded Cauchy problem of B-IV, Sec. 3 and seems to be more general than the one introduced above, but can be reduced to an equation of the type (RE). In fact, since $1 \in \rho(B)$ we have

$$
u(t)=R(1, B) \Phi\left(u_{t}\right) .
$$

Clearly, this equation is of the previous type (with a different "delay functional").\\
2. The choice of "L"-functions" instead of "C-functions" (as in the case of (RCP)) enforces the solutions of (RE) to yield a strongly continuous semigroup of operators (on the space $E$ of initial functions) as in B-IV, Section 3.

In order to solve (RE) we consider the differential operator $A:=\frac{d}{d x}$ on $\mathrm{E}=\mathrm{L}^{\mathrm{I}}([-1,0], \mathrm{F})$ with domain

$$
D(A):=\left\{f \in A C([-1,0], F): f^{\prime} \in E \text { and } f(0)=\Phi(f)\right\} .
$$

We claim that ( $A, D(A))$ generates a strongly continuous semigroup $(T(t))_{t \geqq 0}$ on $E$. To this end we first consider the operator $A_{0} f:=f$ ' with domain

$$
D\left(A_{O}\right):=\left\{f \in E: f \in A C([-1,0], F), f^{\prime} \in E \text { and } f(0)=0\right\} .
$$

Similarly to the special case where $F=\mathbb{R}$ (compare A-I,Ex.2.4.(ii)) it can be seen that the operator $A_{O}$ generates a strongly continuous semigroup $\left(T_{0}(t)\right)_{t \geqslant 0}$ given by

\[
\left(T_{O}(t) f\right)(s)=\left\{\begin{array}{cc}
f(t+s) & \text { if } t+s \leqq 0  \tag{3.1}\\
0 & \text { if } t+s>0
\end{array}\right.
\]

Notice that $\left(\mathrm{T}_{0}(t)\right)_{t \geq 0}$ is a nilpotent semigroup.\\
Now consider the operators $S_{\lambda}: E \rightarrow E: f \rightarrow \varepsilon_{\lambda} \otimes \Phi(f), \lambda>0$, where $\varepsilon_{\lambda}$ denotes the function $s^{\lambda}+e^{\lambda s}$ as an element of $L^{1}[-1,0]$ and $h \otimes x \in E$ is defined by (hox) (s) $:=h(s) \cdot x$ for $h \in L^{l}[-1,0], x \in F$ and $s \in[-1,0]$. Clearly $\left\|_{\lambda}\right\|_{\|}=1 / \lambda \cdot\left(1-e^{-\lambda}\right) \rightarrow 0$ as $\lambda \rightarrow \infty$ and we have $\left\|s_{\lambda} i=\right\| \varepsilon_{\lambda}\|\cdot\| \Phi\left\|=1 / \lambda \cdot\left(1-\mathrm{e}^{-\lambda}\right) \cdot\right\| \Phi\|\leq 1 / \lambda \cdot\| \Phi \|$. For every $\lambda>\|\phi\|$, (Id $-s_{\lambda}$ ) is an isomorphism of $E$ and it is not difficult to see that it induces a bijection from $D(A)$ onto $D\left(A_{0}\right)$ such that


\begin{equation*}
(\lambda-A)=\left(\lambda-A_{0}\right)\left(I d-S_{\lambda}\right) \tag{3,2}
\end{equation*}


Since $A_{0}$ generates a semigroup of contractions $\lambda$ - $A_{0}$ is invertible for each $\lambda>0$. This yields the invertibility of $\lambda$ - A for each $\lambda \geqq\|\phi\|$.\\
In order to obtain an estimate on $\|R(\lambda, A)\|$ we use Formula (3.2).\\
Since $\left\|R\left(1, s_{\lambda}\right)\right\|=\left\|\sum_{n=0}^{\infty} s_{\lambda}^{n}\right\| \leq \sum_{n=0}^{\infty}\left\|\varepsilon_{\lambda}\right\|^{\mathrm{n}} \cdot\|\phi\|^{\mathrm{n}}=\left(1-\left\|\varepsilon_{\lambda}\right\| \cdot\|\phi\|\right)^{-1}$\\
and $\left\|\mathbb{R}\left(\lambda, \mathrm{A}_{0}\right)\right\| \leqq 1 / \lambda$ for $\lambda>0$ we obtain for $\lambda \geqq\|\Phi\|$ :\\
$\|R(\lambda, A)\| \leqq\left(1-\left\|\varepsilon_{\lambda}\right\| \cdot\|\phi\|\right)^{-1} \cdot 1 / \lambda=\left(\lambda-\lambda \cdot\left\|\varepsilon_{\lambda}\right\| \cdot\|\Phi\|\right)^{-1}$

$$
=\left(\lambda-\left(1-e^{-\lambda}\right) \cdot\|\Phi\|^{-1} \leqq\left(\lambda-\|\Phi\|^{-1} .\right.\right.
$$

By using A-II, Cor.1.8 we thus have proved the first assertion of the following theorem:

Theorem 3.1. The operator A defined above is the generator of a semigroup $(T(t))_{t \geq 0}$ on $E$.\\
For every $f \in E, t \geqq 0$ we have for a.e. $s \in[-1,0]$\\
(3.3) $(T(t) f)(s)= \begin{cases}f(t+s) & \text { if } t+s \leqq 0 \\ \Phi(T(t+s) f) & \text { if } t+s>0 .\end{cases}$

Moreover, if $f \in D(A)$ then the translation property (T) (see $B-I V$, Thm. 3.1) is satisfied.

Proof. Consider $\mathrm{E}_{1}:=\mathrm{D}(\mathrm{A})$ endowed with the graph norm and $\mathrm{A}_{1}:=\mathrm{A}$ restricted to $D\left(A_{1}\right):=D\left(A^{2}\right)$. By $(A-I, 3.5) A_{1}$ generates the semigroup $\left(T(t) \mid D(A)^{\prime} t \geq 0^{\circ}\right.$ on $E_{1}$ point evaluation is a continuous mapping and therefore the translation property can be shown as in the proof of B-IV,Thm.3.1. Hence we obtain

(3.4) $(T(t) f)(s)=\left\{\begin{array}{ll}f(t+s) & \text { if } t+s \leqq 0 \\ \Phi(T(t+s) f) & \text { if } t+s>0\end{array}= \begin{cases}f(t+s) & \text { if } t+s \leqq 0 \\ (T(t+s) f)(0) & \text { if } t+s>0 ;\end{cases}\right.$ i.e. (3.3) is valid for $\mathrm{f} \in \mathrm{D}(\mathrm{A})$. It remains to show (3.3) for all $\mathbf{f} \in \mathrm{E}$. Fix $t \in \mathbb{R}_{+}$and $s \in[-t, 0]$. For $t+s>0$ the equality follows immediately by the continuity of $\$$ from (3.4). For the case $t+s \leqq 0$ we consider $g \in L^{\infty}[-1,0]$ with supp $g \in[-1,-t]$. Comparing (3.1) and (3.4) we see that $\left\langle\left(T(t)-T_{0}(t)\right) f, g\right\rangle=0$ for all $f \in D(A)$, and hence for all $f \in E$.\\
Consequently $\left(T(t)-T_{0}(t)\right) \mathrm{f}=0$ a.e. on $[-1,-t]$ which shows $(T(t) f)(s)=f(t+s)$ for a.e. $s \in[-1,-t]$.

The following corollary corresponds to B-IV,Cor.3.2 and assures the we11-posedness of (RE) :

Corollary 3.2. For every $f \in E$ the function $u$ defined by

\[
u(t):=\left\{\begin{array}{lll}
f(t) & \text { if }-1 \leqq t \leqq 0  \tag{3.5}\\
\Phi(T(t) f) & \text { if } t>0
\end{array}\right.
\]

is the unique solution of (RE), in particular (RE) is well-posed. If $f \in D(A)$ then $u(t)=T(t) f(0)$ for $t>0$.

Proof. As in the proof of B-IV, Cor.3.2 we have $u_{t}=T(t) f$ for $t \geq 0$ since $u_{t}(s)=u(t+s)=(T(t) f)(s)$ by the definition of $u$ and by formula $(3,3)$. Thus $u(t)=\Phi(T(t) f)=\Phi\left(u_{t}\right)$ if $t \geq 0$.\\
Also by the definition of $u$ we have $u_{0}=f$.\\
It remains to show uniqueness. Let $w$ be a solution of (RE) with initial function $w_{0}=0$. Then

$$
\begin{aligned}
\|w(t)\|_{F} & =\left\|\Phi\left(w_{t}\right)\right\|_{F} \leqq\|\Phi\| \cdot\left\|_{t}\right\|_{E}=\|\Phi\| \cdot \int_{-1}^{0}\|w t(s)\|_{F} d s \\
& =\|\Phi\| \cdot \int_{-1}^{0}\|w(t+s)\|_{F} d s=\|\Phi\| \cdot \int_{t-1}^{t}\|w(s)\|_{F} d s \\
& \leqq\|\Phi\|_{i} \cdot \int_{-1}^{t}\|w(s)\|_{F} d s=\|\Phi\| \cdot \int_{0}^{t}\|w(s)\|_{F} d s \text { for } t \geq 0 .
\end{aligned}
$$

By Gronwall's lemma $\|w(t)\|_{F}=0$, thus $w(t)=0$.

Now we turn to the aspect of positivity in (RE). We assume $F$ to be a Banach lattice and let $E$ inherit the canonical ordering from $F$ making it a Banach lattice. Additionally, let $\Phi$ be positive. The first observation is that A generates a positive semigroup. Indeed, it follows from equation (3.2) that $R(\lambda, A)=R\left(1, S_{\lambda}\right) R\left(\lambda, A_{0}\right)$ for $\lambda>\|\Phi\|_{\text {, }}$ Since $S_{\lambda}$ is a positive operator we have $R\left(1, S_{\lambda}\right) \geqq 0$. The semigroup $\left(T_{0}(t)\right){ }_{t \geqq 0}$ generated by $A_{0}$ is positive (use (3.1)), hence $R\left(\lambda, A_{0}\right) \geqq 0$. It follows that $R(\lambda, A) \geqq 0$ which is equivalent to the positivity of $(T(t))_{t \geqq 0}$ (see C-II,Prop.4.1).

Proposition 3.3. If $\Phi \in L(E, F)$ is a positive operator, then the solution semigroup $(T(t))_{t \geqslant 0}$ corresponding to (RE) is positive.

For the following considerations concerning spectral poperties of the semigroup $(T(t))_{t \geq 0}$ we always suppose $\Phi$ to be positive. Furthermore we define operators $\Phi_{\lambda} \in L(F), \lambda \in \mathbb{R}$, by\\
(3.6) $\Phi_{\lambda} \mathrm{x}:=\Phi\left(\varepsilon_{\lambda}{ }^{\otimes \mathrm{x}}\right), \mathrm{x} \in \mathrm{F}$.

Evidently, each $\Phi_{\lambda}$ is a positive operator on $F$ and $\lambda \leqq \mu$ implies $\Phi_{\lambda} \geqq \Phi_{\mu}$. From this relation it follows that the spectral bound $s\left(\Phi_{\lambda}\right)$ which coincides with the spectral radius $r\left(\Phi_{\lambda}\right)$ is a decreasing function in $\lambda$.

Furthermore, we shall need the following properties.

Lemma 3.4. The map $h: \mathbb{R} \rightarrow \mathbb{R}_{+}: \lambda \rightarrow s\left(\Phi{ }_{\lambda}\right)$ is continuous from the left. If $\Phi_{\lambda}$ is compact for all $\lambda \in \mathbb{R}$, then $h$ is continuous.

Proof. As indicated above, $h$ is decreasing. Hence continuity from the left follows from the upper semicontinuity of the spectrum (see [Kato (1976), Chap.IV,Thm.3.11).

Now take $\lambda \in \mathbb{R}$ with $s\left(\Phi_{\lambda}\right)>0$ (if $s\left(\Phi_{\lambda}\right)=r\left(\Phi_{\lambda}\right)=0$, then continuity in $\lambda$ is trivial by the continuity from the left and the monotonicity). Since $\Phi_{\lambda}$ is positive and bounded we know that\\
$\left\{s\left(\Phi_{\lambda}\right)\right\}$ is the boundary spectrum $\sigma_{b}\left(\Phi_{\lambda}\right)$ (see C-III, Cor.2.12) of $\Phi_{\lambda}$ Moreover, $s\left(\Phi_{\lambda}\right)$ is a pole of the resolvent with residue of finite rank. Such spectral sets vary continuously under smooth perturbations of $\Phi_{\lambda}$ (see [Dunford-Schwartz (1958),VII.6, Thm.9]), thus $\lambda \rightarrow \mathrm{s}\left(\Phi_{\lambda}\right)$ is continuous.

For the operators $A_{0}$ and $A$ as defined in the beginning of this section we obtain an explicit representation of their resolvents.

Lemma 3.5. For the resolvents of the operators $A_{0}$, resp. A , on $E$ the following statements hold.\\
(a) For every $\lambda \in \mathbb{C}$ we have $\lambda \in \rho\left(A_{0}\right)$ and


\begin{equation*}
R\left(\lambda, A_{0}\right) g(t)=\int_{t}^{0} e^{\lambda(t-s)} g(s) d s, g \in E \tag{3.7}
\end{equation*}


(b) For $\lambda \in \mathbb{C}$ satisfying $1 \in \rho\left(\Phi{ }_{\lambda}\right)$ we have $\lambda \in \rho(A)$ and (3.8) $\quad R(\lambda, A) g=R\left(\lambda, A_{0}\right) g+E_{\lambda} 8 R\left(1, \Phi{ }_{\lambda}\right) \Phi R\left(\lambda, A_{0}\right) g, g \in E$.

Proof. (a) $\rho\left(A_{0}\right)=\mathbb{C}$ follows directly from $\left(T_{0}(t){ }_{t \geq 0}\right.$ being nilpotent (see A-III,Prop.1.1). For $g \in E$ we obtain $R\left(\lambda, A_{0}\right) g=f$ where $\pounds$ is a solution of $\lambda \mathbf{f} \mathrm{E}^{\prime}=\mathrm{g}$.\\
Thus $R\left(\lambda, A_{0}\right) g(t)=\int_{t}^{0} e^{\lambda(t-s)} g(s) d s+e^{\lambda t} \cdot x$ for some $x \in F$. The condition $f \in D\left(A_{0}\right)$ now implies $x=0$ and Formula (3.7).\\
(b) The assertion $\lambda \in \rho(A)$ means that for every $g \in E$ the equation $\lambda f-f^{\prime}=g$ has exactly one solution $f$ in $D(A)$. As in case (a) we have $f(t)=\int_{t}^{0} e^{\lambda(t-s)} g(s) d s+e^{\lambda t} \cdot x$ for some $x \in F$ and hence $R(\lambda, A) g=f=R\left(\lambda, A_{0}\right) g+\varepsilon_{\lambda}(\mathrm{X}$. The condition $R(\lambda, A) g \in D(A)$ implies $x-\Phi{ }_{\lambda}(x)=\Phi R\left(\lambda, A_{0}\right) g$. Hence $x=R\left(1, \Phi_{\lambda}\right) \Phi R\left(\lambda, A_{0}\right) g$ if $1 \in \rho\left(\Phi_{\lambda}\right)$ and thus $(3.8)$ follows.

Proposition 3.6. For each $\lambda \in \mathbb{C}$ the following implications hold.\\
(a) If $\lambda \in \sigma(\mathrm{A})$, then $1 \in \sigma\left(\Phi_{\lambda}\right)$.\\
(b) If $1 \in \operatorname{Po}\left(\Phi_{\lambda}\right)$, then $\lambda \in \operatorname{P\sigma }(A)$.

If, in addition, $\left(D\left(A_{O}\right)\right)=F$ or if $\Phi_{\lambda}$ is compact for all $\lambda \in \mathbb{C}$, then the following equivalence holds:\\
(c) $\lambda \in \sigma(A)$ if and only if $1 \in o\left(\Phi_{\lambda}\right)$.

Proof. (a) This implication follows immediately from Lemma 3.5(b).\\
(b) If $x \neq 0$ satisfies $x-\Phi_{\lambda}(x)=0$, then $f:=\varepsilon_{\lambda} 8 x \in D(A)$ and $\lambda f-f^{\prime}=0$.\\
(c) If $\Phi\left(D\left(A_{0}\right)\right)=F$, then the equation $x-\Phi \lambda_{\lambda}=\Phi R\left(\lambda, A_{0}\right) g$ has a\\
unique solution for every $g \in E$ if and only if $1 \in \rho\left(\phi_{\lambda}\right)$. According to the proof of $3.5(b)$ this is equivalent to $\lambda \in \rho(A)$. If $\Phi_{\lambda}$ is compact, then $\sigma\left(\Phi_{\lambda}\right) \backslash(0\} \subset P \sigma\left(\Phi_{\lambda}\right)$. Thus the assertion follows from (a) and (b).

The previous results will now be used to characterize the spectral bound of $A$ and hence the stability of the solutions of (RE).

Theorem 3.7. Let $A:=\frac{d}{d x}, D(A):=\left\{f \in A C([-1,0], F): f^{\prime} \in L^{1}([-1,0], F)\right.$ and $f(0)=\Phi(f)\}$ be the generator of the solution semigroup on $E:=$ $\mathrm{L}^{1}([-1,0], F)$ corresponding to (RE). If $F$ is a Banach lattice and $0 \leqq \Phi \in L(E, F)$, then the following assertions hold for $\lambda \in \mathbb{R}$.\\
(a) If $\mathrm{s}\left(\Phi_{\lambda}\right)<1$, then $\mathrm{s}(\mathrm{A})<\lambda$.\\
(b) Let $\Phi\left(D\left(A_{0}\right)\right)=F$ or let $\Phi_{\lambda}$ be compact for all $\lambda \in \mathbb{R}$. In addition, suppose that the map $\mu \rightarrow s\left({ }_{\mu}\right)$ is strictly decreasing at $\mu=s(A)$. If $s\left(\Phi_{\lambda}\right)=1$, then $s(A)=\lambda$.\\
(c) Let $\Phi_{\lambda}$ be compact for all $\lambda \in \mathbb{R}$ or let $\Phi\left(D\left(A_{0}\right)\right)=F$ and suppose that $\mu \rightarrow s\left(\Phi_{\mu}\right)$ is continuous from the right. If $s\left(\Phi_{\lambda}\right)>1$, then $s(A)>\lambda$.

Proof. (a) Suppose $r:=s(A) \geqq \lambda$. The positivity of $(T(t))_{t \geqslant 0}$ implies $r \in \sigma(A)$ (see C-III,Thm.1.1.(a)) and by Prop.3.6 (a) this implies $1 \in \sigma\left(\Phi_{r}\right)$ so that $s\left(\Phi_{r}\right) \geqq 1$. Since $\lambda \leqq r$ this yields $s\left(\Phi_{\lambda}\right) \geqq s\left(\Phi_{r}\right) \geqq 1$.\\
(b) Let $s\left(\Phi_{\lambda}\right)=1$. Since $1 \in \sigma\left(\Phi_{\lambda}\right)$ (see C-III,Thm.1.1(a)) $\lambda \in \sigma(\mathrm{A})$ by Prop.3.6(c) whence $s(A) \geqq \lambda$. If $r:=s(A)$ we deduce as in the proof of (a) that $s\left(\Phi_{r}\right) \geqq 1$. Now $r>\lambda$ would imply $s\left(\Phi_{\lambda}\right)>s\left(\Phi_{r}\right)$ $\geqq 1$ (by the strict monotonicity of $\mu \rightarrow s\left(\Phi_{\mu}\right)$, a contradiction. Hence we conclude $s(A)=r=\lambda$.\\
(c) The hypotheses and Lemma 3.4 imply that the map $\mu \rightarrow s_{\mu}\left(\Phi_{\mu}\right.$ is continuous. Let $s\left(\Phi_{\lambda}\right)>1$. Since $s\left(\Phi_{\mu}\right) \leqq\left\|_{\mu}\right\| \leqq\|\Phi\| \cdot\left\|\varepsilon_{\mu}\right\|$ we see that $s\left(\Phi_{\mu}\right)$ tends to zero as $\mu \rightarrow \infty$. Therefore there must exist $\mu^{\prime}>\lambda$ such that $\mathrm{s}\left(\Phi_{\mu},{ }^{\prime}=1\right.$. Now Prop.3.6.(c) implies $\mu^{\prime} \epsilon \sigma(A)$ whence $s(A) \geq \mu^{\prime}>\lambda$.

Corollary 3.8. Under the hypotheses of Thm.3.7, suppose that the mapping $h: \mu \rightarrow s\left(\Phi_{\mu}\right)$ is continuous from the right and strictly decreasing. Then the following equivalence holds.\\
(3.9) $s(A) \leqq \lambda$ if and only if $s\left(\Phi_{\lambda}\right) \leqq 1$.

In particular, $\lambda=s(A)$ is the only real solution of $s\left(\Phi_{\lambda}\right)=1$.

Proof. The first equivalence follows easily from Thm.3.7 . The additional statement is a consequence of the strict monotonicity of $h$.

Remarks. 1. We note that in Prop. 3.6 and Thm. 3.7 it actually suffices that some power of $\Phi_{\lambda}$ is compact.\\
2. The equivalence (3.9) reduces the problem of determining $s(A)$ to the determination of the spectral bounds of the operators $\Phi_{\lambda}$ on the "smaller" Banach space F.\\
In particular, $s(A)<0$ if and only if $s\left(\Phi_{0}\right)<1$.

\begin{enumerate}
  \setcounter{enumi}{2}
  \item We call the identity $" s\left(\Phi_{\lambda}\right)=1 "$ a generalized characteristic equation (see also the remark following B-IV,Thm.3.7). The usual characteristic equation (see for example [Hale (1977), p.168ff] and [Heijmans (1984), Sec.5]) is an equation determining all eigenvalues of the generator A. In fact, if $F$ is finite dimensional the characterization of the spectral values $\lambda$ of $A$ in Prop.3.6.(c) reduces to solving the complex equation $\operatorname{det}\left(I d-\Phi_{\lambda}\right.$ ) $=0$. Obviously, there is no analogous identity characterizing $\sigma(A)$ for infinite dimensional F. However, in order to determine the long term behavior of the solutions of (RE) it is often enough to know the spectral bound $s(A)$. Under the assumptions of Cor. 3.8 (in particular if $\$$ is positive) Formula (3.9) gives a tool to reduce this problem to the determination of the real solution of $s\left(\Phi_{\lambda}\right)=1$.
\end{enumerate}

Example 3.9. We give an example of a large class of operators $\Phi$ satisfying the above assumptions.

For $\psi \in\left(\mathrm{L}^{1}[-1,0]\right)^{\prime}=\mathrm{L}^{\infty}[-1,0]$ and $\mathrm{B} \in L(\mathrm{~F})$ we denote by $\Phi:=\psi \otimes \mathrm{B}$ the operator defined by $\Phi(h \otimes x)=\psi(h) \cdot B x$ for $h \in L^{1}[-1,0], x \in F$. Note that $E=L^{1}([-1,0], F)$ is isomorphic to $L^{1}[-1,0] \tilde{\theta}{ }_{\pi} \quad$ (see [Schaefer (1966), Chap.III,6.5]). The operator $\$$ is bounded from E into F. We assume that $\psi$ and $B$, hence $\Phi$ are positive.

Then the following holds and is stated without proof.

Lemma. (a) If $B$ is compact, then $\Phi$ is compact. If $B$ is surjective, then $\Phi\left(D\left(A_{0}\right)\right)=F$.\\
(b) $\sigma\left(\Phi_{\lambda}\right)=\psi\left(\varepsilon_{\lambda}\right) \cdot \sigma(B)$ for each $\lambda \in \mathbb{C}$. Hence the map $\mu \rightarrow s\left(\Phi_{\mu}\right)$ is continuous and strictly decreasing on $\mathbb{R}$.

For this type of "retarding functionals" $\Phi$ we obtain a simple characterization of the spectral bound.

Corollary. Let $\Phi=\psi B$ where $0 \leqq \psi \in L^{\infty}[-1,0]$ and $0 \leqq B \in L(F)$ such that $B^{n}$ is compact for some $n \Leftrightarrow N$. Then the following holds.


\begin{equation*}
s(A) \xlongequal[>]{\leftrightharpoons} \text { if and only if } \psi\left(\varepsilon_{\lambda}\right) \cdot s(B) \leqq 1 \text {. } \tag{3.10}
\end{equation*}


Example 3.10. Let $F$ be a Banach lattice with the Dunford-Pettis property (see Schaefer(1974), Sec.II.9). Take for example $F=C(K)$ or $F=L^{1}(X, \Sigma, \mu)$. Furthermore define $E=L^{1}([-1,0], F)$ as usual and let $\{K(s): s \in[-1,0]\}$ be a family of positive, irreducible, weakly compact operators on F which is bounded.\\
If we define $\phi f:=\int_{-1}^{0} \mathrm{~K}(\mathrm{~s}) \mathrm{f}(\mathrm{s}) \mathrm{ds}$ for all $f \in E$, then (RE) has the form


\begin{align*}
& f(t)=\int_{-1}^{0} K(s) f(s+t) d s, t \geqq 0  \tag{3.11}\\
& f_{0}=\psi \in E .
\end{align*}


By Cor.3.2 (3.11) has a unique solution $f \in L^{1}([-1, \infty), F)$. For $\Phi_{\lambda}$ we obtain $\Phi_{\lambda} x=\int_{-1}^{0} e^{\lambda s} K(s) x d s, x \in F$. In this case we have

$$
s(A) \leqq \lambda \text { if and only if } s\left(\Phi_{\lambda}\right) \leqq 1 \text {. }
$$

Proof. By Cor. 3.8 it suffices to show that the map $\mathrm{h}: \lambda \rightarrow s\left(\Phi_{\lambda}\right)=$ $r\left(\Phi_{\lambda}\right)$ is strictly decreasing and continuous.\\[0pt]
With the help of [Schaefer (1966), Thm. III.11.4] and [Schaefer (1974), Thm.II.9.9] it is easy to show that $\Phi_{\lambda}{ }^{2}$ is compact and the continuity of $h$ follows by the above remark. It remains to show that $h$ is strictly decreasing.\\
Assume $s\left(\Phi_{\lambda}\right)>0$ for all $\lambda \in \mathbb{R}$. since $\Phi_{\lambda}{ }^{2}$ and $\Phi_{\mu}{ }^{2}$ are compact, $s\left(\Phi_{\lambda}\right)$ and $s\left(\Phi_{\mu}\right)$ are eigenvalues of $\Phi_{\lambda}$ resp. $\Phi_{\mu}$ with corresponding eigenfunctions $x_{\lambda}$ resp. $x_{\mu}$. In the same way $s\left(\Phi_{\lambda}\right)$ and $\mathrm{s}\left(\Phi_{\mu}\right)$ are eigenvalues of $\Phi_{\lambda}{ }^{\prime}$ resp. $\Phi_{\mu}^{\prime}$ with corresponding eigenfunctions $\mathrm{x}_{\lambda}{ }^{\prime}$ resp. $\mathrm{x}_{\mu}{ }^{\prime}$.\\
For $0<x \in F$ and $0<\mu<\lambda$ we obtain,\\
$\Phi_{\mu} x=\int_{-1}^{0} e^{\mu s_{K}(s) x d s=\int_{-1}^{0} e^{(\mu-\lambda) s} e^{\lambda s_{K}}(s) x d s>\int_{-1}^{0} e^{\lambda s} K(s) x d s=\Phi_{\lambda} x . ~}$ since K(s) are positive and irreducible operators.\\
Especially, $\Phi_{\mu} x_{\lambda}>\Phi_{\lambda} x_{\lambda}=r\left(\Phi_{\lambda}\right) x_{\lambda} \quad$ and by evaluation\\
$<\Phi_{\mu} x_{\lambda}, x_{\mu}^{\prime}>>r\left(\Phi_{\lambda}\right)<x_{\lambda}, x_{\mu}^{\prime}>$. Thus $r\left(\Phi_{\mu}\right)<x_{\lambda}, x_{\mu}^{\prime}>>r\left(\Phi{ }_{\lambda}\right)<x_{\lambda}, x_{\mu}^{\prime}>$. since the operators $\Phi_{\lambda}$ are irreducible for each $\lambda$ (due to the irreducibility of $\mathrm{K}(\mathrm{s})$ ) $\mathrm{x}_{\mu}{ }^{\prime}$ is a strictly positive functional on F. Hence $\left\langle x_{\lambda}, x_{\mu}{ }^{\prime}\right\rangle \neq 0$ and thus $r\left(\phi_{\mu}\right)>r\left(\phi_{\lambda}\right)$.

Example 3.11. The next example is of a more special form and occures as a model describing the cell cycle based on unequal division of cells, (see [Arino-Kimmel (1985)]). Let $F=L^{1}[0,1], E=L^{1}([-1,0], F)$ and define an operator $\Phi: E \rightarrow F$ by\\
$\Phi(\psi)(x):=\int_{0}^{1} k\left(x, x^{\prime}\right) \psi(q(x))\left(x^{\prime}\right) d x^{\prime}$ for almost all $x \in[0,1]$.\\
Here $q$ is a continuously differentiable function with strictly positive derivative satisfying $-1 \leqq q(x) \leqq \varepsilon<0$ for all $x \in[0,1]$ and $k$ is a bounded, measurable, strictly positive kernel.\\
Then (RE) has the form


\begin{align*}
& f(t)(x)=\int_{0}^{1} k\left(x, x^{\prime}\right) f(t+q(x))\left(x^{\prime}\right) d x^{\prime}, t \geqq 0  \tag{3.12}\\
& f_{0}=\psi \in E .
\end{align*}


It is easy to show that $\Phi \in L(E, F)$. If we define $k \in L(F)$ by $K f(x)=\int_{0}^{1} k\left(x, x^{\prime}\right) f\left(x^{\prime}\right) d x^{\prime}$ we obtain $\Phi_{\lambda} f=e^{\lambda q(x)} \mathrm{Kf} \quad(f \in F)$. Again we have

$$
s(A) \lesseqgtr \lambda \quad \text { if and only if } s\left(\Phi_{\lambda}\right) \leqq 1 .
$$

Proof. By Cor. 3.8 it suffices to show that the map $h: \lambda \rightarrow s\left(\Phi_{\lambda}\right)$ is strictly decreasing and continuous.\\
Since $k$ is bounded the operator $k$ is weakly compact and so is $\Phi_{\lambda}$. Since $E$ has the Dunford-Pettis property $\left(\Phi_{\lambda}\right)^{2}$ is compact [Schaefer (1974), Thm.II.9.9] and this yields continuity of h.

Let $\lambda>\mu>0$ and $0<\mathrm{f} \in \mathrm{F}_{+}$.\\
Then $\Phi_{\mu} f(x)=e^{\mu q(x)} \operatorname{Kf}(x)=e^{(\mu-\lambda) q(x)} e^{\lambda q(x)} K f(x)=e^{(\mu-\lambda) q(x)} \Phi{ }_{\lambda} f(x)$. Since $q(x) \leqq \varepsilon$ for all $x \in[0,1]$, we obtain, $\Phi_{\mu} f \geqq e^{(\mu-\lambda) \varepsilon_{\Phi} f}$ and, moreover, $\left(\Phi_{\mu}\right)^{n_{\pounds}} \geqq e^{n(\mu-\lambda) \varepsilon} \cdot\left(\Phi_{\lambda}\right)^{n_{f}} \quad$ for every $n \in \mathbb{N}$. Hence $\left\|\left(\Phi_{\mu}\right)^{n}\right\|>e^{n(\mu-\lambda) \varepsilon}\left\|\left(\Phi_{\lambda}\right)^{n}\right\| \quad$ and consequently $\quad r\left(\Phi_{\mu}\right) \geqq e^{(\mu-\lambda) \varepsilon_{r}\left(\Phi_{\lambda}\right)}$. Now $(\mu-\lambda) \varepsilon>0$ implies $r\left(\Phi_{\mu}\right)>r\left(\Phi_{\lambda}\right)$.

The theory developed so far can also be applied to certain population equations. We first notice that (ACP) is isomorphic (in an obvious manner) to the following cauchy problem.\\
For some $x \in \mathbb{R}_{+}$take $E:=L^{1}([0, r], F)$ and let $A:=-\frac{d}{d x}$ on the domain $D(A):=\left\{\pounds \in A C([0, r], F): f^{\prime} \in E\right.$ and $\left.f(0)=\Phi(f)\right\}$ for some $\Phi \in L(E, F)$.\\
We adopt this setting and transform the above results; e.g., $\varepsilon_{\lambda}$ has to be defined as $\varepsilon_{\lambda}(s):=e^{-\lambda s}$ instead of $e^{\lambda s}$. As a concrete example we consider the following.

Example 3.12. Let $F:=\mathbb{C}^{n}, E:=L^{1}([0, x], F)=\Pi_{k=1}^{n} F_{k}, F_{k}=L^{1}[0, r]$. and define $\Phi: E \rightarrow \mathbb{C}^{n}$ by $\Phi=\left(v_{i j}\right)_{n \times n}$ where $\left\langle\nu_{i j}, f>=\int_{0}^{r} \beta_{i j}(a) f(a) d a \quad\right.$ for $f \in L_{[0, r]}{ }^{n}$ and $0 \leqq \beta_{i j} \in L^{\infty}[0, r]$. As $\Phi_{\lambda}$ we obtain the scalar matrix,\\
$\Phi_{\lambda}=\left(\begin{array}{ccc}\left(\int_{0}^{r} B_{11}(a) e^{-\lambda a} d a\right) & \left(\int_{0}^{r} \beta_{12}(a) e^{-\lambda a} d a\right) & \\ \cdot & \cdot & \\ \cdot & \cdot & \cdot \\ \cdot & & \cdot \\ \left(\int_{0}^{r} B_{n 1}(a) e^{-\lambda a} d a\right) & & \left(\int_{0}^{r} B_{n n}(a) e^{-\lambda a} d a\right)\end{array}\right) \cdot$\\
Suppose additionally that $\Phi_{\lambda}$ is irreducible for each $\lambda$, which is, for example, satisfied if $\beta_{i j}(a)>0$ for every $a \in[0, r]$ and $1 \leqq \mathrm{i}, \mathrm{j} \leqq \mathrm{n}$ (see also [Bellman-Cooke (1963),p.257ff]).\\
Since $\Phi$ has finite dimensional range and hence is compact it follows that the function $h: \lambda \rightarrow s\left(\Phi_{\lambda}\right)$ is continuous. Moreover one shows that $h$ is strictly decreasing by using the same arguments as in Example 3.10 and by using the fact that ${ }_{\lambda}{ }_{\lambda}$ is irreducible. The system of differential equations corresponding to A is

$$
\begin{aligned}
& \frac{\partial}{\partial t} u_{i}(t, \alpha)=-\frac{\partial}{\partial \alpha} u_{i}(t, \alpha) \quad(i=1, \ldots, n) \text { for } t \in \mathbb{R}_{+}, \alpha \in[0, r] \\
& \text { with initial condition }
\end{aligned}
$$


\begin{equation*}
u_{i}(0, \alpha)=v_{i}(\alpha) \quad(i=1, \ldots, n) \quad \text { for } \alpha \in[0, r] \tag{3.13}
\end{equation*}


and boundary condition

$$
u_{i}(t, 0)=\int_{0}^{r}\left[\sum_{j=1}^{n} \beta_{i j}(\alpha) u_{j}(t, \alpha)\right] d \alpha \quad(i=1, \ldots, n) \text { for } t \in \mathbb{R}_{+}
$$

This system has a solution for every $\left(v_{1}, \ldots, v_{n}\right) \in D(A)$ and the asymptotic behavior is determined by the identity\\
\includegraphics[max width=\textwidth, center]{2024_12_23_c6487cc0859199a15bd9g-375}\\
whose unique real solution $\lambda$ is $s(A)$.

The infinite dimensional problem of determining s(A) is therefore reduced to solving a real one-dimensional equation.

The differential equation (3.13) may be interpreted as follows. Consider $n$ populations and let $r$ be the máximal age of an individual. Further let $u_{i}(t, \alpha)$ denote the density of the number of members of the population $i$ with respect to age $\alpha$ at time $t$. The birth-rate is denoted by B. The first equation expresses the process of growing old. The second equation defines the initial state at time zero and the last equation describes mutual dependences of birth from individuals of the n populations.

Example 3.13. Take $F:=\mathrm{L}^{1}(\Omega)$ where $\Omega \subset \mathbb{R}^{2}$ is bounded and take E $:=L^{1}([0, r], F)=L^{1}([0, r] \times \Omega)$ for some $r \in \mathbb{R}_{+}$. Furthermore, let $\Phi$ $=B \otimes B$ where $0<B \in L^{\infty}[0, r]$ and $B \in L(F)$ is an integral operator with positive bounded kernel k .\\
The corresponding Cauchy problem is the following.


\begin{align*}
& \frac{\partial}{\partial t} u(t, \alpha, x)=-\frac{\partial}{\partial \alpha} u(t, \alpha, x) \\
& \quad \text { with initial condition } \\
& u(0, \alpha, x)=v(\alpha, x) \tag{3.14}
\end{align*}


and boundary condition

$$
u(t, 0, x)=\int_{0}^{r} B(\alpha) \cdot \int_{\Omega} k(x, y) \cdot u(t, \alpha, y) d y \underset{f o r}{d \in \mathbb{R}_{+}, x \in \Omega,} \quad \alpha \in[0, r]
$$

From Thm.3.1 we conclude that for every $v \in D(A)$ there exists a solution $u \in E$. The boundedness of the integral kernel $k$ yields weak compactness of B (see [Schaefer (1974), Sec.II.5] and thus compactness of $B^{2}$ by the Dunford-Pettis-Property of $L^{1}(\Omega)$ (see [Schaefer (1974), Chap.II,Thm.9.9]).

Thus we are in the situation of Ex.3.9 and from Formula (3.10) we obtain the equivalence

$$
s(A) \leqq \lambda \text { if and only if } \quad \int_{0}^{r} \beta(\alpha) e^{-\lambda \alpha} d \alpha \cdot s(B) \leqq 1 \text {. }
$$

Again we can find a biological interpretation. Let $u(t, \alpha, x)$ denote the density of the number of individuals in a given population with respect to age $\alpha$ and position $x$ at time t. As in Ex.3.12 the first equation in (3.14) corresponds to the aging process. The second equation fixes the initial state of the population and the last equation describes the dependence of newborns on the birth rate $\beta$ and the distribution of the population over the "area" $\Omega$.

NOTES.\\
Section 1. Coincidence of spectral and growth bounds for $\mathrm{L}^{1}$-spaces was proven by Derndinger (1980). For $\mathrm{L}^{2}$-spaces the result is due to Greiner-Nage1 (1983). For the result on AM-spaces we refer to Remark 1.1 of $B$-IV and the corresponding notes. Interpolation techniques in order to obtain results on arbitrary $\mathrm{L}^{\mathrm{p}}$-spaces were used by Voigt (1985). He proved Corollary 1.2(a). Theorem 1.3 as well as Propositions 1.6 , 1.7 , 1.9 are taken from Neubrander (1985a). For a comprehensive discussion of the coincidence of the spectral bound $s(A)$ with other growth bounds of positive semigroups on ordered Banach spaces, see Klein (1984). Similar results for finite dimensional (non-lattice) ordered spaces can be found in Stern (1982). For general results on convergence of the solutions of the inhomogeneous Cauchy problem we refer to Pazy (1983) and the references therein.

Section 2. For quasi-compact semigroups (as considered in theorem 2.1) we refer to the notes of B-IV,Sec.2. Example 2.3 is discussed in more detail in Webb (1984) and Greiner (1984). Further examples of this type are considered in Section 3. It was Lotz (1986) who observed that Doeblin's condition is sufficient for quasi-compactness in reflexive ${ }^{\mathrm{P}}$-spaces. (Obviously this is false in $\mathrm{L}^{\mathrm{L}}$-spaces since in this case the identity operator satisfies Doeblin's condition.)\\
The 0-2-Law for certain bounded operators on L was first established by Ornstein and Sucheston. A special case of the 0-2-Law for one-parameter semigroups (Theorem 2.6) was proven by Winkler (1972) while the general result and its corollaries can be found in Greiner (1982). Corollary 2.11 remains true when the assumption 'T( $t_{0}$ ) is a kernel operator' is replaced by 'T( $t_{0}$ ) is an irreducible Harris operator' (see Lin (1983)).\\
It is well-known that semigroups play an important role in probability theory (see e.g. Dynkin (1965), Feller (1952) and Hille-Phillips (1957)). A more detailed discussion than the one in Example 2.8 is given in Chapter 2 of van Casteren (1985). Convergence to periodic solutions is investigated in Kerscher-Nagel (1984) and Nagel (1984) where Proposition 2.13 is proved. They proved Proposition 2.13. The equation considered in Example 2,15 describes a linear model for cell division with exponential growth of individual cells. The occurring phenomena are conjectured by Diekmann et al. (1984).

Section 3. One of the starting points in the study of retarded equations was the book of Bellmann-Cooke (1963) on differential-difference equations. Initiated by Hale's semigroup approach (see B-IV,Sec.3) to retarded differential equations, Dyson-Villella-Bressan (1979), Villella-Bressan (1985) and Webb (1977) used such methods to investigate retarded equations. These similarly apply to Volterra equations [Miller (1974), Webb (1977)], and to age-dependent population equations [Prüß (1981), Webb (1984), (1985a)]. Recently, the aspect of positivity has led to statements on the asymptotic behavior of solutions of retarded equations. In this context the investigation of population equations by Greiner (1984), Heijmans (1985a) and Webb (1985b) should be mentioned.

\section*{POSItive SEMIGROUPS ON C* - AND W*-ALGEBRAS }


\section*{CHAPTER D-I}
BASIC RESUITSON SEMIGROUPS\\
AND $\quad O P E R A T O R$ A $\mathrm{A} G E B R A S$

This is not a systematic introduction into the theory of strongly continuous semigroups on $\mathrm{C}^{*}$ - and $\mathrm{W}^{*-a l g e b r a s . ~ F o r ~ t h a t ~ w e ~ r e f e r ~ t o ~ B r a-~}$ tteli-Robinson (1979), Davies (1976) and the survey article of Oseledets (1984). We only prepare for the subsequent chapters on spectral theory and asymptotics by fixing the notations and introducing some standard constructions.

\section*{1. NOTATIONS}
\begin{enumerate}
  \item By $M$ we shall denote a $C^{*}$-algebra with unit 1 . $M^{\text {sa }}:=\{x \in M$ : $\left.x^{*}=x\right\}$ is the self-adjoint part of $M$ and $M_{+}:=\{x * x: x \in M\}$ the positive cone in $M$. If $M^{\prime}$ is the dual of $M$, then $M^{\prime}+:=\left\{\psi \mathcal{M}^{\prime}\right.$ : $\left.\psi(\mathrm{x}) \geq 0, x^{\epsilon} \mathrm{M}_{+}\right\}$is a weak*-closed generating cone in $M^{\prime}$. $S(M):=$ $\left\{\psi \epsilon^{\prime}{ }_{+}: \psi(1)=1\right\}$ is called the state space of $M$. For the theory of C*-algebras and related notions we refer to [Pedersen (1979)]. $M$ is called a $\underline{W^{*}-a l g e b r a}$, if there exists a Banach space $M_{*}$, such that its dual $\left(M_{*}\right)^{\prime}$ is (isomorphic to) $M$. We call $M_{*}$ the predual of $M$ and $\psi \mathcal{M}_{*}$ a normal Iinear functional. It is known that $M_{*}$ is unique [Sakai (1971), 1.13.3.]. For further properties of $M_{*}$ we refer to [Takesaki (1979), Chapter III].
\end{enumerate}

\footnotetext{${ }^{*)}$ This is the main part of my "Habilitationsschrift" (accepted by Mathematische Fakultät, Universität Tübingen, 1984). I wish to thank Professor T. Ando for his warm hospitality during a one year stay at the Institute of Applied Electricity, Hokkafdo University, Sapporo. My thanks go also to the Alexander von Humboldt Stiftung and to NIHON GAKUJITSU SHINKOKAL, whose support made this stay possible.
}
2. A map $T \in L(M)$ is called positive (in symbols $T \geq 0$ ) if $T\left(M_{+}\right.$) $\subseteq$ $M_{+}$. $T \in L(M)$ is called n-positive $(n \in \mathbb{N})$ if $T \otimes I d_{n}$ is positive from $M \otimes M_{n}$ in $M \otimes M_{n}$, where $I d_{n}$ is the identity map on the $c^{*}$-algebra $M_{n}$ of all n×n-matrices. Obviously, every n-positive map is positive. We call $T \in L(M)$ a Schwarzmap if $T$ satisfies the inequality
$$
T(x) T(x)^{*} \leqq T\left(x x^{*}\right) \quad, x \in M
$$

Note that such $T$ is necessarily a contraction. It is well known that every n-positive contraction, $n \geq 2$ and that every positive contraction on a commutative $C^{*}$-algebra is a Schwarz map [Takesaki (1979), Corollary IV. 3.8.]. As we shall see, the Schwarz inequality is crucial for our investigations.\\
3. If $M$ is a $C^{*}$-algebra we assume $T=(T(t))_{t \geqq 0}$ to be a strongly continuous semigroup (abbreviated semigroup) while on $\mathrm{W}^{*}$-algebras we consider weak*-semigroups, i.e. the mapping ( $t \rightarrow T(t) x$ ) is continuous from $\mathbb{R}_{+}$into $\left(M, \sigma\left(M, M_{\star}\right)\right), M_{*}$ the predual of $M$, and every $T(t) \in$ $T$ is o $\left(M, M_{*}\right)$-continuous. Note that the preadjoint semigroup

$$
T_{*}=\left\{T(t)_{\star}: T(t) \in T\right\}
$$

is weakly, hence strongly continuous on $M_{\star}$ (see e.g., Davies (1980), Prop.1.23). We call $T$ identity preserving if $T(t) 1=1$ and of Schwarz type if every $T(t) \in T$ is a Schwarz map.

For the notations concerning one-parameter semigroups we refer to Part A. In addition we recommend to compare the results of this section of the book with the corresponding results for commutative $C^{*}-a l g e b r a s$, i.e. for $C_{O}(X), C(K)$ and $L^{\infty}(\mu)$ (see Part B).\\
2. A FUNDAMENTAI INEQUALITY FOR THE RESOLVENT

If $T=(T(t))_{t \geq 0}$ is a strongly continuous semigroup of Schwarz maps on a C*-algebra M (resp. a weak*-semigroup of Schwarz type on a $W^{*}$-algebra M) with generator $A$, then the spectral bound $s(A) \leqq 0$. Then for $\lambda \in \mathbb{C}, \operatorname{Re}(\lambda)>0$, there exists a representation for the resolvent $R(\lambda, A)$ given by the formula

$$
R(\lambda, A) x=\int_{0}^{\infty} e^{-\lambda t} T(t) x d t \quad, \quad x \in M
$$

where the integral exists in the norm topology.

In [Bratteli-Robinson (1979)] it is shown that $T$ is a semigroup of Schwarz type if and only if $\mu \mathrm{R}(\mu, \mathrm{A})$ is a Schwarz map for every $\mu \in \mathbb{R}_{+} \cdot$ Here we relate the domination of two semigroups to an inequality for the corresponding resolvent operator. This inequality will be needed later.

Theorem 2.1. Let $T=(T(t))_{t \geq 0}$ be a semigroup of schwarz type and $T=(S(t))_{t \geqq 0}$ a semigroup on a $C^{*}$-algebra $M$ with generators $A$ and B , respectively. If $(S(t) x)(S(t) x) * \leqq T(t) x x^{*}$\\
for all $x \in M$ and $t \in \mathbb{R}_{+}$, then

$$
(\mu R(\mu, B) x)(\mu R(\mu, B) x)^{*} \leqq \mu R(\mu, A) x x^{*}
$$

for all $x \in M$ and $\mu \in \mathbb{R}_{+}$. The same result holds if $T$ is a weak*-semigroup of Schwarz type and $S$ is a weak*-semigroup on a $W^{*}$-algebra $M$ such that (*) is fulfilled.

Proof. From the assumption (*) it follows that

$$
\begin{aligned}
0 \leqq(S(r) x- & S(t) x)(S(x) x-S(t) x)^{*}= \\
= & (S(r) x)(S(r) x) *-(S(r) x)(S(t) x)^{*}- \\
& -(S(t) x)(S(r) x)^{*}+(S(t) x)(S(t) x)^{*} \leqq \\
\leqq & T(r) x x^{*} \\
+ & T(t) x x^{*}-(S(r) x)(S(t) x)^{*} \\
& -(S(t) x)(S(r) x)^{*}
\end{aligned}
$$

for every $r, t \in \mathbb{R}_{+}$. Hence

$$
(S(r) x)(S(t) x)^{*}+(S(t) x)(S(r) x)^{*} \leqq T(r) x x^{*}+T(t) x x^{*}
$$

Obviously, $\|S(t)\| \leq 1$ for all $t \in \mathbb{R}_{+}$. Then for all $\mu \in \mathbb{R}_{+}$and $\mathrm{x} \in \mathrm{M}$ :\\
$(R(\mu, B) x)(R(\mu, B) x)^{*}=\left(\int_{0}^{\infty} e^{-\mu r} S(r) x d r\right)\left(\int_{0}^{\infty} e^{-\mu t} S(t) x d t\right)^{*}=$

$$
\begin{aligned}
& =\frac{1}{2}\left(\int _ { 0 } ^ { \infty } \int _ { 0 } ^ { \infty } e ^ { - \mu ( r + t ) } \left((S(r) x)(S(t) x)^{*}\right.\right. \\
& \left.\quad+(S(t) x)(S(r) x)^{*}\right) d r d t \\
& \leqq \frac{1}{2}\left(\int_{0}^{\infty} \int_{0}^{\infty} e^{-\mu(r+t)}\left(T(r) x x^{*}+T(t) x x^{*}\right) d r d t\right. \\
& =\left(\int_{0}^{\infty} e^{-\mu s} d s\right)\left(\int_{0}^{\infty} e^{-\mu t} T(t) x x^{*} d t\right)=\mu^{-1} R(\mu, A) x x^{*}
\end{aligned}
$$

where the handling of the integral is justified by [Bourbaki (1955), §8, $\mathrm{n}^{\circ} 4$, Proposition 9].

Corollary 2.2. Let $T$ be a semigroup of Schwarz maps (resp., weak*semigroup of Schwarz maps). Then for all $\lambda \in \mathbb{C}$ with $\operatorname{Re}(\lambda)>0$ :

$$
(R(\lambda, A) x)(\operatorname{R}(\lambda, A) x)^{*} \leqq(\operatorname{Re} \lambda)^{-1} R(\operatorname{Re} \lambda, A) x x^{*} \quad, \quad x \in M
$$

In particular for all $(\mu, \alpha) \in \mathbb{R}_{+} \times \mathbb{R}, x \in M$ :

$$
(\mu R(\mu+i \alpha, A) x)(\mu R(\mu+i \alpha, A) x)^{*} \leqq \mu R(\mu, A)\left(x x^{*}\right)
$$

Proof. Let $\lambda \in \mathbb{C}$ with $\operatorname{Re}(\lambda)>0$. Then the semigroup

$$
S:=\left(e^{\left.-i \operatorname{Im}(\lambda) t_{T}(t)\right)} t \geqq 0\right.
$$

fulfils the assumption of Thm 2.1. and $B:=A-i \lambda$ is the generator of $S$. Consequently $R(\lambda, A)=R(R e \lambda, B)$ and the corollary follows from Theorem 2.1.

As in Section C-III the following notion will be an important tool for the spectral theory of semigroups.

Definition 2.3. Let $E$ be a Banach space and $\varnothing \neq D$ an open subset of $\mathbb{C}$. A family $\mathrm{R}: \mathrm{D} \rightarrow L(\mathrm{E})$ is called a pseudo-resolvent on D with values in E if

$$
R(\lambda)-R(\mu)=-(\lambda-\mu) R(\lambda) R(\mu)
$$

for all $\lambda$ and $\mu$ in $D$.

If $R$ is a pseudo-resolvent on $D=\{\lambda \in \mathbb{C}: \operatorname{Re}(\lambda)>0\}$ with values in a C*- or $W^{*}$-algebra, then $R$ is called of Schwarz type if

$$
(R(\lambda) x)(R(\lambda) x))^{*} \leq(\operatorname{Re} \lambda)^{-1} R(\operatorname{Re} \lambda) x x^{*}
$$

for all $\lambda \in \mathrm{D}$ and $\mathrm{x} \in \mathrm{M} \cdot \mathrm{R}$ is called identity preserving if $\lambda R(\lambda) \lambda=1$ for all $\lambda \in D$.

For examples and properties of a pseudo-resolvent see C-III, 2.5. We state what will be used without further reference.\\
(a) If $\alpha \in \mathbb{C}$ and $x \in E$ such that $(\alpha-\lambda) R(\lambda) x=x$ for some $\lambda \in D$, then $(\alpha-\mu) R(\mu) x=x$ for all $\mu \in D$ (use the "resolvent equation").\\
(b) If $F$ is a closed subspace of $E$ such that $R(\lambda) F \subseteq F$ for some $\lambda \in \mathrm{D}$, then $R(\mu) F \subseteq F$ for all $\mu$ in a neighbourhood of $\lambda$. This follows from the fact that for all $\mu \in D$ near $\lambda$ the pseudo-resolvent in $\mu$ is given by

$$
R(\mu)=\sum_{n}(\lambda-\mu)^{n_{R}(\lambda)^{n+1}} .
$$

Definition 2.4. We call a semigroup $T$ on the predual $M_{*}$ of a W*-algebra M identity preserving and of Schwarz type, if its adjoint weak*-semigroup has these properties. Likewise, a pseudoresolvent $R$ on $D=\{\lambda \in \mathbb{C}: \operatorname{Re}(\lambda)>0\}$ with values in $M_{*}$ is called identity preserving and of Schwarz type, if $R^{\prime}$ has these properties.

Since for a semigroup of contractions on a Banach space

$$
\begin{aligned}
& \operatorname{Fix}(T)=n_{t \geqslant 0} \operatorname{ker}(\operatorname{Id}-T(t))= \\
& =\operatorname{ker}(I d-\lambda R(\lambda, A))=F i x(\lambda R(\lambda, A))
\end{aligned}
$$

for all $\lambda \in \mathbb{C}$ with $\operatorname{Re}(\lambda)>0$, it follows that a semigroup of contractions on $M$ is identity preserving if and only if the (pseudo)resolvent on $D=\{\lambda \in \mathbb{C}: \operatorname{Re}(\lambda)>0\}$ given by

$$
\mathrm{R}(\lambda):=\mathrm{R}(\lambda, \mathrm{~A}) \mid \mathrm{D}
$$

is identity preserving. By Corollary 2.2 an analogous statement holds for 'Schwarz type'.

\section*{3. INDUCTION AND REDUCTION}
(a) If $E$ is a Banach space and $S \subseteq L(E)$ a semigroup of bounded operators, then a closed subspace $F$ is called $S$-invariant, if $S F \subseteq F$ for all $S \in S$. We call the semigroup $S \mid:=\{S \mid F: S \in S$ the reduced semigroup. Note that for a one-parameter semigroup $T$ (resp., pseudoresolvent R , the reduced semigroup is again strongly continuous (resp. R| is again a pseudo-resolvent) (compare the construction in A-I, 3.2).\\
(b) Let $M$ be a $W^{*}$-algebra, $P \in M$ a projection and $S \in L(M)$ such that $S\left(p^{\perp} M\right) \subseteq p^{\perp} M$ and $S\left(M p^{\perp}\right) \subseteq M p^{\perp}$, where $p^{\perp}:=1-p$. Since for all $x \in M=$

$$
p[S(x)-S(p x p)]=p\left[S\left(p^{\perp} x p\right)+S\left(x p^{\perp}\right)\right] p=0
$$

we obtain $p(S x) p=p(S(p x p)) p$. Therefore the map

$$
S_{p}:=(x \rightarrow p(S x) p): p M p \rightarrow p M p
$$

is well defined. We call $s_{p}$ the induced map. If $S$ is an identity preserving schwarz map, then it is easy to see that $s_{p}$ is again a Schwarz map such that $S_{p}(p)=p$.

If $T=(T(t))_{t \geq 0}$ is a weak*-semigroup on $M$ which is of Schwarz type and if $T(t)\left(p^{\perp}\right) \leqq p^{\perp}$ for all $t \in \mathbb{R}_{+}$, then $T$ leaves $\mathrm{p}^{\perp} M$ and $\mathrm{Mp}^{\perp}$ invariant. It is easy to see that the induced semigroup $T_{p}=\left(T(t){ }_{p}^{\prime} t_{\geqq 0}\right.$ is again a weak*-semigroup.

If $R$ is an identity preserving pseudo-resolvent of Schwarz type on $D=\{\lambda \in \mathbb{C}: \operatorname{Re}(\lambda)>0\}$ with values in $M$ such that $R(\mu) p^{\perp} \leqq p^{\perp}$ for some $\mu \in \mathbb{R}_{+}$then $\mathrm{p}^{\perp} \mathrm{M}$ and $\mathrm{Mp}^{\perp}$ are R -invariant. Again, the induced pseudo-resolvent $R_{p}$ is of Schwarz type and identity preserving.\\
(c) Let $\phi$ be a positive normal linear functional on a W*-algebra M such that $\mathrm{T}_{\star} \phi=\phi$ for some identity preserving Schwarz map $T$ on $M$ with preadjoint $\mathrm{T}_{\star} \in L\left(\mathrm{M}_{\star}\right)$. Then $\mathrm{T}(\mathrm{s}(\phi) \perp) \leqq \mathrm{s}(\phi)^{\perp}$ where $s(\phi)$ is the support projection of $\phi$. To see this let $\mathrm{L}_{\phi}:=\left\{x \in M: \phi\left(x x^{*}\right)=0\right\}$ and $\mathrm{M}_{\phi}:=\mathrm{L}_{\phi} \cap \mathrm{L}_{\phi}^{*}$. Since $\phi$ is $\mathrm{T}_{\star}$-invariant, and T is a Schwarz map, the subspaces $\mathrm{L}_{\phi}$ and $\mathrm{M}_{\phi}$ are T -invariant. From $\mathrm{M}_{\phi}=$ $s(\phi)^{\perp} \mathrm{Ms}(\phi)^{\perp}$ and $\mathrm{T}\left(s(\phi)^{\perp}\right) \leqq 1$ it follows that $\mathrm{T}(\mathrm{s}(\phi) \perp) \leqq \mathrm{s}(\phi)^{\perp}$.

Let $T_{S(\phi)}$ be the induced map on $M_{S(\phi)}$. If

$$
s(\phi) M_{\star} s(\phi):=\left\{\psi \in M_{\star}: \psi=s(\phi) \psi s(\phi)\right\}
$$

where $\langle s(\phi) \psi s(\phi), x\rangle:=\langle\psi, s(\phi) x s(\phi)\rangle(x \in M)$, and if $\psi \in s(\phi) M_{\star} s(\phi)$, then for all $x \in M$ :

$$
\begin{gathered}
\left(\mathrm{T}_{*} \psi\right)(\mathrm{x})=\psi(\mathrm{Tx})=\langle\psi, \mathrm{s}(\phi)(\mathrm{Tx}) \mathrm{s}(\phi)\rangle= \\
=\langle\psi, \mathrm{s}(\phi)(\mathrm{T}(\mathrm{~s}(\phi) \mathrm{xs}(\phi))) \mathrm{s}(\phi)\rangle=\left\langle\mathrm{T}_{*} \psi, \mathrm{~s}(\phi) \mathrm{xs}(\phi)\right\rangle,
\end{gathered}
$$

hence $T_{*} \psi \epsilon_{s}(\phi) M_{*} s(\phi)$. Since the dual of $s(\phi) M_{*} s(\phi)$ is $M_{s}(\phi)$, it follows that the adjoint of the reduced map $\mathrm{T}_{*} \mid$ is identity preserving and of Schwarz type.

For example, if $T$ is an identity preserving semigroup of schwarz type on $M_{*}$ such that $\phi \in F i x(T)$, then the semigroup $T \mid\left(s(\phi) M_{*} s(\phi)\right)$ is again identity preserving and of Schwarz type. Furthermore, if R is a pseudo-resolvent on $D=\{\lambda \in \mathbb{C}: \operatorname{Re}(\lambda)>0\}$ with values in $M_{*}$ which is identity preserving and of Schwarz type such that $\mathrm{R}(\mu) \phi=\phi$ for some $\mu \in \mathbb{R}_{+}$, then $R \mid s(\phi) M_{\star} s(\phi)$ has the same properties.

\section*{CHARACTERIZATIONAOPIPOSTIIVE \\
 SEMIGROUPSON $O$ W*-AIGEBRAS }
Since the positive cone of a C*-algebra has non-empty interior many results of Chapter B-II can be applied verbatim to the characterization of the generator of positive semigroups on $C^{*}$-algebras. On the other hand a concret and detailed representation of such generators has been found only in the uniformly continuous case (see Lindblad (1976)).

A third area of active research has been the following: Which maps on c*-algebras (in particular, which derivations) commuting with certain automorphism groups are automatically generators of strongly continuous positive semigroups. For more informations we refer to the survey article of Evans (1984).

\section*{1. POSITIVE SEMIGROUPS ON PROPERLY INFINITE W*-ALGEBRAS}
The aim of this section is to show that strongly continuous semigroups of Schwarz maps on properly infinite w*-algebras are already uniformly continuous. In particular, our theorem is applicable to such semigroups on $B(H)$. It is worthwhile to remark, that the result of Lotz (1985) on the uniformly continuity of every strongly continuous semigroup on $L^{\infty}$ (see A-II, Sec.3) does not extend to arbitrary W*-algebras. For example, take $M=B(H), H$ infinite dimensional, and choose a projection $\mathrm{p} \in \mathrm{M}$ such that Mp is topologically isomorphic to $H$. Therefore $M=H \oplus M_{0}$, where $M_{0}=\operatorname{ker}(x \rightarrow x p)$. Next take $a$ strongly, but not uniformly continuous, semigroup $S$ on $H$ and consider the strongly continuous semigroup $S \oplus \mathrm{Id}$ on M .

For results from the classification theory of $\mathrm{W}^{*}$-algebras needed in our approach we refer to [Sakai (1971), 2.2] and [Takesaki (1979), V.1].

Theorem 1.1. Every strongly continuous one-parameter semigroup of Schwarz type on a properly infinite $\mathrm{W}^{*}$-algebra M is uniformly continuous.

Proof. Let $T=\left(T(t)_{\star}\right)_{t \geq 0}$ be strongly continuous on $M$ and suppose T not to be uniformly continuous. Then there exists a sequence ( $\mathrm{T}_{\mathrm{n}}$ ) $c T$ and $\varepsilon>0$ such that $\left\|T_{n}-I d\right\| \geq \varepsilon$ but $T_{n} \rightarrow I d$ in the strong operator topology. We claim that for every sequence ( $\mathrm{p}_{k}$ ) of mutually orthogonal projections and all bounded sequences $\left(x_{k}\right)$ in $M$

$$
\lim _{n}\left\|\left(T_{n}-I d\right)\left(p_{k} x_{k} p_{k}\right)\right\|=0
$$

uniformly in $k \in \mathbb{N}$. This follows from an application of the Lemma of Phillips and the fact that the sequence $\left(p_{k} x_{k} p_{k}\right)$ is summable in the $s^{*}\left(\mathrm{M}, \mathrm{M}_{*}\right.$ )-topology (compare Elliot (1972)).

Let $\left(p_{k}\right)$ be a sequence of mutually orthogonal projections in M such that every $p_{k}$ is equivalent to 1 via some $u_{k} \in M \quad$ [Sakai (1971), 2.21.

Without loss of generality we may assume $\left\|\left(T_{n}-I d\right)\left(u_{n}\right)\right\| \leqq n^{-1}$ since the semigroup $T$ is strongly continuous. Thus we obtained the following:\\
(1) $\quad \lim _{n}\left\|\left(\mathrm{~T}_{\mathrm{n}}-\mathrm{Id}\right)\left(\mathrm{p}_{\mathrm{k}} \mathrm{x}_{\mathrm{k}} \mathrm{p}_{\mathrm{k}}\right)\right\|=0$ uniformly in $k \in \mathbb{N}$ for every bounded sequence $\left(x_{k}\right)$ in $M$.\\
(2) Every projection $p_{k}$ is equivalent to 1 via some $u_{k} \in M$.\\
(3) $\left\|\left(T_{n}-I d\right) u_{n}\right\| \leqq n^{-1}$ for all $n \in \mathbb{N}$.

For the following construction see A-I, 3.6 and D-II, sec. 2 .

Let $\hat{M}$ be an ultrapower of $M$, let $p:=\left(p_{k}\right)^{\wedge} \in \hat{M}, T:=\left(T_{n}\right)^{\wedge} \epsilon L(\hat{M})$ and $u:=\left(u_{k}\right)^{\wedge} \in \hat{M}$. Then $T$ is identity preserving and of schwarz type on $\hat{M}$. since $u^{*} u=p$ and $u^{*}=1$, it follows pu* $=u^{*}$ and (uu*) $x\left(u^{*}\right)=x$ for all $x \in \hat{M}$. Finally, $T(p x p)=\operatorname{pxp}$ for all $x \in \hat{M}$, which follows from (1), and $T\left(u^{*}\right)=T\left(p u^{*}\right)=p u^{*}=u^{*}$ and $T(u)=u_{\text {) }}$ which follows from (3). Using the Schwarz inequality we obtain

$$
T\left(u u^{*}\right)=T(1) \leqq 1=u u^{*}=T(u) T(u)^{*} .
$$

Using D-III, Lemma 1.1. we conclude $T(u x)=u T(x)$ and $T\left(x u^{*}\right)=$ $T(x) u^{*}$ for all $x \in \hat{M}$. Hence

$$
T(x)=T\left(u u^{*} x u u^{*}\right)=u_{T}\left(u^{*} x u\right) u^{*}=u T\left(p u^{*} x u p\right) u^{*}=
$$

$$
=u_{p u}{ }^{\star} x u p u^{\star}=u u^{\star} x u u^{*}=x
$$

for all $x \in \hat{M}$. From this we obtain that for every bounded sequence $\left(x_{k}\right)$ in $M \quad \lim _{m}\left\|\mathrm{~T}_{\mathrm{m}} \mathrm{x}_{\mathrm{m}}-\mathrm{x}_{\mathrm{m}}\right\|=0$ for some subsequence of the $\mathrm{T}_{\mathrm{n}}$ 's and of the $x_{k}{ }^{3}$ s. This conflicts with our assumption at the beginning, hence the theorem is proved.

\section*{CHAPTER D-III}
SPECTRAI THEORY OF POSITIVE

SEMIGROUPSON

W* - A G G BRAS AND THEIR PREDUALS

Motivated by the classical results of Perron and Frobenius one expects the following spectral properties for the generator $A$ of a positive semigroup: The spectral bound $s(A):=\sup \{\operatorname{Re}(\lambda): \lambda \in \sigma(A)\}$ belongs to the spectrum $\sigma(A)$ and the boundary spectrum

$$
\sigma_{b}(A):=o(A) \cap\left\{s(A)+i^{\mathbb{R}}\right\}
$$

possesses a certain symmetric structure, called cyclicity.

Results of this type have been proved in Chapter B-III for positive semigroups on commutative $C^{*}$-algebras, but in the non-commutative case the situation is more complicated. While 's(A) $\epsilon_{\sigma}(\mathrm{A})$ ' still holds (see Greiner-Voigt-Wolff (1980)) the cyclicity of the boundary spectrum $\sigma_{b}(A)$ is true only under additional assumptions on the semigroup (e.g., irreducibility, see Section 1 below).

For technical reasons we consider mostly strongly continuous semigroups on the predual of a $W^{*-a l g e b r a} M$ or its adjoint semigroup which is a weak*-continuous semigroup on $M$.

\begin{enumerate}
  \item SPECTRAL THEORY FOR POSITIVE SEMIGROUPS ON PREDUALS
\end{enumerate}

The aim of this section is to develop a Perron-Frobenius theory for identity preserving semigroups of schwarz type on $\mathrm{W}^{*}$-algebras. But as we will show in the example preceding Theorem 1.1 below the boundary spectrum is no longer cyclic. The appropriate hypothesis on the semigroup implying the desired results seems to be the concept of irreducibility.

Let us first recall some facts on normal linear functionals. If $\phi$ is a normal linear functional on a $W^{*}$-algebra M then there exists a partial isometry $u \in M$ and a positive linear functional $|\phi| \in M_{*}$ such that

$$
\begin{gathered}
\phi(x)=|\phi|(x u)=:(u|\phi|)(x), x \in M \\
u * u=s(|\phi|),
\end{gathered}
$$

where $s(|\phi|)$ denotes the support projection of $|\phi|$ in $M$. We refer to this as the polar decomposition of $\phi$ [Takesaki (1979), Theorem III.4.2]. In addition, $|\phi|$ is uniquely determined by the following two conditions [Takesaki (1979), Proposition III.4.6]:

$$
\|\phi\|=\||\phi|\|,
$$

(*)

$$
|\phi(x)|^{2} \leqq|\phi|\left(x x^{*}\right) \quad(x \in M)
$$

For the polar decomposition of $\phi^{*}$, where $\phi^{*}(x)=\phi\left(x^{*}\right)^{*}$, we obtain

$$
\phi^{*}=u^{*}\left|\phi^{*}\right|,\left|\phi^{*}\right|=u|\phi| u^{*} \text { and } u^{*}=s\left(\left|\phi^{*}\right|\right) .
$$

It is easy to see that $u * \in s(|\phi|) M$.

If $\Psi$ is a subset of the state space of a C*-algebra $M$, then $\Psi$ is called faithful if $0 \leqq x \in M$ and $\psi(x)=0$ for all $\psi \in \Psi$ implies $\mathrm{x}=0$. $\Psi$ is called subinvariant for a positive map $T \in L(M)$ (resp., positive semigroup T ) if T' $\mathrm{T}^{\prime} \leqq \psi$ for all $\psi \in \psi$ (resp., $\mathrm{T}(t) ' \psi \leq \psi$ for all $T(t) \in T$ and $\psi \in \Psi$ ). Recall that for every positive map $T \in L(M)$ there exists a state $\phi$ on $M$ such that $T^{\prime} \phi=$ $r(T) \phi$ [Groh (1981), Theorem 2.1], where $r(T)$ denotes the spectral radius of T .

Let us start our investigation with two lemmas. Recall that Fix(T) is the fixed space of $T$, i.e. the set $\{\mathrm{x} \in \mathrm{M}$ : $\mathrm{Tx}=\mathrm{x}\}$.

Lemma 1.1. Suppose $M$ to be a $C^{*}$-algebra and $T \in L(M)$ an identity preserving Schwarz map.\\
(a) Let b: $M \times M \rightarrow M$ be a sesquilinear map such that for all $z \in M$ $b(z, z) \geqq 0$. Then $b(x, x)=0$ for some $x \in M$ if and only if $b(x, y)=0$ and $b(y, x)=0$ for all $y \in M$.\\
(b) If there exists a faithful family $\Psi$ of subinvariant states for $T$ on $M$, then Fix(T) is a $C^{*}$-subalgebra of $M$ and $T(x y)=x T(y)$ for all $x \in F i x(T)$ and $y \in M$.

Proof. (a) Take $0 \leqq \psi \in M^{*}$ and consider $f$ := $\psi \circ$. Then $f$ is a positive semidefinite sesquilinear form on $M$ with values in $\mathbb{C}$. From the Cauchy-Schwarz inequality it follows that $f(x, x)=0$ for some $x \in M$ if and only if $f(x, y)=0$ and $f(y, x)=0$ for all $y \in M$. Since the positive cone $M_{+}^{*}$ is generating, assertion (a) is proved.\\
(b) Since $T$ is positive it follows $T(x) *=T\left(x^{*}\right)$ for all $x \in M$. Hence $F i x(T)$ is a self adjoint subspace of M, i.e. invariant under the involution on $M$. For every $x, y \in M$ let

$$
b(x, y):=T\left(x y^{*}\right)-T(x) T(y) *
$$

Then b satisfies the assumptions of (a) . If $x \in F i x(T)$ then

\begin{verbatim}
0\leqq xx* = (Tx) (Tx)* \ T(xx*),
\end{verbatim}

hence

\begin{verbatim}
0\leqq \leqq(T(x\mp@subsup{x}{*}{*})-\textrm{xx*})\leqq0 for all \psi\in\Psi.
\end{verbatim}

But this implies $\mathrm{T}\left(\mathrm{xx} \mathrm{x}^{*}\right)=\mathrm{T}(\mathrm{x}) \mathrm{T}(\mathrm{x})^{*}=\mathrm{xx} *$. Consequently, $\mathrm{b}(\mathrm{x}, \mathrm{x})=0$. Hence $T\left(x y^{*}\right)=x T(y) *$ for all $y \in M$ and (b) is proved.

Lemma 1.2. Let $M$ be a $W^{*-a l g e b r a, ~} T$ an identity preserving Schwarz map on $M$ and $S \in L(M)$ such that $S(x)(S x) * \leqq T\left(x x^{*}\right)$ for every $x \in M$.\\
(a) If $v \in M$ such that $S\left(v^{*}\right)=v^{*}$ and $T\left(v^{*} v\right)=v^{*} v$, then $T(x v)=$ $s(x) v$ for all $x \in M$.\\
(b) Suppose there exists $\phi \in M_{*}$ with polar decomposition $\phi=u|\phi|$ such that $S_{*} \phi=\phi$ and $\mathrm{T}_{*}|\phi|=|\phi|$. If the closed subspace $\mathrm{s}(|\phi|) \mathrm{M}$ is T -invariant, then $\mathrm{Su}^{*}=\mathrm{u}^{*}$ and $\mathrm{T}\left(\mathrm{u}^{*} \mathrm{u}\right)=\mathrm{u}^{*} \mathrm{u}$.

Proof. (a) Define a positive semidefinite sesquilinear map $b: M \times M \rightarrow M$ by

$$
b(x, y):=T\left(x y^{*}\right)-S(x) S(y)^{*} \quad(x, y \in M)
$$

Since $b\left(v^{\star}, v^{*}\right)=0$ we obtain $b\left(x, v^{*}\right)=0$ for all $x \in M \quad$ (Lemma 1.1.a), hence $T(x v)=S(x) v$.\\
(b) Since s(|申|)M is a closed right ideal, the closed face $\mathrm{F}:=\mathrm{s}(|\phi|)\left(\mathrm{M}_{+}\right) \mathrm{s}(|\phi|)$ determines $\mathrm{s}(|\phi|) \mathrm{M}$ uniquely, i.e.,

$$
s(|\phi|) M=\left\{x \in M: x x^{\star} \in F\right\}
$$

[Pedersen (1979), Theorem 1.5.2]. Since $T$ is a Schwarz map and $\mathrm{s}(|\phi|) \mathrm{M}$ is T -invariant, it follows $\mathrm{TF} \subseteq \mathrm{F}$. On the other hand, if $x \in s(|\phi|) M$ then $x x^{*} \in F$. Consequently,

$$
0 \leqq S(x) S(x)^{\star} \leqq T\left(x x^{*}\right) \in F
$$

whence $S(x) \in s(|\phi|) M$.

Next we show $T\left(u^{*} u\right)=u^{*} u$ and $S u^{*}=u^{*}$. For this recall that $u * \in s(|\phi|) M$. First of all

$$
\begin{gathered}
0 \leqq\left(S u^{*}-u^{*}\right)\left(S u^{*}-u^{*}\right) * \leqq \\
\leqq T\left(u^{*} u\right)-u^{*} S\left(u^{*}\right) *-\left(S u^{*}\right) u+u^{*} u . \\
\text { Since } S_{\star} \phi=\phi, T_{*}|\phi|=|\phi| \text { and } \phi=u|\phi| \text { it follows } \\
0 \leqq|\phi|\left(\left(S u^{*}-u^{*}\right)\left(S u^{*}-u^{*}\right)^{*}\right) \leqq \\
\leqq 2|\phi|\left(u^{*} u\right)-|\phi|\left(S\left(u^{*}\right) u^{*}-|\phi|\left(S\left(u^{*}\right) u\right)=\right. \\
=2|\phi|\left(u^{*}\right)-\phi\left(u^{*}\right) *-\phi\left(u^{*}\right)= \\
=2\left(|\phi|\left(u^{*} u\right)-|\phi|\left(u^{*} u\right)\right)=0 .
\end{gathered}
$$

Since $\left(S u^{*}-u^{*}\right)\left(S u^{*}-u^{*}\right) \in F$ and $|\phi|$ is faithful on $F$ we obtain Su* $=u^{*}$. Consequently,

$$
0 \leqq u^{*} u=\left(S u^{*}\right)\left(S u^{*}\right)^{*} \leqq T\left(u^{*} u\right)
$$

Hence $T(u * u)=u * u$ by the faithfulness and $T$-invariance of $|\phi|$.

Remark 1.3. Take $S$ and $T$ as in Lemma 1.2 (b). If $V_{u^{*}}$ (resp. $\mathrm{V}_{\mathrm{u}}$ ) is the map ( $\mathrm{x} \rightarrow \mathrm{xu}^{*}$ ) (resp. $(\mathrm{x} \rightarrow \mathrm{xu})$ ) on $M$, then $\mathrm{V}_{\mathrm{u}}$ is a continuous bijection from $\mathrm{Ms}(|\phi|)$ onto $\mathrm{Ms}\left(\left|\phi^{*}\right|\right)$ with inverse $\mathrm{V}_{\mathrm{u}}$ (because $V_{u}{ }^{\circ} V_{u}^{*}=I d_{M s}(|\phi|)$ and $V_{u}{ }^{*} \circ V_{u}=I d_{M s}\left(\left|\phi^{*}\right|\right)$ ). Let $x \in M$. From $T(x u)=S(x) u$ we obtain $T(x u) u^{*}=S(x) u u^{*}$. In particular, if Ms(| $\left.\phi^{*} \mid\right)$ is S-invariant, then

$$
\left(V_{u *} \circ T \circ V_{u}\right)(x)=T(x u) u^{*}=S(x)
$$

for every $x \in M s(|\phi *|)$. Let $T \mid$ (resp. S $\left.\right|^{\prime}$ ) be the restriction of $T$ to Ms $(|\phi|)$ (resp. of $s$ to $\mathrm{Ms}\left(\left|\phi^{*}\right|\right)$ ). Then the following diagram is commutative :\\
\includegraphics[max width=\textwidth, center]{2024_12_23_c6487cc0859199a15bd9g-392}

In particular, $\sigma\left(S_{\mid}\right)=\sigma(\mathrm{T} \mid)$. Therefore we may deduce spectral properties of S | from $\mathrm{T} \mid$ and vice versa. More concrete applications of Lemma 1.2. will follow.

We now investigate the fixed space $\operatorname{Fix}(\mathrm{R}):=F i x(\lambda R(\lambda)), \lambda \in \mathrm{D}$, of a pseudo-resolvent $R$ with values in the predual of a $\mathrm{W}^{*}$-algebra $M$.

Proposition 1.4. Let $\mathbb{R}$ be a pseudo-resolvent on $D=\{\lambda \in \mathbb{C}: \operatorname{Re}(\lambda)>0\}$ with values in the predual $M_{*}$ of a $W^{*}$-algebra $M$ and suppose $R$ to be identity preserving and of Schwarz type.\\
(a) If $\alpha \in \mathbb{R}$ and $\psi \in M_{*}$ such that $(\gamma-i \alpha) R(\gamma) \psi=\psi$ for some $\gamma \in D$, then $\lambda R(\lambda)|\psi|=|\psi|$ and $\lambda R(\lambda)|\psi *|=|\psi *|$ for all $\lambda \in D$.\\
(b) Fix(R) is invariant under the involution in $M_{*}$. If $\psi \in \mathrm{Fix}(\mathrm{R})$ is self adjoint, then the positive part $\psi^{+}$and the negative part $\psi^{-}$of $\psi$ are elements of Fix(R) .

Proof. If $(\gamma-i \alpha) R(\gamma) \psi=\psi$ then $(\lambda-i \alpha) R(\lambda) \psi=\psi$ for all $\lambda \in D$. In particular, $\left.\mu R(\mu+i \alpha) \psi=\psi(\mu \in \mathbb{R})_{+}\right)$. For all $x \in M$ we obtain

$$
\begin{gathered}
|\psi(x)|^{2}=\mid\left\langle\mu R(\mu+i \alpha)^{\prime} x, \psi>\left.\right|^{2} \leqq\right. \\
\left.\leqq\|\psi\|<\left(\mu R(\mu+i \alpha x)^{\prime} x\right)\left(\mu R(\mu+i \alpha x)^{\prime} x\right)^{*}, \psi\right\rangle \leqq \\
\leqq\|\psi\|<\mu R(\mu)^{\prime}\left(x x^{\star}\right),|\psi|>
\end{gathered}
$$

(D-I, Corollary 2.2). Since

$$
\begin{gathered}
\|\psi\|=\||\psi|\|=|\psi|(1)= \\
=\left\langle\mu R(\mu)^{\prime} 1,\right| \psi| \rangle=\|\mu R(\mu)|\psi|\|
\end{gathered}
$$

we obtain $\mu R(\mu)|\psi|=|\psi|$ by the uniqueness theorem (*) mentioned at the beginning. Therefore $|\psi| \in \mathrm{Fix}(\mathrm{R})$. Since

$$
0 \leqq\left(\mu R(\mu)^{\prime} x\right)\left(\mu R(\mu)^{\prime} x\right)^{*} \leqq \mu R(\mu)^{\prime} x x^{*}
$$

the map $R(\mu)$ is positive. Consequently $(\mu+i \alpha) R(\mu) \psi *=\psi^{*}$ from which $|\psi *| \in F i x(R)$ follows.\\
If $\phi \in \mathrm{Fix}(\mathrm{R})$ is selfadjoint with Jordan decomposition $\phi=\phi^{+}{ }_{-}^{-{ }^{-}}$, then $|\phi|=\phi^{+}+\phi^{-}$[Takesaki (1979), Theorem III.4.2.]. From this we obtain that $\phi^{+}$and $\phi^{-}$are in Fix(R).

Corollary 1.5. Let $T$ be an identity preserving semigroup of Schwarz type on $M_{*}$ with generator $A$ and suppose $\operatorname{Po}(A) \cap$ iR $\neq \emptyset$.\\
(a) If $a \in \mathbb{R}$ and $\psi \in \operatorname{ker}(i \alpha-A)$, then $|\psi|$ and $|\psi *|$ are elements of $F i x(T)=\operatorname{ker}(A)$.\\
(b) Fix(T) is invariant under the involution of $\mathrm{M}_{*}$. If $\psi \in \mathrm{Fix}(T)$ is self adjoint, then the positive part $\psi^{+}$and the negative part $\psi^{-}$of $\psi$ are elements of Fix(T).

The proof follows immediately from $D-I$, Corollary 2.2 and the fact that $\operatorname{ker}(A)=\operatorname{Fix}(\lambda R(\lambda, A))$ for all $\lambda \in \mathbb{C}$ with $\operatorname{Re}(\lambda)>0$.

If $T$ is the semigroup of translations on $L^{1}(\mathbb{R})$ and $A^{\prime}$ the gene-\\
rator of the adjoint weak*-semigroup, then $\operatorname{Po}(\mathrm{A}) \cap \mathrm{iR}=\varnothing$, while Po( $\left.A^{\prime}\right) \cap i \mathbb{R}=i \mathbb{R}$. For that reason we cannot expect a simple connection between these two sets. But as we shall see below, if a semigroup on the predual of a $\mathrm{W}^{*}$-algebra has sufficiently many invariant states, then the point spectra of $A$ and $A^{\prime}$ contained in $i \mathbb{R}$ are identical. Helpful for these investigations will be the next lema.

Lemma 1.6. Let $R$ be a pseudo-resolvent on $D=\{\lambda \in \mathbb{C}: \operatorname{Re}(\lambda)>0\}$ with values in a Banach space $E$ such that $\| \mu R(\mu+i \alpha) \leqq 1$ for all $(\mu, \alpha) \in \mathbb{R}_{+} \times \mathbb{R}$. Then

\begin{verbatim}
dim Fix(\lambdaR(\lambda + i\alpha)) \leqq dim Fix(\lambdaR(\lambda + i\alpha)')
\end{verbatim}

for all $\lambda \in D$.

Proof. Let $(\mu, \alpha) \in \mathbb{R}_{+} \times \mathbb{R}$ and $S:=\mu R(\mu+i \alpha)$. Since $S$ is a contraction, its adjoint $S^{\prime}$ maps the dual unit ball $E^{\prime}{ }_{1}$ into itself. Let $U$ be a free ultrafilter on $[1, \infty)$ which converges to 1 . since $E^{\prime}{ }_{1}$ is $\sigma\left(E^{\prime}, E\right)$-compact,

$$
\psi_{0}:=\lim _{U}(\lambda-1) R(\lambda, s) \cdot \psi
$$

exists for all $\psi \in E^{\prime}{ }_{1}$. Since $S^{\prime}$ is $\sigma^{\prime}\left(E^{\prime}, E\right)$-continuous and since $S^{\prime} R(\lambda, S)^{\prime}=\lambda R\left(\lambda, S^{\prime}\right)-I d$ we conclude $\psi_{0} \in \mathrm{Fix}^{\prime}\left(S^{\prime}\right)$.

Take now $0 \neq x_{0} \in \mathrm{Fix}(\mathrm{S})$ and choose $\psi \in \mathrm{E}_{1}{ }_{1}$ such that $\psi\left(\mathrm{x}_{0}\right)$ is different from zero. From the considerations above it follows

$$
\psi_{0}\left(x_{0}\right)=1 i m_{U}^{(\lambda-1)} \psi\left(\mathrm{R}(\lambda, s) x_{0}\right)=\psi\left(x_{0}\right) \neq 0
$$

hence $0 \neq \psi_{0} \in F i x(S)$. Therefore Fix(S') separates the points of Fix(s) . From this it follows that

\begin{verbatim}
dim Fix(S) \leqq dim Fix(S') •
\end{verbatim}

Since $R$ and $R^{\prime}$ are pseudo-resolvents, the assertion is proved.

Corollary 1.7. Let $T$ be a semigroup of contractions on a Banach space E with generator A . Then

$$
\operatorname{dim} \operatorname{ker}(i \alpha-A) \leqq \operatorname{dim} \operatorname{ker}\left(i \alpha-A^{\prime}\right)
$$

for all $\alpha \in \mathbb{R}$.

This follows from Lemma 1.6 because $\operatorname{Fix}(\lambda R(\lambda+i \alpha))=\operatorname{ker}(i \alpha-A)$.

Proposition 1.8. Let $T$ be an identity preserving semigroup of Schwarz type with generator $A$ on the predual of a $W^{*}$-algebra and suppose that there exists a faithful family $\Psi$ of T-invariant states. Then for all $\alpha \in \mathbb{R}$ we have

$$
\operatorname{dim} \operatorname{ker}(i \alpha-A)=\operatorname{dim} \operatorname{ker}\left(i \alpha-A^{\prime}\right)
$$

and

$$
\operatorname{Po}(A) \cap i \mathbb{R}=\operatorname{Po}\left(A^{\prime}\right) \cap i R \text {. }
$$

Proof. The inequality dim ker(ia - A) $\leqq$ dim ker(ia - A') follows from Corollary 1.7 .

Let $D=\{\lambda \in \mathbb{C}: \operatorname{Re}(\lambda)>0\}$ and $R$ the pseudo-resolvent induced by $R(\lambda, A)$ on $D$. Then $R$ is identity preserving and of schwarz type. Take iaєPo(A) ( $\alpha \in \mathbb{R})$ and choose $0<\mu \in \mathbb{R}$. If $\psi_{\alpha} \in M_{*}$ is of norm one with polar decomposition $\psi_{\alpha}=u_{\alpha}\left|\psi_{\alpha}\right|$ such that $\psi_{\alpha}=(\mu-i \alpha) R(\mu) \psi_{\alpha}$ then $\mu R(\mu)\left|\psi_{\alpha}\right|=\left|\psi_{\alpha}\right| \quad$ (Proposition 1.4.a). Since

$$
\mu R(\mu)^{\prime}\left(1-s\left(\left|\psi_{\alpha}\right|\right)\right) \leqq 1-s\left(\left|\psi_{\alpha}\right|\right)
$$

we obtain $\mu R(\mu) ' s\left(\left|\psi_{\alpha}\right|\right)=s\left(\left|\psi_{\alpha}\right|\right)$ by the faithfulness of $\Psi$. Hence the maps $s:=(\mu-i \alpha) R(\mu)$ and $T:=\mu R(\mu)$ ' fulfil the assumptions of Lemma 1,2.b. Therefore $5 u_{\alpha}{ }^{*}=u_{\alpha}{ }^{*}$ or $(\mu-i \alpha) R(\mu)^{\prime} u_{\alpha}{ }^{*}=u_{\alpha}{ }^{*} \quad$ which implies $u_{\alpha}^{*} \in D\left(A^{\prime}\right)$ and $A^{\prime} u_{\alpha}^{*}=i \alpha u_{\alpha}{ }^{*}$.

If $i \alpha \in \operatorname{Po}\left(\mathrm{~A}^{\prime}\right), \alpha \in \mathbb{R}$, choose $0 \neq \mathrm{v}_{\alpha}$ such that

$$
v_{\alpha}=(\mu-i \alpha) R(\mu)^{\prime} v_{\alpha} \quad\left(\mu \in \mathbb{R}_{+}\right)
$$

and $\psi \in \Psi$ such that $\psi\left(v_{\alpha} v_{\alpha}^{*}\right) \neq 0$. Since

$$
\begin{gathered}
0 \leqq v_{\alpha} v_{\alpha}^{*}=\left((\mu-i \alpha) R(\mu)^{\prime} v_{\alpha}\right)\left((\mu-i \alpha) R(\mu)^{\prime} v_{\alpha}\right) * \leqq \\
\leqq \mu R(\mu)^{\prime}\left(v_{\alpha} v_{\alpha}^{*}\right),
\end{gathered}
$$

we obtain $\mu R(\mu)^{\prime}\left(v_{\alpha} v_{\alpha}{ }^{*}\right)=v_{\alpha} v_{\alpha} *$ because $\Psi$ is faithful.

Hence from Lemma 1.2.a it follows

$$
\mu R(\mu)^{\prime}\left(x v_{\alpha}^{*}\right)=\left(\left(\mu-i_{\alpha}\right) R(\mu)^{\prime} x\right) v_{\alpha}^{*}
$$

for all $x \in M$. Let $\psi_{\alpha}$ be the normal linear functional $\left(x \rightarrow \psi\left(x v_{\alpha}^{*}\right)\right)$ on $M$ and note that $\psi_{\alpha}\left(v_{\alpha}\right) \neq 0$. Then

$$
\begin{gathered}
<x,(\mu-i \alpha) R(\mu) \psi_{\alpha}>=\left\langle\left((\mu-i \alpha) R(\mu)^{\prime} x\right) v_{\alpha}^{*}, \psi\right\rangle= \\
<\mu R(\mu)^{\prime}\left(x v_{\alpha}^{*}\right), \psi>=\psi\left(x v_{\alpha}^{*}\right)=\psi_{\alpha}(x)
\end{gathered}
$$

for all $x \in M$. Consequently ia $\in \operatorname{Po}(\mathrm{A})$ and

$$
\operatorname{dim} \operatorname{ker}\left(i_{\alpha}-A^{\prime}\right) \leqq \operatorname{dim} \operatorname{ker}(i \alpha-A)
$$

which proves the assertion.

Remark 1.9. From the above proof we obtain the following: If $0 \neq \psi_{\alpha} \in \operatorname{ker}\left(i_{\alpha}-A\right)$ with polar decomposition $\psi_{\alpha}=u_{\alpha}!\psi_{\alpha} \mid(\alpha \in \mathbb{R})$ then $A^{\prime} u_{\alpha}=i_{\alpha} u_{\alpha}$. Conversely, if $0 \neq v_{\alpha} \in \operatorname{ker}\left(i_{\alpha}-A^{\prime}\right)$, then there exists $\psi \in \Psi$ such that $\psi\left(v_{\alpha} v_{\alpha}^{*}\right) \neq 0$ and the normal linear form

$$
\psi_{\alpha}:=\left(x \rightarrow \psi\left(x v_{\alpha}{ }^{*}\right)\right)
$$

is an eigenvector of A pertaining to the eigenvalue ia.

If $T$ is a $C_{0}$-semigroup of Markov operators on a commutative C*-algebra with generator A, it has been shown in B-III, that the boundary spectrum $\sigma(A) n \quad i \mathbb{R}$ of its generator is additively cyclic. This is no longer true in the non commutative case:

For $0 \neq \lambda \in i \mathbb{R}$ and $t \in \mathbb{R}$ let

$$
u_{t}:=\left(\begin{array}{ll}
1 & 0 \\
0 & e^{\lambda t}
\end{array}\right) \quad \in M_{2}(\mathbb{C})
$$

The semigroup of *-automorphisms $\left(x \rightarrow u_{t} x u_{t} *\right)$ on $M_{2}(\mathbb{C})$ is identity preserving and of Schwarz type but the spectrum of its generator is $\left\{0, \lambda, \lambda^{*}\right\}$ hence is not additively cyclic.

It turns out that, in order to obtain a non commutative analogue of the Perron-Frobenius theorems, one has to consider semigroups which are irreducible. Recall that a semigroup $S$ of positive operators on an ordered Banach space $\left(E, E_{+}\right)$is called irreducible if no closed face of $E_{+}$, different from $\{0\}$ and $E_{+}$, is invariant under $S$. Here a face $F$ in $E$ is a subcone of $E_{+}$such that the conditions $0 \leqq x \leqq Y, x \in E, Y \in F$ imply $x \in F$ (compare Definitions 3.1 in B-III and C-III).

In the context of $W^{*}$-algebras $M$ we call a semigroup $S$ of positive maps on $M$ weak*-irreducible, if no $\sigma\left(M_{*} M_{*}\right.$-closed face of $M_{+}$is $S$-invariant. Since the norm closed faces of $M_{*}$ and the $\sigma\left(M_{1} M_{*}\right)$ closed faces of $M$ are related by formation of polars with respect to the dual system $\left\langle\mathrm{M}, \mathrm{M}_{*}\right\rangle$ (see [Pedersen (1979), Theorem 3.6.11 and Theorem 3.10.7.1) a semigroup $S$ is (norm) irreducible on $M_{*}$ if and only if its adjoint semigroup is weak*-irreducible.

Theorem 1.10. Let $T$ be an irreducible, identity preserving semigroup of Schwarz type with generator $A$ on the predual of a W*-algebra and suppose $P_{\circ}(A) \cap i \mathbb{R} \neq \emptyset$.\\
(a) The fixed space of $T$ is one dimensional and spanned by a faithful normal state.\\
(b) $P_{\sigma}(A) \cap$ i $i s$ an additive subgroup of $i \mathbb{R}$,

$$
\sigma(A)=\sigma(A)+\left(P_{\sigma}(A) \cap i \mathbb{R}\right)
$$

and every eigenvalue in $i \mathbb{R}$ is simple.\\
(a)* The fixed space of the adjoint weak*-semigroup T' is onedimensional.\\
(b)* $P_{G}\left(A^{\prime}\right) \cap i R=P_{G}(A) \cap i R$ for the generator $A^{\prime}$ of the adjoint semigroup, and every $y_{\operatorname{P}} \mathrm{P}^{\left(A^{\prime}\right)} \cap \mathrm{iR}$ is simple.

Proof. Since $\operatorname{Pof}_{\sigma}(A) \cap$ iR $\neq \varnothing$ there exists $\psi \in F i x(T)+$ of norm one (Corollary 1.5). If $\mathrm{F}:=\left\{\mathrm{x} \in \mathrm{M}_{+}: \psi(\mathrm{x})=0\right\}$ then F is a $\sigma\left(\mathrm{M}_{\mathrm{H}} \mathrm{M}_{*}\right.$ ) closed, T'-invariant face in $M$, hence $F=\{0\}$. Therefore every $0 \neq \psi \in \mathrm{Fix}(T)+$ is faithful. Let $\psi_{1}, \psi_{2} \in \mathrm{Fix}(T)+$ be states such that\\
$\mathrm{f}:=\psi_{1}-\psi_{2}$ is different from zero. If $\mathrm{f}=\mathrm{f}^{+}-\mathrm{f}^{-}$is the Jordan decomposition of $f$, then $f^{+}$and $f^{-}$are elements of $F i x(T)$, whence faithful. Since the support projections of these two normal linear functionals are orthogonal, we obtain $\mathrm{f}^{+}=0$ or $\mathrm{f}^{-}=0$ which implies $\psi_{1} \leqq \psi_{2}$ or $\psi_{2} \leqq \psi_{1}$. Consequently $\psi_{2}=\psi_{1}$. Since Fix $(T)$ is positively generated (Corollary 1.5), $\operatorname{Fix}(T)=\mathbb{C} \phi$ for some faithful normal state $\phi$.

Let $\mu \in \mathbb{R}_{+}$and $\alpha \in \mathbb{R}$ such that $i \alpha \in \operatorname{P\sigma }(A)$. If $\psi_{\alpha}=u_{\alpha}\left|\psi_{\alpha}\right|$ is a normalized eigenvector of $A$ pertaining to $i \alpha$, then $\phi=\left|\psi_{\alpha}\right|=\left|\psi_{\alpha}{ }^{*}\right|$ by Corollary 1.5 and the above considerations. Hence $u_{\alpha} u_{\alpha}{ }^{*}=u_{\alpha}{ }^{*} u_{\alpha}=$ $s(\phi)=1$. Since

$$
(\mu-i \alpha) \mathrm{R}(\mu, \mathrm{~A}) \psi_{\alpha}=\psi_{\alpha}
$$

and

$$
\mu \mathrm{R}(\mu, \mathrm{~A})\left|\psi_{\alpha}\right|=\left|\psi_{\alpha}\right|
$$

we obtain by Lemma 1.2.b that

$$
\text { (1) } \quad \mu \mathrm{R}(\mu, \mathrm{~A})=\mathrm{V}_{\alpha} \circ \mu \mathrm{R}(\mu+i \alpha, \mathrm{~A}) \circ \mathrm{V}_{\alpha}^{-1} \text {, }
$$

where $V_{\alpha}$ is the map $\left(\mathrm{x} \rightarrow \mathrm{xu}_{\alpha}\right.$ ) on M . Similarly for i $\mathrm{B} \in \operatorname{P\sigma }(\mathrm{A})$, we find $v_{B}$ such that $1=u_{B} u_{B}{ }^{*}=u_{B} u_{B}{ }^{*}$ and\\
(2) $\mu R(\mu, A)=V_{\beta} \circ \mu R(\mu+i \beta, A) \circ V_{\beta}^{-1}$.

Hence

$$
\text { (3) } \quad \mu R(\mu, A)=V_{\alpha \beta} \circ \mu R(\mu+i(\alpha+\beta), A) \circ V_{\alpha \beta}^{-1} \text {, }
$$

where $V_{\alpha \beta}:=V_{\alpha} \circ V_{\beta}$. Since $u_{\alpha}$ is unitary in $M$, it follows from (1) that ia is an eigenvalue which is simple because Fix(T)= Fix ( $\mu R(\mu, A))$ is one dimensional. From (3) it follows that $i(\alpha+\beta) \in P \sigma(A)$ since $0 \in P \sigma(A)$ and $V_{\alpha \beta}$ is bijective. From the identity (1) we conclude that $\sigma(R(\mu, A))=\sigma(R(\mu+i \alpha))$, which proves

$$
\sigma(A)+(P \sigma(A) \cap i \mathbb{R}) \subseteq \sigma(A) .
$$

The other inclusion is trivial since $0 \in \mathrm{P}_{\sigma}(\mathrm{A})$.

Remarks 1.11. (a) Let $\phi$ be the normal state on $M$ such that $\operatorname{Fix}(T)=\mathbb{C}_{\phi}$ and let $H:=P_{G}(A) \cap i \mathbb{R}$. From the proof of Theorem 1.10 it follows that there exists a family $\left\{u_{n}: n \in H\right\}$ of unitaries in $M$ such that $A^{\prime} u_{\eta}=-\eta u_{\eta}$ and $A\left(u_{\eta} \phi\right)=n\left(u_{\eta} \phi\right)$ for all $n \in H$.\\
(b) If the group $H$ is generated by a single element, i.e., $H=i \gamma \mathbf{Z}$ for some $\gamma \in \mathbb{R}$ then the family $\left\{u_{\gamma}{ }^{k} ; k \in \mathbb{Z}\right\}$ is a complete family of eigenvectors pertaining to the eigenvalues in $H$, where $u_{\gamma} € M$ is unitary such that $A^{\prime} u_{\gamma}=i_{\gamma} u_{\gamma}$.

Proposition 1.12. Suppose that $T$ and $M$ satisfy the assumptions of Theorem 1.10, and let $N_{\star}$ be the closed linear subspace of $M_{*}$ generated by the eigenvectors of $A$ pertaining to the eigenvalues in $i \mathbb{R}$. Denote by $T_{0}$ the restriction of $T$ to $N_{\star}$. Then\\
(a) $\mathrm{G}:=\left(T_{0}\right)^{-} \subseteq L_{S}\left(N_{*}\right)$ is a compact, Abelian group.\\
(b) $I d \mid N_{\star} \in\left\{T_{0}(t): t>s\right\}^{-} \subseteq L_{s}\left(N_{\star}\right)$ for all o<s $\quad \in \mathbb{R}$.

Proof. For $n \in H:=P_{\sigma}(A) \cap i R$ let

$$
U(n):=\{\psi \in D(A): A \psi=n \psi\}
$$

and $U=\{U(\eta): \eta \in H\}$. Then $(\operatorname{span} U)^{-}=N_{*}$. For each $\psi \in U$ there exists ${ }_{n} \in \mathrm{H}$ such that

$$
\left\{\mathbb{T}_{0}(t) \psi: t \in \mathbb{R}_{+}\right\}=\left\{e^{-n t} \psi: t \in \mathbb{R}_{+}\right\}
$$

Consequently this set is relatively compact in $L_{s}\left(N_{\star}\right)$. From [Schaefer (1966), III.4.5] we obtain that $G$ is compact.

Next choose $\psi_{1}, \ldots, \psi_{n} \in U, 0<s \in \mathbb{R}$ and $\delta>0$, since $T_{0}(t) \psi_{i}=$ $e^{\eta_{i}}{ }_{\phi_{i}}(1 \leq i \leq n)$ for some $\eta_{i} \in H$, it follows from a theorem of Kronekker (see, [Jacobs (1976), Satz 6.1., p.77]) that there exists $s<t$ such that

$$
\left|(1,1, \ldots, 1)-\left(e^{n_{1}}{ }^{t}, \quad e^{n_{2} t}, \ldots, e^{n_{n}^{t}}\right)\right|<\delta
$$

hence

$$
\sup \left\{\left\|\psi_{i}-T_{0}(t) \Psi_{i}\right\|: 1 \leqq i \leqq n\right\}<\delta
$$

or $\operatorname{Id} \mid N_{\star} \in\left\{T_{O}(t): t>s\right\}^{-} \subseteq L_{S}\left(N_{\star}\right)$.

Finally we prove the group property of $G$. Let $V$ be an ultrafilter on $\mathbb{R}$ such that $\lim _{v} \mathrm{~T}_{0}(t)=I d$ in the strong operator topology. For positive $s \in \mathbb{R}$ let $\mathrm{S}:=\lim _{V} \mathrm{~T}(t-s)$. Then $\operatorname{ST}_{\mathrm{O}}(\mathrm{s})=\mathrm{T}_{\mathrm{O}}(\mathrm{s}) \mathrm{s}=\mathrm{Id}$, hence $\mathrm{T}_{\mathrm{O}}(\mathrm{s})^{-1}$ exists in $G$ for all $s \in \mathbb{R}_{+}$. From this it follows that $G$ is a group.

Remark 1.13. (a) Let $k: R \rightarrow G$ be given by

$$
K(t)=\left\{\begin{array}{cc}
T_{0}(t) & \text { if } 0 \leqq t \\
T_{0}(t)^{-1} & \text { if } t \leqq 0
\end{array} .\right.
$$

Then $k$ is a continuous homomorphism with dense range, i.e. (G,k) is solenoidal (see [Hewitt-Ross (1963)]).\\
(b) The compact group $G$ and the discret group $P o(A) \cap i R$ are dual in the sense of locally compact Abelian groups.\\
(c) Let $(G, K)$ be a solenoidal compact group and let $\mathrm{N}_{\star}=\mathrm{L}^{1}(\mathrm{G})$. Then the induced lattice semigroup $T=(k(t))_{t \geq 0}$ fulfils the assertions of Theorem 1.10. For example, if $G$ is the dual of $\mathbb{R}_{\mathrm{d}}$, then $\operatorname{P\sigma }(\mathrm{A}) \cap i \mathbb{R}=i \mathbb{R}$. Since the fixed space of $k(t)$ is given by

$$
\operatorname{Fix}(k(t))=\left(\operatorname{span} U_{k \in \mathbb{Z}} \operatorname{ker}\left(\frac{2 \pi i k}{t}-A\right)\right)^{--},
$$

no $T(t) \in T$ is irreducible.\\
(d) If $T$ is the irreducible semigroup of Schwarz type on the predual of $\mathrm{B}(\mathrm{H})$ given in [Evans (1977)] then $\mathrm{Po}(\mathrm{A}) \mathrm{n} \mathrm{iR}=\emptyset$.\\
2. SPECTRAL PROPERTIES OF UNIFORMLY ERGODIC SEMIGOUPS

The aim of this section is the study of spectral properties of semigroups which are uniformly ergodic, identity preserving and of schwarz type. For the basic theory of uniformly ergodic semigroups on Banach spaces we refer to Dunford-Schwartz (1958).

Our first result yields an estimate for the dimension of the eigenspaces pertaining to eigenvalues of a pseudo-resolvent.

Proposition 2.1. Let R be an identity preserving pseudo-resolvent of Schwarz type on $D=\{\lambda \in \mathbb{C}: \operatorname{Re}(\lambda)>0\}$ with values in the predual of a W*-algebra M . If Fix(AR( $\lambda$ ) is finite dimensional for some $\lambda \in D$, then

\begin{verbatim}
dim Fix((\gamma - ia)R(\gamma)) \leqq dim Fix(\lambdaR(\lambda))
\end{verbatim}

for all $\gamma \in D$ and $\alpha \in \mathbb{R}$.

Proof. By D-IV, Remark 3.2.c we may assume without loss of generality that there exists a faithful family of R-invariant normal states on $M$. In particular the fixed space $N$ of the adjoint pseudoresolvent $R^{\prime}$ is a $W^{\star}-s u b a l g e b r a$ of $M$ with $1 \in N$ (by Lemma 1.1.b). Since N is finite dimensional there exist a natural number n and a set $\mathrm{P}:=\left\{\mathrm{p}_{1}, \ldots, \mathrm{p}_{\mathrm{n}}\right\}$ of minimal, mutually orthogonal projections in N such that $\sum_{k=1}^{n} \mathrm{p}_{\mathrm{k}}=1$. These projections are also mutually orthogonal in $M$ with sum 1 .

Let $R_{j}$ be the $\sigma\left(M, M_{*}\right)$-closed right ideal $P_{j} M$ and $L_{j}$ the closed left invariant subspace $M_{*} \mathrm{P}_{j}(1 \leqq j \leqq n)$. The map $\mu R(\mu) ', \mu \in \mathbb{R}_{+}$is an identity preserving Schwarz map . From Lemma 1.l.b we therefore obtain that for all $x \in N$ and $y \in M$,

$$
\mu R(\mu)^{\prime}(x y)=x\left(\mu R^{\prime}(\mu) y\right) .
$$

In particular, $R_{j}$, resp., $L_{j}$ are invariant under $R^{\prime}$, respectively, $R$. Furthermore, if $\psi \in \mathrm{L}_{j}$ with polar decomposition $\psi=u|\psi|$, then $u^{*} u=s(|\psi|) \leq p_{j}$. Consequently, $|\psi| \epsilon_{j}$. Let now $\alpha \in \mathbb{R}$ and suppose that there exists $\psi_{\alpha} \in \mathrm{L}_{j}$ of norm $1, \psi_{\alpha}=u_{\alpha}\left|\psi_{\alpha}\right|$, such that

$$
\psi_{\alpha} \in F i x((\lambda-i \alpha) R(\lambda)), \lambda \in D .
$$

Since $\lambda \mathbb{R}(\lambda)\left|\psi_{\alpha}\right|=\left|\psi_{\alpha}\right| \quad$ (Proposition 1.4), we obtain

$$
\mu R(\mu)^{\prime}\left(1-s\left(\left|\psi_{\alpha}\right|\right)\right) \leqq\left(1-s\left(\left|\psi_{\alpha}\right|\right), \mu \in \mathbb{R}_{+} .\right.
$$

From the existence of a faithful family of $R$-invariant normal states and since $R^{\prime}$ is identity preserving it follows that

$$
\mu R(\mu)^{\prime} s\left(\left|\psi_{\alpha}\right|\right)=s\left(\left|\psi_{\alpha}\right|\right) \text {. }
$$

Thus $s\left(\left|\psi_{\alpha}\right|\right) \leqq p_{j}$ and even $s\left(\left|\psi_{\alpha}\right|\right)=p_{j}$ by the minimality property of $p_{j}$. On the other hand, $\psi_{\alpha} * \operatorname{EFix}((\lambda+i \alpha) R(\lambda))$. As above we obtain

$$
\mu R(\mu)^{\prime} s\left(\left|\psi_{\alpha} *\right|\right)=s\left(\left|\psi_{\alpha} *\right|\right)
$$

Consequently, the closed left ideals $\operatorname{Ms}\left(\mid \psi_{\alpha} *\right)$ and $\operatorname{Ms}\left(\left|\psi_{\alpha}\right|\right)$ are R'-invariant.

Next fix $\mu_{+} \in \mathbb{R}_{+}$, let $S:=(\mu-i \alpha) R(\mu)^{\prime}$ and $T=\mu R(\mu)^{\prime}$. Then $(S x)(S X)^{*} \leqq \mathrm{~T}\left(\mathrm{x} \mathrm{x}^{*}\right), S_{*}\left(\psi_{\alpha}^{*}\right)=\psi_{\alpha}^{*}, \mathrm{~T}_{*}\left(\left|\psi_{\alpha}{ }^{*}\right|\right)=\left|\psi_{\alpha}{ }^{*}\right|$, and T is an identity preserving Schwarz map. Since $s\left(\left|\psi_{\alpha} *\right|\right) \mathrm{M}$ is T-invariant, the assumptions of Lemma 1.2 are fulfilled and we obtain for every $x \in M$

$$
S(x) u_{\alpha}^{*}=T\left(x u_{\alpha}^{*}\right)
$$

Since the closed left ideal $\mathrm{Mp}_{j}$ is S-invariant it follows

$$
S(x)=T\left(x u_{\alpha}^{*}\right) u_{\alpha} \quad, \quad x \in M p_{j}
$$

(see Remark 1.3). Since $u_{\alpha}$ does not depend on $\mu \in \mathbb{R}_{+}$we obtain for all $\mu \in \mathbb{R}_{+}$

$$
\mu R(\mu+i \alpha)^{\prime} x=\mu R(\mu)^{\prime}\left(x u_{\alpha}^{*}\right) u_{\alpha}
$$

Consequently, the holomorphic functions $\left(\mu \rightarrow \mu R(\mu) '\left(x u_{\alpha}\right) u_{\alpha}{ }^{*}\right)$ and $\left(\mu+\mu R(\mu+i \alpha)^{\prime} x\right)$ coincide on $\mathbb{R}_{+}$from which we conclude

$$
\lambda R(\lambda+i \alpha)^{\prime} x=\lambda R(\lambda)^{\prime}\left(x u_{\alpha}\right)^{\prime} u_{\alpha}
$$

for every $\lambda \in D$ and all $x \in M p_{j}$. Since the map $\left(y \rightarrow y u_{\alpha}\right)$ is a continuous bijection from $M\left(u_{\alpha} u_{\alpha}{ }^{*}\right)$ onto $M_{j}$ and its inverse is the $\operatorname{map}\left(\mathrm{y} \rightarrow \mathrm{yu}_{\alpha}{ }^{*}\right)$, we can deduce that

$$
\begin{aligned}
\operatorname{dim} \operatorname{Fix}\left((\lambda-i \alpha) R(\lambda)^{\prime} \mid M p_{j}\right) & =\operatorname{dim} F i x\left(\lambda R(\lambda)^{\prime}\right) \mid M\left(u_{\alpha} u_{\alpha}^{*}\right) \leqq \\
& \leqq \operatorname{dim} F i x\left(R^{\prime}\right) .
\end{aligned}
$$

since $\oplus_{j=1}^{n} M p_{j}=M$ and $\oplus_{j=1}^{n} L_{j}=M_{*}$ we obtain

$$
\left.\operatorname{dim} \operatorname{Fix}\left((\lambda-i \alpha) R(\lambda)^{\prime}\right)\right)=\operatorname{dim} \operatorname{Fix}\left(\lambda R(\lambda)^{\prime}\right)=
$$

\begin{verbatim}
= dim Fix(\lambdaR(\lambda))
\end{verbatim}

and the assertion follows from Lemma 1.6.

Before going on let us recall the basic facts of the ultrapower $\hat{E}$ of a Banach space $E$ with respect to some free ultrafilter $u$ on $\mathbb{N}$ (compare A-I, 3.6). If $\ell^{\infty}(E)$ is the Banach space of all bounded func-tions on $N$ with values in $E$, then

$$
c_{U}(E):=\left\{\left(x_{n}\right) \in e^{\infty}(E): \lim _{U}\left\|x_{n}\right\|=0\right\}
$$

is a closed subspace of $\ell^{\infty}(E)$ and equal to the kernel of the seminorm

$$
\left\|\left(x_{n}\right)\right\|:=\lim _{u}\left\|x_{n}\right\|,\left(x_{n}\right) \in \ell^{\infty}(E)
$$

$B y$ the ultrapower $\hat{E}$ we understand the quotient space $\ell^{\infty}(E) / C u^{(E)}$ with norm

$$
\|\hat{x}\|=\lim _{u}\left\|x_{n}\right\|, \quad\left(x_{n}\right) \in \hat{x} \in \hat{E}
$$

Moreover, for a bounded linear operator $T \in L(E)$, we denote by $\hat{T}$ the well defined operator $\hat{T x} \hat{x}:=\left(T x_{n}\right)+c_{U}(E),\left(x_{n}\right) \hat{x}$. It is clear by virtue of $\left(x \rightarrow(x, x, \ldots)+c_{U}(E)\right)$ that each $x \in E$ defines an element $\hat{x} \in \hat{E}$. This isometric embedding as well as the operator map $(\mathrm{T} \rightarrow \hat{T})$ are called canonical. In particular, if $R:(D) L(E))$ is a pseudo-resolvent, then

$$
\hat{R}:=\left(\lambda \rightarrow R(\lambda)^{n}\right): D \rightarrow L(\hat{E}),
$$

is a pseudo-resolvent, too. Recall that the approuimative point spectrum $A_{\sigma}(T)$ is equal to the point spectrum $P_{\sigma}(\hat{T})$ (see, e.g., [Schaefer (1974), Chapter v, \$1]). This construction gives us the possibility to characterize uniformly ergodic semigroups with finite dimensional fixed space.

Lemma 2.2. Let $R$ be a pseudo-resolvent on $D=\{\lambda \in \mathbb{C}: \operatorname{Re}(\lambda)>0\}$ such that $\|\mu R(\mu+i \alpha)\| \leq 1$ for all $(\mu, \alpha) \in \mathbb{R}_{+} \times \mathbb{R}$ and suppose

$$
0<\operatorname{dim} \operatorname{Fix}\left(\left(\lambda-i_{\alpha}\right) \hat{R}(\lambda)\right)<\infty \text { for some } \lambda \in D, \alpha \in \mathbb{R}
$$

and the canonical extension $\hat{R}$ on some ultrapower $\hat{E}$.

Then the following assertions hold:\\
(a) $(\lambda-i \alpha)^{-1}$ is a pole of the resolvent $R(., R(\lambda))$ for all $\lambda \in D$.\\
(b) $\operatorname{dim} \operatorname{Fix}((\lambda-i \alpha) R(\lambda))=\operatorname{dim} \operatorname{Fix}((\lambda-i \alpha) \hat{R}(\lambda))$ for all $\lambda \in D$.\\
(c) ia is a pole of the pseudo-resolvent $R$ and the residue of $R$ and $R(., R(\lambda))$ in $i \alpha$ respectively $(\lambda-i \alpha)^{-1}$ are identical.

Proof. Take a normalized sequence $\left(x_{n}\right)$ in $E$ with

$$
\lim _{n}\left\|(\lambda-i \alpha) R(\lambda) x_{n}-x_{n}\right\|=0
$$

The existence of such a sequence follows from the fact that the fixed space of $(\lambda-i \alpha) \hat{R}(\lambda)$ is non trivial. Suppose $\left(x_{n}\right)$ is not relatively compact. Then we may assume that there exists $\delta>0$ such that

$$
\left\|x_{n}-x_{n n}\right\|>\delta \text { for } n \neq m
$$

Take $k \in \mathbb{N}$ and let $\hat{x}_{k}$ be the image of $\left(x_{n+k}\right)$ in $\hat{E}$. Since

$$
\lim _{n}\left\|(\lambda-i \alpha) R(\lambda) x_{n+k}-x_{n+k}\right\|=0
$$

the so defined $\hat{x}_{k}^{\prime}$ s belong to $\operatorname{Fix}((\lambda-j \alpha) \hat{R}(\lambda))$. Since this space is finite dimensional there exist $j<\ell$ such that

$$
\left\|\hat{x}_{j}-\hat{x}_{\ell}\right\| \leqq \frac{\delta}{2} .
$$

From the definition of the norm in $\hat{E}$ it follows that there are natural numbers $n<m$ such that

$$
\left\|x_{n}-x_{m}\right\| \leqq \frac{\delta}{2}
$$

which leads to a contradiction. Therefore every approximate eigenvector of $(\lambda-i \alpha) R(\lambda)$ pertaining to 1 is relatively compact. In particular it has a convergent subsequence from which it follows that the fixed space of $(\lambda-i \alpha) R(\lambda)$ is non trivial.

Obviously

$$
\operatorname{dim} \operatorname{Fix}((\lambda-i \alpha) \mathrm{R}(\lambda)) \leqq \operatorname{dim} \operatorname{Fix}((\lambda-i \alpha) \hat{R}(\lambda))
$$

If the last inequality is strict, then there exists $\gamma>0$ and a normalized $\hat{x} \in F i x((\lambda-i \alpha) \hat{R}(\lambda))$ such that

$$
Y \leqq\|\hat{y}-\hat{x}\|
$$

for all $y \in F i x\left((\lambda-i() R(\lambda))\right.$. Take a normalized sequence $\left(x_{n}\right) \in \hat{x}$. Then $\left(x_{n}\right)$ has a convergent subsequence whence we may assume that $\lim _{n} x_{n}=z$ exists in $E$. Thus $0 \neq z \in F i x((\lambda-i \alpha) R(\lambda))$. From this we obtain the contradiction

$$
\gamma \leqq\|\hat{z}-\hat{x}\|=\lim \left\|z-x_{n}\right\|=0
$$

Consequently

$$
\operatorname{dim} \operatorname{Fix}((\lambda-i \alpha) \mathrm{R}(\lambda))=\operatorname{dim} \operatorname{Fix}((\lambda-i \alpha) \hat{R}(\lambda))
$$

Let $\left\{x_{1}, \ldots, x_{n}\right\}$ be a base of Fix(( $\left.\left.\lambda-i \alpha\right) R(\lambda)\right)$ and choose $\left\{\phi_{1}, \cdots \phi_{n}\right\}$ in Fix(, $\lambda$ ia)R( $\left.\left.\lambda\right)^{\prime}\right)$ such that $\phi_{k}\left(x_{j}\right)=\delta_{k, j}$ (Lemma 1.6). Then

$$
E=\operatorname{Fix}((\lambda-i \alpha) R(\lambda)) \quad \oplus \quad\left(n \underset{j=1}{n} \operatorname{ker}_{j}\right)
$$

where both subspaces on the right are $(\lambda-i \alpha) R(\lambda)$-invariant and 1 is a pole of $(\lambda-i \alpha) R(\lambda) \mid F i x((\lambda-i \alpha) R(\lambda))$ by the finite dimensionality of Fix ( $(\lambda-i a) R(\lambda))$. Suppose 1 belongs to the spectrum of $S$ where $s$ is the restriction of $(\lambda-i \alpha) R(\lambda)$ to $n_{j=1}^{n}$ ker $\phi_{j}$. Then there exists a normalized sequence $\left(y_{n}\right)$ in $n_{j=1}^{n}$ ker. ${ }_{j}$ such that

$$
\lim _{n}\left\|(\lambda-i \alpha) R(\lambda) y_{n}-y_{n}\right\|=0
$$

Therefore $\left(y_{n}\right)$ has an accumulation point different from zero in

$$
\operatorname{Fix}((\lambda-i \alpha) R(\lambda)) \quad \cap \quad\left(n_{j=1}^{n} \operatorname{ker}_{j}\right)
$$

This contradiction implies that 1 does not belong to the spectrum of S . Since Fix(( $\lambda$ - ia)R( $\lambda$ )) is finite dimensional, it follows from general spectral theory that $(\lambda-i \alpha)^{-1}$ is a pole of $R(., R(\lambda))$ for every $\lambda$. Thus (a) and (b) are proved. Assertion (c) follows from the resolvent equality as in the proof of [Greiner (1981), Proposition 1.2].

Proposition 2.3. Let $T$ be a semigroup of contractions on a Banach space E with generator A . Then the following assertions are equivalent:\\
(a) Each $i \alpha, \alpha \in \mathbb{R}$, is a pole of the resolvent $R(., \mathrm{A})$ such that the corresponding residue has finite rank.\\
(b) $\operatorname{dim} \operatorname{Fix}((\lambda-i a) \hat{R}(\lambda, A))<\infty$ for some (hence all) $\lambda \in \mathbb{C}$, $\operatorname{Re}(\lambda)>0$ and the canonical extensions $\hat{R}(\lambda, A)$ of $R(\lambda, A)$ to some ultrapower.

Proof. Let $P_{\alpha}$ be the residue of the resolvent $R(., A)$ in $i \alpha$. Then $\mathbb{P}_{\alpha}=\lim _{\lambda \rightarrow i \alpha}(\lambda-i \alpha) R(\lambda, A)$ in the operator norm of $L(E)$. Since the canonical map $(T) \hat{T})$ is isometric and since $\hat{E}$ is an ultrapower, we obtain

$$
\hat{\mathrm{P}}_{\alpha}=\lim _{\lambda \rightarrow i \alpha}(\lambda-i \alpha) \hat{R}(\lambda, A)
$$

in $L(\hat{E})$ and $\operatorname{rank}\left(\mathrm{P}_{\alpha}\right)=\operatorname{rank}\left(\hat{P}_{\alpha}\right)$. Because of

$$
\hat{\mathrm{P}}_{\alpha}(\hat{\mathrm{E}})=\operatorname{Fix}((\lambda-i \alpha) \hat{R}(\lambda))
$$

one part of the corollary is proved. The other follows from Lemma 2.2.

Remarks 2.4. (a) By the results in [Lin (1974)] a semigroup of contractions on a Banach space is uniformly ergodic if and only if 0 is a pole of the generator with order $\leqq 1$. The residue of the resolvent in 0 and the associated ergodic projection are identical.\\
(b) Let $M$ be a $W^{*}$-algebra with predual $M_{*}$, U a free ultrafilter on $\mathbb{N}$ and $\hat{M}$ (resp. $\left(M_{*}\right)^{\wedge}$ ) the ultrapower of $M$ (resp. $M_{*}$ ) with respect to $U$. Then it is easy to see that $c_{U}(M)$ is a two sided ideal in $\ell^{\infty}(M)$ hence $\hat{M}$ is a $C^{*}$-algebra, but in general not a W*-algebra. Note that the unit of $\hat{M}$ is the canonical image of 1 . For $\hat{\mathrm{x}} \in \hat{\mathrm{M}}$ and $\hat{\phi} \in\left(\mathrm{M}_{*}\right)^{\wedge}$ let $J:\left(M_{*}\right)^{\wedge} \rightarrow \hat{M}^{\prime}$ be defined by

$$
\langle\hat{x}, \mathcal{J}(\hat{\phi})\rangle:=\lim _{U} \phi_{n}\left(x_{n}\right),\left(x_{n}\right) \in \hat{x},\left(\phi_{n}\right) \in \hat{\phi} .
$$

$J$ is well defined and is an isometric embedaing. It turns out that $\left.J\left(M_{\star}\right)^{\wedge}\right)$ is a translation invariant subspace of ( $\left.\mathrm{M}^{\prime}\right)^{\wedge}$. Hence there exists a central projection $z \in \hat{M}^{\prime \prime}$ such that $\left.J\left(M_{*}\right)^{\wedge}\right)=\hat{M}^{\prime \prime z}$ [Groh (1984), Proposition 2.2].

Below we identify $\left(M_{*}\right)^{\wedge}$ via $J$ with this translation invariant subspace. From the construction the following is obvious: If $T$ is an identity preserving Schwarz map with preadjoint $T_{\star} \in L\left(M_{\star}\right)$, then $\hat{T}$ is an identity preserving Schwarz map on $M$ such that $\left(T_{\star}\right)^{\wedge}=$ $\hat{T^{\prime}} \mid\left(M_{*}\right)^{n}$.

Theorem 2.5. Let $T$ be an identity preserving semigroup of Schwarz type with generator $A$ on the predual of a $w^{*}$-algebra $M$. If $T$ is uniformly ergodic with finite dimensional fixed space , then every $\gamma \in \sigma(A) \cap$ iR is a pole of the resolvent $R(., A)$ and dim ker( $\gamma$ - A) $\leqq$ $\operatorname{dim} F i x(T)$.

Proof. Let $D=\{\lambda \in \mathbb{C}: \operatorname{Re}(\lambda)>0\}$ and $R$ the $M_{*}$-valued pseudoresolvent of Schwarz type induced by $R(., \mathrm{A})$ on $D$. Then

$$
P=\lim _{\mu \downarrow 0} \mu R(\mu)
$$

exists in the uniform operator topology and $\operatorname{rank}(\mathrm{p})=\operatorname{dim} \mathrm{Fix}(T)<\infty$. From this we obtain $\operatorname{rank}(\mathrm{P})=\operatorname{rank}(\hat{\mathrm{P}})<\infty$ where $\hat{\mathrm{p}}$ is the canonical extension of $P$ onto $\left(M_{\star}\right)^{\wedge}$. Since $\hat{P}=\lim _{\mu \downarrow 0} \mu R(\mu)^{\wedge}$ it follows that

$$
\operatorname{dim} F i x((\lambda-i \alpha) \hat{R}(\lambda)) \leqq \operatorname{rank}(\hat{P})<\infty
$$

(Proposition 2.1) for all a 6 R . Therefore the assertion follows from Lemma 2.2.

The consequences of this result for the asymptotic behavior of one-parameter semigroups will be discussed in D-IV, section 4 .

NOTES.\\
Section 1. The Perron-Frobenius theory for a single positive operator on a non-commutative operator algebra is worked out in Albeverio-Hфegh-Krohn (1978) and Groh (1981). The limitations of the theory (in the continuous as in the discrete case)\\
are explained by the example following Remark 1.9 (see also Groh (1982a)). Therefore we concentrate on irreducible semigroups. Our main result (Theorem 1.10) extends B-III, Thm. 3.6 to the non-commutative setting.

Section 2. Theorem 2.5 has its roots in the Niiro-Sawashima Theorem for a single irreducible positive operator on a Banach lattice (see Schaefer (1974), V.5.4). The analogous semigroup result on Banach lattices is due to Greiner (1982). The ultrapower technique in our proof is developed in Groh (1984b).

\section*{Chapter D-IV}
\section*{ASYMPTOTICS OF POSTTIVE}
SEMIGROUPS ON $\quad C^{*}-A N D \quad W^{*}-A L G E B R A S$

\section*{1. STABILITY OF POSITIVE SEMIGROUPS}
As explained in A-III, Section 1, it is possible to deduce uniform exponential stability of strongly continuous semigroups from the location of the spectrum of its generator if the spectral bound s(A) and the growth bound $\omega$ coincide. In this section we prove 's $(A)=\omega$ ' for positive semigroups on $C^{*}-a l g e b r a s$ and preduals of W*-algebras. A more general discussion of the "s(A) = $\omega$ " problem can be found in [Greiner-Voigt-Wolff (1981)]. For the results of this section the existence of a unit is essential.

Theorem 1.1. Let $M$ be a $C^{*}$-algebra with unit and $T=(T(t))_{t \geqq 0} a$ positive semigroup on M . Then

$$
-\infty<s(A)=\omega \in \sigma(A) .
$$

Proof. For every $t \geqq 0$ there exists $\phi_{t}$ in the state space $S(M)$ of $M$ such that

$$
T(t){ }^{\prime} \phi_{t}=r(T(t)) \phi_{t}=\exp (\omega t) \phi_{t}
$$

(see, e.g., [Groh (1981), 2.1]). Let $n \in \mathbb{N}$ and

$$
E_{n}:=\left\{\phi \in S(M): T\left(2^{-n}\right)^{\prime} \phi=\exp \left(\omega 2^{-n}\right) \phi\right\}
$$

Then $\emptyset \neq E_{n+1} \subseteq E_{n}$ ( $n \in N$ ). Since $S(M)$ is $\sigma\left(M, M^{\prime}\right)$-compact there exists $\phi \in \cap_{n \in \mathbb{N}} E_{n}$ and $T(t)^{\prime} \phi=\exp (\omega t) \phi$ follows because the adjoint semigroup $\left(T(t)^{\prime}\right)_{t \geq 0}$ is a weak*-semigroup on $M^{\prime}$. Suppose $-\infty=\omega$. Then for $t>0 \quad r(T(t))=0$ (A-III,Prop.1.1) or $T(t)^{\prime} \phi=$ 0 , in particular $\phi(\mathrm{T}(\mathrm{t}) \mathrm{I})=0$. From this we obtain the contra-\\
diction $\phi(1)=0$. Hence $-\infty<\omega$ and $\exp (\omega t) \in P \sigma(T(t)$ ') for every $t \in R_{+}$. Thus $\omega \in \sigma(A)$ or $\omega=s(A)$.

Remark 1.2. (a) If we consider the nilpotent translation semigroup on the $C^{*}$-algebra $C_{0}([0,1))$ then $\sigma(A)=\varnothing$ and $\omega=-\infty$. This shows that the existence of a unit is essential.\\
(b) 's(A) = $\omega$ ' still holds for positive semigroups on commutative C*-algebras without unit (see B-IV, Rem.l.2.b).

Theorem 1.3. Let $M$ be a $W^{*}$-algebra with predual $M_{*}$ and let $(T(t))_{t \geqq 0}$ be a positive semigroup on $M_{\star}$. Then $s(A)=\omega$.

Proof. For all $\lambda>\mathrm{s}(\mathrm{A})$ and $\phi \in \mathrm{M}_{\star}$

$$
R(\lambda, A) \phi=\int_{0}^{\infty} e^{-\lambda t} T(s) \phi d s
$$

which follows as in C-III, Section 1 or [Greiner-Voigt-Wolff (1981), Theorem 3]. Since $\|\phi\|=\phi(1)$ for every $\phi \in \mathrm{M}_{*}^{+}$and since the norm is additive on the positive cone of $\mathrm{M}_{*}$ the integral

$$
\int_{0}^{\infty} e^{\lambda t}\|T(s) \phi\| d s
$$

exists for all $\phi \in \mathrm{M}_{*}$ and all $\lambda>\mathrm{s}(\mathrm{A})$. From this the assumption follows by A-IV,Thm.1.11.

Corollary 1.4. Let $M$ be a $C^{*}$-algebra and (T( $\left.t\right)_{t \geqq 0}$ a positive semigroup on $M^{\prime}$. Then $s(A)=\omega$ holds.

This follows from the fact that the bidual of a $C^{*}$-algebra is a $\mathrm{W}^{*}$ algebra (see [Takesaki (1979), Theorem III.2.4.]).

Remark 1.5. A simple modification of A-III, Example 1.4 (take $\mathrm{C}_{0}$ instead of $\ell^{2}$, shows that Theorem 1.3 is no longer true for nonpositive semigroups (for details see [Groh-Neubrander (1981), Beispiel 2.5]).

While the growth bound $w$ characterizes uniform exponential stability of the semigroup there are other (and weaker) stability concepts (cf. A-IV, Section 1).

Definition 1.6. Let $E$ be a Banach space and $(T(t))_{t \geqslant 0}$ a semigroup on E. We call the semigroup

\begin{enumerate}
  \item uniformly exponentially stable, if $\|\mathrm{T}(\mathrm{t})\| \leqq \mathrm{Me}^{-\mathrm{Wt}}$ for some w , $M>0$ and all $t \geqq 0$.
  \item uniformly stable, if $\lim _{t \rightarrow \infty} T(t)=0$ in the strong operator topology.
  \item weakly stable, if $\lim _{t \rightarrow \infty} T(t)=0$ in the weak operator topology.
\end{enumerate}

Surprisingly all these properties coincide for positive semigroups on C*-algebras with unit.

Theorem 1.7. Let $M$ be a $C^{*}$-algebra with unit and (T(t)) ${ }_{t \geq 0} a$ positive semigroup on $M$. Then the following assertion are equivalent.

\begin{enumerate}
  \item $s(A)<0$.
  \item The semigroup (T(t)) $t \geq 0$ is uniformly exponentially stable.
  \item The semigroup $(T(t))_{t \geq 0}$ is uniformly stable.
  \item The semigroup $(T(t))_{t \geqq 0}$ is weakly stable.
\end{enumerate}

Proof. Since 's(A) = $\omega^{\prime}$ by Theorem 1.3, it suffices to show that 4. implies 1. . For $t>0$ there exists $\phi \in S(M)$ such that

$$
T(t)^{\prime} \phi=r(T(t)) \phi .
$$

Then for $x \in M$

$$
\phi\left(T(t)^{n} x\right)=(r(T(t)))^{n} \phi(x) \rightarrow 0
$$

as $n \rightarrow \infty$. Therefore $r(T(t))<1$ or $\omega<0$. Since $s(A) \leqq \omega$ the assertion follows.

Remark 1.8. If we consider the translation semigroup ( $T(t))_{t \geq 0}$ on $C_{0}\left(\mathbb{R}_{+}\right)$, then $\|T(t)\|=1$, hence $s(A)=1$, but $(T(t))_{t \geqq 0}$ is uniformly stable. The same holds for the translation semigroup on $\mathrm{L}^{1}\left(\mathbb{R}_{+}\right)$. Thus Theorem 1.7 is not true for semigroups on $C^{*}$-algebras with unit or on preduals of $\mathrm{W}^{*-a l g e b r a s . ~ F o r ~ t h e ~ d i s c u s s i o n ~ o f ~ t h e ~}$ commutative situation we refer to B-IV, Section 1.

\section*{2. STABILITY OF IMPLEMENTED SEMIGROUPS}
Let $H$ be a Hilbert space, $U=(U(t))_{t \geqq 0}$ a strongly continuous semigroup on $H$ with generator $B$ and $M \subseteq B(H)$ be a w*-algebra, where $B(H)$ is the $W^{*}$-algebra of all bounded linear operators on H . Suppose U(t)MU(t)* $\subseteq M$. Then one can define a weak*-continuous semigroup $T=(T(t))_{t \geq 0}$ on $M$ by $T(t) x:=U(t) x U(t) *\left(t \in \mathbb{R}_{+}, x \in M\right)$. We call $T$ an implemented semigroup. Every map $T(t)$ of an implemented semigroup is weak*-continuous and n-positive for every $n \in \mathbb{N}$.

Remarks 2.1. (a) Because of

$$
\|T(t)\|=\|T(t) 1\|=\|U(t) U(t) *\|=\|U(t)\|^{2}
$$

it follows that $\omega(T)=2 \omega(u)$.\\
(b) If $(\mathrm{T}(t))_{t \geqq 0}$ is an implemented semigroup, then the preadjoint semigroup is strongly continuous on $M_{*}$. Therefore $s(A)=\omega$ for $(T(t))_{t \geq 0}$ by Theorem 1.3.\\
(c) Since $(U(t))_{t \geqq 0}$ is a (strongly continuous) semigroup the same is true for the adjoint semigroup $\left(U(t)^{*}\right)_{t \geqq 0}$ and its generator is given by B*. In analogy to [Bratteli-Robinson (1979), 3.2.55] the following assertions for $x \in M$ are equivalent:\\
(i) $x \in D(A)$.\\
(ii) For $\xi \in D(B)$ it follows $x \xi \in D\left(B^{*}\right)$ and the linear mapping


\begin{equation*}
\left(\xi \rightarrow x(B \xi)+B^{*}(x \xi)\right): D(B) \rightarrow H \tag{*}
\end{equation*}


has a continuous extension to $H$.

Then Ax is given as the continuous extension of (*) . We shortly write $A x=x B+B * x$.

In the next theorem we give some equivalent conditions for the uniform exponential stability of an implemented semigroup. As we shall see, the operator equality

$$
y^{B}+B^{*} y=-x \quad\left(x, y \in M_{+}\right)
$$

is necessary and sufficient, which is in complete analogy to the classical Iiapunov stability result.

Theorem 2.2. Let $M$ be a $\mathrm{W}^{*}$-algebra on a Hilbert space $H$ and let $T=(T(t))_{t \geqq 0}$ be a weak*-semigroup on $M$ with generator $A$ implemented by the semigroup $(U(t))_{t \geqq 0}$ on $H$ with generator $B$. Then the following assertions are equivalent.\\
(a) $\omega(T)=s(A)<0$.\\
(b) The semigroup $(U(t))_{t \geq 0}$ is uniformly exponentially stable.\\
(c) There exists $0 \leqq x \in D(A)$ such that $A x=-1$.\\
(d) There exists $0 \leqq x \in D(A)$ such that $x(D(B)) \subseteq D\left(B^{*}\right)$ and $\mathrm{xB}+\mathrm{B}^{*} \mathrm{x}=-1$.\\
(e) For every $0 \leqq x \in D(A)$ there exists $0 \leqq y \in D(A)$ such that $A y=-x$.\\
(f) For every $0 \leqq x \in D(A)$ there exists $0 \leqq y \in D(A)$ such that $y(D(B)) \subseteq D\left(B^{*}\right)$ and $y B+B^{*} y=-x$.\\
(g) $\quad \int_{0}^{\infty}\|U(s) \xi\|^{2} \mathrm{ds}$ exists for all $\xi \in \mathrm{H}$.\\
(h) $\quad \int_{0}^{\infty}((T(s) x) \xi \mid \zeta) d s$ exists for all $\xi, \zeta \in H$ and all $x \in M$.

Proof. The equivalence of (a) and (b) follows from Remark 2.1.(a) whereas (c) and (d), resp., (e) and (f) are equivalent by the Remark 2.1.(c).\\
$(a) \rightarrow(c):$ since $s(A)<0$ the resolvent $R(0, A)$ exists and is a positive map on $M$. Therefore $R(0, A) 1 \in D(A)+$ or $A x=-1$ for some $x \in D(A)+$\\
$(c) \rightarrow(e):$ Let $x \in D(A)+$ such that $A x=-1$. Then

$$
T(t) x-x=\int_{0}^{t} T(s) A x d s=-\int_{0}^{t} T(s) 1 d s \quad(t \geq 0),
$$

hence

$$
0 \leqq \int_{0}^{t} T(s) 1 d s \leqq x \quad\left(t \in \mathbb{R}_{+}\right)
$$

Since the family $\left(\int_{0}^{t} T(s) l d s\right) t \geqq 0$ is increasing and bounded,

$$
\lim _{t \rightarrow \infty} \int_{0}^{t} T(s) 1 d s
$$

exists in the weak operator topology on $B(H)$. Since on bounded sets of $M$ the weak operator topology is equivalent to the $\sigma\left(M_{1} M_{\star}\right)-$ topology, [Sakai (1971), 1.15.2.], for every $\phi \in M_{*}$ the integral $\int_{0}^{\infty} \phi(\mathrm{T}(\mathrm{s}) 1) \mathrm{ds}$ exists. Take $\mathrm{x} \in \mathrm{M}_{+}$and $\phi \in \mathrm{M}_{*}^{+}$. Then $\mathrm{x} \leqq\|\mathrm{x}\| 1$ and therefore

$$
\phi(T(s) x) \leqq\|x\| \phi(T(s) 1) \quad\left(s \in \mathbb{R}_{+}\right)
$$

Hence $\int_{0}^{\infty} \phi(T(s) x) d s$ exists. Since the positive cones of $M$ and $M_{*}$ are generating, $\int_{0}^{\infty} \phi(T(s) x) d s$ exists for every $x \in M$ and $\phi \in M_{*}$. Therefore $R(0, A)$ exists and is positive which proves (e).\\
$(c) \rightarrow(g)$ From the last paragraph we obtain that for all $\xi \in H$

$$
\int_{0}^{\infty}\|U(s)\|^{2} d s=\int_{0}^{\infty}(T(s) 1 \xi \mid \xi) d s
$$

exists.\\
$(g) \rightarrow(h): I t$ follows from the polarization identity that the integral

$$
\int_{0}^{\infty}(U(s) \xi \mid U(s) \zeta) d s
$$

exists for all $\xi, \zeta \in H$. Using [Takesaki (1979), Theorem III. 4.2 and Theorem II.2.6] we conclude as in the implication from (c) to (e) that for all $\xi, \zeta \in H$ the integral

$$
\int_{0}^{\infty}(((T(s) x) \xi \mid \zeta) d s \quad(x \in M)
$$

is finite.\\
$(g)$ (a): Since the vector states are dense in the predual of M ([Takesaki (1979), Theorem II.2.6]) and since the preadjoint semigroup of $T$ is strongly continuous, it is easy to see that the integral

$$
\int_{0}^{\infty} \phi(T(s) x) d s
$$

exists for all $x \in M$ and $\phi \in M_{*}$. Therefore, the resolvent $R(0, A)$ exists and is positive, hence $s(\mathrm{~A})<0$.

\section*{3. CONVERGENCE OF POSITIVE SEMIGROUPS}
In this section the asymptotic behavior of positive semigroups $(T(t))_{t \geq 0}$ will be described in more detail. Essentially we distinguish three cases:

\begin{enumerate}
  \item The cesaro means $\frac{1}{s} \int_{0}^{s} T(t) d t$ converge strongly to a projection $P$ onto the fixed space of $(T(t))_{t \geqq 0} \quad$ (see Proposition 3.3 and 3.4 )
  \item The maps $T(t)$ converge strongly to $P$ (see Proposition 3.7, 3.8 and 3.9).
  \item The maps $T(t)$ behave asymptotically as a periodic group (Theorem 3.11).
\end{enumerate}

Much of the following is based on the theory of weakly compact operator semigroups. Therefore the following compactness criterium is quite useful.

Proposition 3.1. Let $M$ be a $W^{*}$-algebra, $T$ a bounded semigroup of positive maps on $M_{*}$ and suppose that there exists a faithful family $\Phi$ of $T$-subinvariant states in $M_{*}$. Then $T$ is relatively compact in the weak operator topology of $L\left(M_{\star}\right)$. In particular, $T$ is strongly ergodic, i.e. $\quad \lim _{s \rightarrow \infty} \frac{1}{s} \int_{0}^{s} T(t) x d t$ exists for every $x$ in $M$ and yields a projection onto Fix (T).

Proof. Since the positive cone of $M_{*}$ is generating, it is enough to show that for every $0 \leqq \psi \in M_{*}$ the orbit $\left\{T(t) \psi: t \in \mathbb{R}_{+}\right\}$is relatively weak compact. For this we use [Takesaki(1979), Theorem III.5.4.(iii)].

Let $\left(p_{n}\right)_{n \in N}$ be a decreasing sequence of projections in $M$ such that $\inf _{\mathrm{n}} \mathrm{P}_{\mathrm{n}}=0$. Then $\lim _{\mathrm{n}} \phi\left(\mathrm{p}_{\mathrm{n}}\right)=0$ for every $\phi \in \Phi$. Since

$$
\left(T(t) p_{n}\right)^{2} \leqq T(t) p_{n}, t \in \mathbb{R}_{+}
$$

we obtain by a classical inequality of Kadison that

$$
0 \leqq \phi\left(\left(T(t) P_{n}\right)^{2}\right) \leqq \phi\left(T(t) P_{n}\right) \leqq \phi\left(p_{n}\right),
$$

hence $\lim _{n} \phi\left(T(t) p_{n}\right)=0$ uniformly in $t \in \mathbb{R}_{+}$. Since the family $\Phi$ is faithful on M, it follows from [Takesaki (1979), Proposition III.5.3] that ( $\mathrm{T}(\mathrm{t}) \mathrm{p}_{\mathrm{n}}$ ) converges to zero in the $s\left(\mathrm{M}_{\mathrm{H}}\right.$ )-topology uniformly in $t \in \mathbb{R}_{+}$. Since this topology is finer than the weak*topology on $M$ we obtain the relative compactness of $T$ which implies the strong ergodicity.

Let $T$ be an identity preserving semigroup of Schwarz type on the predual of a $\mathrm{W}^{*}$-algebra M . We call

$$
p_{r}:=\sup \{s(|\phi|): \phi \in F i x(T)\}
$$

the recurrent projection associated with $T$. For a motivation of this definition compare, e.g., [Davies (1976), Section 6.3].

Since $T(t)|\phi|=|\phi|$ for all $\phi \in F i x(T)$ (D-III, Cor. 1.5) we obtain $T(t) ' P_{r} \geqq P_{r} \quad$ (see $\left.D-I, \operatorname{Sec} .3 .(c)\right)$. Let $T(r)$ be the reduced semigroup on $\mathrm{P}_{r} \mathrm{M}_{\star} \mathrm{P}_{r}$ with generator $A^{(r)}$. Then $T^{(r)}$ is identity preserving and of Schwarz type. Similarly, if $R$ is a pseudo-resolvent on $D=\{\lambda \in \mathbb{C}: \operatorname{Re}(\lambda)>0\}$ with values in $M_{*}$ such that $R$ is identity preserving and of Schwarz type, then the recurrent projecton associated with $R$ is defined using Fix(R) .

Remark 3.2. (a) Let $\phi \in M_{*}$ and $\alpha \in \mathbb{R}$ such that

$$
\left(\mu-i_{\alpha}\right) R(\mu) \phi=\phi \text { for some } \mu \in \mathbb{R}_{+} .
$$

Since s(| $\mid$ ) and $s(|\phi *|)$ are majorized by $p_{r}$ (D-III, Prop.1.4) it follows that $\phi$ and $\phi^{*}$ are in $\mathrm{P}_{r}{ }^{M_{*}} \mathrm{P}_{r}$.\\
(b) From (a) and the observation that the family $\{|\phi|$ : $\phi \in F i x(T)\}$ is\\
faithful on $p_{r} \mathrm{Mp}_{r}$ and consists of $T^{(r)}$-invariant elements, it follows that:\\
(i) $\quad \operatorname{P\sigma }(\mathrm{A}) \cap i \mathbb{R}=\operatorname{P\sigma }\left(\mathrm{A}^{(\mathrm{r})}\right) \cap i \mathbb{R}$.\\
(ii) $\operatorname{ker}(i \alpha-A) \subset \mathrm{p}_{\mathrm{r}} \mathrm{M}_{*} \mathrm{P}_{\mathrm{r}}$ for all $\alpha \in \mathbb{R}$.\\
(iii) The semigroup $T^{(r)}$ is relatively compact in the weak operator topology and therefore strongly ergodic.\\
(c) Similarly, let $R$ be an identity preserving pseudo-resolvent with values in $M_{*}$ on $D=\{\lambda \in \mathbb{C}: \operatorname{Re}(\lambda)>0\}$ which is of Schwarz type. It follows as in (b) that Fix ( $(\lambda-i \alpha) R(\lambda))$ is contained in $P_{r} M_{\star} P_{r}$ for all $\lambda \in D$ and $\alpha \in \mathbb{R}$, where $P_{r}$ is the associated recurrent projection.

We now give a characterization of strong ergodicity of semigroups which are identity preserving and of Schwarz type. For this we need that the Cesàro means $C(s)$, where

$$
C(s) x=\frac{1}{S} \int_{0}^{S} T(t) x d t \quad(x \in M, 0<s \in \mathbb{R})
$$

are Schwarz maps. We omit the simple calculation (compare D-I, Thm.2.1).

Proposition 3.3. Let $T$ be an identity preserving semigroup of Schwarz type on the predual of a $W^{*}$-algbra $M$. Then the following assertions are equivalent:\\
(a) $T$ is strongly ergodic on $M_{*}$.\\
(b) $\quad \sigma\left(M, M_{\star}\right)-\lim _{S \rightarrow \infty} C(s) ' p_{r}=1$.\\
(c) $\quad s^{*}\left(M_{1} M_{*}\right)-\lim _{s \rightarrow \infty} C(s)^{\prime} p_{r}=1$.

Proof. Suppose that (a) holds. Since Fix(T) separates Fix(T') (see [Krengel (1985), Chap.2,Thm.1.4]), the fixed space of $T^{\prime}$ is non trivial, hence $\mathrm{P}_{r} \neq 0$. Let $0 \leqq \psi \in M_{*}$, then

$$
\psi_{0}:=\lim _{s \rightarrow \infty} C(s) \psi \in \operatorname{Fix}(T)
$$

and $s\left(\psi_{0}\right) \leqq \mathrm{p}_{r}$.

Therefore

$$
\begin{aligned}
& \lim _{s \rightarrow \infty} \psi\left(C(s)^{\prime} p_{r}\right)=\lim _{s \rightarrow \infty}(C(s) \psi)\left(p_{r}\right)= \\
& =\psi_{0}\left(p_{r}\right)=\psi_{0}(1)=\lim _{s \rightarrow \infty}(C(s) \psi)(1)=\psi(1),
\end{aligned}
$$

which proves (b).

Suppose that (b) is satisfied. Since $C(s)^{\prime} p_{r} \leqq 1$ for all $s \in \mathbb{R}_{+}$we obtain (c). (Use that for $\left(x_{\alpha}\right) \epsilon_{+}$we have $\lim _{\alpha} x_{\alpha}=0$ in the weak*-topology if and only if $\lim _{\alpha} x_{\alpha}=0$ in the $s^{*}\left(M, M_{*}\right)$-topology.)

Suppose that (c) holds. Since each C(s)' is an identity preserving Schwarz map we obtain for all $\mathrm{x} \in \mathrm{m}$ :

$$
\begin{gathered}
\left.(C(s))^{\prime}\left(\left(1-p_{r}\right) x\right)\right)\left(C(s) \cdot\left(\left(1-p_{r}\right) x\right) *\right) \leqq \\
\leqq C(s)^{\prime}\left(\left(1-p_{r}\right) x \times *\left(1-p_{r}\right)\right) \leqq \\
\leqq\|x\|^{2} C(s)^{\prime}\left(1-p_{r}\right)
\end{gathered}
$$

hence

$$
s^{*}\left(M_{1} M_{\star}\right)-1 i m_{S \rightarrow \infty} C(s)^{\prime}\left(\left(1-p_{r}\right) x\right)=0
$$

In particular we obtain for all $x \in F i x(T ')$ that

$$
x=\sigma\left(M, M_{\star}\right)-1 i m_{s \rightarrow \infty} C(s)^{\prime} x=\sigma\left(M, M_{\star}\right)-1 i m_{s+\infty} C(s)^{\prime}\left(p_{r} x\right) .
$$

Especially for $0 \neq x \in F i x(T)$ we obtain $p_{r} x_{r} \neq 0$. Since the W*-algebra $\mathrm{p}_{r} \mathrm{Mp}_{r}$ is the dual of $\mathrm{p}_{r}{ }^{\mathrm{M}} \mathrm{p}_{r}$ and since $T^{(r)}$ is strongly ergodic, it follows that the fixed space of $T$ separates the points of Fix(T') . Thus $T$ is strongly ergodic ([Krengel (1985), Chap. 2, Thm. 1.4J).

It follows from the result above that the semigroup in [Evans (1977)] cannot be strongly ergodic on $B(\mathrm{H})_{*}$ since the associated recurrent projection is zero. But for irreducible semigroups we have the following result.

Proposition 3.4. Let $T$ be an identity preserving semigroup of Schwarz type on the predual of a $W^{*}$-algebra $M$. Then the following assertions are equivalent.\\
(a) $T$ is irreducible and $\operatorname{Po}(A) \cap i \mathbb{R} \neq \varnothing$.\\
(b) $T$ is relatively compact in the weak operator topology and the fixed space of $T$ is generated by a faithful state.\\
(c) $T$ is strongly ergodic and the fixed space of $T$ is generated by a faithul state.\\
(d) The fixed space of $T$ is generated by a faithful state.

Proof. Suppose (a) is satisfied. Since Fix $(T) \neq\{0\}$ there exists a faithful normal state $\phi$ on $M$ such that $F i x(T)=\phi \mathbb{C}$ (D-III, Thm.1.10.). Therefore $T$ is relatively compact in the weak operator topology by Proposition 3.1., whence (b) holds.

The implications from (b) to (c) and (c) to (d) are trivial.

Suppose that (d) holds. Let $\phi$ be a faithful normal state on M such that $\operatorname{Fix}(T)=\phi \mathbb{C}$. By Proposition 3.1 the semigroup $T$ is strongly ergodic. Therefore the fixed space of $T$ separates the points of Fix(T') . Consequently Fix $\left(T^{\prime}\right)=\mathbb{C} 1$. Thus the ergodic projection associated with $T$ is given by $\mathrm{P}=1 \otimes \phi$, i.e. $\mathrm{P} \psi=\psi(1) \phi$ for all $\psi \in M_{*}$. Let $F$ be a closed $T$-invariant face of $M_{*}^{+}$. If $0 \neq \psi \in \mathrm{F} \quad$ then

$$
\lim _{s \rightarrow \infty} C(s) \psi=\psi(1) \phi \in F .
$$

Hence $\phi \in F$ and therefore $F=M_{*}^{+}$by the faithfulness of $\phi$ which proves (a).

The next theorem is an extension of D-III,Thm.1.10 and shows the usefulness of the theory of semitopological semigroups. Assume $T \subseteq L\left(M_{\star}\right)$ to be relatively compact in the weak operator topology. Since $T$ is commutative its closure $S=(T)^{-} \subseteq L_{W}\left(M_{*}\right)$ contains a unique minimal ideal $K$, called the kernel of $S$, which is a compact Abelian group ([DeLeeuw-Glicksberg (1961); Junghenn (1971); Krengel (1985), § 2.4]. The identity $Q$ of $K$ is a projection onto\\
the closed linear span of all eigenvalues of A pertaining to the eigenvalues in $i \mathbb{R}$. Moreover, the dual group of $K$ can be identified with the subgroup of iR generated by $\mathrm{P}_{\sigma}(\mathrm{A}) \mathrm{n} \mathrm{i} \mathbb{R}$. We call $Q$ the semigroup projection associated with $T$. On the other hand, $T$ is always strongly ergodic with projection P onto Fix(T) . Obviously, the relation

\section*{$O \leqq P \leqq Q \leqq I d$}
holds, where the order relation is defined by the inclusion of the range spaces.

There are two extreme cases: First $Q=I d$ and rank(P) = 1 . This corresponds to the Halmos-von Neumann Theorem in commutative exgodic theory and is discussed, at least for irreducible semigroups, in [Olesen-Pedersen-Takesaki (1980)]. Second, Id > $\mathrm{Q}=\mathrm{P}$, in particular $\operatorname{rank}(P)=1$. This latter case will be investigated in detail for $M=B(H)$, the $W^{*}-a l g e b r a$ of all bounded linear operators on a Hilbert space H . But we first need some preparations.

Theorem 3.5. Let $T$ be an identity preserving semigroup of Schwarz type on the predual of a $W^{*}$-algbra $M$ and suppose there exists a faithful family of $T$-invariant states on $M$. Let $N$ be the $\sigma\left(M_{1} M_{*}\right)$-closed linear span of all eigenvectors of $A^{\prime}$ pertaining to the eigenvalues in $i \mathbb{R}$. If $Q$ is the semigroup projection associated with $T$ the following holds:\\
(a) The adjoint of $Q$ is a faithful normal conditional expectation from $M$ onto the $w^{*}-s u b a l g e b r a N$.\\
(b) The restriction of $T^{\prime}$ to N can be embedded into a $\sigma\left(\mathrm{M}_{1} \mathrm{M}_{*}\right)-$ continuous, one-parameter group of *-automorphisms.\\
(c) If, in addition, $T$ is irreducible and if $\phi$ is the normal state generating the fixed space of $T$, then $\phi / \mathrm{N}$ is a faithful normal trace.

Proof. Consider $H:=\operatorname{Po}(A) \quad \cap \quad i \mathbb{R}$ which is not empty by assumptions. From Proposition 3.1 it follows that $T$ is relatively compact in the weak operator topology, Let $K$ be the semigroup kernel of $T^{-} \subseteq L_{w}\left(M_{*}\right)$ and $Q$ the unit of $K$. Recall that $Q \psi_{n}=\psi_{\eta}$ for all $\psi_{n} \in M_{*}$ such that $A \psi_{n}=n \psi_{n}(n \in H)$. Let $U$ be the family of all\\
eigenvectors of $A^{\prime}$ pertaining to the eigenvalues in $H$. Then $U$ is closed with respect to the multiplication in $M$ and the formation of adjoints. Thus N is a $\mathrm{W}^{*}$-subalgebra of M [Sakai (1971), Corollary 1.7.9.] and $\mathrm{T}_{\mathrm{O}}(t)^{\prime}:=\mathrm{T}(\mathrm{t})^{\prime} \mid \mathrm{N}$ is multiplicative (for this see D-III, Lemma 1.1).\\
since $Q \in T^{-} \subseteq L C_{w}\left(M_{*}\right)$ there exists an ultrafilter $U$ on $\mathbb{R}_{+}$such that $\lim _{U}\langle T(t) \psi, x\rangle=\langle Q \psi, x\rangle$ for all $x \epsilon_{M}$ and $\psi \epsilon_{M_{*}}$. If $n \epsilon_{H}$ and $\psi_{\eta} \in_{M_{*}}$ such that $A \psi_{\eta}=n \psi_{\eta}$, then for all $x \in M$ :

$$
\left\langle\psi_{n}, x\right\rangle=\left\langle Q \psi_{\eta}, x\right\rangle=\lim _{u}\left\langle T(t) \psi_{n} x x^{\rangle}=\left(\lim _{u} e^{n t}\right)\left\langle\psi_{n}, x\right\rangle\right.
$$

hence $\lim _{u} e^{n t}=1$. From this it follows that for all $\psi \epsilon_{M_{*}}$ we have

$$
\begin{aligned}
& \left.<\psi, Q^{\prime}\left(u_{n}\right)>=\lim _{u}<\psi, T(t)^{\prime} u_{n}\right\rangle= \\
& =\left(\lim _{u} e^{n t}\right)<\psi, u_{n}>=\left\langle\psi, u_{n}>\right.
\end{aligned}
$$

Hence $N \subseteq Q^{\prime}(M)$.

For $\gamma$ in the dual group of $K$ and $x \in M$ we define $x_{y}$ by

$$
\psi\left(x_{Y}\right):=\int_{K}<S \psi, x><s, \gamma>* \operatorname{dm}(S) \quad\left(\psi \in M_{*}^{+}\right)
$$

Then $x_{\gamma} \in M$ and $T(t)^{\prime} x_{\gamma}=\left\langle Q T(t), \gamma>x_{\gamma}\right.$. Therefore $x_{\gamma} \in N$. Thus the inclusion $Q^{\prime} M \subseteq \mathrm{~N}$ is proved if we can show that $Q^{\prime} \mathrm{M}$ belongs to the $\sigma\left(M_{,} M_{*}\right)$-closed linear span of $\left\{x_{\gamma}: \gamma \in K, x \in M\right\}$. For this it is enough to show that every linear form $\psi \in M_{*}$ such that $\psi\left(x_{\gamma}\right)=0$ for all $\gamma \in K$ satisfies $\psi(Q x)=0$ for all $x \in M$. But if $\psi\left(x_{\gamma}\right)=0$ then

$$
\int_{K}\langle S \psi, x><S, \gamma>* \operatorname{dm}(S)=0, \gamma \in K .
$$

Since the map $(S \rightarrow \psi(S x)$ ) is continuous on $K$ and since the elements of $K$ form a complete orthonormal basis in $L^{2}(K, d m)$, we obtain $\psi(S x)=0$ for all $S \in K$, in particular $\psi(Q x)=0$ as desired.

Since the range of $Q^{\prime}$ is a $W^{*}$-subalgebra of $M$ it follows from [Takesaki (1979), Theorem III.3.4] that $Q$ ' is a completely positive, normal conditional expectation. Q' is faithful, i.e. ker( $\left.Q^{\prime}\right) \cap M_{+}=(0\}$ since $Q \phi=\phi$ for the faithful linear form $\phi$.

Let $\phi$ be the faithful normal state generating Fix(T) and let $U$ be a family of unitary eigenvectors of $A^{\prime}$ pertaining to the eigenvalues in H (see D-III, Remark 1.11). If $u_{1}, u_{2} \in U$ then

$$
\phi\left(u_{1} u_{2}^{*}\right)=\phi\left(T_{0}(t)^{\prime}\left(u_{1} u_{2}^{*}\right)\right)=e^{\left(n_{1}-n_{2}\right) t} \phi\left(u_{1} u_{2}^{*}\right)
$$

Therefore

$$
\phi\left(u_{1} u_{2}^{*}\right)=\left\{\begin{array}{l}
0 \\
\text { if } n_{1} \neq n_{2} \\
1
\end{array} \text { if } n_{1}=n_{2} .\right.
$$

Hence $\phi\left(u_{1} u_{2} *\right)=\phi\left(u_{2}{ }^{*} u_{1}\right)$ from which it follows that $\tau:=\phi \mid N$ is a faithful normal trace.

Remarks 3.6. (a) Since $Q M_{*}=N_{*}$ and $Q^{\prime} M=N$, where $N_{\star}$ is as in D-III, Proposition 1.12, it follows from general duality theory that $\left(\mathrm{N}_{\star}\right)^{\prime}=\mathrm{N}$.\\
(b) If $\psi \in N_{*}$ then $|\psi| \in N_{*}$. To see this note that $Q \psi=\psi$ and $Q$ is an identity preserving Schwarz map. Then the assertion follows from D-III, Proposition 1.4.\\
(c) If $\psi \in \mathbb{N}_{*}$, then $\left|\mathrm{T}_{\mathrm{O}}(t) \psi\right|=\mathrm{T}_{\mathrm{O}}(t)|\psi|$ for all $t \in \mathbb{R}$. This follows immediately from the fact that $\mathrm{T}_{\mathrm{O}}(t)$ ' is a *-automorphismus on N .\\
(d) Let us add a few words concerning the structure of N : If $T$ is irreducible and $K$ is the semigroup kernel of $T^{-} \subseteq L_{w}\left(M_{*}\right)$, then

$$
\left(S \rightarrow S^{\prime}\right): K \rightarrow \mathrm{~L}\left(\left(\mathrm{~N}, \sigma\left(\mathrm{~N}, \mathrm{~N}_{\star}\right)\right)\right.
$$

is a representation of the compact, Abelian group $K$ as group of *-automorphism such that the fixed space is one dimensional. Therefore we are able to apply the results of [Olesen-Pedersen-Takesaki (1980)]. There are three possibilities for N :\\
(i) $\mathrm{N} \cong \mathrm{L}^{\infty}(K, \mathrm{dm})$ and $\mathrm{T}_{\mathrm{N}}$ is the translation group on N .\\
(ii) $\mathrm{N} \cong \mathrm{R}$ where R is the (unique) hyperfinite factor of typ II ${ }_{1}$. In that case (the image of) $K$ is approximately inner on $R$ [1.c., Theorem 5.81.\\
(iii) There exists a closed subgroup $G$ of $K$ such that

$$
N \cong L^{\infty}\left({ }^{K} / G^{\prime} d m_{/}\right) \& R
$$

where $R$ is as in (ii) and dm , the normalized Haar measure on K/G [l.c., Theorem 5.15].

So far we have studied weak*-semigroups on general $\mathrm{w}^{*}$-algebras. We now want to apply the results of this section to weak*-semigroup on $B(H)$. This is of interest in view of the results in [Davies (1976)]. To do this we call a triple $(M, \Phi, T)$ a $W^{*}$-dynamical system if $M$ is a w*-algebra, a weak*-semigroup of identity preserving Schwarz maps on $M$ and $\Phi$ a faithful family of $T$-invariant normal states. We call ( $M, \Phi, T$ irreducible, if the preadjoint semigroup is irreducible (alternatively, if the fixed space of $T$ is one dimensional).

Proposition 3.7. Let $(B(H), \Phi, T)$ be a $W^{*}$-dynamical system on the W*-algebra $B(H)$ of all bounded linear operators on a Hilbert space H . Then the following assertions are equivalent:\\
(a) $P \circ(A) \cap i \mathbb{R}=\{0\}$.\\
(b) $\lim _{S \rightarrow \infty} \mathrm{~T}(\mathrm{~s})_{\star}=P_{*}$ in the strong operator topology on $L\left(B(H)_{*}\right)$.

Proof. Obviously (b) implies (a). Suppose that (a) is fulfilled. Then the ergodic projection $P_{*}$ of the preadjoint semigroup is equal to the associated semigroup projection. Consequently there exists an ultrafilter $U$ on $\mathbb{R}_{+}$such that $\lim _{U} T(t)=P$ in the weak operator topology. We claim that the convergence holds even in the strong operator topology. Taking this for granted it follows, since for every $t \in \mathbb{R}_{+} T(t)$ is a contraction, that

$$
\lim _{t \rightarrow \infty}\left\|T(t)_{\star} \phi\right\|=0
$$

for all $\phi \in \operatorname{ker}\left(P_{\star}\right)$. Since $T(t)_{*} \psi=\psi$ for every $\psi \in i m\left(P_{*}\right)$ and because

$$
\mathrm{B}(\mathrm{H})_{\star}=\operatorname{im}\left(\mathrm{P}_{\star}\right) \oplus \operatorname{ker}\left(\mathrm{P}_{\star}\right)
$$

the assertion is proved.

It remains to show that $\lim U T(t)_{*}=P_{*}$ in the strong operator topology, Choose $0 \leqq \phi \in B(H)_{*},\|\phi\| \leqq 1$, let $\phi_{t}:=T(t)_{*} \phi(t>0)$,\\
$\phi_{0}:=P_{*} \phi$ and let $\left\{p_{i}: i \in \Delta\right\}$ be an increasing net of projections of Einite rank in $B(H)$ with strong limit 1 . Since the set $K:=\left\{\phi_{t}\right.$ : $t \geqq 0\}$ is relatively compact in the $\sigma(B(H), B(H)$-topology, there exists for every $\delta>0$ an index $i_{0} \in \Delta$ such that

$$
\left\|\left(1-p_{i}\right) \psi\left(1-p_{i}\right)\right\| \leqq \delta
$$

for every $\psi \in K$ and $i \geqq i_{0}[$ Takesaki (1979), Theorem III.5.4.(vi)]. In particular

$$
\left|\psi\left(1-p_{i}\right)\right| \leqq \delta \quad, \psi \in K, \quad i(0) \leqq i
$$

Let $p:=p(i(0))$. Then for all $x$ in the unit ball of $M$ it follows that

$$
\begin{aligned}
& \left|\mid \phi_{t}-\phi_{0}\right)(x) \mid \leqq \\
& \leqq\left|\left(\phi_{t}-\phi_{0}\right)(p x p)\right|+\left|\left(\phi_{t}-\phi_{0}\right)((1-p) x p)\right|+ \\
& +\left|\left(\phi_{t}-\phi_{0}\right)(x(1-p))\right| \leqq \\
& \leqq\left|\left(\phi_{t}-\phi_{0}\right)(p x p)\right|+4 \sqrt{\delta} .
\end{aligned}
$$

Since the $\mathrm{W}^{*}$-algebra $\mathrm{pB}(\mathrm{H}) \mathrm{p}$ is finite dimensional, there exists uEU such that

$$
\left\|\left(\phi_{t}-\phi_{0}\right) \mid \mathrm{pB}(\mathrm{H}) \mathrm{p}\right\| \leqq \delta
$$

for all $t \in U$. Consequently

$$
\left\|\left(\phi_{t}-\phi_{o}\right)\right\| \leqq(\delta+4 \sqrt{\delta})
$$

for all $t \in U$. Therefore $\lim _{U} T(t)_{\star} \phi=P_{*} \phi$ in the strong operator topology. Since the positive cone of $B(H)_{*}$ is generating, the assertion is proved.

For irreducible $\mathrm{W}^{*}$-dynamical systems on $\mathrm{B}(\mathrm{H})$ the above properties always hold.

Theorem 3.8. Let $(B(H), \Phi, T)$ be an irreducible W*-dynamical system. Then

$$
P_{\sigma}(A) \cap i \mathbb{R}=\{0\} \text {. }
$$

Proof. Let $N$ be the $W^{*}$-subalgebra of $M=B(H)$ generated by the eigenvectors of $A$ pertaining to the eigenvalues on $i \mathbb{R}$ and let $Q$ be the faithful normal conditional expectation from M onto N (Theorem 3.7.). Since M is atomic, N is atomic [Størmer (1972)]. N is finite since there exists a finite, faithful normal trace on N . In particular the center of N is isomorphic to $\ell^{\infty}$. Let $S$ be the restriction of $T$ to the center. Then $S$ is a weak*-semigroup such that every $S(t) \in S$ is $\sigma\left(\ell^{\infty}, l^{1}\right)$-continuous and a *-automorphism. From this it follows that $S(t)$ is induced by some continuous flow $k_{t}: \mathbb{N} \rightarrow \mathbb{N}$. Indeed, if $\delta_{n}\left(\left(\xi_{m}\right)\right)=\xi_{n} \quad\left(n \in \mathbb{N},\left(\xi_{m}\right) \in l^{\infty}\right)$, then $\delta_{n} \circ S(t)$ is a normal scalar valued \textit{-homomorphism hence of the form $\delta_{m}$ for some $m=k_{t}(n)$. But the function $\left(t \rightarrow k_{t}\right)$ is continuous from $\mathbb{R}$ into $\mathbb{N}$, whence constant. Hence $S(t)=I d$. But the semigroup $S$ is weak}-irreducible on the center. Consequently the center is one dimensional. Using [Takesaki, Theorem V.1.27] we obtain $\mathrm{N}=\mathrm{B}\left(\mathrm{H}_{\mathrm{n}}\right.$ ) where $H_{n}$ is a finite dimensional Hilbert space. But if $0 \neq i \alpha$ $\in P \sigma(A) \cap i R$ then $i \alpha \mathbf{Z} \subseteq P \sigma(A)$ by $D-I I I, T h m .1 .10$, whence $N$ must be infinite dimensional. Therefore $P \sigma(A) \cap i \mathbb{R}=\{0\}$ as desired.

An immediate and interesting consequence of Theorem 3.8 and Proposition 3.7 is the following.

Corollary 3.9. If (B(H), $\phi, \mathrm{T}$ ) is an irreducible W*-dynamical system, then

$$
\lim _{s \rightarrow \infty} \mathrm{~T}(s)=18 \phi
$$

in the strong operator topology on $L\left(\mathrm{~B}(\mathrm{H})_{*}\right)$, where $\phi$ is the unique normal state generating the fixed space of $\tau_{*}$.

We are now going to discuss the asymptotic behavior of positive semigroups whose generator has boundary point spectrum different from 0 . The standard example is the following:

If $\Gamma$ is the unit circle, $m$ the normalized Haar measure on $\Gamma$ and $0<\tau \in \mathbb{R}$, then we define the maps $R_{\tau}(t), t \in \mathbb{R}{ }_{+}$, on $L^{1}(\mathrm{I}, \mathrm{m})$ by

$$
\left(R_{\tau}(t) f\right)(\xi)=f\left(\xi \exp \left(\frac{2 \pi i}{\tau} t\right)\right) \quad\left(f \in L^{1}(\Gamma, m), \xi \in \Gamma\right)
$$

Then $R:=\left(R_{\tau}(t)\right)_{t \geq 0}$ forms a strongly continuous one parameter semigroup which is identity preserving and of Schwarz type. Since $R$\\
is periodic of period $\tau$ it follows that 0 is a pole of the resolvent of its generator $B$ with residuum $P=1 \otimes 1$ and $\left\{\frac{2 \pi i}{\tau} k: k \in \mathbb{Z}\right\}$ $=\sigma(B)$. Thus $R$ is irreducible and uniformly ergodic on $L^{1}(\Gamma, m)$ (see A-II, Section 5).

Now let $T$ be a semigroup on $M_{*}$. It is called partially periodic, if there exists a projection $Q \in\left(M_{*}\right)$ reducing $T$ such that $Q\left(M_{*}\right)$ $\cong L^{1}(\Gamma, m)$ and $T / \mathrm{im}(Q)$ is conjugate to a periodic semigroup on $\mathrm{L}^{1}(\Gamma, \mathrm{~m})$. In the main result we present a non commutative version of [Nagel (1984)] showing that certain dynamical systems converge to partially periodic semigroups.

Proposition 3.10. Let $T$ be an irreducible, identity preserving semigroup of Schwarz type with generator $A$ on the predual of a $\mathrm{W}^{*}-$ algebra $M$. If $T$ is uniformly ergodic, then $\sigma(\mathrm{A}) \quad \cap \quad i \mathbb{R}=$ po( $A) \cap i \mathbb{R}=i \alpha \mathbb{Z}$ for some $\alpha \in \mathbb{R}$. If additionally $\sigma(A) \cap i \mathbb{R} \neq\{0\}$, there exists a strictly positive projection $Q$ on $M_{*}$ which is identity preserving and completely positive such that:\\
(a) $Q$ reduces $T$ and $Q\left(M_{\star}\right) \cong L^{1}(\Gamma), \Gamma$ being the one dimensional torus.\\
(b) The restriction $T_{0}$ of $T$ to im(Q) is irreducible and conjugate to a rotation semigroup of period $\tau=\frac{2 \pi}{\alpha}$ on $\Gamma$.\\
(c) The spectral bound $s(\mathrm{~A} / \operatorname{ker}(Q)$ ) is strictly smaller than 0 .

Proof. By D-III, Thm.1.11 and D-III,Thm.2.5 it follows that

$$
\sigma(A) \cap i \mathbb{R}=\operatorname{Po}(\mathrm{A}) \cap i \mathbb{R}=\mathbf{i} \alpha \mathbf{Z}
$$

for some $\alpha \in \mathbb{R}$. Suppose $\alpha \neq 0$. Since $\sigma(A)+i \alpha Z=\sigma(A)$ and since every $n \in i \alpha$ is isolated, it follows that there exists $\delta>0$ such that

$$
\sigma(\mathrm{A}) \backslash i \alpha \mathbb{Z} \subseteq\{\lambda \in \mathbb{C}: \operatorname{Re}(\lambda) \leqq \delta\}
$$

Let $\left\{u_{\alpha}^{k}: k \in \mathbb{Z}\right\}$ be a family of unitary eigenvectors of $A^{\prime}$ pertaining to the eigenvalues in iR. Then $Q^{\prime}(\mathrm{M})$ is a commutative $\mathrm{W}^{*}$-algebra. Let $\tau:=\frac{2 \pi}{\alpha}$. Then $T(\tau)^{\prime} u_{\alpha}^{k}=u_{\alpha}^{k}$, hence $T / i m(Q)$ is periodic. From the Halmos-von Neumann theorem (see [Schaefer (1974),

Thm. III.7.11]) it follows that ${ }_{1}^{T}$ im(Q) is conjugate to the rotation semigroup of period $\tau$ on $\mathrm{L}^{1}(\mathrm{~F}, \mathrm{~m})$.

Using this proposition we obtain

Theorem 3.11. Let $T=(T(t))_{t \geqq 0}$ be a uniformly ergodic, identity preserving semigroup of Schwarz type on the predual of a w*-algebra $M$ and suppose $\sigma(A) \cap$ iR $\neq\{0\}$. Then there exists a partially periodic, identity preserving semigroup $S=(S(t))_{t \geqslant 0}$ of Schwarz type on $M_{*}$ such that

$$
\lim _{t \rightarrow \infty}(T(t)-s(t))=0
$$

in the strong operator topology.

Proof. Let $\phi$ be the normal state on $M$ generating the fixed space of $T$. Let $S=(S(t))_{t \geq 0}$ where $S(t):=T(t) \circ Q$ and $Q$ is as in 2.6. Obviously, $S$ is partially periodic and $\phi \in \mathrm{Fix}(\mathrm{S})$. Let $H_{\phi}$ be the GNS-Hilbert space pertaining to $\phi$. Since $\phi$ is fixed under $T, S$ and $Q$ these objects have a canonical extension to $\mathrm{H}_{\phi}$ (in the following denoted by the same symbols). If $\mathrm{H}_{0}:=\operatorname{ker}(\mathrm{Q}) \mathrm{C}_{\phi}$ then it is easy to see that $\mathrm{H}_{0}$ is invariant under the extension to $\mathrm{H}_{\phi}$ of the multiplication maps we defined in D-III, Remark 1.3. Consequently, using the results in Groh-Kummerer (1982) it follows that there exists $c \in \mathbb{R}$ such that for all $\gamma$ near 0 and all $B \in \mathbb{R}$ :

$$
\left\|R\left(\gamma+i \beta, A_{0}\right)\right\| \leqq c \quad(*)
$$

where $A_{0}:=A \mid \operatorname{ker}(Q) \quad$ (the norm taken in $L\left(H_{\phi}\right)$ ). Using the result in A-III, Cor.7.11 it follows that

$$
\lim _{t \rightarrow \infty}\left\|\mathrm{~T}(t) \mid \mathrm{H}_{0}\right\|=0
$$

Since the $s\left(M, M_{*}\right)$-topology on the unit ball of $M$ is nothing else than the restriction of the norm topology on $\mathrm{H}_{\phi}$, we obtain

$$
s\left(M, M_{*}\right)-1 i m_{t \rightarrow \infty}\left(T(t)^{\prime}-s(t)^{\prime}\right)(x)=0
$$

uniformly on $\mathrm{M}_{1}$. From this the assertion follows.

\section*{4. UNIFORM ERGODIC THEOREMS}
As we have seen, uniformly ergodic semigroups have nice spectral properties. In this section we study sufficient conditions which imply uniform ergodicity thereby generalizing results the results of Groh (1984b). We first need some preparations.

Lemma 4.1. Let $R$ be an identity preserving pseudo-resolvent of Schwarz type on $D=\{\lambda \in \mathbb{C}: \operatorname{Re}(\lambda)>0\}$ with values in the predual of a $W^{*}$-algebra M . If the fixed space of $R$ is infinite dimensional, then there exists a sequence of states in $F i x(R)$ such that the corresponding support projections are mutually orthogonal in $M$.

Proof. Let $\Phi=\{\phi \in F i x(R): \phi$ state on $M\}$ and let $p=\sup \{s(\phi)$ $: \phi \in \Phi\}$. Since $\lambda R(\lambda) \phi=\phi$ for all $\phi \in \Phi$ and $\lambda \in D$ it follows $\mu R(\mu)(1-s(\phi)) \leqq(1-s(\phi))$. Hence $\mu R(\mu)(1-p) \leqq(1-p)$ for all $\mu \in \mathbb{R}_{+} \cdot$ Let $R_{\text {, }}$ be the induced pseudo-resolvent on $\mathrm{pM}_{\star} \mathrm{p}$ ( $\mathrm{D}-\mathrm{I}$, Section 3. (c)). Then the family $\Phi$ is faithful on $M_{p}$ and contained in the fixed space of $\mathrm{R}_{\mid}$. The adjoint $\mu \mathrm{R} \mid(\mu)^{\prime}$ is an identity preserving Schwarz map. Consequently it follows from D-III, Lemma 1.1.(b) and the $\sigma\left(M_{p},\left(M_{p}\right)_{\star}\right)$-continuity of $\mu R_{\mid}(\mu)^{\prime}$, that $F i x\left(R^{\prime}\right)$ is a W*-subalgebra of $M_{p}$ and by D-III, Lemma 1.5 it follows that dim Fix (R) § dim Fix(R|'). If Fix(R) is infinite dimensional, let $\left(p_{n}\right)$ be a sequence of mutually orthogonal projections in Fix $\left(\left.R_{\mid}\right|^{\prime}\right) \subseteq$ $M_{p}$ and choose a sequence $\left(\phi_{n}\right)$ in $\Phi$ such that $\phi_{n}\left(p_{n}\right) \neq 0$. For $n \in \mathbb{N}$ let $\psi_{n}$ be the normal state

$$
\psi_{n}(x)=\phi_{n}\left(p_{n}\right)^{-1} \phi_{n}\left(p_{n} x p_{n}\right)
$$

on $M$. Because of $s\left(\psi_{n}\right) \leqq p_{n} \leqq p$, the support projections of the $\psi_{n}$ 's are mutually orthogonal in $M$. For $\mu \in \mathbb{R}_{+}$and $x \in M$ we obtain:

$$
\begin{gathered}
<x, \mu R(\mu) \psi_{n}>=\phi_{n}\left(p_{n}\right)^{-1}<\mu p_{n}\left(R(\mu)^{\prime} x\right) p_{n} \cdot \phi_{n}>= \\
=\phi_{n}\left(p_{n}\right)^{-1}<\mu p_{n} p\left(R(\mu) p^{\prime} x\right) p_{n} \phi_{n}>= \\
=\phi_{n}\left(p_{n}\right)^{-1}<\mu p_{n}\left(p R \mid(\mu)^{\prime} x p\right) p_{n}, \phi_{n}>= \\
=\phi_{n}\left(p_{n}\right)^{-1}<\mu\left(p_{n} R(\mu)^{\prime} x p_{n}\right), \phi_{n}>= \\
=\phi_{n}\left(p_{n}\right)^{-1} \phi_{n}(x)=\psi_{n}(x) .
\end{gathered}
$$

Therefore $\psi_{n} \in \operatorname{Fix}(R)$ for all $n \in \mathbb{N}$.

Remark 4.2. (a) If dim Fix(R) $\geqq 2$ then it follows from the Jordan decomposition of self adjoint linear functionals, that there are at least two states in Fix(R) which have orthogonal support (compare the proof of D-III, Theorem 1.10.(a)).\\
(b) If $R$ is a pseudo-resolvent with values in a w*-algebra such that $F i x\left(R^{\prime}\right)$ is contained in $M_{*}$, then it follows from the proof of D-III, Lemma 1.2 that there exists a sequence of normal states in Fix(R') whith orthogonal supports in M.

Lemma 4.3. Let $R$ be an identity preserving pseudo-resolvent of Schwarz type on $D=\{\lambda \in \mathbb{C}: \operatorname{Re}(\lambda)>0\}$ with values in the predual of a W*-algebra $M$. If the fixed space of the canonical extension $\hat{R}$ of $R$ to some ultrapower of $M_{*}$ is infinite dimensional, then there exists a sequence $\left(z_{n}\right)$ in $M_{1}^{+}$and a sequence of states ( $\phi_{n}$ ) in $M_{*}$ such that:\\
(a) $\lim _{n} z_{n}=0$ in the $s^{*}\left(M, M_{*}\right)$-topology.\\
(b) $\lim _{n}\left\|(I d-\lambda R(\lambda)) \phi_{n}\right\|=0$ for all $\lambda \in D$.\\
(c) $\phi_{n}\left(z_{n}\right) \geq \frac{1}{2}$ for all $n \in \mathbb{N}$.

Proof. Let $\left(M_{*}\right)$, be the ultrapower of $M_{*}$ with respect to some free ultrafilter $U$ on $\mathbb{N}$. Since $\left(M_{\star}\right)^{\wedge}$ is the predual of a W*-subalgebra of $\hat{M}^{\prime \prime}$ (see D-III, Remark 2.4.(b)), there exists a sequence of states $\left(\hat{\psi}_{n}\right)$ in $\operatorname{Fix}(\hat{R})$ such that the corresponding support projections are mutually orthogonal in $\hat{M}^{\prime}$ ' (Lemma 4.1). For every $n \in \mathbb{N}$ let $\left(\psi_{n, k}\right) \in \hat{\psi}_{n}$ be a representing sequence of states, let

$$
\phi:=\sum_{n, k} 2^{-(n+k+1)} \psi_{n, k}
$$

and let

$$
p:=\sup \left\{s\left(\psi_{n, k}\right): n, k=1, \ldots\right\}
$$

in $M$. Then $\phi$ is a normal state on $M$ which is faithful on the $W^{*}$-algebra $M_{p}$. Since

$$
1=\left\langle\psi_{n, k}, s\left(\psi_{n, k}\right)\right\rangle=\psi_{n, k}(p) \quad(n, k \in \mathbb{N})
$$

it follows $\hat{\psi}_{n}(\hat{\mathrm{p}})=1$ where $\hat{\mathrm{p}}$ is the canonical image of p in $\hat{\mathrm{M}}$.

But this implies $s\left(\hat{\psi}_{n}\right) \leq \hat{p}$ in $\hat{M}^{\prime \prime}$. since $\hat{M}_{1}{ }^{+}$is $\sigma\left(\hat{M}^{\prime}, \hat{M}^{\prime}\right)-$ dense in (M'') ${ }^{+}$(Kaplanskys density theorem [Sakai (1971), 1.9.1] in combination with [Sakai (1971), 1.8.9 and 1.8.12]), there exists for all $n \in \mathbb{N}$ a net $\left(\hat{z}_{n, \gamma}\right)$ in $\hat{M}_{1}^{+}$such that

$$
\sigma\left(\hat{M}^{\prime}, \hat{M}^{\prime}\right)-1 \mathrm{im}_{Y} \hat{z}_{n, Y}=s\left(\hat{\psi}_{n}\right)
$$

From [sakai (1971), 1.7.8] and the considerations above we obtain that the net $\left(\hat{p} \hat{z}_{n, \gamma} \hat{p}\right)$ converges to $s\left(\hat{\psi}_{n}\right)$ in the $\sigma\left(\hat{M}^{\prime}, \hat{M}^{\prime}\right)$-topology. Therefore we may assume $\mathrm{z}_{\mathrm{n}, \gamma} \in\left(\mathrm{M}_{\mathrm{p}}^{\wedge}\right)_{1}^{+}$. In the following we denote by $\hat{\phi}$ the canonical image of $\phi$ in $\left(M_{\star}\right)^{\wedge}$.

Since the projections $s\left(\hat{\psi}_{n}\right)$ are mutually orthogonal, there exists a real sequence $\left(r_{n}\right), 0<r_{n}<1, \lim _{n} r_{n}=0$ and $\hat{\phi}\left(s\left(\psi_{n}\right)\right) \leqq \frac{1}{2} r_{n}$. For all $n \in \mathbb{N}$ choose $\hat{z}_{n} \in\left(\hat{M}_{p}\right)_{1}{ }^{+}$such that

$$
\begin{aligned}
& \left|\left\langle\hat{\phi}_{,} s\left(\hat{\psi}_{n}\right)-\hat{z}_{n}\right\rangle\right| \leqq \frac{1}{2} r_{n}, \\
& \left|<\hat{\psi}_{n} s\left(\hat{\psi}_{n}\right)-\hat{z}_{n}\right\rangle \left\lvert\, \leqq \frac{1}{2} r_{n} .\right.
\end{aligned}
$$

Hence $\hat{\phi}\left(\hat{z}_{n}\right) \leqq r_{n}$ and $\hat{\psi}_{n}\left(\hat{z}_{n}\right) \geqq \frac{1}{2}$ for all $n \in \mathbb{N}$. For every $n \in \mathbb{N}$ let $\left(z_{n, k}\right) \in \hat{z}_{n}$ be a representing sequence in $\left(M_{p}\right)_{1}{ }^{+}=p\left(M_{1}{ }^{+}\right) p$ (note that $\left.\bar{m}_{p}^{\wedge}=\left(M_{p}\right)^{\wedge}\right)$ and $f i x \mu \in R_{+}$. Since $\mu R(\mu)^{\wedge} \hat{\psi}_{n}=\psi_{n}$, $\hat{\phi}\left(\hat{z}_{n}\right) \leqq r_{n}$ and $\hat{\psi}_{n}\left(\hat{z}_{n}\right) \geqq \frac{1}{2}$ there exists for all $n \in \mathbb{N}$ an element $U_{n} \in U$ such that for all $k \in U_{n}$ :

$$
\begin{aligned}
& \text { (i)' } \phi\left(z_{n, k}\right) \leqq r_{n}, \\
& \text { (ii)' }\left\|(I d-\mu R(\mu)) \psi_{n, k}\right\| r_{n}, \\
& \text { (iii)' } \psi_{n, k}\left(z_{n, k}\right) \geqq \frac{1}{2} .
\end{aligned}
$$

Inductively we find a sequence $\left(z_{n}\right)$ in $\left(M_{p}\right)_{1}{ }^{+}$and a sequence of states $\left(\phi_{n}\right)$ in $M_{*}$ such that for all $n \in \mathbb{N}$ :\\
(i) $" \quad \lim _{n} \phi_{n}\left(z_{n}\right)=0$,\\
(ii) ' $\lim _{n}\left\|(I d-\mu R(\mu)) \phi_{n}\right\|=0$,\\
(iii)' ' $\phi_{n}\left(z_{n}\right) \geqq \frac{1}{2}$.

Since $\phi$ is faithful on $M_{p}$, condition (i)' implies that $\lim _{n} z_{n}=$ 0 in the $s^{*}\left(\mathrm{M}_{\mathrm{p}},\left(\mathrm{M}_{\mathrm{p}}\right)_{*}\right)$-topology [Takesaki(1979), Proposition III.5.4].

Since $s^{*}\left(M_{p},\left(M_{p}\right)_{*}\right)=s^{*}\left(M_{,} M_{*}\right) \mid M_{p} \quad(a)$ follows immediately from (i)' '. Using the resolvent equation for $R$ it is easy to see that (ii)'' implies

$$
\lim _{n}\left\|(I d-\lambda R(\lambda)) \phi_{n}\right\|=0
$$

for all $\lambda \in D$ and the proof is complete.

Without further comments we will make use of the following facts in the rest of this section :\\
(1) A sequence $\left(\phi_{n}\right)$ in $M^{\prime}+$ converges in the $\sigma\left(M^{\prime}, M\right)$-topology if and only if it converges in o( ${ }^{\prime \prime} \mathrm{M}^{\prime \prime}$ )-topology [Akeman-Dodds-Gamlen (1972)].\\
(2) We can decompose $\phi \in M_{+}^{\prime}$ into its normal and singular part $\phi=$ $\phi^{(n)}+\phi^{(s)}, 0 \leqq \phi^{(n)} \in M_{*}, 0 \leqq \phi^{(s)} \in M_{*} \perp$ and $\|\phi\|=\| \|^{(n)}\|+\|\left\|^{(s)}\right\|$ [Takesaki (1979), Theorem III.2.14].\\
(3) If $\left(\phi_{n}\right)$ is a sequence in $M_{*}$ which converges to zero in the $\sigma\left(M_{*}, M\right)$-topology and if $\left(x_{n}\right)$ is a sequence in $M$ which converges to zero in the $s^{*}\left(M, M_{*}\right)$-topology, then $\lim _{n} \phi_{k}\left(x_{n}\right)=0$ uniformly in $\mathrm{k} \in \mathbb{N}$ [Takesaki (1979), Lemma III.5.57.

Theorem 4.4. Let $R$ be an identity preserving pseudo-resolvent on $D=\{\lambda \in \mathbb{C}: \operatorname{Re}(\lambda)>0\}$ with values in a $W^{*}$-algebra $M$ which is of Schwarz type and let $R^{\prime}$ its adjoint pseudorresolvent. Any one of the following conditions implies dim Fix $(\hat{R})<\infty$ in some ultrapower of M.\\
(a) The fixed space of $R^{\prime}$ is finite dimensional.\\
(b) $\quad 1 i m_{\mu+0} \mu R(\mu)=p$ exists in the strong operator topology and $\operatorname{rank}(P)<\infty$.\\
(c) The fixed space of $R^{\prime}$ is contained in $M_{*}$.\\
(d) Every map $\mu R(\mu), \mu \in \mathbb{R}_{+}$, is irreducible on $M$.

Proof. Suppose that the dimension of the fixed space of ( ' $^{\prime}$ ) in some ultrapower (M')\^{} of $M^{\prime}$ is infinite dimensional. Since (M')\^{}\\
is the predual of the W*-algebra $\hat{M}^{\prime \prime}$ and $R^{\prime}$ is identity preserving (since R''I = RI = 1 ) and of Schwarz type (because $\mu \mathrm{R}^{\prime}(\mu)=$ $(\mu R(\mu))^{\prime \prime}$ is a Schwarz map for all $\mu \in \mathbb{R}_{+}$) we may apply Lemma 4.3.

Suppose that the fixed space of the canonical extension of $R^{\prime}$ to some ultrapower of $\mathrm{M}^{\prime}$ is infinite dimensional. Thus we may choose a sequence of states $\left(\phi_{k}\right)$ in $M^{\prime}$ and a sequence $\left(z_{k}\right)$ in $\left(M^{\prime}\right)_{1}{ }^{+}$ satisfying (a) - (b) of Lemma 4.3. Remark (3) above implies that no subsequence of $\left(\phi_{k}\right)$ can converge in the $\sigma\left(M^{\prime}, M^{\prime}\right)$-topology.\\
(a) If $\phi$ is a $\sigma\left(M^{\prime}, M\right)$-accumulation point of $\left(\phi_{k}\right)$, then $\phi \in F i x\left(R^{\prime}\right)$. Since Fix(R') is finite dimensional, the set of accumulation points of the sequence $\left(\phi_{k}\right)$ is metrizable in the $\sigma\left(M^{\prime}, M\right)-$ topology. Hence there exists a sequence ( $k(n)$ ) of natural numbers, such that $\sigma\left(M^{\prime}, M\right)-\lim _{n} \phi_{k}(n)=\phi$. Consequently, $\phi=\sigma\left(M^{\prime}, M^{\prime}\right)-1 i m_{n}$ $\phi_{k}(n)$ by Remark (1) above. But this leads to a contradiction, which proves (a).\\
(b) Since $\operatorname{dim} F i x(R)=\operatorname{dim} F i x\left(R^{\prime}\right)=\operatorname{rank}(P)<\infty$, (b) follows from (a).\\
(c) Suppose that the fixed space of $R^{\prime}$ is infinite dimensional. Since Fix(R') $\subseteq M_{*}$ there exists a sequence of states ( $\psi_{n}$ ) in Fix(R') with mutually orthogonal support projections in $M$ (Lemma 4.1). Since every o( $\left.\mathrm{M}^{\prime}, \mathrm{M}\right)$-accumulation point of the $\psi_{n}{ }^{\prime} \mathrm{s}$ belongs to Fix(R'), hence is normal, the sequence ( $\psi_{n}$ ) is relatively $\sigma\left(M_{\star}, M\right)$-compact. By Eberleins theorem, we may assume that this sequence is weakly convergent. By the orthogonality of the $s\left(\psi_{n}\right)^{\prime} s$ this sequence converges to zero in the $s^{*}\left(M_{*} M_{*}\right)$-topology, hence $\lim _{n} \psi_{k}\left(s\left(\psi_{n}\right)\right)=0 \quad$ uniformly in $k \in \mathbb{N}$, a contradiction. Consequently dim Fix $(R)<\infty$ and $(c)$ is proved.\\
(d) We prove dim Fix(R') = 1 and apply (a) once again. Useful for this is the following observation : If $\psi$ is a faithful state on M then the normal part is faithful too. Indeed, if $0 \neq x \in M$ such that $\psi^{(n)}(x)=0$ choose a projection $0 \neq p \in M$ such that $\psi^{(n)}(p)=$ $\psi(s)(p)=0$ (use [Takesaki (1979), Theorem III.3.8]), hence $\psi(p)=0$ which conflicts with the faithfulness of $\psi$.

If 2 : dim Fix $\left(R^{\prime}\right)$ there are states $\psi_{1}$ and $\psi_{2}$ in $F i x\left(R^{\prime}\right)$ such that the corresponding support projections are orthogonal in M'' (Remark 4.2). Since every R'-invariant state $\psi$ is faithful on $M$, $\psi_{i}(n) \neq 0$ (otherwise the norm closed face $\left\{\psi(x)=0: x \in M_{+}\right\}$would\\
be non trivial and $\mu \mathrm{R}(\mu)$-invariant). The support projections of the $\psi_{i}(n) ' s$ in M'' are orthogonal (since $\psi_{i}{ }^{(n)} \leqq \psi_{i}$ ) and different from zero. Let $\left(z_{\gamma}\right)$ be a net in $M_{1}{ }^{+}$such that

$$
\sigma\left(M^{\prime \prime}, M^{\prime}\right)-1 i m_{\gamma} z_{\gamma}=s\left(\psi_{1}^{(n)}\right)
$$

Then $\lim _{\gamma} \psi_{1}{ }^{(n)}\left(z_{\gamma}\right)=1$ whereas $\lim _{\gamma} \psi_{2}{ }^{(n)}\left(z_{\gamma}\right)=0$. Let $z$ be a $\sigma\left(M_{1} M_{*}\right)$-accumulation point of $\left(z_{\gamma}\right)$ in $M_{+}$. Since every $\psi_{i}(n)$ is normal, $\psi_{1}(n)(z)=1$ whereas $\psi_{2}^{Y}(n)(z)=0$. The first condition implies $z \neq 0$ whereas the second shows that $\psi_{2}(n)$ cannot be faithful. Since this is a contradiction, it follows dim $F i x\left(R^{\prime}\right)=1$ which proves (d).

The next corollary is an easy application of Theorem 4.4 and of D-III, Proposition 2.3.

Corollary 4.5. Let $T$ be an identity preserving semigroup of Schwarz type on the predual of a $W^{*}$-algebra $M$. Then the following assertions are equivalent:\\
(a) $T$ is uniformly ergodic with finite dimensional fixed space.\\
(b) The adjoint weak*-semigroup is strongly ergodic with finite dimensional fixed space.\\
(c) Every T''-invariant state is normal.

Proof. If (a) is fulfilled then the semigroup $T$ is strongly ergodic on $M_{*}$. Since

$$
\operatorname{dim} \operatorname{Fix}(T)=\operatorname{dim} \operatorname{Fix}\left(T^{+}\right)<\infty
$$

there exist normal states $\phi_{1}, \ldots, \phi_{n}$ in $\operatorname{Fix}(T)$ and $x_{1}, \ldots, x_{k}$ in Fix $\left(T^{\prime}\right)$ such that $\phi_{n}\left(x_{m}\right)=\delta_{n, m} \quad(1 \leqq n, m \leqq k)$ and

$$
P=\sum_{i=1}^{k} \phi_{i} 8 x_{i}
$$

is the associated ergodic projection. If $(C(s))$ s>0 is the family of Césaro means of $T$, then

$$
\lim _{s \rightarrow \infty} c(s) ' \prime(\psi)=\sum_{i=1}^{k} \phi_{i}(\psi) x_{i} \in M_{*}
$$

for every $\psi \in M^{\prime}$. Hence Fix (T'') $\subseteq M_{*}$ which proves (c).

If (c) is fulfilled then $\operatorname{Fix}(T)=\operatorname{Fix}(T ')$. Therefore the fixed space of $T^{\prime}$ separates the points of Fix(T'') which shows that $T^{\prime}$ is strongly ergodic on M ([Krengel (1985), Chap.2, Thm.1.4]). If\\
(b) holds then

$$
P=\lim _{\mu+0} \mu R\left(\mu, A^{\prime}\right)
$$

exists in the strong operator topology, where $A^{\prime}$ is the generator of $T^{\prime}$. Therefore dim Fix $(\mu R(\mu))^{N}<\infty$ in some ultrapower of M (Theorem 4.4.(b)). It follows from D-III, Proposition 2.3 that 0 is a pole of the resolvent of $\mathrm{R}(., \mathrm{A})$. Therefore $T$ is uniformly ergodic.

NOTES.\\
Section 1. The stability concepts appearing in Theorem 1.7 coincide not only for positive semigroups on C*-algebras but on any order unit Banach space. We refer to Batty-Robinson (1984) for this more general setting and to B-IV, Section 1 for the analogous results on $\mathrm{C}_{\mathrm{O}}(\mathrm{X})$.

Section 2. Theorem 2.2 generalizes the Liapunov stability theorem from the matrix algebra $B\left(\mathbb{C}^{\text {n }}\right)$ to arbitrary $\mathrm{W}^{*}$-algebras. For the algebra $\mathrm{B}(\mathrm{H})$ it is due to Mil'stein (1975) and in the general form to Groh-Neubrander (1981).

Section 3. From the many papers dealing more or less explicitely with the asymptotic behavior of semigroups on operator algebras we quote Frigerio-Verri (1982) and Watanabe (1982). The background for our ergodic theorems (Prop.3.3 and Prop. 3.4) can be found best in Krengel (1985). The "automatic" convergence theorem for an irreducible $W^{*}$-dynamical system on $\mathrm{B}(\mathrm{H})$ stated in Corollary 3.9 is the continuous version of a result in Groh (1984c). Finally, the characterization of convergence towards a periodic semigroup through spectral properties of the generator (Thm. 3.11) is due to Nagel (1984) in the commutative, i.e. L ${ }^{2}$ ( $\mu$ ), case (see also C-IV, Thm.2.14).

Section 4. Again we refer to Krengel (1985) for the (uniform) ergodic theory for a single operator or a one-parameter semigroup on a Banach space. The characterization given in Corollary 4.5 for positive semigroups on W*-algebras is based on a sophisticated use of ultrapower techniques and has its discrete forerunners in Lotz (1981) and Groh (1984b).

\section*{BIBLIOGRAPHY}
Abraham, R.; Marsden, J.E.\\[0pt]
[1978] Foundations of Mechandcs.\\
London-Amsterdam: Benjamin / Cummings 1978.

Abramovich, Y.A.\\[0pt]
[1983] Multiplicative representation of disjointness preserving operators. Indag. Math. 45 (1983), 265-279.

Akemann, C.A.; Dodds, P.G.; Gamlen, J.L.B.\\[0pt]
[1972] Weak compactness in the dual space of a W*-algebra. J. Funct. Anal. 16 (1972), 446-450.

Ando, T.\\[0pt]
[1961] Convergent sequences of finitely additive measures. Pacific J. Math. 11 (1961), 395-404.

Albeverto, S.; Hdegh-Krohn, R.\\[0pt]
[1978] Frobenius theory for positive maps on von Neumann algebras. Comm. Math. Phys. 64 (1978), 83-94.

Amann, H.\\[0pt]
[1976] Fixed point equations and nonlinear elgenvalue problems in ordered Banach spaces. SIAM Rev. 18 (1976), 620-709.\\[0pt]
[1983] Dual semigroups and second order linear elliptic boundary value problems. Israel J. Math. 45 (1983), 225-254.

Arendt, W.\\[0pt]
[1982] Kato's equality and spectral decomposition for positive $\mathrm{C}_{\mathrm{o}}$-groups. Manuscripta Math. 40 (1982), 277-298.\\[0pt]
[1983] Spectral properties of Lamperti operators. Indiana Univ. Math. J. 32 (1983), 199-215.\\[0pt]
[1984a] Generators of positive semigroups.\\
In: F. Kappel; W. Schappacher (eds.): Infinite-dimensional Systems, Retzhof 1983. Lecture Notes in Math. 1076, 1-15. Berlin-HeidelbergNew York: Springer 1984.\\[0pt]
[1984b] Kato's inequality. A characterization of generators of positive semigroups. Proc. Roy. Irish Acad, Sect, A 84 (1984), 155-174.\\[0pt]
[1984c] Resolvent positive operators and integrated semigroups. Semesterbericht Funktionalanalysis, Tübingen, Sommersemester 1984, 73-101.\\[0pt]
[1985] Resolvent positive operators. Universität Tübingen, Preprint 1985.

Arendt, W.; Chernoff, P.; Kato, T.\\[0pt]
[1982] A generalization of dissipativity and positive semigroups. J. Operator Theory 8 (1982), 167-180.

Arendt, W.; Greiner, G.\\[0pt]
[1984] The spectral mapping theorem for one-parameter groups of positive operators on $C(X)$.\\
Semigroup Forum 30 (1984), 297-330.\\
Arino, O.; Kimme1, M.\\[0pt]
[1985] Asymptotic analysis of a cell-cycle model based on unequal division. Preprint 1985.

Asimow, L.; Ellis, A.J.\\[0pt]
[1980] Convexity Theory and its Applications in Functional Analysis. London-New York-San Francisco: Academic Press 1980.

Axmann, D.\\[0pt]
[1980] Struktur- und Ergodentheorie irreduzibler Operatoren auf Banachverbänden. Dissertation, Tübingen 1980.

Baras P.; Pierre M.\\[0pt]
[1985] Critère d'existence de solutions positives pour des équations semi-linéaires non monotones.\\
Preprint. Nancy 1985.\\
Bart, H.\\[0pt]
[1977] Periodic strongly continuous semigroups. Ann. Mat. Pura App1. 115 (1977), 311-318.

Batty, C.J.K.\\[0pt]
[1978] Dissipative mappings and well-behaved derivations. J. London Math. Soc. 18 (1978), 527-533.\\[0pt]
[1981] Derivations on compact spaces. Proc. London Math. Soc. 42 (1981), 299-330.

Batty, C.J.K.; Davies, E.B.\\[0pt]
[1982] Positive semigroups and resolvents. J. Operator Theory 10 (1982), 357-363.

Batty, C.J.K.; Robinson, D.W.\\[0pt]
[1984] Positive one-parameter semigroups on ordered spaces. Acta. Appl. Math. $\underline{2}$ (1984), 221-296.\\[0pt]
[1985] The characterization of differential operators by locality: abstract derivations. Ergodic Theory Dynamical Systems 5 (1985), 171-183.

Bauer, H.\\[0pt]
[1966] Harmonische Räume und ihre Potentialtheorie. Berlin-Heidelberg-New York: Springer 1966.

Beals, R.\\[0pt]
[1972] On the abstract Cauchy problem. J. Funct. Anal. 10 (1972), 281-299.

Belleni-Morante, A.\\[0pt]
[1979] Applied Semigroups and Evolution Equations. Oxford: Oxford University Press 1979.

Bellmann, R.; Cooke, K.L.\\[0pt]
[1963] Differential-Difference Equations. London-New York: Academic Press 1963.

Belyi, A.G.; Semenov, Y.A.\\[0pt]
[1975] Kato's inequality and semigroup product-formulas. Functional Anal. App1. $\underline{9}$ (1975), 320-321.

Benchimol, C.D.\\[0pt]
[1978a] A note on weak stabilizability of contraction semigroups. SIAM J. Control Optim. 16 (1978), 373-379.\\[0pt]
[1978b] Feedback stabilizability in Hilbert spaces. Appl. Math. Optim. 4 (1978), 223-248.

Bénilan, P.; Picard, C.\\[0pt]
[1979] Quelques aspects non linéaires du principe du maximum.\\
In: Séminaire de Théorie du Potentiel. Lecture Notes in Math. 713. Springer 1979.

Berg, C.; Forst, G.\\[0pt]
[1975] Potential Theory on Locally Compact Abelian Groups. Berlin-Heidelberg-New York: Springer 1975.

Berger, C.A.; Coburn, L.A.\\[0pt]
[1970] One-parameter semigroups of isometries. Bu11. Amer. Math. Soc. 76 (1970), 1125-1129.

Beurling, A.\\[0pt]
[1970] On analytic extensions of semigroups of operators. J. Funct. Anal. 6 (1970), 387-400.

Beurling, A.; Deny, J.\\[0pt]
[1948] Espaces de Dirichlet I: Le cas èlémentaire. Acta Math. 99 (1948), 203-224.

Di Blasio, G.; Kunisch, E.; Sinestrari, E.\\[0pt]
[1983] The solution operator for a partial differential equation with delay. Atti Accad. Naz. Lincei Rend. C1. Sci. Fis. Mat. Natur. 74 (1983), 228-233.\\[0pt]
[1984] Stability for abstract linear functional equations. To appear in: Israel J. Math.

Bony, J.-M.; Courrège, P.; Priouret, P.\\[0pt]
[1968] Semi-groups de Feller sur une variété à bord compact et problèmes aux limites intégro-différentiels du second ordre donnant lieu au principe du maximum. Ann. Inst. Fourier (Grenoble) 18 (1968), 369-521.

Bourbaki, N.\\[0pt]
[1955] Eléments des Mathématiques, Intégration, Chapitre 5: Intégration des Mesures. Paris: Hermann 1955.

Bourgain, J.\\[0pt]
[1980] Propriétés de rèlevement et projections dans les espace $\mathrm{L}^{1} / \mathrm{H}_{\mathrm{o}}^{1}$ et $\mathrm{H}^{\infty}$. C. R. Acad. Sci. Paris Sér. A-Math. 291 (1980), 607-609.\\[0pt]
[1985] Some new properties of the Banach spaces $\mathrm{L} / \mathrm{H}_{\mathrm{O}}^{1}$ and $\mathrm{H}^{\infty}$ (Part II). preprint 1985.

Boyadzhiev, H.N.\\[0pt]
[1984] Characterization of the generators of $C_{0}$ semigroups which leave a convex set invariant.\\
Commen. Math. Univ. Carolin. 25 (1984), 159-170.\\
Brattell, O.; Digernes, T.; Robinson, D.W.\\[0pt]
[1983] Positive semigroups on ordered Banach spaces. J. Operator Theory $\underline{9}$ (1983), 371-400.

Bratteli, O.; Jørgensen, P.E.T.\\[0pt]
[1984] Positive Semigroups of Operators and Applications. Special issue of Acta Appl. Math. 2 (1984), Dordrecht / Boston: Reide1 1984.

Bratteli, O.; Kishimoto, A.; Robinson, D.W.\\[0pt]
[1980] Positivity and monotonity properties of $C_{0}$-semigroups, I. Comm. Math. Phys. 75 (1980), 67-84.

Bratteli, O.; Robinson, D.W.\\[0pt]
[1975] Unbounded derivations of C*-algebras.\\
Comm. Math. Phys. 42 (1975), 253-268.\\[0pt]
[1979] Operator Algebras and Quantum Statistical Mechanics I. New York-Heidelberg-Berlin: Springer 1979; II, ibid. 1981.\\[0pt]
[1981] Positive C-semigroups on C*-algebras. Math. Scand. 49 (1981), 259-274.

Calvert, B.D.\\[0pt]
[1970] Nonlinear evolution equations in Banach lattice.\\
Bull. Amer. Math. Soc. 76 (1970), 845-850.\\[0pt]
[1971a] Nonlinear equations of evolution.\\
Pacific J. Math. 39 (1971), 293-350.\\[0pt]
[1971b] Semigroups on an ordered Banach space.\\
J. Math. Soc. Japan 23 (1971), 311-319.\\[0pt]
[1972] On T-accretive operators.\\
Ann. Mat. Pura App1. 94 (1972), 291-314.\\
Calvert, B.D.; Picard, C.\\[0pt]
[1975] Opérateurs accrétifs et $\Phi$-accrétifs dans un espace de Banach. Hiroshima Math. J. 5 (1975), 363-370.\\
van Casteren, J.\\[0pt]
[1984] Invariant subsets of strongly continuous semigroups. Integral Equations Operator Theory 7 (1984), 884-892.\\[0pt]
[1985] Generators of Strongly Continuous Semigroups. Boston-London-Melbourne: Pitman 1985.

Chicone, C.; Swanson, R.C.\\[0pt]
[1981] Spectral theory for linearizations of dynamical systems. J. Differential Equations. 40 (1981), 155-167.

Choi, M.-D.\\[0pt]
[1974] A Schwarz inequality for positive linear.maps on C*-algebras. Illinois J. Math. 18 (1974), 565-574.

Choquet, G.; Foias, C.\\[0pt]
[1975] Solution d'un problème sur les itérés d'un opérateur positif sur C(K) et propriétés des moyennes associées.\\
Ann. Inst. Fourier (Grenoble) 25 (1975), 109-125.\\
Coffman, C.V.; Grover, C.L.\\[0pt]
[1980] Obtuse cones in Hilbert spaces and application to partial differential equations.\\
J. Funct. Anal. 35 (1980), 369-396.

Collatz, P.\\[0pt]
[1942] Efnschließungssatz für die charakteristischen Zahlen von Matrizen. Math. Z. 48 (1942), 221-226.

Combes, F.; Delaroche, C.\\[0pt]
[1978] Répresentations des groupes localement compacts et applications aux algèbres d'opérateurs.\\
Astérisque (Séminaire d'Orléans) 55 (1978).\\
Cooke, K.L.; Ferreira, J.M.\\[0pt]
[1983] Stability conditions for linear retarded differential equations. J. Math. Ana1. App1. 96 (1983), 480-504.

Coulhon, T.\\[0pt]
[1984] Suites d'opérateurs sur un espace C(K) de Grothendieck. C. R. Acad. Sci. Paris Sér. I-Math. 298 (1984), 13-15.

Cornfeld, I.P.; Fomin, S.V.; Sinai, Ya.G.\\[0pt]
[1982] Ergodic Theory. Berlin-Heidelberg-New York: Springer 1982.

Crandall, M.G.; Tartar, L.\\[0pt]
[1980] Some relations between nonexpansive and order preserving mappings. Proc. Amer. Math. Soc. 78 (1980), 385-390.

Datko, R.\\[0pt]
[1970] Extending a theorem of A.M. Liapunov to Hilbert space. J. Math. Anal. App1. 32 (1970), 610-616.\\[0pt]
[1972] Uniform asymptotic stability of evolutionary processes in a Banach space. SIAM J. Math. Anal. 3 (1972), 428-445.\\[0pt]
[1983] An example of an unstable neutral differential equation. Internat. J. Control 38 (1983), 263-267.

Davies, E.B.\\[0pt]
[1972] Some contraction semigroups in quantum probability. Z. Wahrsch. Verw. Gebiete 23 (1972), 261-273.\\[0pt]
[1976] Quantum Theory of Open Systems. London-New York-San Francisco: Academic Press 1976.\\[0pt]
[1979] Generators of dynamical semigroups. J. Funct. Anal. 34 (1979), 421-431.\\[0pt]
[1980] One-parameter Semigroups. London-New York-San Francisco: Academic Press 1980.\\[0pt]
[1982] The harmonic functions of mean ergodic semigroups. Math. Z. 181 (1982), 543-552.\\[0pt]
[1986] Spectral properties of some second order elliptic operators on $\mathrm{L}^{\mathrm{P}}$-spaces. In: R. Nage1; U. Schlotterbeck; M.P.H. Wolff (eds.): Aspects of Positivity in Functional Analysis. Amsterdam: North Holland 1986.

Deimling, K.\\[0pt]
[1977] Ordinary Differential Equations in Banach Spaces. Berlin-Heidelberg-New York: Springer 1977.

DeLeeuw, K.; Glicksberg, I.\\[0pt]
[1961] Applications of almost periodic compactifications. Acta Math. 105 (1961), 63-97.

Derndinger, R.\\[0pt]
[1980] Über das Spektrum positiver Generatoren.\\
Math. Z. 172 (1980), 281-293.\\[0pt]
[1984] Betragshalbgruppen normstetiger Operatorhalbgruppen. Arch. Math. 42 (1984), 371-375.

Derndinger, R.; Nagel, R.\\[0pt]
[1979] Der Generator starkstetiger Verbandshalbgruppen auf $\mathrm{C}(\mathrm{X})$ und dessen Spektrum.\\
Math. Ann. 245 (1979), 159-177.\\
Desch, W.; Schappacher, W.\\[0pt]
[1983] Spectral properties of finite-dimensional perturbed linear semigroups. Universität Graz, Preprint 1983.\\[0pt]
[1984] On relatively bounded perturbations of linear C-semigroups. Ann. Scuola Norm. Sup. Pisa 11 (1984), 327-341. ${ }^{\circ}$

Diekmann, O.; Heijmans, H.J.A.M.; Thieme, H.R.\\[0pt]
[1984] On the stability of the cell size distribution. J. Math. Biol. 19 (1984), 227-248.

Dieudonné, J.\\[0pt]
[1971] Eléments d'Analyse (Tome IV). Paris: Gauthier-Villars 1971.

Doetsch, G.\\[0pt]
[1950] Handbuch der Laplace Transformation, Band I, Base1: Birkhäuser 1950.

Dorroh, J.R.\\[0pt]
[1966] Contraction semi-groups in a function space. Pactfic J. Math. 19 (1966), 35-38.

Dunford, N.; Schwartz, J.T.\\[0pt]
[1958] Linear Operators, Part I: General Theory. New York: WLley 1958.

Dynkin, E.B.\\[0pt]
[1965] Markov Processes I , II.\\
Berlin-Göttingen-Heidelberg: Springer 1965.\\
Dyson, J.; Villella-Bressan R.\\[0pt]
[1979] Semigroups of translations associated with functional and functional differential equations Proc. Roy. Soc. Edinburgh Sect. A 82 (1979), 171-188.

Eberlein, W.F.\\[0pt]
[1948] Abstract ergodic theorems and weak almost periodic functions.\\
Trans. Amer. Math. Soc. 67 (1948), 217-240.

Elllot, G.\\[0pt]
[1972] Convergence of automorphisms on certain C*-algebras.\\
J. Funct. Anal. 11 (1972), 204-206.

Evans, D.E.\\[0pt]
[1976] On the spectrum of a one-parameter strongly continuous representation. Math. Scand. 39 (1976), 80-82.\\[0pt]
[1977] Irreducible quantum dynamical semigroups. Comm. Math. Phys. 54 (1977), 293-297.\\[0pt]
[1984] Quantum dynamical semigroups, symetry groups, and locality. Acta. App1. Math $\underline{2}$ (1984), 333-352.

Evans, D.E.; Hanche-Olsen, H.\\[0pt]
[1979] The generators of positive semigroups. J. Funct. Anal. 32 (1979), 207-212.

Fattorini, H.O.\\[0pt]
[1969a] Ordinary differential equations in linear topological spaces, I. J. Differential Equations 5 (1969), 72-105.\\[0pt]
[1969b] Ordinary differential equations in linear topological spaces, II. J. Differential Equations 6 (1969), 50-70.\\[0pt]
[1983] The Cauchy Problem.\\
Reading (Mass.): Addison-Wesley 1983.\\
Feller, W.\\[0pt]
[1952] The parabolic differential equation and the associated semigroups of transformations.\\
Ann. of Math. 55 (1952), 468-519.\\[0pt]
[1953a] On the generation of unbounded semigroups of bounded linear operators. Ann. of Math. 58 (1953), 166-174.\\[0pt]
[1953b] On positivity preserving semigroups of transformations on $\mathrm{C}\left[\mathrm{r}_{1}, \mathrm{r}_{2}\right]$. Ann. Soc. Math. Polon. 25 (1953), 85-94.\\[0pt]
[1954a] The general diffusion operator of positivity preserving semigroups in one dimension.\\
Ann. of Math. 60 (1954), 417-436.\\[0pt]
[1954b] Diffusion processes in one dimension. Trans, Amer. Math. Soc. 77 (1954), 1-31.\\[0pt]
[1955] On second order differential operators. Ann. of Math. 61 (1955), 90-105.\\[0pt]
[1956] Boundarles induced by non-negative matrices. Trans. Amer. Math. Soc. 83 (1956), 19-54.\\[0pt]
[1957] On boundaries defined by stochastic matrices. Applied probability. Proceedings of Symposia in Applied Mathematics, Vol. VII, 35-40, New York: McGraw Hill 1957.\\[0pt]
[1959] Differential operators with the positive maximum property. I111nois J. Math. 3 (1959), 182-186.

Fisher, S.D.\\[0pt]
[1983] Function Theory on Planar Domains. New York: Wiley 1983.

Foias, C.\\[0pt]
[1973] Sur une question de M. Reghis. An. Univ. Timipoara Ser. Stiint. Mat. 11 (1973), 111-114.

Frigerio, A.; Verri, M.\\[0pt]
[1982] Long-time asymptotic properties of dynamical semigroups on W*-algebras. Math. Z. 180 (1982), 275-286.

Frobenius, G.\\[0pt]
[1909] Uber Matrizen aus positiven Elementen. Sitzungsber. Preuß. Akad. Wiss., Physikal.-Math. Kl. (1908), 471-476; ibid. (1909), 514-518.

Fukushima, M.\\[0pt]
[1982] Dirichlet Forms and Markov Processes. London: North Holland 1980.

Gearhart, L.\\[0pt]
[1978] Spectral theory for contraction semigroups on Hilbert spaces. Trans. Amer. Math. Soc. 236 (1978), 385-394.

Gilbarg, D.; Trudinger, N.S.\\[0pt]
[1977] Elliptic Partial Differential Equations of Second Order. Berlin: Springer 1977.

Goldstein, J.A.\\[0pt]
[1981] Some developments in semigroups since Hille Phillips. Integral Equations Operator Theory 4 (1981), 350-365.\\[0pt]
[1985a] Semigroups of Operators and Applications. Oxford University Press 1985.\\[0pt]
[1985b] A (more-or-less) complete bibliography of semigroups of operators through 1984. Preprints and Lecture Notes in Mathematics. Tulane University 1985.\\[0pt]
[1986] Asymptotics for bounded semigroups in Hilbert spaces. In: R. Nage1; U. Schlotterbeck; M.P.H. Wolff (eds.): Aspects of Positivity in Functional Analysis. Amsterdam: North Holland 1986.

Greiner, G.\\[0pt]
[1981] Zur Perron-Frobenius Theorie stark stetiger Halbgruppen. Math. 2. 177 (1981), 401-423.\\[0pt]
[1982] Spektrum und Asymptotik stark stetiger Halbgruppen positiver Operatoren. Sitzungsber. Heidelb. Akad. Wiss., Math.-Naturwiss. KI. (1982) 55-80.\\[0pt]
[1984a] A typical Perron-Frobenius theorem with applications to an age-dependent population equation.\\
In: F. Kappel; W. Schappacher (eds.): Infinite-dimensional Systems, Retzhof 1983. Lecture Notes in Math. 1076, 86-100. Berlin-HeidelbergNew York: Springer 1984.\\[0pt]
[1984b] Spectral properties and asymptotic behavior of the linear transport equation. Math. Z. 185 (1984), 167-177.\\[0pt]
[1984c] A spectral decomposition of strongly continuous groups of positive operators.\\
Quart. J. Oxford (2) 35 (1984), 37-47.\\[0pt]
[1984d] An irreducibility criterion for the linear transport equation.\\
Semesterbericht Funktionalanalysis, Tübingen, Sommersemester 1984, 1-8.\\[0pt]
[1985] Some applications of Fejer's theorem to one-parameter semigroups. Preprint 1985.\\[0pt]
[1986] Perturbing the boundary conditions of a generator.\\
To appear in: Houston J. Math.\\
Greiner, G.; Nage1, R.\\[0pt]
[1982] La loi "zero ou deux" et ses conséquences pour le comportement asymptotique des opérateurs positifs.\\
J. Math. Pures Appl. 9 (1982), 261-273.\\[0pt]
[1983] Op the stability of strongly continuous semigroups of positive operators on $\mathrm{L}^{2}(\beta)$.\\
Ann. Scuola Norm. Sup. Pisa 10 (1983), 257-262.\\
Greiner, G.; Voigt, J.; Wolff, M.P.H,\\[0pt]
[1981] On the spectral bound of the generator of semigroups of positive operators. J. Operator Theory 5 (1981), 245-256.

Groh, U.\\[0pt]
[1981] The peripheral point spectrum of Schwarz operators on C*-algebras Math. Z. 176 (1981), 311-318.\\[0pt]
[1982a] Some observations on the spectra of positive operators on finite dimensional C*-algebras. Linear Algebra Appl. 42 (1982), 213-222.\\[0pt]
[1982b] Asymptotic behavior of dynamical systems on W*-algebras. Semesterbericht Funktionalanalysis, Tübingen, Sommersemester 1982, 15-25.\\[0pt]
[1984a] Uniformly ergodic maps on C*-algebras. Israel J. Math. 47 (1984), 227-235.\\[0pt]
[1984b] Uniform ergodic theorems for identity preserving Schwarz maps on W*-algebras.\\
J. Operator Theory 11 (1984), 395-404.\\[0pt]
[1984c] Spectrum and asymptotic behaviour of completely positive maps on $B(H)$. Math. Japonica 29 (1984), 395-402.

Groh, U.; Kümmerer, B.\\[0pt]
[1982] Bibounded operators on W*-algebras. Math. Scand. 50 (1982), 269-285.

Groh, U.; Neubrander, F.\\[0pt]
[1981] Stabilität starkstetiger positiver Operatorhalbgruppen auf C*-Algebren. Math. Ann. 256 (1981), 129-173.

Grothendieck,A.\\[0pt]
[1953] Sur les applications linéaires faiblement compactes d'espaces du type $C(K)$. Canadian J. Math. 5 (1953), 129-173.

Gustafson, K.; Lumer, G.\\[0pt]
[1972] Multiplicative perturbation of semigroup generators. Pacific J. Math. 41 (1972), 731-742.

Gyllenberg, M.; Heijmans, H.J.A.M.\\[0pt]
[1985] An abstract delay equation modelling size dependent cell growth and division.\\
Preprint 1985.

Hadeler, K.-P.\\[0pt]
[1978] Delay-equations in biology.\\
In: H.-O. Peitgen; H.-O. Walther (eds.) Functional Differential Equations and Approximation of Flxed Points, Bonn 1978. Lecture Notes in Math. 730, 136-156. Berlin-Heidelberg-New York 1978.

Hale, J.\\[0pt]
[1977] Theory of Functional Differential Equations. New York-Heidelberg-Berlin: Springer 1977.

Hamel, G.\\[0pt]
[1905] Eine Basis aller Zahlen und die unstetigen Lösungen der Funktionalgleichung: $f(x+y)=f(x)+f(y)$.\\
Math. Ann. 60 (1905), 459-462.\\
Hasegawa, M.\\[0pt]
[1966] On contraction semigroups and (di)-operators. J. Math. Soc. Japan 18 (1966), 290-302.

Heijmans, H.J.A.M.\\[0pt]
[1985a] Structured populations, linear semigroups and positivity.\\
To appear in: Math. Z.\\[0pt]
[1985b] An eigenvalue problem related to cell growth.\\
J. Math. Anal. Appl. 111 (1985), 253-280.\\[0pt]
[1986] The dynamical behavior of the age-size distribution of a cell population. In: J.A.J. Metz; O. Diekmann (eds.) Dynamics of Physiologically Structured Population. Springer Lecture Notes Biomathematics (to appear).

Hempel, R.; Voigt, J.\\[0pt]
[1985] The spectrum of a Schrödinger operator in $L_{p}\left(\mathbb{R}^{V}\right)$ is p-independent. Preprint. München 1985.

Herbst, I.W.\\[0pt]
[1982] Contraction semigroups and the spectrum of $\mathrm{A}_{1} 8 \mathrm{I}+\mathrm{I}_{2} \mathrm{~A}_{2}$.\\
J. Operator Theory 7 (1982), 61-78.\\[0pt]
[1983] The spectrum of Hilbert space semigroups.\\
J. Operator Theory 10 (1983), 87-94.

Herbst, I.W.; Sloan, A.D.\\[0pt]
[1978] Pertyrbation of translation invariant positivity preserving semigroups on $\mathrm{L}^{2}\left(\mathbb{R}^{\mathrm{N}}\right)$.\\
Trans. Amer. Math. Soc. 236 (1978), 325-360.\\
Hess, H.; Kato, T.\\[0pt]
[1980] On some linear and nonlinear eigenvalue problems with an infinite weight function.\\
Comm. Partiell Differential Equations 5 (1980); 999-1030.\\
Hess, H.; Schrader, R.; Uhlenbrock, D.A.\\[0pt]
[1977] Domination of semigroups and generalization of Kato's inequality. Duke Math. J. 44 (1977), 893-904.

Hewitt, E.; Ross, K.A.\\[0pt]
[1963] Abstract Harmonic Analysis I. Berlin-Heidelberg-New York: Springer 1963.

Hiai, F.\\[0pt]
[1978] Weakly mixing properties of semigroups of linear operators. Kodai Math. J. 1 (1978), 376-393.

\section*{Hille, E.}
[1948] Functional Analysis and Semigroups. Amer. Math. Soc. Col1. Pub1. 31, Providence (R.I.) 1948.\\[0pt]
[1952] Une généralisation du problème de Cauchy. Ann. Inst. Fourier (Grenoble) 4 (1952), 31-48.

Hille, E.; Phillips, R.S.\\[0pt]
[1957] Functional Analysis and Semigroups. Amer. Math. Soc. Co11. Pub1. 31, Providence (R.I.) 1957.

Hirsch, M.W.; Smale, S.\\[0pt]
[1974] Differential Equations, Dynamical Systems and Linear Algebra. New York: Academic Press 1974.

Howland, J.S.\\[0pt]
[1984] On a theorem of Gearhart.\\
Integral Equations Operator Theory 7 (1984), 138-142.

\begin{verbatim}
D'Jacenko, S.V.
\end{verbatim}

[1976] Semigroups of almost negative type and their applications. Soviet Math. Dok1. 17 (1976), 1189-1193.

Jacobs, K.\\[0pt]
[1972] Gleichverteilung mod 1 . Selecta Math. IV, 57-93, Berlin-Heidelberg-New York: Springer 1972.

Jameson, G.J.O.\\[0pt]
[1974] Topology and Normed Spaces. London: Chapman / Hall 1974.

Jørgensen, P.T.\\[0pt]
[1980] Monotone convergence of operators semigroups and the dynamics of infinite particle systems. Aarhus Universitet, Preprint 1980.

Junghenn, H.D.\\[0pt]
[1971] Almost periodic compactifications and applications to one parameter semigroups. Doctoral Dissertation, The George Washington Universtity 1971.

Kadison, R.V.\\[0pt]
[1965] Transformations of states in operator theory and dynamics. Topology 3 (1965), 177-198.

Kallman, R.R.\\[0pt]
[1969] Unitary groups and automorphisms of operator algebras. Amer. J. Math. 91 (1969), 785-806.

Kamke, E.\\[0pt]
[1932] Zur Theorie der Systeme gewöhn1icher Differentialgleichungen II. Acta Math. 58 (1932), 57-85.

Kaper, H.G.; Lekkerkerker, C.G.; Hejtmanek, J.\\[0pt]
[1983a] Spectral Methods in Linear Transport Theory.\\
Basel: Birkhäuser 1982.\\[0pt]
[1983b] Recent progress on the reactor problem of linear transport theory. Argonne National Laboratory, Preprint 1983.

Karlin, s.\\[0pt]
[1959] Positive operators.\\
J. Math. Mech. 8 (1959), 907-937.

Kato, T.\\[0pt]
[1966] Perturbation Theory for Linear Operators 1966. $2^{\text {nd }}$ printing: Berlin-Heidelberg-New York: Springer 1976.\\[0pt]
[1973] Schrödinger operators with singular potentials. Israel J. Math. 13 (1973), 135-148.\\[0pt]
[1982] Superconvexity of the spectral radius, and convexity of the spectral bound and the type. Math. Z. 180 (1982), 265-273.\\[0pt]
[1986] L ${ }^{\mathrm{p}}$-Theory of Schrödinger operators with a singular potential. In: R. Nagel; U. Schlotterbeck; M.P.H. Wolff (eds.): Aspects of Positivity in Functional Analysis. Amsterdam: North Holland 1986.

Katznelson, Y.; Tzafriri, L.\\[0pt]
[1984] On power bounded operators. Hebrew University, Jerusalem, Preprint 1984.

Kerscher, W.\\[0pt]
[1986] Retardierte Cauchy Probleme: Ordnungseigenschaften und Stabilität unabhängig von der Verzögerung. Dissertation, Universität Tübingen 1986.

Kerscher, W.; Nagel, R.\\[0pt]
[1984] Asymptotic behavior of one-parameter semigroups of positive operators. Acta App1. Math. 2 (1984), 297-309.

Kipnis, C.\\[0pt]
[1974] Majoration des semi-groupes de contractions de $\mathrm{L}^{1}$ et applications. Ann. Inst. H. Poincaré (Sect. B) 10 (1974), 369-384.

Kishimoto, A.; Robinson, D.W.\\[0pt]
[1980] Positivity and monotonicity properties of $C_{0}$-semigroups, II. Comm. Math. Phys. 75 (1980), 85-101.\\[0pt]
[1981] Subordinate semigroups and order properties. J. Austral. Math. Soc. Ser. A 31 (1981), 59-76.

Klein, A.; Landau, L.J.\\[0pt]
[1975] Singular perturbations of positivity preserving semigroups via path space techniques.\\
J. Funct. Anal. 20 (1975), 44-82.

Klein, I.\\[0pt]
[1984] Zur Spektraltheorie positiver Halbgruppen auf geordneten Banachräumen. Dissertation, Universität Tübingen 1984.

Komatsu, H.\\[0pt]
[1969] Fractional powers of operators III, Negative powers. J. Math. Soc. Japan 21 (1969), 205-228.

\section*{Konishi, Y.}
[1971] Nonlinear semigroups in Banach lattices. Proc. Japan Acad. Ser. A Math. Sci. 47 (1971), 24-28.

Krasnosel'skif, M.A.\\[0pt]
[1964] Positive Solutions of Operator Equations. Groningen: Noordhoff 1964.

Krein, S.G.\\[0pt]
[1971] Linear Differential Equations in Banach Spaces. Amer. Math. Soc. Trans1. 29, Providence (R.I.) 1971.

Krein, S.G.; Khazan, M.I.\\[0pt]
[1985] Differential equations in a Banach space. J. Soviet Math. 30 (1985), 2154-2239.

Kreiss, H.O.\\[0pt]
[1958] tber sachgemässe Cauchyprobleme für Systeme von linearen partiellen Differentialgleichungen. TRITA-NA, Roy. Inst. Techno1., Stockholm 127 (1958).\\[0pt]
[1959] Uber Matrizen die beschränkte Halbgruppen erzeugen. Math. Scand. 7 (1959), 71-80.

Krengel, U.\\[0pt]
[1985] Ergodic Theorems.\\
Berlin, New York: de Gruyter 1985.\\
Kubokawa, Y.\\[0pt]
[1975] Ergodic theorems for contraction semigroups. J. Math. Soc. Japan 27 (1975), 184-193.

Kümerer, B.; Nagel, R.\\[0pt]
[1979] Mean ergodic semigroups on $\mathrm{W}^{*}$-algebras. Acta Sci. Math. 41 (1979), 151-159.

Kuhn, K.\\[0pt]
[1984] Elliptische Differentialoperatoren als Generatoren auf C( $\Omega$ ). Semesterbericht Funktionalanalysis, Tübingen, Wintersemester 1984/1985, 125-142.

Kundsch, K.; Schappacher, W.\\[0pt]
[1983] Necessary conditions for partial differential equations with delay to generate C-semigroups.\\
J. Differential Equation 50 (1983), 49-79.

Kunita, H.\\[0pt]
[1969] Sub-Markov semi-groups in Banach lattices. Proc. of the Conference on Funct. Anal. and Related Topics, 332-343. Tokyo Press 1969.

Kurose, H.\\[0pt]
[1981] An example of a non quasi well-behaved derivation on C(I). J. Funct. Anal. 43 (1981), 193-201.\\[0pt]
[1982] On a closed derivation in C(I). Mem. Fac. Sci. Kyushu Univ. Ser. A Math. 36 (1982), 193-198.\\[0pt]
[1983] Closed derivations in C(I). Tôhoku Math. J. 35 (1983), 341-347.

Lamperti, J.\\[0pt]
[1977] Stochastic Processes. Berlin-Heidelberg-New York: Springer 1977.\\
de Laubenfels, R.\\[0pt]
[1984] We11 behaved derivation on C[0,1]. Pacific J. Math. 115 (1984), 73-80.

Leader, S.\\[0pt]
[1954] On the infinitesimal generators of a semigoups of positive transformations with local character condition. Proc. Amer. Math. Soc. 5 (1954), 401-406.

Lin, M.\\[0pt]
[1974] On the uniform ergodic theorem II. Proc. Amer. Math. Soc. 46 (1974), 217-225.\\[0pt]
[1975] Quasicompactness and uniform ergodicity of Markov operators. Ann. Inst. H. Poincaré (Sect. B) 11 (1975), 345-354.

Lin, M.; Montgomery, J.; Sine R.\\[0pt]
[1977] Change of velocity and ergodicity in flows and in Markov semi-groups. Z. Wahrsch. Verw. Gebiete 39 (1977), 197-211.

Lindblad, G.\\[0pt]
[1976] On the generators of quantum dynamical semigroups. Comm. Math. Phys. 48 (1976), 119-130.

Lindenstrauss, J.; Tzafriri, L.\\[0pt]
[1979] Classical Banach Spaces II, Function Spaces. Ber1in-Heide1berg-New York: Springer 1979.

Lotz, H.P.\\[0pt]
[1981] Uniform ergodic theorems for Markov operators on $\mathrm{C}(\mathrm{X})$. Math. Z. 178 (1981), 145-156.\\[0pt]
[1982] Uniform convergence of operators on $L^{\infty}$. Semesterbericht Funktionalanalysis, Tubingen, Wintersemester 19882/1983,\\[0pt]
[1984] Tauberian theorems for operators on $\mathrm{L}^{\infty}$ and similar spaces. In: K.D. Bierstedt; B. Fuchssteiner (eds.) Functional Analysis, Surveys and Recent Results III. Amsterdam: North Holland 1984.\\[0pt]
[1985] Uniform convergence of operators on $\mathrm{L}^{\infty}$ and similar spaces. Math. Z. 190 (1985), 207-220.\\[0pt]
[1986] Positive linear operators on $L^{p}$ and the Doeblin condition. In: R. Nagel; U. Schlotterbeck; M.P.H. Wolff (eds.): Aspects of Positivity in Functional Analysis. Ansterdam: North Holland 1986.

Lumer, G.\\[0pt]
[1974a] Perturbations de générateurs infinitesimaux du type "changement de temps". Ann. Inst. Fourier (Grenoble) 23 (1974), 271-279.\\[0pt]
[1974b] Problème de Cauchy pour opérateurs locaux et "changement de temps". Ann. Inst. Fourier (Grenoble) 23 (1974), 409-466.

Lumer, G.; Phillips, R.S.\\[0pt]
[1961] Dissipative operators in a Banach space. Pacific J. Math. 11 (1961), 679-698.

Luxemburg, W.A.J.\\[0pt]
[1979] Some Aspects of the Theory of Riesz Spaces. The University of Arkansas Lecture Notes in Mathematics 4, Fayetteville 1979.

Majewski, A.; Robinson, D.W.\\[0pt]
[1983] Strictly positive and strongly positive semigroups.\\
J. Austral. Math. Soc. Ser. A 34 (1983), 36-48.

Miller, J.; Strang, G.\\[0pt]
[1966] Matrix theorems for partial differential and difference equations. Math. Scand. 18 (1966), 113-133.

Miller, R.K.\\[0pt]
[1974] Linear Volterra integro-differential equations as semigroups. Funkcial. Ekvac. 17 (1974), 39-55.

Mil'stein, G.N.\\[0pt]
[1975] Exponential stability of positive semigroups in a Inear topological space I, II.\\
Soviet. Math. 19 (1975), 35-42, 51-61.\\
Miyadera, I.\\[0pt]
[1952] Generation of a strongly continuous semigroup of operators. Tôhoku Math. J. 4 (1952), 109-114.

Miyajima, S.\\[0pt]
[1986] Generators of positive $\mathrm{C}_{\mathrm{o}}$-semigroups.\\
In: R. Nage1; U. Schlotterbeck; M.P.H. Wolff (eds.): Aspects of Positivity in Functional Analysis. Amsterdam: North Holland 1986.

Miyajima, S.; Okazawa, N.\\[0pt]
[1984] Generators of positive $\mathrm{C}_{\mathrm{o}}$-semigroups on Banach lattices. Preprint 1984.

Montgomery, D.; Zippin, L.\\[0pt]
[1955] Topological Transformation Groups. New York: Interscience Publishers 1955.

Moreau, J.J.\\[0pt]
[1966] Fonctione1les convexes. Séminaire sur les équations aux derivées partielles. Collège de France 1966-67.

Nage 1, R.\\[0pt]
[1973] Mittelergodische Halbgruppen linearer Operatoren. Ann. Inst. Fourier (Grenoble) 23 (1973), 75-87.\\[0pt]
[1983] Sobolev Spaces and Semigroups. Semesterbericht Funktionalanalysis, Tübingen, Sommersemester 1984, 1-19.\\[0pt]
[1984] What can positivity do for stability? In: K.D. Bierstedt; B. Fuchssteiner (eds.): Functional Analysis, Surveys and Recent Results III. Amsterdam: North Holland 1984.\\[0pt]
[1985] Well-posedness and positivity for systems of linear evolution equations. Conferenze del Seminario di Matematica del1'Università di Bari 203 (1985), 1-29.

Nage1, R.; Derndinger, R.; Palm, G.\\[0pt]
[1982] Ergodic Theory in the Functional Analytic Perspective. Tübingen 1982.

Nage 1, R.; Uhlig, H.\\[0pt]
[1981] An abstract Kato inequality for generators of positive semigroups on Banach lattices.\\
J. Operator Theory 6 (1981), 113-123.

Nakano, H.\\[0pt]
[1950] Modern Spectral Theory. Tokyo Mathematical Book Series. Vol. II. Maruzen Co.: Tokyo 1950.

Neubrander, F.\\[0pt]
[1984a] Well-posedness of abstract Cauchy problems. Semigroup Forum 29 (1984), 75-85.\\[0pt]
[1984b] Well-posedness of higher order abstract Cauchy problems. Dissertation, Universität Tübingen 1984.\\[0pt]
[1985a] Laplace transform and asymptotic behavior of strongly continuous semigroups. To appear in: Houston J. Math.\\[0pt]
[1985b] Asymptotic behavior of solutions of inhomogeneous abstract Cauchy problems. In: Proc. Conf. Physical Math. and Nonlinear Part. Differential Equations, Morgantown 1983, 157-73. Marce1 Dekker 1985.\\[0pt]
[1986] Well-posedness of higher order abstract Cauchy problems. To appear in: Trans. Amer. Math. Soc.

Nussbaum, R.D.\\[0pt]
[1984] Positive operators and elliptic eigenvalue problems. Math. Z. 186 (1984), 247-264.

Okazawa, N.\\[0pt]
[1984] An L ${ }^{\text {P }}$-theory for Schrödinger operators with nonnegative potentials. J. Math. Soc. Japan 36 (1984), 675-688.

Olesen, D.; Pedersen, G.K. Takesaki, M.\\[0pt]
[1980] Ergodic actions of compact Abelian groups. J. Operator Theory 3 (1980), 237-269.

Oseledets, V.I.\\[0pt]
[1984] Completely positive linear maps, non Hamiltonian evolution and quantum stochastic processes.\\
J. Soviet Math. 25 (1984), 1529-1557.\\
de Pagter, B.\\[0pt]
[1984] A note on disjointness preserving operators. Proc. Amer. Math. Soc. 90 (1984), 543-550.\\[0pt]
[1986] Irreducible compact operators. To appear in: Math. Z.

Pazy, A.\\[0pt]
[1968] On the differentiability and compactness of linear operators, J. Math. Mech. 17 (1968), 1131-1142.\\[0pt]
[1983] Semigroups of Linear Operators and Applications to Partial Differential Equations. Berlin- Heidelberg-New York-Tokyo: Springer 1983

Pedersen, G.K.\\[0pt]
[1979] C*-Algebras and their Automorphism Groups. London, New York, San Francisco: Academic Press 1979.

Peetre, J.\\[0pt]
[1959] Une charactérisation abstraite des opérateurs differentiels. Math. Scand. 7 (1959), 211-218; et: Rectification à l'article précédent. Math. Scand. $\overline{8}$ (1960), 116-120.

Perron, 0.\\[0pt]
[1907] Zur Theorie der Matrices.\\
Math. Ann. 64 (1907), 248-263.\\
Phillips, R.s.\\[0pt]
[1954] A note on the abstract Cauchy problem.\\
Proc. Nat. Acad. Sci. U.S.A. 40 (1954), 244-248.\\[0pt]
[1962] Semigroups of positive contraction operators.\\
Czechoslovak Math. J. 12 (1962), 294-313.\\[0pt]
[1974] Perturbation theory for semi-groups of linear operators.\\
Trans. Amer. Math. Soc. 74 (1974), 343-369.\\
Picard, C.\\[0pt]
[1972] Opérateurs $\phi_{\text {-accretifs et génération des semi-groupes non linéaires. }}$ C. R. Acad. Sci. Paris Sér. I-Math. 275 (1972), 639-641.

Pietsch, A.\\[0pt]
[1978] Operator Ideals. Berlin: VEB Deutscher Verlag der Wissenschaften 1978.

Plant, A.T.\\[0pt]
[1977] Nonlinear semigroups of linear operators and applications in Banach spaces. J. Math. Anal. Appl. 60 (1977), 67-74.

Protter, M.H.; Weinberger, H.F.\\[0pt]
[1967] Maximum Principles in Differential Equations. New York-Berlin-Heidelberg: Springer 1984.

Prü, J.\\[0pt]
[1981] Equilibrium solutions of age-specific population dynamics of several species.\\
J. Math. Bio1. 11 (1981), 65-84.\\[0pt]
[1984] On the spectrum of $\mathrm{C}_{0}$-semigroups.\\
Trans. Amer. Math. Soc. 284 (1984), 847-857.

Rao, A.S.; Hengartner, W.\\[0pt]
[1974] On the existence of a unique almost periodic solution of an abstract differential equation.\\
J. London Math. Soc. 8 (1974), 577-581.

Reed, M.; Simon, B.\\[0pt]
[1975] Methods of Modern Mathematical Physics II. Fourier Analysis, Self-Adjointness.\\
New York: Academic Press 1975.\\[0pt]
[1978] Methods of Modern Mathematical Physics IV, Analysis of Operators. New York: Academic Press 1978.\\[0pt]
[1979] Methods of Modern Mathematical Physics III. Scattering Theory. New York: Academic Press 1979.

Relch, S.\\[0pt]
[1981] A characterization of nonlinear $\phi$-accretive operators. Manuscripta Math. 36 (1981), 163-178.

Robinson, D.W.\\[0pt]
[1977] The approximation of flows. J. Funct. Ana1. 24 (1977), 280-290.\\[0pt]
[1982] Strongly positive semigroups and faithful invariant states. Comm. Math. Phys. 85 (1982), 129-142.\\[0pt]
[1983] Continuous semigroups on ordered Banach spaces. J. Funct. Anal. 51 (1983), 268-284.\\[0pt]
[1984] On positive semigroups. Publ. Res. Inst. Math. Sci. 20 (1984), 213-224.\\[0pt]
[1985] Differential operators on C*-algebras. Preprint, Canberra 1985.

Robinson, D.W.; Yamamuro, S.\\[0pt]
[1983] The canonical half-norm, dual half-norms, and monotonic norms. Tôhuku Math. J. 35 (1983), 375-386.\\[0pt]
[1984] Hereditary cones, order ideals and half-norms. Pacific J. Math. 110 (1984), 335-343.

Roth, J.P.\\[0pt]
[1976] Opérateurs dissipatifs et semigroupes dans les espaces de fonctions continues.\\
Ann. Inst. Fourfer (Grenoble) 26 (1976), 1-97.\\[0pt]
[1978] Les opérateurs elliptiques comme générateurs infinitésimaux de semigroups de Feller.\\
In: F.Hirsch; G.Mokobodzki (eds.): Séminaire de Théorie du Potential, Paris No. 3 . Lecture Notes in Math. 681, 234-251. Berlin-Heidelberg-New York: Springer 1978.

Sacker,R.J.; Sell, G.R.\\[0pt]
[1978] A spectral theory for linear differential systems.\\
J. Differential Equations 27 (1978), 320-358.

Sakai, S.\\[0pt]
[1971] C*-Algebras and W*-Algebras. Berlin-Heiderberg-New York: Springer 1971.

Sato, K.I.\\[0pt]
[1968] On the generators of non-negative contraction semigroups in Banach lattices. J. Math. Soc. Japan 20 (1968), 423-436.\\[0pt]
[1970a] On dispersive operators in Banach lattices. Pactfic J. Math. 33 (1970), 429-443.\\[0pt]
[1970b] Positive pseudo-resolvents in Banach lattices.\\
J. Fac. Sci. Univ. Tokyo Sect. I A Math. 17 (1970), 305-313.

Sawashima, I.\\[0pt]
[1964] On spectral properties of some positive operators.\\
Natur. Sci. Rep. Ochanomizu Univ. 15 (1964), 53-64.\\
Scarpellini, B.\\[0pt]
[1974] On the spectra of certain semigroups. Math. Ann. 211 (1974), 323-336.

Schaefer, H.H.\\[0pt]
[1966] Topological Vector Spaces 1966.\\
$4^{\text {th }}$ printing: New York-Heide1berg-Berlin: Springer 1980.\\[0pt]
[1968] Invariant ideals of positive operators in $C(X)$, II. Illinois J. Math. 12 (1968), 525-538.\\[0pt]
[1974] Banach Lattices and Positive Operators. New-York Heidelberg-Berlin: Springer 1974.\\[0pt]
[1980] Ordnungsstrukturen In der Operatorentheorie. Jahresber. Deutsch. Math.-Verein. 82 (1980), 33-50.\\[0pt]
[1982] Some recent results on positive groups and semi-groups. In: C.R. Huijsmans; M.A. Kaashoek; W.A.J. Luxemburg; W.K. Vietsch (eds.): From A to Z, Proc. Symp. in Honour of A.C. Zaanen, Leiden 1982. Mathematical Centre Tracts 149, 69-79, Amsterdam: 1982.\\[0pt]
[1985] Existence of spectral values for irreducible $\mathrm{C}_{0}$-semigroups. Tubingen, Preprint 1985.

Schaefer, H.H.; Wolff, M.P.H.; Arendt, W.\\[0pt]
[1978] On lattice isomorphisms with positive real spectrum and groups of positive operators.\\
Math, Z. 164 (1978), 115-123.\\
Schep, A.R.\\[0pt]
[1985] Weak Kato-inequalities and postrive semigroups. Math. Z. 190 (1985), 305-314.

Seever, G.L.\\[0pt]
[1973] Measures on F-spaces.\\
Trans. Amer. Math. Soc. 133 (1973), 267-280.\\
Semadeni, Z.\\[0pt]
[1971] Banach Spaces of Continuous Functions. Warszawa: Polish Scientific Publishers 1971.

\section*{Simon, B.}
[1973] Ergodic semigroups and positivity preserving self-adjoint operators. J. Funct. Anal. 12 (1973), 335-339.\\[0pt]
[1977] An abstract Kato's inequality for generators of positivfty preserving semigroups. Indiana Univ. Math. J. 26 (1977), 1067-1073.\\[0pt]
[1979] Kato's inequality and the comparison of semigroups. J. Funct. Anal. 32 (1979), 97-101.\\[0pt]
[1982] Schrödinger semigroups. Bull. Amer. Math. Soc. 7 (1982), 447-526.

\section*{Slemrod, M.}
[1976] Asymptotic behavior of $C_{0}$-semigroups as determined by the spectrum of the generator.\\
Indiana Univ. Math, J. 25 (1976), 783-892.\\
Stern, R.J.\\[0pt]
[1982] A note on positively invariant cones.\\
App1. Math. Optim. 9 (1982), 67-72.\\
Stewart, H.B.\\[0pt]
[1974] Generation of analytic semigroups by strongly elliptic operators. Trans. Amer. Math. Soc. 199 (1974), 141-162.

Stormer, E.\\[0pt]
[1972] On projection maps on von Neumann algebras. Math. Scand. $\underline{50}$ (1972), 42-50.

Takesaki, M.\\[0pt]
[1979] Theory of Operator Algebras I. New York-Heidelberg-Berlin: Springer 1979.

Travis, C.; Webb, G.F.\\[0pt]
[1974] Existence and stability for partial functional differential equations. Trans. Amer. Math. Soc. 200 (1974), 395-418.

Triggiani, R.\\[0pt]
[1975a] Pathological asymptotic behavior of systems in Banach spaces.\\
J. Math. Anal. Appl. 49 (1975), 411-429.\\[0pt]
[1975b] On the stabilizability problem in Banach spaces. J. Math. Anal. App1. 52 (1975), 383-403.

Trotter, H.F.\\[0pt]
[1974] Approximation and perturbation of semigroups. In: Butzer, Sz.-Nagy: Lineare Operatoren und Approximation II, Proceedings on a Conference in Oberwolfach 1974. Birkhäuser 1974.

Uhlig, H.\\[0pt]
[1979] Derivationen und Verbandshalbgruppen. Dissertation, Tübingen 1979.

Vidav, I.\\[0pt]
[1970] Spectra of perturbed semigroups with applications to transport theory. J. Math. Anal. App1. 30 (1970), 264-279.

V1llella-Bressan R.\\[0pt]
[1985] Functional equation of delay type in $\mathrm{L}^{1}$-spaces. Ann. Polon. Math. 45 (1985), 93-104.

Voigt, J.\\[0pt]
[1980] A perturbation theorem for the essential spectral radius of strongly continuous semigroups.\\
Monatsh. Math. 90 (1980), 153-161.\\[0pt]
[1982] On the abszissa of convergence for the Laplace transform of vector valued measures.\\
Arch. Math. (Base1) 39 (1982), 455-462.\\[0pt]
[1984a] Positivity in time dependent linear transport theory. Acta App1. Math. 2 (1984), 311-331.\\[0pt]
[1984b] Spectral properties of the neutron transport equations. To appear in: J. Math. Anal. Appl.\\[0pt]
[1984c] Absorption semigroups, their generators and Schrödinger semigroups. Preprint 1984.\\[0pt]
[1984d] On substochastic $\mathrm{C}_{\text {- }}$-semigroups and their generators. Semesterbericht Funktionalanalysis, Tübingen, Wintersemester 1984/1985, 71-85.\\[0pt]
[1985] Interpolation for positive $C_{0}$-semigroups on $L^{p}$-spaces.\\
Math. 2. 188 (1985), 283-286. ${ }^{\circ}$

Watanabe, S .\\[0pt]
[1982] Asymptotic behaviour and eigenvalues of dynamical semigroups on operator algebras.\\
J. Math. Anal. Appl. 86 (1982), 411-424.

Webb, G.F.\\[0pt]
[1977] Voltera integral equations and nonlinear semigroups. Nonlinear Analysis $\frac{1}{1}$ (1977), 415-427.\\[0pt]
[1984] A semigroup approach to the Sharpe-Lotka theorem.\\
In: F. Kappel; W. Schappacher (eds.): Infinite-dimensional Systems, Retzhof 1983. Lecture Notes in Math. 1076, 254-268. Berlin-HeidelbergNew York: Springer 1984.\\[0pt]
[1985a] Theory of NonIinear Age-Dependent Population Dynamics. New York: Marcel Dekker 1985.\\[0pt]
[1985b] An operator-theoretic formulation of asynchronous exponential growth. Preprint 1985.

Widder, D. V.\\[0pt]
[1946] The Laplace Transform.\\
Princeton (N.J.): Princeton University Press 1946.\\[0pt]
[1971] An Introduction to Transform Theory. New York: Academic Press 1971.\\[0pt]
[1971] An Introduction to Transform Theory. New York: Academic Press 1971.

Wielandt H.\\[0pt]
[1950] Unzerlegbare, nicht-negative Matrizen. Math. Z. 52 (1950), 642-648.

Winkler, W.\\[0pt]
[1973] A note on continuous one-parameter zero-two law. Ann. Prob. 1 (1973), 341-344.

Wolff, M.P.H.\\[0pt]
[1978] On C-semigroups of lattice homomorphisms on a Banach lattice. Math. Z. 164 (1978), 69-80.\\[0pt]
[1981] A remark on the spectral bound of the generator of a semigroup of positive operators with application to stability theory.\\
In: P.L. Butzer, B. Sz.-Nagy, E. Görlich (eds.): Functional Analysis and Approximation. Proc. Conf. Oberwolfach 1980, 39-50. Basel-Boston-Stuttgart: Birkhäuser 1981.

Yamamuro, S.\\[0pt]
[1984] A note on positive semigroups. Preprint. Canberra 1984.\\[0pt]
[1985] Absolute values in orthogonally decomposable spaces. Bull. Australian Math. Soc. 31 (1985), 215-233.

Yosida, K.\\[0pt]
[1948] On the differentiability and representation of one-parameter semi-groups of linear operators.\\
J. Math. Soc. Japan 1 (1948), 15-21.\\[0pt]
[1965] Fupctional Analysis 1965.\\
$6^{\text {th }}$ printing: Berlin-Heidelberg-New York: Springer 1980.

Yosida, K.; Kakutant, S.\\[0pt]
[1941] Operator-theoretical treatment of Markoff's process and mean ergodic theorem.\\
Ann. of Math. 42 (1941), 188-228.

Zaanen, A.C.\\[0pt]
[1983] Riesz Spaces II.\\
Groningen: North Holland 1983.

Zabczyk, J.\\[0pt]
[1975] A note on C-semigroups. Bull. Acad. Polon. Sci. 23 (1975), 895-898.\\[0pt]
[1979] Stabilization of boundary control systems.\\
Int. Symp. Systems Opt. Anal. 1978. Lecture Notes Control Theory Inform. Berlin-HeideIberg-New York: Springer 1979.

Zaidman, S.D.\\[0pt]
[1979] Abstract Differential Equations.\\
London: Pitman 1979.

\begin{center}
\begin{tabular}{|c|c|}
\hline
$\mathrm{E}_{\mathrm{R}}, \mathrm{E}_{\mathrm{C}}=\mathrm{E}$ & real , complex Banach lattice \\
\hline
$\mathrm{E}_{+}$ & positive cone \\
\hline
E* & dual \\
\hline
E* & semigroup dual \\
\hline
$\mathrm{E}_{\mathrm{F}}^{\text {T }}$ & F-product of $E$ with respect to the semigroup $T$ \\
\hline
$\mathrm{E}_{\mathrm{F}}$ & F-product of E \\
\hline
$\mathrm{E}_{\mathrm{f}}$ & see $C-I, 4$ \\
\hline
$(\mathrm{E}, \phi$ ) & see C-I,4 \\
\hline
E0F & tensor product \\
\hline
L(E) & bounded linear operators on E \\
\hline
Z(E) & center of E \\
\hline
$\mathrm{E}_{\mathrm{n}}$ & n-th Sobolev space \\
\hline
B (H) & W*-algebra of all bounded linear operators on H \\
\hline
$S(\mathrm{M})$ & state space of a $\mathrm{C}^{*}$-algebra M \\
\hline
$\mathrm{M}_{+}$ & positive cone of the C*-algebra M \\
\hline
M* & predual \\
\hline
$\mathrm{M}^{\mathrm{sa}}$ & se1f-adjoint part \\
\hline
$\mathrm{M}_{\mathrm{n}}$ & $C^{*}$-algebra of all n*n-matrices \\
\hline
AC & absolutely continuous functions \\
\hline
BV & functions of bounded variation \\
\hline
K & compact topological space \\
\hline
X & locally compact topological space \\
\hline
$\mathrm{C}(\mathrm{K}), \mathrm{C}(\mathrm{K}, \mathrm{E})$ & continuous functions (with values in E) \\
\hline
\( \begin{aligned} & C_{0}(X), C_{0}(X, E) \\ & C^{B}(X) \end{aligned} \) & continuous functions vanishing in infinity with values in E bounded continuous functions \\
\hline
$C_{\text {bu }}$ (x) & uniformly continuous functions \\
\hline
$c^{1}, c^{(n)}$ & continuous differentiable functions (n-times) \\
\hline
$C_{c}^{\infty}\left(\mathbb{R}^{\mathrm{n}}\right)$ & infinttely differentiable functions with compact support \\
\hline
$\mathrm{L}^{\mathrm{p}}(\mu)$ & p-integrable functions \\
\hline
$S\left(\mathbb{R}^{\mathrm{n}}\right)$ & Schwartz space \\
\hline
M(K) & regular Borel measures \\
\hline
$\mathrm{M}_{\mathrm{b}}(\mathrm{X})$ & bounded regular Borel measures \\
\hline
$T=(T(t))_{t \geq 0}$ & (one-parameter) semigroup \\
\hline
T & subspace (reduced) semigroup \\
\hline
T, & quotient semigroup \\
\hline
Fix(T) & fixed space of $T$ \\
\hline
\end{tabular}
\end{center}

\begin{center}
\begin{tabular}{|c|c|}
\hline
A & generator \\
\hline
$A^{\prime}$ & adjoint \\
\hline
$A^{*}$ & adjoint generator \\
\hline
$\sigma$ (A) & spectrum \\
\hline
$\rho$ (A) & resolvent set \\
\hline
$\sigma_{\text {ess }}$ (A) & essential spectrum \\
\hline
$\sigma_{b}$ (A) & boundary spectrum \\
\hline
$P \sigma$ (A) & point spectrum \\
\hline
$P_{\sigma_{b}}$ (A) & boundary point spectrum \\
\hline
A $\sigma$ (A) & approximate point spectrum \\
\hline
$\mathrm{R}_{\sigma}(\mathrm{A})$ & residual spectrum \\
\hline
$\omega=\omega(A)=\omega(T)=\omega(T(t))$ & growth bound \\
\hline
$\mathrm{s}(\mathrm{A}$ ) & spectral bound \\
\hline
$\omega_{1}$ (A) & growth bound of the solution of the (ACP) \\
\hline
$\omega$ (f) & growth bound of T(.)f \\
\hline
(T) & spectral radius \\
\hline
$\omega_{\text {ess }}$ (A) & essential growth bound \\
\hline
$\mathrm{r}_{\text {ess }}$ (T) & essential spectral radius \\
\hline
$\mathrm{R}(\lambda, \mathrm{A}$ ) & resolvent operator \\
\hline
$I^{d},\left\{I^{d}\right\}^{d}=I^{d d}$ & orthogonal band of I (of I ${ }^{\text {d }}$ ) \\
\hline
- & infimum \\
\hline
$\checkmark$ & supremum \\
\hline
$|\mathrm{T}|$ & modulus of a regular operator \\
\hline
至,考 & Fourier (inverse Fourier) transformation \\
\hline
dp (f) & subdifferential of p in f \\
\hline
dN(f) & subdifferential of the norm in f \\
\hline
$\mathrm{dN}^{+}$(f) & subdifferential of the canonical half-norm in f \\
\hline
im & range \\
\hline
ker & null-space \\
\hline
Im & imaginary part \\
\hline
Re & real part \\
\hline
Ref , Imf & see C-I, 7 \\
\hline
ReT , ImT & see C-1,7 \\
\hline
$\bar{f}$ & complex confugate of f \\
\hline
$S_{f}$ & signum operator with respect to f \\
\hline
sign f & signum of f \\
\hline
signn f & see C-II,2.2 \\
\hline
$f^{[n]}$ & B-III,2.2 ; C-III,2.1 \\
\hline
|f| & absolute value of f \\
\hline
$f^{+}$ & positive part of f \\
\hline
$f^{-}$ & negative part of $f$ \\
\hline
\end{tabular}
\end{center}

\begin{center}
\begin{tabular}{ll}
Id & identity operator \\
$M_{p}$ & multiplication operator \\
${ }^{1} \mathrm{C}$ & function identically 1 \\
$\delta_{\mathrm{x}}$ & characteristic function of the set C \\
tr & Dirac measure in x \\
span M & trace \\
$\mathrm{S}^{\prime}(\alpha)$ & linear subspace generated by M \\
(ACP) & sector in the complex plane \\
(P) & abstract Cauchy problem \\
(P') & positive minimum principle \\
(K) & B-II,1.21 \\
(RCP) & Kato's (equality) inequality \\
(RE) & retarded Cauchy problem \\
(T) & retarded equation \\
translation property &  \\
\end{tabular}
\end{center}

\section*{SUBJECT INDEX}
\begin{verbatim}
Abelian group 390f
    locally compact -- 3905
    solenoidal -- 391
abscissa
    - of absolute convergence 103f
    - of simple convergence 103
    - of holomorphy 101
absolute value 235, 239
abstract Cauchy problem
            4, 26ff, 98ff, 336
adjoint 16f, 400
    - generator 17f
    - operator 16,64f, 77, 141
    - semigroup 16ff
admissible function 154ff
algebra homomorphism 143ff
algebraic multiplicity 73
AL-space 239
AM-space 239
approximation theorems
    32f, 44, 81, 116
asymptotics
    98ff, 204ff, 342ff, 352, 406ff
automorphism group 146ff
\end{verbatim}

Banach lattice 235\\
complex -- $243,260,288$\\
real -- 243\\
band 236\\
- projection 237\\
boundary spectrum 168ff,\\
$296 \mathrm{ff}, 302 \mathrm{ff}, 305,379 \mathrm{ff}, 387$\\
C*-algebra 117, 369\\
Calkin algebra 73\\
Cauchy problem 4, 26ff\\
abstract -- 4, 26ff, 98ff, 336\\
autonomous -- 4, 26ff\\
homogeneous -- 4, 98ff\\
inhomogeneous -- 112ff, 340ff\\
retarded -- $219 \mathrm{ff}, 356 \mathrm{ff}$\\
well-posed -- 26ff\\
center 246, 272, 279f, 288\\
Césaro\\
- mean 346, 406, 408\\
- summable 93f\\
Chapman-Kolmogorov equation 273f\\
characteristic equation $180,229,362$\\
generalized -- 226,362\\
characterization

\begin{itemize}
  \item of generators
\end{itemize}

122ff, 247ff, 260ff, 376ff\\
chain rule 136\\
closable 5f, 52, 128\\
closure 5f, 52, 128\\
cocycle 148ff\\
cone\\
positive - 51, 234, 369\\
conditional expectation\\
normal -- 411f, 416\\
core $5 \mathrm{f}, 46 \mathrm{f}$\\
cyclic 169, 172ff, 192ff, 302ff, 305, 379ff, 388ff\\
imaginary additively 172ff, 192ff, 302ff

Datko's theorem 108 f\\
decomposition 68ff, 325ff, 351ff\\
delay

\begin{itemize}
  \item differential equation 219ff
  \item equation 356ff\\
derivation l43ff\\
derivative\\
first order - $9 \mathrm{ff}, 146$, 184f, 220, 265, 276, 308f, 357\\
higher order - 267 f\\
second order - 11f, 34f, 179, $185,249 \mathrm{f}, 308 \mathrm{f}$\\
differential equation\\
homogeneous -- 4, 98f\\
inhomogeneous --... 112£f, 340£f\\
ordinary -- 152f, 197f, 219ff\\
partial -- 26\\
retarded -- $134 \mathrm{f}, 142$, 179f, 219ff\\
system of -- 365\\
differential operator $9 \mathrm{ff}, 11 \mathrm{ff}$, $34 \mathrm{f}, 146,179,185,220,259 \mathrm{f}$, $265,267 \mathrm{f}, 276,308 \mathrm{f}, 357$\\
disjointness preserving
  \item operator 281\\
-- semigroup 281ff\\
dispersive 249ff\\
strictly - 249 ff\\
dissipative 47ff\\
p- 48ff, 128ff\\
strictly - 48ff\\
Doeblin's condition 218, 345\\
domain 3, 9, 46f\\
Fredholm - $73 f$\\
domain of uniqueness 46 f\\
dominant spectral value\\
177ff, 304, 318ff\\
strictly ---\\
$177 \mathrm{ff}, 210,217,318 \mathrm{f} \pounds$\\
domination $269 \mathrm{ff}, 371$\\
dual 16\\
semigroup - 16 f\\
Dunford-Pettis property 56
\end{itemize}

\begin{verbatim}
eigenspace 64,86
eigenvalue 64,387
    approximate - 64, 314
    simple - 73, 305, 310, 388
    normalized - 389
eigenvector 64, 387
    approximate - 64,314
elliptic differential operator
                $185,190 \mathrm{f}, 260,305,312$
equation
    differential - 4
    heat - 13
    population - $229,344 \mathrm{f}, 354 \mathrm{f}, 364 \mathrm{ff}$
    retarded - 356 ff
    transport - 309f, 320
example
    counter -
        $3,61 \mathrm{ff}, 105,131,265 \mathrm{ff}, 311$
    standard - $7 \mathrm{ff}, 9,10,11,12$,
        $42 \mathrm{ff}, 100 \mathrm{f}, 124,280,416$
exponential estimate $2 f$
\end{verbatim}

F-product $20 \mathrm{f}, 298 \mathrm{ff}, 314 \mathrm{ff}$\\
F-product with respect to a semigroup $20 \mathrm{f}, 74 \mathrm{ff}, 192$\\
face 388 invariant - 388,410\\
faithful subset 380\\
Féjer's theorem 93f\\
Feller property strong -- 213\\
fixed space $343 \mathrm{ff}, 374 \mathrm{ff}, 380 \mathrm{ff}, 414$\\
flow 143 ff\\
continuous - $148,192 \mathrm{ff}, 330$\\
semi - 143ff, 328ff\\
seperately continuous - 149 E\\
forcing term $112 \mathrm{ff}, 340 \mathrm{ff}$ periodic -- 116\\
p-periodic -- 113ff\\
Fourier transformation 12f, 91, 252\\
inverse -- 13, 91\\
coefficient 80\\
Fredholm

\begin{itemize}
  \item domain 73f
  \item operator 73f
\end{itemize}

Gateaux-derivative $50,136,257,283$\\
generalized solution 99,112\\
generator 3ff\\
adjoint - 16\\
bounded - 2, 7, 54ff, 129, 247, 255, 288, 376ff\\
weak* - 16\\
geometric multiplicity 73\\
graph 5\\
graph norm 5\\
Grothendieck space 55 ff\\
group $1,6,9,34,66,146 \mathrm{ff}$, $326 \mathrm{f}, 352 \mathrm{ff}, 390 \mathrm{f}$\\
automorphism - 146ff\\
lattice homomorphism 202\\
one-parameter - 1, 6, 31\\
positive - $146,148 \mathrm{ff}, 295,326 \mathrm{f}$\\
rotation - $10,69,352 \mathrm{ff}$\\
unitary - 13\\
growth bound 2, 6, 60ff\\
-- of a semfgroup 2, 6, 60ff, 74, $99 \mathrm{ff}, 130,168,204 \mathrm{ff}, 295$, 334ff, 343, 400ff\\
-- of mild solutions of a Cauchy problem 99ff\\
-- of solutions of a Cauchy problem 99ff, 204ff, 336ff\\
essential -- 74, 343\\
half-norm 51ff, 127ff\\
canonical - 51ff, 127ff, 255ff\\
strict - 51ff, 127ff\\
heat equation 13\\
Hilbert space $13,62,94 \mathrm{ff}, 105,403$\\
Hille-Yosida theorem 32\\
ideal 236\\
algebraic - 118\\
closed - 118, 236\\
invariant - 182ff, 303, 306ff, 317\\
lattice - 236\\
fmaginary additively cyclic subset 172ff, 192ff, 297\\
inhomogeneous differential equation 112ff, 340ff\\
integral equation 363\\
interpolation $335,348,352$\\
invariant

\begin{itemize}
  \item ideals 182ff, 303, 306ff, 317
  \item subset 24, 346\\
irreducible 138ff, 256ff, 414
  \item semigroup $130,182 \mathrm{ff}, 210,306 \mathrm{ff}$, 311ff, 315ff, 409ff\\
W*-- 388
\end{itemize}

Kakutani-Krein theorem 240, 297, 313, 334\\
Kato's

\begin{itemize}
  \item equality 138ff, 285ff, 325f
  \item inequality 139, 256ff, 258ff, 285\\
classical - 139f, 258f\\
distributional - 259f\\
Krein-Rutman theorem $130,167,334$
\end{itemize}

Laplace transform 101, 107\\
Laplacian 13, 34f, 100, 110, 139, $168,185,205,250 \mathrm{f}, 258,338$\\
lattice

\begin{itemize}
  \item homomorphism $120,243,244,281$
  \item norm 235\\
locality 146f, 268f, 282, 287\\
long term behavior $98 \mathrm{ff}, 204 \mathrm{ff}$, 342ff, 352, 406ff\\
Lumer-Phillips theorem 53f
\end{itemize}

\section*{Markov}
\begin{itemize}
  \item algebra homomorphism 143ff
  \item lattice homomorphism 120, 192ff, 200f
  \item operator 120, 191
  \item process 213 f
  \item semigroup 144, 191
  \item transition function 213f, 347f\\
matrix semigroup 7\\
maximum principle 185, 190\\
mild solution 99,112\\
modulus $136,257,278 \mathrm{ff}, 281$\\
multiplication
  \item operator 7f, 89f, 246
  \item semigroup 7f, $42 \mathrm{ff}, 65 \mathrm{f}, 287 \mathrm{ff}$\\
multiplicity 73ff\\
algebraic - 73f, 209, 310
  \item as a pole 73\\
geometric - 73, 310\\
negative part 235\\
norm\\
graph - 5\\
Sobolev - 18ff\\
normal linear functional 369\\
operator\\
closable - 5 f\\
closed - 5 f\\
contractive - 47ff\\
densely defined - 4\\
differential - $9 f f$\\
dissipative - 47ff\\
dispersive 249ff\\
elliptic - 185, 190f, 260, 305, 312\\
kernel - 184, 189f, 308ff, 320, 349f, 363, 367\\
lattice - 120, 242\\
local - 146f, 268f, 282, 287\\
Laplace - 13, 34f, 100, 110, 139, $168,185,205,250 \mathrm{f}, 258,338$\\
multiplication - 7f, 89f, 246, 287\\
positive - 120ff\\
p-contractive - 48ff\\
p-dissipative - 48ff, 128ff\\
resolvent positive - 127ff\\
Schrödinger - $179,273,278 \mathrm{f}, 336$\\
strictly dissipative - 48 ff\\
strictly p-dissipative - 48ff\\
strictly dispersive 249ff\\
weakly compact - 181, 211f\\
operator semigroup 1ff, 406\\
weakly compact -- 406 f\\
order bounded 238\\
order
  \item complete 234
  \item continuous norm 241
  \item interval 235\\
order continulty $239,286,287 \pounds f$\\
order unit 238\\
weak -- 238\\
ordered
  \item Banach space 234, 295
  \item vector space 234\\
periodic
  \item semigroup $10,79 \mathrm{ff}, 85,313,416$\\
p-periodic 113ff\\
Perron-Frobenius theory 163ff, 172ff, 292ff, 296ff, 379ff\\
perturbation\\
additive - 43ff\\
bounded - 44ff, 307\\
compact - 215f, 319\\
multiplicative - 131ff, 141\\
perturbation by multiplication operators 179, 183, 188, 274ff, 279, 307\\
perturbation theorems 43ff\\
Phillip's theorem 249\\
polar decomposition 380, 392\\
pole 67f, 72ff, 76, 209ff, 305, 315ff\\
algebraically simple - 73f,\\
181, 185, 209ff, 216, 315ff
  \item of order k 73f, 86, 174f, 295, 303ff\\
simple - 73f, 209ff, 310, 315ff\\
first order - 73f, 180\\
\includegraphics[max width=\textwidth, center]{2024_12_23_c6487cc0859199a15bd9g-467}
\end{itemize}

\begin{verbatim}
    tensor product - 21ff, 88f
    translation - 9f, 11, 15, 18, 41,
        66ff, 205
    uniformly continuous - 2, 7, 54ff,
        129, 247, 255, 288, 376ff
    uniformly ergodic - 391ff, 416,
        419, 424f
    weakly continuous - 2
    weak* continuous - 16, 370ff, 403
    weak*-irreducible - 388, 414, 424f
semigroup dual 16f, 77
signum 137ff, 256ff, 276, 296f
    - operator 170ff, 245, 256ff, 296
singularity
    isolated - 72ff
Sobolev space 18ff
    classical -- 19
solid subset 236
solution of a Cauchy problem 4, 27ff
    generalized - 99, 112ff
    mild - 99, 112ff
    p-periodic - 113ff
    strong - 27ff, 99, 112ff,
        219ff, 356ff
spectral
    - decomposition 68ff, 325ff, 351£
    - projection 69f,79
    - theorem 60ff, 82ff
spectral bound 60ff, 101f, 105ff,
        130, 163, 168, 204ff, 225, 292ff,
        316, 334ff, 361, 379, 400ff
    essential -- 73f, 214ff, 318
spectral inclusion theorem 84f
spectral mapping theorem
        60ff, 67, 82ff, 106
    weak --- 65f, 83f, 89ff
    --- for the resolvent 67f
spectral radius 60
    essential -- 73f, 177, 214ff, 318
spectral value
    dominant -- 177ff
    strictly dominant -- 177ff, 210, 217
spectrum 60ff
    approximate point - 64f, 394
    boundary - 169ff, 296ff, 302ff,
        305, 379ff, 387
    cyclic - 169, 172ff, 302ff,
        305, 379ff, 388ff
    essential - 73f
    point - 64f, 394
    residual - 64f
stability
        98ff, 227, 337ff, 361, 402ff
    exponential - 99ff, 227
    uniform - 99f, 339, 402ff
    uniform exponential - 9gF,
        205,402ff
    weak uniform - 111f
state space 369, 400
stationary point 156
stochastic continuity 213f
\end{verbatim}

\begin{verbatim}
weak - 111f, 205f, 402 ff Zero-Two law (0-2 law) 347ff
\end{verbatim}

strictly positive\\
118, 119, 120, 238, 242, 261\\
-- element 261\\
-- functional 238, 261\\
-- operator 242

\begin{itemize}
  \item subset 261ff\\
subdifferential 48ff, 128ff\\
subeigenvector 261\\
positive - 261\\
subinvariant subset 380\\
sublattice 236\\
sublinear function 47 f
\end{itemize}

\section*{tensor product}
-- of Banach spaces 21ff\\
-- of operators 21ff\\
-- of semigroups 21ff, 88f\\
translation property 220,358\\
translation semigroup $9 \mathrm{f}, 15,18$, 61f, 66ff, 75, 205\\
nilpotent -- 11, 41f, 83, 164£\\
periodic -- 66\\
transformation\\
Fourier - 12 f\\
Laplace - 101, 107\\
transport equation 309f, 320\\
type of a semigroup 2\\
ultrapower 315, 377, 394, 480\\
unimodular function 313\\
unitary 390\\
vector

\begin{itemize}
  \item lattice 235
  \item sublattice 236
\end{itemize}

W*-algebra 369\\
W*-dynamical system 414ff\\
irreducible -- 414 ff\\
weakly sequentially compact 242,322\\
well-posedness 26ff

Zero-Two law (0-2 law) 347ff


\end{document}