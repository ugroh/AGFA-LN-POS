%% -- Chapter A-III
%% --

\author{Günther Greiner and Rainer Nagel}
\chapter{Spectral Theory}\label{chap:A-III}
%% --
\section{Introduction}

In this chapter we start a systematic analysis of the spectrum of a strongly continuous semigroup $T = (T(t))_{t\geq 0}$ on a complex Banach space $E$.
By the spectrum of the semigroup we understand the spectrum $\sigma(A)$ of the generator $A$ of $T$.
In particular we are interested in precise relations between $\sigma(A)$ and $\sigma(T(t))$.
The heuristic formula
%% -- 
\[
	T(t) = e^{tA}
\]
%% -- 
serves as a leitmotiv and suggests relations of the form
%% -- 
\[
\sigma(T(t)) = e^{t\sigma(A)} = \{ e^{t\lambda} : \lambda \in \sigma(A) \}
\]
%% -- 
called \emph{spectral mapping theorem}.
These - or similar - relations will be of great use in Chapter IV and enable us to determine the asymptotic behavior of the semigroup $T$ by the spectrum of the generator.

As a motivation as well as a preliminary step we concentrate here on the spectral radius
%% -- 
\[
r(T(t)) := \sup \{ |\lambda| : \lambda \in \sigma(T(t)) \}, \quad t \geq 0
\]
%% -- 
and show how it is related to the spectral bound
%% -- 
\[
s(A) := \sup \{ \Re\lambda : \lambda \in \sigma(A) \}
\]
%% -- 
of the generator $A$ and to the growth bound
%% -- 
\[
\omega := \inf \{\omega \in \mathbb{R} : \|T(t)\| \leq M_{\omega}\cdot e^{\omega t} \text{ for all } t \geq 0 \text{ and suitable } M_{\omega}\}
\]
%% -- 
of the semigroup $T = (T(t))_{t\geq 0}$.
(Recall that we sometimes write $\omega(T)$ or $\omega(A)$ instead of $\omega$).
The Examples 1.3 and 1.4 below illustrate the main difficulties to be encountered.

\begin{proposition}\label{prop:1.1}
Let $\omega$ be the growth bound of the strongly continuous semigroup $T = (T(t))_{t\geq 0}$.
Then
%% -- 
\[
r(T(t)) = e^{\omega t}
\]
%% -- 
for every $t \geq 0$.
\end{proposition}

\newpage
%% -- a3-2

\section{Spectral Theory}\label{sec:a3-1}
\index{Spectral Theory}

\begin{proof}
From A-I, (1.1) we know that
\[ 
    \omega(T) = \lim_{t \to \infty} \frac{1}{t} \log \|T(t)\|
\]

Since the spectral radius of $T(t)$ is given as
\[
    r(T(t)) = \lim_{n \to \infty} \|T(nt)\|^{1/n}
\]

we obtain for $t > 0$
\[
    r(T(t)) = \lim_{n \to \infty} \exp(t(nt)^{-1} \log \|T(nt)\|) = e^{\omega t}
\]
\end{proof}

It was shown in A-I, Prop.1.11 that the spectral bound $s(A)$ is always dominated by the growth bound $\omega$ and therefore $e^{s(A)t} \leq r(T(t))$.
If the above mentioned spectral mapping theorem holds - as is the case for bounded generators (e.g., see Thm. VII.3.11 of Dunford-Schwartz (1958)) we obtain the equality
\[
    e^{s(A)t} = r(T(t)) = e^{\omega t}
\]
hence $s(A) = \omega$.
Therefore the following corollary is a consequence of the definitions of $s(A)$ and $\omega$.
%% --
\begin{corollary}\label{cor:a3-1.2}
\index{Spectral Theory!Corollary}
Consider the semigroup $T = (T(t))_{t \geq 0}$ generated by some bounded linear operator $A \in L(E)$.
If $\Re\lambda < 0$ for each $\lambda \in \sigma(A)$ then $\lim_{t \to \infty}\|T(t)\| = 0$.
\end{corollary}
%% --

Through this corollary we have re-established a famous result of Liapunov which assures that the solutions of the linear Cauchy problem
\[
    \dot{x}(t) = Ax(t), \quad x(0) = x_{0} \in \mathbb{C}^{n} \quad \text{and} \quad A = (a_{ij})_{n\times n}
\]
are \emph{stable} i.e., they converge to zero as $t \to \infty$, if the real parts of all eigenvalues of the matrix $A$ are smaller than zero.

For unbounded generators the situation is much more difficult and $s(A)$ may differ drastically from $\omega$.
%% --
\begin{example}\label{ex:a3-1.3}
\index{Examples!Banach Function Space}
(Banach function space, Greiner-Voigt-Wolff (1981))
Consider the Banach space $E$ of all complex valued continuous functions on $\mathbb{R}_{+}$ which vanish at infinity and are integrable for $e^{x}dx$, i.e.
\[
    E \coloneqq C_{0}(\mathbb{R}_{+}) \cap L^{1}(\mathbb{R}_{+}, e^{x}dx)
\]
endowed with the norm
\[
    \|f\| \coloneqq \|f\|_{\infty} + \|f\|_{1} = \sup\{|f(x)| : x \in \mathbb{R}_{+}\} + \int_{0}^{\infty} |f(x)|e^{x} dx
\]
\end{example}
%% --


\newpage
%% -- a3-3

The translation semigroup
\[
    T(t)f(x) \coloneqq f(x+t)
\]
is strongly continuous on $E$ and one shows as in A-I,2.4 that its generator is given by
\[
    Af = f', \quad D(A) = \{ f \in E : f \in C^{1}(\mathbb{R}_{+}), f' \in E \}
\]

First we observe that $\|T(t)\| = 1$ for every $t \geq 0$, hence $\omega(T) = 0$.
Moreover it is clear that $\lambda$ is an eigenvalue of $A$ as soon as $\Re\lambda < -1$ (in fact: the function
\[
    x \mapsto e_{\lambda}(x) \coloneqq e^{\lambda x}
\]
belongs to $D(A)$ and is an eigenvector of $A$), hence $s(A) \geq -1$.
For $f \in E$, $\Re\lambda > -1$,
\[
    \|\cdot\|_{1}\text{-}\lim_{t \to \infty} \int_{0}^{t} e^{-\lambda s}T(s)f \, ds
\]
exists since $\|T(s)f\|_{1} \leq e^{-s}\|f\|_{1}$, $s \geq 0$, and
%% --
\[
    \|\cdot\|_{\infty}\text{-}\lim_{t \to \infty} \int_{0}^{t} e^{-\lambda s}T(s)f \, ds
\]
%% --
exists since $\int_{0}^{\infty} e^{x}|f(x)| \, dx < \infty$.
Therefore $\int_{0}^{\infty} e^{-\lambda s}T(s)f \, ds$ exists in $E$ for every $f \in E$, $\Re\lambda > -1$.
As we observed in A-I,Prop.1.11 this implies $\lambda \in \rho(A)$.
Therefore $T = (T(t))_{t\geq 0}$ is a semigroup having $s(A) = -1$ but $\omega(T) = 0$.
%% --
\begin{example}\label{ex:a3-1.4}
\index{Examples!Hilbert Space}
(Hilbert space, Zabczyk (1975))
For every $n \in \mathbb{N}$ consider the $n$-dimensional Hilbert space $E_{n} \coloneqq \mathbb{C}^{n}$ and operators $A_{n} \in L(E_{n})$ defined by the matrices
\[
    A_{n} = \begin{pmatrix}
    0 & 1 & \cdots & 0 \\
    \cdot & 0 & 1 & \cdot \\
    \cdot & \cdot & \cdot & 1 \\
    0 & \cdot & \cdot & 0
    \end{pmatrix}_{n \times n}
\]

These matrices are nilpotent and therefore $\sigma(A_{n}) = \{0\}$.
The elements $x_{n} \coloneqq n^{-1/2}(1, \ldots, 1) \in E_{n}$ satisfy the following properties:

\begin{enumerate}[(i)]
\item 
$\|x_{n}\| = 1$ for every $n \in \mathbb{N}$

\item 
$\lim_{n \to \infty} \|A_{n}x_{n} - x_{n}\| = 0$

\item 
$\lim_{n \to \infty} \|\exp(tA_{n})x_{n} - e^{t}x_{n}\| = 0$

\end{enumerate}
%% --
Consider now the Hilbert space $E \coloneqq \bigoplus_{n \in \mathbb{N}} E_{n}$ and the operator $A \coloneqq (A_{n} + 2\pi in)_{n \in \mathbb{N}}$ with maximal domain in $E$.
Analogously we define a semigroup $T = (T(t))_{t \geq 0}$ by
\[
    T(t) \coloneqq (e^{2\pi int}\exp(tA_{n}))_{n \in \mathbb{N}}
\]
\end{example}
%% --

\newpage
%% -- a3-4

Since $\|\exp(tA_{n})\| \leq e^{t}$ for every $n \in \mathbb{N}$, $t \geq 0$, and since $t \mapsto T(t)x$ is continuous on each component $E_{n}$ it follows that $T$ is strongly continuous.
Its generator is the operator $A$ as defined above.

For $\lambda \in \mathbb{C}$, $\Re\lambda > 0$, we have $\lim_{n \to \infty} \|R(\lambda-2\pi in,A_{n})\| = 0$, hence
\[
    (R(\lambda,A_{n}+2\pi in))_{n \in \mathbb{N}} = (R(\lambda-2\pi in,A_{n}))_{n \in \mathbb{N}}
\]
is a bounded operator on $E$ representing the resolvent $R(\lambda,A)$.
Therefore we obtain $s(A) \leq 0$.
On the other hand, each $2\pi in$ is an eigenvalue of $A$, hence $s(A) = 0$.

Take now $x_{n} \in E_{n}$ as above and consider the sequence $(x_{n})_{n \in \mathbb{N}}$.
From (iii) it follows that for $t > 0$ the number $e^{t}$ is an approximate eigenvalue of $T(t)$ with approximate eigenvector $(x_{n})_{n \in \mathbb{N}}$ (see Def.2.1 below).
Therefore $e^{t} \leq r(T(t)) \leq \|T(t)\|$ and hence $\omega(T) \geq 1$.
On the other hand, it is easy to see that $\|T(t)\| = e^{t}$, hence $\omega(T) = 1$.

Finally if we take $S(t) \coloneqq e^{-t/2}T(t)$ we obtain a semigroup having spectral bound $-\frac{1}{2}$ but such that $\lim_{t \to \infty} \|S(t)\| = \infty$ in contrast with Cor. 1.2.

These examples show that neither the conclusion of Cor.1.2, i.e. '$s(A) < 0$ implies stability', nor the 'spectral mapping theorem'
\[
    \sigma(T(t)) = \exp(t\cdot\sigma(A))
\]
is valid for arbitrary strongly continuous semigroups.
A careful analysis of the general situation will be given in Section 6 below, but we will first develop systematically the necessary spectral theoretic tools for unbounded operators.
%% --
\section{The Fine Structure of the Spectrum}\label{sec:a3-2}
\index{Spectrum!Fine Structure}

As usual, with a closed linear operator $A$ with dense domain $D(A)$ in a Banach space $E$, we associate its spectrum $\sigma(A)$, its resolvent set $\rho(A)$ and its resolvent
\[
    \lambda \mapsto R(\lambda,A) \coloneqq (\lambda - A)^{-1}
\]
which is a holomorphic map from $\rho(A)$ into $L(E)$.
In contrast to the finite dimensional situation, where a linear operator fails to be surjective if and only if it fails to be injective, we now have to distinguish different cases of 'non-invertibility' of $\lambda - A$.
This distinction gives rise to a subdivision of $\sigma(A)$ into different subsets.
We point out that these subsets need not be disjoint, but our defini-



