% !TEX root = chap-a1-test.tex
%% -- Chapter A-I

%% -- a1-1

\chapter{Basic Results on Semigroups on Banach Spaces}\label{chap:a1}

Since the basic theory of one-parameter semigroups can be found in several excellent books (e.g. [Davies (1980)], [Goldstein (1985a)], [Pazy (1983)] or [Hille-Phillips (1957)]) we do not want to give a self-contained introduction to this subject here.
It may however be useful to fix our notation, to collect briefly some important definitions and results (Section 1), to present a list of standard examples (Section 2) and to discuss standard constructions of new semigroups from a given one (Section 3).

In the entire chapter we denote by $E$ a (real or) complex Banach space and consider one-parameter semigroups of bounded linear operators $T(t)$ on $E$.
By this we understand a subset $\{T(t) \colon  t \in \mathbb{R}_{+}\}$ of $L(E)$, usually written as $(T(t))_{t\geq0}$, such that
%% --
\begin{align*}
	T(0) &= \text{Id}, \\
	T(s+t) &= T(s) \cdot T(t) \text{ for all } s, t \in \mathbb{R}_{+}
\end{align*}
%% --
In more abstract terms this means that the map $t \to T(t)$ is a homomorphism from the additive semigroup $(\mathbb{R}_{+},+)$ into the multiplicative semigroup $(L(E),\cdot)$.
Similarly, a one-parameter group $(T(t))_{t\in\mathbb{R}}$ will be a homomorphic image of the group $(\mathbb{R},+)$ in $(L(E),\cdot)$.
%% --
\section{Standard Definitions and Results}\label{sec:a1-1}
%% --
We consider a one-parameter semigroup $(T(t))_{t \geq 0}$ on a Banach space $E$ 

%% --
\newpage
%% -- a1-2
and observe that the domain $\mathbb{R}_{+}$ and the range $L(E)$ of the (semi-
Group) homomorphism $\tau \colon t \to T(t)$ are topological semigroups for the natural topology on $\mathbb{R}_{+}$ and any one of the standard operator topologies on $L(E)$.
We single out the strong operator topology on $L(E)$ and require $\tau$ to be continuous.

\begin{definition}\label{def:a1-1.1}
A one-parameter semigroup $(T(t))_{t\geq0}$ is called strongly continuous if the map $t \to T(t)$ is continuous for the strong operator topology on $L(E)$, i.e. 
%% --
\[
	\lim_{t\to t_{0}} \|T(t)f - T(t_{0})f\| = 0
\]
%% --
for every $f \in E$ and $t$, $t_{0} \geq 0$.
\end{definition}
%% --
Clearly one defines in a similar way weakly continuous, resp. uniformly continuous (compare A-II, Def. 1.19) semigroups, but since we concentrate on the strongly continuous case we agree on the following terminology:
%% --
\begin{quote}
If not stated otherwise, a \emph{semigroup } is a strongly continuous one-parameter semigroup of bounded linear operators.
\end{quote}
%% --
Next we collect a few elementary facts on the continuity and boundedness of one-parameter semigroups.
%% --
\begin{remarks}\label{rem:a1-1.2}
%% --
\begin{enumerate}[(i), wide]

\item 
A one-parameter semigroup $(T(t))_{t \geq 0}$ on a Banach space $E$ is strongly continuous if and only if for any $f \in E$ it is true that $T(t)f \to f$ as $t \to 0$.

\item 
For every strongly continuous semigroup there exist constants $M \geq 1$, $w \in \mathbb{R}$ such that $\|T(t)\| \leq M \cdot e^{wt}$ for every $t \geq 0$.
\item If $(T(t))_{t\geq0}$ is a one-parameter semigroup such that $\|T(t)\|$ is bounded for $0 < t \leq \delta$ then it is strongly continuous if and only if $\lim_{t \to 0} T(t)f = f$ for every $f$ in a total subset of $E$.

\end{enumerate}
\end{remarks}
%% --
The exponential estimate from Remark~\ref{rem:a1-1.2}\,(ii) for the growth of $\|T(t)\|$ can be used to define an important characteristic of the semigroup.
%% --
\begin{definition}\label{def:a1-1.3}
By the growth bound (or type) of the semigroup $(T(t))_{t\geq0}$ we understand the number
%% --
%\begin{align*}
%\omega_{0} &:= \inf\{w \in \mathbb{R} \colon \text{There exists } M \in \mathbb{R}_{+} \text{ such that } \|T(t)\| \leq Me^{wt} \text{ for } t \geq 0\} \\
%&= \lim_{t\to\infty} \frac{1}{t}\log\|T(t)\| = \inf_{t>0} \frac{1}{t}\log\|T(t)\|.
%\end{align*}
%%% --
%% --
\begin{align*}\label{eq:a1-1.1}
\omega_{0} \coloneqq{}& \inf\{w \in \mathbb{R} \colon \text{There exists } M \in \mathbb{R}_{+} \text{ such that } \|T(t)\| \leq Me^{wt} \text{ for } t \geq 0\} \\
={}& \lim_{t\to\infty} \frac{1}{t}\log\|T(t)\| = \inf_{t>0} \frac{1}{t}\log\|T(t)\| \notag
\end{align*}
%% --
\end{definition}
%% --
\newpage
%% -- A1-Seite 3

Particularily important are semigroups such that for every $t \geq 0$ we have $\|T(t)\| \leq M$ (bounded semigroups) or $\|T(t)\| \leq 1$ (contraction semigroups).
In both cases we have $\omega_{0} \leq 0$.

It follows from the subsequent examples and from Definition~\ref{def:a1-1.3} that $\omega_{0}$ may be any number $ -\infty \leq \omega < +\infty$.
Moreover the reader should observe that the infimum in (1.3) need not be attained and that $M$ may be larger than $1$ even for bounded semigroups.

\begin{examples}\label{ex:a1-1.4}
%% --
\begin{enumerate}[(i), wide]
\item 
Take $E = \mathbb{C}^2$, 
\[
	A = \begin{pmatrix}0 & 1\\0 & 0\end{pmatrix} 
	\quad \text{and} \quad 
	T(t) = e^{tA} = \begin{pmatrix}1 & t\\0 & 1\end{pmatrix} .
\]
%
Then for the $\ell^{1}$-norm on $E$ we obtain $\|T(t)\| = 1 + t$, hence $(T(t))_{t\geq0}$ is an unbounded semigroup having growth bound $\omega_{0} = 0$.

\item 
Take $E = L^1(\mathbb{R})$ and for $f \in E$, $t \geq 0$ define

%% --
\begin{align*}
T(t)f(x) \coloneqq 
	\begin{cases}
		2\cdot f(x+t) & \text{if } x \in [-t,0] \\
		f(x+t) & \text{otherwise}.
	\end{cases}
\end{align*}
%% --
Each $T(t)$, $t > 0$, satisfies $\|T(t)\| = 2$ as can be seen by taking $f := \chi_{[0,t]}$.
Therefore $(T(t))_{t \geq 0}$ is a strongly continuous semigroup which is bounded, hence has $\omega_{0} = 0$, but the constant $M$ in (1.3) cannot be chosen to be $1$.

\end{enumerate}
\end{examples}
%% --
The most important object associated to a strongly continuous semigroup $(T(t))_{t\geq0}$ is its \emph{generator} which is obtained as the (right)derivative of the map $t \to T(t)$ at $t = 0$.
Since for strongly continuous semigroups the functions $t \to T(t)f$, $f \in E$, are continuous but not always differentiable we have to restrict our attention to those $f \in E$ for which the desired derivative exists.
We then obtain the \emph{generator} as a not necessarily everywhere defined operator.

\begin{definition}\label{def:a1-1.5}
To every semigroup $(T(t))_{t \geq 0}$ there belongs an operator $(A,D(A))$, called the generator and defined on the domain
%% --
\begin{align*}
	D(A) \coloneqq{} & \{f \in E \colon \lim_{h\to0} \frac{T(h)f - f}{h} \text{ exists in } E\}
\end{align*}
%% --
by
%% --
\begin{align*}
	Af \coloneqq{} & \lim_{h\to0} \frac{T(h)f - f}{h} \quad (f \in D(A)).
\end{align*}
%% --
\end{definition}
%% --
Clearly, $D(A)$ is a linear subspace of $E$ and $A$ is linear from $D(A)$ into $E$.
Only in certain special cases (see 2.1) the generator
%% --
\newpage
%% -- A1-Seite 4


%% --
\newpage
%% -- A1-Seite 5
