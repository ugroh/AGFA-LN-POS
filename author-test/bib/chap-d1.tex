% !TEX root = chap-d1-test.tex
%% -- Chapter D-I
%% --

\chapter{Basic Results on Semigroups and Operator Algebras}\label{chap:D-I}

This is not a systematic introduction into the theory of strongly continuous semigroups on \CA- and \WA-algebras.
For that we refer to [Bratteli-Robinson (1979)], [Davies (1976)] and the survey article of [Oseledets (1984)].
We only prepare for the subsequent chapters on spectral theory and asymptotics by fixing the notations and introducing some standard constructions.

\section{Notations}\label{sec:notations}

\begin{enumerate}[1. , wide]

\item
By $ M $ we shall denote a \CA-algebra with unit $ 1 $.
$ M^{sa} := \{x \in M : x^{*} = x\} $ is the self-adjoint part of $ M $ and $ M_{+} := \{ x^{*}x : x \in M\} $ the positive cone in $ M $.
If $ M' $ is the dual of $ M $, then $ M'_{+} := \{\psi \in M' : \psi(x) \geq 0, x \in M_{+}\} $ is a weak*-closed generating cone in $ M' $.
$ S(M) := \{\psi \in M'_{+}: \psi(1) = 1\} $ is called the state space of $ M $.
For the theory of \CA-algebras and related notions we refer to [Pedersen (1979)].

$ M $ is called a \WA-algebra, if there exists a Banach space $ M_{*} $, such that its dual $ (M_{*})' $ is (isomorphic to) $ M $.
We call $ M_{*} $ the predual of $ M $ and $ \psi \in M_{*} $ a normal linear functional.
It is known that $ M_{*} $ is unique [Sakai (1971), 1.13.3].
For further properties of $ M_{*} $ we refer to [Takesaki (1979), Chapter III].

%% -- d1_2
%% --
\item
A map $ T \in L(M) $ is called positive (in symbols $ T \geq 0 $) if $ T(M_{+}) \subseteq M_{+} $.
$ T \in L(M) $ is called n-positive ($ n \in \mathbb{N} $) if $ T \otimes \text{Id}_{n} $ is positive from $ M \otimes M_{n} $ in $ M \otimes M_{n} $, where $ \text{Id}_{n} $ is the identity map on the \CA-algebra $ M_{n} $ of all $ n \times n $-matrices.
Obviously, every n-positive map is positive.
We call $ T \in L(M) $ a Schwarz map if $ T $ satisfies the inequality
%% --
\[
T(x)T(x)^{*} \leq T(xx^{*}) \, , \, x \in M .
\]
%% --
Note that such $ T $ is necessarily a contraction.
It is well known that every n-positive contraction, $ n \geq 2 $ and that every positive contraction on a commutative \CA-algebra is a Schwarz map [Takesaki (1979), Corollary IV. 3.8.].
As we shall see, the Schwarz inequality is crucial for our investigations.

\item If $ M $ is a \CA-algebra we assume $ T = (T(t))_{t \geq 0} $ to be a strongly continuous semigroup (abbreviated semigroup) while on \WA-algebras we consider weak*-semigroups, i.e. the mapping $ (t \mapsto T(t)x) $ is continuous from $ \mathbb{R}_{+} $ into $ (M,\sigma(M,M_{*})) $, $ M_{*} $ the predual of $ M $, and every $ T(t) \in T $ is $ \sigma(M,M_{*}) $-continuous.
Note that the preadjoint semigroup
%% --
\[
T_{*} = \{ T(t)_{*} : T(t) \in T \}
\]
%% --
is weakly, hence strongly continuous on $ M_{*} $ (see e.g., Davies (1980), Prop.1.23).
We call $ T $ identity preserving if $ T(t)1 = 1 $ and of Schwarz type if every $ T(t) \in T $ is a Schwarz map.

\end{enumerate}
%% --
For the notations concerning one-parameter semigroups we refer to Part A.
In addition we recommend to compare the results of this section of the book with the corresponding results for commutative \CA-algebras, i.e. for $ C_{0}(X) $, $ C(K) $ and $ L^\infty(\mu) $ (see Part B).

\section{A Fundamental Inequality for the Resolvent}\label{sec:fundamental}

If $ T = (T(t))_{t \geq 0} $ is a strongly continuous semigroup of Schwarz maps on a \CA-algebra $ M $ (resp. a weak*-semigroup of Schwarz type on a \WA-algebra $ M $) with generator $ A $, then the spectral bound $ s(A) \leq 0 $.
Then for $ \lambda \in \mathbb{C} $, $ \text{Re}(\lambda) > 0 $, there exists a representation for the resolvent $ R(\lambda,A) $ given by the formula
%% --
\[
 R(\lambda,A)x = \int_{0}^\infty e^{-\lambda t} T(t)x \, \dt \, , \, x \in M
\]
%% --
where the integral exists in the norm topology.


%% -- d1_3
%% --

In [Bratteli-Robinson (1979)] it is shown that $ T $ is a semigroup of Schwarz type if and only if $ \mu R(\mu,A) $ is a Schwarz map for every $ \mu \in \mathbb{R}_{+} $.
Here we relate the domination of two semigroups to an inequality for the corresponding resolvent operator.
This inequality will be needed later.

\begin{theorem}\label{thm:2.1}
Let $ T = (T(t))_{t\geq0} $ be a semigroup of Schwarz type and $ T = (S(t))_{t\geq0} $ a semigroup on a \CA-algebra $ M $ with generators $ A $ and $ B $, respectively.
If
%% --
\[
(*) \qquad (S(t)x)(S(t)x)^{*} \leq T(t)xx^{*}
\]
%% --
for all $ x \in M $ and $ t \in \mathbb{R}_{+} $, then
%% --
\[
(\mu R(\mu,B)x)(\mu R(\mu,B)x)^{*} \leq \mu R(\mu,A)xx^{*}
\]
%% --
for all $ x \in M $ and $ \mu \in \mathbb{R}_{+} $.
The same result holds if $ T $ is a weak*-semigroup of Schwarz type and $ S $ is a weak*-semigroup on a \WA-algebra $ M $ such that $ (*) $ is fulfilled.
\end{theorem}

\begin{proof}
From the assumption $ (*) $ it follows that
%% --
\begin{multline*}
	0 	\leq (S(r)x - S(t)x)(S(r)x - S(t)x)^{*} = \\
		 = (S(r)x)(S(r)x)^{*} - (S(r)x)(S(t)x)^{*} 
		 - (S(t)x)(S(r)x)^{*} + (S(t)x)(S(t)x)^{*}  \\
		 \leq T(r)xx^{*} + T(t)xx^{*} - (S(r)x)(S(t)x)^{*} -  
		   (S(t)x)(S(r)x)^{*}
\end{multline*}
%% --
for every $ r $, $ t \in \mathbb{R}_{+} $.
Hence
%% --
\[
	(S(r)x)(S(t)x)^{*} + (S(t)x)(S(r)x)^{*} \leq T(r)xx^{*} + T(t)xx^{*}.
\]
%% --
Obviously, $ \|S(t)\| \leq 1 $ for all $ t \in \mathbb{R}_{+} $.
Then for all $ \mu \in \mathbb{R}_{+} $ and $ x \in M $:
%% --
\begin{multline*}
	(R(\mu,B)x)(R(\mu,B)x)^{*} = (\int_{0}^\infty e^{-\mu r}S(r)x \, dr)(\int_{0}^\infty e^{-\mu t}S(t)x \, 				\dt)^{*}   \\
	= (\int_{0}^\infty \int_{0}^\infty e^{-\mu(r+t)} (S(r)x)(S(t)x)^{*} \, dr \, \dt) \\
	\leq \int_{0}^\infty \int_{0}^\infty e^{-\mu(r+t)} (T(r)xx^{*} + T(t)xx^{*})/2 \, dr \, \dt \\
	= \int_{0}^\infty e^{-\mu r} T(r)xx^{*} \, dr \int_{0}^\infty e^{-\mu t} \, \dt  \\
	= \mu^{-1} \int_{0}^\infty e^{-\mu r} T(r)xx^{*} \, dr = R(\mu,A)xx^{*}.
\end{multline*}
%% --
Here we used the inequality derived above in the first step.
The second step follows from $ S(t)$ being a contraction semigroup and the third step is achieved by integration.
\end{proof}

\begin{remark}
The assumption that $ T $ is a semigroup of Schwarz type cannot be weakened in general to $ T $ being a positive contraction semigroup.
This is shown by examples in [Davies (1980)] where $ S(t)x $ is given by $ e^{tB}x $ for a skew-adjoint generator $ B $ and $ T(t)x \equiv x $.
\end{remark}

\begin{corollary}\label{cor:2.2}
Let $ T = (T(t))_{t\geq0} $ be a semigroup of Schwarz type on a \CA-algebra $ M $ with generator $ A $.
Then for all $ \mu \in \mathbb{R}_{+} $ and $ x \in M $:
%% --
\[
(\mu R(\mu,A)x)(\mu R(\mu,A)x)^{*} \leq \mu R(\mu,A)(xx^{*}).
\]
%% --
\end{corollary}
%%
%
%% -- d1_{4}
%% --
\begin{proof}
Just set $ S = T $ in Theorem~\ref{thm:2.1}.
%% --
\begin{multline*}
= \frac{1}{2}(\int_{0}^\infty \int_{0}^\infty e^{-\mu(r+t)} ((S(r)x)(S(t)x)^{*} \\
+ (S(t)x)(S(r)x)^{*}) \, dr \, dt \\
\leq \frac{1}{2}(\int_{0}^\infty \int_{0}^\infty e^{-\mu(r+t)} (T(r)xx^{*} + T(t)xx^{*}) \, dr \, dt \\
= (\int_{0}^\infty e^{-\mu s} \, ds)(\int_{0}^\infty e^{-\mu t}T(t)xx^{*} \, dt) = \mu^{-1}R(\mu,A)xx^{*}
\end{multline*}
%% --
where the handling of the integral is justified by [Bourbaki (1955), §8, n° 4, Proposition 9].
\end{proof}


\begin{corollary}\label{cor:2.3}
Let $ T $ be a semigroup of Schwarz maps (resp., weak*-semigroup of Schwarz maps).
Then for all $ \lambda \in \mathbb{C} $ with $ \text{Re}(\lambda) > 0 $:
%% --
\[
(R(\lambda,A)x)(R(\lambda,A)x)^{*} \leq (\text{Re}\lambda)^{-1} R(\text{Re}\lambda,A)xx^{*} \, , \, x \in M .
\]
%% --
In particular for all $ (\mu,\alpha) \in \mathbb{R}_{+} \times \mathbb{R} $, $ x \in M $:
%% --
\[
(\mu R(\mu+i\alpha,A)x)(\mu R(\mu+i\alpha,A)x)^{*} \leq \mu R(\mu,A)(xx^{*}).
\]
%% --
\end{corollary}

\begin{proof}
Let $ \lambda \in \mathbb{C} $ with $ \text{Re}(\lambda) > 0 $.
Then the semigroup
%% --
\[
S := (e^{-i\text{Im}(\lambda)t}T(t))_{t\geq0}
\]
%% --
fulfils the assumption of Thm \ref{thm:2.1} and $ B := A - i\lambda $ is the generator of $ S $.
Consequently $ R(\lambda,A) = R(\text{Re}\lambda,B) $ and the corollary follows from Theorem \ref{thm:2.1}.
\qed
\end{proof}

As in Section C-III the following notion will be an important tool for the spectral theory of semigroups.

\begin{definition}\label{def:2.3}
Let $ E $ be a Banach space and $ \emptyset \neq D $ an open subset of $ \mathbb{C} $.
A family $ R: D \to L(E) $ is called a pseudo-resolvent on $ D $ with values in $ E $ if
%% --
\[
R(\lambda) - R(\mu) = -(\lambda - \mu)R(\lambda)R(\mu)
\]
%% --
for all $ \lambda $ and $ \mu $ in $ D $.
\end{definition}

%% -- d1_5
%% --

If $ R $ is a pseudo-resolvent on $ D = \{\lambda \in \mathbb{C} : \text{Re}(\lambda) > 0\} $ with values in a \CA- or \WA-algebra, then $ R $ is called of Schwarz type if
%% --
\[
(R(\lambda)x)(R(\lambda)x)^{*} \leq (\text{Re}\lambda)^{-1} R(\text{Re}\lambda)xx^{*}
\]
%% --
for all $ \lambda \in D $ and $ x \in M $.
$ R $ is called identity preserving if $ \lambda R(\lambda)1 = 1 $ for all $ \lambda \in D $.

For examples and properties of a pseudo-resolvent see C-III, 2.5.
We state what will be used without further reference.

\begin{enumerate}[(a)]
\item 
If $ \alpha \in \mathbb{C} $ and $ x \in E $ such that $ (\alpha - \lambda)R(\lambda)x = x $ for some $ \lambda \in D $, then $ (\alpha - \mu)R(\mu)x = x $ for all $ \mu \in D $ (use the \enquote{resolvent equation}).

\item 
If $ F $ is a closed subspace of $ E $ such that $ R(\lambda)F \subseteq F $ for some $ \lambda \in D $, then $ R(\mu)F \subseteq F $ for all $ \mu $ in a neighbourhood of $ \lambda $.
This follows from the fact that for all $ \mu \in D $ near $ \lambda $ the pseudo-resolvent in $ \mu $ is given by
%% --
\[
R(\mu) = \sum_{n} (\lambda - \mu)^{n} R(\lambda)^{n+1}.
\]
%% --
\end{enumerate}

\begin{definition}\label{def:2.4}
We call a semigroup $ T $ on the predual $ M_{*} $ of a \WA-algebra $ M $ identity preserving and of Schwarz type, if its adjoint weak*-semigroup has these properties.
Likewise, a pseudo-resolvent $ R $ on $ D = \{\lambda \in \mathbb{C} : \text{Re}(\lambda) > 0\} $ with values in $ M_{*} $ is called identity preserving and of Schwarz type, if $ R' $ has these properties.
\end{definition}

Since for a semigroup of contractions on a Banach space
%% --
\[
\text{Fix}(T) = \bigcap_{t \geq 0} \text{ker}(\text{Id} - T(t)) =
\]
%% --
\[
= \text{ker}(\text{Id} - \lambda R(\lambda,A)) = \text{Fix}(\lambda R(\lambda,A))
\]
%% --
for all $ \lambda \in \mathbb{C} $ with $ \text{Re}(\lambda) > 0 $, it follows that a semigroup of contractions on $ M $ is identity preserving if and only if the (pseudo)-resolvent on $ D = \{\lambda \in \mathbb{C} : \text{Re}(\lambda) > 0\} $ given by
%% --
\[
R(\lambda) := R(\lambda,A)|_{D}
\]
%% --
is identity preserving.
By Corollary \ref{cor:2.2} an analogous statement holds for \enquote{Schwarz type}.

%% -- d1_6
%% --

\section{Induction and Reduction}

\begin{enumerate}[1. , wide]
\item
If $ E $ is a Banach space and $ S \subseteq L(E) $ a semigroup of bounded operators, then a closed subspace $ F $ is called $ S $-invariant, if $ SF \subseteq F $ for all $ S \in S $.
We call the semigroup $ S|_{F} := \{S|_{F} : S \in S \} $ the reduced semigroup.
Note that for a one-parameter semigroup $ T $ (resp., pseudo-resolvent $ R $) the reduced semigroup is again strongly continuous (resp. $ R|_{F} $ is again a pseudo-resolvent) (compare the construction in A-I,3.2).

\item
Let $ M $ be a \WA-algebra, $ p \in M $ a projection and $ S \in L(M) $ such that $ S(p^{\perp}M) \subseteq p^{\perp}M $ and $ S(Mp^{\perp}) \subseteq Mp^{\perp} $, where $ p^{\perp} := 1-p $.
Since for all $ x \in M $:
%% --
\[
p[S(x) - S(pxp)] = p[S(p^{\perp}xp) + S(xp^{\perp})]p = 0,
\]
%% --
we obtain $ p(Sx)p = p(S(pxp))p $.
Therefore the map
%% --
\[
S_{p} := (x \mapsto p(Sx)p) : pMp \to pMp
\]
%% --
is well defined.
We call $ S_{p} $ the induced map.
If $ S $ is an identity preserving Schwarz map, then it is easy to see that $ S_{p} $ is again a Schwarz map such that $ S_{p}(p) = p $.

If $ T = (T(t))_{t\geq0} $ is a weak*-semigroup on $ M $ which is of Schwarz type and if $ T(t)(p^{\perp}) \leq p^{\perp} $ for all $ t \in \mathbb{R}_{+} $, then $ T $ leaves $ p^{\perp}M $ and $ Mp^{\perp} $ invariant.
It is easy to see that the induced semigroup $ T_{p} = (T(t)_{p})_{t\geq0} $ is again a weak*-semigroup.

If $ R $ is an identity preserving pseudo-resolvent of Schwarz type on $ D = \{\lambda \in \mathbb{C} : \text{Re}(\lambda) > 0\} $ with values in $ M $ such that $ R(\mu)p^{\perp} \leq p^{\perp} $ for some $ \mu \in \mathbb{R}_{+} $, then $ p^{\perp}M $ and $ Mp^{\perp} $ are $ R $-invariant.
Again, the induced pseudo-resolvent $ R_{p} $ is of Schwarz type and identity preserving.

\item
Let $ \phi $ be a positive normal linear functional on a \WA-algebra $ M $ such that $ T_{*}\phi = \phi $ for some identity preserving Schwarz map $ T $ on $ M $ with preadjoint $ T_{*} \in L(M_{*}) $.
Then $ T(s(\phi)^{\perp}) \leq s(\phi)^{\perp} $ where $ s(\phi) $ is the support projection of $ \phi $.

To see this let $ L_{\phi} := \{x \in M: \phi(xx^{*}) = 0\} $ and $ M_{\phi} := L_{\phi} \cap L_{\phi}^{*} $.
Since $ \phi $ is $ T_{*} $-invariant, and $ T $ is a Schwarz map, the subspaces $ L_{\phi} $ and $ M_{\phi} $ are $ T $-invariant.
From $ M_{\phi} = s(\phi)^{\perp}Ms(\phi)^{\perp} $ and $ T(s(\phi)^{\perp}) \leq 1 $ it follows that $ T(s(\phi)^{\perp}) \leq s(\phi)^{\perp} $.

\end{enumerate}
%
%% -- d1_7
%% --

Let $ T_{s(\phi)} $ be the induced map on $ M_{s(\phi)} $. 
If
%% --
\[
s(\phi)M_{*}s(\phi) := \{\psi \in M_{*} : \psi = s(\phi)\psi s(\phi)\}
\]
%% --
where $ \langle s(\phi)\psi s(\phi),x \rangle := \langle \psi,s(\phi)xs(\phi) \rangle $ ($ x \in M $), and if $ \psi \in s(\phi)M_{*}s(\phi) $, then for all $ x \in M $:
%% --
\[
(T_{*}\psi)(x) = \psi(Tx) = \langle \psi,s(\phi)(Tx)s(\phi) \rangle =
\]
%% --
\[
= \langle \psi,s(\phi)(T(s(\phi)xs(\phi)))s(\phi) \rangle = \langle T_{*}\psi,s(\phi)xs(\phi) \rangle,
\]
%% --
hence $ T_{*}\psi \in s(\phi)M_{*}s(\phi) $.
Since the dual of $ s(\phi)M_{*}s(\phi) $ is $ M_{s(\phi)} $, it follows that the adjoint of the reduced map $ T_{*}| $ is identity preserving and of Schwarz type.

For example, if $ T $ is an identity preserving semigroup of Schwarz type on $ M_{*} $ such that $ \phi \in \text{Fix}(T) $, then the semigroup $ T|(s(\phi)M_{*}s(\phi)) $ is again identity preserving and of Schwarz type.
Furthermore, if $ R $ is a pseudo-resolvent on $ D = \{\lambda \in \mathbb{C} : \text{Re}(\lambda) > 0\} $ with values in $ M_{*} $ which is identity preserving and of Schwarz type such that $ R(\mu)\phi = \phi $ for some $ \mu \in \mathbb{R}_{+} $, then $ R|s(\phi)M_{*}s(\phi) $ has the same properties.

%% -- Ende d1
%% --
