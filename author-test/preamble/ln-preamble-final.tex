%% --
%% Welche Pakete verwenden wir 
%% -- 
%\usepackage{graphicx}        
%\usepackage{libertine}

%% -- Nummerierung Abschnittsweise
%% --
%\renewcommand\thesection{\arabic{section}}
%\renewcommand\thesubsection{\thesection.\arabic{subsection}}

%% -- AMS-math und mathtools
%% --
%\usepackage{amsmath, amssymb, amsthm}	
%\usepackage{mathtools}

%% -- Die Theorem-Umgebungen: Nummerierung nach Abschnitten
%% --

%% --
%\theoremstyle{plain}
%\newtheorem{theorem}{Theorem}[section]
%\newtheorem{proposition}[theorem]{Proposition}
%\newtheorem{corollary}[theorem]{Corollary}
%\newtheorem{lemma}[theorem]{Lemma} 
%%% --
%\theoremstyle{definition}
%\newtheorem{example}[theorem]{Example}
%\newtheorem{examples}[theorem]{Examples} 
%\newtheorem{definition}[theorem]{Definition}
%%% --
%\theoremstyle{remark}
%\newtheorem{remark}[theorem]{Remark}
%\newtheorem{remarks}[theorem]{Remarks}



%% -- Unsere Pakete
%% --
\usepackage[english]{babel}					% Trennungen richtig
\usepackage{csquotes}						% \enquote{Text} gibt die US Variante von "Text"
\usepackage[inline,shortlabels]{enumitem}	% \begin{enumerate}[(i)] oder [(a)] 
\setlist{parsep=0.0em}					
\usepackage{ragged2e}						% \RaggedRight = Flattersatz
%% --
\usepackage{tikz}
\usepackage{tikz-cd}
\usetikzlibrary{matrix,arrows.meta,calc}
%% --
\usepackage{comment}
\usepackage{xspace}

%% --
%\usepackage[round]{natbib}
%%\bibliographystyle{spmpsci}
%\bibliographystyle{abbrvnat}
%\newcommand{\citeidx}[2]{\citet{#1}\index{References!#2!cited on}} %% Für Index
%%% -- Eintrag im Index \citeidx{kato:1966}{Kato} 

%% --

%\usepackage{hyperref}
%\hypersetup{
%    colorlinks=true,
%    linkcolor=blue,
%    citecolor=blue,
%    urlcolor=blue,
%    linktoc=all
%}



