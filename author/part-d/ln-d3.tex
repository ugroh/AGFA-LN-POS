% chapter-D-III.tex
%% -- Datnd 2025/01/18

\chapter{Spectral Theory of Positive Semigroups on W*-Algebras and their Preduals}

Motivated by the classical results of Perron and Frobenius one expects the following spectral properties for the generator $A$ of a positive semigroup: The spectral bound $s(A) := \sup\{\Re(\lambda) : \lambda \in \sigma(A)\}$ belongs to the spectrum $\sigma(A)$ and the boundary spectrum
%% -- 
\[
\sigma_{b}(A) := \sigma(A) \cap \{s(A)+i\mathbb{R}\}
\]
%% -- 
possesses a certain symmetric structure, called cyclicity.

Results of this type have been proved in Chapter B-III for positive semigroups on commutative C*-algebras, but in the non-commutative case the situation is more complicated.
While \enquote{$s(A) \in \sigma(A)$} still holds (see Greiner-Voigt-Wolff (1980)) the cyclicity of the boundary spectrum $\sigma_{b}(A)$ is true only under additional assumptions on the semigroup (e.g., irreducibility, see Section 1 below).

For technical reasons we consider mostly strongly continuous semigroups on the predual of a W*-algebra $M$ or its adjoint semigroup which is a weak*-continuous semigroup on $M$.

\section{Spectral Theory for Positive Semigroups on Preduals}

The aim of this section is to develop a Perron-Frobenius theory for identity preserving semigroups of Schwarz type on W*-algebras.
But as we will show in the example preceding Theorem 1.1 below the boundary spectrum is no longer cyclic.
The appropriate hypothesis on the semigroup implying the desired results seems to be the concept of irreducibility.

Let us first recall some facts on normal linear functionals.
If $\phi$ is a normal linear functional on a W*-algebra $M$ then there exists a partial isometry $u\in M$ and a positive linear functional $|\phi|\in M_{*}$ such that
%% -- 
\[
\phi(x) = |\phi|(xu) =: (u|\phi|)(x), x\in M
\]
%% -- 
%% -- 
\[
u^*u = s(|\phi|),
\]
%% -- 
where $s(|\phi|)$ denotes the support projection of $|\phi|$ in $M$.
We refer to this as the \emph{polar decomposition} of $\phi$ [Takesaki (1979), Theorem III.4.2].
In addition, $|\phi|$ is uniquely determined by the following two conditions [Takesaki (1979), Proposition III.4.6]:
%% -- 
\[
	\|\phi\| = \| |\phi| \|,
\]
%% -- 
$(*)$ 
%% -- 
\[
	|\phi(x)|^{2} \leq |\phi|(xx^*) \quad (x\in M).
\]
%% -- 
For the polar decomposition of $\phi^*$, where $\phi^*(x) = \phi(x^*)^*$, we obtain
%% -- 
\[
	\phi^* = u^*|\phi^*|, \quad |\phi^*| = u|\phi|u^* \quad \text{and} \quad 		uu^* = s(|\phi^*|).
\]
%% -- 
It is easy to see that $u^*\in s(|\phi|)M$.

If $\Psi$ is a subset of the state space of a C*-algebra $M$, then $\Psi$ is called \emph{faithful} if $0 \leq x\in M$ and $\psi(x) = 0$ for all $\psi\in\Psi$ implies $x = 0$.
$\Psi$ is called \emph{subinvariant} for a positive map $T\in\mathcal{L}(M)$ (resp., positive semigroup $T$) if $T'\psi \leq \psi$ for all $\psi\in\Psi$ (resp., $T(t)'\psi \leq \psi$ for all $T(t)\in T$ and $\psi\in\Psi$).
Recall that for every positive map $T\in\mathcal{L}(M)$ there exists a state $\phi$ on $M$ such that $T'\phi = r(T)\phi$ [Groh (1981), Theorem 2.1], where $r(T)$ denotes the spectral radius of $T$.

Let us start our investigation with two lemmas.
Recall that $\Fix(T)$ is the fixed space of $T$, i.e. the set $\{x\in M: Tx=x\}$.

% ln-part-d3_3.tex

\begin{lemma}\label{lem:1.1}
Suppose $M$ to be a C*-algebra and $T\in\mathcal{L}(M)$ an identity preserving Schwarz map.

\begin{enumerate}[(i)]
\item Let $b: M\times M \to M$ be a sesquilinear map such that for all $z\in M$ $b(z,z) \geq 0$.
Then $b(x,x) = 0$ for some $x\in M$ if and only if $b(x,y) = 0$ and $b(y,x) = 0$ for all $y\in M$.

\item If there exists a faithful family $\Psi$ of subinvariant states for $T$ on $M$, then $\Fix(T)$ is a \CA-subalgebra of $M$ and $T(xy) = xT(y)$ for all $x\in\Fix(T)$ and $y\in M$.

\end{enumerate}
\end{lemma}
%%  --
\begin{proof} 
(i) Take $0 \leq \psi\in M^*$ and consider $f := \psi\circ b$.
Then $f$ is a positive semidefinite sesquilinear form on $M$ with values in $\mathbb{C}$.
From the Cauchy-Schwarz inequality it follows that $f(x,x) = 0$ for some $x\in M$ if and only if $f(x,y) = 0$ and $f(y,x) = 0$ for all $y\in M$.
Since the positive cone $M^*_{+}$ is generating, assertion (a) is proved.

(ii) Since $T$ is positive it follows $T(x)^* = T(x^*)$ for all $x\in M$.
Hence $\Fix(T)$ is a self adjoint subspace of $M$, i.e. invariant under the involution on $M$.
For every $x,y\in M$ let
%% -- 
\[
	b(x,y) := T(xy^*) - T(x)T(y)^*.
\]
%% -- 
Then $b$ satisfies the assumptions of (i).

If $x\in\Fix(T)$ then
%% -- 
\[
0 \leq xx^* = (Tx)(Tx)^* \leq T(xx^*),
\]
%% -- 
hence
%% -- 
\[
0 \leq \psi(T(xx^*) - xx^*) \leq 0 \quad \text{for all } \psi\in\Psi.
\]
%% -- 
But this implies $T(xx^*) = T(x)T(x)^* = xx^*$.
Consequently, $b(x,x) = 0$.
Hence $T(xy^*) = xT(y)^*$ for all $y\in M$ and (ii) is proved.
\end{proof}

\begin{lemma}\label{lem:1.2}
Let $M$ be a \WA-algebra, $T$ an identity preserving Schwarz map on $M$ and $S\in\mathcal{L}(M)$ such that $S(x)(Sx)^* \leq T(xx^*)$ for every $x\in M$.

\begin{enumerate}[(a)]
\item If $v\in M$ such that $S(v^*) = v^*$ and $T(v^*v) = v^*v$, then $T(xv) = S(x)v$ for all $x\in M$.
\item Suppose there exists $\phi\in M_{*}$ with polar decomposition $\phi = u|\phi|$ such that $S_{*}\phi = \phi$ and $T_{*}|\phi| = |\phi|$.
If the closed subspace $s(|\phi|)M$ is T-invariant, then $Su^* = u^*$ and $T(u^*u) = u^*u$.
\end{enumerate}
\end{lemma}
%% --
\begin{proof}
(a) Define a positive semidefinite sesquilinear map $b: M\times M \to M$ by
%% -- 
\[
b(x,y) := T(xy^*) - S(x)S(y)^* \quad (x,y\in M).
\]
%% -- 
\end{proof}

%% -- ln-part-d3_4 --%%

Since $b(v^{*},v^{*}) = 0$ we obtain $b(x,v^{*}) = 0$ for all $x \in M$ (Lemma 1.1.a), hence $T(xv) = S(x)v$.

\begin{enumerate}[(a)]
\item 
Since $s(|\phi|)M$ is a closed right ideal, the closed face $F := s(|\phi|)(M_{+})s(|\phi|)$ determines $s(|\phi|)M$ uniquely, i.e.,
%% --
\[
s(|\phi|)M = \{x \in M : xx^{*} \in F\}
\]
%% --
[Pedersen (1979), Theorem 1.5.2].
Since $T$ is a Schwarz map and $s(|\phi|)M$ is $T$-invariant, it follows $TF \subseteq F$.
On the other hand, if $x \in s(|\phi|)M$ then $xx^{*} \in F$.
Consequently,
%% --
\[
0 \leq S(x)S(x)^{*} \leq T(xx^{*}) \in F,
\]
%% --
whence $S(x) \in s(|\phi|)M$.

Next we show $T(u^{*}u) = u^{*}u$ and $Su^{*} = u^{*} \in s(|\phi|)M$.
First of all
%% --
\[
0 \leq (Su^{*} - u^{*})(Su^{*} - u^{*})^{*} \leq
\]
%% --
%% --
\[
\leq T(u^{*}u) - u^{*}S(u^{*})^{*} - (Su^{*})u + u^{*}u.
\]
%% --
Since $S_{*}\phi = \phi$, $T_{*}|\phi| = |\phi|$ and $\phi = u|\phi|$ it follows
%% --
\[
0 \leq |\phi|((Su^{*} - u^{*})(Su^{*} - u^{*})^{*}) \leq
\]
%% --
%% --
\[
\leq 2|\phi|(u^{*}u) - |\phi|(S(u^{*})u)^{*} - |\phi|(S(u^{*})u) =
\]
%% --
%% --
\[
= 2|\phi|(uu^{*}) - \phi(u^{*})^{*} - \phi(u^{*}) =
\]
%% --
%% --
\[
= 2(|\phi|(u^{*}u) - |\phi|(u^{*}u)) = 0.
\]
%% --

Since $(Su^{*} - u^{*})(Su^{*} - u^{*}) \in F$ and $|\phi|$ is faithful on $F$ we obtain $Su^{*} = u^{*}$.
Consequently,
%% --
\[
0 \leq u^{*}u = (Su^{*})(Su^{*})^{*} \leq T(u^{*}u).
\]
%% --

Hence $T(u^{*}u) = u^{*}u$ by the faithfulness and $T$-invariance of $|\phi|$.
\end{enumerate}

% !TEX root = ln-part-d3-test.tex
%% -- ln-part-d3_5 --%%

\begin{remark}\label{rem:1.3}
Take $S$ and $T$ as in Lemma 1.2 (b).
If $V_{u^*}$ (resp. $V_u$) is the map $(x \mapsto xu^*)$ (resp. $(x \mapsto xu)$) on $M$, then $V_{u^*}$ is a continuous bijection from $Ms(|\phi|)$ onto $Ms(|\phi^*|)$ with inverse $V_u$ (because $V_u \circ V_{u^*} = \operatorname{Id}_{Ms(|\phi|)}$ and $V_{u^*} \circ V_u = \operatorname{Id}_{Ms(|\phi^*|)}$).
Let $x \in M$.
From $T(xu) = S(x)u$ we obtain $T(xu)u^* = S(x)uu^*$.
In particular, if $Ms(|\phi^*|)$ is $S$-invariant, then
%% --
\[
(V_{u^*} \circ T \circ V_u)(x) = T(xu)u^* = S(x)
\]
%% --
for every $x \in Ms(|\phi^*|)$.
Let $T|$ (resp. $S|$) be the restriction of $T$ to $Ms(|\phi|)$ (resp. of $S$ to $Ms(|\phi^*|)$).
Then the following diagram is commutative:
%% --
\begin{equation*}
\begin{tikzcd}[column sep=large, row sep=large, scale=1.5]
Ms(|\phi|) \arrow[r, "T|"] \arrow[d, "V_u"'] & Ms(|\phi|) \arrow[d, "V_{u^*}"] \\
Ms(|\phi^*|) \arrow[r, "S|"'] & Ms(|\phi^*|)
\end{tikzcd}
\end{equation*}
%% --
In particular, $\sigma(S|) = \sigma(T|)$.
Therefore we may deduce spectral properties of $S|$ from $T|$ and vice versa.
More concrete applications of Lemma 1.2 will follow.
\end{remark}
%% --
We now investigate the fixed space $\operatorname{Fix}(R) := \operatorname{Fix}(\lambda R(\lambda))$, $\lambda \in D$, of a pseudo-resolvent $R$ with values in the predual of a W*-algebra $M$.

\begin{proposition}\label{prop:1.4}
Let $R$ be a pseudo-resolvent on $D = \{\lambda \in \mathbb{C}: \operatorname{Re}(\lambda) > 0\}$ with values in the predual $M_*$ of a W*-algebra $M$ and suppose $R$ to be identity preserving and of Schwarz type.

\begin{enumerate}[(a)]
\item 
If $a \in \mathbb{R}$ and $\psi \in M_*$ such that $(\gamma - ia)R(\gamma)\psi = \psi$ for some $\gamma \in D$, then $\lambda R(\lambda)|\psi| = |\psi|$ and $\lambda R(\lambda)|\psi^*| = |\psi^*|$ for all $\lambda \in D$.

\item 
$\operatorname{Fix}(R)$ is invariant under the involution in $M_*$.
If $\psi \in \operatorname{Fix}(R)$ is self adjoint, then the positive part $\psi^+$ and the negative part $\psi^-$ of $\psi$ are elements of $\operatorname{Fix}(R)$.
\end{enumerate}
\end{proposition}

%% -- ln-part-d3_6 --%%

\begin{proof}
If $(\gamma - i\alpha)R(\gamma)\psi = \psi$ then $(\lambda - i\alpha)R(\lambda)\psi = \psi$ for all $\lambda \in D$.
In particular, $\mu R(\mu + i\alpha)\psi = \psi$ ($\mu \in \mathbb{R}_+$).
For all $x \in M$ we obtain
%% --
\[
|\psi(x)|^2 = |<\mu R(\mu+i\alpha)'x,\psi>|^2 \leq
\]
%% --
%% --
\[
\leq \|\psi\| <(\mu R(\mu+i\alpha)'x)(\mu R(\mu+i\alpha)'x)^*,\psi> \leq
\]
%% --
%% --
\[
\leq \|\psi\| <\mu R(\mu)'(xx^*),|\psi|>
\]
%% --
(D-I, Corollary 2.2).
Since
%% --
\[
\|\psi\| = \| |\psi| \| = |\psi|(1) =
\]
%% --
%% --
\[
= <\mu R(\mu)'1,|\psi|> = \| \mu R(\mu)|\psi| \|,
\]
%% --
we obtain $\mu R(\mu)|\psi| = |\psi|$ by the uniqueness theorem (*) mentioned at the beginning.
Therefore $|\psi| \in \operatorname{Fix}(R)$.
Since
%% --
\[
0 \leq (\mu R(\mu)'x)(\mu R(\mu)'x)^* \leq \mu R(\mu)'xx^*,
\]
%% --
the map $R(\mu)$ is positive.
Consequently $(\mu+i\alpha)R(\mu)\psi^* = \psi^*$ from which $|\psi^*| \in \operatorname{Fix}(R)$ follows.
If $\phi \in \operatorname{Fix}(R)$ is selfadjoint with Jordan decomposition $\phi = \phi^+ - \phi^-$, then $|\phi| = \phi^+ + \phi^-$ [Takesaki (1979), Theorem III.4.2.].
From this we obtain that $\phi^+$ and $\phi^-$ are in $\operatorname{Fix}(R)$.
\end{proof}
%% --
\begin{corollary}\label{cor:1.5}
Let $T$ be an identity preserving semigroup of Schwarz type on $M_*$ with generator $A$ and suppose $P\sigma(A) \cap i\mathbb{R} \neq \emptyset$.

\begin{enumerate}[(i)]
\item
If $\alpha \in \mathbb{R}$ and $\psi \in \operatorname{ker}(i\alpha - A)$, then $|\psi|$ and $|\psi^*|$ are elements of $\operatorname{Fix}(T) = \operatorname{ker}(A)$.

\item 
$\operatorname{Fix}(T)$ is invariant under the involution of $M_*$.
If $\psi \in \operatorname{Fix}(T)$ is self adjoint, then the positive part $\psi^+$ and the negative part $\psi^-$ of $\psi$ are elements of $\operatorname{Fix}(T)$.
\end{enumerate}

\end{corollary}
%% --
The proof follows immediately from D-I, Corollary 2.2 and the fact that $\operatorname{ker}(A) = \operatorname{Fix}(\lambda R(\lambda,A))$ for all $\lambda \in \mathbb{C}$ with $\operatorname{Re}(\lambda) > 0$.

If $T$ is the semigroup of translations on $L^1(\mathbb{R})$ and $A'$ the gene-

%% --
