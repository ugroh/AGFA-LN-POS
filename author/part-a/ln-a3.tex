\chapter{Spectral Theory}

\author{Günther Greiner and Rainer Nagel}

\section{Introduction}

In this chapter we start a systematic analysis of the spectrum of a strongly continuous semigroup $T = (T(t))_{t\geq 0}$ on a complex Banach space $E$.
By the spectrum of the semigroup we understand the spectrum $\sigma(A)$ of the generator $A$ of $T$.
In particular we are interested in precise relations between $\sigma(A)$ and $\sigma(T(t))$.
The heuristic formula
%% -- 
\[
\text{\enquote{$T(t) = e^{tA}$}}
\]
%% -- 
serves as a leitmotiv and suggests relations of the form
%% -- 
\[
\text{\enquote{$\sigma(T(t)) = e^{t\sigma(A)} = \{ e^{t\lambda} : \lambda \in \sigma(A) \}$}},
\]
%% -- 
called \enquote{spectral mapping theorem}.
These - or similar - relations will be of great use in Chapter IV and enable us to determine the asymptotic behavior of the semigroup $T$ by the spectrum of the generator.

As a motivation as well as a preliminary step we concentrate here on the spectral radius
%% -- 
\[
r(T(t)) := \sup \{ |\lambda| : \lambda \in \sigma(T(t)) \}, \quad t \geq 0
\]
%% -- 
and show how it is related to the spectral bound
%% -- 
\[
s(A) := \sup \{ \Re\lambda : \lambda \in \sigma(A) \}
\]
%% -- 
of the generator $A$ and to the growth bound
%% -- 
\[
\omega := \inf \{\omega \in \mathbb{R} : \|T(t)\| \leq M_{\omega}\cdot e^{\omega t} \text{ for all } t \geq 0 \text{ and suitable } M_{\omega}\}
\]
%% -- 
of the semigroup $T = (T(t))_{t\geq 0}$.
(Recall that we sometimes write $\omega(T)$ or $\omega(A)$ instead of $\omega$).
The Examples 1.3 and 1.4 below illustrate the main difficulties to be encountered.

\begin{proposition}\label{prop:1.1}
Let $\omega$ be the growth bound of the strongly continuous semigroup $T = (T(t))_{t\geq 0}$.
Then
%% -- 
\[
r(T(t)) = e^{\omega t}
\]
%% -- 
for every $t \geq 0$.
\end{proposition}