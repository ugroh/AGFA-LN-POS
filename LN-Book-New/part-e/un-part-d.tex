%% --
\setcounter{section}{0}
%% --
\chapternopage{Updated Notes Part IV}
\chaptermark{Updated Notes Part IV}
%% --
\section*{Chapter 1}
\addcontentsline{toc}{section}{Chapter 1}
%% --
An overview of positive operators on operator algebras can be found in \textcite{zbmath06128372}, but there seems to be no systematic reference for positive $C_{0}$-semigroups on operator algebras.
However, many papers deal with Markov semigroups (see \eg \textcite{zbmath03762672}) or with so-called E-semigroups (see \textcite{zbmath01949821}).
%% --
\section*{Chapter 2}
\addcontentsline{toc}{section}{Chapter 2}
%% -- 
\begin{enumerate}
\item
As we have seen in Part-I, Section~\ref{sec:a2-3}, strongly continuous semigroups on commutative \WA-algebras, that is, on $L^{\infty}$, are already norm-continuous. 
The proof depends heavily on the Grothendieck property and the Dunford-Pettis property of these Banach spaces.

In the noncommutative case it still holds that every \WA-algebra has the Grothendieck property as shown in \textcite{zbmath00537321}, with an alternative approach in \textcite{zbmath00125315}.
Surprisingly, if every strongly continuous $C_{0}$-semigroup on a \CA-algebra has a bounded generator, then it is a Grothendieck space.
To prove this, one uses the fact that a \CA-algebra is a Grothendieck space if and only if $c_{0}$ is not a complemented subspace (see 
\textcite[Proposition 3.1.13 and Proposition 4.2.1]{zbmath07458830}).
But on $\BH$ (where $H$ is an infinite-dimensional Hilbert space), there always exist $C_{0}$-semigroups that are strongly continuous but not uniformly continuous (see Part-IV, Example~\ref{ex:d2-1.1}). 
It follows from Part-I, Theorem~\ref{thm:a2-3.5} that $\BH$ with infinite-dimensional $H$ 
does not have the Dunford-Pettis property. 
This also follows from the example given above above and the fact that the Hilbert space $H$ can be identified with a direct factor of $\LH$.

For the characterization of \WA-algebras with the Dunford-Pettis property one needs the  concept of finite type I \WA-algebras.
According to \textcite[Theorem V.1.27]{zbmath01692441} a \WA-algebra  $M$ is of finite type I if and only if there exist finite dimensional Hilbert spaces $H_{j}$ and commutative \WA-algebras $M_{j}$ such that $ M = \oplus_{j} ( \mathcal{B}(H_{j}) \overline{\otimes} M_{j} )$.

In \textcite{zbmath00125315} it is shown that a \WA-algebra has the Dunford-Pettis property if and only if it is of finite type I with $\sup_{j} \dim H_{j} < \infty$.
On the other hand, by \textcite{zbmath00097659}, the predual of a \WA-algebra has the Dunford-Pettis property if and only if it is of finite type I without further restrictions on the dimensions of the Hilbert spaces.

\item 
In contrast to all of the above, a strongly continuous $C_{0}$-semigroup of completely positive operators on a \WA-algebra has a bounded generator. 
According to \textcite{zbmath01495754}, this follows from the fact that for sequences of completely positive maps on \WA-algebras, strong and norm convergence to the identity operator are equivalent. 
This result---the equivalence of strong and norm convergence 
for completely positive maps---holds not just for \WA-algebras 
but extends to \AW-algebras (\textcite{zbmath01495754}).
\end{enumerate}
%%% --
\section*{Chapter 3}
\addcontentsline{toc}{section}{Chapter 3}
%% -- 
\begin{enumerate}
\item 
By \textcite[Theorem 5.3.1]{zbmath03736445}] one has $s(A) \in \sigma(A)$ whenever $A$ generates a positive semigroup on an ordered Banach space with normal cone and if the spectrum of $A$ is non-empty.
Moreover, $s(A) = \omega_{0}$ if the norm is additive on the dual positive cone, see \textcite[Theorem 2.4.4]{zbmath03883065}\,---\,\CA-algebras enjoy both properties.

However, it is helpful to provide a summary in the context of \CA-algebras and preduals of \WA-algebras, as this allows for a more direct treatment of this topic.
For this see \textcite{groh:2026}.


\item 
While \WA-algebras are not stable under the ultraproduct construction (see \textcite[p. 79]{zbmath03640303} or \textcite{zbmath03467832}), their preduals are and allow the spectral theoretic techniques discussed in Chapter 3 and Chapter 4.
More on ultraproducts of \WA-algebras can be found in \textcite{zbmath06326930}.

\item 
Another approach to the spectral theory on \WA-algebras can be found in \textcite{zbmath06031834}, where a Jacobs-de Leeuw-Glicksberg decomposition is constructed.
This leads to a noncommutative version of the Perron-Frobenius theorem for \WA-algebras and is applied to the asymptotics of \WA-dynamical systems.
A similar approach is in \textcite{zbmath06728793}.

\end{enumerate}
%% --
\section*{Chapter 4}
\addcontentsline{toc}{section}{Chapter 4}
%% --
As in Part-IV, Chapter 3,  results from \textcite{zbmath07964911} on semigroups on ordered Banach spaces with a normal cone and order unit can be applied. 
More precise asymptotic results on \WA-algebras and their preduals can be found in \textcite{zbmath02244715}.
%% --
%% -- References 
%% --
\section*{References}
\addcontentsline{toc}{section}{References}
{\RaggedRight
  \printbibliography[heading=none]
 }