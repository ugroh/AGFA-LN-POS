%% --
\setcounter{section}{0}
\chapternopage{Asymptotics of Positive Semigroups on Banach Lattices}\label{chap:c4}%
\index{semigroup!positive semigroup}
%% --
In this chapter we describe the long term behavior of positive semigroups and discuss some concrete examples in more detail.
The first section is devoted to the stability of positive semigroups, and we give sufficient and necessary conditions which ensure that the semigroup (and the solution of the abstract Cauchy problem, respectively) converges to zero as $t \to \infty$.
It is shown that for positive semigroups stability is determined fairly well by spectral properties of the generator.
In the second section we describe conditions ensuring convergence of the semigroup (as $t \to \infty$) to an equilibrium point or to a periodic solution.
Again we are interested in spectral conditions ensuring such a behavior.
In the final section a series of examples is discussed in more detail.
In particular we consider semigroups related to retarded equations and discuss existence of solutions, spectral properties and asymptotic behavior.
\index{asymptotic behavior}
Most of the examples are motivated by biological models.

\section{Stability of Positive Semigroups on Banach Lattices}\label{sec:c4-1} 
\index{stability|(}
%% --
\hspace{1cm}{\Large by Günther Greiner and Frank Neubrander}
\vspace{.5cm}
\newline
%% --
In Section~1 of Part-II, Chapter 4 we have seen that the growth bound of a positive semigroup on spaces $C_{0}(X)$ coincides with the spectral bound of the generator $A$, which is --- for positive semigroups --- an element of the spectrum of\/ $A$.
Now, using the results of Part-I, Chapter 3,  Part-I, Chapter 4,  Part-II, Chapter 4,  Section~1 and Part-III, Chapter 3,  it can be shown that this is valid for positive semigroups on AM- , AL- and Hilbert spaces.

\begin{theorem}\label{thm:c4-1.1}
Let $A$ be the generator of a positive semigroup $(T(t))_{t \geq 0}$ on a Banach lattice $E$ such that $s(A) > -\infty$.
Each of the subsequent conditions implies
%% --
\begin{equation*}\label{eq:c4-1.0-kgk}
s(A) = \omega_{1}(A) = \omega_{0}(A) \in \sigma(A) .
\end{equation*}
%% --
\begin{enumerate}[\upshape (i)] 
    \item \label{thm:c4-1.1-1}
    Either $E$ is an AM-space or an $L^{2}$-space or an $L^{1}$-space.
    
    \item \label{thm:c4-1.1-2}
    There exist $\tau > 0$, $h \in E_{+}$ such that $T(\tau)E \subset E_{h}$.
    
    \item \label{thm:c4-1.1-3}
    There exist $\tau > 0$, $\phi \in E'_{+}$ such that $\|T(\tau)f\| \leq \langle f,\phi\rangle$ for all $f \in E_{+}$.
\end{enumerate}
\end{theorem}
%% --
\begin{proof}
We know that $s(A) \leq \omega_{1}(A) \leq \omega_{0}(A)$ (see Part-I, Chapter 4,  Corollary 1.5) and $s(A) \in \sigma(A)$ (see Corollary~\ref{cor:c3-1.4} ).
Thus we have to show $s(A) = \omega_{0}(A)$.

\begin{enumerate}[\upshape (i), wide, labelindent=.5em]
\item 
%\ref{thm:c4-1.1-1} 
For AM-spaces the proof given in Section 1 of Part-II, Chapter 4 works (cf.\ Part-II, Remark~\ref{rem:b4-1.5}.).
Since for positive semigroups we always have $\|R(\lambda,A)\| \leq \|R(\Re(\lambda), A)\|$ ($\Re(\lambda) > s(A)$) (see Part-III, Chapter 3,  Corollar~1.3), the assertion for $L^{2}$-spaces follows from Part-I, Chapter 3,  Corollary~7.11.
If\/ $E$ is an $L^{1}$-space the assumptions of  \ref{thm:c4-1.1-3} are satisfied.

\item 
%\ref{thm:c4-1.1-2} 
We identify $E_{h}$ according to the \emph{Kakutani-Krein Theorem} with a space $C(K)$, $K$ compact.
Considering $T(\tau)$ as operator from $E$ into $C(K)$ we denote it by $T_{0}$.
Then $T_{0}$ is positive hence continuous (see \textcite[Theorem~5.3]{schaefer:1974}).

Let $j \colon C(K) = E_{h} \to E$ be the canonical inclusion.
The spectral radii of\/ $T(\tau) = j \circ T_{0}$ and $T_{0} \circ j$ coincide and are given by $\rho \coloneq  \exp(\tau \omega_{0}(A))$.
By the Krein-Rutman Theorem (cf.\ the Corollary to Theorem~2.6 in the Appendix of \textcite{schaefer:1966}) there exists $0 < \mu \in C(K)'$ such that $(T_{0} \circ j)'\mu = \rho \cdot \mu$.
Then $\phi \coloneq  T_{0}'\mu$ is an eigenvector of\/ $(j \circ T_{0})'$ with eigenvalue $\rho$.
Thus $\rho \in R\sigma(T(\tau))$ and hence $s(A) \geq \omega_{0}(A)$ by Part-I, Chapter 3,  Theorem~6.2.

\item
%\ref{thm:c4-1.1-3} 
For $\alpha > s(A)$, $r > \tau$, $f \in E_{+}$ we have
%% --
\begin{align*}
\int_{0}^{r} \eu^{-\alpha s}\|T(s)f\| \, \ds &= \int_{0}^{\tau} \eu^{-\alpha s}\|T(s)f\| \, \ds + \eu^{-\alpha \tau}\int_{0}^{r-\tau} \eu^{-\alpha s}\|T(\tau+s)f\| \, \ds \\
&\leq \int_{0}^{\tau} \eu^{-\alpha s}\|T(s)f\| \, \ds + \eu^{-\alpha \tau}\int_{0}^{r-\tau} \eu^{-\alpha s}\langle T(s)f,\phi \rangle \, \ds \\
&\leq \int_{0}^{\tau} \eu^{-\alpha s}\|T(s)f\| \, \ds + \|R(\alpha,A)f\|
\end{align*}
%% --
(the last inequality follows from Theorem~\ref{thm:c3-1.2}).
\index{Datko's theorem}
Now Datko's theorem (Part-I, Chapter 4,  Theorem~1.11) implies $\omega_{0}(A) < \alpha$.
\end{enumerate}
\end{proof}
For $L^{p}$-spaces, $p \neq 1$, $2$, $\infty$, it is not known whether spectral- and growth bound of an arbitrary positive semigroup coincide.
Using interpolation techniques and  Theorem~\ref{thm:c4-1.1}  one can treat some special cases.
Before doing this we have to recall some facts on interpolation.
For details we refer to 
\textcite[VI.10]{dunfordschwartz:1958} or \textcite[IX.4.]{reedsimon:1975}.

Let $(X,\Sigma,\mu)$ be a $\sigma$-finite measure space, $1 \leq p < q < \infty$ and suppose that 
%% --
\[
T_{0} \colon L^{p}(\mu) \cap L^{q}(\mu) \to L^{p}(\mu) \cap L^{q}(\mu)
\]
%% --
is a linear operator which satisfies 
%% --
\[
\|T_{0}f\|_{p} \leq C_{p}\|f\|_{p} \ \text{ and } \ \|T_{0}f\|_{q} \leq C_{q}\|f\|_{q} 
\ \text{for every } \ f\in L^{\,p}(\mu) \cap L^{q}(\mu)\,.
\]
%% --
Then for every $r \in [p,q]$, $T_{0}$ has a (unique) continuous extension $T_{r} \colon L^{r}(\mu) \to L^{r}(\mu)$.
Moreover,
%% --
\begin{equation}\label{eq:c4-1.1}
u \mapsto \log\left\|T_{1/u}\right\| \text{ is a convex function on the interval } \left[\frac{1}{q},\frac{1}{p}\right].
\end{equation}
%% --
Applying this result to the powers $T_{r}^{n}$ $(n \in\N)$ and using the fact that the pointwise limit of convex functions is convex, we obtain that the logarithm of the spectral radius is convex, \ie
%% --
\begin{equation}\label{eq:c4-1.2}
u \mapsto \log(r(T_{1/u})) = \lim_{n\to\infty} \frac{1}{n}\log\left\|T_{1/u}^{n}\right\| \text{ is convex on }  \left[\frac{1}{q},\frac{1}{p}\right].
\end{equation}
%% --
In the following we suppose that for every $r \in [p,q]$ we have a strongly continuous semigroup $(T_{r}(t))_{t\geq 0}$ on $L^{r}(\mu)$ such that
%% --
\begin{equation}\label{eq:c4-1.3}
T_{r}(t)_{|L^{r} \cap L^{s}} = T_{s}(t)_{|L^{r} \cap L^{s}} \text{ for all } r,s \in [p,q], t \geq 0.
\end{equation}
%% --
Let $A_{r}$ be the generator of\/ $(T_{r}(t))$, $\omega_{0}(r)$ its growth bound and $s(r)$ the spectral bound of\/ $A_{r}$.
In this situation we have the following corollary of Theorem~\ref{thm:c4-1.1}.

\begin{corollary}\label{cor:c4-1.2}
Suppose that the semigroups $(T_{r}(t))_{t\geq 0}$ are positive.
\begin{enumerate}[\upshape(i)]
\item \label{cor:c4-1.2-1}
In case $p < 2 < q$ and $\omega_{0}(r)$ independent of\/ $r \in [p,q]$, one has $s(r) = \omega_{0}(r)$ for all $r \in [p,q]$.
%%--
\item \label{cor:c4-1.2-2}
If\/ $p = 1$, $q \geq 2$ and $s(r)$ independent of\/ $r \in [p,q]$, then $s(r) = \omega_{0}(r)$ for $r \in [1,2]$.
\end{enumerate}
\end{corollary}
%% --
\begin{proof}
Once it is shown that both functions $u \mapsto s(1/u)$ and $u \mapsto \omega_{0}(1/u)$ are convex on $[\frac{1}{q},\frac{1}{p}]$, the assertion follows from  Theorem~\ref{thm:c4-1.1} and the relation $s(r) \leq \omega_{0}(r)$ for every $r$.

Since $\omega_{0}(u) = \log r(T_{u}(1))$ (see Part-I, Chapter 3, \,(1.4)), Eq.~\eqref{eq:c4-1.2} implies that $u \mapsto \omega_0(1/u)$ is a convex function.
By Theorem~\ref{thm:c3-1.1} we have $r(R(k,A_{u})) = (k-s(u))^{-1}$ for $k \in \N$ sufficiently large.
The assumption Eq.~\eqref{eq:c4-1.3} implies that 
%% --
\[
R(\lambda,A_{r})_{|L^{r} \cap L^{s}} = R(\lambda,A_{s})_{|L^{r} \cap L^{s}}
\]
%% --
for $r$, $s \in [p,q]$ and $\lambda \in \C$ with $\Re(\lambda)$ large enough.
Hence by Eq.~\eqref{eq:c4-1.2}, the mapping  $u \mapsto \log [r(R(k,A_{1/u}))]$ is a convex function for large $k \in \N$.
We have
%% --
\begin{align*}
\log  \left[(1 - \frac{1}{k}s(1/u))^{-k}\right] &= k \cdot \log k + k \cdot \log [k - s(1/u)]^{-1} \\
&= k \cdot \log k + k \cdot \log [r(R(k,A_{1/u}))]^{-1},
\end{align*}
%% --
hence all the functions $u \mapsto \log[(1 - \frac{1}{k}s(1/u))^{-k}]$, $k \in \N$, are convex.
It follows that $u \mapsto s(1/u) = \lim_{k\to\infty}(\log [(1 - \frac{1}{k}s(1/u))^{-k}])$ is convex as well.
\end{proof}
%% --
\index{Schr\"odinger operator}
\index{operator!Schr\"odinger}
\index{operator!Laplacian|(}
One can apply the corollary to Schrödinger operators on the spaces $L^{p}(\R^{n})$, \ie operators $A = \Delta + V$ where $\Delta$ is the Laplacian and $V$ is a multiplication operator, see  \textcite{simon:1982} for details.
\index{operator!multiplication}
In Theorem B.5.1 (l.c.) it is shown that for certain potentials $V$ the growth bound is independent of\/ $p \in [1,\infty)$.
Thus the assumptions of \ref{cor:c4-1.2-1} are satisfied.
Part \ref{cor:c4-1.2-2} of the Corollary can be applied if\/ $q > 2$ and if\/ $A_{1}$ has compact resolvent.
Then all operators $A_{r}$, $1 \leq r < q$, have compact resolvent and therefore their spectra coincide.
In particular, $s(A_{r})$ is independent of\/ $r \in [1,q)$.

As shown in Part-I, Chapter 4,  Example~1.2\,(ii), the equality $s(A) = \omega_{0}(A)$ may not hold for positive semigroups on arbitrary Banach lattices.
\index{abstract Cauchy problem}
However, the knowledge of\/ $s(A)$ is still sufficient to determine the growth bound $\omega_{1}(A)$ of the strong solutions of the abstract Cauchy problem.
In fact, combining Theorems 1.1 and 1.2 of Part-3, Chapter 3  with Theorem 1.4 of Part-I, Chapter 4 we obtain the following fundamental result for the stability of positive semigroups.
%% --
\begin{theorem}\label{thm:c4-1.3}
Let $A$ be the generator of a positive semigroup $(T(t))_{t\geq 0}$ on a Banach lattice.
Then $s(A) = \omega_{1}(A) \in \sigma(A)$.
\end{theorem}
%% --
Recalling the definition of\/ $\omega_{1}(A)$ (see Part-I, Chapter 4,  Definition~1.1) and the fact that $s(A)$ is always an element of\/ $\sigma(A)$, we can reformulate the statement of Theorem~\ref{thm:c4-1.3} as follows.
%% --
\begin{corollary}\label{cor:c4-1.4}
%
\index{stability!exponentially stable}
Let $(T(t))_{t\geq 0}$ be a positive semigroup on a (real or complex) Banach lattice with generator $A$.
Each of the following conditions implies that the solutions of the abstract Cauchy problem are exponentially stable, \ie  there is $\delta > 0$ such that $\lim_{t \to \infty} \eu^{\delta t}T(t)f = 0$ for every $f \in D(A)$.
%% --
\begin{enumerate}[\upshape (i)]
    \item \label{cor:c4-1.4-1}
    $\lambda - A$ is invertible for every $\lambda \geq 0$;
    %%--
    \item \label{cor:c4-1.4-2}
    $A$ is invertible and $A^{-1} \leq 0$.
    \end{enumerate}
\end{corollary}
%% --
\begin{proof}
In case of a real Banach lattice we consider the complexification (see Section 7 of Part-3, Chapter 1 ).
Note that both, the hypotheses and the statement remain preserved.
Since $s(A) \in \sigma(A)$ assertion \ref{cor:c4-1.4-1} implies $s(A) < 0$.
If \ref{cor:c4-1.4-2} is satisfied then $R(0,A) \geq 0$, hence $s(A) < 0$ by Theorem~\ref{thm:c3-1.1} \,(ii)).
It follows from Theorem~\ref{thm:c4-1.3} that $\sup\{\omega(f) \colon f \in D(A)\} = \omega_{1}(A) < 0$.
\end{proof}
%% --
In the following we give a spectral characterization of stability for eventually norm-continuous positive semigroups.
An important tool in the proof is the following result on power bounded operators due to \textcite{katznelsontzafriri:1984}.
%% --
\begin{equation}\label{eq:c4-1.4}
\begin{minipage}{.75\linewidth}
Let $R$ be a linear operator on a Banach space such that $\sup\{ \|R^{n}\| \colon n \in \N \} < \infty$. 
Then one has $\sigma(R) \cap \Gamma \subseteq \{1\}$ if and only if 
$ \lim_{n \to \infty} \|R^{n} - R^{n+1}\| = 0$.
\end{minipage}
\end{equation}
%% --
\begin{theorem}\label{thm:c4-1.5}%
\index{semigroup!eventually norm-continuous semigroup}
%% --
Let $(T(t))_{t\geq 0}$ be a positive semigroup on a Banach lattice $E$ which is bounded and eventually norm-continuous.
The following two assertions are equivalent.
\index{stability!uniformly stable}
\begin{enumerate}[\upshape (a)]
    \item \label{thm:c4-1.5-1}
    $(T(t))_{t\geq 0}$ is uniformly stable;
    
    \item \label{thm:c4-1.5-2}
    $0 \notin R\sigma(A)$ (\ie  $\Kern{A'} 
    = \{0\}$).
\end{enumerate}
%% --
In case $E$ is reflexive \ref{thm:c4-1.5-1} and \ref{thm:c4-1.5-2} are equivalent to
%% --
\begin{enumerate}[\upshape (a), resume]

\item  \label{thm:c4-1.5-3}
$0 \notin \text{P}\sigma(A)$ (\ie  $\Kern{A} = \{0\}$).
\end{enumerate}
\end{theorem}
%% --
\begin{proof}
\ref{thm:c4-1.5-1} $\Rightarrow$ \ref{thm:c4-1.5-2} was proven in Part-I, Chapter 4,  Theorem~1.12 in a more general setting.

\ref{thm:c4-1.5-2} $\Rightarrow$ \ref{thm:c4-1.5-1}:  In case $\omega_{0}(A) < 0$ one trivially has \ref{thm:c4-1.5-1}.
Therefore we can assume $\omega_{0}(A) = 0$.
By Corollary~2.13 and Proposition~2.9 of Part-III, Chapter 3,  we have $\sigma(A) \cap \im\R = \{0\}$.
Since the spectral mapping theorem holds (cf.\ Theorem~6.6 and Theorem~6.3 of Part-I, Chapter 3) we have
%% --
\begin{equation}\label{eq:c4-1.5}
\sigma(T(1)) \cap \Gamma = \{1\} \text{ and } 1 \notin R{\sigma}(T(1)).
\end{equation}
%% --
From \ref{eq:c4-1.4} it follows that $\lim_{n \to \infty} \|T(n) - T(n+1)\| = 0$ and therefore $\lim_{t \to \infty} \|T(t) - T(t+1)\| = 0$.
Thus given $g \in \Image{\Id - T(1)}$ then $g = f - T(1)f$ for some $f \in E$ hence 
%% --
\[
\|T(t)g\| = \|(T(t) - T(t+1))f\| \leq \|(T(t) - T(t+1))\| \cdot \|f\| \to 0.
\]
%% --
The second assertion of \ref{eq:c4-1.5} ensures that $\Image{\Id - T(1)}$ is dense in $E$.
Since the semigroup is bounded, we have $\lim_{t \to \infty} \|T(t)f\| = 0$ for every $f \in \overline{\Image{\Id - T(1)}} = E$, \ie  $(T(t))$ is uniformly stable.

\ref{thm:c4-1.5-1} $\Rightarrow$ \ref{thm:c4-1.5-3} is always true and follows from Part-I, Chapter 4,  Theorem~1.13.

\ref{thm:c4-1.5-3} $\Rightarrow$ \ref{thm:c4-1.5-2}:  
\index{adjoint semigroup}
\index{semigroup!adjoint}
The adjoint semigroup $(T(t)^*)_{t \geq 0}$ is eventually norm-continuous and boun\-ded and we have $R{\sigma}(A^*) = P{\sigma}(A^{**}) = P{\sigma}(A)$.
Thus the implication \enquote{\ref{thm:c4-1.5-2} $\Rightarrow$ \ref{thm:c4-1.5-1}} can be applied and we obtain that $(T(t)^*)_{t \geq 0}$ is stable.
Then Part-I, Chapter 4,  Theorem~1.13 yields $0 \notin P{\sigma}(A^*) = R{\sigma}(A)$.
\end{proof}
% --
As an application of Theorem~\ref{thm:c4-1.5}  we consider the Laplacian as generator on $L^{p}(\R^{n})$, $1 \leq p < \infty$, (see Part-I, Chapter 1,  2.8).
For $p = 1$ the constant functions are eigenvectors of the adjoint operator, hence $0 \in R{\sigma}(\Delta)$.
Thus the semigroup is not stable on $L^{1}(\R^{n})$.
On the other hand, $\Kern\Delta =\{ 0 \}$ for $ 1 \leq p < \infty$ (see below). 
Hence $\Delta$ generates a stable semigroup on $L^{p}(\R^{n})$ for $1 < p < \infty$.

That $\Kern\Delta = \{0\}$ can be deduced from the following two facts.
%% --
\begin{enumerate}[\upshape (i), wide, labelindent=.5em]
\item 
Since the semigroup consists of contractions and since the norm is strictly monotone on $E_{+}$, it follows that $\Kern\Delta$ is a sublattice.
Thus irreducibility of the semigroup (see Part-I, Chapter 1,  2.8 and Part-III, Chapter 3,  Example~3.4\,(i)) implies that $\dim \Kern\Delta \leq 1$.

\item 
The semigroup commutes with the translations on $\R^{n}$, hence $\Kern\Delta$ is invariant under translations.

\end{enumerate}
%% --
In the next results we give conditions on the range of the generator which ensure stability.
We begin with a generalization of Corollary \ref{cor:c4-1.4}\,\ref{cor:c4-1.4-2}.
%%
\begin{proposition}\label{prop:c4-1.6}
%
Let $A$ be the generator of a positive semigroup on a (real or complex) Banach lattice, $D(A)_{-} \coloneqq -(D(A) \cap E_{+})$.
Then 
%% --
\[ 
\text{$\omega_{1}(A) < 0 $ \quad if and only if \quad $E_{+} \subset \Image{A(D(A)_{-})}$.}
\]
\end{proposition}
%% --
\begin{proof}
If\/ $\omega_{1}(A) < 0$, then $s(A) < 0$ (Part-I, Chapter 4,  Corollary 1.5), hence $A^{-1} = -R(0,A) \leq 0$ by Theorem~\ref{thm:c3-1.1} .

If\/ $E_{+} \subset \Image{A(D(A)_{-})}$, then, for every $f \in E_{+}$, there exists $-g \in D(A)_{+}$ such that $Ag = -f$.
We have 
%% --
\[
0 \leq T(t)g = g + \int_{0}^{t} T(s)Ag \, \ds = g - \int_{0}^{t} T(s)f \, \ds,
\]
%% --
hence 
%% --
\[
0 \leq \int_{0}^{t} T(s)f \, \ds \leq g 
\quad \text{for every $t \geq 0$.}
\]
%% --
For $\alpha > 0$ we have 
%% --
\[
\int_{0}^{t} \eu^{-\alpha s}T(s)f \, \ds \leq \int_{0}^{t} T(s)f \, \ds \leq g .
\]
%% --
Using Theorem~\ref{thm:c3-1.2}  we conclude that $R(\alpha,A)f \leq g$ for $\alpha > \max\{0,s(A)\}$.

By the Uniform Boundedness Principle we know that $\{R(\alpha,A) \colon \alpha > \max\{0,s(A)\}\}$ is uniformly bounded.
Since $\omega_{1}(A) = s(A) \in \sigma(A)$ (see Theorem~\ref{thm:c4-1.3}) it follows that $\omega_{1}(A) < 0$.
\end{proof}
Next we show that weak uniform stability implies uniform stability provided that $E$ is weakly sequentially complete (see Chapter~\ref{sec:c1-5}) and $(\Image{A})_{+} \coloneqq A(D(A)) \cap E_{+}$ is a total subset of\/ $E$.
The left translations on $L^{2}(\R_{+})$ are stable.
Hence, by Part-I, Chapter 4, Remark 1.17(a), $\Image{A} = \{f \in L^{2}(\R_{+}) \colon \int_{0}^{\infty} f(x) \, \dx \text{ exists}\}$ and we see that $(\Image{A})_{+}$ is a total subset of\/ $L^{2}(\R_{+})$.
On the other hand, $(\Image{A})_{+} = \{0\}$ for the generator of the non stable, but weakly stable semigroup of left translations on $L^{2}(\R)$.

\begin{proposition}\label{prop:c4-1.7}
%
Let $A$ be the generator of a positive semigroup $(T(t))_{t\geq 0}$ on a weakly sequentially complete Banach lattice $E$, such that $(\Image{A})_{+}$ is total in $E$.
Then $(T(t))_{t\geq 0}$ is uniformly stable if and only if it is weakly uniformly stable.
\end{proposition}
%% --
\begin{proof}
If\/ $(T(t))_{t\geq 0}$ is weakly uniformly stable, then $(T(t))$ is bounded by the Uniform Boundedness Principle.
Using the weak version of Part-I, Chapter 4,  Theorem~1.14, 
$\int_{0}^{\infty} \langle T(t)f,\phi \rangle \, \dt$ exists for every $f \in (\Image{A})_{+}$ and $\phi \in E'_{+}$.
It follows that the net $(\int_{0}^{r} T(t)f \, \dt)_{r\geq 0}$ is weakly Cauchy.
Hence $\sigma(E',E)$-$\lim_{r \to \infty} \int_{0}^{r} T(t)f \, \dt$ exists for every $f \in (\Image{A})_{+}$.
Since the net is monotone, one obtains convergence in norm by Dini's Theorem (see \textcite [II. Theorem~5.9]{schaefer:1974}).
Now uniform stability follows from Part-I, Chapter 4,  Theorem 1.16.
\end{proof}
%% --
In Part-I, Chapter 4,  Theorem 1.13 we have seen that a generator $A$ of a stable semigroup satisfies necessarily $s(A) \leq 0$, $\Re(\lambda) < 0$ for all $\lambda \in P{\sigma}(A) \cup R{\sigma}(A)$ and from $\lambda R(\lambda,A)f = R(\lambda,A)Af + f$ we conclude that $\lim_{\lambda \to 0+} R(\lambda,A)g$ exists for all $g \in \Image{A}$. 
For positive semigroups similar properties are even sufficient for stability.

\begin{lemma}\label{lem:c4-1.8}
%
Let $A$ be the generator of a positive semigroup $(T(t))_{t\geq 0}$ on a Banach lattice $E$ with $s(A) \leq 0$.
Given $f \in E_{+}$ then $\lim_{\lambda \to 0+} R(\lambda,A)f$ exists if and only if\/ $\lim_{t \to \infty} \int_{0}^{t} T(s)f \, \ds$ exists.
\end{lemma}
%% --
\begin{proof}
In view of Theorem~\ref{thm:c3-1.2}  we have for $\phi \in E'_{+}$:
%% --
\begin{align*}
\lim_{t \to \infty}  \left\langle\int_{0}^{t} T(s)f \, \ds \,,\,\phi\right\rangle &= \sup_{t>0} \int_{0}^{t}\langle T(s)f,\phi\rangle \, \ds \\
&= \sup_{t>0}\sup_{\lambda>0} \int_{0}^{t}\eu^{-\lambda s}\langle T(s)f,\phi\rangle \, \ds \\
&= \sup_{\lambda>0}\sup_{t>0} \int_{0}^{t}\eu^{-\lambda s}\langle T(s)f,\phi\rangle \, \ds \\
&= \sup_{\lambda>0} \langle R(\lambda,A)f,\phi\rangle \\
&= \lim_{\lambda \downarrow 0} \langle R(\lambda,A)f,\phi\rangle.
\end{align*}
%% --
Thus either both limits exist with respect to $\sigma(E,E')$-topology or none.
Since both nets are monotonically increasing, the assertion follows from Dini's Theorem (see \textcite[II. Theorem~5.9]{schaefer:1974}).
\end{proof}

\begin{proposition}\label{prop:c4-1.9}
%
Let $A$ be the generator of a positive, bounded semigroup $(T(t))_{t \geq 0}$ on a Banach lattice $E$.
If there is a subset $D \subset E_{+}$ which is total in $E$ such that $\lim_{\lambda \to 0+} R(\lambda,A)f$ exists for every $f \in D$, then $(T(t))_{t \geq 0}$ is uniformly stable.
\index{stability!uniformly stable}
\end{proposition}
%% --
\begin{proof}
By Lemma \ref{lem:c4-1.8} $\int_{0}^{\infty} T(t)f \, \dt$ exists for every $f$ in the linear hull of\/ $D$.
But $D$ is total, $(T(t))_{t \geq 0}$ is bounded and hence, by Part-I, Chapter 4,  Theorem~1.16, uniformly stable.
\end{proof}

\begin{remark}\label{rem:c4-1.10}
If\/ $A$ is the generator of a positive semigroup, then for every $n \in \N$, 
%% --
\[
D(A^{n})_{+}
\quad \text{and} \quad 
D_{+} =  \left(\bigcap_{n=0}^{\infty} D(A^{n})\right)_{+}
\]
%% --
are total subsets of\/ $E$.
This follows from 
%% --
\begin{align*}
    f &\in D(A^{n}), \\
    f &= R(\lambda,A)^{n}g = R(\lambda,A)^{n}(g_{1} - g_{2}) = f_{1} - f_{2},
\end{align*}
%% --
where $f_{1}$, $f_{2} \in D(A^{n})_{+}$ and Theorem~1.43 in \textcite{davies:1980}.
\end{remark}
%% --
In the rest of this section we discuss the long term behavior of the solutions of the inhomogeneous equation
\index{Cauchy problem!inhomogeneous}
%% --
\begin{equation}\label{eq:c4-1.6}
\dot{u}(t) = Au(t) + F(t), \quad u(0) = u_{0} \in D(A)\,,
\end{equation}
%% --
where the forcing term $F(t)$ converges to some $f_{0} \in E$ as $t \to \infty$.
In case that $A$ generates a positive semigroup the assumption \emph{$\omega_{0}(A) < 0$}, which is needed to prove the next proposition for arbitrary generators (see \textcite[ Theorem~4.4.4]{pazy:1983}), can be replaced by the \emph{stability} of the semigroup.
We recall that some important generators as, for example, the Laplacian on $L^{p}(\R^{n})$, $1 < p < \infty$, generate positive, stable semigroups which are not uniformly exponentially stable.
Therefore, the weakening of the assumptions on $A$ mentioned above - \ie  replacing \emph{$\omega_{0}(A) < 0$} by \emph{positive and stable} - widens the class of equations \ref{eq:c4-1.6} for which the following stability result is applicable.
For additional results of this kind see \textcite{neubrander:1985b}.
\index{operator!Laplacian|)}

\begin{proposition}\label{prop:c4-1.11}
\index{semigroup!stable semigroup}
%% --
Let $A$ be the generator of a positive, stable semigroup $(T(t))_{t \geq 0}$ on a Banach lattice $E$.
Let $F(\cdot)$ be a locally integrable function from $\R_{+}$ into $E$.
If there are $G(\cdot) \in C_{0}(\R_{+},\R_{+})$, $f_{0} \in  \Image{A}$ and $g_{0} \in \Image{A_{+}}$ such that $|F(s) - f_{0}| \leq G(s)g_{0}$ for every $s \geq 0$, then every mild solution $u(\cdot)$ of Eq.~\eqref{eq:c4-1.6} converges as $t \to \infty$ and $\lim_{t \to \infty} u(t) = -h$ where $h \in D(A)$ with $Ah = -f_{0}$.
\end{proposition}
%% --
\begin{proof}
Recall that every solution of \ref{eq:c4-1.6} satisfies
%% --
\begin{equation}\label{eq:c4-1.7}
u(t) = T(t)f + \int_{0}^{t} T(t-s)f_{0} \, \ds + \int_{0}^{t} T(t-s)(F(s) - f_{0}) \, \ds.
\end{equation}
%% --
By the stability of the semigroup and $f \in D(A)$, the first term converges to zero as $t \to \infty$.
Since $f_{0} \in \Image{A}$, the second term converges to $h \coloneqq  \int_{0}^{\infty} T(s)f_{0} \, \ds \in \Image{A}$ (Part-I, Chapter 4,  Theorem 1.16) and $Ah = -f_{0}$.
Define $H(s) \coloneqq F(s) - f_{0}$, 
$ H_{+}(s) \coloneqq (H(s))_{+} $ 
and similarly $H_{-}(\cdot)$.

We have to show that $\int_{0}^{t} T(t-s)H_{\pm}(s) \, \ds \to 0$ as $t \to \infty$.
Again, the assumption $g_{0} \in \Image{A}$ is equivalent to the existence of\/ $\int_{0}^{\infty} T(t)g_{0} \, \dt$.

We choose a constant $M$  such that 
%% --
\begin{enumerate}[\upshape (i)]
    \item 
    $0 \leq H_{\pm}(s) \leq H_{+}(s) + H_{-}(s) = |H(s)| \leq G(s)g_{0} \leq Mg_{0}$,
\end{enumerate}
%% --
and a constant $t'$ such that
%% --
\begin{enumerate}[\upshape (i), resume]
    \item $ \left\|\int_{t'}^{\infty} T(s)g_{0} \, \ds\right\| \leq \frac{\epsilon}{2M}$ and  
$G(s) \leq \frac{\epsilon}{2} \,  \left\|\int_{0}^{\infty} T(s)g_{0} \, \ds\right\|$ for every $s \geq t'$.
\end{enumerate}
%% --
Then, for $t > 2t'$,
%% --
\begin{align*}
0 &\leq \int_{0}^{t} T(t)H_{\pm}(s) \, \ds \leq \int_{0}^{t} T(t)G(s)g_{0} \, \ds\\
&= \int_{0}^{t'} T(t)G(s)g_{0} \, \ds + \int_{t'}^{t} T(t)G(s)g_{0} \, \ds\\
&\leq M \int_{t-t'}^{\infty} T(t)g_{0} \, \ds + \epsilon/2 \,  \left\|\int_{0}^{\infty} T(t)g_{0} \, \ds\, \right\|^{-1} \int_{0}^{t-t'} T(t)g_{0} \, \ds\\
&\leq M \int_{t'}^{\infty} T(t)g_{0} \, \ds + \epsilon/2 \,  \left\|\int_{0}^{\infty} T(t)g_{0} \, \ds\, \right\|^{-1} \int_{0}^{\infty} T(t)g_{0} \, \ds.
\end{align*}
%% --

Hence $\left\|\int_{0}^{t} T(t)H_{\pm}(s) \, \ds\right\| \leq \epsilon$ for every $t > 2t'$.
\end{proof}

We conclude with a result similar to the previous proposition.
Instead of uniform stability we now require $s(A) < 0$ while the assumption on the forcing term is weaker than in Proposition~\ref{prop:c4-1.11}.

\begin{proposition}\label{prop:c4-1.12}
Let $(T(t))_{t \geq 0}$ be a positive semigroup with $s(A) < 0$.
Assume that the forcing term $F$ has values in $D(A)$, that it is continuous with respect to the graph norm and that $f_{0} \coloneq  \|\cdot\|_{A}$-$\lim_{t \to \infty} F(t)$ exists.
Then for every solution $u(\cdot)$ of Eq.~\eqref{eq:c4-1.6} we have $\lim_{t \to \infty} u(t) = -A^{-1}f_{0}$.
%%
\footnote{Note, that the assumptions imply that Eq.~\eqref{eq:c4-1.6} has a unique strong solution, see \textcite[ Theorem~4.2.4]{pazy:1983}.}
\end{proposition}
%% --
\begin{proof}
The solution of Eq.~\eqref{eq:c4-1.6} is given by
%% --
\begin{equation}\label{eq:c4-1.8}
u(t) = T(t)u_{0} + \int_{0}^{t} T(s)f_{0} \, \ds + \int_{0}^{t} T(s)(F(t-s)-f_{0}) \, \ds\,.
\end{equation}
%% --
The first term tends to zero by Corollary \ref{cor:c4-1.4},  
the second term to $R(0,A)f_{0} = -A^{-1}f_{0}$ by Theorem~\ref{thm:c3-1.2} .
By assumption we have $\lim_{s \to \infty}\|A(F(s)-f_{0})\| = 0$ and from Theorem~\ref{thm:c4-1.3} and Part-I, Chapter 4, \,(1.3) we deduce that $\|T(s)R(0,A)\| \leq M\eu^{-\epsilon s}$ for $s \geq 0$ and suitable constants $M \geq 1$, $\epsilon > 0$.
Thus for the third term we have
%% --
\begin{align*}
 \left\|\int_{0}^{t} T(s)(F(t-s)-f_{0}) \, \ds\,\right\| &\leq \int_{0}^{t} \|T(s)R(0,A)\| \|A(F(t-s)-f_{0})\| \, \ds \\
&= \int_{0}^{t/2} \ldots \, \ds + \int_{t/2}^{t} \ldots \, \ds.
\end{align*}
%% --
The first integral can be estimated by
%% --
\[
 \sup  \left\{\|A(F(s)-f_{0})\| \colon s \in 
 \left[ t/2 , t \right] \right\} 
\cdot \int_{0}^{\infty} M \cdot \eu^{-\epsilon s} \, \ds
\]
%% --
while the second integral can be estimated by 
%% --
\[
\sup\{\|A(F(s)-f_{0})\| \colon s \geq 0\}
\cdot \int_{t/2}^{t} M \cdot \eu^{- \epsilon s} \, \ds.
\]
%% -- 
It follows that the third term in Eq.~\eqref{eq:c4-1.8} tends to zero.
\end{proof}
\index{stability|)}
%% --
\clearpage
%% --
\section{Convergence of Positive Semigroups}\label{sec:c4-2}
%
\hspace{1cm}{\Large by Günther Greiner and Rainer Nagel}
\vspace{.5cm}
\newline
%% --
The considerations in this section are motivated by the following guideline.
%% --
\index{asymptotic behavior}
\begin{quote} 
The asymptotic behavior of a strongly continuous semigroup $(T(t))_{t \geq 0}$ is determined by the structure and location of the spectrum $\sigma(A)$ of the generator $A$.
%The asymptotic behavior of a strongly continuous semigroup $(T(t))_{t \geq 0}$ is determined by the (structure, location of the) spectrum $\sigma(A)$ of the generator $A$.
\end{quote}
%% --
Unfortunately, this principle does not hold in general, \eg  there are semigroups with spectral bound less than zero and growth bound greater than zero (see Part-I, Chapter 3,  Example 1.3 \& 1.4).
In order to prove results in the above direction we have to assume additional hypotheses on the semigroup.
Positivity may serve to this purpose.
For example, the norm convergence to zero, \ie  $\lim_{t \to \infty} \|T(t)\| = 0$ for a positive semigroup on certain Banach lattices, is characterized by the condition $s(A) < 0$ (see Theorem~\ref{thm:c4-1.1}).
Thus in this case the location of the spectrum determines the norm convergence of the semigroup.

Here we concentrate on the case $s(A) = 0$.
At first we observe that
$\lim_{t \to \infty} T(t)$ --- if it exists in some operator topology --- is always a projection $P$ onto the fixed space of\/ $(T(t))_{t \geq 0}$ which coincides with the kernel of\/ $A$.
In case $P = 0$ we have stability which was discussed in Section~1.
In this section we mainly consider the case $s(A) = 0 \in P{\sigma}(A)$ and show that the symmetric structure of the boundary spectrum of the generator of a positive semigroup yields interesting results.
\index{spectrum!boundary} 

\index{semigroup!quasi-compact semigroup}
We begin our discussion by considering quasi-compact semigroups.
Using the general results presented in Section 2 of Part-II, Chapter 4 and the spectral theoretical result of Part-III, Chapter 3 we obtain the following.

\begin{theorem}\label{thm:c4-2.1}
Let $(T(t))_{t \geq 0}$ be a positive semigroup on a Banach lattice $E$ which is bounded, quasi-compact and has spectral bound zero.
\index{spectral bound $s(A)$}
Then there exists a positive projection $P$ of finite rank and suitable constants $\delta > 0$, $M \geq 1$ such that
%% --
\begin{equation}\label{eq:c4-2.1}
\|T(t) - P\| \leq M \cdot \eu^{-\delta t} \text{ for all } t \geq 0.
\end{equation}
%% --
\end{theorem}
%% --
\begin{proof}
By Theorem 2.10 of Part-II, Chapter 4 the set $\{\lambda \in \sigma(A) \colon \Re(\lambda) = 0\}$ is finite and by Theorem 2.10 of Part-3, Chapter 3  even imaginary additively cyclic.
Thus it contains only the value $s(A) = 0$.
Then by Part-II, Chapter 4, \,(2.5) we have
%% --
\begin{equation*}\label{eq:c4-2.1-kgk}
T(t) = \sum_{j=0}^{k-1} \frac{1}{j!} \cdot t^{j}A^{j}\circ P + R(t) \quad (t \geq 0),
\end{equation*}
%% --
where $P$ is the residue of\/ $R(\cdot,A)$ at $0$, $k$ is the pole order and $\|R(t)\| \leq M \cdot \eu^{-\delta t}$ for suitable constants $\delta > 0$, $M \geq 1$.
Since we assumed that $(T(t))_{t \geq 0}$ is bounded, the pole order $k$ has to be $1$.
\end{proof}
%% --
Before discussing a concrete example we formulate some remarks related to Theorem~\ref{thm:c4-2.1}.
%% --
\begin{remarks}\label{rem:c4-2.2}
\begin{enumerate}[\upshape (i), wide, labelindent=.5em]

\item \label{rem:c4-2.2-1}
If one has a positive semigroup\, $\TT = (T(t))_{t \geq 0}$\,  satisfying
%% --
\[
\omega_{ess}(\TT) <  \omega_0(\TT), 
\]
%% -- 
then the rescaled semigroup with 
%% --
\[
\widehat{T}(t) \coloneqq \exp(-\omega_0(\TT)t)T(t)
\]
%% --
is quasi-compact and has spectral bound zero.

In order to apply Theorem~\ref{thm:c4-2.1} we still need the boundedness of\/ $(\widehat{T}(t))_{t\geq 0}$ (see the following remarks).

\item \label{rem:c4-2.2-2}
Without assuming boundedness of the semigroup, one can conclude that
%% --
\[
T(t) - \sum_{j=0}^{k-1} \frac{1}{j!} \cdot t^{j}A^{j}{\circ}P
\]
%% --
tends to zero exponentially.

\item \label{rem:c4-2.2-3}
In the proof of Theorem~\ref{thm:c4-2.1} we saw that a quasi-compact semigroup of positive operators having spectral bound zero is bounded if and only if the pole order at zero is one.
This is automatically true whenever there exists a fixed vector which is a quasi-interior point of\/ $E_{+}$.
\index{quasi-interior point}
Indeed, if\/ $k$ is the order of the pole at $s(A) = 0$, then we have 
%% --
\[
0 \neq A^{k-1}P = \lim_{\lambda \to 0} \lambda^{k}R(\lambda,A).
\]
%% --
Thus $A^{k-1}P$ is a positive operator.

Assuming $k > 1$ and denoting the quasi-interior fixed vector by $u$, we have $Au = 0$ hence $A^{k-1}Pu = PA^{k-1}u = 0$.
Since $A^{k-1}P$ is positive, it vanishes on the principal ideal generated by $u$.
Since this ideal is dense, we obtain $A^{k-1}P = 0$ which is a contradiction.

\item  \label{rem:c4-2.2-4}
\index{semigroup!irreducible semigroup|(}
If\/ $\TT = (T(t))_{t \geq 0}$ is an irreducible semigroup with $s(A) = 0$, then quasi-compactness implies boundedness of\/ $\TT$ (This follows from \ref{rem:c4-2.2-3} and Proposition~\ref{prop:c3-3.5} ).
Moreover, in this case the projection $P$ has the form $P = \phi \otimes h$ where $h$ 
is a quasi-interior point of\/ $E_{+}$ and $\phi$ is a strictly positive linear form on $E$.
This also is a consequence of Proposition~\ref{prop:c3-3.5} .

\item  \label{rem:c4-2.2-5}
If\/ $\TT = (T(t))_{t \geq 0}$ is irreducible and eventually compact, then the rescaled semigroup $(\exp(-\omega_0(\TT)t)T(t))$ satisfies the assumptions of Theorem~\ref{thm:c4-2.1}.

Indeed, by Theorem~\ref{thm:c3-3.7} we know that $\omega_0(\TT) > -\infty$, while $\omega_{ess}(\TT) = -\infty$.
It follows that the rescaled semigroup is quasi-compact, hence \ref{rem:c4-2.2-4} is applicable.
\end{enumerate}
\end{remarks}

\index{population semigroup}
\index{population!age-structured}
\index{equation!population}
The following example has a biological background, and the semigroup considered describes the time evolution of an age-structured population.
For more details we refer to \textcite{greiner:1984a} or \textcite{webb:1984}.

\begin{example}\label{ex:c4-2.3}
On the Banach lattice $E = L^{1}( \left[0,\infty\right))$ we consider the operator $A$ defined by
%% --
\begin{equation}\label{eq:c4-2.2}
\begin{split}
	Af &:= -f' - \mu f \quad \text{with domain}\\ 
	D(A) &:= \left\{ \vphantom{\int_{0}^{\infty}}f \in E \colon f \text{ absolutely continuous}, f' \in E  \right. \\
	 \qquad & \left. \text{and } f(0) = \int_{0}^{\infty} \beta(a) f(a)  \diff{a} \vphantom{\}} \right\}.
\end{split}
\end{equation}
%% --
Here we assume that $\mu$ and $\beta$ are positive, measurable, bounded functions on $[0,\infty)$.
One can show that $A$ generates a strongly continuous semigroup $\TT$ of positive operators.
Assuming that $\mu(\infty) \coloneqq \lim_{a \to \infty}\mu(a)$ exists, we obtain $\omega_{ess}(\TT) = -\mu(\infty)$.
We suppose in addition that $\beta$ and $\mu$ satisfy
%% --
\begin{equation}\label{eq:c4-2.3}
\int_{0}^{\infty} \beta(a)(\exp(-\int_{0}^{a} \mu(x) \, \dx))\, \diff{a} = 1 \quad \text{and} \quad \mu(\infty) > 0.
\end{equation}
%% --
The function $h$ with $h(a) \coloneqq \exp(-\int_{0}^{a} \mu(s) \, \ds)$ is differentiable, $h \in E$ and $h' = -\mu h$.
Moreover, Eq.~Eq.~\eqref{eq:c4-2.3} implies 
%
\[
	 \int_{0}^{\infty} \beta(a)h(a) \, \diff{a} = 1 = h(0) .
\]
%
Thus $h \in D(A)$ and $Ah = 0$.
It follows that $s(A) = 0$.
Indeed, since $s(A)$ is a pole of the resolvent, there exists a positive eigenvector $w$ of\/ $A'$ corresponding to $s(A)$.
Since $h$ is strictly positive, we have $\langle h,w \rangle > 0$. 
Hence 
%% --
\[
s(A)\langle h,w \rangle = \langle h,A'w \rangle = \langle Ah,w \rangle = 0
\]
%% --
which implies $s(A) = 0$.

Consequently the semigroup generated by $A$ satisfies all the assumptions of Theorem~\ref{thm:c4-2.1} provided that $\mu$ and $\beta$ satisfy Eq.~\eqref{eq:c4-2.3} (the boundedness of the semigroup follows from Remark \ref{rem:c4-2.2}\,\ref{rem:c4-2.2-3}).

It is not difficult to see that (up to a constant) $h$ is the unique eigenfunction of\/ $A$ corresponding to $0$.
Thus the projection $P$ has the form $P = v \otimes h$ for a suitable positive $v \in L^{\infty}(\left[0,\infty\right))$.
For more general generators of the type Eq.~\eqref{eq:c4-2.2} we refer to Chapter~\ref{sec:c4-3}.
\end{example}
Clearly, quasi-compactness was essential in the above example as well as in Theorem~\ref{thm:c4-2.1}.
For spaces $C_{0}(X)$ we proved in 
\textbf{Part-II, Theorem~\ref{thm:b4-2.12} ??} 
that Doeblin's condition is sufficient for quasi-compactness.
Actually this is true in $L^{p}$-spaces with $1 < p < \infty$ as well.
We quote the result from \textcite{lotz:1986}.

\begin{proposition}\label{prop:c4-2.4}
%
\index{Doeblin's condition}
%
Let $(T(t))_{t \geq 0}$ be a bounded positive semigroup on $E = L^{p}(\mu)$, $1 < p < \infty$.
Assume that there exist $t_{0} \geq 0$, $\phi \in E'_{+}$,\, $b < 1$ such that
%% --
\begin{equation}\label{eq:c4-2.4}
\|T(t_{0})f\| \leq \langle f,\phi \rangle + b\|f\| \quad \text{for all} \quad f \geq 0.
\end{equation}
%% --
Then $(T(t))_{t \geq 0}$ is quasi-compact.
\end{proposition}
\index{semigroup!eventually norm-continuous semigroup|(}
In the following result we replace quasi-compactness by eventual norm-continuity of the semigroup.

\begin{theorem} \label{thm:c4-2.5}
%
Let $\TT = (T(t))_{t \geq 0}$ be a bounded, eventually norm-continuous positive semigroup with generator $A$ on a reflexive Banach lattice $E$.
Then $Pf \coloneqq  \lim_{t \to \infty} T(t)f$ exists for every $f \in E$, and
$P$ is a positive projection onto the fixed space $\Fix{\TT} = \Kern{A}$. 
\end{theorem}
%% --
\begin{proof}
In view of Theorem~\ref{thm:c4-1.5} it suffices to consider the case $s(A) = 0 \in P{\sigma}(A)$.
We define $F \coloneqq \{f \in E \colon \lim_{t \to \infty} T(t)f \text{ exists}\}$.
Then $F$ is closed since $(T(t))_{t \geq 0}$ is bounded and obviously $\Kern{A} \subset F$.
Since $\sigma(A) \cap \im\R$ is cyclic and bounded (see Theorem~\ref{thm:c3-2.10} and Part-I, Chapter 2,  Theorem 1.20 resp.), we have $\sigma(A) \cap \im\R = \{0\}$.
Since the spectral mapping theorem holds (cf.\ Part-I, Chapter 3,  Theorem~6.6), we conclude $\sigma(T(t)) \cap \Gamma = \{1\}$ for all $t \geq 0$.
Then (\refeq{eq:c4-1.4}) implies $\lim_{n \to \infty} \|T(n) - T(n+1)\| = 0$, hence $\lim_{t \to \infty}\|T(t) - T(t+1)\| = 0$.
Take $f = g - T(1)g$.
Then 
%
\[
	\|T(t)f\| = \|T(t)g - T(t+1)g\| \leq \|T(t) - T(t+1)\| \cdot \|g\|
\]
%
implies $\lim_{t \to \infty} T(t)f = 0$.
Thus
%% --
\begin{equation}\label{eq:c4-2.5}
\lim_{t \to \infty} T(t)f = 0\, \text{ for every }\, f \in \Image{\Id - T(1)}.
\end{equation}
%% --
That is, $\mathrm{im}(\Id - T(1)) \subset F$.
Since $\Kern{A} = \bigcap_{t \geq 0} \Kern{\Id - T(t)} = \Kern{\Id - T(1)}$ (cf.\ Part-I, Chapter 3,  Corollary~6.4), we have $\Image{\Id - T(1)} + \Kern{\Id - T(1)} \subset F$.
Since power bounded operators on a reflexive Banach space are mean ergodic (\eg  see \textcite[Chapter~2, Theorem~1.2]{krengel:1985}), we obtain that $\Image{\Id - T(1)} + \Kern{\Id - T(1)}$ is dense in $E$, hence $F = E$.
\end{proof}
%% --
Strong convergence of the semigroup $\TT = (T(t))_{t \geq 0}$ implies strong convergence of the Césaro means $C(t)f \coloneqq  \frac{1}{t}\int_{0}^{t} T(s)f \, \ds$, $f \in E$ which (by definition) is \emph{mean ergodicity} of the semigroup $\TT$ (see \textcite[Chap.5.1]{davies:1980}).
\index{mean ergodicity}
\index{semigroup!mean ergodic semigroup}
On the other hand, an inspection of the proof of Theorem~\ref{thm:c4-2.5} shows that reflexivity of the underlying space can be replaced by the assumption that $\TT$ is a mean ergodic semigroup.

This remark also shows where to look for examples of semigroups not converging as $t \to \infty$.

Consider the positive contraction $R$ defined by $(Rf)(x) \coloneqq f(x+1)$ on $E = L^{1}(\R)$.
Then $T(t) \coloneqq \eu^{t(R-\Id)}$ defines a positive norm-continuous semigroup on $E$.
Since $\Kern{R - \Id} = \Fix{R} = \{0\}$, but 
\[
\|T(t)f\| = \eu^{-t} \left\|\sum_{n=0}^{\infty} \frac{t^n}{n!}R^{n}f\right\| = 
\eu^{-t}\sum_{n=0}^{\infty} \frac{t^n}{n!}\|R^{n}f\|= \|f\| > 0
\] 
for every $0 < f \in E$, we see that $\lim_{t\to\infty} T(t)$ does not exist for the strong operator topology.

Finally we remark that in Theorem~\ref{thm:c4-2.5} \enquote{eventual norm continuity} is crucial as well.
This can be seen by considering the translation (semi-) groups on $L^{p}(\R)$.

In the next few results we study semigroups which are not necessarily eventually norm-continuous, but restrict our attention to positive semigroups on $L^{p}$-spaces, for $1 \leq p < \infty$.
The essential tool will be the following \emph{0-2 Law} which we quote from \textcite[Theorem~3.7]{greiner:1982}.

If\/ $(X, \Sigma, \mu)$ is a measure space and $\TT = (T(t))_{t \geq 0}$ is a positive semigroup on $L^{p}(\mu)$, then a subset $C \in \Sigma$ is called \emph{$\TT$-invariant} if the principal ideal generated by the characteristic function $\1_{C}$ is $(T(t))$-invariant in the usual sense.

\begin{theorem}\label{thm:c4-2.6}
%
\index{zero-two law (0-2 Law)}
\index{semigroup!0-2 law}
%
Let $(T(t))_{t \geq 0}$ be a positive contraction semigroup on $L^{p}(\mu)$, $1 \leq p < \infty$, and assume that there exists a strictly positive fixed function $e \in \Kern{A}$. 
Then the following holds. 
\footnote{The regular operators on $L^p$-spaces form a vector lattice, thus $\sup\{T,S\},\ \inf\{T,S\}$ and $|T|$ exist.}
%%  
\begin{enumerate}[\upshape (i)]
\item \label{thm:c4-2.6-1}
For every $\tau > 0$ there exists a disjoint decomposition $X = X_{0} \cup X_{2}$ into $(T(t))$-invariant measurable subsets such that
%% --
\begin{enumerate}
    %%--	
    \item[(0)] \label{thm:c4-2.6-1.1} 
    $|T(t) - T(t+\tau)|e_{0} \to 0$ for $e_{0} = e \cdot \1_{X_{0}}$ as $t \to \infty$,
    %%-
    \item[(2)] \label{thm:c4-2.6-1.2} 
    $|T(t) - T(t+\tau)|e_{2} \geq 2e_{2}$ for $e_{2} = e \cdot \1_{X_{2}}$ and all $t \geq 0$.
    \end{enumerate}
%%--

\item \label{thm:c4-2.6-2} 
If the semigroup is irreducible, then for every $\tau > 0$ one has the alternative
\begin{enumerate}
    %%--    
    \item[(0)] \label{thm:c4-2.6-2.1} 
    $|T(t) - T(t+\tau)|e \to 0$ as $t \to \infty$ or
    %%--
    \item[(2)] \label{thm:c4-2.6-2.2} 
    $|T(t) - T(t+\tau)|e = 2e$ for all $t \geq 0$.
\end{enumerate}
\end{enumerate}
\end{theorem}
%% --
The \emph{0-2 Law} can be used in order to obtain results on convergence of positive semigroups.

\begin{corollary}\label{cor:c4-2.7}
%
Assume that --- in addition to the assumptions made in Theorem~\ref{thm:c4-2.6} --- $P{\sigma}(A) \cap \im\R = \{0\}$.
If we decompose $X = X_{0} \cup X_{2}$ for some $\tau > 0$ according to assertion \ref{thm:c4-2.6-1} , then $\lim_{t \to \infty} T(t)f$ exists for every $f \in L^{p}(\mu)$ vanishing $\mu$-a.e. on $X_{2}$.
\end{corollary}
%% --
\begin{proof}
From $T(t)e_{j} \leq T(t)e = e$, we obtain $T(t)e_{j} \leq e_{j}$ since $X_{0}$ and $X_{2}$ are $(T(t))$-invariant.
Then $T(t)e_{0} + T(t)e_{2} = T(t)e = e$ implies $T(t)e_{j} = e_{j}$ for $j=0,2$.
Thus we can assume $X = X_{0}$, $e = e_{0}$.
Given $g \in L^{p}(\mu)$ such that $|g| \leq e$, we have
$|T(t)(\Id - T(\tau))g| \leq |T(t) - T(t+\tau)|e \to 0$ for $t \to \infty$.
Since $\{g \in L^{p}(\mu) \colon |g| \leq e\}$ is a total subset of\/ $E$ ($e$ is strictly positive) and $(T(t))_{t \geq 0}$ is bounded, we conclude
%% --
\begin{equation}\label{eq:c4-2.6}
\lim_{t \to \infty} T(t)f = 0 \text{ for every } f \in \overline{\Image{\Id - T(\tau)}}.
\end{equation}
%% --
The assumption $P{\sigma}(A) \cap \im\R = \{0\}$ implies $\Kern{\Id - T(\tau)} = \Kern{A}$ (cf.\ Part-I, Chapter 3,  Corollar 6.4), hence we have convergence on $\Kern{\Id -T(\tau)}$.
Since $T(\tau)$ is a contraction on a reflexive Banach space, we have
\[
L^{p}(\mu) = \Kern{\Id - T(\tau)} \oplus \overline{\Image{\Id - T(\tau)}} \quad \text{(see \textcite [p.74]{krengel:1985})}
\]  
which finally proves the convergence on the whole space.
\end{proof}
%% --
Typical examples for which Theorem~\ref{thm:c4-2.6} and Corollary \ref{cor:c4-2.7}  can be applied, occur in the theory of stochastic processes (see also Part-II, Example~\ref{ex:b4-2.6}).
We briefly describe this situation and remind that in this context the sets $X_{0}$ and $X_{2}$ have a probabilistic meaning (see \textcite{greinernagel:1982} or the Supplement in \textcite{krengel:1985}).

\begin{example}\label{ex:c4-2.8}
%
\index{Markov transition function|(}
%% --
Let $X$ be a set and $\Sigma$ be a $\sigma$-algebra of subsets of\/ $X$.
We consider a Markov transition function $(P_{t})_{t \geq 0}$ on $(X,\Sigma)$, \ie  each $P_{t}$ is a real-valued function on $X \times \Sigma$ such that
\stepcounter{equation}
\begin{enumerate}[\upshape (i)]
%% \setcounter{enumi}{2}
\item \label{ex:c4-2.7-1}
$P_{t}(x,\cdot)$ is a probability measure for $x \in X$, $t > 0$\,,

%% --
\item \label{ex:c4-2.7-2}
$P_{t}(\cdot,C)$ is a measurable function for $C \in \Sigma$, $t > 0$\, ,

%% --
\item \label{ex:c4-2.7-3}
$P_{t+s}(x,C) = \int_{K} P_{s}(y,C)P_{t}(x,dy)$ for all $s,t>0$, $x \in K$, $C \in \Sigma$.
\end{enumerate}
%% --
We assume that $(P_{t})$ possesses an invariant probability measure $\mu$, \ie  we assume
%% --
\begin{enumerate}[\upshape (i)]
\setcounter{enumi}{3}
\item \label{ex:c4-2.7-4}
$\mu(C) = \int P_{t}(x,C) \, d\mu(x) \quad \text{for every} \quad C \in \Sigma$.
\end{enumerate}
%% --

Finally, we assume that the following continuity condition holds true.
%% --
\begin{enumerate}[\upshape (i)]
\setcounter{enumi}{4}
\item \label{ex:c4-2.7-5}
$\text{For every } C \in \Sigma \text{ one has } \lim_{t \to 0} P_{t}(x,C) = 1_{C}(x) \quad \mu\text{-a.e.} .$
\end{enumerate}
%% --
Given $h \in L^{1}(\mu)$ we define a measure\, $P_{t}h$\, on\, $\Sigma$\, by
\[
P_{t}h(C) \coloneqq \int P_{t}(x,C)h(x) \, \diff{\mu(x)}\,.
\]
In case $\mu(C) = 0$, then, by \ref{ex:c4-2.7-4}, $P_{t}(x,C) = 0$ $\mu$-a.e. on $X$, hence $P_{t}h(C) = 0$.
That is, $P_{t}h$ is absolutely continuous with respect to $\mu$.
By the Radon-Nikodym theorem, $P_{t}h$ has an integrable density with respect to $\mu$.
We define $T(t)h$ to be this density (which is unique as an element of\/ $L^{1}(\mu)$).
Thus for $h \in L^{1}(\mu)$, $C \in \Sigma$, we have
%% --
\begin{equation}\label{eq:c4-2.8}
\int_{C}(T(t)h)(x) \, d\mu(x) = \int P_{t}(x,C)h(x) \, d\mu(x) \quad \text{for all } C \in \Sigma.
\end{equation}
%% --
It is not difficult to see that $T(t)$ is a positive linear contraction on $L^{1}(\mu)$.
We have $T(t)'\1_{X} = \1_{X}$ and $T(t)\1_{X} = \1_{X}$ for all $t \geq 0$ and $T(t)T(s) = T(t+s)$ for $t$, $s \geq 0$.
This follows from \ref{ex:c4-2.7-1} , \ref{ex:c4-2.7-4} and \ref{ex:c4-2.7-3}, respectively.
Moreover \ref{ex:c4-2.7-5} implies strong continuity of the semigroup $(T(t))_{t \geq 0}$.
In fact, by \textcite[Proposition 1.23]{davies:1980}, we only have to show weak continuity at $t = 0$.
Since the characteristic functions are total in $L^{\infty}(\mu)$, this is true provided that $\lim_{t \to 0} \langle T(t)h,\1_{C}\rangle = \langle h,\1_{C}\rangle$ for $h \in L^{1}(\mu)$, $C \in \Sigma$.
Given $h \in L^{1}(\mu)$, then, by \ref{ex:c4-2.7-5}, 
\[
\lim_{t \to 0} P_{t}(x,C)h(x) = \1_{C}(x)h(x)\quad\mu-\text{a.e.}.
\]
By Lebesgue's Theorem 
\[
\langle T(t)h,\1_{C}\rangle = \int P_{t}(x,C)h(x) \, \diff{\mu(x)}
\]
tends to 
\[
\int \1_{C}(x)h(x) \, \diff{\mu(x)} = \langle h,\1_{C}\rangle \ \text{ as } \ t \to 0
\]
and we have weak, hence strong continuity.

\index{Markov transition function|)}
Therefore a Markov transition function satisfying all the assumptions of \ref{cor:c4-2.7} induces a strongly continuous semigroup on $L^{1}(\mu)$, and, by interpolation on $L^{p}(\mu)$, it satisfies the hypotheses of Theorem~\ref{thm:c4-2.6}.
\end{example}

In the following corollaries of Theorem~\ref{thm:c4-2.6}, we give criteria which ensure convergence on the whole space.
In view of Corollary \ref{cor:c4-2.7} it is enough to show $X_{2} = \emptyset$.
%% --
\begin{corollary}\label{cor:c4-2.9}
%
Let $(T(t))_{t \geq 0}$ be a positive semigroup of contractions on the Banach lattice $L^{1}(\mu)$ and assume that there exists a strictly positive eigenfunction $e \in \Kern{A}$.
If\/ $(T(t))_{t \geq 0}$ is eventually norm-continuous, then $\lim_{t \to \infty}T(t)f$ exists for every $f \in L^{1}(\mu)$.
\index{semigroup!eventually norm-continuous semigroup|)}
\end{corollary}
%% --
\begin{proof}
Since the semigroup is positive and eventually norm-continuous, its boundary spectrum is cyclic and bounded, \ie  we have $P{\sigma}(A) \cap \im\R = \{0\}$.
Moreover, there exist $t_{0} > 0$ and $\tau > 0$ such that $\|T(t_{0}) - T(t_{0}+\tau)\| < 1$.

For bounded linear operators $S \in \L{L^{1}}$ 
one has $\|S\| = \|\,|S|\,\|$ (see \textcite[IV, Theorem 1.5]{schaefer:1974}), hence $\|\,|T(t_{0}) - T(t_{0}+\tau)| f\| < \|f\|$ for every $f \in L^{1}(\mu)$, $f \neq 0$.
This shows that condition (2) of Theorem~\ref{thm:c4-2.6}\,(i) can be true only when $e_{2} = 0$, \ie  $X_{2} = \emptyset$.
\end{proof}

\begin{corollary}\label{cor:c4-2.10}
%
\index{irreducible semigroup}
%
Let $(T(t))_{t\geq 0}$ be an irreducible semigroup on $L^{p}(\mu)$ satisfying the assumptions of Theorem~\ref{thm:c4-2.6}.
If\/ $P{\sigma}(A) \cap \im\R = \{0\}$ and if there exist $0 \leq r < s$ such that $\inf\{T(r),T(s)\} > 0$, then there exists a strictly positive function $h \in L^{q}(\mu)$, $p^{-1}+q^{-1} = 1$, such that $\lim_{t \to \infty} T(t)f = \langle f,\,h \rangle e$ for every $f \in L^{p}(\mu)$.
\end{corollary}
%% --
\begin{proof}
Since $\inf\{T(r),T(s)\} > 0$, we have $(\inf\{T(r),T(s)\})e > 0$ for the strictly positive fixed vector $e$.
Since the regular operators on $L^{p}(\mu)$ form a vector lattice, it follows by \textcite[II.1.4, Formula (5) \& (5')]{schaefer:1974} that 
\[
|T(r) - T(s)|e = T(r)e + T(s)e - 2(\inf\{T(r),T(s)\})e < 2e\,.
\]
Consequently, the first alternative of Theorem~\ref{thm:c4-2.6} holds true with $\tau \coloneqq s-r$.
Equivalently, we have $X_{2} = \emptyset$ and by Corollary \ref{cor:c4-2.7} $Pf \coloneqq \lim_{t \to \infty} T(t)f$ exists for every $f \in L^{p}(\mu)$.
The limit $P$ is a positive projection satisfying $PT(t) = T(t)P = P$ for all $t \geq 0$.
It follows that $\Image{P} \subset \Kern{A}$ and $\Image{P'} \subset \Kern{A'}$.
Since Pe = e, hence $P \neq 0$, we conclude that $\Kern{A'}$ contains positive elements.
Now Proposition~\ref{prop:c3-3.5} (i)-(iii) implies that $P$ has the form $P = h \otimes e$ for a strictly positive function $h \in L^{q}(\mu) = (L^{p}(\mu))'$.
\end{proof}
%%% --
\index{operator!kernel}
In a last corollary we consider the case where one operator $T(t_{0})$ is a kernel operator, \ie  $T(t_{0})$ is induced by a $\mu \otimes \mu$-measurable kernel on $X \times X$.
The corollary is of particular interest for semigroups on spaces $\ell^{p}$, $1 \leq p < \infty$, where every positive operator is a kernel operator.
For a precise definition and fundamental properties of kernel operators we refer to \textcite[Section IV.9]{schaefer:1974} or \textcite[Chapter 13]{zaanen:1983}.
In particular, we recall that the restriction of a kernel operator to a sublattice is again a kernel operator and that
the identity on $L^{p}(\mu)$, $1 \leq p < \infty$, is a kernel operator if and only if the measure space $(X,\Sigma,\mu)$ is purely atomic, \ie  $L^{p}(\mu) \cong \ell^{p}_{I}$ for some index set $I$.
\ifnum 0 = 1 
For the proof of the following corollary we refer to 
\textcite[Theorem 1]{gerlach:2017}
\fi
%%%%%%%%%%%%%%%  TEIL 1  %%%%%%%
Moreover, from  \textcite{axmann:1980} we quote the following result (see Satz~3.5 l.c.).

%\begin{equation}
%\begin{split}
%    \text{If $T$ is an irreducible kernel operator, then }\\ 
%    \text{$\inf\{T^n,T^m\} > 0$  for some  $n,M \in \N$, $n \neq m $\,.}
%\end{split}
%\end{equation}
%%%%%%%%%%%%%%%%  ENDE Teil 1  %%%%
\begin{equation}
\begin{aligned}
\text{If $T$ is an irreducible kernel operator, then there exist } \\
n,m \in \mathbb N,\ n \neq m,
\text{such that } T^n \wedge T^m > 0.
\end{aligned}
\end{equation}

\begin{corollary}\label{cor:c4-2.11}
%
\index{operator!kernel}%
Let $\TT = (T(t))_{t\geq 0}$ be a semigroup on $L^{p}(\mu)$ satisfying the assumptions of Theorem~\ref{thm:c4-2.6} and assume that one operator $T(t_{0})$ is a kernel operator.
Then $\lim_{t \to \infty}T(t)f$ exists for every $f \in L^{p}(\mu)$.
\end{corollary}
%%%%%%%%%%%%  2.Teil  %%%%%
%% --
\begin{proof}
First we note that $\Kern{A} = \Fix{\TT}$ is a sublattice of $L^p(\mu)$\,, hence is itself an $L^p$-space. 
Since $T(t_0)_{\vert \Kern{A}} = \Id$  we conclude that  $\Kern{A} \cong \ell^p_I$\,.  
Thus $L^p(\mu)$ contains an orthogonal system $\{e_j \in \Kern{A} \colon j \in I\}$  of atoms such that $\sup_{j \in I} e_j = e$\,.
The closed principal ideal $E_j$ generated by $e_j$ in  $L^p(\mu)$ is $(T(t))$-invariant and the restriction of $(T(t))_{t \geq 0}$  to this ideal yields an irreducible semigroup $(T_j(t))_{t \geq 0}$ having generator $A_j$. 
From Corollary~\ref{cor:c3-3.9}  we conclude that  $P\sigma(A_j) \cap \im\R = {0}$\,. 
It follows that $T_j(t_0)$ is an irreducible kernel operator, hence by (2.9) all the assumptions of Corollary~2.10 are satisfied
Thus we have convergence on the the principal ideal $E_j$\,. 
Since the semigroup is bounded and the union of these ideals is total in  $L^p(\mu)$  we have convergence on the whole space.
\end{proof}
%%%%%%%%%%%%%%%%% ENDE  Teil 2 %%%%

%% --
In all the results obtained so far we had to show or to assume that $P{\sigma}(A) \cap \im\R = \{0\}$.
This is not surprising since for an eigenvector $g \in E$ corresponding to $\im\alpha \neq 0$, $\alpha \in \R$, we have $T(t)g = \eu^{\im\alpha t}g$ and so $\lim_{t\to\infty} T(t)g$ does not exist.
Nevertheless, in some cases with $P{\sigma}(A)\cap \im\R \neq \{0\}$ one can describe the asymptotic behavior of\/ $(T(t))_{t \geq 0}$ for large $t$.
Instead of convergence to an equilibrium point one obtains that $T(t)f$  \enquote{converges to a periodic function}.

To that purpose we consider a bounded, irreducible semigroup $\TT = (T(t))_{t \geq 0}$ of positive operators on some Banach lattice $E$ having order continuous norm.
Under the assumption that the spectral bound $s(A) = 0$ is a pole of the resolvent, we can apply Theorem~\ref{thm:c3-3.12}.
In particular, if\/ $0$ is not the only point in the boundary spectrum $\sigma(A)\cap \im\R$, we obtain that
%% --
\begin{equation*}\label{eq:c4-2.9-kgk}
\sigma(A)\cap \im\R = P{\sigma}(A)\cap \im\R = \im\alpha\Z \text{ for some } 0 < \alpha \in \R.
\end{equation*}
%% --
Therefore the assumptions of Theorem~\ref{thm:c3-3.8} are satisfied and formula Part-III, Chapter 3, \,(3.13) implies
%% --
\begin{equation}\label{eq:c4-2.10}
\rho(A) = \rho(A) + \im\alpha\Z \quad \text{and} \quad \|R(\lambda,A)\| = \|R(\lambda+\im\alpha k,A)\|
\end{equation}
%% --
for $\lambda \in \rho(A)$, $k \in \Z$.

Since $0$ was supposed to be a pole of the resolvent, we can decompose
%% --
\begin{equation*}\label{eq:c4-2.10-kgk}
\sigma(A) = \sigma_{1} \cup \sigma_{2},
\end{equation*}
%% --
where $\sigma_{1} = \im\alpha\Z$, $0 < \alpha \in \R$, and $\sup\{\Re(\lambda) \colon \lambda \in \sigma_{2}\} < 0$.
Moreover, for small $\epsilon > 0$, $\|R(-\epsilon+\im\lambda,A)\|$ is uniformly bounded for $\lambda \in \R$.

\index{spectral decomposition}
Next, we construct a spectral decomposition of\/ $E$ and $\TT$ corresponding to $\sigma_{1}$ and $\sigma_{2}$ (compare Part-I, Chapter 3,  Section 3).
Since $0$ is an eigenvalue of\/ $A$, it follows that $\TT$ has a quasi-interior fixed point $h \in E_{+}$ (use Proposition~\ref{prop:c3-3.5} (i)).
Hence, $\{T(t)f \colon t \geq 0\}$ is contained in the weakly compact (see Chapter~\ref{sec:c1-5}) order interval $[-h,h]$ whenever $|f| \leq h$.
Since $h$ is a quasi-interior point and $\TT$ is bounded, it follows that $\TT$ is relatively compact for the weak operator topology on $\LE$.

Therefore the \emph{Jacobs-DeLeeuw-Glicksberg Splitting Theorem} (see \textcite[Chapter 2, Theorem 4.4 and 4.5]{krengel:1985}) can be applied to (the weak closure of) $\TT$ and we obtain a projection $Q \in \LE$ onto the closed subspace $E_{1}$ generated by the eigenvectors $h_{k}$ of\/ $A$ corresponding to the eigenvalues $\im\alpha k$, $k \in \Z$.

Clearly, $Q$ splits the semigroup $\TT$ into the restricted semigroups $\TT_{1}$ on $E_{1} \coloneqq QE$ and $\TT_{2}$ on $E_{2} \coloneqq \Kern{Q}$.
We first describe $\TT_{1}$ in more detail.

The projection $Q$ is positive as an element of the weak closure of\/ $\TT$ and even strictly positive by the irreducibilitiy of\/ $\TT$.
Its range $E_{1}$ is a closed sublattice of\/ $E$ (use \textcite[Proposition III.11.5]{schaefer:1974}) on which the semigroup $\TT_{1}$ is periodic, irreducible and positive.
In fact, $T(2\pi/\alpha)f = f$ for every $f = h_{k}$, $k \in \Z$, and hence for every $f \in E_{1}$, while irreducibility and positivity are inherited from $\TT$.
It now follows from Part-I, Chapter 3,  Lemma~5.2 that the generator $A_{1} = A_{|E_{1}}$ of\/ $\TT_{1}$ has spectrum $\sigma(A_{1}) = \im\alpha\Z$.
Moreover, in view of Part-I, Chapter 2,  Proposition~5.2 and Corollary~5.3(ii) 
we have $\sigma(A_{2}) = \sigma(A) \setminus \im\alpha\Z$.
Therefore the decomposition $E = E_{1} \oplus E_{2}$ is a spectral decomposition corresponding to $\sigma_{1}$ and $\sigma_{2}$.
This proves the first part of the following lemma.

\begin{lemma}\label{lem:c4-2.12}
%
\index{spectral decomposition}
Under the above assumptions there exists a positive projection $Q$ with range $E_{1} \coloneqq QE$ and kernel $E_{2} \coloneqq Q^{-1}(0)$ such that the following holds.
%% --
\begin{enumerate}[\upshape (i)]
\item \label{lem:c4-2.12-1}
$E = E_{1} \oplus E_{2}$, $T = T_{1} \oplus T_{2}$ and $A = A_{1} \oplus A_{2}$ is a spectral decomposition corresponding to the decomposition $\sigma(A) = \sigma_{1} \cup \sigma_{2}$ where $\sigma_{1} = \sigma(A_{1}) = \im\alpha\Z$ and $\sigma_{2} = \sigma(A_{2}) = \sigma(A) \setminus \im\alpha\Z$.

\item \label{lem:c4-2.12-2}
$s(A_{2}) < 0$ and $\|R(\lambda,A_{2})\|$ is uniformly bounded in each semiplane\\ 
$\{\lambda \in \C \colon \Re(\lambda) > s(A_{2}) + \epsilon\}$ with $\epsilon > 0$.

\item \label{lem:c4-2.12-3}
$E_{1}$ is a closed sublattice of\/ $E$ and $\TT_{1}$ is a periodic, irreducible, positive semigroup on $E_{1}$.
In particular, $(E_{1},\TT_{1})$ is isomorphic to $(L,R_{\tau}(t))$ where $L$ is a function lattice between $C(\Gamma)$ and $L^{1}(\Gamma)$ and $R_{\tau}(t)$ is the rotation group with period $\tau = 2\pi/\alpha$.
\index{semigroup!rotation semigroup}
\end{enumerate}
\end{lemma}
%% --
\begin{proof}
\ref{lem:c4-2.12-1} has been derived above while \ref{lem:c4-2.12-2} follows immediately from Eq.~\eqref{eq:c4-2.10}.
The properties of\/ $\TT_{1}$ mentioned in \ref{lem:c4-2.12-3} have been stated above.
Hence the representation of\/ $\TT_{1}$ as a rotation group follows from Corollary~\ref{cor:c3-3.9} .
\end{proof}
%% --
For Hilbert spaces $L^{2}(\mu)$ the property \ref{lem:c4-2.12-2} of the above lemma and Part-I, Chapter 3,  Corollary~7.11 imply that the growth bound $\omega_{0}(A_{2})$ is less than zero.
Therefore we obtain the following result on the behavior of\/ $\TT$.
%% --
\begin{proposition}\label{prop:c4-2.13}
\index{resolvent!pole}
Let $\TT = (T(t))_{t\geq 0}$ be a bounded, irreducible, positive semigroup on a Hilbert lattice $E = L^{2}(\mu)$.
Assume that $s(A) = 0$ is a pole of the resolvent of the generator $A$ and that $\im\alpha \in \sigma(A)$ for some $0 \neq \alpha \in \R$.
Then $\TT$ behaves asymptotically as the rotation group $(R_{\tau}(t))_{t \geq 0}$ with period $\tau = 2\pi n/\alpha$ for some $n \in \N$ on $L^{2}(\Gamma)$.

More precisely, we can identify $L^{2}(\Gamma)$ with a sublattice of\/ $E$, being the range of a strictly positive projection $Q$, and we find constants $\epsilon > 0$ and $M \geq 1$ such that for every $f \in E$ we have
%% --
\begin{equation}\label{eq:c4-2.11}
\|T(t)f - R_{\tau}(t)g\| \leq M\eu^{-\epsilon t}\|f\| \text{ for every } t \geq 0 \text{ where } g \coloneqq Qf.
\end{equation}
\end{proposition}
%% --
For $L^{p}$-spaces the analogous statement can be shown only for a weaker type of convergence.
The proof of this result uses interpolation for operators, mainly the Riesz Convexity Theorem (see the remarks preceding Corollary \ref{cor:c4-1.2}).

\begin{theorem}\label{thm:c4-2.14}
%
\index{periodic semigroup}
\index{semigroup!irreducible semigroup|)}
%
%
Let $\TT = (T(t))_{t \geq 0}$ be a bounded, irreducible positive semigroup on a Banach lattice $E = L^{p}(\mu)$, $1 \leq p < \infty$.
Assume that $s(A) = 0$ is a pole of the resolvent of the generator $A$ and that $\im\alpha \in \sigma(A)$ for some $0 \neq \alpha \in \R$.
Then $\TT$ behaves asymptotically as the rotation group $(R_{\tau}(t))_{t \geq 0}$ with period $\tau > 0$ on $L^{p}(\Gamma)$, \ie  we can identify $L^{p}(\Gamma)$ with a sublattice of\/ $E$ such that for every $f \in E$ there exists $g \in L^{p}(\Gamma)$ satisfying
%% --
\begin{equation}\label{eq:c4-2.12}
\lim_{t \to \infty} \|T(t)f - R_{\tau}(t)g\| = 0.
\end{equation}
\end{theorem}
%% --
\begin{proof}
We only consider $1 \leq p < 2$.
The assertion for $p > 2$ then follows by duality while $p = 2$ was treated in Proposition~\ref{prop:c4-2.13}.

At first we observe that without loss of generality we may assume that $\mu$ is a probability measure and that $T(t)\1 = \1$ for every $t \geq 0$.
In fact, the assumptions imply that $T(t)h = h$ for some $h \gg 0$, $\|h\|_{p} = 1$.
We consider the measure $\nu$ which has the density $h^{p}$ with respect to $\mu$.
Then $\nu$ is a probability measure and 
$M \colon L^{p}(\nu) \to L^{p}(\mu)$, defined by $Mh \coloneqq h\cdot f$, is an isometric lattice isomorphism of\/ $L^{p}(\nu)$ onto $L^{p}(\mu)$.
The semigroup defined by $\widehat{T}(t) \coloneqq M^{-1}T(t)M$ possesses the same properties as $(T(t))$ and satisfies $\widehat{T}(t)\1 = \1$ for $t \geq 0$.

Now the properties $T(t)\1 = \1$, and $T(t) \geq 0$ imply that $L^{\infty}(\mu)$ is an invariant subspace for every operator $T(t)$ which is contractive with respect to the $L^{\infty}$-norm.
The \emph{Riesz Convexity Theorem} (see \textcite[VI.10.11]{dunfordschwartz:1958}) then implies that by restricting the semigroup $(T(t))$ to $L^{q}(\mu)$ $(p < q < \infty)$ we obtain a strongly continuous semigroup $(T_{q}(t))_{t \geq 0}$ on $L^{q}(\mu)$ such that $\|T_{q}(t)\| \leq \|T(t)\|^{p/q}$ for $t \geq 0$, $q \geq p$.

Let $A_{q}$ be the generator of\/ $(T_{q}(t))$.
In order to apply Proposition~\ref{prop:c4-2.13} we have to show that $0$ is a pole of the resolvent of\/ $A_{2}$.
Denoting the residue of\/ $R(\cdot,A)$ at $0$ by $P$, then $P = h \otimes l$ for a suitable $h \in (L^{p}(\mu))'$.
Since $(L^{p}(\mu))' \subset (L^{2}(\mu))'$, $P$ can also be considered as bounded operator on $L^{2}(\mu)$.
We denote it by $P_{2}$.
From $AP = PA = 0$ it follows that
%% --
\begin{align*}
(R(1,A)(\Id-P))^{n} &= R(1,A)^{n} - P \quad (n \in \N) \quad \text{and} \\
(R(1,A_{2})(\Id-P_{2}))^{n} &= R(1,A_{2})^{n} - P_{2} \quad (n \in \N).
\end{align*}
%% --
The Riesz Convexity Theorem yields the following estimate for the operator norm:
%% --
\begin{align*}
\|R(1,A_{2})^{n} - P_{2}\| &\leq \|R(1,A)^{n} - P\|^{2/p}\|R(1,A)_{\infty}^{n} - P_{\infty}\|^{1-2/p} \\
&\leq \|R(1,A)^{n} - P\|^{2/p}(1 + \|P_{\infty}\|)^{1-2/p}.
\end{align*}
%% --
Since $0$ is a pole with residue $P$, the spectral radius of the operator $R(1,A)(1-P)$ is less than $1$.
Thus the right hand side of the inequality tends to $0$ as $n \to \infty$.
It follows that $r_{ess}(R(1,A_{2})) < 1$, hence $1$ is a pole of the resolvent of\/ $R(1,A_{2})$, or equivalently, $0$ is a pole of\/ $R(\cdot,A_{2})$ (see Part-I, Chapter 3,  Proposition.2.5).

Now we can apply Proposition~\ref{prop:c4-2.13} and obtain a projection $Q$ such that 
%% --
\[ 
\lim_{t \to \infty}\|T(t)f - R_{\tau}(t) \circ Qf\|_{2} = 0 \ \text{ for every 
} \ f \in L^{2}(\mu)\,.
\]
%% --
On order intervals of\/ $L^{\infty}(\mu)$, both $L^{p}$- and $L^{2}$-norm induce the same topology (see \textcite[V.8.3]{schaefer:1974}), 
hence $\lim_{t \to \infty} \|T(t)f - R_{\tau}(t) \circ Qf\|_{p} = 0$ for every $f \in L^{\infty}(\mu)$.
Since $(T(t))$ is bounded, we finally obtain convergence in the $L^{p}$-norm for every $f \in L^{p}(\mu)$.
\end{proof}
%% --
%
\index{cell division model}
\index{population semigroup}
We give an example for the situation treated in Theorem~\ref{thm:c4-2.14}.
The equation we consider describes the division of a cell population.
For details we refer to \textcite{diekmannetal:1984}.

\begin{example}\label{ex:c4-2.15}
Let $E = L^{1}([\frac{1}{4},1],w\,\dx)$, where the density $w$ is a continuous positive function on $[\frac{1}{4},1]$, vanishes at $x = 1$ and is strictly positive in $[\frac{1}{4},1)$.
We consider the operator $C = A + B$ where $A$ is defined by $(Af)(x) \coloneq  -xf'(x)$ on the domain $D(A) \coloneqq \{f \in AC \colon f(\frac{1}{2}) = 0\}$ and $B$ is defined by
%% --
\begin{equation*}\label{eq:c4-2.12-kgk}
Bf(x) \coloneqq \begin{cases}
    k(x)f(2x) & \text{if } x \leq \frac{1}{2}, \\
    0 & \text{if } x > \frac{1}{2}.
\end{cases}
\end{equation*}
%% --
Here $k$ is a positive continuous function on $[\frac{1}{4},1]$ satisfying
%% --
\begin{equation}\label{eq:c4-2.13}
k(x) > 0 \quad \text{ for } \quad \frac{1}{4} < x < \frac{1}{2} \quad \text{ and }  \int_{1/4}^{1/2} \frac{k(y)}{y} \, \dy = 1.
\end{equation}
%% --
In the following we show that under these hypotheses and for suitable $w$ the semigroup generated by $C$ fulfills the assertions of Theorem~\ref{thm:c4-2.14}.

The operator $A$ generates the nilpotent semigroup $(T(t))$ defined by
\index{semigroup!nilpotent semigroup}
%% --
\begin{equation*}\label{eq:c4-2.13-kgk}
(T(t)f)(x) = \begin{cases}
    f(\eu^{-t}x) & \text{if } \eu^{-t}x \geq \frac{1}{4}, \\
    0 & \text{otherwise}.
\end{cases}
\end{equation*}
%% --
We have $(R(\lambda,A)f)(x) = x^{-\lambda}\int_{1/4}^{x} y^{\lambda-1}f(y) \, \dy$ $(f \in E, x \in [\frac{1}{4},1])$, hence $A$ has compact resolvent.
Since $B$ is bounded and positive, $C$ is the generator of a positive semigroup $(S(t))$ having compact resolvent as well.
Using Proposition~\ref{prop:c3-3.3}  one can show that $(S(t))$ is irreducible.
Indeed, the non-trivial $(T(t))$-ideals are of the form $I_{s} = \{f \in E \colon f \text{ vanishes on } [\frac{1}{4},s]\}$ with $s$ satisfying $\frac{1}{4} < s < 1$.
Since none of these ideals is invariant under $B$, the semigroup $(S(t))$ is irreducible.

A suitable choice of the weight function $w$ ensures that $(S(t))$ is bounded.
Take
%% --
\begin{equation}\label{eq:c4-2.14}
w(x) \coloneqq \begin{cases}
    \text{\sfrac{1}{x}} & \text{for } x \leq \frac{1}{2}, \\
    \text{\sfrac{1}{x}} \left\{1 - \int_{1/4}^{x/2} \frac{k(y)}{y} \, \dy \right\} & \text{for } x \geq \frac{1}{2}.
\end{cases}
\end{equation}
%% --
Then integration by parts yields for $f \in D(A) = D(C)$ that
%% --
\begin{align*}
\langle Cf,\1 \rangle = \int_{1/4}^{1/2} (-xf'(x) + k(x)f(2x))w(x) \, \dx - \int_{1/2}^{1} xf'(x)w(x) \, \dx = 0.
\end{align*}
%% --
Thus $\1 \in D(C')$ and $C'\1 = 0$, equivalently $S(t)'\1 = \1$ for all $t$.
This shows that $(S(t))$ is a semigroup of contractions on $E$.

It remains to show that there is $\alpha > 0$ such that $\im\alpha \in \sigma(C)$.
In fact, considering $\alpha \coloneqq 2\pi(\log 2)^{-1}$,  then $\im\alpha$ is an eigenvalue of\/ $C$.
A corresponding eigenfunction is given by $h_{1}(x) \coloneq  x^{-\im\alpha}h_{0}(x)$, where $h_{0}$ is the eigenfunction corresponding to $0$ which is given by
%% --
\begin{equation}\label{eq:c4-2.15}
h_{0}(x) \coloneq  \begin{cases}
    \int_{1/4}^{x} \frac{k(y)}{y} \, \dy & \text{for } \frac{1}{4} \leq x \leq \frac{1}{2}, \\
    1 & \text{for } \frac{1}{2} \leq x \leq 1.
\end{cases}
\end{equation}
%% --
The verification of these statements is left as an excercise.
\end{example}

In several of the above results we had to assume that the positive semigroup $(T(t))_{t\geq 0}$ is bounded and has spectral bound zero.
In general, these conditions are difficult to verify, in particular, when only the generator is known.
In the final example we described a method how to cope with this problem. 

If\/ $s(A)$ is an eigenvalue of the adjoint $A'$ with a strictly positive eigenvector $\phi$, then $(T(t))_{t\geq 0}$ induces, in a canonical way, a positive semigroup $(T_{\phi}(t))_{t \geq 0}$ on the AL-space $(E,\phi)$.
This semigroup satisfies $\|T_{\phi}(t)\| \leq \exp(t \cdot s(A))$ and has spectral bound $s(A)$.
Hence one may apply the results of this section to the rescaled semigroup $(\exp(-t \cdot s(A))T_{\phi}(t))_{t \geq 0}$, thus obtaining convergence of\/ $(T(t))_{t \geq 0}$ for the weaker topology on $E$ which is induced by $(E,\phi)$.

\newpage
\section{A Semigroup Approach to Retarded Equations}\label{sec:c4-3}
%
%
\index{equation!retarded}
%
%% --
\hspace{1cm}{\Large by Annette Grabosch und Ulrich Moustakas}
\vspace{.5cm}
\newline
\index{retarded equation|(}
\index{retarded semigroup|(}
As indicated by the above title of this section there is a close relationship to Part-II, Chapter 4,  Section~3.
First, the considered Cauchy problems are \enquote{similar} to (RCP).
Second, there again is a correspondence to a class of semigroups generated by the first derivative.

Instead of the differential equation in (RCP) we will study equations of the form
\index{equation!retarded|(}
%% --
\begin{equation*}\label{eq:c4-re} \tag{RE}
\begin{split} 
u(t) &= \Phi(u_t), \, t \geq 0,\\
u_0 &= g.
\end{split}
\end{equation*}
%% --
We use the following setting: Let $F$ be a Banach space, consider $E \coloneqq L^{1}([-1,0],F)$ and take $\Phi \in \L{E,F}$.
For $u \in L^{1}_{\mathrm{loc}}([-1,\infty),F)$ we denote by $u_t \in E$ the function given by $u_t(s) \coloneqq u(t+s)$, $t \geq 0$, $s \in [-1,0]$.

By a solution of Eq.~\eqref{eq:c4-re} with initial function $g \in E$ we understand a function $u \in L^{1}_{\mathrm{loc}}([-1,\infty),F)$ which satisfies equation Eq.~\eqref{eq:c4-re}.
The problem Eq.~\eqref{eq:c4-re} is called \emph{well-posed} if for each $g \in E$ there exists exactly one solution.
%% --
\begin{remarks*}
\begin{enumerate}[\upshape (i), wide, labelindent=.5em] 
\item \label{rem:c4-kgk-1}
The equation
%% --
\begin{align*}
u(t) &= Bu(t) + \Phi(u_t), t \geq 0,\\
u_0 &= g,
\end{align*}
%% --
(where $B$ is the generator of a bounded semigroup on $F$) seems to be in better analogy to the retarded Cauchy problem of Part-II, Chapter 4,  Section~3 and can be reduced to an equation of the type Eq.~\eqref{eq:c4-re}.

In fact, since $1 \in \rho(B)$ we have,
%% --
\begin{equation*}\label{eq:c4-3.0-KGK1}
u(t) = R(1,B)\Phi(u_{t}).
\end{equation*}
%% --
\index{Cauchy problem!retarded}
Clearly, this equation is of the previous type (with a different \enquote{delay functional}).
%%--
\item \label{rem:c4-kgk-2}
The choice of \enquote{$L^{1}$-functions} instead of \enquote{C-functions} (as in the case of (RCP)) enforces the solutions of Eq.~\eqref{eq:c4-re} to yield a strongly continuous semigroup of operators (on the space $E$ of initial functions) as in Part-II, Chapter 4,  Section 3.
\end{enumerate}
\end{remarks*}
%% --
\index{differential operator}
In order to solve Eq.~\eqref{eq:c4-re} we consider on $E = L^{1}([-1,0],F)$ the differential operator $A \coloneq  \frac{\diff{}}{\dx}$ with domain
%% --
\begin{equation*}\label{eq:c4-3.0-KGK2}
D(A):=\{f \in AC([-1,0],F) \colon f' \in E \text{ and } f(0) = \Phi(f)\}.
\end{equation*}
%% --
We claim that $(A,D(A))$ generates a strongly continuous semigroup $(T(t))_{t\geq 0}$ on $E$.
To this end we first consider the operator $A_0 f \coloneqq f'$ with domain
%% --
\begin{equation*}\label{eq:c4-3.0-KGK3}
D(A_0) \coloneqq \{f \in E \colon f \in AC([-1,0],F), f' \in E \text{ and } f(0) = 0\}.
\end{equation*}
%% --
Similarly to the special case where $F = \R$ (compare Part-I, Chapter 1,  Example 2.4.(ii)), it can be seen that the operator $A_0$ generates a strongly continuous semigroup $(T_0(t))_{t\geq 0}$ given by
%% --
\begin{equation}\label{eq:c4-3.1}
(T_0(t)f)(s) = \begin{cases}
    f(t+s) & \text{if } t+s \leq 0 \\
    0 & \text{if } t+s > 0.
\end{cases}
\end{equation}
%% --
Notice that $(T_0(t))_{t \geq 0}$ is a nilpotent semigroup.

Now consider the operators $S_{\lambda} \colon E \to E \colon f \mapsto \epsilon_{\lambda} \otimes \Phi(f)$, $\lambda > 0$, where $\epsilon_{\lambda}$ denotes the function $s \mapsto \eu^{\lambda s}$ as an element of\/ $L^{1}[-1,0]$, 
and $h \otimes x \in E$ is defined by 
%\linebreak[2] 
$(h \otimes x)(s) \coloneqq h(s) \cdot x$ for $h \in L^{1}[-1,0]$, $x \in F$ and $s \in [-1,0]$.
Clearly 
%\linebreak[4] 
$\|\epsilon_{\lambda}\| = 1/\lambda \cdot (1 - \eu^{-\lambda}) \to 0$ as $\lambda \to \infty$ and we have
%% --
\[
\|S_{\lambda}\| = \|\epsilon_{\lambda}\| \cdot \|\Phi\| = 1/\lambda \cdot (1 - \eu^{-\lambda}) \cdot \|\Phi\| \leq 1/\lambda \cdot \|\Phi\|\,.
\]
%% --
For every $\lambda > \|\Phi\|$, the operator $(\Id - S_{\lambda})$ is an isomorphism of\/ $E$ and it induces a bijection from $D(A)$ onto $D(A_{0})$ such that
%% --
\begin{equation}\label{eq:c4-3.2}
(\lambda - A) = (\lambda - A_{0})(\Id - S_{\lambda}).
\end{equation}
%% --
Since $A_{0}$ generates a semigroup of contractions, $\lambda - A_{0}$ is invertible for each $\lambda > 0$.
This yields the invertibility of\/ $\lambda - A$ for each $\lambda \geq \|\Phi\|$.

In order to obtain an estimate on $\|R(\lambda,A)\|$ we use Formula Eq.~\eqref{eq:c4-3.2}.
Since 
\[ 
\|R(1,S_{\lambda})\| = \|\sum_{n=0}^{\infty} S_{\lambda}^n\| \leq \sum_{n=0}^{\infty} \|\epsilon_{\lambda}\|^n \cdot \|\Phi\|^n = (1-\|\epsilon_{\lambda}\| \cdot \|\Phi\|)^{-1}
\]
and $\|R(\lambda,A_0)\| \leq 1/\lambda$ for $\lambda > 0$, we obtain for $\lambda \geq \|\Phi\|$ that
%% --
\begin{align*}
\|R(\lambda,A)\| &\leq (1 - \|\epsilon_{\lambda}\| \cdot \|\Phi\|)^{-1} \cdot 1/\lambda = (\lambda - \lambda \cdot \|\epsilon_{\lambda}\| \cdot \|\Phi\|)^{-1} \\
&= (\lambda - (1 - \eu^{-\lambda}) \cdot \|\Phi\|)^{-1} \leq (\lambda - \|\Phi\|)^{-1}.
\end{align*}
%% --
By using Part-I, Chapter 2,  Corollary~1.8 we thus have proved the first assertion of the following theorem.
%% --
\begin{theorem}\label{thm:c4-3.1}
%
%
The operator $A$, defined above, is the generator of a semigroup $(T(t))_{t \geq 0}$ on $E$.
For every $f \in E$, $t \geq 0$, we have for a.e. $s \in [-1,0]$
%% --
\begin{equation}\label{eq:c4-3.3}
(T(t)f)(s) = \begin{cases}
    f(t+s) & \text{if } \ t+s \leq 0 \\
    \Phi(T(t+s)f) & \text{if } \ t+s > 0.
\end{cases}
\end{equation}
%% --
Moreover, if\/ $f \in D(A)$, then the translation property (T) (see Part-II, Theorem~\ref{thm:b4-3.1}) is satisfied.
\end{theorem}

\begin{proof}
Consider $E_1 \coloneqq D(A)$ endowed with the graph norm 
and on $E_1$ the operator $A_1$ which is the restriction of\/ $A$ to $D(A_1) \coloneqq D(A^{2})\subset E_1$.
By (Part-I, Chapter 1,  3.6), $A_1$ generates the semigroup $(T(t)_{|D(A)})_{t \geq 0}$.
On $E_1$ the point evaluation is a continuous mapping and therefore the translation property can be shown as in the proof of Part-II, Theorem~\ref{thm:b4-3.1}.
Hence we obtain
%% --
\begin{equation}\label{eq:c4-3.4}
(T(t)f)(s)= \begin{cases}
    f(t+s) &\text{if } t+s \leq 0 \\
    \Phi(T(t+s)f) &\text{if } t+s > 0
\end{cases} = \begin{cases}
    f(t+s) &\text{if } t+s \leq 0 \\
    (T(t+s)f)(0) &\text{if } t+s > 0
\end{cases};
\end{equation}
%% --
\ie Eq.~\eqref{eq:c4-3.3} is valid for $f \in D(A)$.
It remains to show Eq.~\eqref{eq:c4-3.3} for all $f \in E$.
Let $t \in \R_+$ and $s \in [-t,0]$.
For $t + s > 0$ the equality follows immediately by the continuity of\/ $\Phi$ from Eq.~\eqref{eq:c4-3.4}.
For the case $t + s \leq 0$ we consider $g \in L^{\infty}[-1,0]$ with supp $g \subset [-1,-t]$.
Comparing Eq.~\eqref{eq:c4-3.1} and Eq.~\eqref{eq:c4-3.4} we see that $\langle(T(t)-T_0(t))f,g\rangle = 0$ for all $f \in D(A)$, and hence for all $f \in E$.
Consequently $(T(t)-T_{0}(t))f = 0$ a.e. on $[-1,-t]$ which shows $(T(t)f)(s) = f(t+s)$ for a.e. $s \in [-1,-t]$.
\end{proof}
%% --
The following corollary corresponds to Part-II, Corollary~\ref{cor:b4-3.2} and assures the well-posed\-ness of Eq.~\eqref{eq:c4-re}.

\begin{corollary}\label{cor:c4-3.2}
%
For every $f \in E$ the function $u$ defined by
%% --
\begin{equation}\label{eq:c4-3.5}
u(t) \coloneqq  \begin{cases}
    f(t) & \text{if } \ -1 \leq t \leq 0 \\
    \Phi(T(t)f) & \text{if } \ t > 0
\end{cases}
\end{equation}
%% --
is the unique solution of Eq.~\eqref{eq:c4-re}, in particular Eq.~\eqref{eq:c4-re} is well-posed.
If\/ $f \in D(A)$, then $u(t) = (T(t)f)(0)$ for $t > 0$.
\end{corollary}
%% --
\begin{proof}
As in the proof of Part-II, Corollary~\ref{cor:b4-3.2} we have $u_t = T(t)f$ for $t \geq 0$ since $u_t(s) = u(t+s) = (T(t)f)(s)$ by the definition of\/ $u$ and by formula Eq.~\eqref{eq:c4-3.3}.
Thus $u(t) = \Phi(T(t)f) = \Phi(u_t)$ if\/ $t \geq 0$.
Also by the definition of\/ $u$, we have $u_{0} = f$.
It remains to show uniqueness.
Let $w$ be a solution of Eq.~\eqref{eq:c4-re} with initial function $w_{0} = 0$.
Then
\begin{align*}
\|w(t)\|_F & = \|\Phi(w_{t})\| \leq \|\Phi\| \cdot  \|w_{t}\|_{E} =  \|\Phi\| \cdot \int_{-1}^{0} \|w_{t}(s)\_{F}\| \, \ds \\
&=  \|\Phi\| \int_{-1}^0 \|w(t+s)\|_{F} \ds = \|\Phi\| \int_{t-1}^{t} \|w(s)\|_{F} \ds \\
&\leq \|\Phi\| \int_{-1}^{t} \|w(s)\|_{F} \ds \ \text{ for } \ t \geq 0\,.
\end{align*}
By Gronwall's lemma $\|w(t)\|_F \leq 0$, thus $w(t) = 0$.
\end{proof}
%% --
Now we turn to the aspect of positivity in Eq.~\eqref{eq:c4-re}. We assume $F$ to be a Banach lattice and let $E$ inherit the canonical ordering from $F$ making it a Banach lattice again. Additionally, let $\Phi$ be positive.

The first observation is that $A$ generates a positive semigroup.
Indeed, it follows from equation Eq.~\eqref{eq:c4-3.2}, that $R(\lambda,A) = R(1,S_\lambda)R(\lambda,A_{0})$ for $\lambda > \omega$. Since $S_\lambda$ is a positive operator, we have $R(1,S_\lambda) \geq 0$.
The semigroup generated by $A_{0}$ is positive (use Eq.~\eqref{eq:c4-re}), hence $R(\lambda,A_{0}) \geq 0$.
It follows that $R(\lambda,A) \geq 0$ which is equivalent to the positivity of\/ $(T(t))_{t \geq 0}$ (see Proposition~\ref{prop:c2-4.1} ).

\begin{proposition}\label{prop:c4-3.3}
If\/ $\Phi \in \L{E,F}$ is a positive operator, then the solution semigroup corresponding to Eq.~\eqref{eq:c4-re} is positive.
\end{proposition}
%% --
For the following considerations concerning spectral poperties of the semigroup we always suppose $\Phi$ to be positive. Furthermore, we define operators $\Phi_{\lambda} \in \L{F}$, $\lambda \in \R$, by
%% --
\begin{equation}\label{eq:c4-3.6}
\Phi_\lambda x \coloneqq \Phi(\epsilon_{\lambda} \cdot x), \quad x \in F\,.
\end{equation}
%% --
Evidently, each $\Phi_\lambda$ is a positive operator on $F$ and $\lambda \leq \mu$ implies $\Phi_{\lambda} \geq \Phi_{\mu}$.
From this relation it follows that the spectral bound $s(\Phi_{\lambda})$ which coincides with the spectral radius $r(\Phi_{\lambda})$ is a decreasing function in $\lambda$.

Furthermore, we shall need the following properties.

\begin{lemma}\label{lem:c4-3.4}
%
The map $h \colon \R \to \R_{+}$\,, $\lambda \mapsto s(\Phi_{\lambda})$ is continuous from the left. 
If\/ $\Phi_{\lambda}$ is compact for all $\lambda \in \R$, then $h$ is continuous.
\end{lemma}
%% --
\begin{proof}
As indicated above, $h$ is decreasing. Hence continuity from the left follows from the upper semicontinuity of the spectrum (see \textcite[Chapter IV, Theorem~3.1]{kato:1976}).

Now take $\lambda \in \R$ with $s(\Phi_{\lambda}) > 0$ (if\/ $s( \Phi_{\lambda}) = 0$, then continuity in $\lambda$ is trivial by the continuity from the left and the monotonicity). Since $\Phi_{\lambda}$ is positive and bounded, we know that 
$s(\Phi_{\lambda})$ is the boundary spectrum $\sigma_b(\Phi_{\lambda})$ (see Corollary~\ref{cor:c3-2.12} ) of\/ $\Phi_{\lambda}$. Moreover, $s(\Phi_{\lambda})$ is a pole of the resolvent with residue of finite rank. Such spectral sets vary continuously under smooth perturbations of\/ $\Phi_{\lambda}$ (see \textcite[VII.6, Theorem 9]{dunfordschwartz:1958}), thus $\lambda \mapsto s(\Phi_{\lambda})$ is continuous.
\end{proof}
%% --
For the operators $A_{0}$ and $A$ as defined in the beginning of this section we obtain an explicit representation of their resolvents.

\begin{lemma}\label{lem:c4-3.5}
For the resolvents of the operators $A_{0}$, \resp $A$, on $E$ the following statements hold.
%% --
\begin{enumerate}[\upshape (i)]
\item \label{lem:c4-3.5-1}
For every $\lambda \in \C$ we have $\lambda \in \rho(A_{0})$ and
%% --
\begin{equation}\label{eq:c4-3.7}
(R(\lambda,A_{0})g)(t) = \int_{0}^{t} \eu^{\lambda(t-s)}g(s)\, \ds, \quad g \in E.
\end{equation}
%% --
\item \label{lem:c4-3.5-2}
For $\lambda \in \C$ satisfying $1 \in \rho(\Phi_{\lambda})$ we have $\lambda \in \rho(A)$ and
%% --
\begin{equation}\label{eq:c4-3.8}
R(\lambda,A)g = R(\lambda,A_{0})g + \epsilon_{\lambda} \otimes R(1,\Phi_{\lambda})\Phi R(\lambda,A_{0})g\,, \quad g \in E.
\end{equation}
%% --
\end{enumerate}
\end{lemma}
%% --
\begin{proof}
\begin{enumerate}[\upshape (i), wide, labelindent=.5em] 
    \item 
    %\ref{lem:c4-3.5-1}
    That $\rho(A_{0}) = \C$ follows directly from $(T_{0}(t))_{t \geq 0}$ being nilpotent (see Part-I, Chapter 3,  Proposition 1.1). For $g \in E$ we obtain $R(\lambda,A_{0})g = f$ where $f$ is a solution of\/ $\lambda f - f' = g$.
    Thus $R(\lambda,A_{0})g(t) = \int_{0}^{t} \eu^{\lambda(t-s)}g(s)\, \ds + \eu^{\lambda t} \cdot x$ for some $x \in F$. The condition $f \in D(A_{0})$ now implies $x = 0$ and Formula Eq.~\eqref{eq:c4-3.7}.
    
    \item 
    %\ref{lem:c4-3.5-2}
    The assertion $\lambda \in \rho(A)$ means that for every $g \in E$ the equation $\lambda f - f' = g$ has exactly one solution $f$ in $D(A)$. As in \ref{lem:c4-3.5-1} we have 
    $f(t) = \int_{0}^{t} \eu^{\lambda(t-s)}g(s)\, \ds + \eu^{\lambda t} \cdot x$ for some $x \in F$ and hence
    $R(\lambda,A)g = f = R(\lambda,A_{0})g + \epsilon_{\lambda}  \otimes x$. 
    The condition $R(\lambda,A)g \in D(A)$ implies $x - \Phi_{\lambda}(x) = \Phi R(\lambda,A_{0})g$. 
\end{enumerate}    
    Hence, $x = R(1,\Phi_{\lambda})\Phi R(\lambda,A_{0})g$ if\/ $1 \in \rho(\Phi_{\lambda})$ and thus Eq.~\eqref{eq:c4-3.8} follows.
\end{proof}
%% --
\begin{proposition}\label{prop:c4-3.6}
For each $\lambda \in \C$ the following implications hold.
\begin{enumerate}[\upshape (i)]
\item \label{prop:c4-3.6-1}
If \, $\lambda \in \sigma(A)$, then $1 \in \sigma(\Phi_{\lambda})$.
%%--
\item \label{prop:c4-3.6-2}
If \, $1 \in P\sigma(\Phi_{\lambda})$, then $\lambda \in P\sigma(A)$.
\end{enumerate}
If, in addition, $\Phi(D(A_{0})) = F$ or if\/ $\Phi_{\lambda}$ is compact for all $\lambda \in \C$, then the following equivalence holds:
\begin{enumerate}[\upshape (i)]
\setcounter{enumi}{2}
\item \label{prop:c4-3.6-3}
$\lambda \in \sigma(A)$ if and only if\/ $1 \in \sigma(\Phi_{\lambda})$.
\end{enumerate}
\end{proposition}
%% --
\begin{proof}
\begin{enumerate}[\upshape (i), wide, labelindent=.5em] 
\item 
%\ref{prop:c4-3.6-1} 
This implication follows immediately from Lemma \ref{lem:c4-3.5}\,(ii).

\item 
%\ref{prop:c4-3.6-2} 
If\/ $x \neq 0$ satisfies $x - \Phi_{\lambda}(x) = 0$, then $f \coloneq  \epsilon_{\lambda} \otimes x \in D(A)$ and $\lambda f - f' = 0$.

\item 
%\ref{prop:c4-3.6-3} 
If\/ $\Phi(D(A_{0}))$ $= F$,  then the equation $x - \Phi_{\lambda}(x) = 0$ has a
unique solution for every $g \in E$ if and only if\, $1 \in \rho(\Phi_{\lambda})$. 
According to the proof of \ref{lem:c4-3.5}\,\ref{lem:c4-3.5-2} this is equivalent to $\lambda \in \sigma(A)$.
\end{enumerate}
If\/ $\Phi_{\lambda}$ is compact, then $\sigma(\Phi_{\lambda})\backslash\{0\} \subset P\sigma(\Phi_{\lambda})$. Thus the assertion follows from \ref{prop:c4-3.6-1} and \ref{prop:c4-3.6-2}.
\end{proof}
%% --
The previous results will now be used to characterize the spectral bound of\/ $A$ and hence the stability of the solutions of Eq.~\eqref{eq:c4-re}.

\begin{theorem}\label{thm:c4-3.7}
\index{stability}
\index{spectral bound $s(A)$}
Let $A \coloneqq \frac{\diff{}}{\dt}, D(A) \coloneqq \{f \in AC([-1,0],F) \colon f' \in L^{1}([-1,0],F)$ and $f(0) = \Phi(f)\}$ be the generator of the solution semigroup on $E \coloneqq L^{1}([-1,0],F)$ corresponding to Eq.~\eqref{eq:c4-re}. 

If\/ $F$ is a Banach lattice and $0 < \Phi \in \L{E,F}$, then the following assertions hold for $\lambda \in \R$.
\begin{enumerate}[\upshape (i)]
\item \label{thm:c4-3.7-1}
If\/ $s(\Phi_{\lambda}) < 1$, then $s(A) < \lambda$.
%%.
\item \label{thm:c4-3.7-2}
Let $\Phi(D(A_{0})) = F$ or let $\Phi_{\lambda}$ be compact for all $\lambda \in \R$. 
In addition, suppose that the map $\mu \mapsto s(\Phi_\mu)$ is strictly decreasing at $\mu = s(A)$. 

If\/ $s(\Phi_{\lambda}) = 1$, then $s(A) = \lambda$.
%%--
\item \label{thm:c4-3.7-3}
Let $\Phi_{\lambda}$ be compact for all $\lambda \in \R$ \quad or\quad let $\Phi(D(A_{0})) = F$ and suppose that the map $\mu \mapsto s(\Phi_\mu)$ is continuous from the right. 

If\/ $s(\Phi_{\lambda}) > 1$, then $s(A) > \lambda$.
\end{enumerate}
\end{theorem}

\begin{proof}
\begin{enumerate}[\upshape (i), wide, labelindent=.5em] 
\item% 
%\ref{thm:c4-3.7-1} 
Suppose $r \coloneqq s(A) \geq \lambda$. 
The positivity of\/ $(T(t))_{t \geq 0}$ implies $r \in \sigma(A)$ (see Theorem~\ref{thm:c3-1.1} .(i)) and by Proposition~\ref{prop:c4-3.6}\,\ref{prop:c4-3.6-1}  this implies $1 \in \sigma(\Phi_r)$, so that $s(\Phi_{r)} \geq 1$. 
Since $r \geq \lambda$, this yields $s(\Phi_{\lambda}) \geq s(\Phi_{r}) \geq 1$.

\item 
%\ref{thm:c4-3.7-2} 
Let $s(\Phi_{\lambda}) = 1$. 
Since $1 \in \sigma(\Phi_{\lambda})$ (see Theorem~\ref{thm:c3-1.1} (i)), $\lambda \in \sigma(A)$ by Proposition~\ref{prop:c4-3.6}\,\ref{prop:c4-3.6-3} whence $s(A) \geq \lambda$. 
If\/ $r \coloneqq s(A)$, we deduce as in the proof of \ref{thm:c4-3.7-1} that $s(\Phi_{r}) \geq 1$. 
Now $r > \lambda$ would imply $s(\Phi_{\lambda}) > s(\Phi_{r}) \geq 1$ (by the strict monotonicity of\/ $\lambda \mapsto s(\Phi_{\lambda})$), a contradiction. Hence we conclude $s(A) = r = \lambda$.

\item 
%\ref{thm:c4-3.7-3} 
The hypotheses and Lemma \ref{lem:c4-3.4} imply that the map $\lambda \mapsto s(\Phi_{\lambda})$ is continuous. 
Let $s(\Phi_{\lambda}) > 1$. 
Since $s(\Phi_{\mu}) \leq \|\Phi_{\mu}\| \leq \|\Phi\| \cdot \|\epsilon_{\mu}\|$ we see that $s(\Phi_{\mu})$ tends to zero as $\mu \to \infty$. 
Therefore there exists $\mu' > \lambda$ such that $s(\Phi_{\mu'}) = 1$. 
\end{enumerate} 
Now Proposition~\ref{prop:c4-3.6}\,\ref{prop:c4-3.6-3}  implies $\mu' \in \sigma(A)$ whence $s(A) \geq \mu' > \lambda$.
\end{proof}
%% --
\begin{corollary}\label{cor:c4-3.8}
%
Under the hypotheses of Theorem~\ref{thm:c4-3.7}, suppose additionally that the mapping $h: \mu \mapsto s(\Phi_{\mu})$ is continuous from the right and strictly decreasing. Then the following equivalence holds.
%% --
\begin{equation}\label{eq:c4-3.9}
s(A) 
\overset{<}{ \underset{>}{=} }
%\begin{array}{c} < \\[-2mm] = \\[-2mm] > \end{array}
\lambda 
\text{ if and only if } 
s(\Phi_{\lambda}) 
\overset{<}{ \underset{>}{=} }
%\begin{array}{c} < \\[-2mm] = \\[-2mm] > \end{array}
1\,.
\end{equation}
%% --
In particular, $\lambda = s(A)$ is the only real solution of\/ $s(\Phi_{\lambda}) = 1$.
\end{corollary}
%% --
\begin{proof}
The first equivalence follows easily from Theorem~\ref{thm:c4-3.7}. 
The additional statement is a consequence of the strict monotonicity of\/ $h$.
\end{proof}
%% -- 
\begin{remarks*}\label{rem:c4-3.0-kgk}
%
\begin{enumerate}[\upshape (i), wide, labelindent=.5em] 
\item \label{rem:c4-3.1-1}
We note that in Proposition~\ref{prop:c4-3.6} and Theorem~\ref{thm:c4-3.7} it actually suffices that some power of\/ $\Phi_{\lambda}$ is compact.

\item \label{rem:c4-3.1-2}
The equivalence Eq.~\eqref{eq:c4-3.9} reduces the problem of determining $s(A)$ to the determination of the spectral bounds of the operators $\Phi_{\lambda}$ on the \enquote{smaller} Banach space $F$.
In particular, $s(A) < 0$ if and only if\/ $s(\Phi_{0}) < 1$.

\item \label{rem:c4-3.1-3}
\index{characteristic equation}
We call the identity\, \enquote{$s(\Phi_{\lambda}) = 1$} a \emph{generalized characteristic equation} (see also the remark following Part-II, Theorem~\ref{thm:b4-3.7}). The usual characteristic equation (see for example  \textcite[p.168ff]{hale:1977} and \textcite[Section~5]{heijmans:1985b}
is an equation determining all eigenvalues of the generator $A$. 
In fact, if\/ $F$ is finite dimensional, the characterization of the spectral values $\lambda$ of\/ $A$ in Proposition.~\ref{prop:c4-3.6}\,\ref{prop:c4-3.6-3} reduces to solving the complex equation $\det(\Id - \Phi_{\lambda}) = 0$. 
Obviously, there is no analogous identity characterizing $\sigma(A)$ for infinite dimensional $F$. 
However, in order to determine the long term behavior of the solutions of Eq.~\eqref{eq:c4-re}, it is often enough to know the spectral bound $s(A)$. 
Under the assumptions of Corollary~\ref{cor:c4-3.8} (in particular if\/ $\Phi$ is positive) formula Eq.~\eqref{eq:c4-3.9} gives a tool to reduce this problem to the determination of the real solution of\/ $s(\Phi_{\lambda}) = 1$.
\end{enumerate}
\end{remarks*}
%%--
\begin{example}\label{ex:c4-3.9}
%
We give an example of a large class of operators $\Phi$ satisfying the above assumptions.

For $\psi \in (L^{1}[-1,0])' = L^{\infty}[-1,0]$ and $B \in \L{F}$ we denote by $\Phi \coloneqq \psi \otimes B$ the operator defined by $\Phi(h \otimes x) = \psi(h) \cdot Bx$ for $h \in L^{1}[-1,0]$, $x \in F$.
Note that $E = (L^{1}[-1,0],F)$ is isomorphic to $L^{1}[-1,0] \otimes F$ (see \textcite[Chapter III,6.5]{schaefer:1966}).
The operator $\Phi$ is bounded from $E$ into $F$.
%% --
We assume that $\psi$ and $B$, hence $\Phi$, are positive.
Then the following holds and is stated without proof.

\begin{lemma*}\label{lem:c4-3.9-kgk}
%
The following statements hold.
\begin{enumerate}[\upshape (i), wide,labelindent=.5em] 
\item \label{lem:c4-3.9-kgk.1}
If\/ $B$ is compact, then $\Phi$ is compact. 
If\/ $B$ is surjective, then $\Phi(D(A_{0}))$ $= F$.

\item \label{lem:c4-3.9-kgk.2}
$\sigma(\Phi_{\lambda}) = \psi(\epsilon_{\lambda}) \cdot \sigma(B)$ for each $\lambda \in \C$. 
Hence the map $\mu \mapsto s(\Phi_{\mu})$ is continuous and strictly decreasing on $\R$.
\end{enumerate}
\end{lemma*}
\end{example}

For this type of \enquote{retarding functionals} $\Phi$ we obtain a simple characterization of the spectral bound.
%% --
\begin{corollary*}\label{cor:c4-3.9-kgk}
Let $\Phi = \psi \otimes B$ where $0 \leq \psi \in L^{\infty}[-1,0]$ and $0 \leq B \in L(F)$ such that $B^{n}$ is compact for some $n \in \N$.
Then the following holds.
%%--
\begin{equation}\label{eq:c4-3.10}
s(A) \overset{<}{ \underset{>}{=} } \lambda \quad 
\text{if and only if} 
\quad 
\psi(e_{\lambda}) \cdot s(B) 
\overset{<}{ \underset{>}{=} } 1 \,.
\end{equation}
%%--
\end{corollary*}
%% --
\begin{example}\label{ex:c4-3.10}
%
Let $F$ be a Banach lattice with the Dunford-Pettis property (see \textcite[Section II.9]{schaefer:1974}).
Take for example $F = C(K)$ or $F = L^{1}(X, \Sigma, \mu)$.
Furthermore define $E = L^{1}([-1,0],F)$ as usual and let $\{K(s) \colon s \in [-1,0]\}$ be a bounded family of positive, irreducible, weakly compact operators on $F$.
  \index{operator!weakly compact}
If we define $\Phi f \coloneqq \int_{-1}^{0} K(s)f(s) \, \ds$ for all $f \in E$, then (RE) has the form
%% --
\begin{equation}\label{eq:c4-3.11}
f(t) = \int_{-1}^{0} K(s)f(s+t) \, \ds \,, \quad t \geq 0 \,,\quad f_{0} = \psi \in E\,.
\end{equation}
%% --
By Corollary \ref{cor:c4-3.2}, \ Eq.~\eqref{eq:c4-3.11} has a unique solution $f \in L^{1}([-1,\infty),F)$.
For $\Phi_{\lambda}$ we obtain $\Phi_{\lambda}x = \int_{-1}^{0} \eu^{\lambda s}K(s)x \, \ds,\, x \in F$.
In this case we have 
%% --
\[
\text{$s(A) \overset{<}{ \underset{>}{=} } \lambda$  if and only if 
$s(\Phi_{\lambda}) \overset{<}{ \underset{>}{=} } 1$.}
\]
%% --
\end{example}

\begin{proof}
By Corollary \ref{cor:c4-3.8} it suffices to show that the map $h \colon \lambda \mapsto s(\Phi_{\lambda}) = r(\Phi_{\lambda})$ is strictly decreasing and continuous.
With the help of \textcite[Theorem III.11.4]{schaefer:1966} and \textcite[Theorem II.9.9]{schaefer:1974} it is easy to show that $\Phi_{\lambda}^{2}$ is compact and the continuity of\/ $h$ follows by the above remark.
It remains to show that $h$ is strictly decreasing.
Assume $s(\Phi_{\lambda}) > 0$ for all $\lambda \in \R$.
Since $\Phi_{\lambda}^{2}$ and $\Phi_{\mu}^{2}$ are compact, 
$s(\Phi_{\lambda})$ and $s(\Phi_{\mu})$ are eigenvalues of\/ $\Phi_{\lambda}$ resp.\ $\Phi_{\mu}$ with corresponding eigenfunctions $x_{\lambda}$ resp.\ $x_{\mu}$.
In the same way $s(\Phi_{\lambda})$ and $s(\Phi_{\mu})$ are eigenvalues of\/ $\Phi_{\lambda}'$ resp.\ $\Phi_{\mu}'$ with corresponding eigenfunctions $x_{\lambda}'$ resp.\ $x_{\mu}'$.
For $0 < x \in F$ and $0 < \mu < \lambda$ we obtain
%% --
\begin{align*}
\Phi_{\mu}x &= \int_{-1}^{0} \eu^{\mu s}K(s)x \, \ds = \int_{-1}^{0} \eu^{(\mu-\lambda) s}\eu^{\lambda s}K(s)x \, \ds > \int_{-1}^{0} \eu^{\lambda s}K(s)x \, \ds = \Phi_{\lambda}x
\end{align*}
%% --
since $K(s)$ are positive and irreducible operators.
In particular, we have 
%\linebreak[4] 
$\Phi_{\mu}x_{\lambda} > \Phi_{\lambda}x_{\lambda} = r(\Phi_{\lambda})x_{\lambda}$ and by evaluation $\langle\Phi_{\mu}x_{\lambda},x_{\mu}'\rangle \,>\, r(\Phi_{\lambda})\langle x_{\lambda},x_{\mu}'\rangle$.
Thus $r(\Phi_{\mu})\langle x_{\lambda},x_{\mu}'\rangle > r(\Phi_{\lambda})\langle x_{\lambda},x_{\mu}'\rangle$.
Since the operators $\Phi_{\lambda}$ are irreducible for each $\lambda$ (due to the irreducibility of\/ $K(s)$)\,, $x_{\mu}'$ is a strictly positive functional on $F$.
Hence \ $\langle x_{\lambda},x_{\mu}'\rangle \neq 0$ \ and therefore
%\linebreak[3]
$r(\Phi_{\mu}) > r(\Phi_{\lambda})$.
\end{proof}
%% --
\begin{example}\label{ex:c4-3.11}
The next example is of a more special form and occures as a model describing the cell cycle based on unequal division of cells, (see  \textcite{arinokimmel:1985}).
Let $F = L^{1}[0,1]$,\, $E = L^{1}([-1,0], F)$ and define an operator $\Phi \colon E \to F$ by
%% --
\[
\Phi(\psi)(x) \coloneqq \int_0^1 k(x,x')\psi(q(x))(x') \dx' \quad \text{for almost all} \quad x \in [0,1].
\]
%% --
Here $q$ is a continuously differentiable function with strictly positive derivative satisfying $-1 \leq q(x) \leq \epsilon < 0$ for all $x \in [0,1]$ and $k$ is a bounded, measurable, strictly positive kernel.

Then Eq.~\eqref{eq:c4-re} has the form
%% --
\begin{equation} \label{eq:c4-3.12}
\quad f(t)(x) = \int_{0}^{1} k(x,x')f(t+q(x))(x') \, \dx', \quad t \geq 0, f_{0} = \psi \in E.
\end{equation}
%% --
It is easy to show that $\phi \in \L{E,F}$.
If we define $K \in \L{F}$  as 
%% --
\[
(Kf)(x) \coloneqq \int_{0}^{1} k(x,x')f(x') \, \dx' \ ,
\ \text{we obtain} \quad 
\Phi_{\lambda}f = \eu^{\lambda q(x)}Kf \,\, (f \in F)\,.
\]
%% --
Again we have 
%%--
\begin{equation*}\label{eq:c4-3.12-kgk1}
s(A) \overset{<}{ \underset{>}{=} }\lambda \quad \text{if and only if} \quad s(\Phi_{\lambda}) \overset{<}{ \underset{>}{=} } 1 \,.
\end{equation*}
%% --
\end{example}

\begin{proof}
By Corollary \ref{cor:c4-3.8} it suffices to show that the map $h \colon \lambda \mapsto s(\Phi_{\lambda})$ is strictly decreasing and continuous.

Since $k$ is bounded, the operator $K$ is weakly compact and so is $\Phi_{\lambda}$.
Since $E$ has the Dunford-Pettis property, $(\Phi_{\lambda})^{2}$ is compact [\textcite[Theorem II.9.9]{schaefer:1974}] which yields continuity of\/ $h$.

Let $\lambda > \mu > 0$ and $0 < f \in F_{+}$.
Then 
%%$
\[
\Phi_{\mu}(x) = 
\eu^{\mu q(x)}(Kf)(x) = 
\eu^{(\mu-\lambda)q(x)}\eu^{\lambda q(x)}(Kf)(x) =
\eu^{(\mu-\lambda)q(x)} \cdot \Phi_{\lambda}f(x)\,.
\]
%%$
Since $q(x) \leq \epsilon$ for all $x \in [0,1]$, we obtain 
%% --
\[
\Phi_{\mu}f \geq \eu^{(\mu-\lambda)\epsilon} \cdot \Phi_{\lambda}f
\]
%% --
and, moreover, 
%% --
\[
(\Phi_{\mu})^n f \geq \eu^{n(\mu-\lambda)\epsilon} \cdot (\Phi_{\lambda})^n f
\]
%% --
for every $n \in \N$.
Hence 
%% --
\[
\|(\Phi_{\mu})^n\| > \eu^{n(\mu-\lambda)\epsilon}\|(\Phi_{\lambda})^n\|
\]
%% --
and consequently 
%% --
\[
r(\Phi_{\mu}) \geq \eu^{(\mu-\lambda)\epsilon}r(\Phi_{\lambda}) .
\]
%% --
Now 
%% --
\[
\text{$(\mu-\lambda)\epsilon > 0$ implies 
$r(\Phi_{\mu}) > r(\Phi_{\lambda})$.}
\]
%% --
\end{proof}

\index{equation!population}
\index{population!equation}
The theory developed so far can also be applied to certain population equations.
We first notice that (ACP) is isomorphic (in an obvious manner) to the following Cauchy problem.

For some $r \in \R_{+}$ take $E \coloneqq L^{1}([0,r], F)$ and let $A \coloneqq - \frac{\diff{}}{\dt}$ on the domain $D(A) \coloneqq \{f \in AC([0,r], F) \colon f' \in E$ and $f(0) = \Phi(f)\}$ for some $\Phi \in \L{E,F}$.

We adopt this setting and transform the above results, \eg $\epsilon_{\lambda}$ has to be defined as $\epsilon_{\lambda}(s) \coloneqq \eu^{-\lambda s}$ instead of\/ $\eu^{\lambda s}$.
As a concrete example we consider the following.



\begin{example}\label{ex:c4-3.12}
%
\index{differential equation}
%
Let $F \coloneqq \C^{n}$, $E:= L^{1}([0,r],F) = \prod_{k=1}^{n} F_{k}$, $F_{k} = L^{1}[0,r]$. and define $\Phi \colon E \to \C^{n}$ by $\Phi = (v_{ij})_{{n\times n}}$ where
%% --
\[
\langle v_{ij}, f \rangle = \int_{0}^{r} \beta_{ij}(a)f(a) \, \diff{a} \quad \text{for } f \in L^{1}[0,r] \text{ and } 0 \leq \beta_{ij} \in L^{\infty}[0,r] \,.
\]
%% --
As $\Phi_{\lambda}$ we obtain the scalar matrix
%% --
\begin{equation*}\label{eq:c4-3.12-kgk2}
\Phi_{\lambda} = \begin{pmatrix}
(\int_0^r \beta_{11}(a)\eu^{-\lambda a}\diff{a}) & (\int_0^r \beta_{12}(a)\eu^{-\lambda a}\diff{a}) & \cdots & \cdot \\
\cdot & \cdot & \cdot & \cdot \\
\cdot & \cdot & \cdot & \cdot \\
\cdot & \cdot & \cdot & \cdot \\
(\int_0^r \beta_{n1}(a)\eu^{-\lambda a}\diff{a}) & \cdots & \cdots & (\int_0^r \beta_{nn}(a)\eu^{-\lambda a}\diff{a})
\end{pmatrix}\,. 
\end{equation*}
%% --
Suppose additionally that $\Phi_{\lambda}$ is irreducible for each $\lambda$, which is, for example, satisfied if\/ $\beta_{ij}(a) > 0$ for every $a \in [0,r]$ and $1 \leq i,j \leq n$ (see also \textcite[p.257ff]{bellmancooke:1963}\,).

Since $\Phi$ has finite dimensional range and hence is compact, it follows that the function $h \colon \lambda \mapsto s(\Phi_{\lambda})$ is continuous.
Moreover one shows that $h$ is strictly decreasing by using the same arguments as in Example \ref{ex:c4-3.10} and the fact that $\Phi_{\lambda}$ is irreducible.

\index{differential equation!system of}
The system of differential equations corresponding to $A$ is
%% --
%%\begin{empheq}
\begin{equation} \label{eq:c4-3.13}
\begin{array}{l}
\frac{\partial}{\partial t}u_i(t,a) = - \frac{\partial}{\partial a}u_i(t,a) \quad (i=1,...,n) \quad \text{for } t\in\R_+, a\in[0,r]\\[2mm]
%% --
\quad \text{with initial condition} \\[2mm]
%% --
u_i(0,a) = v_i(a) \quad (i=1,...,n) \quad \text{for } a\in[0,r] \\[2mm]
%% --
\quad \text{and boundary condition} \\[2mm]
%% --
u_i(t,0) = \int_0^r [\sum_{j=1}^n \beta_{ij}(a)u_j(t,a)]\diff{a} \quad (i=1,...,n) \quad \text{for } t\in\R_+.
\end{array}
\end{equation}
%% --
%% \end{empheq}
\index{asymptotic behavior}
This system has a solution for every $(v_{1}, \ldots, v_{n}) \in D(A)$ and the asymptotic behavior is determined by the identity
%% --
\[
0 = \det
\begin{pmatrix}\scriptstyle{
 \left(1 - \int_{0}^{r} \beta_{11}(a)\eu^{-\lambda a} \diff{a}\right)} 
  & \scriptstyle{ \left(-\int_{0}^{r} \beta_{12}(a)\eu^{-\lambda a} \diff{a}\right)} 
  & \scriptstyle{\cdots} 
  & \scriptstyle{ \left(-\int_{0}^{r} \beta_{1n}(a)\eu^{-\lambda a} \diff{a}\right)} \\
\scriptstyle{ \left(-\int_{0}^{r} \beta_{21}(a)\eu^{-\lambda a} \diff{a}\right)} 
  & \scriptstyle{\ddots} 
  & \scriptstyle{\cdots}
  & \scriptstyle{\vdots} \\
\scriptstyle{\vdots}
  & \scriptstyle{\ddots}  
  & \scriptstyle{\ddots}   
  & \scriptstyle{\vdots}  \\
\vdots
  & \scriptstyle{\cdots}  
  & \scriptstyle{\cdots}
  & \scriptstyle{ \left(-\int_{0}^{r} \beta_{n-1,n}(a)\eu^{-\lambda a} \diff{a}\right)} \\
\scriptstyle{ \left(-\int_{0}^{r} \beta_{n1}(a)\eu^{-\lambda a} \diff{a}\right)} 
  & \scriptstyle{\cdots} 
  & \scriptstyle{\cdots}
  & \scriptstyle{ \left(1 - \int_{0}^{r} \beta_{nn}(a)\eu^{-\lambda a} \diff{a}\right)}
\end{pmatrix}
\]
%% --
whose unique real solution $\lambda$ is $s(A)$.

The infinite dimensional problem of determining $s(A)$ is therefore reduced to solving a one-dimensional equation.
\end{example}

The differential equation Eq.~\eqref{eq:c4-3.13} may be interpreted as follows.
\begin{quote}
Consider $n$ populations and let $r$ be the maximal age of an individual.
Further let $u_{i}(t,a)$ denote the density of the number of members of the population $i$ with respect to age $a$ at time $t$.
The birth-rate is denoted by $\beta$.
The first equation expresses the process of growing old.
The second equation defines the initial state at time zero and the last equation describes mutual dependences of birth from individuals of the $n$ populations.
\end{quote}
%% --
\begin{example}\label{ex:c4-3.13}
%
Take $F \coloneqq L^{1}(\Omega)$ where $\Omega \subset \R^{2}$ is bounded and take \\ $E \coloneqq L^{1}([0,r], F) = L^{1}([0,r] \times \Omega)$ for some $r \in \R_{+}$.
%%FN\index{operator!integral}
Furthermore, let $\Phi = \beta \otimes B$ where $0 < \beta \in L^{\infty}[0,r]$ and $B \in \L{F}$ is an integral operator with positive bounded kernel $k$.

The corresponding Cauchy problem is
%% --
\begin{equation} \label{eq:c4-3.14}
\begin{array}{l}
\frac{\partial}{\partial t}u_i(t,a) = - \frac{\partial}{\partial a}u_i(t,a) \quad (i=1,...,n) \quad \text{for } t\in\R_+, a\in[0,r]\\[2mm]
%% --
\quad \text{with initial condition} \\
%% --
u_i(0,a) = v_i(a) \quad (i=1,...,n) \quad \text{for } a\in[0,r] \\[2mm]
%% --
\quad \text{and boundary condition} \\
%% --
u_i(t,0) = \int_0^r [\sum_{j=1}^n \beta_{ij}(a)u_j(t,a)]\diff{a} \quad (i=1,...,n) \quad \text{for } t\in\R_+.
\end{array}
\end{equation}
%% --
From Theorem~\ref{thm:c4-3.1} we conclude that for every $v \in D(A)$ there exists a solution $u \in E$.
The boundedness of the integral kernel $k$ yields weak compactness of\/ $B$ (see \textcite[Section II.5]{schaefer:1974} and thus compactness of\/ $B^{2}$ by the Dunford-Pettis-Property of\/ $L^{1}(\Omega)$ (see \textcite[Chapter II, Theorem.9.9]{schaefer:1974}).
\index{equation!retarded|)}

Thus we are in the situation of Example \ref{ex:c4-3.9} and from Formula Eq.~\eqref{eq:c4-3.10} we obtain the equivalence
%% --
\[
s(A) \overset{<}{ \underset{>}{=} } \lambda \quad \text{if and only if} \quad \int_{0}^{r} \beta(a)\eu^{-\lambda a} \diff{a} \cdot s(B) \overset{<}{ \underset{>}{=} } 1\,.
\]
%% --
\end{example}
%% --
Again we can find a biological interpretation.
%% --
\index{population!age-structured}
\begin{quote}
Let $u(t,a,x)$ denote the density of the number of individuals in a given population with respect to age $a$ and position $x$ at time $t$.
As in Example \ref{ex:c4-3.13} the first equation in Eq.~\eqref{eq:c4-3.14} corresponds to the aging process.
The second equation fixes the initial state of the population and the last equation describes the dependence of newborns on the birth rate $\beta$ and the distribution of the population over the \enquote{area} $\Omega$\,.
\end{quote}
\index{retarded equation|)}
\index{retarded semigroup|)}
%% --
\section*{Notes}
\addcontentsline{toc}{section}{Notes}

\begin{enumerate}[label=\emph{Section \arabic*:}, wide, itemsep=1ex]

\item%\label{se:c4-notes-1}
Coincidence of spectral and growth bounds for $L^{1}$-spaces was proven by \textcite{derndinger:1980}.
For $L^{2}$-spaces the result is due to \textcite{greinernagel:1983}.
For the result on AM-spaces we refer to Remark~1.1 of Part-II, Chapter 4 and the corresponding notes.

Interpolation techniques in order to obtain results on arbitrary $L^{p}$-spaces were used by \textcite{voigt:1985}.
He proved Corollary \ref{cor:c4-1.2}\,\ref{cor:c4-1.2-1}.
Theorem~\ref{thm:c4-1.3} as well as Propositions \ref{prop:c4-1.6}, \ref{prop:c4-1.7}, \ref{prop:c4-1.9}  are taken from \textcite{neubrander:1985a}.
For a comprehensive discussion of the coincidence of the spectral bound $s(A)$ with other growth bounds of positive semigroups on ordered Banach spaces, see \textcite{klein:1984}.
Similar results for finite dimensional (non-lattice) ordered spaces can be found in \textcite{stern:1982}.
For general results on the convergence of the solutions of the inhomogeneous Cauchy problem we refer to \textcite{pazy:1983} and the references therein.
\index{Cauchy problem!inhomogeneous}

\item%  \label{se:c4-notes-2}
\index{semigroup!quasi-compact semigroup}
\index{Doeblin's condition}
For quasi-compact semigroups (as considered in Theorem~\ref{thm:c4-2.1} we refer to the notes of Part-II, Chapter 4,  Section~2.
Example \ref{ex:c4-2.3} is discussed in more detail in \textcite{webb:1984} and \textcite{greiner:1984}.
Further examples of this type are considered in Section~3.
It was \textcite{lotz:1986} who observed that Doeblin's condition is sufficient for quasi-com\-pactness in reflexive $L^{p}$-spaces.
(Obviously this is false in $L^{\infty}$-spaces since in this case the identity operator satisfies Doeblin's condition.)

\index{zero-two law (0-2 Law)}
The 0-2 Law for certain bounded operators on $L^{1}$ was first established by Ornstein and Sucheston.
A special case of the 0-2 Law for one-parameter semigroups (Theorem~\ref{thm:c4-2.6}) was proven by \textcite{winkler:1972} while the general result and its corollaries can be found in \textcite{greiner:1982}.
\index{operator!kernel}
Corollary \ref{cor:c4-2.11} remains true when the assumption \enquote{$T(t)$ is a kernel operator} is replaced by \enquote{$T(t)$ is an irreducible Harris operator} (see \textcite{lin:1974}).

It is well-known that semigroups play an important role in probability theory (\eg see \textcite{dynkin:1965}, \textcite{feller:1952} and \textcite{hillephillips:1957}).
A more detailed discussion than Example \ref{ex:c4-2.8} is given in Chapter~2 of \textcite{vancasteren:1985}.

Convergence to periodic solutions is investigated in \textcite{kerschernagel:1984} and \textcite{nagel:1984} where Proposition~\ref{prop:c4-2.13} is proved.
The equation considered in Example \ref{ex:c4-2.15} describes a linear model for cell division with exponential growth of individual cells.
The occurring phenomena are conjectured by \textcite{diekmannetal:1984}.

\item% \label{se:c4-notes-3}
One of the starting points in the study of retarded equations was the book of \textcite{bellmancooke:1963} on differential-difference equations.
Initiated by Hale's semigroup approach (see Part-II, Chapter 4,  Sec.3) to retarded differential equations, \textcite{dysonvillella:1979}, \textcite{villellabressan:1985} and \textcite{webb:1977} used such methods to investigate retarded equations.
These similarly apply to Volterra equations, see \textcite{miller:1974}, \textcite{webb:1977} and to 
\index{equation!population}
\index{population!equation}
age-dependent population equations, see \textcite{pruess:1981}, \textcite{webb:1984}, \textcite{webb:1985a}.
Recently, the aspect of positivity has led to statements on the asymptotic behavior of solutions of retarded equations.
In this context the investigation of population equations by \textcite{greiner:1984}, \textcite{heijmans:1985a} and \textcite{webb:1985b} should be mentioned.

\end{enumerate}
%% --
%% -- References 
%% --
\section*{References}
\addcontentsline{toc}{section}{References}
{\RaggedRight
  \printbibliography[heading=none]
 }