\documentclass[10pt]{scrartcl}
\usepackage{longtable}
\usepackage{array}
\begin{document}

\section*{Subject Index (Fortsetzung)}

\begin{longtable}{>{\bfseries}p{6cm}p{8cm}}
Term} 	& \textbf{Page Numbers \\
\hline
\endhead

Population equation 	& 251ff, 373ff, 382ff, 391ff \\

Positive part 	& 262 \\

Positive minimum principle 	& 204ff, 152ff, 280ff, 395 \\

Positive subeigenvector 	& 230 \\

Positivity 	& 138, 138, 139, 147ff, 261, 269, 270, 404 \\
	& n- 	& 404, 439 \\
	& strict 	& 138, 138, 139, 261, 269, 335, 340 \\

Predual 	& 353 \\

Projection 	& 83, 232ff, 372ff, 444ff, 454 \\
	& ergodic 	& 444ff, 455 \\
	& recurrent 	& 443 \\
	& semigroup 	& 232ff, 335, 372ff, 445 \\
	& spectral 	& 96ff \\

Pseudo-resolvent 	& 231ff, 339ff, 405ff, 421ff, 429ff, 451ff \\
	& positive 	& 326ff \\

Quasi-compact 	& 236ff, 372ff \\

Quasi-interior point 	& 265, 332 \\

Range condition 	& 60ff, 165ff, 277, 295 \\

Regular mapping 	& 269, 297, 303ff \\

Regularity 	& 405, 298, 303ff \\

Residue 	& 78ff, 83ff, 332ff, 422ff \\

Resolvent 	& 75ff, 404 \\
	& compact 	& 48, 84, 150, 187, 196, 331, 339, 364 \\
	& positive 	& 152ff \\
	& pseudo 	& 325ff, 339ff, 405ff, 421ff, 429ff, 451ff \\
	& slowly growing 	& 327ff \\

Resolvent 	& 8, 75ff, 404 \\
	& equation 	& 156, 326 \\
	& integral representation 	& 8, 320ff \\
	& positive 	& 156 \\
	& set 	& 75ff, 86 \\

Retarded 	& \\
	& differential equation 	& 241ff \\
	& equation 	& 384ff \\

Riesz Decomposition theorem 	& 264 \\

Riesz Schauder theory 	& 83ff \\

Schrödinger operator 	& 298ff, 302ff, 364 \\

Schwarz map 	& 404ff, 417ff, 419ff, 443ff \\
	& identity preserving 	& 404ff, 417ff, 419ff, 443ff \\

Schwarz inequality 	& 404 \\

Schwartz space 	& 24, 278 \\

Self-adjoint part 	& 403 \\

Semiflow 	& 162ff, 347ff \\
	& continuous 	& 163ff, 210 \\
	& injective 	& 211 \\
	& surjective 	& 211 \\

Semigroup 	& 3ff \\
	& adjoint 	& 20ff, 88, 437 \\
	& analytic 	& 94ff \\
	& bounded 	& 5 \\
	& bounded holomorphic (of angle α) 	& 94ff, 124 \\
	& compact 	& 48ff \\
	& commuting 	& 61 \\
	& contraction 	& 5, 55ff, 276ff, 310ff, 433 \\
	& convolution 	& 17 \\
	& differentiable 	& 45ff, 49 \\
	& diffusion 	& 15ff \\
	& disjointness preserving 	& 305ff \\
	& eventually compact 	& 48ff, 232, 234, 236 \\
	& eventually differentiable 	& 45, 49 \\
	& eventually norm continuous 	& 46ff, 49, 97ff, 120, 197, 330ff, 342, 365, 374 \\

F-product 	& 25ff, 85ff, 210 \\
	& holomorphic (of angle α) 	& 42ff, 49, 113, 185, 331ff, 336ff \\
	& identity preserving 	& 404ff, 417ff, 419ff, 443ff, 455 \\
	& implemented 	& 439 \\
	& induced 	& 18ff, 85ff, 229, 407 \\
	& irreducible 	& 201ff, 233, 339ff, 423ff, 444ff \\
	& lattice homomorphism 	& 155ff, 162ff, 212ff, 262, 418ff \\
	& Markovian 	& 163ff, 209 \\
	& matrix 	& 9 \\
	& mean-ergodic 	& 375 \\
	& modulus 	& 301ff, 420ff \\
	& multiplication 	& 9ff, 50ff, 76ff, 310ff \\
	& nilpotent 	& 15, 49ff, 85ff \\
	& norm continuous 	& 46ff, 49 \\
	& of Schwarz type 	& 404ff, 417ff, 443ff, 455 \\
	& one-parameter 	& 3 \\
	& partially periodic 	& 380ff, 444ff \\
	& periodic 	& 90ff, 95, 338, 448 \\
	& positive 	& 145ff \\
	& preadjoint 	& 447 \\
	& quasi-compact 	& 236ff, 372ff \\
	& quotient 	& 19, 85 \\
	& reduced 	& 407, 443 \\
	& rescaled 	& 18 \\
	& rotation 	& 14, 80, 207, 338, 380ff \\
	& similar 	& 19ff \\
	& Sobolev 	& 24ff \\
	& strongly continuous 	& 4ff \\
	& strongly ergodic 	& 442, 443ff, 455 \\
	& subspace 	& 18ff, 85 \\

\end{longtable}

\end{document}