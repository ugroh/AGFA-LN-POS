%% --
%% -- Updated Notes
%% -- Part D - final -
%% -- Stand 2025-12-16
%% --
\chapter{Updated Notes Part D}
%% --
\section*{Updated Notes D-I}
\addcontentsline{toc}{section}{Updated Notes D-I}
%% --
An overview of positive operators on operator algebras can be found in \mycite{zbmath06128372}, but there seems to be no systematic reference for positive $C_{0}$-semigroups on operator algebras.
However, many papers deal with Markov semigroups (see \eg \mycite{zbmath03762672}) or with so-called E-semigroups (see \mycite{zbmath01949821}).
%% --
\section*{Updated Notes D-II}
\addcontentsline{toc}{section}{Updated Notes D-II}
%% -- 
\begin{enumerate}
\item
As we have seen in Chapter A-II, Section 3, strongly continuous semigroups on commutative \WA-algebras, that is, on $L^{\infty}$, are already norm-continuous. 
The proof depends heavily on the Grothendieck property and the Dunford-Pettis property of these Banach spaces.

In the noncommutative case it still holds that every \WA-algebra has the Grothendieck property as shown in \mycite{zbmath00537321}, with an alternative approach in \mycite{zbmath00125315}.
Surprisingly, if every strongly continuous $C_{0}$-semigroup on a \CA-algebra has a bounded generator, then it is a Grothendieck space.
To prove this, one uses the fact that a \CA-algebra is a Grothendieck space if and only if $c_{0}$ is not a complemented subspace (see 
\mycite[Prop. 3.1.13 and Prop. 4.2.1]{zbmath07458830}).
But on $\BH$ (where $H$ is an infinite-dimensional Hilbert space), there always exist $C_{0}$-semigroups that are strongly continuous but not uniformly continuous (see the example in D-II-1.1).

\item 
It follows from A-II, Theorem 2.5 that $\BH$ with infinite-dimensional $H$ 
does not have the Dunford-Pettis property. This also follows from Example D-II-1.1 
and the fact that the Hilbert space $H$ is a direct factor of $\LH$.

For the characterization of \WA-algebras with the Dunford-Pettis property one needs the  concept of finite type I \WA-algebras.
According to \mycite[Theorem V.1.27]{zbmath01692441} a \WA-algebra  $M$ is of finite type I if and only if there exist finite dimensional Hilbert spaces $H_{j}$ and commutative \WA-algebras $M_{j}$ such that $ M = \oplus_{j} ( \mathcal{B}(H_{j}) \overline{\otimes} M_{j} )$.

In \mycite{zbmath00125315} it is shown that a \WA-algebra has the Dunford-Pettis property if and only if it is of finite type I with $\sup_{j} \dim H_{j} < \infty$.
On the other hand, by \mycite{zbmath00097659}, the predual of a \WA-algebra has the Dunford-Pettis property if and only if it is of finite type I without further restrictions on the dimensions of the Hilbert spaces.

\item 
In contrast to all of the above, a strongly continuous $C_{0}$-
-semigroup of completely positive operators on a \WA-algebra is always norm-continuous and thus has a bounded generator. According to \mycite{zbmath01495754}, this follows from the fact that for sequences of completely positive maps on \WA-algebras, strong and norm convergence to the identity operator are equivalent. 
This result---the equivalence of strong and norm convergence 
for completely positive maps---holds not just for \WA-algebras 
but extends to \AW-algebras (\mycite{zbmath01495754}).
\end{enumerate}
%%% --
\section*{Updated Notes D-III}
\addcontentsline{toc}{section}{Updated Notes D-III}
%% -- 
\begin{enumerate}
\item 
Using \mycite[Theorem 2.4.4]{zbmath03883065} and \mycite{zbmath03736445}, one can derive properties such as $s(A) \in \sigma(A)$ or $s(A) = \omega_{0}$ for positive semigroups on \CA-algebras from the theory of semigroups on certain ordered Banach spaces.
This is possible since the positive cone of a \CA-algebra possesses the required properties.

\item 
While \WA-algebras are not stable under the ultraproduct construction (see \mycite[p. 79]{zbmath03640303} or \mycite{zbmath03467832}), their preduals are and allow the spectral theoretic techniques discussed in Chapters D-III and D-IV.
More on ultraproducts of \WA-algebras can be found in \mycite{zbmath06326930}.

\item 
Another approach to the spectral theory on \WA-algebras can be found in \mycite{zbmath06031834}, where a Jacobs-de Leeuw-Glicksberg decomposition is constructed.
This leads to a noncommutative version of the Perron-Frobenius theorem for \WA-algebras and is applied to the asymptotics of \WA-dynamical systems.
A similar approach is in \mycite{zbmath06728793}.

\end{enumerate}
%% --
\section*{Updated Notes D-IV}
\addcontentsline{toc}{section}{Updated Notes D-IV}
%% --
As in D-III, results from \mycite{zbmath07964911} on semigroups on ordered Banach spaces with a normal cone and order unit can be applied. 
More precise asymptotic results on \WA-algebras and their preduals can be found in \mycite{zbmath02244715}.
