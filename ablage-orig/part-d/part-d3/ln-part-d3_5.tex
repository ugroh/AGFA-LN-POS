% !TEX root = ln-part-d3-test.tex
%% -- ln-part-d3_5 --%%

\begin{remark}\label{rem:1.3}
Take $S$ and $T$ as in Lemma 1.2 (b).
If $V_{u^*}$ (resp. $V_u$) is the map $(x \mapsto xu^*)$ (resp. $(x \mapsto xu)$) on $M$, then $V_{u^*}$ is a continuous bijection from $Ms(|\phi|)$ onto $Ms(|\phi^*|)$ with inverse $V_u$ (because $V_u \circ V_{u^*} = \operatorname{Id}_{Ms(|\phi|)}$ and $V_{u^*} \circ V_u = \operatorname{Id}_{Ms(|\phi^*|)}$).
Let $x \in M$.
From $T(xu) = S(x)u$ we obtain $T(xu)u^* = S(x)uu^*$.
In particular, if $Ms(|\phi^*|)$ is $S$-invariant, then
%% --
\[
(V_{u^*} \circ T \circ V_u)(x) = T(xu)u^* = S(x)
\]
%% --
for every $x \in Ms(|\phi^*|)$.
Let $T|$ (resp. $S|$) be the restriction of $T$ to $Ms(|\phi|)$ (resp. of $S$ to $Ms(|\phi^*|)$).
Then the following diagram is commutative:
%% --
\begin{equation*}
\begin{tikzcd}[column sep=large, row sep=large, scale=1.5]
Ms(|\phi|) \arrow[r, "T|"] \arrow[d, "V_u"'] & Ms(|\phi|) \arrow[d, "V_{u^*}"] \\
Ms(|\phi^*|) \arrow[r, "S|"'] & Ms(|\phi^*|)
\end{tikzcd}
\end{equation*}
%% --
In particular, $\sigma(S|) = \sigma(T|)$.
Therefore we may deduce spectral properties of $S|$ from $T|$ and vice versa.
More concrete applications of Lemma 1.2 will follow.
\end{remark}
%% --
We now investigate the fixed space $\operatorname{Fix}(R) := \operatorname{Fix}(\lambda R(\lambda))$, $\lambda \in D$, of a pseudo-resolvent $R$ with values in the predual of a W*-algebra $M$.

\begin{proposition}\label{prop:1.4}
Let $R$ be a pseudo-resolvent on $D = \{\lambda \in \mathbb{C}: \operatorname{Re}(\lambda) > 0\}$ with values in the predual $M_*$ of a W*-algebra $M$ and suppose $R$ to be identity preserving and of Schwarz type.

\begin{enumerate}[(a)]
\item 
If $a \in \mathbb{R}$ and $\psi \in M_*$ such that $(\gamma - ia)R(\gamma)\psi = \psi$ for some $\gamma \in D$, then $\lambda R(\lambda)|\psi| = |\psi|$ and $\lambda R(\lambda)|\psi^*| = |\psi^*|$ for all $\lambda \in D$.

\item 
$\operatorname{Fix}(R)$ is invariant under the involution in $M_*$.
If $\psi \in \operatorname{Fix}(R)$ is self adjoint, then the positive part $\psi^+$ and the negative part $\psi^-$ of $\psi$ are elements of $\operatorname{Fix}(R)$.
\end{enumerate}
\end{proposition}