% chapter-D-III.tex
% for i in {1..21}; do touch ln-part-d3_$i.tex; done

\chapter{Spectral Theory of Positive Semigroups on W*-Algebras and their Preduals}

Motivated by the classical results of Perron and Frobenius one expects the following spectral properties for the generator $A$ of a positive semigroup: The spectral bound $s(A) := \sup\{\Re(\lambda) : \lambda \in \sigma(A)\}$ belongs to the spectrum $\sigma(A)$ and the boundary spectrum
%% -- 
\[
\sigma_{b}(A) := \sigma(A) \cap \{s(A)+i\mathbb{R}\}
\]
%% -- 
possesses a certain symmetric structure, called cyclicity.

Results of this type have been proved in Chapter B-III for positive semigroups on commutative C*-algebras, but in the non-commutative case the situation is more complicated.
While \enquote{$s(A) \in \sigma(A)$} still holds (see Greiner-Voigt-Wolff (1980)) the cyclicity of the boundary spectrum $\sigma_{b}(A)$ is true only under additional assumptions on the semigroup (e.g., irreducibility, see Section 1 below).

For technical reasons we consider mostly strongly continuous semigroups on the predual of a W*-algebra $M$ or its adjoint semigroup which is a weak*-continuous semigroup on $M$.

\section{Spectral Theory for Positive Semigroups on Preduals}

The aim of this section is to develop a Perron-Frobenius theory for identity preserving semigroups of Schwarz type on W*-algebras.
But as we will show in the example preceding Theorem 1.1 below the boundary spectrum is no longer cyclic.
The appropriate hypothesis on the semigroup implying the desired results seems to be the concept of irreducibility.

