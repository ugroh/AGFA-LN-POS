% ln-part-d3_2.tex

Let us first recall some facts on normal linear functionals.
If $\phi$ is a normal linear functional on a W*-algebra $M$ then there exists a partial isometry $u\in M$ and a positive linear functional $|\phi|\in M_{*}$ such that
%% -- 
\[
\phi(x) = |\phi|(xu) =: (u|\phi|)(x), x\in M
\]
%% -- 
%% -- 
\[
u^*u = s(|\phi|),
\]
%% -- 
where $s(|\phi|)$ denotes the support projection of $|\phi|$ in $M$.
We refer to this as the \emph{polar decomposition} of $\phi$ [Takesaki (1979), Theorem III.4.2].
In addition, $|\phi|$ is uniquely determined by the following two conditions [Takesaki (1979), Proposition III.4.6]:
%% -- 
\[
	\|\phi\| = \| |\phi| \|,
\]
%% -- 
$(*)$ 
%% -- 
\[
	|\phi(x)|^{2} \leq |\phi|(xx^*) \quad (x\in M).
\]
%% -- 
For the polar decomposition of $\phi^*$, where $\phi^*(x) = \phi(x^*)^*$, we obtain
%% -- 
\[
	\phi^* = u^*|\phi^*|, \quad |\phi^*| = u|\phi|u^* \quad \text{and} \quad 		uu^* = s(|\phi^*|).
\]
%% -- 
It is easy to see that $u^*\in s(|\phi|)M$.

If $\Psi$ is a subset of the state space of a C*-algebra $M$, then $\Psi$ is called \emph{faithful} if $0 \leq x\in M$ and $\psi(x) = 0$ for all $\psi\in\Psi$ implies $x = 0$.
$\Psi$ is called \emph{subinvariant} for a positive map $T\in\mathcal{L}(M)$ (resp., positive semigroup $T$) if $T'\psi \leq \psi$ for all $\psi\in\Psi$ (resp., $T(t)'\psi \leq \psi$ for all $T(t)\in T$ and $\psi\in\Psi$).
Recall that for every positive map $T\in\mathcal{L}(M)$ there exists a state $\phi$ on $M$ such that $T'\phi = r(T)\phi$ [Groh (1981), Theorem 2.1], where $r(T)$ denotes the spectral radius of $T$.

Let us start our investigation with two lemmas.
Recall that $\Fix(T)$ is the fixed space of $T$, i.e. the set $\{x\in M: Tx=x\}$.


