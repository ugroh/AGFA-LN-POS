% continuation of chapter-A-II-1-section1.tex

In order to show uniqueness, assume that $u$ is a solution of (ACP) with initial value $0$.
We have to show that $u \equiv 0$.
Let $v(t) = \int_{0}^{t} u(s)ds$.
Then $v(t) \in D(A)$ and $Av(t) = \int_{0}^{t} Au(s)ds = \int_{0}^{t} \dot{u}(s)ds = u(t) \in D(A)$.
Consequently, $v(t) \in D(A^{2})$ for all $t\geq0$.
Moreover, $\dot{v}(t) = u(t) = Av(t)$ and $\frac{d}{dt} Av(t) = Au(t) = A\dot{v}(t) = A^{2}v(t)$.
Thus $v \in C^{1}([0,\infty),E_{1})$ and $\dot{v}(t) = A_{1}v(t)$.
Since $v(0) = 0$, it follows that $v \equiv 0$.
Thus $u \equiv v \equiv 0$.
\end{proof}

If (ACP) has a unique solution for every initial value in $D(A)$, then $A$ is the generator of a strongly continuous semigroup only if some additional assumptions on the solutions (continuous dependence from the initial value) or on $A$ $(\rho(A) \neq \emptyset)$ are made.

\begin{corollary}\label{cor:1.2}
Let $A$ be a closed operator.
Consider the following existence and uniqueness condition.

(EU) For every $f \in D(A)$ there exists a unique solution $u(\cdot,f)\in C^{1}([0,\infty),E)$ of the Cauchy problem associated with $A$ having the initial value $u(0,f) = f$.

The following assertions are equivalent.
\begin{enumerate}[(a)]
\item $A$ is the generator of a strongly continuous semigroup.
\item $A$ satisfies (EU) and $\rho(A) \neq \emptyset$.
\item $A$ satisfies (EU) and for every $\mu \in \mathbb{R}$ there exists $\lambda > \mu$ such that $(\lambda-A)D(A) = E$.
\item $A$ satisfies (EU), has dense domain and for every sequence $(f_{n})$ in $D(A)$ satisfying $\lim_{n \to \infty}f_{n} = 0$ one has $\lim_{n \to \infty}u(t,f_{n}) = 0$ uniformly in $t \in [0,1]$.
\end{enumerate}
\end{corollary}

\begin{proof}
It is clear that (i) implies the remaining assertions.
So assume that $A$ satisfy (EU).
Then by Theorem 1.1., $A_{1}$ is a generator.
If there exists $\lambda \in \rho(A)$, then $(\lambda-A)$ is an isomorphism from $E_{1}$ onto $E$ and $A$ is similar to $A_{1}$ via this isomorphism [since $D(A_{1}) = \{(\lambda-A)^{-1}f : f \in D(A)\}$ and $Af = (\lambda-A)A_{1}(\lambda-A)^{-1}f$ for all $f \in D(A)$, see A-I,3.0].
Thus $A$ is a generator on $E$ and we have shown that (ii) implies (i).

If (iii) holds, then there exists $\lambda > s(A_{1})$ such that $(\lambda-A)D(A) = E$.
We show that $(\lambda-A)$ is injective.
Then $\lambda \in \rho(A)$ since $A$ is closed.
Assume that $Af = \lambda f$ for some $f \in D(A)$.
Then $f \in D(A^{2}) = D(A_{1})$, and so $f = 0$ since $\lambda \in \rho(A_{1})$.
This proves that (iii) implies (ii).