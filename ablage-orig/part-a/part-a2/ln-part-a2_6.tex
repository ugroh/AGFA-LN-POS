% continuation of chapter-A-II-1-section1.tex

It remains to show that (iv) implies (i).
Assertion (iv) implies that for all $t \geq 0$ there exist bounded operators $T(t) \in \mathcal{L}(E)$ such that $u(t,f) = T(t)f$ if $f \in D(A)$.
Moreover, $\sup_{0\leq t\leq1} \|T(t)\| < \infty$.
It follows that $T(\cdot)f$ is strongly continuous for all $f \in E$ (since it is so for $f \in D(A)$ and $D(A)$ is dense).
Let $t > 1$.
There exist unique $n \in \mathbb{N}$ and $s \in [0,1)$ such that $t = n + s$.
Let $T(t) := T(1)^{n}T(s)$.
From the uniqueness of the solutions it follows that $T(t)f = u(t,f)$ for all $t \geq 0$ as well as $T(t+s)f = T(s)T(t)f$ for all $f \in D(A)$ and $s,t \geq 0$.
Thus $(T(t))_{t\geq0}$ is a semigroup.
Denote by $B$ its generator.
It follows from the definition that $A \subset B$.
Moreover, $D(A)$ is invariant under the semigroup.
So by A-I,Prop.1.9.(ii) $D(A)$ is a core of $B$.
Since $A$ is closed this implies that $A = B$.
\end{proof}

\begin{remark}\label{rem:1.3}
It is surprising that from condition (ii) and (iii) in the corollary it follows automatically that $D(A)$ is dense.
On the other hand this condition cannot be omitted in (iv).
In fact, consider any generator $\tilde{A}$ and its restriction $A$ with domain $D(A) = \{0\}$.
Then $A$ satisfies the remaining conditions in (iv) but is not a generator (if $\dim E > 0$).
\end{remark}

\begin{example}\label{ex:1.4}
We give a densely defined closed operator $A$, such that there exists a unique solution of (ACP) for all initial values in $D(A)$, but $A$ is not a generator.
Let $B$ be a densely defined unbounded closed operator on a Banach space $F$.
Consider $E = F \oplus F$ and $A$ on $E$ given by
%% -- 
\[
A = \begin{pmatrix} 0 & B \\ 0 & 0 \end{pmatrix}
\]
%% -- 
with domain $F \times D(B)$.

Then $D(A^{2}) = \{(f,g) \in F \times D(B) : Bg \in F\} = D(A)$ and so $A_{1} \in \mathcal{L}(E_{1})$.
In particular, $A_{1}$ is a generator.
But $A$ is not.
For instance condition (ii) in Corollary 1.2.\ does not hold, since for each $\lambda \in \mathbb{C}$,
%% -- 
\[
(\lambda-A)D(A) = \{(\lambda f-Bg,\lambda g) : f \in F, g \in D(B)\} \subset F \times D(B) \neq F \times F = E.
\]
%% -- 
So $\rho(A) = \emptyset$.
\end{example}

As a further illustration, we note that the solution of the corresponding abstract Cauchy problem for the initial value $(f,g) \in F \times D(B)$ is given by $u(t) = (f + tBg,g)$.
Since $B$ is unbounded, condition (iv) of Corollary 1.2.\ is clearly violated.