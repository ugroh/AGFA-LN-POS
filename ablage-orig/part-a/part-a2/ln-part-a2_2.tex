% chapter-A-II-1.tex

\section{The Abstract Cauchy Problem, Special Semigroups and Perturbation}

Linear differential equations in Banach spaces are intimately connected with the theory of one-parameter semigroups.
In fact, given a closed linear operator $A$ with dense domain $D(A)$ the following statement is true (with some reservation regarding a technical detail): The abstract Cauchy problem
%% -- 
\[
\begin{aligned}
\dot{u}(t) &= Au(t) \quad (t \geq 0) \\
u(0) &= f
\end{aligned}
\]
%% -- 
has a unique solution for every $f \in D(A)$ if and only if $A$ is the generator of a strongly continuous semigroup.

This is one characterization of generators which illustrates their important role for applications.
The fundamental Hille-Yosida theorem gives a different characterization in terms of the resolvent and yields a powerful tool for actually proving that a given operator is the generator of a semigroup.

Another problem we will treat here is how diverse properties of a semigroup can be described in terms of its generator.
This is a reasonable question from the theoretical point of view (since the generator uniquely determines the semigroup).
It is of interest from the practical point of view as well: the generator is the given object, defined by the differential equation.
It is useful to dispose of conditions of the generator itself giving information on the solutions (which might not be known explicitly).
We discuss smoothness properties such as analyticity, differentiability, norm continuity and compactness of the semigroup.

A frequent method to obtain new generators out of a given one is by perturbation.
We will have a brief look at this circle of problems at the end of this section.

The results are explained and illustrated by examples.
Proofs are only given when new aspects are presented which are not yet contained in the literature, otherwise we refer to the recent monographs Davies (1980), Goldstein (1985a), Pazy (1983).