% !TEX root = chap-d3-test.tex
%% -- Chapter D-III
%% --

\chapter{Spectral Theory of Positive Semigroups on \WA-Algebras and their Preduals \\
\Large{U. Groh}}\label{chap:D-III}

%Motivated by the classical results of Perron and Frobenius one expects the following spectral properties for the generator $A$ of a positive semigroup: The spectral bound $s(A) := \sup\{\Re(\lambda) : \lambda \in \sigma(A)\}$ belongs to the spectrum $\sigma(A)$ and the boundary spectrum
%%% -- 
%\[
%\sigma_{b}(A) := \sigma(A) \cap \{s(A)+i\mathbb{R}\}
%\]
%%% -- 
%possesses a certain symmetric structure, called cyclicity.
%
%Results of this type have been proved in Chapter B-III for positive semigroups on commutative C*-algebras, but in the non-commutative case the situation is more complicated.
%While \enquote{$s(A) \in \sigma(A)$} still holds (see [Greiner-Voigt-Wolff (1980)]) the cyclicity of the boundary spectrum $\sigma_{b}(A)$ is true only under additional assumptions on the semigroup (e.g., irreducibility, see Section 1 below).
%
%For technical reasons we consider mostly strongly continuous semigroups on the predual of a W*-algebra $M$ or its adjoint semigroup which is a weak*-continuous semigroup on $M$.
%
%\section{Spectral Theory for Positive Semigroups on Preduals}\label{sec:d3-1}
%
%The aim of this section is to develop a Perron-Frobenius theory for identity preserving semigroups of Schwarz type on W*-algebras.
%But as we will show in the example preceding Theorem 1.1 below the boundary spectrum is no longer cyclic.
%The appropriate hypothesis on the semigroup implying the desired results seems to be the concept of irreducibility.
%
%% ln-part-d3_2.tex
%
%Let us first recall some facts on normal linear functionals.
%If $\phi$ is a normal linear functional on a W*-algebra $M$ then there exists a partial isometry $u\in M$ and a positive linear functional $|\phi|\in M_{*}$ such that
%%% -- 
%\[
%\phi(x) = |\phi|(xu) =: (u|\phi|)(x), x\in M
%\]
%%% -- 
%%% -- 
%\[
%u^*u = s(|\phi|),
%\]
%%% -- 
%where $s(|\phi|)$ denotes the support projection of $|\phi|$ in $M$.
%We refer to this as the \emph{polar decomposition} of $\phi$ [Takesaki (1979), Theorem III.4.2].
%In addition, $|\phi|$ is uniquely determined by the following two conditions [Takesaki (1979), Proposition III.4.6]:
%%% -- 
%\[
%	\|\phi\| = \| |\phi| \|,
%\]
%%% -- 
%$(*)$ 
%%% -- 
%\[
%	|\phi(x)|^{2} \leq |\phi|(xx^*) \quad (x\in M).
%\]
%%% -- 
%For the polar decomposition of $\phi^*$, where $\phi^*(x) = \phi(x^*)^*$, we obtain
%%% -- 
%\[
%	\phi^* = u^*|\phi^*|, \quad |\phi^*| = u|\phi|u^* \quad \text{and} \quad 		uu^* = s(|\phi^*|).
%\]
%%% -- 
%It is easy to see that $u^*\in s(|\phi|)M$.
%
%If $\Psi$ is a subset of the state space of a C*-algebra $M$, then $\Psi$ is called \emph{faithful} if $0 \leq x\in M$ and $\psi(x) = 0$ for all $\psi\in\Psi$ implies $x = 0$.
%$\Psi$ is called \emph{subinvariant} for a positive map $T\in\mathcal{L}(M)$ (resp., positive semigroup $T$) if $T'\psi \leq \psi$ for all $\psi\in\Psi$ (resp., $T(t)'\psi \leq \psi$ for all $T(t)\in T$ and $\psi\in\Psi$).
%Recall that for every positive map $T\in\mathcal{L}(M)$ there exists a state $\phi$ on $M$ such that $T'\phi = r(T)\phi$ [Groh (1981), Theorem 2.1], where $r(T)$ denotes the spectral radius of $T$.
%
%Let us start our investigation with two lemmas.
%Recall that $\Fix(T)$ is the fixed space of $T$, i.e. the set $\{x\in M: Tx=x\}$.
%
%% ln-part-d3_3.tex
%
%\begin{lemma}\label{lem:d3-1.1}
%Suppose $M$ to be a C*-algebra and $T\in\mathcal{L}(M)$ an identity preserving Schwarz map.
%
%\begin{enumerate}[(i)]
%\item Let $b: M\times M \to M$ be a sesquilinear map such that for all $z\in M$ $b(z,z) \geq 0$.
%Then $b(x,x) = 0$ for some $x\in M$ if and only if $b(x,y) = 0$ and $b(y,x) = 0$ for all $y\in M$.
%
%\item If there exists a faithful family $\Psi$ of subinvariant states for $T$ on $M$, then $\Fix(T)$ is a \CA-subalgebra of $M$ and $T(xy) = xT(y)$ for all $x\in\Fix(T)$ and $y\in M$.
%
%\end{enumerate}
%\end{lemma}
%%%  --
%\begin{proof} 
%(i) Take $0 \leq \psi\in M^*$ and consider $f := \psi\circ b$.
%Then $f$ is a positive semidefinite sesquilinear form on $M$ with values in $\mathbb{C}$.
%From the Cauchy-Schwarz inequality it follows that $f(x,x) = 0$ for some $x\in M$ if and only if $f(x,y) = 0$ and $f(y,x) = 0$ for all $y\in M$.
%Since the positive cone $M^*_{+}$ is generating, assertion (a) is proved.
%
%(ii) Since $T$ is positive it follows $T(x)^* = T(x^*)$ for all $x\in M$.
%Hence $\Fix(T)$ is a self adjoint subspace of $M$, i.e. invariant under the involution on $M$.
%For every $x,y\in M$ let
%%% -- 
%\[
%	b(x,y) := T(xy^*) - T(x)T(y)^*.
%\]
%%% -- 
%Then $b$ satisfies the assumptions of (i).
%
%If $x\in\Fix(T)$ then
%%% -- 
%\[
%0 \leq xx^* = (Tx)(Tx)^* \leq T(xx^*),
%\]
%%% -- 
%hence
%%% -- 
%\[
%0 \leq \psi(T(xx^*) - xx^*) \leq 0 \quad \text{for all } \psi\in\Psi.
%\]
%%% -- 
%But this implies $T(xx^*) = T(x)T(x)^* = xx^*$.
%Consequently, $b(x,x) = 0$.
%Hence $T(xy^*) = xT(y)^*$ for all $y\in M$ and (ii) is proved.
%\end{proof}
%
%\begin{lemma}\label{lem:d3-1.2}
%Let $M$ be a \WA-algebra, $T$ an identity preserving Schwarz map on $M$ and $S\in\mathcal{L}(M)$ such that $S(x)(Sx)^* \leq T(xx^*)$ for every $x\in M$.
%
%\begin{enumerate}[(a)]
%\item If $v\in M$ such that $S(v^*) = v^*$ and $T(v^*v) = v^*v$, then $T(xv) = S(x)v$ for all $x\in M$.
%\item Suppose there exists $\phi\in M_{*}$ with polar decomposition $\phi = u|\phi|$ such that $S_{*}\phi = \phi$ and $T_{*}|\phi| = |\phi|$.
%If the closed subspace $s(|\phi|)M$ is T-invariant, then $Su^* = u^*$ and $T(u^*u) = u^*u$.
%\end{enumerate}
%\end{lemma}
%%% --
%\begin{proof}
%(a) Define a positive semidefinite sesquilinear map $b: M\times M \to M$ by
%%% -- 
%\[
%b(x,y) := T(xy^*) - S(x)S(y)^* \quad (x,y\in M).
%\]
%%% -- 
%\end{proof}
%
%%% -- ln-part-d3_4 --%%
%Since $b(v^{*},v^{*}) = 0$ we obtain $b(x,v^{*}) = 0$ for all $x \in M$ (Lemma 1.1.a), hence $T(xv) = S(x)v$.
%
%\begin{enumerate}[(a)]
%\item 
%Since $s(|\phi|)M$ is a closed right ideal, the closed face $F := s(|\phi|)(M_{+})s(|\phi|)$ determines $s(|\phi|)M$ uniquely, i.e.,
%%% --
%\[
%s(|\phi|)M = \{x \in M : xx^{*} \in F\}
%\]
%%% --
%[Pedersen (1979), Theorem 1.5.2].
%Since $T$ is a Schwarz map and $s(|\phi|)M$ is $T$-invariant, it follows $TF \subseteq F$.
%On the other hand, if $x \in s(|\phi|)M$ then $xx^{*} \in F$.
%Consequently,
%%% --
%\[
%0 \leq S(x)S(x)^{*} \leq T(xx^{*}) \in F,
%\]
%%% --
%whence $S(x) \in s(|\phi|)M$.
%
%Next we show $T(u^{*}u) = u^{*}u$ and $Su^{*} = u^{*} \in s(|\phi|)M$.
%First of all
%%% --
%\[
%0 \leq (Su^{*} - u^{*})(Su^{*} - u^{*})^{*} \leq
%\]
%%% --
%%% --
%\[
%\leq T(u^{*}u) - u^{*}S(u^{*})^{*} - (Su^{*})u + u^{*}u.
%\]
%%% --
%Since $S_{*}\phi = \phi$, $T_{*}|\phi| = |\phi|$ and $\phi = u|\phi|$ it follows
%%% --
%\[
%0 \leq |\phi|((Su^{*} - u^{*})(Su^{*} - u^{*})^{*}) \leq
%\]
%%% --
%%% --
%\[
%\leq 2|\phi|(u^{*}u) - |\phi|(S(u^{*})u)^{*} - |\phi|(S(u^{*})u) =
%\]
%%% --
%%% --
%\[
%= 2|\phi|(uu^{*}) - \phi(u^{*})^{*} - \phi(u^{*}) =
%\]
%%% --
%%% --
%\[
%= 2(|\phi|(u^{*}u) - |\phi|(u^{*}u)) = 0.
%\]
%%% --
%
%Since $(Su^{*} - u^{*})(Su^{*} - u^{*}) \in F$ and $|\phi|$ is faithful on $F$ we obtain $Su^{*} = u^{*}$.
%Consequently,
%%% --
%\[
%0 \leq u^{*}u = (Su^{*})(Su^{*})^{*} \leq T(u^{*}u).
%\]
%%% --
%
%Hence $T(u^{*}u) = u^{*}u$ by the faithfulness and $T$-invariance of $|\phi|$.
%\end{enumerate}
%
%%% -- ln-part-d3_5 --%%
%
%\begin{remark}\label{rem:1.3}
%Take $S$ and $T$ as in Lemma 1.2 (b).
%If $V_{u^*}$ (resp. $V_u$) is the map $(x \mapsto xu^*)$ (resp. $(x \mapsto xu)$) on $M$, then $V_{u^*}$ is a continuous bijection from $Ms(|\phi|)$ onto $Ms(|\phi^*|)$ with inverse $V_u$ (because $V_u \circ V_{u^*} = \operatorname{Id}_{Ms(|\phi|)}$ and $V_{u^*} \circ V_u = \operatorname{Id}_{Ms(|\phi^*|)}$).
%Let $x \in M$.
%From $T(xu) = S(x)u$ we obtain $T(xu)u^* = S(x)uu^*$.
%In particular, if $Ms(|\phi^*|)$ is $S$-invariant, then
%%% --
%\[
%(V_{u^*} \circ T \circ V_u)(x) = T(xu)u^* = S(x)
%\]
%%% --
%for every $x \in Ms(|\phi^*|)$.
%Let $T|$ (resp. $S|$) be the restriction of $T$ to $Ms(|\phi|)$ (resp. of $S$ to $Ms(|\phi^*|)$).
%Then the following diagram is commutative:
%%% --
%\begin{equation*}
%\begin{tikzcd}[column sep=large, row sep=large, scale=1.5]
%Ms(|\phi|) \arrow[r, "T|"] \arrow[d, "V_u"'] & Ms(|\phi|) \arrow[d, "V_{u^*}"] \\
%Ms(|\phi^*|) \arrow[r, "S|"'] & Ms(|\phi^*|)
%\end{tikzcd}
%\end{equation*}
%%% --
%In particular, $\sigma(S|) = \sigma(T|)$.
%Therefore we may deduce spectral properties of $S|$ from $T|$ and vice versa.
%More concrete applications of Lemma 1.2 will follow.
%\end{remark}
%%% --
%We now investigate the fixed space $\operatorname{Fix}(R) := \operatorname{Fix}(\lambda R(\lambda))$, $\lambda \in D$, of a pseudo-resolvent $R$ with values in the predual of a W*-algebra $M$.
%
%\begin{proposition}\label{prop:d3-1.4}
%Let $R$ be a pseudo-resolvent on $D = \{\lambda \in \mathbb{C}: \operatorname{Re}(\lambda) > 0\}$ with values in the predual $M_*$ of a W*-algebra $M$ and suppose $R$ to be identity preserving and of Schwarz type.
%
%\begin{enumerate}[(a)]
%\item 
%If $a \in \mathbb{R}$ and $\psi \in M_*$ such that $(\gamma - ia)R(\gamma)\psi = \psi$ for some $\gamma \in D$, then $\lambda R(\lambda)|\psi| = |\psi|$ and $\lambda R(\lambda)|\psi^*| = |\psi^*|$ for all $\lambda \in D$.
%
%\item 
%$\operatorname{Fix}(R)$ is invariant under the involution in $M_*$.
%If $\psi \in \operatorname{Fix}(R)$ is self adjoint, then the positive part $\psi^+$ and the negative part $\psi^-$ of $\psi$ are elements of $\operatorname{Fix}(R)$.
%\end{enumerate}
%\end{proposition}
%
%%% -- ln-part-d3_6 --%%
%
%\begin{proof}
%If $(\gamma - i\alpha)R(\gamma)\psi = \psi$ then $(\lambda - i\alpha)R(\lambda)\psi = \psi$ for all $\lambda \in D$.
%In particular, $\mu R(\mu + i\alpha)\psi = \psi$ ($\mu \in \mathbb{R}_+$).
%For all $x \in M$ we obtain
%%% --
%\[
%|\psi(x)|^2 = |<\mu R(\mu+i\alpha)'x,\psi>|^2 \leq
%\]
%%% --
%%% --
%\[
%\leq \|\psi\| <(\mu R(\mu+i\alpha)'x)(\mu R(\mu+i\alpha)'x)^*,\psi> \leq
%\]
%%% --
%%% --
%\[
%\leq \|\psi\| <\mu R(\mu)'(xx^*),|\psi|>
%\]
%%% --
%(D-I, Corollary 2.2).
%Since
%%% --
%\[
%\|\psi\| = \| |\psi| \| = |\psi|(1) =
%\]
%%% --
%%% --
%\[
%= <\mu R(\mu)'1,|\psi|> = \| \mu R(\mu)|\psi| \|,
%\]
%%% --
%we obtain $\mu R(\mu)|\psi| = |\psi|$ by the uniqueness theorem (*) mentioned at the beginning.
%Therefore $|\psi| \in \operatorname{Fix}(R)$.
%Since
%%% --
%\[
%0 \leq (\mu R(\mu)'x)(\mu R(\mu)'x)^* \leq \mu R(\mu)'xx^*,
%\]
%%% --
%the map $R(\mu)$ is positive.
%Consequently $(\mu+i\alpha)R(\mu)\psi^* = \psi^*$ from which $|\psi^*| \in \operatorname{Fix}(R)$ follows.
%If $\phi \in \operatorname{Fix}(R)$ is selfadjoint with Jordan decomposition $\phi = \phi^+ - \phi^-$, then $|\phi| = \phi^+ + \phi^-$ [Takesaki (1979), Theorem III.4.2.].
%From this we obtain that $\phi^+$ and $\phi^-$ are in $\operatorname{Fix}(R)$.
%\end{proof}
%%% --
%\begin{corollary}\label{cor:d3-1.5}
%Let $T$ be an identity preserving semigroup of Schwarz type on $M_*$ with generator $A$ and suppose $P\sigma(A) \cap i\mathbb{R} \neq \emptyset$.
%
%\begin{enumerate}[(i)]
%\item
%If $\alpha \in \mathbb{R}$ and $\psi \in \operatorname{ker}(i\alpha - A)$, then $|\psi|$ and $|\psi^*|$ are elements of $\operatorname{Fix}(T) = \operatorname{ker}(A)$.
%
%\item 
%$\operatorname{Fix}(T)$ is invariant under the involution of $M_*$.
%If $\psi \in \operatorname{Fix}(T)$ is self adjoint, then the positive part $\psi^+$ and the negative part $\psi^-$ of $\psi$ are elements of $\operatorname{Fix}(T)$.
%\end{enumerate}
%
%\end{corollary}
%%% --
%The proof follows immediately from D-I, Corollary 2.2 and the fact that $\operatorname{ker}(A) = \operatorname{Fix}(\lambda R(\lambda,A))$ for all $\lambda \in \mathbb{C}$ with $\operatorname{Re}(\lambda) > 0$.
%
%%% -- d3-7
%
%If $T$ is the semigroup of translations on $L^1(\mathbb{R})$ and $A'$ the genenerator 
%of the adjoint weak*-semigroup, then $P_{\sigma}(A) \cap i\mathbb{R} = \emptyset$, while $P_{\sigma}(A') \cap i\mathbb{R} = i\mathbb{R}$.
%
%For that reason we cannot expect a simple connection between these two sets.
%
%But as we shall see below, if a semigroup on the predual of a W*-algebra has sufficiently many invariant states, then the point spectra of $A$ and $A'$ contained in $i\mathbb{R}$ are identical.
%
%Helpful for these investigations will be the next lemma.
%
%\begin{lemma}\label{lem:d3-1-6}
%Let $R$ be a pseudo-resolvent on $D = \{\lambda \in \mathbb{C} : \Re(\lambda) > 0\}$ with values in a Banach space $E$ such that $\|R(\mu + i\alpha)\| \leq 1$ for all $(\mu,\alpha) \in \mathbb{R}_{+} \times \mathbb{R}$.
%
%Then
%%% --
%\[
%\dim \Fix(\lambda R(\lambda + i\alpha)) \leq \dim \Fix(\lambda R(\lambda + i\alpha)')
%\]
%%% --
%for all $\lambda \in D$.
%\end{lemma}
%
%\begin{proof}
%Let $(\mu,\alpha) \in \mathbb{R}_{+} \times \mathbb{R}$ and $S := \mu R(\mu + i\alpha)$.
%Since $S$ is a contraction, its adjoint $S'$ maps the dual unit ball $E'_{1}$ into itself.
%
%Let $U$ be a free ultrafilter on $[1,\infty)$ which converges to $1$.
%Since $E'_{1}$ is $\sigma(E',E)$-compact,
%%% --
%\[
%\psi_{o} := \lim_{U}(\lambda - 1)R(\lambda,S)'\psi
%\]
%%% --
%exists for all $\psi \in E'_{1}$.
%Since $S'$ is $\sigma(E',E)$-continuous and since $S'R(\lambda,S)' = \lambda R(\lambda,S')-\Id$ we conclude $\psi_{o} \in \Fix(S')$.
%
%Take now $0 \neq x_{o} \in \Fix(S)$ and choose $\psi \in E'_{1}$ such that $\psi(x_{o})$ is different from zero.
%
%From the considerations above it follows
%%% --
%\[
%\psi_{o}(x_{o}) = \lim_{U}(\lambda - 1)\psi(R(\lambda,S)x_{o}) = \psi(x_{o}) \neq 0
%\]
%%% --
%hence $0 \neq \psi_{o} \in \Fix(S)$.
%
%Therefore $\Fix(S')$ separates the points of $\Fix(S)$.
%
%From this it follows that
%%% --
%\[
%\dim \Fix(S) \leq \dim \Fix(S')
%\]
%%% --
%
%Since $R$ and $R'$ are pseudo-resolvents, the assertion is proved.
%\end{proof}
%
%\begin{corollary}\label{cor:d3-1-7}
%Let $T$ be a semigroup of contractions on a Banach space $E$ with generator $A$.
%Then
%%% --
%\[
%\dim \ker(i\alpha - A) \leq \dim \ker(i\alpha - A')
%\]
%%% --
%for all $\alpha \in \mathbb{R}$.
%\end{corollary}
%
%%% -- d3-8
%This follows from Lemma \ref{lem:d3-1-6} because $\Fix(\lambda R(\lambda+i\alpha)) = \ker(i\alpha-A)$.
%%% --
%\begin{proposition}\label{prop:d3-1-8}
%Let $T$ be an identity preserving semigroup of Schwarz type with generator $A$ on the predual of a W*-algebra and suppose that there exists a faithful family $\Psi$ of $T$-invariant states.
%
%Then for all $\alpha \in \mathbb{R}$ we have
%%% --
%\[
%\dim \ker(i\alpha - A) = \dim \ker(i\alpha - A')
%\]
%%% --
%and
%%% --
%\[
%P_{\sigma}(A) \cap i\mathbb{R} = P_{\sigma}(A') \cap i\mathbb{R}
%\]
%%% --
%\end{proposition}
%
%\begin{proof}
%The inequality $\dim \ker(i\alpha - A) \leq \dim \ker(i\alpha - A')$ follows from Corollary \ref{cor:d3-1-7}.
%
%Let $D = \{\lambda \in \mathbb{C} : \Re(\lambda) > 0\}$ and $R$ the pseudo-resolvent induced by $R(\lambda,A)$ on $D$.
%
%Then $R$ is identity preserving and of Schwarz type.
%
%Take $i\alpha \in P_{\sigma}(A)$ ($\alpha \in \mathbb{R}$) and choose $0 < \mu \in \mathbb{R}$.
%
%If $\psi_{\alpha} \in M_{*}$ is of norm one with polar decomposition $\psi_{\alpha} = u_{\alpha}|\psi_{\alpha}|$ such that $\psi_{\alpha} = (\mu - i\alpha)R(\mu)\psi_{\alpha}$ then $\mu R(\mu)|\psi_{\alpha}| = |\psi_{\alpha}|$ (Proposition 1.4.a).
%
%Since
%%% --
%\[
%\mu R(\mu)'(1 - s(|\psi_{\alpha}|)) \leq 1 - s(|\psi_{\alpha}|)
%\]
%%% --
%we obtain $\mu R(\mu)'s(|\psi_{\alpha}|) = s(|\psi_{\alpha}|)$ by the faithfulness of $\Psi$.
%
%Hence the maps $S := (\mu - i\alpha)R(\mu)'$ and $T := \mu R(\mu)'$ fulfil the assumptions of Lemma 1.2.b.
%
%Therefore $Su_{\alpha}^{*} = u_{\alpha}^{*}$ or $(\mu-i\alpha)R(\mu)'u_{\alpha}^{*} = u_{\alpha}^{*}$ which implies $u_{\alpha}^{*} \in D(A')$ and $A'u_{\alpha}^{*} = i\alpha u_{\alpha}^{*}$.
%
%If $i\alpha \in P_{\sigma}(A')$, $\alpha \in \mathbb{R}$, choose $0 \neq v_{\alpha}$ such that
%%% --
%\[
%v_{\alpha} = (\mu - i\alpha)R(\mu)'v_{\alpha} \quad (\mu \in \mathbb{R}_{+})
%\]
%%% --
%and $\psi \in \Psi$ such that $\psi(v_{\alpha}v_{\alpha}^{*}) \neq 0$.
%
%Since
%%% --
%\[
%0 \leq v_{\alpha}v_{\alpha}^{*} = ((\mu - i\alpha)R(\mu)'v_{\alpha})((\mu - i\alpha)R(\mu)'v_{\alpha})^{*} \leq \mu R(\mu)'(v_{\alpha}v_{\alpha}^{*})
%\]
%%% --
%we obtain $\mu R(\mu)'(v_{\alpha}v_{\alpha}^{*}) = v_{\alpha}v_{\alpha}^{*}$ because $\Psi$ is faithful.
%
%%% -- d3-9
%
%Hence from Lemma~\ref{lem:d3-1.2}, Teil (a) it follows
%%% --
%\[
%\mu R(\mu)'(xv_{\alpha}^{*}) = ((\mu - i\alpha)R(\mu)'x)v_{\alpha}^{*}
%\]
%%% --
%for all $x \in M$.
%
%Let $\psi_{\alpha}$ be the normal linear functional $(x \mapsto \psi(xv_{\alpha}^{*}))$ on $M$ and note that $\psi_{\alpha}(v_{\alpha}) \neq 0$.
%
%Then
%%% --
%\begin{align*}
%\langle x, (\mu - i\alpha)R(\mu)\psi_{\alpha} \rangle &= \langle ((\mu - i\alpha)R(\mu)'x)v_{\alpha}^{*},\psi \rangle \\
%&= \langle \mu R(\mu)'(xv_{\alpha}^{*}),\psi \rangle = \psi(xv_{\alpha}^{*}) = \psi_{\alpha}(x)
%\end{align*}
%%% --
%for all $x \in M$.
%
%Consequently $i\alpha \in P_{\sigma}(A)$ and
%%% --
%\[
%\dim \ker(i\alpha - A') \leq \dim \ker(i\alpha - A)
%\]
%%% --
%which proves the assertion.
%\end{proof}
%
%\begin{remark}\label{rem:d3-1-9}
%From the above proof we obtain the following: If $0 \neq \psi_{\alpha} \in \ker(i\alpha - A)$ with polar decomposition $\psi_{\alpha} = u_{\alpha}|\psi_{\alpha}|$ ($\alpha \in \mathbb{R}$) then $A'u_{\alpha} = i\alpha u_{\alpha}$.
%
%Conversely, if $0 \neq v_{\alpha} \in \ker(i\alpha - A')$, then there exists $\psi \in \Psi$ such that $\psi(v_{\alpha}v_{\alpha}^{*}) \neq 0$ and the normal linear form
%%% --
%\[
%\psi_{\alpha} := (x \mapsto \psi(xv_{\alpha}^{*}))
%\]
%%% --
%is an eigenvector of $A$ pertaining to the eigenvalue $i\alpha$.
%\end{remark}
%
%If $T$ is a $C_{0}$-semigroup of Markov operators on a commutative C*-algebra with generator $A$, it has been shown in B-III, that the boundary spectrum $\sigma(A) \cap i\mathbb{R}$ of its generator is additively cyclic.
%This is no longer true in the non commutative case:
%
%For $0 \neq \lambda \in i\mathbb{R}$ and $t \in \mathbb{R}$ let
%%% --
%\[
%u_{t} := \begin{pmatrix} 1 & 0 \\ 0 & e^{\lambda t} \end{pmatrix} \in M_{2}(\mathbb{C})
%\]
%%% --
%
%%% -- d3-10
%%% --
%
%I'll continue converting the document to LaTeX following your rules:
%
%The semigroup of *-automorphisms $(x \mapsto u_{t}xu_{t}^{*})$ on $M_{2}(\mathbb{C})$ is identity preserving and of Schwarz type but the spectrum of its generator is $\{0, \lambda, \lambda^{*}\}$ hence is not additively cyclic.
%
%It turns out that, in order to obtain a non commutative analogue of the Perron-Frobenius theorems, one has to consider semigroups which are irreducible.
%
%Recall that a semigroup $S$ of positive operators on an ordered Banach space $(E,E_{+})$ is called \emph{irreducible} if no closed face of $E_{+}$, different from $\{0\}$ and $E_{+}$, is invariant under $S$.
%
%Here a face $F$ in $E$ is a subcone of $E_{+}$ such that the conditions $0 \leq x \leq y$, $x \in E$, $y \in F$ imply $x \in F$ (compare Definitions 3.1 in B-III and C-III).
%
%In the context of W*-algebras $M$ we call a semigroup $S$ of positive maps on $M$ \emph{weak*-irreducible}, if no $\sigma(M,M_{*})$-closed face of $M_{+}$ is $S$-invariant.
%
%Since the norm closed faces of $M_{*}$ and the $\sigma(M,M_{*})$-closed faces of $M$ are related by formation of polars with respect to the dual system $\langle M,M_{*} \rangle$ (see [Pedersen (1979), Theorem 3.6.11 and Theorem 3.10.7.]) a semigroup $S$ is (norm) irreducible on $M_{*}$ if and only if its adjoint semigroup is weak*-irreducible.
%
%\begin{theorem}\label{thm:d3-1-10}
%Let $T$ be an irreducible, identity preserving semigroup of Schwarz type with generator $A$ on the predual of a W*-algebra and suppose $P_{\sigma}(A) \cap i\mathbb{R} \neq \emptyset$.
%
%\begin{enumerate}[(a)]
%\item The fixed space of $T$ is one dimensional and spanned by a faithful normal state.
%
%\item $P_{\sigma}(A) \cap i\mathbb{R}$ is an additive subgroup of $i\mathbb{R}$,
%%% --
%\[
%\sigma(A) = \sigma(A) + (P_{\sigma}(A) \cap i\mathbb{R})
%\]
%%% --
%and every eigenvalue in $i\mathbb{R}$ is simple.
%\end{enumerate}
%
%\begin{enumerate}[(i)]
%\item The fixed space of the adjoint weak*-semigroup $T'$ is one-dimensional.
%
%\item $P_{\sigma}(A') \cap i\mathbb{R} = P_{\sigma}(A) \cap i\mathbb{R}$ for the generator $A'$ of the adjoint semigroup, and every $\gamma \in P_{\sigma}(A') \cap i\mathbb{R}$ is simple.
%\end{enumerate}
%\end{theorem}
%
%\begin{proof}
%Since $P_{\sigma}(A) \cap i\mathbb{R} \neq \emptyset$ there exists $\psi \in \Fix(T)_{+}$ of norm one (Corollary 1.5).
%If $F := \{x \in M_{+} : \psi(x) = 0\}$ then $F$ is a $\sigma(M,M_{*})$-closed, $T'$-invariant face in $M$, hence $F = \{0\}$.
%Therefore every $0 \neq \psi \in \Fix(T)_{+}$ is faithful.
%Let $\psi_{1}, \psi_{2} \in \Fix(T)_{+}$ be states such that
%
%%% -- d3-11
%%% -- 
%
%$f := \psi_{1} - \psi_{2}$ is different from zero.
%
%If $f = f^{+} - f^{-}$ is the Jordan decomposition of $f$, then $f^{+}$ and $f^{-}$ are elements of $\Fix(T)$, whence faithful.
%Since the support projections of these two normal linear functionals are orthogonal, we obtain $f^{+} = 0$ or $f^{-} = 0$ which implies $\psi_{1} \leq \psi_{2}$ or $\psi_{2} \leq \psi_{1}$.
%
%Consequently $\psi_{2} = \psi_{1}$.
%
%Since $\Fix(T)$ is positively generated (Corollary 1.5), $\Fix(T) = \mathbb{C}\phi$ for some faithful normal state $\phi$.
%
%Let $\mu \in \mathbb{R}_{+}$ and $\alpha \in \mathbb{R}$ such that $i\alpha \in P_{\sigma}(A)$.
%
%If $\psi_{\alpha} = u_{\alpha}|\psi_{\alpha}|$ is a normalized eigenvector of $A$ pertaining to $i\alpha$, then $\phi = |\psi_{\alpha}| = |\psi_{\alpha}^{*}|$ by Corollary 1.5 and the above considerations.
%
%Hence $u_{\alpha}u_{\alpha}^{*} = u_{\alpha}^{*}u_{\alpha} = s(\phi) = 1$.
%
%Since
%%% --
%\[
%(\mu - i\alpha)R(\mu,A)\psi_{\alpha} = \psi_{\alpha}
%\]
%%% --
%and
%%% --
%\[
%\mu R(\mu,A)|\psi_{\alpha}| = |\psi_{\alpha}|
%\]
%%% --
%we obtain by Lemma 1.2.b that
%%% --
%\[
%(1) \quad \mu R(\mu,A) = V_{\alpha} \circ \mu R(\mu+i\alpha,A) \circ V_{\alpha}^{-1}
%\]
%%% --
%where $V_{\alpha}$ is the map $(x \mapsto xu_{\alpha})$ on $M$.
%
%Similarly for $i\beta \in P_{\sigma}(A)$, we find $V_{\beta}$ such that $1 = u_{\beta}u_{\beta}^{*} = u_{\beta}u_{\beta}^{*}$ and
%%% --
%\[
%(2) \quad \mu R(\mu,A) = V_{\beta} \circ \mu R(\mu+i\beta,A) \circ V_{\beta}^{-1}
%\]
%%% --
%Hence
%%% --
%\[
%(3) \quad \mu R(\mu,A) = V_{\alpha\beta} \circ \mu R(\mu+i(\alpha+\beta),A) \circ V_{\alpha\beta}^{-1}
%\]
%%% --
%where $V_{\alpha\beta} := V_{\alpha} \circ V_{\beta}$.
%
%Since $u_{\alpha}$ is unitary in $M$, it follows from (1) that $i\alpha$ is an eigenvalue which is simple because $\Fix(T) = \Fix(\mu R(\mu,A))$ is one dimensional.
%
%From (3) it follows that $i(\alpha+\beta) \in P_{\sigma}(A)$ since $0 \in P_{\sigma}(A)$ and $V_{\alpha\beta}$ is bijective.
%
%From the identity (1) we conclude that $\sigma(R(\mu,A)) = \sigma(R(\mu+i\alpha))$, which proves
%%% --
%\[
%\sigma(A) + (P_{\sigma}(A) \cap i\mathbb{R}) \subseteq \sigma(A)
%\]
%%% --
%
%The other inclusion is trivial since $0 \in P_{\sigma}(A)$.
%\end{proof}
%
%%% -- d3-12
%%% --
%
%I'll convert this part of the document to LaTeX following your rules:
%
%\begin{remarks}\label{rem:d3-1-11}
%\begin{enumerate}[(a)]
%\item Let $\phi$ be the normal state on $M$ such that $\Fix(T) = \mathbb{C}\phi$ and let $H := P_{\sigma}(A) \cap i\mathbb{R}$.
%
%From the proof of Theorem 1.10 it follows that there exists a family $\{u_{\eta} : \eta \in H\}$ of unitaries in $M$ such that $A'u_{\eta} = -\eta u_{\eta}$ and $A(u_{\eta}\phi) = \eta(u_{\eta}\phi)$ for all $\eta \in H$.
%
%\item If the group $H$ is generated by a single element, i.e., $H = i\gamma\mathbb{Z}$ for some $\gamma \in \mathbb{R}$ then the family $\{u_{\gamma}^{k} : k \in \mathbb{Z}\}$ is a complete family of eigenvectors pertaining to the eigenvalues in $H$, where $u_{\gamma} \in M$ is unitary such that $A'u_{\gamma} = i\gamma u_{\gamma}$.
%\end{enumerate}
%\end{remarks}
%
%\begin{proposition}\label{prop:d3-1-12}
%Suppose that $T$ and $M$ satisfy the assumptions of Theorem 1.10, and let $N_{*}$ be the closed linear subspace of $M_{*}$ generated by the eigenvectors of $A$ pertaining to the eigenvalues in $i\mathbb{R}$.
%
%Denote by $T_{o}$ the restriction of $T$ to $N_{*}$.
%
%Then
%\begin{enumerate}[(a)]
%\item $G := (T_{o})^{-} \subseteq L_{s}(N_{*})$ is a compact, Abelian group.
%
%\item $\Id|N_{*}\in\{T_{o}(t) : t>s\}^{-} \subseteq L_{s}(N_{*})$ for all $0<s \in \mathbb{R}$.
%\end{enumerate}
%\end{proposition}
%
%\begin{proof}
%For $\eta \in H := P_{\sigma}(A) \cap i\mathbb{R}$ let
%%% --
%\[
%U(\eta) := \{\psi \in D(A): A\psi = \eta\psi\}
%\]
%%% --
%and $U = \{U(\eta) : \eta \in H\}$.
%
%Then $(U)^{-} = N_{*}$.
%
%For each $\psi \in U$ there exists $\eta \in H$ such that
%%% --
%\[
%\{T_{o}(t)\psi : t \in \mathbb{R}_{+}\} = \{e^{-\eta t}\psi : t \in \mathbb{R}_{+}\}
%\]
%%% --
%
%Consequently this set is relatively compact in $L_{s}(N_{*})$.
%
%From [Schaefer (1966),III.4.5] we obtain that $G$ is compact.
%
%Next choose $\psi_{1}, ..., \psi_{n} \in U$, $0 < s \in \mathbb{R}$ and $\delta > 0$.
%
%Since $T_{o}(t)\psi_{i} = e^{\eta_{i}t}\psi_{i}$ $(1 \leq i \leq n)$ for some $\eta_{i} \in H$, it follows from a theorem of Kronek-ker (see, [Jacobs (1976), Satz 6.1., p.77]) that there exists $s < t$ such that
%%% --
%\[
%|(1,1, ..., 1) - (e^{\eta_{1}t}, e^{\eta_{2}t}, ..., e^{\eta_{n}t})| < \delta
%\]
%%% --
%hence
%%% --
%\[
%\sup\{\|\psi_{i} - T_{o}(t)\psi_{i}\| : 1 \leq i \leq n\} < \delta
%\]
%%% --
%or $\Id|N_{*} \in \{T_{o}(t) : t>s\}^{-} \subseteq L_{s}(N_{*})$.
%
%
%
%
%%% -- d3-13
%%% --
%
%
%Finally we prove the group property of $G$.
%
%Let $\mathcal{U}$ be an ultrafilter on $\mathbb{R}$ such that $\lim_{\mathcal{U}}T_{o}(t) = \Id$ in the strong operator topology.
%
%For positive $s \in \mathbb{R}$ let $S := \lim_{\mathcal{U}}T(t-s)$.
%
%Then $ST_{o}(s) = T_{o}(s)S = \Id$, hence $T_{o}(s)^{-1}$ exists in $G$ for all $s \in \mathbb{R}_{+}$.
%
%From this it follows that $G$ is a group.
%\end{proof}
%
%\begin{remark}\label{rem:d3-1-13}
%\begin{enumerate}[(i), wide]
%\item Let $\kappa:\mathbb{R} \to G$ be given by
%%% --
%\[
%\kappa(t) = \begin{cases}
%T_{o}(t) & \text{if } 0 \leq t \\
%T_{o}(t)^{-1} & \text{if } t \leq 0
%\end{cases}
%\]
%%% --
%Then $\kappa$ is a continuous homomorphism with dense range, i.e. $(G,\kappa)$ is solenoidal (see [Hewitt-Ross (1963)]).
%
%\item The compact group $G$ and the discret group $P_{\sigma}(A) \cap i\mathbb{R}$ are dual in the sense of locally compact Abelian groups.
%
%\item Let $(G,\kappa)$ be a solenoidal compact group and let $N_{*} = L^{1}(G)$.
%Then the induced lattice semigroup $T = (\kappa(t))_{t \geq 0}$ fulfils the assertions of Theorem 1.10.
%For example, if $G$ is the dual of $\mathbb{R}_{d}$, then $P_{\sigma}(A) \cap i\mathbb{R} = i\mathbb{R}$.
%Since the fixed space of $\kappa(t)$ is given by
%%% --
%\[
%\Fix(\kappa(t)) = (\bigcup_{k \in \mathbb{Z}} \ker(\frac{2\pi ik}{t} - A))^{--}
%\]
%%% --
%no $T(t) \in T$ is irreducible.
%
%\item If $T$ is the irreducible semigroup of Schwarz type on the predual of $B(H)$ given in [Evans (1977)] then $P_{\sigma}(A) \cap i\mathbb{R} = \emptyset$.
%\end{enumerate}
%\end{remark}
%
%\section{Spectral Properties of Uniformly Ergodic Semigoups}\label{sec:d3-2}
%
%The aim of this section is the study of spectral properties of semigroups which are uniformly ergodic, identity preserving and of Schwarz type.
%For the basic theory of uniformly ergodic semigroups on Banach spaces we refer to [Dunford-Schwartz (1958)].
%
%
%
%%% -- d3-14
%%% --
%
%
%
%Our first result yields an estimate for the dimension of the eigenspaces pertaining to eigenvalues of a pseudo-resolvent.
%
%\begin{proposition}\label{prop:d3-2-1}
%Let $R$ be an identity preserving pseudo-resolvent of Schwarz type on $D = \{\lambda \in \mathbb{C} : \Re(\lambda) > 0\}$ with values in the predual of a W*-algebra $M$.
%
%If $\Fix(\lambda R(\lambda))$ is finite dimensional for some $\lambda \in D$, then
%%% --
%\[
%\dim \Fix((\gamma - i\alpha)R(\gamma)) \leq \dim \Fix(\lambda R(\lambda))
%\]
%%% --
%for all $\gamma \in D$ and $\alpha \in \mathbb{R}$.
%\end{proposition}
%
%\begin{proof}
%By D-IV, Remark 3.2.c we may assume without loss of generality that there exists a faithful family of $R$-invariant normal states on $M$.
%
%In particular the fixed space $N$ of the adjoint pseudo-resolvent $R'$ is a W*-subalgebra of $M$ with $1 \in N$ (by Lemma 1.1.b).
%
%Since $N$ is finite dimensional there exist a natural number $n$ and a set $P := \{p_{1}, .., p_{n}\}$ of minimal, mutually orthogonal projections in $N$ such that $\sum_{k=1}^{n} p_{k} = 1$.
%
%These projections are also mutually orthogonal in $M$ with sum $1$.
%
%Let $R_{j}$ be the $\sigma(M,M_{*})$-closed right ideal $p_{j}M$ and $L_{j}$ the closed left invariant subspace $M_{*}p_{j}$ $(1 \leq j \leq n)$.
%
%The map $\mu R(\mu)'$, $\mu \in \mathbb{R}_{+}$ is an identity preserving Schwarz map.
%
%From Lemma 1.1.b we therefore obtain that for all $x \in N$ and $y \in M$,
%%% --
%\[
%\mu R(\mu)'(xy) = x(\mu R'(\mu)y)
%\]
%%% --
%
%In particular, $R_{j}$, resp., $L_{j}$ are invariant under $R'$, respectively, $R$.
%
%Furthermore, if $\psi \in L_{j}$ with polar decomposition $\psi = u|\psi|$, then $u^{*}u \leq s(|\psi|) \leq p_{j}$.
%
%Consequently, $|\psi| \in L_{j}$.
%
%Let now $\alpha \in \mathbb{R}$ and suppose that there exists $\psi_{\alpha} \in L_{j}$ of norm 1, $\psi_{\alpha} = u_{\alpha}|\psi_{\alpha}|$, such that
%%% --
%\[
%\psi_{\alpha} \in \Fix((\lambda - i\alpha)R(\lambda)) , \lambda \in D
%\]
%%% --
%
%Since $\lambda R(\lambda)|\psi_{\alpha}| = |\psi_{\alpha}|$ (Proposition 1.4), we obtain
%%% --
%\[
%\mu R(\mu)'(1-s(|\psi_{\alpha}|)) \leq (1-s(|\psi_{\alpha}|) , \mu \in \mathbb{R}_{+}
%\]
%%% --
%
%From the existence of a faithful family of $R$-invariant normal states and since $R'$ is identity preserving it follows that
%%% --
%\[
%\mu R(\mu)'s(|\psi_{\alpha}|) = s(|\psi_{\alpha}|)
%\]
%%% --
%
%
%
%
%%% -- d3-15
%%% --
%
%
%
%Thus $s(|\psi_{\alpha}|) \leq p_{j}$ and even $s(|\psi_{\alpha}|) = p_{j}$ by the minimality property of $p_{j}$.
%
%On the other hand, $\psi_{\alpha}^{*} \in \Fix((\lambda + i\alpha)R(\lambda))$.
%
%As above we obtain
%%% --
%\[
%\mu R(\mu)'s(|\psi_{\alpha}^{*}|) = s(|\psi_{\alpha}^{*}|)
%\]
%%% --
%
%Consequently, the closed left ideals $Ms(|\psi_{\alpha}^{*}|)$ and $Ms(|\psi_{\alpha}|)$ are $R'$-invariant.
%
%Next fix $\mu \in \mathbb{R}_{+}$, let $S := (\mu - i\alpha)R(\mu)'$ and $T = \mu R(\mu)'$.
%Then $(Sx)(Sx)^{*} \leq T(xx^{*})$, $S_{*}(\psi_{\alpha}^{*}) = \psi_{\alpha}^{*}$, $T_{*}(|\psi_{\alpha}^{*}|) = |\psi_{\alpha}^{*}|$, and $T$ is an identity preserving Schwarz map.
%
%Since $s(|\psi_{\alpha}^{*}|)M$ is $T$-invariant, the assumptions of Lemma 1.2 are fulfilled and we obtain for every $x \in M$
%%% --
%\[
%S(x)u_{\alpha}^{*} = T(xu_{\alpha}^{*})
%\]
%%% --
%Since the closed left ideal $Mp_{j}$ is $S$-invariant it follows
%%% --
%\[
%S(x) = T(xu_{\alpha}^{*})u_{\alpha} \, , \, x \in Mp_{j}
%\]
%%% --
%(see Remark 1.3).
%Since $u_{\alpha}$ does not depend on $\mu \in \mathbb{R}_{+}$ we obtain for all $\mu \in \mathbb{R}_{+}$
%%% --
%\[
%\mu R(\mu+i\alpha)'x = \mu R(\mu)'(xu_{\alpha}^{*})u_{\alpha}
%\]
%%% --
%Consequently, the holomorphic functions $(\mu \mapsto \mu R(\mu)'(xu_{\alpha})u_{\alpha}^{*})$ and $(\mu \mapsto \mu R(\mu+i\alpha)'x)$ coincide on $\mathbb{R}_{+}$ from which we conclude
%%% --
%\[
%\lambda R(\lambda+i\alpha)'x = \lambda R(\lambda)'(xu_{\alpha}^{*})u_{\alpha}
%\]
%%% --
%for every $\lambda \in D$ and all $x \in Mp_{j}$.
%
%Since the map $(y \mapsto yu_{\alpha})$ is a continuous bijection from $M(u_{\alpha}u_{\alpha}^{*})$ onto $Mp_{j}$ and its inverse is the map $(y \mapsto yu_{\alpha}^{*})$, we can deduce that
%%% --
%\[
%\dim \Fix((\lambda-i\alpha)R(\lambda)'|Mp_{j}) = \dim \Fix(\lambda R(\lambda)')|M(u_{\alpha}u_{\alpha}^{*}) \leq \dim \Fix(R')
%\]
%%% --
%
%Since $\bigoplus_{j=1}^{n} Mp_{j} = M$ and $\bigoplus_{j=1}^{n} L_{j} = M_{*}$ we obtain
%%% --
%\begin{align*}
%\dim \Fix((\lambda - i\alpha)R(\lambda)')) &= \dim \Fix(\lambda R(\lambda)') = \\
%&= \dim \Fix(\lambda R(\lambda))
%\end{align*}
%%% --
%
%
%
%
%%% -- d3-16
%%% --
%
%
%and the assertion follows from Lemma 1.6.
%\end{proof}
%%% --
%Before going on let us recall the basic facts of the \emph{ultrapower} $\hat{E}$ of a Banach space $E$ with respect to some free ultrafilter $\mathcal{U}$ on $\mathbb{N}$ (compare A-I,3.6).
%
%If $\ell^{\infty}(E)$ is the Banach space of all bounded functions on $\mathbb{N}$ with values in $E$, then
%%% --
%\[
%c_{\mathcal{U}}(E) := \{(x_{n}) \in \ell^{\infty}(E) : \lim_{\mathcal{U}}\|x_{n}\| = 0\}
%\]
%%% --
%is a closed subspace of $\ell^{\infty}(E)$ and equal to the kernel of the seminorm
%%% --
%\[
%\|(x_{n})\| := \lim_{\mathcal{U}}\|x_{n}\| \, , \, (x_{n}) \in \ell^{\infty}(E)
%\]
%%% --
%
%By the ultrapower $\hat{E}$ we understand the quotient space $\ell^{\infty}(E)/c_{\mathcal{U}}(E)$ with norm
%%% --
%\[
%\|\hat{x}\| = \lim_{\mathcal{U}}\|x_{n}\| \, , \, (x_{n}) \in \hat{x} \in \hat{E}
%\]
%%% --
%
%Moreover, for a bounded linear operator $T \in L(E)$, we denote by $\hat{T}$ the well defined operator $\hat{T}\hat{x} := (Tx_{n}) + c_{\mathcal{U}}(E)$, $(x_{n}) \in \hat{x}$.
%
%It is clear by virtue of $(x \mapsto (x, x, ..) + c_{\mathcal{U}}(E))$ that each $x \in E$ defines an element $\hat{x} \in \hat{E}$.
%
%This isometric embedding as well as the operator map $(T \mapsto \hat{T})$ are called canonical.
%
%In particular, if $R: (D \to L(E))$ is a pseudo-resolvent, then
%%% --
%\[
%\hat{R} := (\lambda \mapsto R(\lambda)^{\wedge}): D \to L(\hat{E})
%\]
%%% --
%is a pseudo-resolvent, too.
%
%Recall that the approximative point spectrum $A_{\sigma}(T)$ is equal to the point spectrum $P_{\sigma}(\hat{T})$ (see, e.g., [Schaefer (1974), Chapter V, §1]).
%
%This construction gives us the possibility to characterize uniformly ergodic semigroups with finite dimensional fixed space.
%
%\begin{lemma}\label{lem:d3-2-2}
%Let $R$ be a pseudo-resolvent on $D = \{\lambda \in \mathbb{C}: \Re(\lambda) > 0\}$ such that $\|R(\mu+i\alpha)\| \leq 1$ for all $(\mu,\alpha) \in \mathbb{R}_{+} \times \mathbb{R}$ and suppose
%%% --
%\[
%0 < \dim \Fix((\lambda-i\alpha)\hat{R}(\lambda)) < \infty \quad \text{for some} \quad \lambda \in D \, , \, \alpha \in \mathbb{R}
%\]
%%% --
%and the canonical extension $\hat{R}$ on some ultrapower $\hat{E}$.
%
%
%
%%% -- d3-17
%%% --
%
%
%Then the following assertions hold:
%
%\begin{enumerate}[(i)]
%\item $(\lambda - i\alpha)^{-1}$ is a pole of the resolvent $R(.,R(\lambda))$ for all $\lambda \in D$.
%
%\item $\dim \Fix((\lambda-i\alpha)R(\lambda)) = \dim \Fix((\lambda-i\alpha)\hat{R}(\lambda))$ for all $\lambda \in D$.
%
%\item $i\alpha$ is a pole of the pseudo-resolvent $R$ and the residue of $R$ and $R(.,R(\lambda))$ in $i\alpha$ respectively $(\lambda - i\alpha)^{-1}$ are identical.
%\end{enumerate}
%\end{lemma}
%
%\begin{proof}
%Take a normalized sequence $(x_{n})$ in $E$ with
%%% --
%\[
%\lim_{n}\|(\lambda - i\alpha)R(\lambda)x_{n} - x_{n}\| = 0
%\]
%%% --
%
%The existence of such a sequence follows from the fact that the fixed space of $(\lambda-i\alpha)\hat{R}(\lambda)$ is non trivial.
%
%Suppose $(x_{n})$ is not relatively compact.
%
%Then we may assume that there exists $\delta > 0$ such that
%%% --
%\[
%\|x_{n} - x_{m}\| > \delta \quad \text{for} \quad n \neq m
%\]
%%% --
%
%Take $k \in \mathbb{N}$ and let $\hat{x}_{k}$ be the image of $(x_{n+k})$ in $\hat{E}$.
%
%Since
%%% --
%\[
%\lim_{n}\|(\lambda - i\alpha)R(\lambda)x_{n+k} - x_{n+k}\| = 0
%\]
%%% --
%the so defined $\hat{x}_{k}$'s belong to $\Fix((\lambda - i\alpha)\hat{R}(\lambda))$.
%
%Since this space is finite dimensional there exist $j < \ell$ such that
%%% --
%\[
%\|\hat{x}_{j} - \hat{x}_{\ell}\| \leq \frac{\delta}{2}
%\]
%%% --
%
%From the definition of the norm in $\hat{E}$ it follows that there are natural numbers $n < m$ such that
%%% --
%\[
%\|x_{n} - x_{m}\| \leq \frac{\delta}{2}
%\]
%%% --
%which leads to a contradiction.
%
%Therefore every approximate eigenvector of $(\lambda - i\alpha)R(\lambda)$ pertaining to $\alpha$ is relatively compact.
%In particular it has a convergent subsequence from which it follows that the fixed space of $(\lambda - i\alpha)R(\lambda)$ is non trivial.
%
%Obviously
%%% --
%\[
%\dim \Fix((\lambda - i\alpha)R(\lambda)) \leq \dim \Fix((\lambda - i\alpha)\hat{R}(\lambda))
%\]
%%% --
%
%
%
%
%%% -- d3-18
%%% --
%
%
%If the last inequality is strict, then there exists $\gamma > 0$ and a normalized $\hat{x} \in \Fix((\lambda - i\alpha)\hat{R}(\lambda))$ such that
%%% --
%\[
%\gamma \leq \|\hat{y} - \hat{x}\|
%\]
%%% --
%for all $y \in \Fix((\lambda - i\alpha)R(\lambda))$.
%
%Take a normalized sequence $(x_{n}) \in \hat{x}$.
%
%Then $(x_{n})$ has a convergent subsequence whence we may assume that $\lim_{n} x_{n} = z$ exists in $E$.
%
%Thus $0 \neq z \in \Fix((\lambda - i\alpha)R(\lambda))$.
%
%From this we obtain the contradiction
%%% --
%\[
%\gamma \leq \|\hat{z} - \hat{x}\| = \lim \|z - x_{n}\| = 0
%\]
%%% --
%
%Consequently
%%% --
%\[
%\dim \Fix((\lambda - i\alpha)R(\lambda)) = \dim \Fix((\lambda - i\alpha)\hat{R}(\lambda))
%\]
%%% --
%
%Let $\{x_{1},...,x_{n}\}$ be a base of $\Fix((\lambda - i\alpha)R(\lambda))$ and choose $\{\phi_{1},...,\phi_{n}\}$ in $\Fix((\lambda - i\alpha)R(\lambda)')$ such that $\phi_{k}(x_{j}) = \delta_{k,j}$ (Lemma 1.6).
%
%Then
%%% --
%\[
%E = \Fix((\lambda - i\alpha)R(\lambda)) \oplus (\bigcap_{j=1}^{n} \ker\phi_{j})
%\]
%%% --
%where both subspaces on the right are $(\lambda - i\alpha)R(\lambda)$-invariant and $1$ is a pole of $(\lambda-i\alpha)R(\lambda)|_{\Fix((\lambda-i\alpha)R(\lambda))}$ by the finite dimensionality of $\Fix((\lambda-i\alpha)R(\lambda))$.
%
%Suppose $1$ belongs to the spectrum of $S$ where $S$ is the restriction of $(\lambda-i\alpha)R(\lambda)$ to $\bigcap_{j=1}^{n} \ker\phi_{j}$.
%
%Then there exists a normalized sequence $(y_{n})$ in $\bigcap_{j=1}^{n} \ker\phi_{j}$ such that
%%% --
%\[
%\lim_{n} \|(\lambda - i\alpha)R(\lambda)y_{n} - y_{n}\| = 0
%\]
%%% --
%
%Therefore $(y_{n})$ has an accumulation point different from zero in
%%% --
%\[
%\Fix((\lambda - i\alpha)R(\lambda)) \cap (\bigcap_{j=1}^{n} \ker\phi_{j})
%\]
%%% --
%
%This contradiction implies that $1$ does not belong to the spectrum of $S$.
%
%Since $\Fix((\lambda - i\alpha)R(\lambda))$ is finite dimensional, it follows from general spectral theory that $(\lambda - i\alpha)^{-1}$ is a pole of $R(.,R(\lambda))$ for every $\lambda$.
%
%Thus (a) and (b) are proved.
%
%Assertion (c) follows from the resolvent equality as in the proof of [Greiner (1981), Proposition 1.2].
%
%\end{proof}
%
%
%%% -- d3-19
%%% --
%
%
%\begin{proposition}\label{prop:d3-2-3}
%Let $T$ be a semigroup of contractions on a Banach space $E$ with generator $A$.
%Then the following assertions are equivalent:
%\begin{enumerate}[(a)]
%\item
%Each $i\alpha$, $\alpha \in \mathbb{R}$, is a pole of the resolvent $R(.,A)$ such that the corresponding residue has finite rank.
%\item
%$\dim \Fix((\lambda - i\alpha)\hat{R}(\lambda,A)) < \infty$ for some (hence all) $\lambda \in \mathbb{C}$, $\Re(\lambda) > 0$ and the canonical extensions $\hat{R}(\lambda,A)$ of $R(\lambda,A)$ to some ultrapower.
%\end{enumerate}
%\end{proposition}
%
%\begin{proof}
%Let $P_{\alpha}$ be the residue of the resolvent $R(.,A)$ in $i\alpha$.
%Then $P_{\alpha} = \lim_{\lambda \to i\alpha}(\lambda - i\alpha)R(\lambda,A)$ in the operator norm of $L(E)$.
%Since the canonical map $(T \mapsto \hat{T})$ is isometric and since $\hat{E}$ is an ultrapower, we obtain
%%% --
%\[
%\hat{P}_{\alpha} = \lim_{\lambda \to i\alpha}(\lambda - i\alpha)\hat{R}(\lambda,A)
%\]
%%% --
%in $L(\hat{E})$ and $\rank(P_{\alpha}) = \rank(\hat{P}_{\alpha})$.
%Because of
%%% --
%\[
%\hat{P}_{\alpha}(\hat{E}) = \Fix((\lambda-i\alpha)\hat{R}(\lambda))
%\]
%%% --
%one part of the corollary is proved. The other follows from Lemma 2.2.
%\end{proof}
%
%\begin{remarks}\label{rem:d3-2-4}
%\begin{enumerate}[(a)]
%\item
%By the results in [Lin (1974)] a semigroup of contractions on a Banach space is uniformly ergodic if and only if $0$ is a pole of the generator with order $\leq 1$.
%The residue of the resolvent in $0$ and the associated ergodic projection are identical.
%\item
%Let $M$ be a W*-algebra with predual $M_{*}$, $\mathcal{U}$ a free ultrafilter on $\mathbb{N}$ and $\hat{M}$ (resp. $(M_{*})^{\wedge}$) the ultrapower of $M$ (resp. $M_{*}$) with respect to $\mathcal{U}$.
%Then it is easy to see that $c_{\mathcal{U}}(M)$ is a two sided ideal in $\ell^{\infty}(M)$ hence $\hat{M}$ is a C*-algebra, but in general not a W*-algebra.
%Note that the unit of $\hat{M}$ is the canonical image of $1$.
%For $\hat{x} \in \hat{M}$ and $\hat{\phi} \in (M_{*})^{\wedge}$ let $J: (M_{*})^{\wedge} \to \hat{M}'$ be defined by
%%% --
%\[
%\langle x,J(\hat{\phi}) \rangle := \lim_{\mathcal{U}}\phi_{n}(x_{n}) \, , \, (x_{n}) \in \hat{x} \, , \, (\phi_{n}) \in \hat{\phi}
%\]
%%% --
%$J$ is well defined and is an isometric embedding.
%It turns out that $J((M_{*})^{\wedge})$ is a translation invariant subspace of $(\hat{M}')^{\wedge}$.
%Hence there exists a central projection $z \in \hat{M}''$ such that $J((M_{*})^{\wedge}) = \hat{M}''z$ [Groh (1984), Proposition 2.2].
%\end{enumerate}
%\end{remarks}
%
%
%
%
%%% -- d3-20
%%% --
%
%Below we identify $(M_{*})^{\wedge}$ via $J$ with this translation invariant subspace.
%From the construction the following is obvious: If $T$ is an identity preserving Schwarz map with preadjoint $T_{*} \in L(M_{*})$, then $\hat{T}$ is an identity preserving Schwarz map on $\hat{M}$ such that $(T_{*})^{\wedge} = \hat{T}'|(M_{*})^{\wedge}$.
%
%\begin{theorem}\label{thm:d3-2-5}
%Let $T$ be an identity preserving semigroup of Schwarz type with generator $A$ on the predual of a W*-algebra $M$.
%If $T$ is uniformly ergodic with finite dimensional fixed space, then every $\gamma \in \sigma(A) \cap i\mathbb{R}$ is a pole of the resolvent $R(.,A)$ and $\dim \ker(\gamma - A) \leq \dim \Fix(T)$.
%\end{theorem}
%
%\begin{proof}
%Let $D = \{\lambda \in \mathbb{C} : \Re(\lambda) > 0\}$ and $R$ the $M_{*}$-valued pseudo-resolvent of Schwarz type induced by $R(.,A)$ on $D$.
%Then
%%% --
%\[
%P = \lim_{\mu \downarrow 0}\mu R(\mu)
%\]
%%% --
%exists in the uniform operator topology and $\rank(P) = \dim \Fix(T) < \infty$.
%From this we obtain $\rank(P) = \rank(\hat{P}) < \infty$ where $\hat{P}$ is the canonical extension of $P$ onto $(M_{*})^{\wedge}$.
%Since $\hat{P} = \lim_{\mu \downarrow 0} \mu R(\mu)^{\wedge}$ it follows that
%%% --
%\[
%\dim \Fix((\lambda - i\alpha)\hat{R}(\lambda)) \leq \rank(\hat{P}) < \infty
%\]
%%% --
%(Proposition 2.1) for all $\alpha \in \mathbb{R}$.
%Therefore the assertion follows from Lemma 2.2.
%\end{proof}
%
%The consequences of this result for the asymptotic behavior of one-parameter semigroups will be discussed in D-IV, Section 4.
%
%\section*{Notes}\label{notes:d3-notes}
%
%\begin{enumerate}[label=\emph{Section \arabic*:}, wide]
%
%\item
%The Perron-Frobenius theory for a single positive operator on a non-commutative operator algebra is worked out in [Albeverio-Höegh-Krohn (1978)] and [Groh (1981)].
%
%
%
%
%%% -- d3-21
%%% --
%
%
%The limitations of the theory (in the continuous as in the discrete case)
%are explained by the example following Remark 1.9 (see also [Groh (1982a)]).
%Therefore we concentrate on irreducible semigroups.
%Our main result (Theorem 1.10) extends B-III, Thm.3.6 to the non-commutative setting.
%
%\item
%Theorem 2.5 has its roots in the Niiro-Sawashima Theorem for a single irreducible positive operator on a Banach lattice (see [Schaefer (1974), V.5.4]).
%The analogous semigroup result on Banach lattices is due to [Greiner (1982)].
%The ultrapower technique in our proof is developed in [Groh (1984b)].
%
%\end{enumerate}
%



































