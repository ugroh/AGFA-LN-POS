\documentclass{article}
\usepackage{amsmath}
\usepackage{amsthm}
\usepackage{amssymb}
\usepackage{array}
\usepackage[inline,shortlabels]{enumitem}

\newtheorem{definition}{Definition}
\newtheorem{theorem}{Theorem}
\newtheorem{proposition}{Proposition}
\newtheorem{remark}{Remark}
\newtheorem{example}{Example}

\begin{document}

\section{ONE-PARAMETER SEMIGROUPS ON BANACH SPACES}

\subsection*{CHAPTER A-I}

\subsection*{BASIC RESULTS ON SEMIGROUPS ON BANACH SPACES}

Since the basic theory of one-parameter semigroups can be found in several excellent books (e.g. Davies (1980), Goldstein (1985a), Pazy (1983) or Hille-Phillips (1957)) we do not want to give a self-contained introduction to this subject here.
It may however be useful to fix our notation, to collect briefly some important definitions and results (Section 1), to present a list of standard examples (Section 2) and to discuss standard constructions of new semigroups from a given one (Section 3).
In the entire chapter we denote by $E$ a (real or) complex Banach space and consider one-parameter semigroups of bounded linear operators $T(t)$ on $E$.
By this we understand a subset $\{T(t): t \in \mathbb{R}_{+}\}$ of $L(E)$, usually written as $(T(t))_{t \geq 0}$, such that
$T(0) = Id$
$T(s+t) = T(s) \cdot T(t)$ for all $s, t \in \mathbb{R}_{+}$.
In more abstract terms this means that the map $t \mapsto T(t)$ is a homomorphism from the additive semigroup $\mathbb{R}_{+}$ into the multiplicative semigroup $(L(E), \cdot)$.
Similarly, a one-parameter group $(T(t))_{t \in \mathbb{R}}$ will be a homomorphic image of the group $(\mathbb{R},+)$ in $(L(E), \cdot)$.

\section{STANDARD DEFINITIONS AND RESULTS}

We consider a one-parameter semigroup $(T(t))_{t \geq 0}$ on a Banach space $E$ and observe that the domain $\mathbb{R}_{+}$ and the range $L(E)$ of the (semigroup) homomorphism $\tau: t \mapsto T(t)$ are topological semigroups for the natural topology on $\mathbb{R}_{+}$ and any one of the standard operator topologies on $L(E)$.
We single out the strong operator topology on $L(E)$ and require $\tau$ to be continuous.

\begin{definition}
A one-parameter semigroup $(T(t))_{t \geq 0}$ is called strongly continuous if the map $t \mapsto T(t)$ is continuous for the strong operator topology on $L(E)$, i.e., $\lim_{t \to t_0} \|T(t)f - T(t_0)f\| = 0$ for every $f \in E$ and $t, t_0 \geq 0$.
\end{definition}

Clearly one defines in a similar way weakly continuous, resp. uniformly continuous (compare A-II, Def. 1.19) semigroups, but since we concentrate on the strongly continuous case we agree on the following terminology:
From now on 'semigroup' always means strongly continuous one-parameter semigroup of bounded linear operators.

Next we collect a few elementary facts on the continuity and boundedness of one-parameter semigroups.

\begin{remark}

\begin{enumerate}[(i)]
\item
A one-parameter semigroup $(T(t))_{t \geq 0}$ on a Banach space $E$ is strongly continuous if and only if for any $f \in E$ it is true that $T(t)f \to f$ as $t \to 0$.

\item
For every strongly continuous semigroup $(T(t))_{t \geq 0}$ there exist constants $M \geq 1$, $\omega \in \mathbb{R}$ such that $\|T(t)\| \leq M \cdot e^{\omega t}$ for every $t \geq 0$.

\item
If $(T(t))_{t \geq 0}$ is a one-parameter semigroup such that $\|T(t)\|$ is bounded for $0 \leq t \leq \delta$ then it is strongly continuous if and only if $\lim_{t \to 0} T(t)f = f$ for every $f$ in a total subset of $E$.

\end{enumerate}

\end{remark}

\begin{definition}
By the growth bound (or type) of the semigroup $(T(t))_{t \geq 0}$ we understand the number
\begin{equation}
\omega := \inf\{w \in \mathbb{R}: \text{There exists } M \in \mathbb{R}_{+} \text{ such that } \|T(t)\| \leq M e^{wt} \text{ for } t \geq 0\}
\end{equation}
$= \lim_{t \to \infty} \frac{1}{t} \cdot \log\|T(t)\| = \inf_{t > 0} \frac{1}{t} \cdot \log\|T(t)\|$
\end{definition}

Particularly important are semigroups such that for every $t \geq 0$ we have $\|T(t)\| \leq M$ (bounded semigroups) or $\|T(t)\| \leq 1$ (contraction semigroups).
In both cases we have $\omega \leq 0$.
It follows from the subsequent examples and from 3.1 that $\omega$ may be any number $-\infty \leq \omega < +\infty$.
Moreover the reader should observe that the infimum in (1.1) need not be attained and that $M$ may be larger than 1 even for bounded semigroups.

\begin{example}
\begin{enumerate}[(i)]
\item
Take $E = \mathbb{C}^2$, $A = \begin{pmatrix} 0 & 1 \\ 0 & 0 \end{pmatrix}$ and $T(t) = e^{tA} = \begin{pmatrix} 1 & t \\ 0 & 1 \end{pmatrix}$.
Then for the 1-norm on $E$ we obtain $\|T(t)\| = 1 + t$, hence $(T(t))_{t \geq 0}$ is an unbounded semigroup having growth bound $\omega = 0$.

\item
Take $E = L^1(\mathbb{R})$ and for $f \in E$, $t \geq 0$ define
\[
T(t)f(x) := \begin{cases}
2 \cdot f(x+t) & \text{if } x \in [-t,0] \\
f(x+t) & \text{otherwise.}
\end{cases}
\]
Each $T(t)$, $t > 0$, satisfies $\|T(t)\| = 2$ as can be seen by taking $f := 1_{[0,t]}$.
Therefore $(T(t))_{t \geq 0}$ is a strongly continuous semigroup which is bounded, hence has $\omega = 0$, but the constant $M$ in (1.1) cannot be chosen to be 1.

\end{enumerate}
\end{example}

\begin{definition}
To every semigroup $(T(t))_{t \geq 0}$ there belongs an operator $(A,D(A))$, called the generator and defined on the domain
\[
D(A) := \left\{f \in E: \lim_{h \to 0} \frac{T(h)f-f}{h} \text{ exists in } E\right\}
\]
by $Af := \lim_{h \to 0} \frac{T(h)f-f}{h}$ for $f \in D(A)$.
\end{definition}

Clearly, $D(A)$ is a linear subspace of $E$ and $A$ is linear from $D(A)$ into $E$.
Only in certain special cases (see 2.1) the generator is everywhere defined and therefore bounded (use Prop. 1.9(i)).
In general the precise extent of the domain $D(A)$ is essential for the characterization of the generator.
But since the domain is canonically associated to the generator of a semigroup we shall write in most cases $A$ instead of $(A,D(A))$.

\begin{proposition}
For the generator $A$ of a semigroup $(T(t))_{t \geq 0}$ on a Banach space $E$ the following assertions hold:
(i) If $f \in D(A)$ then $T(t)f \in D(A)$ for every $t \geq 0$.
(ii) The map $t \mapsto T(t)f$ is differentiable on $\mathbb{R}_{+}$ if and only if $f \in D(A)$.
In that case one has
\begin{equation}
\frac{d}{dt}T(t)f = AT(t)f = T(t)Af.
\end{equation}
(iii) For every $f \in E$ one has $\int_0^t T(s)f ds \in D(A)$ and
\begin{equation}
A\int_0^t T(s)f ds = T(t)f - f.
\end{equation}
(iv) If $f \in D(A)$ then
\[
T(t)f = f + \int_0^t AT(s)f ds = f + \int_0^t T(s)Af ds.
\]
(v) The domain $D(A)$ is dense in $E$.
\end{proposition}

\begin{theorem}
Let $(A,D(A))$ be the generator of a strongly continuous semigroup $(T(t))_{t \geq 0}$ on the Banach space $E$.
Then the 'abstract Cauchy problem' (ACP)
\[
\frac{d}{dt}\xi(t) = A\xi(t), \quad \xi(0) = f_0,
\]
has a unique solution $\xi: \mathbb{R}_{+} \to D(A)$ in $C^1(\mathbb{R}_{+},E)$ for every $f_0 \in D(A)$.
In fact, this solution is given by $\xi(t) := T(t)f_0$.
\end{theorem}

For the important relation of semigroups to abstract Cauchy problems we refer to A-II, Section 1.
Here we only point out that the above theorem implies that a semigroup is uniquely determined by its generator.
While the generator is bounded only for uniformly continuous semigroups (see 2.1 below), it always enjoys a weaker but useful property.

\begin{definition}
An operator $B$ with domain $D(B)$ on a Banach space $E$ is called closed if $D(B)$ endowed with the graph norm
\[
\|f\|_B := \|f\| + \|Bf\|
\]
becomes a Banach space.
Equivalently, $(B,D(B))$ is closed if and only if its graph $\{(f,Bf): f \in D(B)\}$ is closed in $E \times E$, i.e.
\begin{equation}
f_n \in D(B), f_n \to f \text{ and } Bf_n \to g \text{ implies } f \in D(B) \text{ and } Bf = g.
\end{equation}
\end{definition}

It is clear from this definition that the 'closedness' of an operator $B$ depends very much on the size of the domain $D(B)$.
For example, a bounded and densely defined operator $(B,D(B))$ is closed if and only if $D(B) = E$.
On the other hand it may happen that $(B,D(B))$ is not closed but has a closed extension $(C,D(C))$, i.e., $D(B) \subset D(C)$ and $Bf = Cf$ for every $f \in D(B)$.
In that case, $B$ is called closable, a property which is equivalent to the following:
\begin{equation}
f_n \in D(B), f_n \to 0 \text{ and } Bf_n \to g \text{ implies } g = 0.
\end{equation}

The smallest closed extension of $(B,D(B))$ will be called the closure $\bar{B}$ with domain $D(\bar{B})$.
In other words, the graph of $\bar{B}$ is the closure of $\{(f,Bf): f \in D(B)\}$ in $E \times E$.
Finally we call a subset $D_0$ of $D(B)$ a core for $B$ if $D_0$ is $\|\cdot\|_B$-dense in $D(B)$.
This means that a closed operator is determined (via closure) by its restriction to a core in its domain.

\begin{proposition}
For the generator $A$ of a strongly continuous semigroup $(T(t))_{t \geq 0}$ the following holds:
(i) The generator $A$ is a closed operator.
(ii) If a subspace $D_0$ of the domain $D(A)$ is dense in $E$ and $(T(t))$-invariant, then it is a core for $A$.
(iii) Define $D(A^n) := \{f \in D(A^{n-1}): Af \in D(A^{n-1})\}$, $D(A^1) = D(A)$.
Then $D(A^\infty) := \bigcap_{n \in \mathbb{N}} D(A^n)$ is dense in $E$ and a core for $A$.
\end{proposition}

\begin{example}
Property (iii) above does not hold for general densely defined closed operators.
Take $E = C[0,1]$, $D(B) = C^1[0,1]$ and $Bf = q \cdot f$ for some nowhere differentiable function $q \in C[0,1]$.
Then $B$ is closed, but $D(B^2) = (0)$.
\end{example}

\begin{proposition}
For the generator $A$ of a strongly continuous semigroup $(T(t))_{t \geq 0}$ on a Banach space $E$ the following holds.
If $\int_0^\infty e^{-\lambda t}T(t)f dt$ exists for every $f \in E$ and some $\lambda \in \mathbb{C}$, then $\lambda \in \rho(A)$ and $R(\lambda,A)f = \int_0^\infty e^{-\lambda t}T(t)f dt$.
In particular,
\begin{equation}
R(\lambda,A)^{n+1}f = \frac{(-1)^n}{n!}\left(\frac{d}{d\lambda}\right)^n R(\lambda,A)f = \int_0^\infty e^{-\lambda t}\frac{t^n}{n!}T(t)f dt
\end{equation}
for $n \in \mathbb{N}$, $f \in E$ and all $\lambda$ with $\operatorname{Re}\lambda > \omega$.
\end{proposition}

\begin{remark}
(1) For continuous Banach space valued functions such as $t \mapsto T(t)f$ we consider the Riemann integral and define $\int_0^\infty T(t)f dt$ as $\lim_{t \to \infty} \int_0^t T(s)f ds$.
Sometimes such integrals for strongly continuous semigroups $(T(t))_{t \geq 0}$ are written as $\int_a^b T(t)dt$ and understood in the strong sense.
(2) Since the generator $(A,D(A))$ determines the semigroup $(T(t))_{t > 0}$ uniquely, we will speak occasionally of the growth bound of the generator instead of the semigroup, i.e., we write $\omega = \omega(A) = \omega((T(t))_{t \geq 0})$.
(3) For one-parameter groups it might seem to be more natural to define the generator as the `derivative' rather than just the `right derivative' at $t = 0$.
This yields the same operator as the following result shows:
The strongly continuous semigroup $(T(t))_{t \geq 0}$ with generator $A$ can be extended to a strongly continuous one-parameter group $(U(t))_{t \in \mathbb{R}}$ if and only if $-A$ generates a semigroup $(S(t))_{t \geq 0}$.
In that case $(U(t))_{t \in \mathbb{R}}$ is obtained as
\[
U(t) := \begin{cases}
T(t) & \text{for } t \geq 0 \\
S(-t) & \text{for } t \leq 0
\end{cases}
\]
We refer to [Davies (1980), Prop.~1.14] for the details.
\end{remark}

\end{document}