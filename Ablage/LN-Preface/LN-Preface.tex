\documentclass{article}
\usepackage{amsmath}
\usepackage{amsthm}
\usepackage{amssymb}
\usepackage{array}
\usepackage{booktabs}
\usepackage{enumitem}
\usepackage[english]{babel}
\usepackage{csquotes}




\begin{document}

\section*{Preface}

As early as 1948 in the first edition of his fundamental treatise on Semigroups and Functional Analysis, E.~Hille expressed the need for \enquote{developing an adequate theory of transformation semigroups operating in partially ordered spaces} (l.c., Foreword). 
In the meantime the theory of one-parameter semigroups of positive linear operators has grown continuously. 
Motivated by problems in probability theory and partial differential equations W.~Feller (1952) and R.~S.~Phillips (1962) laid the first cornerstones by characterizing the generators of special positive semigroups. 
In the 60's and 70's the theory of positive operators on ordered Banach spaces was built systematically and is well documented in the monographs of H.~H.~Schaefer (1974) and A.~C.~Zaanen (1983). 
But in this process the original ties with the applications and, in particular, with initial value problems were at times obscured. 
Only in recent years an adequate and up-to-date theory emerged, largely based on the techniques developed for positive operators and thus recombining the functional analytic theory with the investigation of Cauchy problems having positive solutions to each positive initial value. 
Even though this development --- in particular with respect to applications to concrete evolution equations in transport theory, mathematical biology, and physics --- is far from being complete, the present volume is a first attempt to shape the multitude of available results into a coherent theory of one-parameter semigroups of positive linear operators on ordered Banach spaces.

The book is organized as follows.
We concentrate our attention on three subjects of semigroup theory: \emph{characterization}, \emph{spectral theory} and \emph{asymptotic behavior}. 
By \emph{characterization}, we understand the problem of describing special properties of a semigroup, such as positivity, through the generator. 
By \emph{spectral theory} we mean the investigation of the spectrum of a generator. 
\emph{Asymptotic behavior} refers to the orbits of the initial values under a given semigroup and phenomena such as stability.

This program (characterization, spectral theory, asymptotic behavior) is worked out on four different types of underlying spaces:
%% --
\begin{enumerate}[label=(\Alph*)]
\item 
On Banach spaces --- Here we present the background for the subsequent discussions related to order.

\item 
On spaces $C_{0}(X)$ ($X$ locally compact), which constitute an important class of ordered Banach spaces and where our results can be presented in a form which makes them accessible also for the non-expert in order-theory.

\item 
On Banach lattices, which admit a rich theory and are still sufficiently general as to include many concrete spaces appearing in analysis; e.g., $C_0(X)$, $\mathcal{L}^p(k)$ or $l^p$.

\item 
On non-commutative operator algebras such as $C^*$- or $W^*$-algebras, which are not lattice ordered but still possess an interesting order structure of great importance in mathematical physics.

\end{enumerate}

In each of these cases we start with a short collection of basic results and notations, so that the contents of the book may be visualized in the form of a $4 \times 4$ matrix in a way which will allow \enquote{row readers} (interested in semigroups on certain types of spaces) and \enquote{column readers} (interested in certain aspects) to find a path through the book corresponding to their interest.

We display this matrix, together with the names of the authors contributing to the subjects defined through this scheme:

\begin{table}[ht]
\centering
\begin{tabular}{l|c|c|c|c|}
\cline{2-5}
 & I & II & III & IV \\
 & Basic & Characterization & Spectral & Asymptotics \\
 & Results &  & Theory & \\
\hline
\multicolumn{1}{|l|}{A. Banach} & R. Nagel & W. Arendt & G. Greiner & F. Neubrander \\
\multicolumn{1}{|l|}{Spaces} & U. Schlotterbeck & H. P. Lotz & R. Nagel & \\
\hline
\multicolumn{1}{|l|}{B. $C_0(X)$} & R. Nagel & W. Arendt & G. Greiner & A. Grabosch \\
\multicolumn{1}{|l|}{} & U. Schlotterbeck & & & G. Greiner \\
\multicolumn{1}{|l|}{} & & & & U. Moustakas \\
\multicolumn{1}{|l|}{} & & & & F. Neubrander \\
\hline
\multicolumn{1}{|l|}{C. Banach} & R. Nagel & G. Arendt & G. Greiner & A. Grabosch \\
\multicolumn{1}{|l|}{Lattices} & U. Schlotterbeck & & & G. Greiner \\
\multicolumn{1}{|l|}{} & & & & U. Moustakas \\
\multicolumn{1}{|l|}{} & & & & R. Nagel \\
\multicolumn{1}{|l|}{} & & & & F. Neubrander \\
\hline
\multicolumn{1}{|l|}{D. Operator} & U. Groh & U. Groh & U. Groh & U. Groh \\
\multicolumn{1}{|l|}{Algebras} & & & & \\
\hline
\end{tabular}
\end{table}

This \enquote{matrix of contents} has been an indispensable guide line in our discussions on the scope and the spirit of the various contributions. 
However, we would not have succeeded in completing this manuscript, as a collection of independent contributions (personally accounted for by the authors), under less favorable conditions than we have actually met. 
For one thing, Rainer Nagel was an unfaltering and undisputed spiritus rector from the very beginning of the project. 
On the other hand we gratefully acknowledge the influence of Helmut H.~Schaefer and his pioneering work on order structures in analysis. 
It was the team spirit produced by this common mathematical background which, with a little help from our friends, made it possible to overcome most difficulties.

We have prepared the manuscript with the aid of a word processor, but we confess that without the assistance of Klaus Kuhn the pitfalls of such a system would have been greater than its benefits.

\vspace{\baselineskip}
\begin{flushright}\noindent
${}$\hfill {\it The authors} \\
\end{flushright}

\end{document}