\documentclass{article}
\usepackage{amsmath}
\usepackage{amsthm}
\usepackage{amssymb}
\usepackage[inline,shortlabels]{enumitem}

\usepackage{amsthm}

% Theorem styles
\theoremstyle{plain} % for theorems, lemmas, propositions (italic text)
\newtheorem{theorem}{Theorem}[section]
\newtheorem{proposition}[theorem]{Proposition}
\newtheorem{lemma}[theorem]{Lemma}
\newtheorem{corollary}[theorem]{Corollary}

% Definition style
\theoremstyle{definition} % for definitions, examples (upright text)
\newtheorem{definition}[theorem]{Definition}
\newtheorem{example}[theorem]{Example}
\newtheorem{remark}[theorem]{Remark}

\title{Characterization of Positive}
\author{}
\date{}

\begin{document}
\maketitle

Since the positive cone of a C*-algebra has non-empty interior many results of Chapter B-II can be applied verbatim to the characterization of the generator of positive semigroups on $C^*$.
Other hand a concrete and detailed representation of such generators has been found only in the uniformly continuous case (see Lindblad (1976)).
A third area of active research has been the following: Which maps on $C^*$-algebras (in particular, which derivations) commuting with certain automorphism groups are automatically generators of strongly continuous positive semigroups.
For more informations we refer to the survey article of Evans (1984).

\section{Positive Semigroups on Properly Infinite $ W^{*} $-Algebras}

The aim of this section is to show that strongly continuous semigroups of schwarz maps on properly infinite $W^*$-algebras are already uniformly continuous.
In particular, our theorem is applicable to such semigroups on $B(H)$.
It is worthwhile to remark, that the result of Lotz (1985) on the uniformly continuity of every strongly continuous semigroup on $L^m$ (see A-II, sec.3) does not extend to arbitrary $W^*$ algebras.
For example, take $M=B(H)$, $H$ infinite dimensional, and choose a projection $p \in M$ such that $Mp$ is topologically isomorphic to $H$.
Therefore $M=H \oplus M_o$, where $M_o=ker(x \rightarrow xp)$.
Next take a strongly, but not uniformly continuous, semigroup $S$ on $H$ and consider the strongly continuous semigroup $S \oplus \text{Id}$ on $M$.
Our approach we refer to [Sakai (1971), 2.2] and [Takesaki (1979), V.1].

\begin{theorem}
Every strongly continuous one-parameter semigroup of Schwarz type on a properly infinite $W^*$-algebra $M$ is uniformly continuous.
\end{theorem}

\begin{proof}
Let $T=(T(t)_{*}, t \geq 0)$ be strongly continuous on $M$ and suppose $T$ not to be uniformly continuous.
Then there exists a sequence $(T_n) \subset T$ and $\varepsilon > 0$ such that $\|T_n-\text{Id}\| \geq \varepsilon$ but $T_n \to \text{Id}$ in the strong operator topology.
We claim that for every sequence $(p_k)$ of mutually orthogonal projections and all bounded sequences $(x_k)$ in $M$
%% --
\[
\lim_{n}\|(T_n-\text{Id})(p_k x_k p_k)\|=0
\]
%% --
uniformly in $k \in \mathbb{N}$.
This follows from an application of the Lemma of Phillips and the fact that the sequence $(p_k x_k p_k)$ is sumable in the $\sigma^*(M,M_*)$-topology (compare Elliot (1972)).

Let $(p_k)$ be a sequence of mutually orthogonal projections in $M$ such that every $p_k$ is equivalent to 1 via some $u_k \in M$ [Sakai (1971), 2.2].
Without loss of generality we may assume $\|(T_n-\text{Id})(u_n)\| \leq n^{-1}$ since the semigroup $T$ is strongly continuous.
Thus we obtained the following:
\begin{enumerate}[(i)]
\item $\lim_{n}\|(T_n-\text{Id})(p_k x_k p_k)\|=0$ uniformly in $k \in \mathbb{N}$ for every bounded sequence $(x_k)$ in $M$.
\item Every projection $p_k$ is equivalent to 1 via some $u_k \in M$.
\item $\|(T_n-\text{Id})u_n\| \leq n^{-1}$ for all $n \in \mathbb{N}$.
\end{enumerate}

For the following construction see A-I, 3.6 and D-II, Sec.2.
Let $\hat{M}$ be an ultrapower of $M$, let $p:=(p_k)^\wedge \in \hat{M}$, $T:=(T_n)^\wedge \in (\hat{M})$ and $u:=(u_k)^\wedge \in \hat{M}$.
Then $T$ is identity preserving and of schwarz type on $\hat{M}$.
Since $u^*u=p$ and $uu^*=1$, it follows $pu^*=u^*$ and $(uu^*)x(uu^*)=x$ for all $x \in \hat{M}$.
Finally, $T(pxp)=pxp$ for all $x \in \hat{M}$, which follows from (1), and $T(u^*)=T(pu^*)=pu^*=u^*$ and $T(u)=u$, which follows from (3).
Using the schwarz inequality we obtain
%% --
\[
T(uu^*)=T(1) \leq 1=uu^*=T(u)T(u)^*.
\]
%% --

Using D-II, Lemma 1.1 we conclude $T(ux)=uT(x)$ and $T(xu^*)=T(x)u^*$ for all $x \in \hat{M}$.
Hence
$T(x)=T(uu^*xuu^*)=uT(u^*xu)u^*=uT(pu^*xup)u^*=upu^*xuu^*=uu^*xuu^*=x$
for all $x \in \hat{M}$.
From this we obtain that for every bounded sequence $(x_k)$ in $M$:
$\lim_{n}\|(T_n-\text{Id})(x_k)\|=0$ uniformly in $k$ and independently of the $x_k$'s.
This conflicts with our assumption at the beginning, hence the theorem is proved.
\end{proof}

\end{document}