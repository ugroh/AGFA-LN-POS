% !TEX TS-program = pdflatexmk
%% --  
%% -- AB04 Ultraproducts I
%% -- Status 2024-08-12-08
%% -- 

\documentclass[%
	,english 
	,headings	= small 
	,leqno
	,parskip		= half+
	,DIV			= 14
	,BCOR 			= 10mm	
%	,fontsize		= 12pt	 
%	,framed
%	,mydina4
		]{scrartcl}


%% -- Definitions etc.
%% --
\usepackage{ablatt-ug}

%% -- Number of the worksheet
%% --
\setcounter{nummer}{5}

%% -- Headings
%% --
\newcommand{\ueone}{Spectral Theory of Positive Operators} 
\newcommand{\uetwo}{Ultraproducts in Spectral Theory}

%% -- What is it
%% --
\newcommand{\was}{Worksheet}

%% -- Literature
%% --
\ExecuteBibliographyOptions{
 	,backref=true
	,backrefstyle=three
	,doi=false
	,url=false
	,maxcitenames = 1
	,maxbibnames = 10
	}

%% -- Linked cross-references
%% -- 
\hypersetup{%
	,breaklinks = true	%
	,colorlinks	= true  %                                                            
	,urlcolor	= blue  %                                                              
	,citecolor	= blue  %                                                          
	,linkcolor	= blue	%  
%  	,hidelinks 			% Remove % before printing
	}
	
%% -- Settings for Dictum; see KOMA page 138-140
%% --
\renewcommand{\dictumwidth}{0.45\textwidth}

%% --
\begin{document}

%% -- Header/Footer
%% --
\myhead{\ueone}{\was}{\today}{\uetwo}
\myfoot{\was}{\uetwo}

%% -- Quote
%% --

% For a quote
\dictum[\href{https://en.wikipedia.org/wiki/Confucius}{Confucius}]{Life is really simple, but we insist on making it complicated.}
% --------------------------
\begin{multicols}{2}[%
\section*{Ultraproducts in Spectral Theory}
The method of \enquote{ultraproduct technique} from \emph{Non Standard Analysis} proves to be very useful in the theory of Banach spaces and their operators.
The goal of this paper is to summarize the application of this techniques to the spectral theory of operators and semigroups on Banach spaces. 
The survey article by S. Heinrich \cite{heinrich:1980} contains further details.
The  \href{https://terrytao.wordpress.com/2011/10/15/254a-notes-6-ultraproducts-as-a-bridge-between-hard-analysis-and-soft-analysis/}{article about ultraproducts} by T. Tao was a reat source of inspiration for this paper.
We refrain from the most general case of the construction, \ie for arbitrary families of normed vector spaces, but restrict ourselves to the presentation as we will then use it.
]
\RaggedRight
% 
\subsection*{Ultrafilters}
Let me first recall some definitions about ultrafilter and applications.
More about this can be found \eg in \textcite[Chapter I §6]{bourbaki:topology-1} or in the above mentioned paper of T. Tao. 
\subsubsection{}
If $ X $ is a non-empty set, we call a set  $\omega$ of subsets of $ X $ an \emph{ultrafilter} if
%
\begin{myenumerate}
	\item (Properness)
	$ \emptyset \notin \omega $ and $ \omega \neq \emptyset $.
	
	\item (Intersection)
	If $ b{ 1 } $ and $ b_{ 2 } $ are in $ \omega $, then so is $ b_{ 1 } \cap b_{ 2 } $.
	
	\item (Monotonicity)
	If $a \subset b \subset X$, and $a \in \omega$, then $b \in \omega$.	
	
	\item (Maximality)
	$ a \subset X$ , then  $ a $ or $X \setminus a$ lies in $\omega$.
	
	\item
	If $ \omega $ fulfils just (i) and (ii), then we call  $ \omega $ a \emph{filter basis} and a \emph{filter}, if it fulfills (i) -- (ii).
	
	\item
	If $ \bigcup \omega = \emptyset $, then $ \omega $ is called a \emph{free ultrafilter}. 
	 
\end{myenumerate}
%% -- 
 

If 

Here are some examples:

\begin{myenumerate}
	

\item
\emph{Point filter: } For $ x \in X $ we have
%
\[
   	  \mathfrak{F}_{ x } = \{ F \subseteq X \colon x \in F \} 
\]
%
an ultrafilter on $ X $ with $ \bigcap  \mathfrak{F}_{ x } = \{ x \} $. 
Filters with this intersection property are called \emph{fixed} filters and $ \bigcup \mathfrak{F}_{ x } = \{ x \}$. 
	
\item
\emph{Fréchet filter: } Let $ f $ be a function from $ \N $ to $ X $ and for $ k \in \N $ let 
%
\[
   	F_{ k } = \{ f(n) \in \N \colon n \geq k \} \subseteq X . 
\]
%
Then 
%
\[
   	\mathfrak{B} = \{ F_{ k } \colon k \in \N \}
\]
%
is a filter basis on $ X $.
The filter $ \mathfrak{F}_{ f } $ generated by it is called the \emph{Fréchet filter} generated by $ f $ on $ X $ and this filter is a free filter but not an ultrafilter since $ F \in \mathfrak{F}_{ f } $ iff $ \N \setminus F $ is finite.

\item
If $ \omega $ is a free ultrafilter on $ \N $, then it contains the Fréchet filter.

%
%	
%	Just as a remark: Every 
%	
%	\item
%	\emph{Cofinite filters: }
%	On $ X $ countably infinite, 
%	%
%	\[
%    	 \mathfrak{F} = \{ F \subseteq X  \colon \text{ $ X \setminus $ F is finite} \}
%	\]
%	is a filter on $ X $ with $ \bigcap  \mathfrak{F} = \emptyset $.
%	Filters with this intersection property are called \emph{free} filters.
	
\item
\emph{Neighborhood filter: } If $ ( X , \tau ) $ is a topological space and $ x \in X $, then
%
\[
   	\U{ x } = \{ U \subseteq X \colon \text{There exists $ O \in \tau $ with $ O \subseteq U $} \} 
\]
%
is a filter on $ X $, the \emph{neighborhood filter} of $ x $.
	
\end{myenumerate}
%
\begin{remark}
We call a filter $ \mathfrak{F}_{ 2 } $ finer than the filter $ \mathfrak{F}_{ 1 } $ if every element of the filter $ \mathfrak{F}_{ 1 } $ is contained in $ \mathfrak{F}_{ 2 } $ and write $ \mathfrak{F}_{ 1 } \preccurlyeq \mathfrak{F}_{ 2 } $.
This is then a partial order on the set of all filters on $ X $ and an application of Zorn's lemma shows that every filter is contained in a maximal filter with respect to this order.

Then a filter $ \omega $ is maximal with respect to this order if and only if it is an ultrafilter
(see \textcite[Chap. I, §6.4 Prop. 5]{bourbaki:topology-1}.

\end{remark}

%
%Consequences from \vref{prop:charakt-ultrafilter}: 
%Every point filter is always an ultrafilter, but the cofinite filter is not an ultrafilter.
%
\begin{example}
If $ X = \N $, we can identify the fixed ultrafilters with $ \N $.
If $ \beta( \N ) $ is the set of all ultrafilters on $ \N $, then $ \beta( \N ) \setminus \N $ is precisely the set of all free ultrafilters. 
\end{example}
%
\subsubsection{}
If $ ( X , \tau ) $ is a topological space, $ x \in X $ with neighborhood filter $ \U{x} $ and $ \mathfrak{F} $ is a filter on $ X $, we call \emph{$ \mathfrak{F} $ convergent to $ x $} if the filter $ \mathfrak{F} $ is finer than the neighborhood filter $ \U{x} $.
We then write $ \lim_{ \mathfrak{F} } = x $.
%
\begin{example}
Let $ f $ be a sequence in $ \R $.
Then $ f $ converges to some $ r $ in the classical sense if and only if $ \lim_{ \mathfrak{F}_{ f } } f = r $, where $ \mathfrak{F}_{ f } $ is the Fréchet filter.
In other words: For $ \epsilon > 0 $ there exists an $ F_{ k } $ such that $ \abs{ f(n) - r } \leq \epsilon $ for all $ n \in F_{ k } $.
\end{example}
%
\subsubsection{}
For what follows we need some topological concepts.

\begin{myenumerate}
	\item (Cluster point)
	 We call $ x $ a cluster point of $ F \subseteq X $ if $ U \cap F \neq \emptyset $ for all $ U \in \U{ x } $ and denote by $ \overline{F} $ the set of all cluster points of $ F $.
	
	\item (Closed set) 
	We call $ F $ closed if $ F = \overline{F} $.
	
	\item (Closure of a set)
	If $ A \subseteq X $, then its closure is 
	%
	\[
    	\overline{ A } = \bigcap \{ F \colon \text{ $ A \subseteq F $ and $ F $ is closed} \}
	\]
	%
	\item (Interior of a set)
	If $ A \subseteq X $, then its interior is
	%
	\[ 
    	\interior{ A } = \bigcup \{ O \colon \text{ $ O \subseteq A $ and $ O $ is open} \}
	\]
	%

\end{myenumerate}
%%
%\begin{exercise}
%Show:
%\begin{myenumerate}
%	\item
%	A set $ O $ is open if and only if $ O = \interior{ O } $.
%	
%	\item
%	A set $ F $ is closed if and only if $ F = \overline{F} $.
%	
%	\item
%	If $ O $ is open, then $ X \setminus O $ is closed.
%	
%	\item
%	If $ F $ is closed, then $ X \setminus F $ is open. 
%	
%	
%\end{myenumerate}
%\end{exercise}

\subsubsection{}
Filters can now be used to elegantly define compact topological spaces.
%
\begin{proposition}
Let $ ( X , \tau ) $ be a topological space.
Then the following are equivalent:
%
\begin{myequivalent}
	\item (Lebesgue-Borel axiom)
	Every open cover of $ X $ has a finite subcover.
	
	\item (Finite intersection property)
	Every family of closed subsets of $ X $ whose intersection is empty contains finitely many elements with empty intersection.	
	
	\item (Bolzano-Weierstraß)
	Every filter on $ X $ has a cluster point.
	
	\item
	Every ultrafilter on $ X $ is convergent.
	
\end{myequivalent}
\end{proposition}
%
The equivalence of (a) and (b) is given by taking complements.
Property (a) is also called the \enquote{open covering property}.

If (b) holds and $ \mathfrak{ F } $ is a filter without cluster points, then
%
\[
    \bigcap_{ F \in \mathfrak{ F } } \overline{ F } = \emptyset  . 
\]
%
Since $ X $ is compact, we then already have 
%
\[
    \overline{F_{ 1 }} \cap \ldots \cap \overline{F_{ n }} = \emptyset 
\]
%
for finitely many elements of $ \mathfrak{ F } $.
But then $ \emptyset \in \mathfrak{ F } $.

If (c) holds and $ x $ is a cluster point of an ultrafilter $ \omega $, then
%
\[
    \{ U \cap w \colon U \in \U{x} ,  w \in \omega \}
\]
%
is a filter finer than $ \omega $.
Since this is an ultrafilter, both filters are equal, \ie $ \omega $ converges to $ x $.

If (d) holds and $ \mathfrak{ F } $ is a filter, then it is contained in an ultrafilter that is convergent.
The limit of this filter is however a cluster point of the original filter (why?)

Suppose $ \mathfrak{ F } $ is a family of closed sets for which there is no finite subfamily with empty intersection.
Then this is a filter basis, \ie the filter generated by it has cluster points and thus this family cannot have empty intersection.

\begin{definition}
A topological space with one of the above equivalent conditions is called a \emph{compact} topological space.
\end{definition}
%
\begin{example}
Let $ f $ be a bounded sequence in $ \R $, \ie $ f(n) \in \interval{-m,m} $ and let $ \omega $ be an ultrafilter on $ \N $, then 
%
\[
    f( \omega ) = \{ f(w) \colon w \in \omega \}
\]
%
is an ultrafilter on $ \interval{-m,m} $, \ie convergent.
We write for this
%
\[
    \lim_{ n \to \omega } f(f) = r  . 
\]
%
or simply
%
\[
    \lim_{ \omega } f(f) = r  . 
\]
%
If $ f $ is already convergent and $ \omega $ is a free ultrafilter, then $ \lim_{n} f(n) = \lim_{ \omega } f(n) $, since $ \omega $ is finder then the Fréchet filter. 

\end{example}
%
%\begin{example}
%Let $ E = \ell^{ \infty } $, $ \omega $ an ultrafilter on $ \N $ and $ \phi_{ \omega } $ the mapping
%%
%\[
%    \phi_{ \omega } \colon ( \xi_{ n } ) \mapsto \lim_{ \omega } \abs{\xi_{  n }} 
%\]
%%
%on $ E $.
%Then $ \phi_{ \omega } $ is a multiplicative linear form that is not shift-invariant but extends the limit on $ c $.
%
%\emph{Small exercise:} We have $ E = C( K ) $, thus $ \beta( \N ) \subseteq K $.
%Why is every multiplicative linear form already given by an ultrafilter on $ \N $, \ie why does $ \beta( \N ) = K $ hold?
%Literature on this: \textcite{walker} or \textcite{gillman}.
%See also the worksheet on \enquote{Banach limits} from the FAU lecture.
%\end{example}
%
\subsection*{The Ultraproduct Construction}
\subsubsection{}
For a Banach space $ E $ let
%
\[
   \ell^{ \infty }( E ) = \{ x \coloneq ( x_{ j } ) \colon 
   x_{ j } \in E  ,  \sup_{ j } \norm{ x_{ j } } < \infty \}  .  
\]
%
Equipped with the norm 
%
\[
	\norm{x} \coloneq \sup_{j} \norm{x_{j}}, \quad x = (x_{j})
\]
%
is vector space is a Banach space.
Let $ \omega $ be a free ultrafilter on $ \N $ and $ p_{ \omega } $  the seminorm
%
\[
    p_{ \omega }(x) =  \lim_{ \omega } \norm{ x_{ j } } , \quad  x = (x_{j}), 
\]
%
then
%
\[
    c_{ \omega } ( E ) = p_{ \omega }^{ -1 }( \{ 0 \} )
\]
%
is a closed subspace of $ \ell^{ \infty }( E )  $ and the quotient space
%
\[
    E_{ \omega } = \ell^{ \infty }( E ) / c_{ \omega } ( E )
\]
%
equipped with the quotient norm is a Banach space.
We call $ E_{ \omega } $ a \emph{ultraproduct} of the normed vector space $ E $ with respect to $ \omega $.

Obviously, the mapping 
%
\[
    x \mapsto ( x )_{ \omega }
\]
%
is an isometry from $ E $ into $ E_{ \omega } $ and we identify $ E $  with a closed subspace of $ E_{ \omega } $. 
%
\begin{proposition}
For the quotient norm for an $ x_{ \omega } $ we then have 
%
\[
    \norm{ x_{ \omega } } = \lim_{ \omega } \, \norm{ x_{ j } }  ,  \quad ( x_{ j } ) \in x_{ \omega } ) .  
\]
\end{proposition}
%% --
%For the next proposition we use an essential property of an ultrafilter $ \omega $ on a set $ K $: If $ A $ is a subset of $ K $ then $ A \in \omega $ or $ K \setminus A \in \omega $.
%
\begin{proposition}\label{prop:normiert}
Let $ E_{ \omega } $ be the ultraproduct of the Banach space $ E $ with respect to a free ultrafilter $ \omega $ and let $ x_{ \omega} \in E_{ \omega } $

\begin{myenumerate}
\item
If $ \norm{ x_{ \omega } } $ is the quotient norm of $ x_{ \omega } $ then 
%
\[
    \norm{ x_{ \omega } } = \lim_{ \omega }   \norm{ x_{ j } }  ,  \quad ( x_{ j } ) \in x_{ \omega }  .  
\]

\item
There exists a sequence $ ( x_{ j } ) \in x_{ \omega } $ such that $ \norm{x_{j}} = \norm{ x_{ \omega } } $.

\end{myenumerate}

\end{proposition}
%
\begin{proof}
\begin{enumerate}[\upshape (i), wide, labelindent=.5em]
\item
Let $ (x_{j}) \in x_{\omega} $, then $ \alpha = \lim_{ \omega } \norm{ x_{j} } $ exists.
If $  \epsilon > 0 $, then then exists $  a \in \omega $, such that
%
\[
	\abs{ \alpha - \norm{ x_{j} } } \leq \epsilon \quad \text{for all $  j \in a  $.}
\]
%
Hence $ \norm{ x_{j} } \leq \alpha + \epsilon $ for all $  j \in a  $.

Let $ z = (z_{j}) \in  \in \ell^{ \infty }( E ) $ such that
%
\[
	z_{j} =
	\begin{cases}
	0 , 	& j \notin a , \\
	x_{j}, 	& j \in a .
	\end{cases}
\]
%
Then
%
\[
	x_{j} - z_{j} =
	\begin{cases}
	x_{j} , 	& j \notin a , \\
	0, 			& j \in a .
	\end{cases}
\]
%
or
%
\[
	\norm{ x_{j} z_{j} } \leq \frac{1}{k} \quad \text{for all $  k \geq 1 $ and $ j \in a $.}
\]
%
Thus $ z_{ \omega } = x_{ \omega } $ and we obtain
%% --
\begin{align*}
	\norm{ x_{ \omega } } = \norm{ z_{ \omega } } \leq \norm{ (z_{j}) } = \sup_{j} \norm{ (z_{j}) } \\
	\leq \sup_{j} \norm{ (x_{j}) } \leq \alpha + \epsilon = \lim_{j} \norm{ x_{j} }.
\end{align*} 
%% --
Hence 
%
\[
	\norm{ x_{ \omega } } \leq \lim_{j} \norm{ x_{j} } \quad \text{for all $ (x_{j}) \in x_{ \omega } $.}
\]
%
Conversely, let $ \epsilon > 0 $.
Then there exists $ (z_{j}) \in c_{ \omega }(E) $ and $ (x_{j}) \in x_{ \omega } $ such that
%
\[
	\norm{ (x_{j}) - (z_{j}) } \leq \norm{ x_{ \omega } } + \epsilon .
\]
%
But then 
%
\[
	\lim_{j }\norm{ x_{j} } = \lim_{j} \norm{ x_{j}  -  z_{j}  } \leq \norm{ x_{ \omega } } + \epsilon 
\]
%
and (i) is proved.
\item
Obviously it is enough to prove this for $ \norm{ x_{ \omega } } = 1$.

Let $ ( y_{ j } ) $ be any sequence in $ x_{ \omega } $ and let 
%
\[
    A = \{ j  \colon y_{ j } = 0 \}  . 
\]
%
If $ A \in \omega $, then because of
%
\[
    A \subseteq A_{ k } = \{ j \colon \norm{ y_{ j } } \leq \frac{ 1 }{ k } \}
\]
%%
the set $ A_{ k } \in \omega $ as well, \ie $ \norm{ x_{ \omega } } \leq 1/k $.
Thus $ \N \setminus A \in \omega $.

Let $ k \in \N \setminus A $ and define
%
\[
    x_{ j } = 
    	\begin{cases}
    		y_{ k }/\norm{ 	y_{ k } }  ,  	& j \in A  ,  \\
    		y_{ j }/\norm{ 	y_{ j } }  ,  	& j \in \N \setminus A  . 
    	\end{cases}
\]
%
Then $ \lim_{ \omega } \norm{ x_{ j } - y_{ j } } = 0 $ and the proposition is thus proved.

\end{enumerate}
\end{proof}
%
\subsubsection{}
Here are some properties of an ultraproduct of a Banach space.
%
\begin{examples}
\begin{enumerate}[\upshape (i), wide, labelindent=.5em]

	\item
	If $ E $ is a Banach algebra or a \CA-algebra or a Banach lattice, then so is $ E_{ \omega } $.
	
	\item
	If $ \BA $ is a commutative \CA-algebra, \ie $ \BA = C(K) $, then $ \BA_{ \omega } $ is a commutative \CA-algebra and its spectrum (Gelfand space) is the Stone-C\v{e}ch compactification of the countable disjoint union of $ K $ (see \textcite[Theorem 4.1]{heinrich:1980}).
%	
	\item
	If $ E $ is a Banach space of finite dimension, then it is isometrically isomorphic to its ultraproduct. 
	For the closed unit ball is compact for the norm topology, \ie every bounded sequence is convergent along the ultrafilter in $ E $.
	Therefore the canonical embedding is a surjective isometry.
	
	\item
	If $ \NA $ is the \WA-algebra $ \ell^{ \infty } $, then $ M_{ \omega } $ for a free ultrafilter $ \omega $ is not a \WA-algebra.
	For this see \textcite[p. 79]{heinrich:1980}.
\end{enumerate}	

\end{examples}
%% --
\subsubsection{}
If $ E_{ \omega } $ is the ultraproduct of a normed vector space $ E $, $ (E')_{ \omega } $ is the ultraproduct of the dual $ E' $ and $ ( E_{ \omega } ){}' $ is the dual of the ultraproduct of $ E_{ \omega } $, then there is a canonical isometry $ \tau $ from $ (E')_{ \omega } $ into $ ( E_{ \omega } ){}' $ given by
%
\[
    < \tau( x'_{ \omega } ), x_{ \omega } > 	= \lim_{ \omega }   < x'_{ j } , x_{ j } >
\]
%
for $ ( x'_{ j } ) \in x'_{ \omega }  $ and $ ( x_{ j } ) \in x_{ \omega } $
Then $ \tau $ is well defined and an isometry.
We therefore identify $ (E')_{ \omega } $ with the closed subspace $ \tau( (E')_{ \omega } ) $ in 
$ ( E_{ \omega } ){}' $.

It is worth to remark, that in general $ (E')_{ \omega }  $ is distinct from $ ( E_{ \omega } ){}' $.
More about this can be found in the \textcite[1.7]{heinrich:1980}.

%\begin{example}\label{examp:predual}
%Since in general the ultraproduct of a \WA-algabra is not a \WA-algebra.
%Motivated by the commutataive situation, \ie the duality $ ( L^{1}, L^{\infty} ) $, we consider the predual $ \NA_{*} $ of a \WA-algebra $ \NA $.
%\end{example}
 

% --------------------------
\subsubsection{}
If $ T $ is a bounded operator on $ E $, then for $ x = ( x_{ j } ) \in \ell^{ \infty }( E ) $ the mapping 
%
\[
    \widetilde{ T } ( x ) = ( T x_{ j } )
\]
%
is a bounded operator on $ \ell^{ \infty }( E ) $ that leaves the subspace $ c_{ \omega }( E ) $ invariant.
We denote by $ \hat{T} $ the operator defined on the ultraproduct $ E_{ \omega } $, \ie
%
\[
    \hat{T}( x_{ \omega } ) = ( T x_{ j } )_{ \omega }  ,  
    \quad ( x_{ j } ) \in x_{ \omega }  . 
\]
%
\begin{proposition}\label{lem:einbettung-le}
\begin{myenumerate}
\item
The mapping $ T \mapsto \hat{ T } $ from $ \LE $ into $ \L{ E_{ \omega } } $ is an algebra homomorphism from the algebra $ \LE $ into the algebra $ \L{ E_{ \omega } } $ and an isometry.

\item
$ \hat{ T } $ restricted to the canonical image of $ E $ in $ E_{ \omega } $ is the original operator $ T $, \ie $ \hat{T} $ is an extension of $ T $.

\item
If $ T' $ is the adjoint of $ T $ and if we identify canonically $ (E')_{ \omega } $ with the closed subspace $ \tau( (E')_{ \omega } ) $ in $ ( E_{ \omega } ){}' $, then 
%
\[
   	( \hat{ T } ){}' = \hat{ T' }_{ \vert (E')_{ \omega } }  . 
\]
%
\end{myenumerate}

\end{proposition}
%
\begin{proof}

\end{proof}
%
\subsection*{Spectral Theory on Ultraproducts}
\subsubsection{}
Important: Look at \emph{Chap04-AB01-Spectral Theory, Epilogue} again.
There we had shown that one can split the spectrum of an operator into two essential parts: The approximate point spectrum (not bounded below) and the compression spectrum (does not have dense image).

First an initial observation: If $ T $ is invertible, then so is $ \hat{T} $, and if this operator is invertible, then so is $ T $, since $ \hat{T} $ leaves the canonical embedding of $ E $ into its ultraproduct invariant and there it is the original operator.
From this we immediately obtain:
%
\begin{proposition}
For all bounded operators we always have 
%
\[
    \Sp{ T } = \Sp{ \hat{ T } } 
\]
%
and for all $ \lambda \in \Res{ T } $
%
\[
    R( \lambda, \hat{ T } ) = R( \lambda , T )\hat \,  . 
\]
%
\end{proposition}
%
Now to two properties that show the connection between the parts of the spectrum.
%
\begin{proposition}\label{prop:injektiv}
For $ T \in \LE $ the following are equivalent:
%
\begin{myequivalent}
	\item
	$ T $ is bounded below (in particular injective).
	
	\item
	$ \hat{ T } $ is bounded below.
	
	\item
	$ \hat{ T } $ is injective.
\end{myequivalent}
\end{proposition}
%
\begin{exercise}
Prove these equivalences.
\end{exercise}
%
\begin{corollary}
We have
%
\[
    \Spap{ T } = \Spap{ \hat{ T } } = \Spp{ \hat{ T } }  . 
\]
%
\end{corollary}
%
\emph{Note: } If one makes the original space large enough, approximate eigenvalues suddenly become eigenvalues.
%
\begin{exercise}
What does this mean for normal operators on a Hilbert space?
\end{exercise}
%
\subsubsection{}
In \vref{prop:injektiv} we characterized the injectivity of $ \hat{ T } $. 
For surjectivity we have 
% 
\begin{proposition}\label{prop:surjektiv}
For $ T \in \LE $ the following are equivalent:
%
\begin{myequivalent}
	\item
	$ T $ is open ($ = $ surjective).
	
	\item
	$ \hat{ T } $ is open ($ = $ surjective).
	
	\item
	$ \hat{ T } $ has dense image.
\end{myequivalent}
\end{proposition}
%
If $ T $ is open, then for every bounded sequence $ ( y_{ j } ) $ there exists a bounded sequence $ ( x_{ j } ) $ in $ E $ with $ T x_{ j } = y_{ j } $.
Thus $ \hat{ T } $ is surjective and therefore open.

If $ \hat{ T } $ has dense image, then $ \hat{ T }' $ is injective on $ ( E_{ \omega } ){}' $.
Because of the canonical embeddings
%
\[
    E' \hookrightarrow ( E' )_{ \omega } \hookrightarrow ( E_{ \omega } ){}'
\]
%
$ T' $ is thus injective on $ E' $.
But then $ T $ must already be open (see the next section).
%
\begin{corollary}
For $ \hat{ T } $ we therefore always have 
%
\[
    \Sp{ \hat{ T } } = \Spp{ \hat{ T } } \cup \Spd{ \hat{ T } }  . 
\]
%
\end{corollary}
%
\emph{Conclusion: } We are back in Linear Algebra, at least for operators on $ E_{ \omega } $ that are extensions of operators on $ E $.
%
\subsubsection{\emph{Supplement: }}
Some more general spectral theory: For $ T \in \LE $ the following are always equivalent (see \textcite[Theorem 57.16 \& 57.18]{berberian:1973} or \textcite[Chap. 2, Theorem 51 \& Theorem 52]{mathieu:1998}):
%
\begin{myequivalent}
	\item
	$ T $ is surjective.
	
	\item
	$ T' $ is bounded below.
\end{myequivalent}
%
And as a \enquote{dual} version the following are equivalent:
%
\begin{myequivalent}
	\item
	$ T $ is bounded below.
	
	\item
	$ T' $ is surjective.
\end{myequivalent} 
%
As an overall result, everything together then in a small exercise (please do):
%
\begin{exercise}
For $ T \in \LE $ we have:
%
\begin{myenumerate}

	\item
	$ \Sp{ T } = \Sp{ \hat{ T } } $.
	
	\item
	$ R( \lambda , \hat{ T } ) =  R( \lambda , T )\hat{}\, $.
	
	\item
	$ \Spap{ T } = \Spap{ T'' } $.
	
	\item
	$ \Spp{ T'' } \subseteq \Spap{ T } $.
	
	\item
	$ \Spap{ T } = \Spap{ \hat{ T } } = \Spp{ \hat{ T } } $.
	
	\item
	$ \Spd{ T } = \Spd{ \hat{ T } } = \Spcp{ \hat{ T } } $.
	
\end{myenumerate}


\end{exercise}
%
\subsection*{An Application}
\subsubsection{}
The following properties are intended to show the usefulness of the ultraproduct construction and conclude this worksheet.
%
\begin{proposition}
Let $ T \in \LE $.
%
\begin{myenumerate}
	
	\item
	If the fixed space of $ \hat{ T } $ in $ E_{ \omega } $ is finite-dimensional, then $ ( I - T ) E $ is a closed subspace in $ E $.
	
	\item
	If $ \lambda \in \Spap{ T } \setminus \Spp{ T } $, then the dimension of the subspace $ \Kern{ \lambda - \hat{ T } } $ is not finite.
	
	\item
	If the dimension of the subspace $ \Kern{ \lambda - \hat{ T } } $ is finite, then $ \lambda $ is a pole of the resolvent.
\end{myenumerate}
\end{proposition}
%%
\begin{proof}
See lecture.
\end{proof}

% -----------------------------------------------------
%
% \setcounter{unbalance}{5}		
% \nocite{}
\printbibliography
\end{multicols}
%
\end{document}