% !TEX TS-program = pdflatexmk
%% --  
%% -- AB05 Ultraproducts II
%% -- Status 2024-08-30
%% -- 

\documentclass[%
	,english 
	,headings	= small 
	,leqno
	,parskip		= half+
	,DIV			= 14
	,BCOR 			= 10mm	
%	,fontsize		= 12pt	 
%	,framed
%	,mydina4
		]{scrartcl}


%% -- Definitions etc.
%% --
\usepackage{ablatt-ug}

%% -- Number of the worksheet
%% --
\setcounter{nummer}{4}

%% -- Headings
%% --
\newcommand{\ueone}{Spectral Theory of Positive Operators} 
\newcommand{\uetwo}{Ultraproducts II}

%% -- What is it
%% --
\newcommand{\was}{Worksheet}

%% -- Literature
%% --
\ExecuteBibliographyOptions{
 	,backref=true
	,backrefstyle=three
	,doi=false
	,url=false
	,maxcitenames = 1
	,maxbibnames = 10
	}

%% -- Linked cross-references
%% -- 
\hypersetup{%
	,breaklinks = true	%
	,colorlinks	= true  %                                                            
	,urlcolor	= blue  %                                                              
	,citecolor	= blue  %                                                          
	,linkcolor	= blue	%  
%  	,hidelinks 			% Remove % before printing
	}
	
%% -- Settings for Dictum; see KOMA page 138-140
%% --
\renewcommand{\dictumwidth}{0.45\textwidth}

%% --
\begin{document}

%% -- Header/Footer
%% --
\myhead{\ueone}{\was}{January 17, 2025}{\uetwo}
\myfoot{\was}{\uetwo}

%% -- Quote
%% --

%%
\dictum[\href{https://en.wikipedia.org/wiki/Jules_Renard}{Jules Renard}]{There are moments when everything works out for us. No need to be frightened: it will pass.}
%%
% -------------------------------------------------------
\begin{multicols}{2}[%
\section*{Ultraproducts of \CA- and \WA-Algebras}
This is a first draft that needs to be elaborated. Perhaps someone is interested in this.]
\RaggedRight
\subsection*{Ultraproducts of \CA- and \WA-Algebras}
\subsubsection{}
For \CA-algebras we have
%
\begin{proposition}
If $ \BA_{ \omega } $ is the ultraproduct of a \CA-algebra $ \BA $, then:
%
\begin{myenumerate}
	
	\item
	If $ a $ is self-adjoint in $ \BA_{ \omega } $, then there exists a sequence $ ( a_{ j } ) $ in $ \BA_{ h } $ with $ a = ( a_{ j } )_{ \omega } $.
	
	\item
	If $ a $ is positive in $ \BA_{ \omega } $, then there exists a sequence $ ( a_{ j } ) $ in $ \BA_{ + } $ with $ a = ( a_{ j } )_{ \omega } $.
	
	\item
	If $ p $ is a projection in $ \BA_{ \omega } $, then there exists a sequence $ ( p_{ j } ) $ of projections in $ \BA $ with $ p = ( p_{ j } )_{ \omega } $.
	
	\item
	If $ u $ is unitary in $ \BA_{ \omega } $, then there exists a sequence $ ( u_{ j } ) $ of unitary elements in $ \BA $ with $ u = ( u_{ j } )_{ \omega } $.
	
	\item
	If $ p = ( p_{ j } )_{ \omega } $ and $ q = ( q_{ j } )_{ \omega } $ are projections in $ \BA_{ \omega } $ and $ v $ is a partial isometry with $ v^{ * } v = p $, $ vv^{ * } = q $, then there exist $ v_{ j } $ in $ \BA $ with
	%
\[
   v_{ j }^{ * } v_{ j } = p_{ j } 
   \quad \text{and} \quad
   v_{ j }v_{ j }^{ * }  = q_{ j } 
\]
%
along $ \omega $.
	
	\item
	If $ p = ( p_{ j } )_{ \omega } $ is a projection, $ p \leq q $ for some projection $ q $, then there exist projections $ q_{ j } $ in $ \BA $ with  
	%
\[
    p_{ j } \leq q_{ j }
    \quad \text{and} \quad
    q = ( q_{ j } )_{ \omega }
\]
%
\end{myenumerate}
%
\end{proposition}
%
Proof: Literature citation
\subsubsection{}
If $ E = L^{ 1 } $, then $ E $ is the predual of a \WA-algebra and the ultraproduct of $ E $ is again an $ L^{ 1 } $ and thus again a predual.
This also holds for general \WA-algebras (see \textcite{ulgr-optheory11}).
If $ \NA $ is a \WA-algebra, then its ultraproduct is in general not a \WA-algebra, since the kernel of the seminorm $ p_{ \omega } $ need not be a $ \sigma^{ * } $-closed ideal in the \WA-algebra $ \ell^{ \infty }( \NA ) $.
However, we have:
%
\begin{proposition}
The ultraproduct of the predual of a \WA-algebra is the predual of a \WA-algebra.
\end{proposition}
%
Let $ \NA $ be a \WA-algebra with predual $ \NA_{ * } $, then the mapping
%
\[
    \tau \colon ( \NA_{ * } )_{ \omega } \mapsto ( \NA_{ \omega } ){}'
\]
%
with
%
\[
    < \tau( \phi_{ \omega } ) , x_{ \omega } > = \lim_{ \omega } \phi_{ j }( x_{ j } )
\]
%
for $ ( x_{ j } ) \in x_{ \omega } $ and $ ( \phi_{ j } ) \in \phi_{ \omega } $ is an isometry.
Let $ \hat{\MA} $ be the closed subspace $ \tau ( ( \NA_{ * } )_{ \omega } ) $. 

Show: This subspace is invariant under multiplication from the right and from the left with elements of $ \NA_{ \omega } $ and $ \NA_{ \omega }'' $. 
Here one only needs to note that for $ \phi \in \NA_{ * } $ and $ x \in \NA $ we always have $ x . \phi $ and $ \phi . x $ are again elements of the predual.
Thus the polar of $ \hat{\MA} $ in $ \NA_{ \omega }'' $ is a $ \sigma^{ * } $-closed ideal and there exists a central projection $ z $ in $ \NA_{ \omega }'' $ with
%
\[
    \hat{ \MA } = \NA_{ \omega }' . z = \left[ ( \NA_{ \omega } ){}'' . z \right]_{ * } \, .
\]
%
\begin{proposition}
For the mapping $ \tau $ defined above, we have for all $ \phi_{ \omega } \in ( \NA_{ * } )_{ \omega } $
%
\[
    \tau ( \abs{ \phi_{ \omega } } ) = \abs{ \tau( \phi_{ \omega } ) } \, .
\]
%
\end{proposition}
The proof is based on \textcite[III.4.10]{takesaki:1979} and remains as \fbox{ Exercise }.
%
\begin{example}
Example where $ \NA_{ \omega } $ is not a \WA-algebra ->
\end{example}
%
\subsection*{Compatibility with Positive Maps}
\subsubsection{}
We have:
%
\begin{myenumerate}
	
	\item
	$ T $ is positive if and only if $ \hat{ T } $ is positive.
	
	\item
	$ T $ is a Schwarz operator if and only if $ \hat{ T } $ is a Schwarz operator.
	
	\item
	$ T $ is an $ n $-positive operator if and only if $ \hat{ T } $ is an $ n $-positive operator.
\end{myenumerate}
%
For the proof of the last claim one must show that we always have
%
\[
    M_{ n }( A )_{ \omega } = M_{ n }( A _{ \omega } )
\]
%
% -------------------------------------------------------
% \setcounter{unbalance}{5}		
% \nocite{}
\printbibliography
\end{multicols}
%
\end{document}