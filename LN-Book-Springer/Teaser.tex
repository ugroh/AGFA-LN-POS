\documentclass[%
	,english
	,fontsize	= 12pt
	,paper 		= a4			 			 
	,smallheadings=true
	,parskip	= half	
	,DIV	= classic						
	,oneside
]{scrartcl}
%% --
%% -- 
%% --
\begin{document}
%% --
% =========================
% Part A — Abstract
% =========================
\section*{Part A}
This part provides the general framework for the theory of one-parameter semigroups of
bounded linear operators on Banach spaces, with particular emphasis on structures
relevant to positivity. 
Rather than offering a self-contained treatment, the focus is on
fixing notation, recalling fundamental definitions and results, and collecting standard
examples and constructions that recur throughout the book. 
Topics include the abstract
Cauchy problem, perturbation and characterization results, as well as spectral and
asymptotic theory. 
This material serves as a foundation for the subsequent parts, where positivity plays a central role and the basic objects are Banach lattices.


% =========================
% Part B — Abstract
% =========================
\section*{Part B}
This part is devoted to positive semigroups acting on spaces of continuous functions of
type $C_{0}(X)$, $X$ locally compact. 
Owing to their concrete structure as Banach lattices, these spaces provide
an accessible and intuitive setting in which many fundamental phenomena of positive
semigroups can be studied in detail. After fixing notation and reviewing the basic
properties of $C_0(X)$ and $C(K)$, $K$ compact, the theory develops characterizations of positive
semigroups, their generators, and, as a special case, associated flows. 
Spectral and asymptotic properties, including stability, irreducibility, and compactness, are treated extensively. The results in this part serve both as a motivating model and as a testing ground for the
more abstract theory developed later.


% =========================
% Part C — Abstract
% =========================
\section*{Part C}
This part develops the theory of positive semigroups in the general setting of Banach
lattices. 
It begins with a concise introduction to ordered Banach spaces and Banach
lattices, emphasizing concepts and structural properties essential for the study of
positivity. Classical function spaces such as $C(K)$ and $L^p$ spaces serve as guiding
examples, illustrating the close connection between abstract lattice theory and concrete
models. Building on this foundation, the part presents characterization results for
positive semigroups and their generators, domination and disjointness properties, as
well as spectral and asymptotic theory. This framework unifies and extends the phenomena
encountered already in the $C_{0}(X)$ setting.


% =========================
% Part D — Abstract
% =========================
\section*{Part D}
This part is concerned with positive semigroups acting on $C^*$- and $W^*$-algebras and their preduals. 
It is an introduction to the theory of positive semigroups on operator algebras addressing similar subjects as in Parts B and C.
After reviewing basic structural properties of semigroups on operator algebras, the focus shifts to positivity-preserving dynamics, spectral properties, and asymptotic behavior. 
Topics include characterization results on $W^*$-algebras, spectral theory on preduals, and stability and ergodic phenomena. 
This part extends the theory of positive semigroups on Banach lattices to a noncommutative
setting.

%% --
\end{document}
