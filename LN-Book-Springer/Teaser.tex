\documentclass[%
	,english
	,fontsize	= 12pt
	,paper 		= a4			 			 
	,smallheadings=true
	,parskip	= half	
	,DIV	= classic						
	,oneside
]{scrartcl}
%% --
%% -- 
%% --
\begin{document}
%% --
% =========================
% Part A — Abstract
% =========================
\section*{Part A}
This part outlines the general framework of the theory of one-parameter semigroups of bounded linear operators on Banach spaces, paying particular attention to structures relevant to positivity. 
Rather than providing a self-contained treatment, the focus is on establishing notation, recalling key definitions and results, and collecting standard examples and constructions that recur throughout the book.
Topics include the abstract Cauchy problem, perturbation and characterization results, and spectral and asymptotic theory. 
This material forms the foundation for the subsequent parts, in which positivity plays a central role and Banach lattices are the primary objects of study.


% =========================
% Part B — Abstract
% =========================
\section*{Part B}
This part focuses on positive semigroups acting on $C_{0}(X)$ spaces, where $X$ is a locally compact set. 
Due to their concrete structure as Banach lattices, these spaces provide an accessible and intuitive setting in which many fundamental phenomena of positive semigroups can be studied in detail. 
After fixing notation and reviewing the basic properties of $C_{0}(X)$ and $C(K)$, where K is compact, the theory proceeds to characterize positive semigroups, their generators and, as a special case, associated flows. 
Spectral and asymptotic properties, including stability, irreducibility and compactness, are covered in depth. 
The results in this part serve as both a motivating model and a testing ground for the more abstract theory developed later.


% =========================
% Part C — Abstract
% =========================
\section*{Part C}
This part outlines the theory of positive semigroups within the broader context of Banach lattices. 
It begins with a concise introduction to ordered Banach spaces and Banach lattices, emphasizing the concepts and structural properties that are essential for studying positivity. 
Classical function spaces, such as $C(K)$ and $L^p$ spaces, serve as guiding examples, illustrating the close connection between abstract lattice theory and concrete models. 
Building on this foundation, the part presents characterization results for positive semigroups and their generators, as well as domination and disjointness properties and spectral and asymptotic theory. 
This framework unifies and extends phenomena already encountered in the $C_{0}(X)$ setting.


% =========================
% Part D — Abstract
% =========================
\section*{Part D}
This part focuses on positive semigroups acting on $C^*$- and $W^*$-algebras and their preduals. 
It introduces the theory of positive semigroups on operator algebras, covering topics similar to those in Parts B and C. 
After reviewing basic structural properties of semigroups on operator algebras, the focus shifts to positivity-preserving dynamics, spectral properties and asymptotic behavior. 
Topics include characterization results on $W^*$-algebras, spectral theory on preduals, and stability and ergodic phenomena. 
This part extends the theory of positive semigroups on Banach lattices to a non-commutative setting.

%% --
\end{document}
