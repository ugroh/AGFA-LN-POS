%% --
\chapternopage{Updated Notes Part C}
%% --
\section*{Updated Notes C-I}
\addcontentsline{toc}{section}{Updated Notes C-I}
%% --
\begin{enumerate}

\item 
Our main source for the theory of Banach lattices and positive operators is \mycite{zbmath03464348}.
Other %useful 
references are \mycite{zbmath03983937}, \mycite{zbmath00051953}, and \mycite{zbmath00967648}.

\item
A gentle introduction to  semigroups of positive operators is \mycite{zbmath06695787}, starting from finite dimensions and leading to many concrete applications.

\item
Motivated by concrete PDEs (see, \eg \mycite{zbmath06487326}), \enquote{eventually positive} semigroups form another active research area.
We refer to the survey article by \mycite{zbmath07497413}.

\item 
The monograph by \mycite{zbmath06255542} is devoted to (positive) semigroups on networks.


\end{enumerate}
%% --
\section*{Updated Notes C-II}
\addcontentsline{toc}{section}{Updated Notes C-II}
%% --
\begin{enumerate} 

\item 
It is interesting that some properties of semigroups on Banach lattices are preserved by domination.
We mention the following  result by \mycite{zbmath07735826}.
%% --
\begin{quote}
\textit{%
Let $(T(t))_{t\geq 0}$ and $(S(t))_{t\geq 0}$ be positive semigroups on a Banach lattice $E$ such that $S(t) \leq T(t)$ for all $t\geq 0$. 
If the semigroup  $(T(t))_{t\geq 0}$ is holomorphic, then so is the semigroup $ (S(t))_{ t\geq 0 } $.} %% so ist es richtig
\end{quote}
%% --
The proof uses a result by \mycite{zbmath01080532} about the preservation of spectral and asymptotic behavior of semigroups under domination.

Also mean ergodicity is preserved under domination  if the underlying Banach lattice $E$ has order continuous norm, see \mycite{zbmath00031435}. 
Specifically, this is valid for complex Banach lattices, even if the semigroup $ (S(t))$ is not necessarily positive (which is needed for the preceding result, though). 
Thus the weaker domination property  
$ \lvert S(t)f \rvert \leq T(t)\lvert f\rvert $ for all $ t\geq 0 $, $ f\in E $ suffices. However, on a space of type $C(K)$ mean ergodicity is not necessarily inherited from a dominating semigroup, see Section~3 in \mycite{zbmath00031435}.

Another interesting result involving domination of semigroups is proved in the article \mycite{zbmath00537201}:
%% --
\begin{quote}
\textit{%
Let $(T(t))_{t\geq 0}$ and $(S(t))_{t\geq 0}$ be positive semigroups on a Banach lattice $E$ with order continuous norm such that $S(t) \leq T(t)$ for all $t\geq 0$.
If\/ $T(t)f$ converges to $Pf$ as $t\rightarrow \infty$ for all $f\in E$ and $P$ has finite rank, then also $S(t)f$ converges as $t\rightarrow \infty$ for all $f\in E$.}
\end{quote}
%% --
For further properties inherited by domination we refer to \mycite{zbmath00561284} and the literature mentioned there.

\item
Kato's classical inequality is frequently used to prove uniqueness results, while a generalisation of Kato's inequality has been obtained by \mycite{zbmath02093937}. 
The abstract Kato inequality (K) in C-II, Theorem 3.8 for generators of positive semigroups has interesting applications to semi-linear evolution equations, see \mycite{zbmath07978278}.


\item 
Form  methods are important for generation of holomorphic semigroups on a Hil\-bert space.
The Beurling-Deny criterion is a most efficient tool to characterise positivity of a semigroup on $L^{2}$ associated with a form. 
\mycite{zbmath00001168} extended this criterion to describe invariance of arbitrary closed convex sets in the underlying Hilbert space. This allows him in particular to characterise irreducibility of the associated semigroups in a very simple way. 
We refer to Ouhabaz' monograph \mycite{zbmath02168554} for this and a comprehensive theory of forms. 
In particular, semigroups generated by elliptic operators with diverse boundary conditions on $L^{2}$ can be analyzed via form methods. 

\item 

Domination can be proved most conveniently for semigroups associated with a form, see e.g., \mycite{zbmath05059635}, \mycite{zbmath02168554}. 
More general criteria for domination, valid in ordered Banach spaces, are given by \mycite{zbmath05119932}.
The modulus semigroup has been determined in a series of concrete cases, see \mycite{zbmath05262326}, \mycite{zbmath02106405}, \mycite{zbmath02167818} 

\item 
Kernel estimates for positive semigroups, and in particular Gaussian estimates, play an important role. 
They imply that a semigroup defined and holomorphic on $L^{2}$ extends to all $L^{p}$-spaces and is holomorphic on each of these spaces (and in particular on $L^{1}$), see \mycite{zbmath00788452}. 
Even the spectrum of the generator is independent on p in this case, see \mycite{zbmath01344331}. 

\item
In \mycite{zbmath01538134} Gaussian estimates are proved for semigroups generated by elliptic operators with measurable coefficients under various boundary conditions. A comprehensive account is given in \mycite{zbmath02168554}.


\item 
It is most remarkable that a positive contractive semigroup on $L^{p}$ for  
\mbox{$1 < p < \infty$}  enjoys \emph{maximal regularity}, a result due to  \mycite{zbmath01661089}, after an impressive development of regularity theory by many authors.
We refer to Chapter 17 in the monograph \mycite{zbmath07784626} for a comprehensive treatment of maximal regularity. 

\item 
The Dirichlet-to-Neumann operator generates a holomorphic, positive, irreducible semigroup on $L^{2}(\partial\Omega)$ whenever $ \Omega$  is a bounded, connected  Lipschitz domain (see \mycite{zbmath06171001} and the Updated Notes of B-II). This is again proved via form methods. 
Kernel estimates for this semigroup are obtained in \mycite{zbmath07063341}

\item 
Much research has been done on so-called Ornstein-Uhlenbeck semigroups which are  explicitly given by a Gaussian kernel. 
Such a semigroup acts on all $L^{p}$-spaces with respect to the Lebesgue measure and also with respect to the invariant measure $\mu$ when the drift matrix $A$ is real with eigenvalues in the open left halfplane. 
The domain of its generator can be described explicitely, see \mycite{zbmath02217252}. 
For regularity properties and the spectrum of Ornstein-Uhlenbeck operators we refer to the survey article \mycite{MR4176390} and the monograph \mycite{zbmath06593873}. 
For quantitative and qualitative properties of more general Kolmogorov operators see \mycite{zbmath06593873}, \mycite{zbmath01837428} and \mycite{zbmath01789518}.

\item 
In the monograph \mycite{zbmath07274768} semigroups generated by elliptic operators are studied with special attention to Schauder estimates and
regularity properties. From the maximum principle they obtain positivity
of the generated semigroups.


\item
An elliptic operator with Robin boundary conditions (sometimes called boundary conditions of the third kind) generates a positive semigroup for very general functions defining the Robin boundary, see \mycite{zbmath05551221} . 
Also  non-local boundary conditions lead to positive semigroups,  see, \eg
\mycite{arXiv:2502.03216}. 
 
\item 
The survey article \mycite{zbmath07402424} shows the role positivity plays in models and also gives some new perturbation results (in Section 6). 


\item Also in control theory positive semigroups play an important role, see \eg \mycite{zbmath04120022} and \mycite{zbmath08056303}. 

\item 
Semigroups of lattice homomorphisms from the Koopman point of view 
on $L^{p}$-spaces are the subject of \mycite{zbmath07036266} (see also the extended notes of Chapter B-II concerning Koopman semigroups). In fact, the authors characterize generators of Markov lattice semigroups on $L^{p}$ in the Koopman spirit. As a consequence, every measure preserving flow on a standard probability space is isomorphic to a continuous flow on a compact Borel probability space.


\item
In C-II, Proposition 5.16 it is shown that any strongly continuous group in the center of a real Banach lattice has a bounded generator. 
In this context it is interesting to mention the \emph{Markov conjecture}: \textit{Any generator of a strongly continuous positive semigroup on $\ell^{1}$ which is norm-preserving on the positive cone and which extends to a group has a bounded generator.} 
This conjecture is still open, but a special case has been proved by \mycite{zbmath07367032}.

\item

Perturbation of positive semigroups is systematically studied in the monograph 
\mycite{zbmath05030445}. Further results are obtained in  \mycite{zbmath07971662}  and \mycite{zbmath08137521}.
\item 
Evolution on networks is another subject where semigroups play an important role for the theory and concrete models. 
A comprehensive book describing such semigroups has been written by \mycite{zbmath06255542} and a short introduction is given in Chapter 18 of \mycite{zbmath06695787}, with special attention to positive semigroups. The basic idea is to consider either transport on an interval as in 
B-II, Section 3 or diffusion on an interval, which then is identified with an edge of a graph.  With suitable boundary conditions at the nodes of the graph one obtains the generator of a positive semigroup. Concerning transport, we refer in particular to \mycite{zbmath05773159} and \mycite{zbmath05010468}, while diffusion is investigated in  \mycite{zbmath05180115}, \mycite{zbmath05244453}, \mycite{zbmath07952050} and \mycite{zbmath05167359}.
Eventually positive semigroups occur for semigroups on networks if the bi-Laplacian is considered on the edges and suitable node-conditions are requested, see \mycite{zbmath07189579}.

\end{enumerate}
%% --
\section*{Updated Notes C-III}
\addcontentsline{toc}{section}{Updated Notes C-III}
%% --
\begin{enumerate}
\item 
%Wolfgangs Text

The question whether the generator of a positive semigroup on a Banach lattice always has additively cyclic boundary spectrum is still open. 
As in C-III (and B-III) additional assumptions, essentially on the growth of the resolvent, are needed. 
In the analogous case of a bounded positive operator they are relaxed in \mycite{zbmath06591946}.


\item 
Concerning the additive cyclicity of the boundary point spectrum the situation is clearer. 
C-III, Corollary~4.3 establishes additive cyclicity under additional assumptions, while C-III, Example~4.4 shows that the boundary point spectrum may not be additively cyclic, in general.
If a positive semigroup is irreducible and bounded and if $s(A) = 0$,  then the boundary point spectrum $P\sigma_{b}(A)$ of its generator $A$ is a subgroup of $\im\R$. 
This is a consequence of C-III, Theorem~3.8, see also Proposition~3.1 in   \mycite{zbmath07423250}.
However, there exists a bounded, irreducible, positive semigroup on an $L^{1}$-space,  preserving the norm  on the positive cone, such that the  boundary spectrum 
$\sigma_{b}(A)$ of its generator $A$ 
is not a subgroup of $\im\R$, see Theorem~3.2 in \mycite{zbmath07423250}.
This solves the problem formulated before  B-III, Theorem~3.11.

\item 
There is also the notion of the \emph{ergodic spectrum} $E\sigma(A)$ of a bounded semigroup $(T(t))_{t\geq 0}$ consisting of all points $s\in\R$ such that the semigroup 
%$(\exp(-\im st)\cdot T(t))_{t\geq 0}$
$(\mathrm{e}^{-\im st}\,T(t))_{t\geq 0}$
is not mean ergodic. 
If the semigroup is positive and the underlying Banach lattice has order continuous norm, then $E\sigma(A)$ is additively cyclic,
see \mycite{zbmath00763800}. 
This is no longer true on $C(K)$ . 

\item
Part of the results of Chapter C-III have been extended  to bounded,  uniformly eventually positive semigroups with $s(A)=0$. 
By Theorem~4.7 in \mycite{zbmath08029955} their generator has cyclic boundary spectrum. 
Moreover, assume that such a semigroup $(T(t))_{t\geq 0}$, defined on a  Banach lattice $E$, is \emph{persistently irreducible} (\ie if $J$ is a closed ideal such that $T(t)J \subset J$ for all $t\geq t_0$ for some  $t_0 > 0$, then $J = {0}$ or $J = E$).  
Then the following holds. 
If $s(A) = 0$ is a pole of the resolvent, then $P\sigma(A) = i\alpha \mathbb{Z}$ for some $\alpha \in \mathbb{R}$, see Theorem~4.3 in \mycite{zbmath08029955} .
More information on persistently irreducible semigroups is in \mycite{zbmath07889246}.

\end{enumerate}
%% --
\section*{Updated Notes C-IV}
\addcontentsline{toc}{section}{Updated Notes C-IV}
%% --
\begin{enumerate}
\item 
The problem formulated after C-IV, Theorem~1.1 has been solved by \mycite{zbmath00868262}:
The growth bound and spectral bound coincide for positive semigroups on all $L^{p}$-spaces  for $1 \leq p \ < \infty$. 
This proof is reproduced with more details in the monograph \mycite{zbmath00921898}, and a different proof is given in \mycite[Theorem 5.3.6]{zbmath05842872}.  
Recently a short and elegant proof of Weis' Theorem has been found by \mycite{zbmath07574448}, which is even valid for eventually positive semigroups.

\item 
A survey on the asymptotic behavior of positive semigroups can be found in \mycite{mr4176382}, where also countable boundary spectrum is discussed.


\item 
In C-IV, Corollary 2.12 non-spectral conditions imply strong convergence of a semigroup as $t \rightarrow \infty$. 
The essential property is that one operator $T(t_0)$ is a kernel operator. 
This surprising phenomenon has been studied systematically in \mycite{zbmath07124949}, with various generalizations and different arguments. 
The main hypothesis is that one of the semigroup operators $T(t_0)$ is AM-compact (which includes kernel operators and compact operators). 
For a special case and a particularly elegant argument we refer to \mycite{zbmath06790441}.
These ideas are developed further in \mycite{zbmath07201923}. 

\item 
 More generally, a wealth of non-spectral results on the convergence of positive semigroups is known, see e.g., \mycite{zbmath00482883}  and \mycite{zbmath05080033}. 
 Very general results of this kind have been obtained by \mycite{zbmath06971625}, where an (individual) lower bound for the semigroup is the essential hypothesis.

 
\item 
In C-IV, Theorem 2.14, conditions are given implying that  a positive, irreducible, bounded semigroup converges strongly to a periodic group. 
Further results of analogous asymptotic behavior are in \mycite{zbmath05262335}.
Also, certain flows on a network converge to a periodic flow as shown in \mycite{zbmath02146799} and \mycite{zbmath05598906}.
Many of these results are based on the Jacobs-DeLeeuw-Glicksberg Theorem, see  \mycite[Theorem  V.2.8]{zbmath01354832} and \mycite{zbmath06423122}.
The strongest results of this sort are obtained if the semigroup has strongly compact orbits, see  \mycite[Theorem V.2.14]{zbmath01354832}. 
A different approach to such a decomposition into a group part and a part which converges to $0$ is given in \mycite[Chapter V]{zbmath05842872}, see also the notes to Section 5.4 in that book.
\mycite{zbmath07201923} introduce the notion \emph{semigoup at infinity} which allows them not only to generalize the results by \mycite{zbmath07124949} mentioned above, but also to obtain structure theorems for positive groups. 


\end{enumerate}

%% --