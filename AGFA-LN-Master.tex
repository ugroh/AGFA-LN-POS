% !TEX TS-program = pdflatexmk
%% -- 
%% -- Das Masterefile: AGFA-LN-Master.tex  
%% -- Stand:	2025/01/12 
%% --
\documentclass[%
	,ngerman				
	,bibliography 	= leveldown % nicht Chapter	
	]{scrbook}

%% -- KOMA Otions
%% -- 

\KOMAoptions{%
	,paper 		= a4
	,pagesize
	,BCOR		= 15mm		% siehe KOMA-Script 2.6, S. 39
	,twoside	= true		% siehe KOMA-Script 2.6, S. 47
	,fontsize	= 11pt		% Standard ist 11pt
%   ,DIV		= 12		% calc		% siehe KOMA-Script 2.6, S. 42
	,headings 	= small		% siehe KOMA-Script 3.16, S. 113 
	}		 

%% -- Alle benötigten sty-Files befinden sich in ./Preamble/AGFA-LN-Buch; 
%% --

\usepackage[
  	,LN2Buch
%	,LN2Online
	]{./Preamble/AGFA-LN-Buch}

%% -- Was soll bearbeitet werden
%% --

%\includeonly{%
%,./Content/AGFA-LN-Buch-Vorwort
%}
	
%% -- Für den Druck % entfernen; 
%% -- Notwendige Bindekorrektur
%% --
% \KOMAoptions{BCOR = 12mm}

%% -- Literaturverzeichnis seperat für jeden Beitrag
%% -- mittels \begin{refsection} .. \end{refsection}
%% --

\addbibresource{./content/content-part-a1/bib-part-a/Biblio-part-a.bib}

%% -- Gesamtbibliothek zum Ende des Buches
%% --
\addbibresource{./bib/agfa-ln-bib.bib}	

%% -- Eventuell alles auf false setzen bei der Druckversion
%% --	
\ExecuteBibliographyOptions{%
	,backref	= true	% Auf was ist verwiesen; Rest weg
 	,url		= true	% true: URL wird angezeigt 
 						% false: URL wird nicht angezeigt
 	,doi		= false	% DOI werden im Titel hinterlegt 
 						% true: wird noch separat angezeigt
	,eprint		= false	% true: wird separat angezeigt
 	,maxcitenames  = 1 	% viele Autoren -> nur der erste erscheint  
    ,maxbibnames   = 10 % im Verzeichnis tauchen alles auf
    ,mincitenames  = 1	% Muss sein
	}

%% --  Querverweise
%% --  siehe AGFA-readme.pdf bzw. LaTeX-Tipps dazu
%% --
\hypersetup{
	,colorlinks	= true    	%  Für PDF-Version true                                                           
	,urlcolor	= blue      %                                                              
	,citecolor	= blue      %                                                          
	,linkcolor	= blue		% oder black
%  	,hidelinks 				% Vor dem Druck % entfernen
	}

\usepackage{lucida-otf}

%% -- Start
%% --
\begin{document}

%% -- Frontmatter
%% --

\frontmatter

%\ifthenelse{\boolean{Rom2Buch}}
%	{\input{./content/Rom-Titel-Buch}} 		% Titelseite für Buch
%	{\input{./content/Rom-Titel-Online}}	% Titelseite für online
%	
%\tableofcontents
%
%\include{./content/Rom-Vorwort}
%\include{./content/Rom-Agenda}
%%


%% -- Mainmatter
%% -- 


%% -- Die einzelnen Abschnitte werden über \includeonly eingebaut 
%% -- Dies bewirkt einen Seitenvorschub  und man muss zum Schluss
%% -- sehen, was dieser bewirkt hat. Im Zweifel dann das 
%% -- \include in ein \input abändern. Es ist aber bei der Erstellung
%% -- so bequemer, da die Querverweise etc. erhalten bleiben.
%% --

\include{./content/AGFA-LN-Part-A}		% Part A
\include{./content/AGFA-LN-Part-B}		% PartB
%% -- ...
\include{./content/AGFA-LN-Part-D}		% Part D

\cleardoubleoddpage 

%% -- Backmatter Literatur
%% --

\backmatter 

% -- Literaturverzeichnis
%% --
				
\RaggedRight		% Kein Blocksatz
\printbibliography

%% --

\end{document}
