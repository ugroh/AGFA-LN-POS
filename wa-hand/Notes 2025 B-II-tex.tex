\documentclass[11pt]{article}
\usepackage[utf8]{inputenc}
\usepackage[T1]{fontenc}
\usepackage{amsmath,amssymb,amsthm}
\usepackage{geometry}
\geometry{a4paper, margin=2.5cm}

\newtheorem{theorem}{Theorem}[section]
\newtheorem{corollary}[theorem]{Corollary}
\newtheorem{definition}[theorem]{Definition}
\newtheorem{remark}[theorem]{Remark}

\DeclareMathOperator{\dom}{D}
\DeclareMathOperator{\spbound}{s}
\DeclareMathOperator{\TFAE}{TFAE}

\title{4. Notes to B-II, 40 years later}
\author{Wolfgang (nach Handschrift)}
\date{4.6.2025}

\begin{document}

\maketitle

\section*{Vorwort}

Liebe Koautoren,

hier ist ein Versuch Resultate aus den letzten 40 Jahren darzustellen. Ich habe mal versucht es in Form von Notes zu tun. Allerdings denke ich, dass wir sicherlich 50-100 Seiten schreiben sollten, nur ein vollständiges Bild zu präsentieren.

Vorbild könnten die Notes zu Dirk Werners FA-Buch sein. Aber wir sollten einen neuen Text weiterhin im Auge behalten.

Gruß
Wolfgang

\section{Notes to B-II, 40 years later}
\subsection*{oder 2025 (?)}

Many results on positive semigroups are known today on Banach lattices and also ordered Banach spaces. Specific to $C(\overline{\Omega})$ or $C_0(\Omega)$ are generation results for elliptic operators with diverse boundary conditions. Usually they are first treated in $L^2(\Omega)$ and later one succeeded in proving invariance of $C_0(\Omega)$ or $C(\overline{\Omega})$. But sometimes the case $C_0(\Omega)$ allows a separate, even easier proof.

In the case of differential operators in non-divergence form there are cases where generation results are valid on $C_0(\Omega)$ but unknown on $L^p(\Omega)$.

In all cases it is a challenge to prove generation results under minimal conditions on the open set $\Omega$ and the coefficients of the operator. But on spaces of continuous functions irreducibility is a strong property and not easy to prove.

We start considering the Laplacian with diverse boundary conditions.

\subsection{The Laplacian}

Let $\Omega \subset \mathbb{R}^d$ be open and bounded. We say that $\Omega$ is \textbf{Dirichlet-regular} if for every $g \in C(\partial\Omega)$ there exists a function $u \in C(\overline{\Omega}) \cap C^2(\Omega)$ such that
\begin{align}
\Delta u &= 0 \quad \text{and}\\
u|_{\partial\Omega} &= g.
\end{align}

If $\Omega$ has Lipschitz boundary, then $\Omega$ is Dirichlet regular. In dimension $d = 2$ it suffices that $\Omega$ is simply connected.

We refer to Arendt-Urban 2024 Section 6.9 or Gilbarg-Trudinger 1977 for further information on the Dirichlet Problem.

The Dirichlet Laplacian $\Delta_0$ on $C_0(\Omega)$ is defined by
\begin{align}
\Delta_0 u &:= \Delta u\\
\dom(\Delta_0) &:= \{u \in C_0(\Omega) : \Delta u \in C_0(\Omega)\}.
\end{align}

Here $\Delta u$ is to be understood in the sense of distributions.

\begin{theorem}[4.1]
$\TFAE$:
\begin{enumerate}
\item[(a)] $\Omega$ is Dirichlet regular;
\item[(b)] $\Delta_0$ generates a positive semigroup $\mathcal{T}$ on $C_0(\Omega)$.
\end{enumerate}
In that case the semigroup $\mathcal{T}$ is holomorphic of angle $\pi/2$. Moreover $T(t)$ is compact for all $t > 0$ and $\omega_0(\Delta_0) < 0$.
\end{theorem}

This result extends Example C-I 1.52) which also fits into Chapter B-I. It is due to Arendt-Bénilan 1999, Theorem 2.3 and 2.4.

If $\Omega$ is connected, the semigroup $\mathcal{T}$ in the theorem above is irreducible. On the space $C_0(\Omega)$, this is not so easy to prove. In fact, it means that for $0 < f \in C_0(\Omega)$, $(T(t)f)(x) > 0$ for all $x \in \Omega$ and not just almost everywhere.

The paper Arendt, ter Elst, Glück 2020 is devoted to the study of irreducibility on spaces of continuous functions. In our situation it leads to the following result.

\begin{theorem}[4.2]
Assume that $\Omega \subset \mathbb{R}^d$ is connected, open, bounded and Dirichlet regular. Then the semigroup $\mathcal{T}$ generated by $\Delta_0$ is irreducible.
\end{theorem}

Next we consider Robin boundary conditions. We assume that $\Omega$ has Lipschitz boundary. Then there exists a unique bounded operator $\text{tr} : H^1(\Omega) \to L^2(\partial\Omega)$ such that $\text{tr}u = u|_{\partial\Omega}$ for all $u \in H^1(\Omega) \cap C(\overline{\Omega})$. It is called the trace operator. The space $L^2(\partial\Omega)$ is defined with the surface measure (i.e. the $(d-1)$-dimensional Hausdorff measure) on $\partial\Omega$.

The normal derivative $\partial_\nu u$ of $u$ is defined as follows. Let $u \in H^1(\Omega)$ such that $\Delta u \in L^2(\Omega)$. Let $h \in L^2(\partial\Omega)$. We say that $h$ is the (outer) normal derivative of $u$ and write $\partial_\nu u = h$ if
\begin{equation}
\int_\Omega \Delta u v + \int_\Omega \nabla u \nabla v = \int_{\partial\Omega} h \text{tr}v
\end{equation}
for all $v \in H^1(\Omega)$.

If $u \in H^1(\Omega)$ such that $\Delta u \in L^2(\Omega)$ we say $\partial_\nu u \in L^2(\partial\Omega)$ if there exists $h \in L^2(\partial\Omega)$ such that $\partial_\nu u = h$.

\textbf{Remark:} Since $\Omega$ has Lipschitz boundary the outer normal $\nu(z)$ exists for almost all $z \in \partial\Omega$ and $\nu \in L^\infty(\partial\Omega)$. But we do not use this outer normal and rather define $\partial_\nu u$ weakly by the validity of Green's formula.

Let $\beta \in L^\infty(\partial\Omega)$. We define the Laplacian $\Delta^\beta$ with Robin boundary conditions as follows. Let
\begin{align}
\dom(\Delta^\beta) &:= \{u \in H^1(\Omega) : \Delta u \in L^2(\Omega),\\
&\qquad \partial_\nu u + \beta \text{tr}(u) = 0\}\\
\Delta^\beta u &:= \Delta u.
\end{align}

We call $\Delta^\beta$ briefly the Robin-Laplacian. Note that for $\beta = 0$, we obtain Neumann boundary conditions and $\Delta^N := \Delta^0$ is the Neumann Laplacian.

The following result is valid.

\begin{theorem}[4.3]
Assume that $\Omega \subset \mathbb{R}^d$ is bounded, open, connected with Lipschitz boundary, and let $\beta \in L^\infty(\partial\Omega)$. Then $\Delta^\beta$ generates a positive, irreducible, holomorphic semigroup $\mathcal{T} = (T(t))_{t \geq 0}$ on $C(\overline{\Omega})$. Moreover, $T(t)$ is compact for all $t > 0$.
\end{theorem}

Irreducibility has strong consequences. One has $\sigma(\Delta^\beta) = \sigma_p(\Delta^\beta) \subset \mathbb{R}$. Denote by $\spbound(\Delta^\beta)$ the spectral bound of $\Delta^\beta$. Then $\spbound(\Delta^\beta)$ is the largest eigenvalue of $\Delta^\beta$. It is the unique eigenvalue with a positive eigenfunction $0 < u_0 \in \dom(\Delta^\beta)$. The eigenfunction $u_0$ is strictly positive; i.e. there exists $\delta > 0$ such that $u_0(x) \geq \delta > 0$ for all $x \in \overline{\Omega}$.

The spectral bound $\spbound(\Delta^\beta)$ determines the asymptotic behavior of the semigroup $\mathcal{T}$. In fact, the following follows from B-II Proposition 3.5.

\begin{corollary}[4.4]
There exist a strictly positive Borel measure $\mu$ on $\overline{\Omega}$, $M \geq 0$ and $\varepsilon > 0$ such that $\langle \mu, u_0 \rangle = 1$ and
\begin{equation}
\|T(t) - e^{\spbound(\Delta^\beta)t} P\| \leq M e^{-\varepsilon t}
\end{equation}
for all $t \geq 0$, where $P \in \mathcal{L}(C(\overline{\Omega}))$ is given by
\begin{equation}
Pf = \langle \mu, f \rangle u_0.
\end{equation}
\end{corollary}

The theorem says that the semigroup converges in the operator norm to the rank-1-projection $P$ exponentially fast.

\section{Elliptic operators in divergence form}

The preceding results extend to elliptic operators in divergence form for bounded measurable coefficients.

Let $\Omega \subset \mathbb{R}^d$ be open and bounded. Let $a_{k,\ell}, b_k, c_k, c_0 \in L^\infty(\Omega)$, $k, \ell = 1, \ldots, d$ such that for some $\alpha > 0$
\begin{equation}
\sum_{k,\ell=1}^d a_{k,\ell}(x) \xi_k \xi_\ell \geq \alpha |\xi|^2
\end{equation}
for all $x \in \Omega$, $\xi \in \mathbb{R}^d$, where $|\xi|^2 = \xi_1^2 + \ldots + \xi_d^2$.

Let $H^1_{loc}(\Omega) := \{v \in L^2_{loc}(\Omega) : D_k v \in L^2_{loc}(\Omega), k = 1, \ldots, d\}$.

Define $\mathcal{A} : H^1_{loc}(\Omega) \to C_0'(\Omega)$ by
\begin{align}
\langle \mathcal{A}u, v \rangle &= \sum_{k,\ell=1}^d \int_\Omega a_{k,\ell}(x) D_\ell u D_k v \, dx + \sum_{k=1}^d \int_\Omega b_k(x) D_k u v \, dx\\
&\quad + \sum_{k=1}^d \int_\Omega c_k(x) u D_k v \, dx + \int_\Omega c_0(x) u v \, dx.
\end{align}

We define $A_0$ as the part of $\mathcal{A}$ in $C_0(\Omega)$; i.e.
\begin{align}
\dom(A_0) &:= \{u \in C_0(\Omega) \cap H^1_0(\Omega) : \mathcal{A}u \in C_0(\Omega)\}\\
A_0 u &:= \mathcal{A}u.
\end{align}

Then Theorem 4.1 holds with $\Delta_0$ replaced by $A_0$. It is remarkable that Dirichlet regularity of $\Omega$ is the right boundary condition again. This is due to fundamental results of Stampacchia and co-authors. We refer to Arendt and Bénilan 1999 for a proof of the following result.

\begin{theorem}[4.4]
Assume that $\Omega \subset \mathbb{R}^d$ is a bounded, open, connected Dirichlet regular set. Then $A_0$ generates a positive, irreducible, holomorphic semigroup $\mathcal{T} = (T(t))_{t \geq 0}$ on $C_0(\Omega)$. Moreover, $T(t)$ is compact for all $t > 0$.
\end{theorem}

Also the results for Robin boundary conditions Theorem 4.3 and 4.4 can be extended for elliptic operators in divergence form on $C_0(\Omega)$; see [...]

\section{Elliptic operators in non-divergence form on $C_0(\Omega)$}

\textbf{To Do}

The Dirichlet-to-Neumann operator on $C(\partial\Omega)$ -- for this case irreducibility is very surprising.

\section*{References for Notes to B-II 2025}

W. Arendt, A.F.M. ter Elst, J. Glück: Strict positivity for the principal eigenfunction of elliptic operators with various boundary conditions. Adv. Nonlinear Stud. 2020; 20(3): 633--650

\textbf{To Do} [weitere Referenzen hinzufügen]

\end{document}