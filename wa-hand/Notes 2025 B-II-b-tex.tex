\documentclass[11pt]{article}
\usepackage[utf8]{inputenc}
\usepackage[T1]{fontenc}
\usepackage{amsmath,amssymb,amsthm}
\usepackage{geometry}
\geometry{a4paper, margin=2.5cm}

\newtheorem{theorem}{Theorem}[section]
\newtheorem{corollary}[theorem]{Corollary}
\newtheorem{definition}[theorem]{Definition}
\newtheorem{remark}[theorem]{Remark}

\DeclareMathOperator{\dom}{D}
\DeclareMathOperator{\spbound}{s}
\DeclareMathOperator{\tr}{tr}

\title{Green's Formula and Robin-Laplacian}
\author{Mathematical Notes (from Handwriting)}
\date{}

\begin{document}

\maketitle

\section{Green's Formula}

Let $\beta \in L^\infty(\partial\Omega)$. We define the Laplacian $\Delta^\beta$ with Robin boundary conditions as follows. Let
\begin{align}
\dom(\Delta^\beta) &:= \{u \in H^1(\Omega) : \Delta u \in L^2(\Omega),\\
&\qquad \partial_\nu u + \beta \tr(u) = 0\}\\
\Delta^\beta u &:= \Delta u.
\end{align}

We call $\Delta^\beta$ briefly the \textbf{Robin-Laplacian}. Note that for $\beta = 0$, we obtain \textbf{Neumann boundary conditions}, and $\Delta^N := \Delta^0$ is the \textbf{Neumann Laplacian}.

The following result is valid.

\begin{theorem}[4.3]
Assume that $\Omega \subset \mathbb{R}^d$ is bounded, open, connected with Lipschitz boundary, and let $\beta \in L^\infty(\partial\Omega)$. Then $\Delta^\beta$ generates a positive, irreducible, holomorphic semigroup $\mathcal{T} = (T(t))_{t \geq 0}$ on $C(\overline{\Omega})$. Moreover, $T(t)$ is compact for all $t > 0$.
\end{theorem}

Irreducibility has strong consequences. One has $\sigma(\Delta^\beta) = \sigma_p(\Delta^\beta) \subset \mathbb{R}$. Denote by $\spbound(\Delta^\beta)$ the spectral bound of $\Delta^\beta$. Then $\spbound(\Delta^\beta)$ is the largest eigenvalue of $\Delta^\beta$. It is the unique eigenvalue with a positive eigenfunction $0 < u_0 \in \dom(\Delta^\beta)$. The eigenfunction $u_0$ is strictly positive; i.e. there exists $\delta > 0$ such that $u_0(x) \geq \delta > 0$ for all $x \in \overline{\Omega}$.

The spectral bound $\spbound(\Delta^\beta)$ determines the asymptotic behavior of the semigroup $\mathcal{T}$. In fact, the following follows from B-II Proposition 3.5.

\begin{corollary}[4.4]
There exist a strictly positive Borel measure $\mu$ on $\overline{\Omega}$, $M \geq 0$ and $\varepsilon > 0$ such that $\langle \mu, u_0 \rangle = 1$ and
\begin{equation}
\|T(t) - e^{\spbound(\Delta^\beta)t} P\| \leq M e^{-\varepsilon t}
\end{equation}
for all $t \geq 0$, where $P \in \mathcal{L}(C(\overline{\Omega}))$ is given by
\begin{equation}
Pf = \langle \mu, f \rangle u_0.
\end{equation}
\end{corollary}

The theorem says that the semigroup converges in the operator norm to the rank-1-projection $P$ exponentially fast.

\section{Elliptic Operators in Divergence Form}

The preceding results extend to elliptic operators in divergence form for bounded measurable coefficients.

Let $\Omega \subset \mathbb{R}^d$ be open and bounded. Let $a_{k,\ell}, b_k, c_k, c_0 \in L^\infty(\Omega)$, $k, \ell = 1, \ldots, d$ such that for some $\alpha > 0$
\begin{equation}
\sum_{k,\ell=1}^d a_{k,\ell}(x) \xi_k \xi_\ell \geq \alpha |\xi|^2
\end{equation}
for all $x \in \Omega$, $\xi \in \mathbb{R}^d$, where $|\xi|^2 = \xi_1^2 + \ldots + \xi_d^2$.

Let $H^1_{loc}(\Omega) := \{v \in L^2_{loc}(\Omega) : D_k v \in L^2_{loc}(\Omega), k = 1, \ldots, d\}$.

Define $\mathcal{A} : H^1_{loc}(\Omega) \to C_0'(\Omega)$ by
\begin{align}
\langle \mathcal{A}u, v \rangle &= \sum_{k,\ell=1}^d \int_\Omega a_{k,\ell}(x) D_\ell u D_k v \, dx + \sum_{k=1}^d \int_\Omega b_k(x) D_k u v \, dx\\
&\quad + \sum_{k=1}^d \int_\Omega c_k(x) u D_k v \, dx + \int_\Omega c_0(x) u v \, dx.
\end{align}

We define $A_0$ as the part of $\mathcal{A}$ in $C_0(\Omega)$; i.e.
\begin{align}
\dom(A_0) &:= \{u \in C_0(\Omega) \cap H^1_0(\Omega) : \mathcal{A}u \in C_0(\Omega)\}\\
A_0 u &:= \mathcal{A}u.
\end{align}

Then Theorem 4.1 holds with $\Delta_0$ replaced by $A_0$. It is remarkable that Dirichlet regularity of $\Omega$ is the right boundary condition again. This is due to fundamental results of Stampacchia and co-authors. We refer to Arendt and Bénilan 1999 for a proof of the following result.

\begin{theorem}[4.4]
Assume that $\Omega \subset \mathbb{R}^d$ is a bounded, open, connected Dirichlet regular set. Then $A_0$ generates a positive, irreducible, holomorphic semigroup $\mathcal{T} = (T(t))_{t \geq 0}$ on $C_0(\Omega)$. Moreover, $T(t)$ is compact for all $t > 0$.
\end{theorem}

Also the results for Robin boundary conditions Theorems 4.3 and 4.4 can be extended for elliptic operators in divergence form on $C_0(\Omega)$; see Arendt and Bénilan 1999 for a proof of the following result.

\section{Elliptic Operators in Non-Divergence Form on $C_0(\Omega)$}

\textbf{To Do}

The Dirichlet-to-Neumann operator on $C(\partial\Omega)$ -- for this case irreducibility is very surprising.

\section*{References for Notes to B-II 2025}

W. Arendt, A.F.M. ter Elst, J. Glück: Strict positivity for the principal eigenfunction of elliptic operators with various boundary conditions. Adv. Nonlinear Stud. 2020; 20(3): 633--650

\textbf{To Do} [weitere Referenzen hinzufügen]

\end{document}